\documentclass[a4paper,14pt]{extarticle}


\usepackage[utf8]{inputenc}
\usepackage[T2A]{fontenc}
\usepackage[english,russian]{babel}
\usepackage{ragged2e}
\usepackage[utf8]{inputenc}
\usepackage{hyperref}
\usepackage{minted}
\setmintedinline{frame=lines, framesep=2mm, baselinestretch=1.5, bgcolor=LightGray, breaklines,fontsize=\scriptsize}
\setminted{frame=lines, framesep=2mm, baselinestretch=1.5, bgcolor=LightGray, breaklines,fontsize=\scriptsize}
\usepackage{xcolor}
\definecolor{LightGray}{gray}{0.9}
\usepackage{graphicx}
\usepackage[export]{adjustbox}
\usepackage[left=1cm,right=1cm, top=1cm,bottom=1cm,bindingoffset=0cm]{geometry}
\usepackage{fontspec}
\usepackage{ upgreek }
\usepackage[shortlabels]{enumitem}
\usepackage{adjustbox}
\usepackage{multirow}
\usepackage{amsmath}
\usepackage{amssymb}
\usepackage{pifont}
\usepackage{pgfplots}
\usepackage{longtable}
\usepackage{array}
\graphicspath{ {./images/} }
\makeatletter
\AddEnumerateCounter{\asbuk}{\russian@alph}{щ}
\makeatother
\setmonofont{Ubuntu Mono}
\setmainfont{Times New Roman}

\newcommand\textbox[1]{
	\parbox{.45\textwidth}{#1}
} 

\newcommand{\specialcell}[2][c]{%
	\begin{tabular}[#1]{@{}c@{}}#2\end{tabular}}

\begin{document}
\pagenumbering{gobble}
\begin{center}
    \small{
        \textbf{МИНИCТЕРCТВО НАУКИ И ВЫCШЕГО ОБРАЗОВАНИЯ РОCCИЙCКОЙ ФЕДЕРАЦИИ}\\
        ФЕДЕРАЛЬНОЕ ГОCУДАРCТВЕННОЕ БЮДЖЕТНОЕ ОБРАЗОВАТЕЛЬНОЕ УЧРЕЖДЕНИЕ\\ВЫCШЕГО ОБРАЗОВАНИЯ \\
        \textbf{«БЕЛГОРОДCКИЙ ГОCУДАРCТВЕННЫЙ ТЕХНОЛОГИЧЕCКИЙ\\УНИВЕРCИТЕТ им. В. Г. ШУХОВА»\\ (БГТУ им. В.Г. Шухова)} \\
        \bigbreak
        \includegraphics[width=70mm]{log}\\
        ИНСТИТУТ ИНФОРМАЦИОННЫХ ТЕХНОЛОГИЙ И УПРАВЛЯЮЩИХ СИСТЕМ\\}
\end{center}

\vfill
\begin{center}
    \large{
        \textbf{
            Лабораторная работа №2}}\\
    \normalsize{
        по дисциплине: Операционные системы \\
        тема: «Процессы и потоки в ОС Linux (Ubuntu): сравнение, механизмы синхронизации. Парадигмы межпроцессорного взаимодействия.»}
\end{center}
\vfill
\hfill\textbox{
    Выполнил: ст. группы ПВ-223\\Пахомов Владислав Андреевич
    \bigbreak
    Проверили: \\доц. Островский Алексей Мичеславович\\
    асс. Четвертухин Виктор Романович
}
\vfill\begin{center}
    Белгород 2024 г.
\end{center}
\newpage
\underline{\textbf{Цель работы: }}Изучить различия между процессами и потоками в ОС Linux (Ubuntu), а также
освоить механизмы синхронизации и межпроцессорного взаимодействия для обеспечения
корректной работы программ в многозадачной среде.\\
\underline{\textbf{Условие индивидуального задания: }}Вводится дополнительное условие в логику задачи. Каждая голова Змея Горыныча должна
уметь проверять содержимое склада несколько раз подряд, прежде чем выбрать продукт. Для
корректной работы программы, проверка и забор продукции должны быть защищены, в случае
потоков, рекурсивными мьютексами (pthread\_mutexattr\_settype), а в случае процессов —
семафорами (sys/sem.h). Голова может вызвать функцию проверки доступности продукта
несколько раз (например, чтобы проверить разные типы продукции). Важно, чтобы каждая
голова корректно освобождала мьютекс после завершения работы.\\
\begin{center}
\textbf{Ход выполнения работы}
\end{center}

\textbf{Текст программы:}\bigbreak
\textbf{shared.h}
\begin{minted}{C}
#ifndef SHARED
#define SHARED

#include <pthread.h>
#include <stdio.h>
#include <stdlib.h>
#include <sys/time.h>
#include <stdbool.h>

// Лаба слишком вкусная... Пойду точить...

// Хранилище. В стандартных настройках первые 4 бита занимают тортики, 
// 4-8 биты: пироженки, 8-12: конфетки, 12-16: пряники
u_int64_t storage = 0;
// Мьютексы для потоков
pthread_mutexattr_t mutex_attr;
pthread_mutex_t mutex;

// Флаг, работает ли основной цикл
bool is_running = true;

// Максимальное количество вкусняшек на складе
#define MAX_STORAGE_SIZE 10
// Сколько битов занимает склад для одной вкусняшки
#define STORAGE_CHUNK_SIZE 4
// Максимальное количество разновидностей вкусняшек
#define MAX_CHUNKS_AMOUNT (sizeof(storage) * 8 / STORAGE_CHUNK_SIZE)
// Базовая маска для вкусняшки 
#define STORAGE_CHUNK_MASK ((1 << STORAGE_CHUNK_SIZE) - 1)

// Отступ для тортиков
#define CAKE_OFFSET (STORAGE_CHUNK_SIZE * 0)
// Отступ для пироженок
#define BROWNIE_OFFSET (STORAGE_CHUNK_SIZE * 1)
// Отступ для конфеток
#define CANDY_OFFSET (STORAGE_CHUNK_SIZE * 2)
// Отступ для пряников
#define GINGERBREAD_OFFSET (STORAGE_CHUNK_SIZE * 3)

long current_millis()
{
    struct timeval tp;

    gettimeofday(&tp, NULL);
    return tp.tv_sec * 1000 + tp.tv_usec / 1000;
}

// Устанавливает количество вкусняшек по offset в storage
void set_amount_src(u_int64_t* storage_address, int offset, u_int64_t amount)
{
    (*storage_address) = ((*storage_address) & (~(STORAGE_CHUNK_MASK << offset))) | ((amount & STORAGE_CHUNK_MASK) << offset);
}

void set_amount(int offset, u_int64_t amount)
{
    set_amount_src(&storage, offset, amount);
}

// Возвращает количество вкусняшек по offset в storage
u_int64_t get_amount_src(u_int64_t* storage_address, int offset)
{
    return (((*storage_address) & (STORAGE_CHUNK_MASK << offset)) >> offset);
}

u_int64_t get_amount(int offset)
{
    return get_amount_src(&storage, offset);
}

// Возвращает общее количество вкусняшек
u_int64_t get_storage_amount_src(u_int64_t* storage_address)
{
    return get_amount_src(storage_address, CAKE_OFFSET) + get_amount_src(storage_address, BROWNIE_OFFSET) +
           get_amount_src(storage_address, CANDY_OFFSET) + get_amount_src(storage_address, GINGERBREAD_OFFSET);
}

u_int64_t get_storage_amount() {
    return get_storage_amount_src(&storage);
}

u_int64_t cakes_thrown = 0;
u_int64_t brownies_thrown = 0;
u_int64_t candies_thrown = 0;
u_int64_t gingerbreads_thrown = 0;

u_int64_t cakes_baked = 0;
u_int64_t brownies_baked = 0;
u_int64_t candies_baked = 0;
u_int64_t gingerbreads_baked = 0;

u_int64_t cakes_ate = 0;
u_int64_t brownies_ate = 0;
u_int64_t candies_ate = 0;
u_int64_t gingerbreads_ate = 0;

void print_report() {
    printf("А теперь посчитаем...\n");
    printf("Тортики:\n");
    printf("- Испечено: %lu\n", cakes_baked);
    printf("- Выброшено: %lu\n", cakes_thrown);
    printf("- Слопано: %lu\n\n", cakes_ate);
    printf("Пироженки:\n");
    printf("- Испечено: %lu\n", brownies_baked);
    printf("- Выброшено: %lu\n", brownies_thrown);
    printf("- Слопано: %lu\n\n", brownies_ate);
    printf("Конфетки:\n");
    printf("- Испечено: %lu\n", candies_baked);
    printf("- Выброшено: %lu\n", candies_thrown);
    printf("- Слопано: %lu\n\n", candies_ate);
    printf("Пряники:\n");
    printf("- Испечено: %lu\n", gingerbreads_baked);
    printf("- Выброшено: %lu\n", gingerbreads_thrown);
    printf("- Слопано: %lu\n", gingerbreads_ate);
}

#endif
\end{minted}

\textbf{threads.c}
\begin{minted}{C}
#include "shared.h"
#include <unistd.h>

void *workshop(void *args_outer)
{
    u_int64_t *args = (u_int64_t *)args_outer;

    while (is_running)
    {
        u_int64_t plus_amnt = args[0];
        u_int64_t plus_offs = args[1];
        // Производим товар в цехе
        usleep(args[2] * 1000);
        // Начинаем транспортировку, требующую синхронизацию
        pthread_mutex_lock(&mutex);
        switch (plus_offs)
        {
        case CAKE_OFFSET:
            cakes_baked += plus_amnt;
            break;
        case CANDY_OFFSET:
            candies_baked += plus_amnt;
            break;
        case BROWNIE_OFFSET:
            brownies_baked += plus_amnt;
            break;
        case GINGERBREAD_OFFSET:
            gingerbreads_baked += plus_amnt;
            break;
        }
        u_int64_t total = get_storage_amount();
        if (total + plus_amnt > MAX_STORAGE_SIZE)
        {
            switch (plus_offs)
            {
            case CAKE_OFFSET:
                cakes_thrown += (total + plus_amnt - MAX_STORAGE_SIZE);
                break;
            case CANDY_OFFSET:
                candies_thrown += (total + plus_amnt - MAX_STORAGE_SIZE);
                break;
            case BROWNIE_OFFSET:
                brownies_thrown += (total + plus_amnt - MAX_STORAGE_SIZE);
                break;
            case GINGERBREAD_OFFSET:
                gingerbreads_thrown += (total + plus_amnt - MAX_STORAGE_SIZE);
                break;
            }

            plus_amnt -= total + plus_amnt - MAX_STORAGE_SIZE;
        }

        set_amount(plus_offs, get_amount(plus_offs) + plus_amnt);
        pthread_mutex_unlock(&mutex);
    }

    pthread_exit(0);
}

void start_factory(u_int64_t *data)
{
    // Фабрику подразделяем на несколько цехов, которые будут заняты производством товара
    for (int i = 0; i < data[0]; i++)
    {
        pthread_t thread_workshop;
        pthread_create(&thread_workshop, NULL, workshop, (void *)(data + i * 3 + 1));
        pthread_detach(thread_workshop);
    }
}

bool serpent_head_rec(u_int64_t serpent_preferences, int current_offset, int max_offset)
{
    u_int64_t offset = STORAGE_CHUNK_SIZE * current_offset;
    u_int64_t mask_with_offset = STORAGE_CHUNK_MASK << offset;
    // Змей Горыныч будет искать предпочтительный склад с наибольшим количеством вкусняшек, чтобы
    // оптимизировать работу фабрик. Какой заботливый!
    pthread_mutex_lock(&mutex);
    if ((storage & mask_with_offset) && (serpent_preferences & mask_with_offset))
    {
        u_int64_t storage_amount = get_amount(offset);
        if (max_offset == -1 || max_offset != -1 && storage_amount > get_amount(max_offset))
        {
            max_offset = offset;
        }
    }

    if (current_offset != 0)
    {
        bool result = serpent_head_rec(serpent_preferences, current_offset - 1, max_offset);
        pthread_mutex_unlock(&mutex);
        return result;
    }

    if (max_offset == -1)
    {
        pthread_mutex_unlock(&mutex);
        return false;
    }

    switch (max_offset)
    {
    case CAKE_OFFSET:
        cakes_ate++;
        break;
    case CANDY_OFFSET:
        candies_ate++;
        break;
    case BROWNIE_OFFSET:
        brownies_ate++;
        break;
    case GINGERBREAD_OFFSET:
        gingerbreads_ate++;
        break;
    }
    set_amount(max_offset, get_amount(max_offset) - 1);

    pthread_mutex_unlock(&mutex);

    return true;
}

void *serpent_head(void *paramOuter)
{
    u_int64_t *param = (u_int64_t *)paramOuter;

    while (is_running)
    {
        bool result = serpent_head_rec(param[0], MAX_CHUNKS_AMOUNT - 1, -1);
        if (result)
            usleep(param[1] * 1000);
    }

    pthread_exit(0);
}

/*
Вариант программы, когда под ожиданием подразумевается полезная нагрузка.
Кроме того, не так оптимизирован, так как требует большего количества потоков.
Но зато можно задавать нулевые задержки.
*/
int main()
{
    pthread_t serpent_head_a, serpent_head_b, serpent_head_c,
        workshop_a, workshop_b, workshop_c, workshop_d;

    u_int64_t factory_a_args[] = {2,
                                  1, CAKE_OFFSET, 70,
                                  2, BROWNIE_OFFSET, 50};

    u_int64_t factory_b_args[] = {2,
                                  3, CANDY_OFFSET, 60,
                                  1, GINGERBREAD_OFFSET, 40};

    u_int64_t serpent_head_a_args[] = {(STORAGE_CHUNK_MASK << CAKE_OFFSET) |
                                           (STORAGE_CHUNK_MASK << CANDY_OFFSET),
                                       80};

    u_int64_t serpent_head_b_args[] = {(STORAGE_CHUNK_MASK << BROWNIE_OFFSET) |
                                           (STORAGE_CHUNK_MASK << GINGERBREAD_OFFSET),
                                       90};

    u_int64_t serpent_head_c_args[] = {(STORAGE_CHUNK_MASK << BROWNIE_OFFSET) |
                                           (STORAGE_CHUNK_MASK << GINGERBREAD_OFFSET) |
                                           (STORAGE_CHUNK_MASK << CAKE_OFFSET) |
                                           (STORAGE_CHUNK_MASK << CANDY_OFFSET),
                                       70};

    pthread_mutexattr_init(&mutex_attr);
    pthread_mutexattr_settype(&mutex_attr, PTHREAD_MUTEX_RECURSIVE);
    pthread_mutex_init(&mutex, &mutex_attr);

    start_factory(factory_a_args);
    start_factory(factory_b_args);
    pthread_create(&serpent_head_a, NULL, serpent_head, (void *)serpent_head_a_args);
    pthread_create(&serpent_head_b, NULL, serpent_head, (void *)serpent_head_b_args);
    pthread_create(&serpent_head_c, NULL, serpent_head, (void *)serpent_head_c_args);

    pthread_detach(serpent_head_a);
    pthread_detach(serpent_head_b);
    pthread_detach(serpent_head_c);
    getchar();

    // Выходим - is_running = false. Поток должен завершиться сам.
    is_running = false;

    pthread_mutex_lock(&mutex);
    print_report();
    pthread_mutex_unlock(&mutex);

    pthread_mutex_destroy(&mutex);
    pthread_mutexattr_destroy(&mutex_attr);

    return 0;
}
\end{minted}

\textbf{threads\_emul.c}
\begin{minted}{C}
#include "shared.h"

/* Контроль переходит потоковой функции */
void *factory(void *paramOuter)
{
    u_int64_t *param = (u_int64_t *)paramOuter;
    u_int64_t prod_amount = param[0];
    long init = current_millis();
    long last_check = init;

    while (is_running)
    {
        long new_check = current_millis();

        pthread_mutex_lock(&mutex);
        for (u_int64_t i = 0; i < prod_amount; i++)
        {
            u_int64_t plus_amnt = param[3 * i + 1];
            u_int64_t plus_offs = param[3 * i + 2];
            u_int64_t plus_time = param[3 * i + 3];
            u_int64_t total_ticks = (new_check - init) / plus_time;
            u_int64_t last_ticks = (last_check - init) / plus_time;
            u_int64_t amount_after = total_ticks > last_ticks ? (total_ticks - last_ticks) : 0;
            plus_amnt *= amount_after;
            // Количество продукции увеличивается в соответствие с
            // количеством "тиков".

            switch (plus_offs)
            {
            case CAKE_OFFSET:
                cakes_baked += plus_amnt;
                break;
            case CANDY_OFFSET:
                candies_baked += plus_amnt;
                break;
            case BROWNIE_OFFSET:
                brownies_baked += plus_amnt;
                break;
            case GINGERBREAD_OFFSET:
                gingerbreads_baked += plus_amnt;
                break;
            }

            u_int64_t total = get_storage_amount();
            if (total + plus_amnt > MAX_STORAGE_SIZE)
            {
                switch (plus_offs)
                {
                case CAKE_OFFSET:
                    cakes_thrown += (total + plus_amnt - MAX_STORAGE_SIZE);
                    break;
                case CANDY_OFFSET:
                    candies_thrown += (total + plus_amnt - MAX_STORAGE_SIZE);
                    break;
                case BROWNIE_OFFSET:
                    brownies_thrown += (total + plus_amnt - MAX_STORAGE_SIZE);
                    break;
                case GINGERBREAD_OFFSET:
                    gingerbreads_thrown += (total + plus_amnt - MAX_STORAGE_SIZE);
                    break;
                }

                plus_amnt -= total + plus_amnt - MAX_STORAGE_SIZE;
            }

            set_amount(plus_offs, get_amount(plus_offs) + plus_amnt);
        }
        pthread_mutex_unlock(&mutex);

        last_check = new_check;
    }

    pthread_exit(0);
}

void serpent_head_rec(u_int64_t serpent_preferences, int left)
{
    // Здесь обход по складу несколько раз обоснован тем, что
    // может пройти несколько тиков. Рекурсивно мы их и обрабатываем
    if (left == 0)
        return;

    pthread_mutex_lock(&mutex);
    int max_offset = -1;
    u_int64_t max_amount = 0;
    for (int i = 0; i < MAX_CHUNKS_AMOUNT; i++)
    {
        u_int64_t offset = STORAGE_CHUNK_SIZE * i;
        u_int64_t mask_with_offset = STORAGE_CHUNK_MASK << offset;
        if ((storage & mask_with_offset) && (serpent_preferences & mask_with_offset))
        {
            u_int64_t storage_amount = get_amount(offset);
            if (max_offset == -1 || max_offset != -1 && storage_amount > max_amount)
            {
                max_amount = storage_amount;
                max_offset = offset;
            }
        }
    }
    // Грустная голова поняла, что ей здесь пока делать нечего,
    // и ушла, освободив ресурсы другим головам
    if (max_offset == -1)
    {
        pthread_mutex_unlock(&mutex);
        return;
    }

    switch (max_offset)
    {
    case CAKE_OFFSET:
        cakes_ate++;
        break;
    case CANDY_OFFSET:
        candies_ate++;
        break;
    case BROWNIE_OFFSET:
        brownies_ate++;
        break;
    case GINGERBREAD_OFFSET:
        gingerbreads_ate++;
        break;
    }
    set_amount(max_offset, max_amount - 1);
    serpent_head_rec(serpent_preferences, left - 1);
    pthread_mutex_unlock(&mutex);
}

void *serpent_head(void *paramOuter)
{
    u_int64_t *param = (u_int64_t *)paramOuter;
    u_int64_t prod_amount = param[0];
    long init = current_millis();
    long last_check = init;

    while (is_running)
    {
        long new_check = current_millis();
        u_int64_t total_ticks = (new_check - init) / param[1];
        u_int64_t last_ticks = (last_check - init) / param[1];
        u_int64_t amount_after = total_ticks > last_ticks ? (total_ticks - last_ticks) : 0;
        // Здесь также определяем кол-во тиков

        serpent_head_rec(param[0], amount_after);

        last_check = new_check;
    }

    pthread_exit(0);
}

/*
Данный вариант не совсем корректен.
Под потреблением фабрик и змеем горынычем
могут подразумеваться полезные нагрузки,
однако здесь мы моделируем работу
этой системы.

Эта программа тоже имеет право на жизнь,
например если мы создаём игру про фабрики и драконов,
здесь гораздо важнее справедливость модели и
следование установленным правилам,
именно поэтому мы высчитываем, сколько "тиков" прошло и
следовательно сколько было выполнено работ.
*/
int main()
{
    pthread_t factory_a, factory_b,
        serpent_head_a, serpent_head_b, serpent_head_c;

    u_int64_t factory_a_args[] = {2,
                                  1, CAKE_OFFSET, 70,
                                  2, BROWNIE_OFFSET, 50};

    u_int64_t factory_b_args[] = {2,
                                  3, CANDY_OFFSET, 40,
                                  1, GINGERBREAD_OFFSET, 60};

    u_int64_t serpent_head_a_args[] = {(STORAGE_CHUNK_MASK << CAKE_OFFSET) |
                                           (STORAGE_CHUNK_MASK << CANDY_OFFSET),
                                       80};

    u_int64_t serpent_head_b_args[] = {(STORAGE_CHUNK_MASK << BROWNIE_OFFSET) |
                                           (STORAGE_CHUNK_MASK << GINGERBREAD_OFFSET),
                                       90};

    u_int64_t serpent_head_c_args[] = {(STORAGE_CHUNK_MASK << BROWNIE_OFFSET) |
                                           (STORAGE_CHUNK_MASK << GINGERBREAD_OFFSET) |
                                           (STORAGE_CHUNK_MASK << CAKE_OFFSET) |
                                           (STORAGE_CHUNK_MASK << CANDY_OFFSET),
                                       70};

    pthread_mutexattr_init(&mutex_attr);
    pthread_mutexattr_settype(&mutex_attr, PTHREAD_MUTEX_RECURSIVE);
    pthread_mutex_init(&mutex, &mutex_attr);

    pthread_create(&factory_a, NULL, factory, (void *)factory_a_args);
    pthread_create(&factory_b, NULL, factory, (void *)factory_b_args);

    pthread_create(&serpent_head_a, NULL, serpent_head, (void *)serpent_head_a_args);
    pthread_create(&serpent_head_b, NULL, serpent_head, (void *)serpent_head_b_args);
    pthread_create(&serpent_head_c, NULL, serpent_head, (void *)serpent_head_c_args);

    pthread_detach(factory_a);
    pthread_detach(factory_b);
    pthread_detach(serpent_head_a);
    pthread_detach(serpent_head_b);
    pthread_detach(serpent_head_c);
    getchar();

    // Выходим - is_running = false. Поток должен завершиться сам.
    is_running = false;

    pthread_mutex_lock(&mutex);
    print_report();
    pthread_mutex_unlock(&mutex);

    pthread_mutex_destroy(&mutex);
    pthread_mutexattr_destroy(&mutex_attr);

    return 0;
}
\end{minted}

\textbf{processes.c}
\begin{minted}{C}
#include <unistd.h>
#include <semaphore.h>
#include <sys/ipc.h>
#include <sys/shm.h>
#include <fcntl.h>    /* константы O_* */
#include <sys/stat.h> /* константы для mode */
#include <semaphore.h>
#include <stdio.h>
#include <sys/mman.h>
#include <string.h>
#include <signal.h>
#include "shared.h"

#define STORAGE_SHM_KEY "os_lab_2_task_2_processes_storage"
#define SEMAPHORE_KEY "os_lab_2_task_2_processes_semaphore"

u_int64_t *shm_storage;
sem_t *shm_sem;
pid_t PIDS[7] = {};
int pids_size = 0;

void create_workshop(u_int64_t *args)
{
    pid_t workshop_pid = fork();
    if (workshop_pid == -1)
    {
        printf(stderr, "Не удалось создать цех.\n");
        exit(1);
    }

    if (workshop_pid != 0)
    {
        PIDS[pids_size++] = workshop_pid;
        return;
    }

    attach();
    while (is_running)
    {
        u_int64_t plus_amnt = args[0];
        u_int64_t plus_offs = args[1];
        usleep(args[2] * 1000);
        sem_wait(shm_sem);
        switch (plus_offs)
        {
        case CAKE_OFFSET:
            cakes_baked += plus_amnt;
            break;
        case CANDY_OFFSET:
            candies_baked += plus_amnt;
            break;
        case BROWNIE_OFFSET:
            brownies_baked += plus_amnt;
            break;
        case GINGERBREAD_OFFSET:
            gingerbreads_baked += plus_amnt;
            break;
        }
        u_int64_t total = get_storage_amount_src(shm_storage);
        if (total + plus_amnt > MAX_STORAGE_SIZE)
        {
            switch (plus_offs)
            {
            case CAKE_OFFSET:
                cakes_thrown += (total + plus_amnt - MAX_STORAGE_SIZE);
                break;
            case CANDY_OFFSET:
                candies_thrown += (total + plus_amnt - MAX_STORAGE_SIZE);
                break;
            case BROWNIE_OFFSET:
                brownies_thrown += (total + plus_amnt - MAX_STORAGE_SIZE);
                break;
            case GINGERBREAD_OFFSET:
                gingerbreads_thrown += (total + plus_amnt - MAX_STORAGE_SIZE);
                break;
            }

            plus_amnt -= total + plus_amnt - MAX_STORAGE_SIZE;
        }

                set_amount_src(shm_storage, plus_offs, get_amount_src(shm_storage, plus_offs) + plus_amnt);
        sem_post(shm_sem);
    }
    printf("Цех умер!\n");
    print_report();
    exit(0);
}

void create_factory(u_int64_t *data)
{
    for (int i = 0; i < data[0]; i++)
    {
        create_workshop(data + i * 3 + 1);
    }
}

bool serpent_head_rec(u_int64_t serpent_preferences, int current_offset, int max_offset)
{
    u_int64_t offset = STORAGE_CHUNK_SIZE * current_offset;
    u_int64_t mask_with_offset = STORAGE_CHUNK_MASK << offset;
    if (((*shm_storage) & mask_with_offset) && (serpent_preferences & mask_with_offset))
    {
        u_int64_t storage_amount = get_amount_src(shm_storage, offset);
        if (max_offset == -1 || max_offset != -1 && storage_amount > get_amount_src(shm_storage, max_offset))
        {
            max_offset = offset;
        }
    }

    if (current_offset != 0)
    {
        bool result = serpent_head_rec(serpent_preferences, current_offset - 1, max_offset);
        return result;
    }

    if (max_offset == -1)
    {
        return false;
    }

    switch (max_offset)
    {
    case CAKE_OFFSET:
        cakes_ate++;
        break;
    case CANDY_OFFSET:
        candies_ate++;
        break;
    case BROWNIE_OFFSET:
        brownies_ate++;
        break;
    case GINGERBREAD_OFFSET:
        gingerbreads_ate++;
        break;
    }
    set_amount_src(shm_storage, max_offset, get_amount_src(shm_storage, max_offset) - 1);

    return true;
}

void create_serpent(u_int64_t *data)
{
    pid_t serpent_pid = fork();
    if (serpent_pid == -1)
    {
        fprintf(stderr, "Не удалось создать голову змея.\n");
        exit(1);
    }

    if (serpent_pid != 0)
    {
        PIDS[pids_size++] = serpent_pid;
        return;
    }

    attach();
    while (is_running)
    {
        sem_wait(shm_sem);
        bool result = serpent_head_rec(data[0], MAX_CHUNKS_AMOUNT - 1, -1);
        sem_post(shm_sem);
        if (result)
            usleep(data[1] * 1000);
    }
    printf("Змей Горыныч умер!\n");
    print_report();
    exit(0);
}

// Нам прилетел SIGUSR1, ставим флаг на "работающий" на false.
// Можно было бы убивать процесс вручную, однако так делать не стоит.
// Нужно дать процессу завершиться самому
void sigusr1()
{
    printf("Нужно прекращать работу!\n");
    is_running = false;
}

// Здесь мы присоединяемся к семафору и shared memory
void attach()
{
    signal(SIGUSR1, sigusr1);

    // storage должен существовать в обязательном порядке
    int storage_descriptor = shm_open(STORAGE_SHM_KEY,
                                      O_RDWR,
                                      0644);
    if (storage_descriptor < 0)
    {
        fprintf(stderr, "Невозможно получить доступ к складу.\n");
        exit(1);
    }
    ftruncate(storage_descriptor, sizeof(storage));

    caddr_t storage_ptr = mmap(NULL,
                               sizeof(storage),
                               PROT_READ | PROT_WRITE,
                               MAP_SHARED,
                               storage_descriptor,
                               0);
    shm_storage = (u_int64_t *)storage_ptr;

    // Семафор также должен существовать
    shm_sem = sem_open(SEMAPHORE_KEY, 0, 0644);

    if (shm_sem == SEM_FAILED)
    {
        fprintf(stderr, "Невозможно получить доступ к семафору.\n");
        exit(1);
    }
}

void init()
{
    // Запишем в SHM наш склад, чтобы дочерние процессы могли к нему обращаться
    // Создадим область в shared memory
    int fd = shm_open(STORAGE_SHM_KEY,
                      O_RDWR | O_CREAT,
                      0644);
    if (fd < 0)
    {
        fprintf(stderr, "Не удалось инициализировать склад.\n");
        exit(1);
    }

    // Установим размер нашего shared memory участка
    ftruncate(fd, sizeof(storage));

    // Теперь отобразим область в shared memory в адресное пространство программы и синхронизируем его
    caddr_t memptr = mmap(NULL,
                          sizeof(storage),
                          PROT_READ | PROT_WRITE,
                          MAP_SHARED,
                          fd,
                          0);
    if ((caddr_t)-1 == memptr)
        exit(1);

    // Теперь остаётся просто скопировать исходное значение storage в SHM
    memcpy(memptr, &storage, sizeof(storage));

    sem_t *semaphore = sem_open(SEMAPHORE_KEY, O_CREAT, 0644, 1);
}

void on_close()
{
    // Делаем анлинк склада и семафора
    shm_unlink(STORAGE_SHM_KEY);
    sem_unlink(SEMAPHORE_KEY);
}

int main()
{
    init();

    u_int64_t factory_a_args[] = {2,
                                  1, CAKE_OFFSET, 70,
                                  2, BROWNIE_OFFSET, 50};

    u_int64_t factory_b_args[] = {2,
                                  3, CANDY_OFFSET, 60,
                                  1, GINGERBREAD_OFFSET, 40};

    u_int64_t serpent_head_a_args[] = {(STORAGE_CHUNK_MASK << CAKE_OFFSET) |
                                           (STORAGE_CHUNK_MASK << CANDY_OFFSET),
                                       80};

    u_int64_t serpent_head_b_args[] = {(STORAGE_CHUNK_MASK << BROWNIE_OFFSET) |
                                           (STORAGE_CHUNK_MASK << GINGERBREAD_OFFSET),
                                       90};

    u_int64_t serpent_head_c_args[] = {(STORAGE_CHUNK_MASK << BROWNIE_OFFSET) |
                                           (STORAGE_CHUNK_MASK << GINGERBREAD_OFFSET) |
                                           (STORAGE_CHUNK_MASK << CAKE_OFFSET) |
                                           (STORAGE_CHUNK_MASK << CANDY_OFFSET),
                                       70};

    create_factory(factory_a_args);
    create_factory(factory_b_args);
    create_serpent(serpent_head_a_args);
    create_serpent(serpent_head_b_args);
    create_serpent(serpent_head_c_args);

    getchar();

    // Посылаем всем рабочим процессам сигнал SIGUSR1 и ждём, пока они закончатся
    for (int i = 0; i < pids_size; i++)
    {
        kill(PIDS[i], SIGUSR1);
        int status;
        waitpid(PIDS[i], &status, 0);
        while (!WIFEXITED(status))
        {
            waitpid(PIDS[i], &status, 0);
        }
    }

    on_close();

    return 0;
}
\end{minted}

\textbf{Протоколы, логи, скриншоты, графики:}\bigbreak
\textbf{Количество ядер: 6}\\
\textbf{Первая программа (threads):}\\
Вывод программы:\\
\includegraphics[width=140mm]{threads_output_6}\\
Ресуры:\\
\includegraphics[width=140mm]{threads_resources_6}\\
\textbf{Вторая программа (threads\_emul):}\\
Вывод программы:\\
\includegraphics[width=140mm]{threads_emul_output_6}\\
Ресуры:\\
\includegraphics[width=140mm]{threads_emul_resources_6}\\
\textbf{Третья программа (processes):}\\
Вывод программы:\\
\includegraphics[width=70mm]{processes_output_6_1}
\includegraphics[width=70mm]{processes_output_6_2}
\includegraphics[width=70mm]{processes_output_6_3}
\includegraphics[width=70mm]{processes_output_6_4}
\includegraphics[width=70mm]{processes_output_6_5}
\includegraphics[width=70mm]{processes_output_6_6}
\includegraphics[width=70mm]{processes_output_6_7}\\
Ресуры:\\
\includegraphics[width=140mm]{processes_resources_6}\bigbreak
\textbf{Количество ядер: 4}\\
\textbf{Первая программа (threads):}\\
Вывод программы:\\
\includegraphics[width=140mm]{threads_output_4}\\
Ресуры:\\
\includegraphics[width=140mm]{threads_resources_4}\\
\textbf{Вторая программа (threads\_emul):}\\
Вывод программы:\\
\includegraphics[width=140mm]{threads_emul_output_4}\\
Ресуры:\\
\includegraphics[width=140mm]{threads_emul_resources_4}\\
\textbf{Третья программа (processes):}\\
Вывод программы:\\
\includegraphics[width=70mm]{processes_output_4_1}
\includegraphics[width=70mm]{processes_output_4_2}
\includegraphics[width=70mm]{processes_output_4_3}
\includegraphics[width=70mm]{processes_output_4_4}
\includegraphics[width=70mm]{processes_output_4_5}
\includegraphics[width=70mm]{processes_output_4_6}
\includegraphics[width=70mm]{processes_output_4_7}\\
Ресуры:\\
\includegraphics[width=140mm]{processes_resources_4}\bigbreak
\textbf{Количество ядер: 3}\\
\textbf{Первая программа (threads):}\\
Вывод программы:\\
\includegraphics[width=140mm]{threads_output_3}\\
Ресуры:\\
\includegraphics[width=140mm]{threads_resources_3}\\
\textbf{Вторая программа (threads\_emul):}\\
Вывод программы:\\
\includegraphics[width=140mm]{threads_emul_output_3}\\
Ресуры:\\
\includegraphics[width=140mm]{threads_emul_resources_3}\\
\textbf{Третья программа (processes):}\\
Вывод программы:\\
\includegraphics[width=70mm]{processes_output_3_1}
\includegraphics[width=70mm]{processes_output_3_2}
\includegraphics[width=70mm]{processes_output_3_3}
\includegraphics[width=70mm]{processes_output_3_4}
\includegraphics[width=70mm]{processes_output_3_5}
\includegraphics[width=70mm]{processes_output_3_6}
\includegraphics[width=70mm]{processes_output_3_7}\\
Ресуры:\\
\includegraphics[width=140mm]{processes_resources_3}\bigbreak
\textbf{Количество ядер: 2}\\
\textbf{Первая программа (threads):}\\
Вывод программы:\\
\includegraphics[width=140mm]{threads_output_2}\\
Ресуры:\\
\includegraphics[width=140mm]{threads_resources_2}\\
\textbf{Вторая программа (threads\_emul):}\\
Вывод программы:\\
\includegraphics[width=140mm]{threads_emul_output_2}\\
Ресуры:\\
\includegraphics[width=140mm]{threads_emul_resources_2}\\
\textbf{Третья программа (processes):}\\
Вывод программы:\\
\includegraphics[width=70mm]{processes_output_2_1}
\includegraphics[width=70mm]{processes_output_2_2}
\includegraphics[width=70mm]{processes_output_2_3}
\includegraphics[width=70mm]{processes_output_2_4}
\includegraphics[width=70mm]{processes_output_2_5}
\includegraphics[width=70mm]{processes_output_2_6}
\includegraphics[width=70mm]{processes_output_2_7}\\
Ресуры:\\
\includegraphics[width=140mm]{processes_resources_2}



\textbf{Вывод: } в ходе лабораторной работы изучили различия между процессами и потоками в ОС Linux (Ubuntu), а также
освоить механизмы синхронизации и межпроцессорного взаимодействия для обеспечения корректной работы программ в многозадачной среде. 
Потоки, хоть в Linux и являются по сути процессами, однако имеют с родительсикм потоком общее адресное пространство. 
Процессы же имеют собственное адресное пространство. Для синхронизации можно использовать частный случай семафора - 
мьютекс, рекурсивный мьютекс. Создать семафор для межпроцессного взаимодействия можно при помощи sem\_open. По сути 
эта команда инициализирует семафор, затем записывает его в shared memory и вызывает mmap для отображения созданного 
файла в виртуальной файловой системе с семафором в адресное пространство. Была также разработана программа, моделирующая 
указанное поведение. Данная программа более справедлива в генерации продуктов, так как при разных конфигурациях системы
было испечено одинаковое количество продуктов, в отличие от программ, выполняющих реальную работу. Однако полученная 
программа непригодна для выполнения реальной работы, так как она лишь моделирует заданную систему. Также она занимает больше процессорных ресурсов, 
потому что использует цикл без ожиданий. Стоит также отметить, 
что shared-ресурсов для потоков оказалось больше, так как потоки имеют доступ ко всему адресному пространству родителя, 
в то время как процессы получают лишь ограниченное количество, выделенное при помощи shm\_open. В целом, программы написанные
при использовании потоков и процессов, теряли производительность с уменьшением количество ядер одинаково. Программа для моделирования
системы производительность не теряла с уменьшением количества процессоров. Также стоит отметить, что данные моделирующие программы 
не совпали с полученными результатами выполнения настоящих программ. 

\end{document}