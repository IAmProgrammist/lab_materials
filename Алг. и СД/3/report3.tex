\documentclass[a4paper,14pt]{extarticle}


\usepackage[english,russian]{babel}
\usepackage[T2A]{fontenc}
\usepackage[utf8]{inputenc}
\usepackage{ragged2e}
\usepackage[utf8]{inputenc}
\usepackage{hyperref}
\usepackage{minted}
\setmintedinline{frame=lines, framesep=2mm, baselinestretch=1.5, bgcolor=LightGray, breaklines,fontsize=\footnotesize}
\setminted{frame=lines, framesep=2mm, baselinestretch=1.5, bgcolor=LightGray, breaklines,fontsize=\footnotesize}
\usepackage{xcolor}
\definecolor{LightGray}{gray}{0.9}
\usepackage{graphicx}
\usepackage[export]{adjustbox}
\usepackage[left=1cm,right=1cm, top=1cm,bottom=1cm,bindingoffset=0cm]{geometry}
\usepackage{fontspec}
\usepackage{ upgreek }
\usepackage[shortlabels]{enumitem}
\usepackage{adjustbox}
\usepackage{multirow}
\usepackage{amsmath}
\usepackage{amssymb}
\usepackage{pifont}
\usepackage{pgfplots}
\usepackage{longtable}
\usepackage{array}
\graphicspath{ {./images/} }
\makeatletter
\AddEnumerateCounter{\asbuk}{\russian@alph}{щ}
\makeatother
\setmonofont{Consolas}
\setmainfont{Times New Roman}

\newcommand\textbox[1]{
	\parbox{.45\textwidth}{#1}
}

\newcommand{\specialcell}[2][c]{%
	\begin{tabular}[#1]{@{}c@{}}#2\end{tabular}}

\begin{document}
\pagenumbering{gobble}
\begin{center}
	\small{
		МИНИCТЕРCТВО НАУКИ И ВЫCШЕГО ОБРАЗОВАНИЯ \\РОCCИЙCКОЙ ФЕДЕРАЦИИ
		\bigbreak
		ФЕДЕРАЛЬНОЕ ГОCУДАРCТВЕННОЕ БЮДЖЕТНОЕ ОБРАЗОВАТЕЛЬНОЕ УЧРЕЖДЕНИЕ ВЫCШЕГО ОБРАЗОВАНИЯ \\
		\bigbreak
		\textbf{«БЕЛГОРОДCКИЙ ГОCУДАРCТВЕННЫЙ \\ТЕХНОЛОГИЧЕCКИЙ УНИВЕРCИТЕТ им. В. Г. ШУХОВА»\\ (БГТУ им. В.Г. Шухова)} \\
		\bigbreak
		Кафедра программного обеспечения вычислительной техники и автоматизированных систем\\}
\end{center}

\vfill
\begin{center}
	\large{
		\textbf{
			Лабораторная работа №3}}\\
	\normalsize{
		по дисциплине: Алгоритмы и структуры данных \\
		тема: Сравнительный анализ методов сортировки (Pascal/C)}
\end{center}
\vfill
\hfill\textbox{
	Выполнил: ст. группы ПВ-223\\Пахомов Владислав Андреевич
	\bigbreak
	Проверили: асс. Солонченко Роман\\Евгеньевич
}
\vfill\begin{center}
	Белгород 2023 г.
\end{center}
\newpage
\begin{center}
	\textbf{Лабораторная работа №3}\\
	Сравнительный анализ методов сортировки (Pascal/C)
\end{center}
\textbf{Цель работы: }изучение методов сортировки массивов и приобретение
навыков в проведении сравнительного анализа различных методов сортировки.
\begin{enumerate}
	\item Листинг программы.
	      \textit{main.c}
	      \begin{minted}{C}
#include <time.h>

#include <algc.h>

#define LOW 5
#define HIGH 45
#define STEP 5

int main() {
    srand(time(0));

    comparesExperiment(insertionSort, "insertion sort", LOW, HIGH, STEP);
    comparesExperiment(selectionSort, "selection sort", LOW, HIGH, STEP);
    comparesExperiment(bubbleSort, "bubble sort", LOW, HIGH, STEP);
    comparesExperiment(bubbleSortMod1, "bubble sort first modification", LOW, HIGH, STEP);
    comparesExperiment(bubbleSortMod2, "bubble sort second modification", LOW, HIGH, STEP);
    comparesExperiment(shellSort, "shell sort", LOW, HIGH, STEP);
    comparesExperiment(hoarSort, "quick sort", LOW, HIGH, STEP);
    comparesExperiment(heapSort, "heap sort", LOW, HIGH, STEP);

    return 0;
}
	\end{minted}
	      \textit{sorts.h}
	      \begin{minted}{C}
#ifndef SORTS
#define SORTS

#include <stdbool.h>

typedef void (*SortingFunction)(int*, int, int*);

#define INC_COMPARES(comps) ((!comps || ++(*comps)))

void swap(int* a, int* b);
bool isOrdered(int* a, int size);
bool isOrderedBackwards(int* a, int size);
void genOrdered(int *a, int size);
void genOrderedBackwards(int *a, int size);
void genRandom(int *a, int size);

void insertionSort(int* data, int size, int* comps);
void selectionSort(int* data, int size, int* comps);
void bubbleSort(int* data, int size, int* comps);
void bubbleSortMod1(int* data, int size, int* comps);
void bubbleSortMod2(int* data, int size, int* comps);
void shellSort(int* data, int size, int* comps);
void hoarSort(int* data, int size, int* comps);
void heapSort(int* data, int size, int* comps);

void comparesExperiment(SortingFunction function, char *sortingFunctionName, int low, int high, int step);

#endif
	\end{minted}
	      \textit{utility.c}
	      \begin{minted}{C}
#include <lab3/sorts.h>

#include <assert.h>
#include <limits.h>
#include <stdlib.h>
#include <stdbool.h>
#include <stdio.h>

void swap(int* a, int* b) {
    int t = *a;
    *a = *b;
    *b = t;
}

bool isOrdered(int* a, int size) {
    for (int i = 0; i < size - 1; i++)
        if (a[i] > a[i + 1]) return false;

    return true;
}

bool isOrderedBackwards(int* a, int size) {
    for (int i = 0; i < size - 1; i++) 
        if (a[i] < a[i + 1]) return false;
    
    return true;
}

void genOrdered(int *a, int size) {
    int current = INT_MIN;

    for (int i = 0; i < size; i++) {
        a[i] = current;
        current += rand() % RAND_MAX;
    }
}

void genOrderedBackwards(int *a, int size) {
    int current = INT_MAX;

    for (int i = 0; i < size; i++) {
        a[i] = current;
        current -= rand() % RAND_MAX;
    }
}

void genRandom(int *a, int size) {
    for (int i = 0; i < size; i++)
        a[i] = rand() % RAND_MAX;
}

void comparesExperiment(SortingFunction function, char *sortingFunctionName, int low, int high, int step) {
    printf("==============================================\n\n");
    printf("Launching comparing experiment for function %s\n", sortingFunctionName);

    printf("\nOrdered array results:\n");
    for (int i = low; i <= high; i += step) {
        int* array = malloc(i * sizeof(int));
        genOrdered(array, i);
        int compares = 0;
        function(array, i, &compares);
        assert(isOrdered(array, i));
        printf("For %3d elements: %7d compares\n", i, compares);
        free(array);
    }

    printf("\nOrdered backwards array results:\n");
    for (int i = low; i <= high; i += step) {
        int* array = malloc(i *sizeof(int));
        genOrderedBackwards(array, i);
        int compares = 0;
        function(array, i, &compares);
        assert(isOrdered(array, i));
        printf("For %3d elements: %7d compares\n", i, compares);
        free(array);
    }

    printf("\nRandom order array results:\n");
    for (int i = low; i <= high; i += step) {
        int* array = malloc(i * sizeof(int));
        genRandom(array, i);
        int compares = 0;
        function(array, i, &compares);
        assert(isOrdered(array, i));
        printf("For %3d elements: %7d compares\n", i, compares);
        free(array);
    }

    printf("\n");
}
	\end{minted}
	      \textit{insertionsort.c}
	      \begin{minted}{C}
#include <lab3/sorts.h>

#include <string.h>

void insertionSort(int* data, int size, int* comps) {
    for (int i = 1; INC_COMPARES(comps) && i < size; i++) {
        int j = i - 1;
        while (INC_COMPARES(comps) && j >= 0 && INC_COMPARES(comps) && data[j] > data[i])
            j--;

        int element = data[i];
        memcpy(data + j + 2, data + j + 1, (i - j - 1) * sizeof(int));
        data[j + 1] = element;
    }
}
	\end{minted}
	      \textit{selectionsort.c}
	      \begin{minted}{C}
#include <lab3/sorts.h>

void selectionSort(int* data, int size, int* comps) {
    for (int i = 0; INC_COMPARES(comps) && i < size - 1; i++) {
        int minIndex = i;
        for (int j = i + 1; INC_COMPARES(comps) && j < size; j++) {
            if (INC_COMPARES(comps) && data[j] < data[minIndex])
                minIndex = j;
        }

        swap(data + i, data + minIndex);
    }
}
	\end{minted}
	      \textit{bubblesort.c}
	      \begin{minted}{C}
#include <lab3/sorts.h>

void bubbleSort(int* data, int size, int* comps) {
    for (int i = 0; INC_COMPARES(comps) && i < size; i++) {
        for (int j = size - 1; INC_COMPARES(comps) && j > i; j--) {
            if (INC_COMPARES(comps) && data[j] < data[j - 1])
                swap(data + j, data + j - 1);
        }
    }
}
	\end{minted}
	      \textit{bubblesortmod1.c}
	      \begin{minted}{C}
#include <lab3/sorts.h>

#include <stdbool.h>

void bubbleSortMod1(int* data, int size, int* comps) {
    for (int i = 0; INC_COMPARES(comps) && i < size; i++) {
        bool anyCompsDone = false;

        for (int j = size - 1; INC_COMPARES(comps) && j > i; j--) {
            if (INC_COMPARES(comps) && data[j] < data[j - 1]) {
                swap(data + j, data + j - 1);
                anyCompsDone = true;
            }
        }

        if (!anyCompsDone)
            break;
    }
}
	\end{minted}
	      \textit{bubblesortmod2.c}
	      \begin{minted}{C}
#include <lab3/sorts.h>

#include <stdbool.h>

void bubbleSortMod2(int *data, int size, int *comps)
{
    for (int i = 0; INC_COMPARES(comps) && i < size; i++) {
        int k = size;

        for (int j = size - 1; INC_COMPARES(comps) && j > i; j--) {
            if (INC_COMPARES(comps) && data[j] < data[j - 1]) {
                swap(data + j, data + j - 1);
                k = j - 1;
            }
        }

        i = k;
    }
}
	\end{minted}
	      \textit{shellsort.c}
	      \begin{minted}{C}
#include <lab3/sorts.h>

#include <math.h>

void shellSort(int* data, int size, int* comps) {
    int t = log2(size) / log2(3) - 1;
    t = INC_COMPARES(comps) && t < 0 ? 0 : t;

    int h = 1;
    for (int i = 0; INC_COMPARES(comps) && i < t; i++) 
        h = h * 3 + 1;

    while (INC_COMPARES(comps) && h >= 1) {
        for (int i = h; INC_COMPARES(comps) && i < size; i++) {
            int j = i;
            
            while (INC_COMPARES(comps) && j - h >= 0 && INC_COMPARES(comps) && data[j - h] > data[j]) {
                swap(data + j - h, data + j);
                j -= h;
            }
        }

        h = (h - 1) / 3;
    }
}
	\end{minted}
	      \textit{hoarsort.c}
	      \begin{minted}{C}
#include <lab3/sorts.h>

#include <malloc.h>

void hoarSort(int *data, int size, int *comps) {
    if (INC_COMPARES(comps) && size <= 1) return;

    int i = 0, j = size - 1;
    int midElement = data[size / 2];
    while (INC_COMPARES(comps)) {
        if (INC_COMPARES(comps) && data[i] < midElement) {
            i++;
            continue;
        }

        if (INC_COMPARES(comps) && data[j] > midElement) {
            j--;
            continue;
        }

        if (INC_COMPARES(comps) && i >= j)
            break;

        swap(data + (i++), data + (j--));
    }

    hoarSort(data, i, comps);
    hoarSort(data + i, size - i, comps);
}
	\end{minted}
	      \textit{heapsort.c}
	      \begin{minted}{C}
#include <lab3/sorts.h>

void heapSort(int* data, int size, int* comps) {
    
    // Создаём кучу
    for (int i = 1; INC_COMPARES(comps) && i < size; i++) {
        int currentIndex = i;
        int parentIndex = (currentIndex - 1) / 2;

        while (INC_COMPARES(comps) && currentIndex != 0 && INC_COMPARES(comps) && data[parentIndex] < data[currentIndex]) {
            swap(data + parentIndex, data + currentIndex);
            currentIndex = parentIndex;
            parentIndex = (currentIndex - 1) / 2;
        }    
    }

    // Удаляем из кучи элементы
    for (int i = 0; INC_COMPARES(comps) && i < size - 1; i++) {
        swap(data, data + size - 1 - i);

        int currentIndex = 0;
        int heapSize = size - i - 1;
        while (1) {
            int leftChildIndex = 2 * currentIndex + 1; 
            int rightChildIndex = 2 * currentIndex + 2;

            if (INC_COMPARES(comps) && leftChildIndex < heapSize && INC_COMPARES(comps) && rightChildIndex < heapSize) {
                if (INC_COMPARES(comps) && data[currentIndex] < data[leftChildIndex] && 
                    INC_COMPARES(comps) && data[leftChildIndex] > data[rightChildIndex]) {
                    swap(data + leftChildIndex, data + currentIndex);
                    currentIndex = leftChildIndex;
                } else if (INC_COMPARES(comps) && data[currentIndex] < data[rightChildIndex] && 
                           INC_COMPARES(comps) && data[rightChildIndex] > data[leftChildIndex]) {
                    swap(data + rightChildIndex, data + currentIndex);
                    currentIndex = rightChildIndex;
                } else break;
            } else if (INC_COMPARES(comps) && leftChildIndex < heapSize && INC_COMPARES(comps) && data[currentIndex] < data[leftChildIndex]) {
                swap(data + leftChildIndex, data + currentIndex);
                break;
            } else if (INC_COMPARES(comps) && rightChildIndex < heapSize && INC_COMPARES(comps) && data[rightChildIndex] > data[rightChildIndex]) {
                swap(data + rightChildIndex, data + currentIndex);
                break;
            } else break;
        }
    }
}
	\end{minted}
	\item Результаты работы программы.
	      \begin{center}
		      \textbf{Результаты экспериментов для упорядоченных по возрастанию массивов}
		      \begin{tabular}{|c|c|c|c|c|c|c|c|c|c|}
			      \hline
			      \multirow{2}{*}{Сортировка} & \multicolumn{9}{|c|}{Количество элементов в массиве}                                                    \\
			      \cline{2-10}
			                                  & 5                                                    & 10  & 15  & 20  & 25  & 30  & 35   & 40   & 45   \\
			      \hline
			      Включением                  & 13                                                   & 28  & 43  & 58  & 73  & 88  & 103  & 118  & 133  \\
			      \hline
			      Выбором                     & 29                                                   & 109 & 239 & 419 & 649 & 929 & 1259 & 1639 & 2069 \\
			      \hline
			      Обменом                     & 31                                                   & 111 & 241 & 421 & 651 & 931 & 1261 & 1641 & 2071 \\
			      \hline
			      Обменом 1                   & 10                                                   & 20  & 30  & 40  & 50  & 60  & 70   & 80   & 90   \\
			      \hline
			      Обменом 2                   & 11                                                   & 21  & 31  & 41  & 51  & 61  & 71   & 81   & 91   \\
			      \hline
			      Шелла                       & 17                                                   & 53  & 83  & 113 & 143 & 227 & 272  & 317  & 362  \\
			      \hline
			      Хоара                       & 44                                                   & 115 & 192 & 282 & 373 & 461 & 561  & 666  & 771  \\
			      \hline
			      Пирамидальная               & 50                                                   & 147 & 264 & 403 & 551 & 705 & 871  & 1047 & 1229 \\ \hline
		      \end{tabular}\bigbreak
		      \textbf{Результаты экспериментов для упорядоченных по убыванию массивов}
		      \begin{tabular}{|c|c|c|c|c|c|c|c|c|c|}
			      \hline
			      \multirow{2}{*}{Сортировка} & \multicolumn{9}{|c|}{Количество элементов в массиве}                                                    \\
			      \cline{2-10}
			                                  & 5                                                    & 10  & 15  & 20  & 25  & 30  & 35   & 40   & 45   \\
			      \hline
			      Включением                  & 29                                                   & 109 & 239 & 419 & 649 & 929 & 1259 & 1639 & 2069 \\
			      \hline
			      Выбором                     & 29                                                   & 109 & 239 & 419 & 649 & 929 & 1259 & 1639 & 2069 \\
			      \hline
			      Обменом                     & 31                                                   & 111 & 241 & 421 & 651 & 931 & 1261 & 1641 & 2071 \\
			      \hline
			      Обменом 1                   & 30                                                   & 110 & 240 & 420 & 650 & 930 & 1260 & 1640 & 2070 \\
			      \hline
			      Обменом 2                   & 31                                                   & 111 & 241 & 421 & 651 & 931 & 1261 & 1641 & 2071 \\
			      \hline
			      Шелла                       & 33                                                   & 72  & 136 & 234 & 290 & 333 & 403  & 548  & 535  \\
			      \hline
			      Хоара                       & 42                                                   & 113 & 185 & 275 & 361 & 449 & 544  & 649  & 749  \\
			      \hline
			      Пирамидальная               & 42                                                   & 119 & 217 & 339 & 437 & 571 & 700  & 816  & 984  \\ \hline
		      \end{tabular}\bigbreak
		      \textbf{Результаты экспериментов для неупорядоченных массивов}
		      \begin{tabular}{|c|c|c|c|c|c|c|c|c|c|}
			      \hline
			      \multirow{2}{*}{Сортировка} & \multicolumn{9}{|c|}{Количество элементов в массиве}                                                    \\
			      \cline{2-10}
			                                  & 5                                                    & 10  & 15  & 20  & 25  & 30  & 35   & 40   & 45   \\
			      \hline
			      Включением                  & 5                                                    & 22  & 50  & 142 & 263 & 459 & 520  & 796  & 1245 \\
			      \hline
			      Выбором                     & 29                                                   & 109 & 239 & 419 & 649 & 929 & 1259 & 1639 & 2069 \\
			      \hline
			      Обменом                     & 31                                                   & 111 & 241 & 421 & 651 & 931 & 1261 & 1641 & 2071 \\
			      \hline
			      Обменом 1                   & 30                                                   & 98  & 184 & 400 & 648 & 888 & 1188 & 1550 & 2068 \\
			      \hline
			      Обменом 2                   & 29                                                   & 99  & 175 & 349 & 645 & 785 & 1195 & 1391 & 1885 \\
			      \hline
			      Шелла                       & 29                                                   & 77  & 136 & 223 & 292 & 345 & 465  & 494  & 639  \\
			      \hline
			      Хоара                       & 51                                                   & 117 & 237 & 325 & 489 & 543 & 616  & 869  & 829  \\
			      \hline
			      Пирамидальная               & 45                                                   & 142 & 240 & 366 & 502 & 635 & 777  & 916  & 1069 \\ \hline
		      \end{tabular}\\
			\end{center}
		     \item Графики зависимостей ФВС.\\
		     \begin{center} 
			  \begin{longtable}{|>{\centering\arraybackslash}p{0.13\textwidth}|>{\centering\arraybackslash}p{0.27\textwidth}|>{\centering\arraybackslash}p{0.27\textwidth}|>{\centering\arraybackslash}p{0.27\textwidth}|}
			      \hline
			      Сортировка                                      & \begin{tabular}{c}Упоряд. по воз-\\растанию массив\end{tabular} & \begin{tabular}{c}Упоряд. по убы-\\ванию массив\end{tabular} & \begin{tabular}{c}Неупорядочен-\\ный массив\end{tabular} \\
			      \hline
			      \begin{tabular}{c}Включе-\\нием\end{tabular}    & \begin{tikzpicture}
				                                                        \begin{axis}[
						      axis lines = left,
						      xlabel = \(x\),
						      ylabel = {\(y\)},
						      width=0.25\textwidth,
						      height=0.25\textwidth
					      ]
					      \addplot[
						      mark size=2pt,
						      only marks,
					      ]
					      table {
							      x  y
							      5  13
							      10 28
							      15 43
							      20 58
							      25 73
							      30 88
							      35 103
							      40 118
							      45 133
						      };
					      \addplot[
						      mark size=0pt,
						      only marks,
					      ]
					      table {
							      x  y
							      0  0
						      };
				      \end{axis}
			                                                        \end{tikzpicture}                                      & \begin{tikzpicture}
				                                                                                                                 \begin{axis}[
						      axis lines = left,
						      xlabel = \(x\),
						      ylabel = {\(y\)},
						      width=0.25\textwidth,
						      height=0.25\textwidth
					      ]
					      \addplot[
						      mark size=2pt,
						      only marks,
					      ]
					      table {
							      x  y
							      5  29
							      10 109
							      15 239
							      20 419
							      25 649
							      30 929
							      35 1259
							      40 1639
							      45 2069
						      };
					      \addplot[
						      mark size=0pt,
						      only marks,
					      ]
					      table {
							      x  y
							      0  0
						      };
				      \end{axis}
			                                                                                                                 \end{tikzpicture}                                   & \begin{tikzpicture}
				                                                                                                                                                                       \begin{axis}[
						      axis lines = left,
						      xlabel = \(x\),
						      ylabel = {\(y\)},
						      width=0.25\textwidth,
						      height=0.25\textwidth
					      ]
					      \addplot[
						      mark size=2pt,
						      only marks,
					      ]
					      table {
							      x  y
							      5  5
							      10 22
							      15 50
							      20 142
							      25 263
							      30 459
							      35 520
							      40 796
							      45 1245
						      };
					      \addplot[
						      mark size=0pt,
						      only marks,
					      ]
					      table {
							      x  y
							      0  0
						      };
				      \end{axis}
			                                                                                                                                                                       \end{tikzpicture}                                                  \\
			      \hline
			      Выбором                                         & \begin{tikzpicture}
				                                                        \begin{axis}[
						      axis lines = left,
						      xlabel = \(x\),
						      ylabel = {\(y\)},
						      width=0.25\textwidth,
						      height=0.25\textwidth
					      ]
					      \addplot[
						      mark size=2pt,
						      only marks,
					      ]
					      table {
							      x  y
							      5  29
							      10 109
							      15 239
							      20 419
							      25 649
							      30 929
							      35 1259
							      40 1639
							      45 2069
						      };
					      \addplot[
						      mark size=0pt,
						      only marks,
					      ]
					      table {
							      x  y
							      0  0
						      };
				      \end{axis}
			                                                        \end{tikzpicture}                                      & \begin{tikzpicture}
				                                                                                                                 \begin{axis}[
						      axis lines = left,
						      xlabel = \(x\),
						      ylabel = {\(y\)},
						      width=0.25\textwidth,
						      height=0.25\textwidth
					      ]
					      \addplot[
						      mark size=2pt,
						      only marks,
					      ]
					      table {
							      x  y
							      5  29
							      10 109
							      15 239
							      20 419
							      25 649
							      30 929
							      35 1259
							      40 1639
							      45 2069
						      };
					      \addplot[
						      mark size=0pt,
						      only marks,
					      ]
					      table {
							      x  y
							      0  0
						      };
				      \end{axis}
			                                                                                                                 \end{tikzpicture}                                   & \begin{tikzpicture}
				                                                                                                                                                                       \begin{axis}[
						      axis lines = left,
						      xlabel = \(x\),
						      ylabel = {\(y\)},
						      width=0.25\textwidth,
						      height=0.25\textwidth
					      ]
					      \addplot[
						      mark size=2pt,
						      only marks,
					      ]
					      table {
							      x  y
							      5  29
							      10 109
							      15 239
							      20 419
							      25 649
							      30 929
							      35 1259
							      40 1639
							      45 2069
						      };
					      \addplot[
						      mark size=0pt,
						      only marks,
					      ]
					      table {
							      x  y
							      0  0
						      };
				      \end{axis}
			                                                                                                                                                                       \end{tikzpicture}                                                  \\
			      \hline
			      Обменом                                         & \begin{tikzpicture}
				                                                        \begin{axis}[
						      axis lines = left,
						      xlabel = \(x\),
						      ylabel = {\(y\)},
						      width=0.25\textwidth,
						      height=0.25\textwidth
					      ]
					      \addplot[
						      mark size=2pt,
						      only marks,
					      ]
					      table {
							      x  y
							      5  31
							      10 111
							      15 241
							      20 421
							      25 651
							      30 931
							      35 1261
							      40 1641
							      45 2071
						      };
					      \addplot[
						      mark size=0pt,
						      only marks,
					      ]
					      table {
							      x  y
							      0  0
						      };
				      \end{axis}
			                                                        \end{tikzpicture}                                      & \begin{tikzpicture}
				                                                                                                                 \begin{axis}[
						      axis lines = left,
						      xlabel = \(x\),
						      ylabel = {\(y\)},
						      width=0.25\textwidth,
						      height=0.25\textwidth
					      ]
					      \addplot[
						      mark size=2pt,
						      only marks,
					      ]
					      table {
							      x  y
							      5  31
							      10 111
							      15 241
							      20 421
							      25 651
							      30 931
							      35 1261
							      40 1641
							      45 2071
						      };
					      \addplot[
						      mark size=0pt,
						      only marks,
					      ]
					      table {
							      x  y
							      0  0
						      };
				      \end{axis}
			                                                                                                                 \end{tikzpicture}                                   & \begin{tikzpicture}
				                                                                                                                                                                       \begin{axis}[
						      axis lines = left,
						      xlabel = \(x\),
						      ylabel = {\(y\)},
						      width=0.25\textwidth,
						      height=0.25\textwidth
					      ]
					      \addplot[
						      mark size=2pt,
						      only marks,
					      ]
					      table {
							      x  y
							      5  31
							      10 111
							      15 241
							      20 421
							      25 651
							      30 931
							      35 1261
							      40 1641
							      45 2071
						      };
					      \addplot[
						      mark size=0pt,
						      only marks,
					      ]
					      table {
							      x  y
							      0  0
						      };
				      \end{axis}
			                                                                                                                                                                       \end{tikzpicture}                                                  \\
			      \hline
			      Обменом 1                                       & \begin{tikzpicture}
				                                                        \begin{axis}[
						      axis lines = left,
						      xlabel = \(x\),
						      ylabel = {\(y\)},
						      width=0.25\textwidth,
						      height=0.25\textwidth
					      ]
					      \addplot[
						      mark size=2pt,
						      only marks,
					      ]
					      table {
							      x  y
							      5  10
							      10 20
							      15 30
							      20 40
							      25 50
							      30 60
							      35 70
							      40 80
							      45 90
						      };
					      \addplot[
						      mark size=0pt,
						      only marks,
					      ]
					      table {
							      x  y
							      0  0
						      };
				      \end{axis}
			                                                        \end{tikzpicture}                                      & \begin{tikzpicture}
				                                                                                                                 \begin{axis}[
						      axis lines = left,
						      xlabel = \(x\),
						      ylabel = {\(y\)},
						      width=0.25\textwidth,
						      height=0.25\textwidth
					      ]
					      \addplot[
						      mark size=2pt,
						      only marks,
					      ]
					      table {
							      x  y
							      5  30
							      10 110
							      15 240
							      20 420
							      25 650
							      30 930
							      35 1260
							      40 1640
							      45 2070
						      };
					      \addplot[
						      mark size=0pt,
						      only marks,
					      ]
					      table {
							      x  y
							      0  0
						      };
				      \end{axis}
			                                                                                                                 \end{tikzpicture}                                   & \begin{tikzpicture}
				                                                                                                                                                                       \begin{axis}[
						      axis lines = left,
						      xlabel = \(x\),
						      ylabel = {\(y\)},
						      width=0.25\textwidth,
						      height=0.25\textwidth
					      ]
					      \addplot[
						      mark size=2pt,
						      only marks,
					      ]
					      table {
							      x  y
							      5  30
							      10 98
							      15 184
							      20 400
							      25 648
							      30 888
							      35 1188
							      40 1550
							      45 2068
						      };
					      \addplot[
						      mark size=0pt,
						      only marks,
					      ]
					      table {
							      x  y
							      0  0
						      };
				      \end{axis}
			                                                                                                                                                                       \end{tikzpicture}                                                  \\
			      \hline
			      Обменом 2                                       & \begin{tikzpicture}
				                                                        \begin{axis}[
						      axis lines = left,
						      xlabel = \(x\),
						      ylabel = {\(y\)},
						      width=0.25\textwidth,
						      height=0.25\textwidth
					      ]
					      \addplot[
						      mark size=2pt,
						      only marks,
					      ]
					      table {
							      x  y
							      5  11
							      10 21
							      15 31
							      20 41
							      25 51
							      30 61
							      35 71
							      40 81
							      45 91
						      };
					      \addplot[
						      mark size=0pt,
						      only marks,
					      ]
					      table {
							      x  y
							      0  0
						      };
				      \end{axis}
			                                                        \end{tikzpicture}                                      & \begin{tikzpicture}
				                                                                                                                 \begin{axis}[
						      axis lines = left,
						      xlabel = \(x\),
						      ylabel = {\(y\)},
						      width=0.25\textwidth,
						      height=0.25\textwidth
					      ]
					      \addplot[
						      mark size=2pt,
						      only marks,
					      ]
					      table {
							      x  y
							      5  31
							      10 111
							      15 241
							      20 421
							      25 651
							      30 931
							      35 1261
							      40 1641
							      45 2071
						      };
					      \addplot[
						      mark size=0pt,
						      only marks,
					      ]
					      table {
							      x  y
							      0  0
						      };
				      \end{axis}
			                                                                                                                 \end{tikzpicture}                                   & \begin{tikzpicture}
				                                                                                                                                                                       \begin{axis}[
						      axis lines = left,
						      xlabel = \(x\),
						      ylabel = {\(y\)},
						      width=0.25\textwidth,
						      height=0.25\textwidth
					      ]
					      \addplot[
						      mark size=2pt,
						      only marks,
					      ]
					      table {
							      x  y
							      5  29
							      10 99
							      15 175
							      20 349
							      25 645
							      30 785
							      35 1195
							      40 1391
							      45 1885
						      };
					      \addplot[
						      mark size=0pt,
						      only marks,
					      ]
					      table {
							      x  y
							      0  0
						      };
				      \end{axis}
			                                                                                                                                                                       \end{tikzpicture}                                                  \\
			      \hline
			      Шелла                                           & \begin{tikzpicture}
				                                                        \begin{axis}[
						      axis lines = left,
						      xlabel = \(x\),
						      ylabel = {\(y\)},
						      width=0.25\textwidth,
						      height=0.25\textwidth
					      ]
					      \addplot[
						      mark size=2pt,
						      only marks,
					      ]
					      table {
							      x  y
							      5  17
							      10 53
							      15 83
							      20 113
							      25 143
							      30 227
							      35 272
							      40 317
							      45 362
						      };
					      \addplot[
						      mark size=0pt,
						      only marks,
					      ]
					      table {
							      x  y
							      0  0
						      };
				      \end{axis}
			                                                        \end{tikzpicture}                                      & \begin{tikzpicture}
				                                                                                                                 \begin{axis}[
						      axis lines = left,
						      xlabel = \(x\),
						      ylabel = {\(y\)},
						      width=0.25\textwidth,
						      height=0.25\textwidth
					      ]
					      \addplot[
						      mark size=2pt,
						      only marks,
					      ]
					      table {
							      x  y
							      5  33
							      10 72
							      15 136
							      20 234
							      25 290
							      30 333
							      35 403
							      40 548
							      45 535
						      };
					      \addplot[
						      mark size=0pt,
						      only marks,
					      ]
					      table {
							      x  y
							      0  0
						      };
				      \end{axis}
			                                                                                                                 \end{tikzpicture}                                   & \begin{tikzpicture}
				                                                                                                                                                                       \begin{axis}[
						      axis lines = left,
						      xlabel = \(x\),
						      ylabel = {\(y\)},
						      width=0.25\textwidth,
						      height=0.25\textwidth
					      ]
					      \addplot[
						      mark size=2pt,
						      only marks,
					      ]
					      table {
							      x  y
							      5  29
							      10 77
							      15 136
							      20 223
							      25 292
							      30 345
							      35 465
							      40 494
							      45 639
						      };
					      \addplot[
						      mark size=0pt,
						      only marks,
					      ]
					      table {
							      x  y
							      0  0
						      };
				      \end{axis}
			                                                                                                                                                                       \end{tikzpicture}                                                  \\
			      \hline
			      Хоара                                           & \begin{tikzpicture}
				                                                        \begin{axis}[
						      axis lines = left,
						      xlabel = \(x\),
						      ylabel = {\(y\)},
						      width=0.25\textwidth,
						      height=0.25\textwidth
					      ]
					      \addplot[
						      mark size=2pt,
						      only marks,
					      ]
					      table {
							      x  y
							      5  44
							      10 115
							      15 192
							      20 282
							      25 373
							      30 461
							      35 561
							      40 666
							      45 771
						      };
					      \addplot[
						      mark size=0pt,
						      only marks,
					      ]
					      table {
							      x  y
							      0  0
						      };
				      \end{axis}
			                                                        \end{tikzpicture}                                      & \begin{tikzpicture}
				                                                                                                                 \begin{axis}[
						      axis lines = left,
						      xlabel = \(x\),
						      ylabel = {\(y\)},
						      width=0.25\textwidth,
						      height=0.25\textwidth
					      ]
					      \addplot[
						      mark size=2pt,
						      only marks,
					      ]
					      table {
							      x  y
							      5  42
							      10 113
							      15 185
							      20 275
							      25 361
							      30 449
							      35 544
							      40 649
							      45 749
						      };
					      \addplot[
						      mark size=0pt,
						      only marks,
					      ]
					      table {
							      x  y
							      0  0
						      };
				      \end{axis}
			                                                                                                                 \end{tikzpicture}                                   & \begin{tikzpicture}
				                                                                                                                                                                       \begin{axis}[
						      axis lines = left,
						      xlabel = \(x\),
						      ylabel = {\(y\)},
						      width=0.25\textwidth,
						      height=0.25\textwidth
					      ]
					      \addplot[
						      mark size=2pt,
						      only marks,
					      ]
					      table {
							      x  y
							      5  51
							      10 117
							      15 237
							      20 325
							      25 489
							      30 543
							      35 616
							      40 869
							      45 829
						      };
					      \addplot[
						      mark size=0pt,
						      only marks,
					      ]
					      table {
							      x  y
							      0  0
						      };
				      \end{axis}
			                                                                                                                                                                       \end{tikzpicture}                                                  \\
			      \hline
			      \begin{tabular}{c}Пирами-\\дальная\end{tabular} & \begin{tikzpicture}
				                                                        \begin{axis}[
						      axis lines = left,
						      xlabel = \(x\),
						      ylabel = {\(y\)},
						      width=0.25\textwidth,
						      height=0.25\textwidth
					      ]
					      \addplot[
						      mark size=2pt,
						      only marks,
					      ]
					      table {
							      x  y
							      5  50
							      10 147
							      15 264
							      20 403
							      25 551
							      30 705
							      35 871
							      40 1047
							      45 1229
						      };
					      \addplot[
						      mark size=0pt,
						      only marks,
					      ]
					      table {
							      x  y
							      0  0
						      };
				      \end{axis}
			                                                        \end{tikzpicture}                                      & \begin{tikzpicture}
				                                                                                                                 \begin{axis}[
						      axis lines = left,
						      xlabel = \(x\),
						      ylabel = {\(y\)},
						      width=0.25\textwidth,
						      height=0.25\textwidth
					      ]
					      \addplot[
						      mark size=2pt,
						      only marks,
					      ]
					      table {
							      x  y
							      5  42
							      10 119
							      15 217
							      20 339
							      25 437
							      30 571
							      35 700
							      40 816
							      45 984
						      };
					      \addplot[
						      mark size=0pt,
						      only marks,
					      ]
					      table {
							      x  y
							      0  0
						      };
				      \end{axis}
			                                                                                                                 \end{tikzpicture}                                   & \begin{tikzpicture}
				                                                                                                                                                                       \begin{axis}[
						      axis lines = left,
						      xlabel = \(x\),
						      ylabel = {\(y\)},
						      width=0.25\textwidth,
						      height=0.25\textwidth
					      ]
					      \addplot[
						      mark size=2pt,
						      only marks,
					      ]
					      table {
							      x  y
							      5  45
							      10 142
							      15 240
							      20 366
							      25 502
							      30 635
							      35 777
							      40 916
							      45 1069
						      };
					      \addplot[
						      mark size=0pt,
						      only marks,
					      ]
					      table {
							      x  y
							      0  0
						      };
				      \end{axis}
			                                                                                                                                                                       \end{tikzpicture}                                                  \\
			      \hline
		      \end{longtable}
	      \end{center}
	\item Выводы по работе.\\
	      \textbf{Сортировка вставками:}\\
		  В лучшем случае, если массив отсортирован по возрастанию, $t = 1 + N\cdot5$, 
		  в худшем случае, если массив отсортирован по неубыванию, 
		  $t = 1 + N\cdot(1 + 3 \cdot (1 + 2 + 3 + ... + N - 1) + 2 + (1 + 2 + 3 + ... + N)) = 1 + N \cdot (3 + 4 \cdot \frac{N - 1}{2}) = 
		  1 + 5 \cdot N + 4 \cdot N^2$.\bigbreak
		  Для отсортированного по возрастанию массива: $\Omega(N)$\\
		  Для отсортированного по убыванию массива: $O(N^2)$\\
		  Для неотсортированного массива: $\Theta(N^2)$\\
	      \textbf{Сортировка выбором:}\\
	      $t = 1 + N\cdot(3 + 3 \cdot (N - 1 + N - 2 + ... + 3 + 2 + 1)) = 1 + N\cdot(3 + \frac{3}{2}(N - 1)) = 1 + 3\cdot N + \frac{3}{2}(N ^ 2 - N) = 1 + \frac{3}{2} \cdot N + \frac{3}{2} \cdot N^2$.\bigbreak
		  Для отсортированного по возрастанию массива: $\Omega(N^2)$\\
		  Для отсортированного по убыванию массива: $O(N^2)$\\
		  Для неотсортированного массива: $\Theta(N^2)$\\
	      \textbf{Обменом:}\\
	      $t = 1 + N\cdot(1 + 3 \cdot (N - 1 + N - 2 + ... + 3 + 2 + 1)) = 1 + N\cdot(1 + \frac{3}{2}(N - 1)) = 1 + N + \frac{3}{2}(N ^ 2 - N) = 1 - \frac{1}{2} \cdot N + \frac{3}{2} \cdot N^2$.\bigbreak
		  Для отсортированного по возрастанию массива: $\Omega(N^2)$\\
		  Для отсортированного по убыванию массива: $O(N^2)$\\
		  Для неотсортированного массива: $\Theta(N^2)$\\
	      \textbf{Обменом 1:}\\
	      В лучшем случае, если массив отсортирован по возрастанию, $t = 5 + 3 \cdot (N - 1 + N - 2 + ... + 3 + 2 + 1) = 3.5 + 1.5\cdot N$, 
		  в худшем случае, если массив отсортирован по неубыванию, 
		  $t = 1 + N\cdot (5 + 5 \cdot (N - 1 + N - 2 + ... + 3 + 2 + 1)) = 1 + N\cdot(5 + \frac{5N - 5}{2}) = 1 + N \cdot (2.5 + 2.5N) = 1 + 2.5N + 2.5N^2$.\bigbreak
		  Для отсортированного по возрастанию массива: $\Omega(N)$\\
		  Для отсортированного по убыванию массива: $O(N^2)$\\
		  Для неотсортированного массива: $\Theta(N^2)$\\
	      \textbf{Обменом 2:}\\
	      В лучшем случае, если массив отсортирован по возрастанию, $t = 5 + 3\cdot(N - 1 + N - 2 + ... + 3 + 2 + 1) = 3.5 + 1.5\cdot N$, 
		  в худшем случае, если массив отсортирован по неубыванию, 
		  $t = 1 + N\cdot (5 + 5 \cdot (N - 1 + N - 2 + ... + 3 + 2 + 1)) = 1 + N\cdot(5 + \frac{5N - 5}{2}) = 1 + N \cdot (2.5 + 2.5N) = 1 + 2.5N + 2.5N^2$.\bigbreak
		  Для отсортированного по возрастанию массива: $\Omega(N)$\\
		  Для отсортированного по убыванию массива: $O(N^2)$\\
		  Для неотсортированного массива: $\Theta(N^2)$\\
	      \textbf{Сортировка Шелла:}\\
		  $t = 3 + ((\log_{3}N) - 1) + ((\log_{3}N) - 1) \cdot ((N / (3 * h_i + 1)) \cdot (1 + 2 \cdot ((1 + (1 + 3 * h_i + 1) + (1 + 2\cdot(3 * h_i + 1)) + (1 + 3\cdot(3 * h_i + 1)) + ... + (1 + N / (3 * h_i + 1))))) + 1)$\bigbreak
		  Анализ сортировки Шелла математически сложен, в случае правильного выбора шагов порядок ФВС будет выглядеть как $O(N^{1.2})$ или $O(N(\log_2N))$.
		  
	      \textbf{Сортировка Хоара:}\\
		  $t = 4 + 4\cdot N + 2\cdot t(N / 2)$\bigbreak
		  ФВС алгоритма не зависит от упорядоченности массива, она зависит от алгоритма нахождения разделителя. 
		  Поэтому для отсортированных по убыванию, возрастанию и неотсортированных массивов \\
		  $O(N^2)$ - если разделитель выбран неудачно\\
		  $\Omega(N \cdot \log N)$ - если разделитель выбран удачно\\
		  $\Theta(N \cdot \log N)$.
		  
	      \textbf{Пирамидальная сортировка:}\\
		  $t = 2 + N\cdot (4 + 4 \cdot (\log_2 1 + \log_2 2 + \log_2 3 + ... + \log_2 N)) + N \cdot (5 + 10 \cdot (\log_2 1 + \log_2 2 + \log_2 3 + ... + \log_2 N))$\bigbreak
		  $O(N \cdot \log N)$\\
		  $\Omega(N \cdot \log N)$\\
		  $\Theta(N \cdot \log N)$.
		\end{enumerate}
\href{https://github.com/IAmProgrammist/algorithms_and_data_structures/tree/main}{Ссылка на репозиторий}\\
\textbf{Вывод: } в ходе лабораторной работы изучили методы сортировки массивов и приобрели
навыки в проведении сравнительного анализа различных методов сортировки.

\end{document}
