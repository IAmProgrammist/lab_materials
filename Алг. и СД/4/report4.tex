\documentclass[a4paper,14pt]{extarticle}


\usepackage[english,russian]{babel}
\usepackage[T2A]{fontenc}
\usepackage[utf8]{inputenc}
\usepackage{ragged2e}
\usepackage[utf8]{inputenc}
\usepackage{hyperref}
\usepackage{minted}
\setmintedinline{frame=lines, framesep=2mm, baselinestretch=1.5, bgcolor=LightGray, breaklines,fontsize=\footnotesize}
\setminted{frame=lines, framesep=2mm, baselinestretch=1.5, bgcolor=LightGray, breaklines,fontsize=\footnotesize}
\usepackage{xcolor}
\definecolor{LightGray}{gray}{0.9}
\usepackage{graphicx}
\usepackage[export]{adjustbox}
\usepackage[left=1cm,right=1cm, top=1cm,bottom=1cm,bindingoffset=0cm]{geometry}
\usepackage{fontspec}
\usepackage{ upgreek }
\usepackage[shortlabels]{enumitem}
\usepackage{adjustbox}
\usepackage{multirow}
\usepackage{amsmath}
\usepackage{amssymb}
\usepackage{pifont}
\usepackage{pgfplots}
\usepackage{longtable}
\usepackage{array}
\graphicspath{ {./images/} }
\makeatletter
\AddEnumerateCounter{\asbuk}{\russian@alph}{щ}
\makeatother
\setmonofont{Consolas}
\setmainfont{Times New Roman}

\newcommand\textbox[1]{
	\parbox{.45\textwidth}{#1}
}

\newcommand{\specialcell}[2][c]{%
	\begin{tabular}[#1]{@{}c@{}}#2\end{tabular}}

\begin{document}
\pagenumbering{gobble}
\begin{center}
	\small{
		МИНИCТЕРCТВО НАУКИ И ВЫCШЕГО ОБРАЗОВАНИЯ \\РОCCИЙCКОЙ ФЕДЕРАЦИИ
		\bigbreak
		ФЕДЕРАЛЬНОЕ ГОCУДАРCТВЕННОЕ БЮДЖЕТНОЕ ОБРАЗОВАТЕЛЬНОЕ УЧРЕЖДЕНИЕ ВЫCШЕГО ОБРАЗОВАНИЯ \\
		\bigbreak
		\textbf{«БЕЛГОРОДCКИЙ ГОCУДАРCТВЕННЫЙ \\ТЕХНОЛОГИЧЕCКИЙ УНИВЕРCИТЕТ им. В. Г. ШУХОВА»\\ (БГТУ им. В.Г. Шухова)} \\
		\bigbreak
		Кафедра программного обеспечения вычислительной техники и автоматизированных систем\\}
\end{center}

\vfill
\begin{center}
	\large{
		\textbf{
			Лабораторная работа №4}}\\
	\normalsize{
		по дисциплине: Алгоритмы и структуры данных \\
		тема: Сравнительный анализ алгоритмов поиска (Pascal/С)}
\end{center}
\vfill
\hfill\textbox{
	Выполнил: ст. группы ПВ-223\\Пахомов Владислав Андреевич
	\bigbreak
	Проверили: асс. Солонченко Роман\\Евгеньевич
}
\vfill\begin{center}
	Белгород 2023 г.
\end{center}
\newpage
\begin{center}
	\textbf{Лабораторная работа №4}\\
	Сравнительный анализ алгоритмов поиска (Pascal/С)
\end{center}
\textbf{Цель работы: }изучение алгоритмов поиска элемента в массиве и закрепление
навыков в проведении сравнительного анализа алгоритмов.
\begin{enumerate}
	\item Листинг программы.
	      \textit{main.c}
	      \begin{minted}{C}
#include <time.h>

#include <algc.h>

#define LOW 50
#define HIGH 450
#define STEP 50

int main() {
    srand(time(0));

    comparesSearchExperiment       (linearSearch,             "linear search",                          LOW, HIGH, STEP);
    comparesSearchExperiment       (linearQuickSearch,        "quick linear search",                    LOW, HIGH, STEP);
    comparesOrderedSearchExperiment(orderedLinearQuickSearch, "quick linear search for ordered arrays", LOW, HIGH, STEP);
    comparesOrderedSearchExperiment(orderedBinarySearch,      "binary search",                          LOW, HIGH, STEP);
    comparesOrderedSearchExperiment(orderedBlockSearch,       "block search",                           LOW, HIGH, STEP);

    return 0;
}
	\end{minted}
	      \textit{search.h}
	      \begin{minted}{C}
#ifndef SEARCH
#define SEARCH

#include <lab3/sorts.h>

#include <stdbool.h>

#define SEARCH_UTILITY_EXPERIMENT_ITERATIONS_AMOUNT 10000

typedef int (*SearchingFunction)(int*, int, int, int*);
typedef int (*OrderedSearchingFunction)(int*, int, int, int*);

int linearSearch(int* a, int size,  int searchElement, int* comps);
int linearQuickSearch(int* a, int size,  int searchElement, int* comps);
int orderedLinearQuickSearch(int* a, int size,  int searchElement, int* comps);
int orderedBinarySearch(int* a, int size,  int searchElement, int* comps);
int orderedBlockSearch(int* a, int size,  int searchElement, int* comps);

void comparesSearchExperiment(SearchingFunction function, char *searchingFunctionName, int low, int high, int step);
void comparesOrderedSearchExperiment(SearchingFunction function, char *searchingFunctionName, int low, int high, int step);

#endif
	\end{minted}
	      \textit{utility.c}
	      \begin{minted}{C}
#include <lab4/search.h>

#include <assert.h>
#include <limits.h>
#include <stdlib.h>
#include <stdbool.h>
#include <stdio.h>

void comparesSearchExperiment(SearchingFunction function, char *searchingFunctionName, int low, int high, int step) {
    printf("==============================================\n\n");
    printf("Launching comparing experiment for function %s\n\n", searchingFunctionName);

    for (int i = low; i <= high; i += step) {
        int sumCompares = 0;
        int maxCompares = INT_MIN;
        for (int j = 0; j < SEARCH_UTILITY_EXPERIMENT_ITERATIONS_AMOUNT; j++) {
            int* array = malloc((i + 1) * sizeof(int));
            genRandom(array, i);
            int compares = 0;
            int randomIndex = rand() % i;
            int foundIndex = function(array, i, array[randomIndex], &compares);
            sumCompares += compares;
            maxCompares = compares > maxCompares ? compares : maxCompares;
            assert(array[foundIndex] == array[randomIndex]);
            free(array);
        }

        int avgCompares = sumCompares / SEARCH_UTILITY_EXPERIMENT_ITERATIONS_AMOUNT;
        printf("For %3d elements. Average: %7d compares; maximum: %7d compares\n", i, avgCompares, maxCompares);
    }

    printf("\n");
}

void comparesOrderedSearchExperiment(SearchingFunction function, char *searchingFunctionName, int low, int high, int step) {
    printf("==============================================\n\n");
    printf("Launching comparing experiment for function %s\n\n", searchingFunctionName);

    for (int i = low; i <= high; i += step) {
        int sumCompares = 0;
        int maxCompares = INT_MIN;
        for (int j = 0; j < SEARCH_UTILITY_EXPERIMENT_ITERATIONS_AMOUNT; j++) {
            int* array = malloc((i + 1) * sizeof(int));
            genOrdered(array, i);
            int compares = 0;
            int randomIndex = rand() % i;
            int foundIndex = function(array, i, array[randomIndex], &compares);
            sumCompares += compares;
            maxCompares = compares > maxCompares ? compares : maxCompares;
            assert(array[foundIndex] == array[randomIndex]);
            free(array);
        }

        int avgCompares = sumCompares / SEARCH_UTILITY_EXPERIMENT_ITERATIONS_AMOUNT;
        printf("For %3d elements. Average: %7d compares; maximum: %7d compares\n", i, avgCompares, maxCompares);
    }

    printf("\n");
}
	\end{minted}
	      \textit{linearsearch.c}
	      \begin{minted}{C}
#include <lab4/search.h>

int linearSearch(int* a, int size,  int searchElement, int* comps) {
    int i = 0;
    while (INC_COMPARES(comps) && i < size && INC_COMPARES(comps) && a[i] != searchElement)
        i++;

    return i;
}
	\end{minted}
	      \textit{linearquicksearch.c}
	      \begin{minted}{C}
#include <lab4/search.h>

int linearQuickSearch(int* a, int size,  int searchElement, int* comps) {
    int i = 0;
    a[size] = searchElement;
    while (INC_COMPARES(comps) && a[i] != searchElement)
        i++;

    return i;
}
	\end{minted}
	      \textit{orderedlinearquicksearch.c}
	      \begin{minted}{C}
#include <lab4/search.h>

int orderedLinearQuickSearch(int* a, int size,  int searchElement, int* comps) {
    int i = 0;
    while (INC_COMPARES(comps) && i < size && INC_COMPARES(comps) && a[i] != searchElement) {
        if (INC_COMPARES(comps) && a[i] > searchElement)
           i = size - 1;
        
        i++;
    }

    return i;
}
	\end{minted}
	      \textit{binarysearch.c}
	      \begin{minted}{C}
#include <lab4/search.h>

int orderedBinarySearch(int* a, int size, int searchElement, int* comps) {
    int left = 0, right = size - 1;

    while (INC_COMPARES(comps) && left <= right) {
        int middle = left + (right - left) / 2;

        if (INC_COMPARES(comps) && a[middle] < searchElement) 
            left = middle + 1;
        else if (INC_COMPARES(comps) && a[middle] > searchElement)
            right = middle - 1;
        else
            return middle;
    }
    
    return size;
}
	\end{minted}
	      \textit{blocksearch.c}
	      \begin{minted}{C}
#include <lab4/search.h>

#include <math.h>

int orderedBlockSearch(int* a, int size, int searchElement, int* comps) {
    int blockSize = sqrtl(size);
    int blocksAmount = size / blockSize + size % blockSize;
    for (int i = 0; INC_COMPARES(comps) && i < blocksAmount; i++) {
        int blockBeginIndex = i * blockSize;
        int blockEndIndex = (i + 1) * blockSize;
        blockEndIndex = INC_COMPARES(comps) && blockEndIndex > size ? size : blockEndIndex;

        if (INC_COMPARES(comps) && a[blockEndIndex - 1] < searchElement) continue;

        int j = blockBeginIndex;
        while (INC_COMPARES(comps) && j < blockEndIndex && INC_COMPARES(comps) && a[j] != searchElement) {
            if (INC_COMPARES(comps) && a[j] > searchElement)
                j = size - 1;
            
            j++;
        }

        return j;
    }

    return size;
}
	\end{minted}
	\item Результаты работы программы.
	      \begin{center}
		      \textbf{Максимальное количество операций сравнения}
		      \begin{tabular}{|c|c|c|c|c|c|c|c|c|c|}
			      \hline
			      \multirow{2}{*}{Алгоритмы поиска}                                              & \multicolumn{9}{|c|}{Количество элементов в массиве}                                                    \\
			      \cline{2-10}
			                                                                                     & 50                                                   & 100 & 150 & 200 & 250 & 300 & 350  & 400  & 450  \\
			      \hline
			      \begin{tabular}{c}Линейный\\(неупорядочен-\\ный массив)\end{tabular}           & 100                                                  & 200 & 300 & 400 & 500 & 600 & 700  & 800  & 900  \\
			      \hline
			      \begin{tabular}{c}Быстрый линей-\\ный (неупорядо-\\ченный массив)\end{tabular} & 50                                                   & 100 & 150 & 200 & 250 & 300 & 350  & 400  & 450  \\
			      \hline
			      \begin{tabular}{c}Быстрый линей-\\ный (упорядо-\\ченный массив)\end{tabular}   & 149                                                  & 299 & 449 & 599 & 749 & 899 & 1049 & 1199 & 1349 \\
			      \hline
			      \begin{tabular}{c}Бинарный (упо-\\рядоченный мас-\\сив)\end{tabular}           & 17                                                   & 20  & 21  & 23  & 23  & 24  & 25   & 26   & 26   \\
			      \hline
			      \begin{tabular}{c}Блочный (упо-\\рядоченный мас-\\сив)\end{tabular}            & 41                                                   & 59  & 71  & 83  & 92  & 101 & 110  & 119  & 125  \\
			      \hline
		      \end{tabular}\bigbreak
		      \textbf{Среднее количество операций сравнения}
		      \begin{tabular}{|c|c|c|c|c|c|c|c|c|c|}
			      \hline
			      \multirow{2}{*}{Алгоритмы поиска}                                              & \multicolumn{9}{|c|}{Количество элементов в массиве}                                                 \\
			      \cline{2-10}
			                                                                                     & 50                                                   & 100 & 150 & 200 & 250 & 300 & 350 & 400 & 450 \\
			      \hline
			      \begin{tabular}{c}Линейный\\(неупорядочен-\\ный массив)\end{tabular}           & 51                                                   & 100 & 152 & 200 & 248 & 302 & 345 & 396 & 451 \\
			      \hline
			      \begin{tabular}{c}Быстрый линей-\\ный (неупорядо-\\ченный массив)\end{tabular} & 25                                                   & 50  & 75  & 101 & 125 & 148 & 174 & 200 & 222 \\
			      \hline
			      \begin{tabular}{c}Быстрый линей-\\ный (упорядо-\\ченный массив)\end{tabular}   & 75                                                   & 150 & 224 & 298 & 374 & 454 & 529 & 601 & 675 \\
			      \hline
			      \begin{tabular}{c}Бинарный (упо-\\рядоченный мас-\\сив)\end{tabular}           & 12                                                   & 14  & 16  & 17  & 18  & 18  & 19  & 19  & 20  \\
			      \hline
			      \begin{tabular}{c}Блочный (упо-\\рядоченный мас-\\сив)\end{tabular}            & 23                                                   & 32  & 38  & 44  & 49  & 53  & 57  & 61  & 65  \\
			      \hline
		      \end{tabular}
	      \end{center}
	\item Графики зависимостей ФВС.\\
	      \begin{center}
		      \begin{longtable}{|>{\centering\arraybackslash}p{0.13\textwidth}|>{\centering\arraybackslash}p{0.43\textwidth}|>{\centering\arraybackslash}p{0.43\textwidth}|}
			      \hline
			      Поиск                                                                                             & \begin{tabular}{c}Максимальное количество\\операций сравнения\end{tabular} & \begin{tabular}{c}Среднее количество\\операций сравнения\end{tabular} \\
			      \hline
			      \begin{tabular}{c}Линейный\\поиск\end{tabular}                                                    & \begin{tikzpicture}
				                                                                                                          \begin{axis}[
						      axis lines = left,
						      xlabel = \(x\),
						      ylabel = {\(y\)},
						      width=0.41\textwidth,
						      height=0.41\textwidth
					      ]
					      \addplot[
						      mark size=2pt,
						      only marks,
					      ]
					      table {
							      x  y
							      50  100
							      100 200
							      150 300
							      200 400
							      250 500
							      300 600
							      350 700
							      400 800
							      450 900
						      };
					      \addplot[
						      mark size=0pt,
						      only marks,
					      ]
					      table {
							      x  y
							      0  0
						      };
				      \end{axis}
			                                                                                                          \end{tikzpicture}                                                 & \begin{tikzpicture}
				                                                                                                                                                                              \begin{axis}[
						      axis lines = left,
						      xlabel = \(x\),
						      ylabel = {\(y\)},
						      width=0.41\textwidth,
						      height=0.41\textwidth
					      ]
					      \addplot[
						      mark size=2pt,
						      only marks,
					      ]
					      table {
							      x  y
							      50  51
							      100 100
							      150 152
							      200 200
							      250 248
							      300 302
							      350 345
							      400 396
							      450 451
						      };
					      \addplot[
						      mark size=0pt,
						      only marks,
					      ]
					      table {
							      x  y
							      0  0
						      };
				      \end{axis}
			                                                                                                                                                                              \end{tikzpicture}                                                      \\
			      \hline \begin{tabular}{c}Быстрый\\линейный\\поиск\\(неупо-\\рядочен-\\ный мас-\\сив)\end{tabular} & \begin{tikzpicture}
				                                                                                                          \begin{axis}[
						      axis lines = left,
						      xlabel = \(x\),
						      ylabel = {\(y\)},
						      width=0.41\textwidth,
						      height=0.41\textwidth
					      ]
					      \addplot[
						      mark size=2pt,
						      only marks,
					      ]
					      table {
							      x  y
							      50  50
							      100 100
							      150 150
							      200 200
							      250 250
							      300 300
							      350 350
							      400 400
							      450 450
						      };
					      \addplot[
						      mark size=0pt,
						      only marks,
					      ]
					      table {
							      x  y
							      0  0
						      };
				      \end{axis}
			                                                                                                          \end{tikzpicture}                                                 & \begin{tikzpicture}
				                                                                                                                                                                              \begin{axis}[
						      axis lines = left,
						      xlabel = \(x\),
						      ylabel = {\(y\)},
						      width=0.41\textwidth,
						      height=0.41\textwidth
					      ]
					      \addplot[
						      mark size=2pt,
						      only marks,
					      ]
					      table {
							      x  y
							      50  25
							      100 50
							      150 75
							      200 101
							      250 125
							      300 148
							      350 174
							      400 200
							      450 222
						      };
					      \addplot[
						      mark size=0pt,
						      only marks,
					      ]
					      table {
							      x  y
							      0  0
						      };
				      \end{axis}
			                                                                                                                                                                              \end{tikzpicture}                                                      \\
			      \hline \begin{tabular}{c}Быстрый\\линейный\\поиск\\(упо-\\рядочен-\\ный мас-\\сив)\end{tabular}   & \begin{tikzpicture}
				                                                                                                          \begin{axis}[
						      axis lines = left,
						      xlabel = \(x\),
						      ylabel = {\(y\)},
						      width=0.41\textwidth,
						      height=0.41\textwidth
					      ]
					      \addplot[
						      mark size=2pt,
						      only marks,
					      ]
					      table {
							      x  y
							      50  149
							      100 299
							      150 449
							      200 599
							      250 749
							      300 899
							      350 1049
							      400 1199
							      450 1349
						      };
					      \addplot[
						      mark size=0pt,
						      only marks,
					      ]
					      table {
							      x  y
							      0  0
						      };
				      \end{axis}
			                                                                                                          \end{tikzpicture}                                                 & \begin{tikzpicture}
				                                                                                                                                                                              \begin{axis}[
						      axis lines = left,
						      xlabel = \(x\),
						      ylabel = {\(y\)},
						      width=0.41\textwidth,
						      height=0.41\textwidth
					      ]
					      \addplot[
						      mark size=2pt,
						      only marks,
					      ]
					      table {
							      x  y
							      50  75
							      100 150
							      150 224
							      200 298
							      250 374
							      300 454
							      350 529
							      400 601
							      450 675
						      };
					      \addplot[
						      mark size=0pt,
						      only marks,
					      ]
					      table {
							      x  y
							      0  0
						      };
				      \end{axis}
			                                                                                                                                                                              \end{tikzpicture}                                                      \\
			      \hline \begin{tabular}{c}Бинарный\\поиск\\(упо-\\рядочен-\\ный мас-\\сив)\end{tabular}            & \begin{tikzpicture}
				                                                                                                          \begin{axis}[
						      axis lines = left,
						      xlabel = \(x\),
						      ylabel = {\(y\)},
						      width=0.41\textwidth,
						      height=0.41\textwidth
					      ]
					      \addplot[
						      mark size=2pt,
						      only marks,
					      ]
					      table {
							      x  y
							      50  17
							      100 20
							      150 21
							      200 23
							      250 23
							      300 24
							      350 25
							      400 26
							      450 26
						      };
					      \addplot[
						      mark size=0pt,
						      only marks,
					      ]
					      table {
							      x  y
							      0  0
						      };
				      \end{axis}
			                                                                                                          \end{tikzpicture}                                                 & \begin{tikzpicture}
				                                                                                                                                                                              \begin{axis}[
						      axis lines = left,
						      xlabel = \(x\),
						      ylabel = {\(y\)},
						      width=0.41\textwidth,
						      height=0.41\textwidth
					      ]
					      \addplot[
						      mark size=2pt,
						      only marks,
					      ]
					      table {
							      x  y
							      50  12
							      100 14
							      150 16
							      200 17
							      250 18
							      300 18
							      350 19
							      400 19
							      450 20
						      };
					      \addplot[
						      mark size=0pt,
						      only marks,
					      ]
					      table {
							      x  y
							      0  0
						      };
				      \end{axis}
			                                                                                                                                                                              \end{tikzpicture}                                                      \\
			      \hline \begin{tabular}{c}Блочный\\поиск\\(упо-\\рядочен-\\ный мас-\\сив)\end{tabular}             & \begin{tikzpicture}
				                                                                                                          \begin{axis}[
						      axis lines = left,
						      xlabel = \(x\),
						      ylabel = {\(y\)},
						      width=0.41\textwidth,
						      height=0.41\textwidth
					      ]
					      \addplot[
						      mark size=2pt,
						      only marks,
					      ]
					      table {
							      x  y
							      50  41
							      100 59
							      150 71
							      200 83
							      250 92
							      300 101
							      350 110
							      400 119
							      450 125
						      };
					      \addplot[
						      mark size=0pt,
						      only marks,
					      ]
					      table {
							      x  y
							      0  0
						      };
				      \end{axis}
			                                                                                                          \end{tikzpicture}                                                 & \begin{tikzpicture}
				                                                                                                                                                                              \begin{axis}[
						      axis lines = left,
						      xlabel = \(x\),
						      ylabel = {\(y\)},
						      width=0.41\textwidth,
						      height=0.41\textwidth
					      ]
					      \addplot[
						      mark size=2pt,
						      only marks,
					      ]
					      table {
							      x  y
							      50  23
							      100 32
							      150 38
							      200 44
							      250 49
							      300 53
							      350 57
							      400 61
							      450 65
						      };
					      \addplot[
						      mark size=0pt,
						      only marks,
					      ]
					      table {
							      x  y
							      0  0
						      };
				      \end{axis}
			                                                                                                                                                                              \end{tikzpicture}                                                      \\
		      \end{longtable}
	      \end{center}
	\item Выводы по работе.\\
	      \textbf{Линейный поиск (неупорядоченный массив):}\\
	      $t = 2 + 3\cdot N$\bigbreak
	      Если искомый элемент в начале массива: $\Omega(1)$\\
	      Если искомый элемент в конце массива или искомого элемента нет в массиве: $O(N)$\\
	      В общем случае: $\Theta(N)$\\
	      \textbf{Быстрый линейный поиск (неупорядоченный массив):}\\
	      $t = 3 + 2\cdot N$\bigbreak
	      Если искомый элемент в начале массива: $\Omega(1)$\\
	      Если искомый элемент в конце массива или искомого элемента нет в массиве: $O(N)$\\
	      В общем случае: $\Theta(N)$\\
	      \textbf{Быстрый линейный поиск (упорядоченный массив):}\\
	      $t = 2 + 5 \cdot N$
		  Если искомый элемент в начале массива: $\Omega(1)$\\
	      Все элементы массива меньше искомого: $O(N)$\\
	      В общем случае: $\Theta(N)$\\
	      \textbf{Бинарный поиск:}\\
	      $t = 3 + 5 \cdot \log_2 N$
		  Если искомый элемент в середине массива: $\Omega(1)$\\
	      Если элемента нет в массиве: $O(\log_2 N)$\\
	      В общем случае: $\Theta(\log_2 N)$\\
	      \textbf{Блочный поиск:}\\
	      $t = 2 + 7 \cdot \sqrt{N} $
		  Если искомый элемент в начале массива: $\Omega(1)$\\
	      Если искомый элемент в конце массива: $O(\sqrt{N})$\\
	      В общем случае: $O(\sqrt{N})$
\end{enumerate}
\href{https://github.com/IAmProgrammist/algorithms_and_data_structures/tree/main}{Ссылка на репозиторий}\\
\textbf{Вывод: } в ходе лабораторной работы изучили алгоритмы поиска элемента в массиве и закрепили
навыки в проведении сравнительного анализа алгоритмов.

\end{document}
