\documentclass[a4paper,14pt]{extarticle}


\usepackage[english,russian]{babel}
\usepackage[T2A]{fontenc}
\usepackage[utf8]{inputenc}
\usepackage{ragged2e}
\usepackage[utf8]{inputenc}
\usepackage{hyperref}
\usepackage{minted}
\setmintedinline{frame=lines, framesep=2mm, baselinestretch=1.5, bgcolor=LightGray, breaklines,fontsize=\scriptsize}
\setminted{frame=lines, framesep=2mm, baselinestretch=1.5, bgcolor=LightGray, breaklines,fontsize=\scriptsize}
\usepackage{xcolor}
\definecolor{LightGray}{gray}{0.9}
\usepackage{graphicx}
\usepackage[export]{adjustbox}
\usepackage[left=1cm,right=1cm, top=1cm,bottom=1cm,bindingoffset=0cm]{geometry}
\usepackage{fontspec}
\usepackage{ upgreek }
\usepackage[shortlabels]{enumitem}
\usepackage{adjustbox}
\usepackage{multirow}
\usepackage{amsmath}
\usepackage{amssymb}
\usepackage{pifont}
\usepackage{pgfplots}
\usepackage{longtable}
\usepackage{array}
\graphicspath{ {./images/} }
\makeatletter
\AddEnumerateCounter{\asbuk}{\russian@alph}{щ}
\makeatother
\setmonofont{Consolas}
\setmainfont{Times New Roman}

\newcommand\textbox[1]{
	\parbox{.45\textwidth}{#1}
} 

\newcommand{\specialcell}[2][c]{%
	\begin{tabular}[#1]{@{}c@{}}#2\end{tabular}}

\begin{document}
\pagenumbering{gobble}
\begin{center}
    \small{
        \textbf{МИНИCТЕРCТВО НАУКИ И ВЫCШЕГО ОБРАЗОВАНИЯ РОCCИЙCКОЙ ФЕДЕРАЦИИ}\\
        ФЕДЕРАЛЬНОЕ ГОCУДАРCТВЕННОЕ БЮДЖЕТНОЕ ОБРАЗОВАТЕЛЬНОЕ УЧРЕЖДЕНИЕ\\ВЫCШЕГО ОБРАЗОВАНИЯ \\
        \textbf{«БЕЛГОРОДCКИЙ ГОCУДАРCТВЕННЫЙ ТЕХНОЛОГИЧЕCКИЙ\\УНИВЕРCИТЕТ им. В. Г. ШУХОВА»\\ (БГТУ им. В.Г. Шухова)} \\
        \bigbreak
        \includegraphics[width=70mm]{log}\\
        ИНСТИТУТ ИНФОРМАЦИОННЫХ ТЕХНОЛОГИЙ И УПРАВЛЯЮЩИХ СИСТЕМ\\}
\end{center}

\vfill
\begin{center}
	\large{
		\textbf{
			Лабораторная работа №7}}\\
	\normalsize{
		по дисциплине: Алгоритмы и структуры данных \\
		тема: «Структуры данных типа «дерево» (Pascal/С)»}
\end{center}
\vfill
\hfill\textbox{
	Выполнил: ст. группы ПВ-223\\Пахомов Владислав Андреевич
	\bigbreak
	Проверили: асс. Солонченко Роман\\Евгеньевич
}
\vfill\begin{center}
	Белгород \the\year г.
\end{center}
\newpage
\begin{center}
	\textbf{Лабораторная работа №7}\\
	«Структуры данных типа «дерево» (Pascal/С)»\\
	Вариант 10
\end{center}
\textbf{Цель работы: }изучить СД типа «дерево», научиться их программно 
реализовывать и использовать.

\begin{enumerate}
	\item Для СД типа дерево список» определить:
	      \begin{enumerate}[label*=\arabic*.]
		      \item Абстрактный уровень представления СД:

		            \begin{enumerate}[label*=\arabic*.]
			            \item Характер организованности и изменчивости.\\
			            Характер организованности - \textbf{дерево}, один ко многим. Характер изменчивости - \textbf{динамический}.
			            \item Набор допустимых операций.\\
			            Инициализация, создание корня, запись данных в дерево, 
                        чтение данных из дерева, проверка на наличие левого/правого сына,
                        получение правого/левого сына, проверка дерева на пустоту, 
                        удаление листа
		            \end{enumerate}

		      \item Физический уровень представления СД:

		            \begin{enumerate}[label*=\arabic*.]
			            \item Схему хранения.\\
			            Схема хранения - \textbf{последовательность.}
			            \item Объем памяти, занимаемый экземпляром СД.\\
			            Элемент состоит из трёх полей: указатель произвольного типа, 
                        индекс для элемента слева и индекс на элемент справа. Все они имеют размер 4 байт. 
                        $V = 12 \cdot N$, где N - количество элемента деревья.
			            \item Формат внутреннего представления СД и способ его интерпретации.\\
			            Элементы содержатся в статическом массиве, свободные элементы объединяются
                        в список (ССЭ), на начало которого указывает левый сын первого элемента массива.
			            \item Характеристику допустимых значений.\\
			            $Car(C) = \sum_{1}^{max} \frac{(2 \cdot i)!}{(i + 1)(i!)^2} \cdot Car(BaseType) + 1$.
			            \item Тип доступа к элементам.\\
						Тип доступа к элементам - \textbf{прямой}.
		            \end{enumerate}

		      \item Логический уровень представления СД.
		            \begin{enumerate}[label*=\arabic*.]
			            \item Способ описания СД и экземпляра СД на языке программирования.\\
						\begin{minted}{C}
InitTree(10);
Tree root = CreateRoot();
												  \end{minted}
		            \end{enumerate}
	      \end{enumerate}
		  \item Реализовать СД типа «дерево» в соответствии с 
          вариантом индивидуального задания (см. табл. 17) в виде модуля.\\
main.c (тесты)
		  \begin{minted}{C}
#include <assert.h>

#include <algc.h>

#define TAKEN_ELEMENTS ((bool *)MemTree[0].data)

void testInitTree() {
    Tree root = InitTree(0);
    assert(TreeError == TreeUnder);

    root = InitTree(10);
    assert(TreeError == TreeOk && Size == 10);

    root = InitTree(TreeBufferSize + 1);
    assert(TreeError == TreeNotMem);
}

void testCreateRoot() {
    Tree init = InitTree(1);
    Tree root = CreateRoot();
    assert(TreeError == TreeNotMem);

    init = InitTree(2);
    root = CreateRoot();
    assert(TreeError == TreeOk);
    assert(IsEmptyTree(root));
    root = CreateRoot();
    assert(TreeError == TreeNotMem);
}

void testMemEmpty() {
    Tree init = InitTree(1);
    assert(EmptyMem());

    init = InitTree(2);
    assert(!EmptyMem());

    init = InitTree(2);
    Tree root = CreateRoot();
    assert(EmptyMem());
}

void testNewMem() {
    Tree init = InitTree(1);
    Tree newEl = NewMem();
    assert(TreeError == TreeNotMem);

    init = InitTree(2);
    newEl = NewMem();
    assert(TreeError == TreeOk && TAKEN_ELEMENTS[newEl]);
    NewMem();
    assert(TreeError == TreeNotMem);
}

void testDisposeMem() {
    Tree init = InitTree(2);
    Tree newEl = NewMem();
    assert(TreeError == TreeOk && TAKEN_ELEMENTS[newEl]);
    DisposeMem(newEl);
    assert(TreeError == TreeOk && !TAKEN_ELEMENTS[newEl]);
}

void testWriteReadDataTree() {
    Tree init = InitTree(2);
    Tree root = CreateRoot();

    int someVal = 15;
    WriteDataTree(root, &someVal);
    assert(TreeError == TreeOk);
    assert(*(int*)ReadDataTree(root) == someVal && TreeError == TreeOk);
}

void testIsLSonMoveToLSon() {
    Tree init = InitTree(6);
    Tree root = CreateRoot();

    Tree ex = MoveToLSon(root);
    assert(TreeError == TreeUnder);

    Tree newEl = NewMem();
    MemTree[root].LSon = newEl;
    assert(IsLSon(root) && TreeError == TreeOk && MoveToLSon(root) == newEl && TreeError == TreeOk);
}

void testIsRSonMoveToRSon() {
    Tree init = InitTree(6);
    Tree root = CreateRoot();

    Tree ex = MoveToRSon(root);
    assert(TreeError == TreeUnder);

    Tree newEl = NewMem();
    MemTree[root].RSon = newEl;
    assert(IsRSon(root) && TreeError == TreeOk && MoveToRSon(root) == newEl && TreeError == TreeOk);
}

void testDelTree() {
    Tree init = InitTree(6);
    Tree root = CreateRoot();

    Tree newEl1 = NewMem();
    MemTree[root].LSon = newEl1;
    Tree newEl2 = NewMem();
    MemTree[root].RSon = newEl2;

    DelTree(root);
    assert(TreeError == TreeOk && 
    !TAKEN_ELEMENTS[root] && !TAKEN_ELEMENTS[newEl1] && !TAKEN_ELEMENTS[newEl2]);
}

void test() {
    testInitTree();
    testCreateRoot();
    testMemEmpty();
    testNewMem();
    testDisposeMem();
    testWriteReadDataTree();
    testIsLSonMoveToLSon();
    testIsRSonMoveToRSon();
    testDelTree();
}

int main() {
    test();

    return 0;
}
\end{minted}
algc.h (заголовки)
\begin{minted}{C}
#ifndef TREE
#define TREE

#include <stdint.h>

#define TreeBufferSize 1000

#define TreeOk 0
#define TreeNotMem 1
#define TreeUnder 2

typedef void* TreeBaseType;
typedef size_t PtrEl;

typedef struct {
    TreeBaseType data;
    PtrEl LSon;
    PtrEl RSon;
} Element;

typedef PtrEl Tree;

extern Element MemTree[TreeBufferSize];
extern int TreeError;
extern size_t Size;
// инициализация дерева
Tree InitTree(unsigned size);

// создание корня
Tree CreateRoot(); 

//запись данных
void WriteDataTree(Tree T, TreeBaseType E); 

//чтение
TreeBaseType ReadDataTree(Tree T);

//1 — есть левый сын, 0 — нет
int IsLSon(Tree T);

//1 — есть правый сын, 0 — нет
int IsRSon(Tree T);

// перейти к левому сыну, где T — адрес ячейки, содержащей адрес текущей вершины, TS — адрес ячейки, содержащей адрес корня левого поддерева(левого сына)
Tree MoveToLSon(Tree T);

//перейти к правому сыну
Tree MoveToRSon(Tree T);

//1 — пустое дерево,0 — не пустое
int IsEmptyTree(Tree T);

//удаление листа
void DelTree(Tree T);

/*связывает все элементы массива в список свободных
элементов*/
void InitMem(); 

/*возвращает 1, если в массиве нет свободных элемен-
тов,0 — в противном случае*/
int EmptyMem(); 

/*возвращает номер свободного элемента и ис-
ключает его из ССЭ*/
size_t NewMem(); 

/*делает n-й элемент массива свободным и
включает его в ССЭ*/
void DisposeMem(size_t n);

int BuildTree(Tree T, char* input);
void CopyTree(Tree dst, Tree src);
bool CompTree(Tree T1, Tree T2);

#endif 
\end{minted}
tree.c (реализации функций)
\begin{minted}{C}
#include <algc.h>
#include <stdbool.h>
#include <stdlib.h>

Element MemTree[TreeBufferSize];
int TreeError = TreeOk;
size_t Size = 0;

#define TAKEN_ELEMENTS ((bool *)MemTree[0].data)

Tree InitTree(unsigned size) {
    if (size < 1) {
        TreeError = TreeUnder;
        return 0;
    }

    if (size > TreeBufferSize) {
        TreeError = TreeNotMem;
        return 0;
    }
    Size = size;

    TreeError = TreeOk;
    InitMem();

    return 0;
}

void InitMem() {
    if (Size < 1) {
        TreeError = TreeUnder;
        return;
    }

    if (Size > TreeBufferSize) {
        TreeError = TreeNotMem;
        return;
    }
    
    TreeError = TreeOk;
    bool *takenElements = calloc(Size, sizeof(bool));

    MemTree[0].LSon = 0;
    MemTree[0].RSon = 0;
    MemTree[0].data = takenElements;
    TAKEN_ELEMENTS[0] = true;
}

Tree CreateRoot() {
    size_t newInd = NewMem();
    if (TreeError != TreeOk) return 0;

    TAKEN_ELEMENTS[newInd] = true;
    MemTree[newInd].data = NULL;
    MemTree[newInd].RSon = 0;
    MemTree[newInd].LSon = 0;

    return newInd;
} 


int EmptyMem() {
    TreeError = TreeOk;

    for (size_t i = 0; i < Size; i++)
        if (!TAKEN_ELEMENTS[i]) return false;

    return true;
}

size_t NewMem() {
    TreeError = TreeOk;
    for (size_t i = 0; i < Size; i++) {
        if (!TAKEN_ELEMENTS[i]) {
            TAKEN_ELEMENTS[i] = true;
            MemTree[i].data = NULL;
            MemTree[i].LSon = 0;
            MemTree[i].RSon = 0;

            return i;
        };
    }

    TreeError = TreeNotMem;
    return 0;
}

void DisposeMem(size_t n) {
    TreeError = TreeOk;
    TAKEN_ELEMENTS[n] = false;
}

void WriteDataTree(Tree T, TreeBaseType E) {
    if (!TAKEN_ELEMENTS[T]) {
        TreeError = TreeUnder;
        return;
    }

    TreeError = TreeOk;
    MemTree[T].data = E;
}

TreeBaseType ReadDataTree(Tree T) {
    if (!TAKEN_ELEMENTS[T]) {
        TreeError = TreeUnder;
        return NULL;
    }

    TreeError = TreeOk;
    return MemTree[T].data;
}

int IsLSon(Tree T) {
    if (!TAKEN_ELEMENTS[T]) {
        TreeError = TreeUnder;
        return false;
    }

    TreeError = TreeOk;
    return MemTree[T].LSon != 0 && TAKEN_ELEMENTS[MemTree[T].LSon];
}

int IsRSon(Tree T) {
    if (!TAKEN_ELEMENTS[T]) {
        TreeError = TreeUnder;
        return false;
    }

    TreeError = TreeOk;
    return MemTree[T].RSon != 0 && TAKEN_ELEMENTS[MemTree[T].RSon];
}

Tree MoveToLSon(Tree T) {
    if (IsLSon(T)) return MemTree[T].LSon;

    TreeError = TreeUnder;
    return 0;
}

Tree MoveToRSon(Tree T) {
    if (IsRSon(T)) return MemTree[T].RSon;

    TreeError = TreeUnder;
    return 0;
}

int IsEmptyTree(Tree T) {
    TreeError = TreeOk;

    return MemTree[T].RSon == 0;
}

void _DelSubTree(Tree T) {
    if (!TAKEN_ELEMENTS[T]) {
        TreeError = TreeUnder;
        return;
    }

    if (IsRSon(T))
        _DelSubTree(MemTree[T].RSon);
    MemTree[T].RSon = 0;
        
    if (IsLSon(T))
        _DelSubTree(MemTree[T].LSon);
    MemTree[T].LSon = 0;
    MemTree[T].data = NULL;
    
    DisposeMem(T);
}

void DelTree(Tree T) {
    TreeError = TreeOk;
    // Please do not the important
    if (T == 0)
        return;

    if (!TAKEN_ELEMENTS[T]) {
        TreeError = TreeUnder;
        return;
    }

    if (IsRSon(T)) {
        _DelSubTree(MemTree[T].RSon);
        MemTree[T].RSon = 0;
    }

    if (IsLSon(T)) {
        _DelSubTree(MemTree[T].LSon);
        MemTree[T].LSon = 0;
    }
    DisposeMem(T);
}
\end{minted}
\item Разработать программу для решения задачи в соответствии с вариан-
том индивидуального задания (см. табл.17) с использованием модуля, по-
лученного в результате выполнения пункта 2 задания.\\
main.c (основная программа)
\begin{minted}{C}
#include <algc.h>
#include <ctype.h>
#include <malloc.h>
#include <stdbool.h>
#include <stdio.h>
#include <assert.h>

int main() {
    InitTree(TreeBufferSize);
    Tree fromBrackets = CreateRoot();
    int res = BuildTree(fromBrackets, "(A(B(C)(D))(e(F)(G)(H)))");

    if (res == -1) {
        printf("unable to parse");
        return 1;
    }

    Tree secondRoot = CreateRoot();
    CopyTree(secondRoot, fromBrackets);
    assert(CompTree(fromBrackets, secondRoot));

    return 0;
}
\end{minted}
tree.c (реализации функций)
\begin{minted}{C}
#define NAME_BUFFER_SIZE 100

int BuildTree(Tree T, char* input) {
    char* startInput = input;
    while (isspace(*input))
        input++;
    
    if (*input != '(')
        return -1;
    
    input++;

    char *buffer = calloc(NAME_BUFFER_SIZE, sizeof(char));
    int bufferIndex = 0;
    bool shouldWriteData = true;
    bool anyChild = false;

    while (*input != ')') {
        if (*input == '\0')
            return -1;
        else if (*input == '(') {
            if (shouldWriteData) {
                WriteDataTree(T, buffer);
                shouldWriteData = false;
            }

            size_t newIndex = NewMem();
            if (!anyChild) { 
                anyChild = true;
                MemTree[T].RSon = newIndex;
            } else 
                MemTree[T].LSon = newIndex;

            int res = BuildTree(newIndex, input);
            if (res == -1) return -1;

            input += res + 1;
            T = newIndex;
        } else if (shouldWriteData)
            buffer[bufferIndex] = *(input++);
        else input++;
    }

    if (shouldWriteData) WriteDataTree(T, buffer);

    return input - startInput;
}

void CopyTree(Tree dst, Tree src) {
    WriteDataTree(dst, ReadDataTree(src));
    if (TreeError != TreeOk) return;

    Tree RSon;
    if ((RSon = MoveToRSon(src)) && TreeError == TreeOk) {
        Tree newTree = NewMem();
        if (TreeError != TreeOk)
            return;
        MemTree[dst].RSon = newTree;

        CopyTree(newTree, RSon);
    }

    Tree LSon;
    if ((LSon = MoveToLSon(src)) && TreeError == TreeOk) {
        Tree newTree = NewMem();
        if (TreeError != TreeOk)
            return;
        MemTree[dst].LSon = newTree;

        CopyTree(newTree, LSon);
    }
}

bool CompTree(Tree T1, Tree T2) {
    return ((ReadDataTree(T1) == ReadDataTree(T2)) && TreeError == TreeOk) && 
    (IsRSon(T1) == IsRSon(T2) ? !IsRSon(T1) || CompTree(MemTree[T1].RSon, MemTree[T2].RSon) : false ) && 
    (IsLSon(T1) == IsLSon(T2) ? !IsLSon(T1) || CompTree(MemTree[T1].LSon, MemTree[T2].LSon) : false );
}
\end{minted}
\end{enumerate}
\textbf{Вывод: } в ходе лабораторной работы изучили СД типа «дерево», научились их программно 
реализовывать и использовать.

\end{document}