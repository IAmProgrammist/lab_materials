\documentclass[a4paper,14pt]{extarticle}


\usepackage[english,russian]{babel}
\usepackage[T2A]{fontenc}
\usepackage[utf8]{inputenc}
\usepackage{ragged2e}
\usepackage[utf8]{inputenc}
\usepackage{hyperref}
\usepackage{minted}
\setmintedinline{frame=lines, framesep=2mm, baselinestretch=1.5, bgcolor=LightGray, breaklines,fontsize=\scriptsize}
\setminted{frame=lines, framesep=2mm, baselinestretch=1.5, bgcolor=LightGray, breaklines,fontsize=\scriptsize}
\usepackage{xcolor}
\definecolor{LightGray}{gray}{0.9}
\usepackage{graphicx}
\usepackage[export]{adjustbox}
\usepackage[left=1cm,right=1cm, top=1cm,bottom=1cm,bindingoffset=0cm]{geometry}
\usepackage{fontspec}
\usepackage{ upgreek }
\usepackage[shortlabels]{enumitem}
\usepackage{adjustbox}
\usepackage{multirow}
\usepackage{amsmath}
\usepackage{amssymb}
\usepackage{pifont}
\usepackage{pgfplots}
\usepackage{longtable}
\usepackage{array}
\graphicspath{ {./images/} }
\makeatletter
\AddEnumerateCounter{\asbuk}{\russian@alph}{щ}
\makeatother
\setmonofont{Consolas}
\setmainfont{Times New Roman}

\newcommand\textbox[1]{
	\parbox{.45\textwidth}{#1}
} 

\newcommand{\specialcell}[2][c]{%
	\begin{tabular}[#1]{@{}c@{}}#2\end{tabular}}

\begin{document}
\pagenumbering{gobble}
\begin{center}
    \small{
        \textbf{МИНИCТЕРCТВО НАУКИ И ВЫCШЕГО ОБРАЗОВАНИЯ РОCCИЙCКОЙ ФЕДЕРАЦИИ}\\
        ФЕДЕРАЛЬНОЕ ГОCУДАРCТВЕННОЕ БЮДЖЕТНОЕ ОБРАЗОВАТЕЛЬНОЕ УЧРЕЖДЕНИЕ\\ВЫCШЕГО ОБРАЗОВАНИЯ \\
        \textbf{«БЕЛГОРОДCКИЙ ГОCУДАРCТВЕННЫЙ ТЕХНОЛОГИЧЕCКИЙ\\УНИВЕРCИТЕТ им. В. Г. ШУХОВА»\\ (БГТУ им. В.Г. Шухова)} \\
        \bigbreak
        \includegraphics[width=70mm]{log}\\
        ИНСТИТУТ ИНФОРМАЦИОННЫХ ТЕХНОЛОГИЙ И УПРАВЛЯЮЩИХ СИСТЕМ\\}
\end{center}

\vfill
\begin{center}
	\large{
		\textbf{
			РГЗ}}\\
	\normalsize{
		по дисциплине: Алгоритмы и структуры данных \\
		тема: «Структуры данных типа «таблица» (Pascal/С)»}
\end{center}
\vfill
\hfill\textbox{
	Выполнил: ст. группы ПВ-223\\Пахомов Владислав Андреевич
	\bigbreak
	Проверили: асс. Солонченко Роман\\Евгеньевич
}
\vfill\begin{center}
	Белгород \the\year г.
\end{center}
\newpage
\begin{center}
	\textbf{РГЗ}\\
	«Структуры данных типа «таблица» (Pascal/С)»\\
	Вариант 5
\end{center}
\textbf{Цель работы: }изучить СД типа «таблица», научиться их программно 
реализовывать и использовать.

\begin{enumerate}
	\item Для СД типа дерево список» определить:
	      \begin{enumerate}[label*=\arabic*.]
		      \item Абстрактный уровень представления СД:

		            \begin{enumerate}[label*=\arabic*.]
			            \item Характер организованности и изменчивости.\\
			            Характер организованности - \textbf{множество}. Характер изменчивости - \textbf{динамический}.
			            \item Набор допустимых операций.\\
			            Инициализация, включение элемента, исключение, чтение элемента, изменение 
                        элемента, проверка таблицы на пустоту, уничтожение таблицы.
		            \end{enumerate}

		      \item Физический уровень представления СД:

		            \begin{enumerate}[label*=\arabic*.]
			            \item Схему хранения.\\
			            Схема хранения - \textbf{зависит от реализации}, может быть как последовательной, так и связной.
			            \item Объем памяти, занимаемый экземпляром СД.\\
			            Объём памяти зависит от реализации. В текущей реализации
                        таблица является ОЛС из 5 лаб. работы, объём которой равен 
                        $V = 13 + N\cdot(sizeof(BaseType) + 4)$. $BaseType$ - элемент
                        таблицы, который содержит ключ и значение. По умолчанию они оба int, 
                        то есть занимают по 4 байт. Итого:
                        $V = 13 + N\cdot12$.
			            \item Формат внутреннего представления СД и способ его интерпретации.\\
			            Элемент таблицы представляет из себя СД запись, содержащую ключ и значение. Расположение
                        элементов таблицы зависит от реализации, в данном случае таблица реализована на ОЛС, элементы
                        которой хранятся в динамической памяти.  
			            \item Характеристику допустимых значений.\\
			            $Car(C) = Car(BaseType) ^ 0 + Car(BaseType) ^ 1 + Car(BaseType) ^ 2 + ... + Car(BaseType) ^ {max}$.
			            \item Тип доступа к элементам.\\
						Тип доступа к элементам - \textbf{последовательный}.
		            \end{enumerate}

		      \item Логический уровень представления СД.
		            \begin{enumerate}[label*=\arabic*.]
			            \item Способ описания СД и экземпляра СД на языке программирования.\\
						\begin{minted}{C}
Table t = InitTable();
												  \end{minted}
		            \end{enumerate}
	      \end{enumerate}
		  \item Реализовать СД типа «таблица» в соответствии с 
          вариантом индивидуального задания (см. табл. 18) в виде модуля.\\
main.c (тесты)
		  \begin{minted}{C}
#include "../../libs/alg/lab8/table.c"

#include <assert.h>

void testInitTable() {
    Table t = InitTable();
    assert(TableError == TableOk);
}

void testEmptyTable() {
    Table t = InitTable();
    
    assert(EmptyTable(&t) && TableError == TableOk);

    PutTable(&t, (BaseType){42, 42});
    assert(!EmptyTable(&t) && TableError == TableOk);
}

void testPutTable() {
    Table t = InitTable();
    assert(PutTable(&t, (BaseType){42, 42}) && TableError == TableOk && !EmptyTable(&t) && TableError == TableOk);

    BaseType an;
    assert(GetTable(&t, &an, 42) && TableError == TableOk && an.Key == 42 && an.Value == 42 );
    PutTable(&t, (BaseType){42, 42});
    assert(!PutTable(&t, (BaseType){42, 42}));

    PutTable(&t, (BaseType){39, 39});
    PutTable(&t, (BaseType){41, 41});
    PutTable(&t, (BaseType){40, 40});
    PutTable(&t, (BaseType){38, 38});
    BeginPtr(&t);
    assert(t.ptr->next->data.Key == 38 && t.ptr->next->data.Value == 38);
    MovePtr(&t);
    assert(t.ptr->next->data.Key == 39 && t.ptr->next->data.Value == 39);
    MovePtr(&t);
    assert(t.ptr->next->data.Key == 40 && t.ptr->next->data.Value == 40);
    MovePtr(&t);
    assert(t.ptr->next->data.Key == 41 && t.ptr->next->data.Value == 41);
    MovePtr(&t);
    assert(t.ptr->next->data.Key == 42 && t.ptr->next->data.Value == 42);
}

void testGetTable() {
    Table t = InitTable();
    PutTable(&t, (BaseType){42, 42});
    PutTable(&t, (BaseType){39, 39});
    PutTable(&t, (BaseType){41, 41});
    PutTable(&t, (BaseType){40, 40});
    PutTable(&t, (BaseType){38, 38});

    BaseType an;
    assert(GetTable(&t, &an, 41) && an.Key == 41 && an.Value == 41 && TableError == TableOk);
    assert(!GetTable(&t, &an, 41));
    assert(!GetTable(&t, &an, 90));
}

void testReadTable() {
    Table t = InitTable();
    PutTable(&t, (BaseType){42, 42});
    PutTable(&t, (BaseType){39, 39});
    PutTable(&t, (BaseType){41, 41});
    PutTable(&t, (BaseType){40, 40});
    PutTable(&t, (BaseType){38, 38});

    BaseType an;
    assert(ReadTable(&t, &an, 40) && TableError == TableOk && an.Key == 40 && an.Value == 40);
    assert(ReadTable(&t, &an, 40) && TableError == TableOk && an.Key == 40 && an.Value == 40);
    assert(!ReadTable(&t, &an, 90) && TableError == TableOk);
}

void testWriteTable() {
    Table t = InitTable();
    PutTable(&t, (BaseType){42, 42});
    PutTable(&t, (BaseType){39, 39});
    PutTable(&t, (BaseType){41, 41});
    PutTable(&t, (BaseType){40, 40});
    PutTable(&t, (BaseType){38, 38});

    BaseType an;
    assert(WriteTable(&t, (BaseType){40, 55}) && TableError == TableOk);
    assert(ReadTable(&t, &an, 40) && TableError == TableOk && an.Key == 40 && an.Value == 55);
    assert(!WriteTable(&t, (BaseType){123, 55}) && TableError == TableOk);
}

void test() {
    testInitTable();
    testEmptyTable();
    testPutTable();
    testGetTable();
    testReadTable();
    testWriteTable();
}

int main() {
    test();
}
\end{minted}
algc.h (заголовки)
\begin{minted}{C}
#ifndef TABLE
#define TABLE

#ifndef CUSTOM_TYPES_TABLE
typedef int T_Key;
typedef int T_Value;
#endif

typedef struct {
    T_Key Key;
    T_Value Value;
} TableElement;

#define CUSTOM_BASE_TYPE
typedef TableElement BaseType;

#include <lab5/singlyconnectedlist.h>

#include <stdbool.h>

#define TableOk 0
#define TableNotMem 1
#define TableUnder 2

extern int TableError;

typedef List Table;

Table InitTable();

bool EmptyTable(Table *T);

bool PutTable(Table *T, BaseType E);

bool GetTable(Table *T, BaseType *E, T_Key Key);

bool ReadTable(Table *T, BaseType *E, T_Key Key);

bool WriteTable(Table *T, BaseType E);

void DoneTable(Table *T);

#endif
\end{minted}
table.c (реализации функций)
\begin{minted}{C}
#include <lab8/table.h>

#include "../lab5/task2.c"

#include <stddef.h>

int TableError = TableOk;

Table InitTable() {
    TableError = TableOk;
    Table result;
    InitList(&result);

    return result;
}

bool EmptyTable(Table *T) {
    TableError = TableOk;
    bool result = (Count(T) == 0);

    if (ListError == ListNotMem) TableError = TableNotMem;
    if (ListError == ListUnder) TableError = TableUnder;

    return result;
}

bool PutTable(Table *T, BaseType E) {
    TableError = TableOk;
    BeginPtr(T);
    while (!EndList(T) && T->ptr->next->data.Key < E.Key)
        MovePtr(T);

    if (!EndList(T) && T->ptr->next->data.Key == E.Key) {
        return false;
    }

    PutList(T, E);

    if (ListError == ListNotMem) {
        TableError = TableNotMem;
        return false;
    }

    return true;
}

bool GetTable(Table *T, BaseType *E, T_Key Key) {
    TableError = TableOk;
    BeginPtr(T);
    while (!EndList(T) && T->ptr->next->data.Key < Key)
        MovePtr(T);
    
    if (EndList(T)) return false;

    if (T->ptr->next->data.Key == Key) {
        GetList(T, E);

        return true;
    }

    return false;
}

bool ReadTable(Table *T, BaseType *E, T_Key Key) {
    TableError = TableOk;
    BeginPtr(T);
    while (!EndList(T) && T->ptr->next->data.Key < Key)
        MovePtr(T);
    
    if (EndList(T)) return false;

    if (T->ptr->next->data.Key == Key) {
        ReadList(T, E);

        return true;
    }

    return false;
}

bool WriteTable(Table *T, BaseType E) {
    TableError = TableOk;
    BeginPtr(T);
    while (!EndList(T) && T->ptr->next->data.Key < E.Key)
        MovePtr(T);
    
    if (EndList(T)) return false;

    if (T->ptr->next->data.Key == E.Key) {
        MovePtr(T);
        T->ptr->data = E;

        return true;
    }

    return false;
}

void DoneTable(Table *T) {
    TableError = TableOk;
    DoneList(T);
}
\end{minted}
\item Разработать программу для решения задачи в соответствии с вариантом
индивидуального задания (см. табл.18) с использованием модуля, полученного
в результате выполнения пункта 2 задания.\\
main.c (основная программа)
\begin{minted}{C}
#include <stdint.h>

#define CUSTOM_TYPES_TABLE
typedef uint64_t T_Key;
typedef float T_Value;

#include "../../libs/alg/lab8/table.c"
#include "../../libs/alg/lab6/stack.c"

#include <stdio.h>
#include <stdbool.h>
#include <ctype.h>
#include <string.h>
#include <math.h>

#define TOKEN_BUFFER_SIZE sizeof(T_Key)

bool isStringFloat(const char *s) {
    int len;
    float ignore;
    int ret = sscanf(s, "%f %n", &ignore, &len);
    return ret == 1 && !s[len];
}

int main() {
    Stack tokens;
    InitStack(&tokens);
    Table tokenValues = InitTable();
    char formula[1024];
    int fIndex = 0;
    gets(formula);

    int input;
    char buffer[TOKEN_BUFFER_SIZE] = {0};
    int bufferCurrentIndex = 0;
    while (1) {
        input = formula[fIndex++];

        if (input == ' ' || input == '\0') {
            if (isStringFloat(buffer)) {
                float val = atof(buffer);
                PutStack(&tokens, (BaseType) {0, val});
            } else if (!strcmp(buffer, "+") || !strcmp(buffer, "-") || !strcmp(buffer, "*") || !strcmp(buffer, "/")) {
                BaseType leftOperand, rightOperand;
                GetStack(&tokens, &leftOperand);
                if (StackError != StackOk) {
                    fprintf(stderr, "Unable to parse");
                    return 1;
                }

                GetStack(&tokens, &rightOperand);
                if (StackError != StackOk) {
                    fprintf(stderr, "Unable to parse");
                    return 1;
                }

                if (!strcmp(buffer, "+"))
                    PutStack(&tokens, (BaseType){0, leftOperand.Value + rightOperand.Value});
                else if (!strcmp(buffer, "-"))
                    PutStack(&tokens, (BaseType){0, rightOperand.Value - leftOperand.Value}); 
                else if (!strcmp(buffer, "*"))
                    PutStack(&tokens, (BaseType){0, leftOperand.Value * rightOperand.Value}); 
                else if (!strcmp(buffer, "/"))
                    PutStack(&tokens, (BaseType){0, rightOperand.Value / leftOperand.Value}); 
            } else if (!strcmp(buffer, "sin") || !strcmp(buffer, "cos") || !strcmp(buffer, "arctan") || 
                       !strcmp(buffer, "abs") || !strcmp(buffer, "exp") || !strcmp(buffer, "ln") ||
                       !strcmp(buffer, "sqr") || !strcmp(buffer, "sqrt")) {
                BaseType operand;
                GetStack(&tokens, &operand);
                if (StackError != StackOk) {
                    fprintf(stderr, "Unable to parse");
                    return 1;
                }

                if (!strcmp(buffer, "sin"))
                    PutStack(&tokens, (BaseType){0, sinf(operand.Value)});
                if (!strcmp(buffer, "cos"))
                    PutStack(&tokens, (BaseType){0, cosf(operand.Value)});
                if (!strcmp(buffer, "arctan"))
                    PutStack(&tokens, (BaseType){0, atanf(operand.Value)});
                if (!strcmp(buffer, "abs"))
                    PutStack(&tokens, (BaseType){0, fabs(operand.Value)});
                if (!strcmp(buffer, "exp"))
                    PutStack(&tokens, (BaseType){0, expf(operand.Value)});
                if (!strcmp(buffer, "ln"))
                    PutStack(&tokens, (BaseType){0, log2f(operand.Value) / log2f(expf(1))});
                if (!strcmp(buffer, "sqr"))
                    PutStack(&tokens, (BaseType){0, operand.Value * operand.Value});
                if (!strcmp(buffer, "sqrt"))
                    PutStack(&tokens, (BaseType){0, sqrtf(operand.Value)});
            } else {
                T_Key key;
                memcpy(&key, buffer, sizeof(T_Key));

                BaseType token = (BaseType) {key, 0};
                if (!GetTable(&tokenValues, &token, key)) {
                    printf("Input value of variable %s: ", buffer);
                    scanf("%f", &token.Value);

                    PutTable(&tokenValues, token);
                }

                PutStack(&tokens, token);
            }

            memset(buffer, 0, TOKEN_BUFFER_SIZE);
            bufferCurrentIndex = 0;

            if (input == '\0') break;
        } else {
            if (bufferCurrentIndex >= TOKEN_BUFFER_SIZE - 1) {
                fprintf(stderr, "Unable to parse, %s is too long, you can use only 7 symbols long varialbes.", buffer);
                return 1;
            }

            buffer[bufferCurrentIndex++] = input;
        }
    }

    BaseType result;
    GetStack(&tokens, &result);
    if (StackError != StackOk || !EmptyStack(tokens)) {
        fprintf(stderr, "Unable to parse");
        return 1;
    }
    
    printf("Result: %f", result.Value);

    return 0;
}
\end{minted}
\end{enumerate}
\href{https://github.com/IAmProgrammist/algorithms_and_data_structures/tree/main}{Ссылка на репозиторий}\\
\textbf{Вывод: } в ходе лабораторной работы изучили СД типа «таблица», научились их программно 
реализовывать и использовать.

\end{document}