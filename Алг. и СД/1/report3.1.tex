\documentclass[a4paper,14pt]{extarticle}


\usepackage[english,russian]{babel}
\usepackage[T2A]{fontenc}
\usepackage[utf8]{inputenc}
\usepackage{ragged2e}
\usepackage[utf8]{inputenc}
\usepackage{hyperref}
\usepackage{minted}
\setmintedinline{frame=lines, framesep=2mm, baselinestretch=1.5, bgcolor=LightGray, breaklines,fontsize=\footnotesize}
\setminted{frame=lines, framesep=2mm, baselinestretch=1.5, bgcolor=LightGray, breaklines,fontsize=\footnotesize}
\usepackage{xcolor}
\definecolor{LightGray}{gray}{0.9}
\usepackage{graphicx}
\usepackage[export]{adjustbox}
\usepackage[left=1cm,right=1cm, top=1cm,bottom=1cm,bindingoffset=0cm]{geometry}
\usepackage{fontspec}
\usepackage{ upgreek }
\usepackage[shortlabels]{enumitem}
\usepackage[mathletters]{ucs}
\usepackage{adjustbox}
\usepackage{multirow}
\usepackage{amsmath}
\usepackage{amssymb}
\usepackage{pifont}
\usepackage{pgfplots}
\graphicspath{ {./images/} }
\makeatletter
\AddEnumerateCounter{\asbuk}{\russian@alph}{щ}
\makeatother
\setmonofont{Consolas}
\setmainfont{Times New Roman}

\newcommand\textbox[1]{
	\parbox{.45\textwidth}{#1}
}

\newcommand{\specialcell}[2][c]{%
	\begin{tabular}[#1]{@{}c@{}}#2\end{tabular}}

\begin{document}
\pagenumbering{gobble}
\begin{center}
	\small{
		МИНИCТЕРCТВО НАУКИ И ВЫCШЕГО ОБРАЗОВАНИЯ \\РОCCИЙCКОЙ ФЕДЕРАЦИИ
		\bigbreak
		ФЕДЕРАЛЬНОЕ ГОCУДАРCТВЕННОЕ БЮДЖЕТНОЕ ОБРАЗОВАТЕЛЬНОЕ УЧРЕЖДЕНИЕ ВЫCШЕГО ОБРАЗОВАНИЯ \\
		\bigbreak
		\textbf{«БЕЛГОРОДCКИЙ ГОCУДАРCТВЕННЫЙ \\ТЕХНОЛОГИЧЕCКИЙ УНИВЕРCИТЕТ им. В. Г. ШУХОВА»\\ (БГТУ им. В.Г. Шухова)} \\
		\bigbreak
		Кафедра программного обеспечения вычислительной техники и автоматизированных систем\\}
\end{center}

\vfill
\begin{center}
	\large{
		\textbf{
			Лабораторная работа №1}}\\
	\normalsize{
		по дисциплине: Алгоритмы и структуры данных \\
		тема: «Встроенные структуры данных (Pascal/С)»}
\end{center}
\vfill
\hfill\textbox{
	Выполнил: ст. группы ПВ-223\\Пахомов Владислав Андреевич
	\bigbreak
	Проверили: асс. Солонченко Роман\\Евгеньевич
}
\vfill\begin{center}
	Белгород 2023 г.
\end{center}
\newpage
\begin{center}
	\textbf{Лабораторная работа №1}\\
	Встроенные структуры данных (Pascal/С)\\
	Вариант 10
\end{center}
\textbf{Цель работы: }изучение базовых типов данных языка Pascal/C как структур данных (СД).

\begin{enumerate}
	\item Для типов данных определить:
	      \begin{enumerate}[label*=\arabic*.]
		      \item Абстрактный уровень представления СД:

		            \begin{enumerate}[label*=\arabic*.]
			            \item Характер организованности и изменчивости.
			            \item Набор допустимых операций.
		            \end{enumerate}

		      \item Физический уровень представления СД:

		            \begin{enumerate}[label*=\arabic*.]
			            \item Схему хранения.
			            \item Объем памяти, занимаемый экземпляром СД.
			            \item Формат внутреннего представления СД и способ его интерпретации.
			            \item Характеристику допустимых значений.
			            \item Тип доступа к элементам.
		            \end{enumerate}

		      \item Логический уровень представления СД.
		            \begin{enumerate}[label*=\arabic*.]
			            \item Способ описания СД и экземпляра СД на языке программирования.
		            \end{enumerate}
	      \end{enumerate}

	      \begin{center}Задания для C\\
		      \begin{tabular}{|c|c|c|}
			      \hline
			      Тип 1       & Тип 2 & Тип 3                         \\
			      \hline
			      signed char & float & \{red, yellow, green\} colors \\
			      \hline
		      \end{tabular}
	      \end{center}
	      \textbf{signed char}
	      \begin{enumerate}[label*=\arabic*.]
		      \item Абстрактный уровень представления СД:

		            \begin{enumerate}[label*=\arabic*.]
			            \item Характер организованности - \textbf{простой}\\
			                  Характер изменчивости - \textbf{статический}
			            \item Набор допустимых операций - \textbf{математические операции, побитовые операции, присваивание, инициализация, логические операции,
				                  приведение типа, взятие адреса}
		            \end{enumerate}

		      \item Физический уровень представления СД:

		            \begin{enumerate}[label*=\arabic*.]
			            \item Схема хранения - \textbf{последовательная память}.
			            \item Объем памяти, занимаемый экземпляром СД.
			                  Размер \mintinline{C}|signed char| в современном C гарантированно равен \textbf{1 байту или 8 битам}.
			            \item Формат внутреннего представления СД и способ его интерпретации.
			                  \textbf{8-битное число}. Старший бит отводится под хранение знака числа, отрицательные числа
			                  хранятся в \textbf{дополнительном коде}, положительные - \textbf{в прямом}.
			            \item Характеристика допустимых значений.
			                  $E(\text{signed char}) \in [-2^7; 2^7 - 1]$ или
			                  $E(\text{signed char}) \in [-128; 127]$.
			            \item Тип доступа к элементам - \textbf{прямой}.
		            \end{enumerate}

		      \item Логический уровень представления СД.
		            \begin{enumerate}[label*=\arabic*.]
			            \item Способ описания СД и экземпляра СД на языке программирования.
			                  \begin{minted}{C}
signed char a;
char a;
					  \end{minted}
		            \end{enumerate}
	      \end{enumerate}

	      \textbf{float}
	      \begin{enumerate}[label*=\arabic*.]
		      \item Абстрактный уровень представления СД:

		            \begin{enumerate}[label*=\arabic*.]
			            \item Характер организованности - \textbf{простой}\\
			                  Характер изменчивости - \textbf{статический}
			            \item Набор допустимых операций - \textbf{математические операции, побитовые операции, присваивание, инициализация, логические операции,
				                  приведение типа, взятие адреса}
		            \end{enumerate}

		      \item Физический уровень представления СД:

		            \begin{enumerate}[label*=\arabic*.]
			            \item Схема хранения - \textbf{последовательная память}.
			            \item Объем памяти, занимаемый экземпляром СД.
			                  Размер \mintinline{C}|float| может различаться на различных системах.
			                  Однако на большинстве ПК размер float равен \textbf{4 байтам или 32 битам}.
			            \item Формат внутреннего представления СД и способ его интерпретации.
			                  \textbf{32-битное число}. Старший бит \mintinline{C}|s| отводится под хранение знака числа.
			                  Следующие 8 бит \mintinline{C}|e| содержат порядок числа, последние 23 бита содержат мантиссу числа \mintinline{C}|m|.
			            \item Характеристика допустимых значений.\\
			                  $E(\text{float}) \in [1.1754943\cdot10^{-38}; 1.1754943\cdot10^{38}]$.
			            \item Тип доступа к элементам - \textbf{прямой}.
		            \end{enumerate}

		      \item Логический уровень представления СД.
		            \begin{enumerate}[label*=\arabic*.]
			            \item Способ описания СД и экземпляра СД на языке программирования.
			                  \begin{minted}{C}
float a;
					  \end{minted}
		            \end{enumerate}
	      \end{enumerate}

	      \textbf{\{red, yellow, green\} colors}
	      \begin{enumerate}[label*=\arabic*.]
		      \item Абстрактный уровень представления СД:

		            \begin{enumerate}[label*=\arabic*.]
			            \item Характер организованности - \textbf{линейный}\\
			                  Характер изменчивости - \textbf{статический}
			            \item Набор допустимых операций - \textbf{объявление, получение значения по идентификатору}
		            \end{enumerate}

		      \item Физический уровень представления СД:

		            \begin{enumerate}[label*=\arabic*.]
			            \item Схема хранения - \textbf{последовательная память}.
			            \item Объем памяти, занимаемый экземпляром СД.\\
			                  enum представляет из себя список констант, каждой из которых присвоено
			                  значение типа int. Размер СД будет равен $N\cdot S$, где N - количество элементов, $S = sizeof(int)$.
			            \item Формат внутреннего представления СД и способ его интерпретации.\\
			                  Последовательность N элементов одного типа int.
			            \item Характеристика допустимых значений.\\
			                  Максимальная мощность равна $2 ^ {sizeof(int) \cdot 8} = 4294967296$.
			            \item Тип доступа к элементам - \textbf{прямой}.
		            \end{enumerate}

		      \item Логический уровень представления СД.
		            \begin{enumerate}[label*=\arabic*.]
			            \item Способ описания СД и экземпляра СД на языке программирования.
			                  \begin{minted}{C}
typedef enum {
    RED, // 0
    GREEN, // 1
    BLUE // 2
} Color;
// Можно также вручную задать значения констант.
typedef enum {
    RED_C = 45, 
    GREEN_C, // 46
    BLUE_C = RED_C + 4 //49
} ColorCustomNumeration;
					  \end{minted}
		            \end{enumerate}
	      \end{enumerate}
	\item Для заданных типов данных определить набор значений, необходимый
	      для изучения физического уровня представления СД.\\
	      \textbf{signed char}
		  \begin{enumerate}[1. ]
			\item -12
			\item 55
		  \end{enumerate}
	      \textbf{float}
		  \begin{enumerate}[1. ]
			\item 12.5
			\item -0.75
		  \end{enumerate}
	      \textbf{\{red, yellow, green\} colors}
		  \begin{enumerate}[1. ]
			\item red
			\item green
		  \end{enumerate}
\item Преобразовать значения в двоичный код.
\textbf{signed char}
\begin{enumerate}[1. ]
  \item -12\\
  $-12_{10} = -00001100_2 (\textit{прямой код}) = 1'1110011_2 (\textit{обратный код}) = $\\$ = 1'1110100 (\textit{дополнительный код})$
  \item 55\\
  $55_{10} = 00110111_2 (\textit{прямой код})$\\
\end{enumerate}
\textbf{float}
\begin{enumerate}[1. ]
  \item 12.5
  \item -0.75
\end{enumerate}
\textbf{\{red, yellow, green\} colors}
\begin{enumerate}[1. ]
  \item red
  \item green
\end{enumerate}
\end{enumerate}
\textbf{Вывод: } в ходе лабораторной работы изучили способы задания отношений, операции над отношениями и свойства отношений, научились программно реализовывать операции и определять свойства отношений.

\end{document}