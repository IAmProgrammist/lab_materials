\documentclass[a4paper,14pt]{extarticle}


\usepackage[english,russian]{babel}
\usepackage[T2A]{fontenc}
\usepackage[utf8]{inputenc}
\usepackage{ragged2e}
\usepackage[utf8]{inputenc}
\usepackage{hyperref}
\usepackage{minted}
\setmintedinline{frame=lines, framesep=2mm, baselinestretch=1.5, bgcolor=LightGray, breaklines,fontsize=\footnotesize}
\setminted{frame=lines, framesep=2mm, baselinestretch=1.5, bgcolor=LightGray, breaklines,fontsize=\footnotesize}
\usepackage{xcolor}
\definecolor{LightGray}{gray}{0.9}
\usepackage{graphicx}
\usepackage[export]{adjustbox}
\usepackage[left=1cm,right=1cm, top=1cm,bottom=1cm,bindingoffset=0cm]{geometry}
\usepackage{fontspec}
\usepackage{ upgreek }
\usepackage[shortlabels]{enumitem}
\usepackage[mathletters]{ucs}
\usepackage{adjustbox}
\usepackage{multirow}
\usepackage{amsmath}
\usepackage{amssymb}
\usepackage{pifont}
\usepackage{pgfplots}
\graphicspath{ {./images/} }
\makeatletter
\AddEnumerateCounter{\asbuk}{\russian@alph}{щ}
\makeatother
\setmonofont{Consolas}
\setmainfont{Times New Roman}

\newcommand\textbox[1]{
	\parbox{.45\textwidth}{#1}
}

\newcommand{\specialcell}[2][c]{%
	\begin{tabular}[#1]{@{}c@{}}#2\end{tabular}}

\begin{document}
\pagenumbering{gobble}
\begin{center}
	\small{
		МИНИCТЕРCТВО НАУКИ И ВЫCШЕГО ОБРАЗОВАНИЯ \\РОCCИЙCКОЙ ФЕДЕРАЦИИ
		\bigbreak
		ФЕДЕРАЛЬНОЕ ГОCУДАРCТВЕННОЕ БЮДЖЕТНОЕ ОБРАЗОВАТЕЛЬНОЕ УЧРЕЖДЕНИЕ ВЫCШЕГО ОБРАЗОВАНИЯ \\
		\bigbreak
		\textbf{«БЕЛГОРОДCКИЙ ГОCУДАРCТВЕННЫЙ \\ТЕХНОЛОГИЧЕCКИЙ УНИВЕРCИТЕТ им. В. Г. ШУХОВА»\\ (БГТУ им. В.Г. Шухова)} \\
		\bigbreak
		Кафедра программного обеспечения вычислительной техники и автоматизированных систем\\}
\end{center}

\vfill
\begin{center}
	\large{
		\textbf{
			Лабораторная работа №1}}\\
	\normalsize{
		по дисциплине: Алгоритмы и структуры данных \\
		тема: «Встроенные структуры данных (Pascal/С)»}
\end{center}
\vfill
\hfill\textbox{
	Выполнил: ст. группы ПВ-223\\Пахомов Владислав Андреевич
	\bigbreak
	Проверили: асс. Солонченко Роман\\Евгеньевич
}
\vfill\begin{center}
	Белгород 2023 г.
\end{center}
\newpage
\begin{center}
	\textbf{Лабораторная работа №1}\\
	Встроенные структуры данных (Pascal/С)\\
	Вариант 10
\end{center}
\textbf{Цель работы: }изучение базовых типов данных языка Pascal/C как структур данных (СД).

\begin{enumerate}
	\item Для типов данных определить:
	      \begin{enumerate}[label*=\arabic*.]
		      \item Абстрактный уровень представления СД:

		            \begin{enumerate}[label*=\arabic*.]
			            \item Характер организованности и изменчивости.
			            \item Набор допустимых операций.
		            \end{enumerate}

		      \item Физический уровень представления СД:

		            \begin{enumerate}[label*=\arabic*.]
			            \item Схему хранения.
			            \item Объем памяти, занимаемый экземпляром СД.
			            \item Формат внутреннего представления СД и способ его интерпретации.
			            \item Характеристику допустимых значений.
			            \item Тип доступа к элементам.
		            \end{enumerate}

		      \item Логический уровень представления СД.
		            \begin{enumerate}[label*=\arabic*.]
			            \item Способ описания СД и экземпляра СД на языке программирования.
		            \end{enumerate}
	      \end{enumerate}

	      \begin{center}Задания для C\\
		      \begin{tabular}{|c|c|c|}
			      \hline
			      Тип 1       & Тип 2 & Тип 3                         \\
			      \hline
			      signed char & float & \{red, yellow, green\} colors \\
			      \hline
		      \end{tabular}
	      \end{center}
	      \textbf{signed char}
	      \begin{enumerate}[label*=\arabic*.]
		      \item Абстрактный уровень представления СД:

		            \begin{enumerate}[label*=\arabic*.]
			            \item Характер организованности - \textbf{простой}\\
			                  Характер изменчивости - \textbf{статический}
			            \item Набор допустимых операций - \textbf{математические операции, побитовые операции, присваивание, инициализация, логические операции,
				                  приведение типа, взятие адреса}
		            \end{enumerate}

		      \item Физический уровень представления СД:

		            \begin{enumerate}[label*=\arabic*.]
			            \item Схема хранения - \textbf{последовательная память}.
			            \item Объем памяти, занимаемый экземпляром СД.
			                  Размер \mintinline{C}|signed char| в современном C гарантированно равен \textbf{1 байту или 8 битам}.
			            \item Формат внутреннего представления СД и способ его интерпретации.
			                  \textbf{8-битное число}. Старший бит отводится под хранение знака числа, отрицательные числа
			                  хранятся в \textbf{дополнительном коде}, положительные - \textbf{в прямом}.
			            \item Характеристика допустимых значений.
			                  $E(\text{signed char}) \in [-2^7; 2^7 - 1]$ или
			                  $E(\text{signed char}) \in [-128; 127]$.
			            \item Тип доступа к элементам - \textbf{прямой}.
		            \end{enumerate}

		      \item Логический уровень представления СД.
		            \begin{enumerate}[label*=\arabic*.]
			            \item Способ описания СД и экземпляра СД на языке программирования.
			                  \begin{minted}{C}
signed char a;
char a;
					  \end{minted}
		            \end{enumerate}
	      \end{enumerate}

	      \textbf{float}
	      \begin{enumerate}[label*=\arabic*.]
		      \item Абстрактный уровень представления СД:

		            \begin{enumerate}[label*=\arabic*.]
			            \item Характер организованности - \textbf{простой}\\
			                  Характер изменчивости - \textbf{статический}
			            \item Набор допустимых операций - \textbf{математические операции, побитовые операции, присваивание, инициализация, логические операции,
				                  приведение типа, взятие адреса}
		            \end{enumerate}

		      \item Физический уровень представления СД:

		            \begin{enumerate}[label*=\arabic*.]
			            \item Схема хранения - \textbf{последовательная память}.
			            \item Объем памяти, занимаемый экземпляром СД.
			                  Размер \mintinline{C}|float| может различаться на различных системах.
			                  Однако на большинстве ПК размер float равен \textbf{4 байтам или 32 битам}.
			            \item Формат внутреннего представления СД и способ его интерпретации.
			                  \textbf{32-битное число}. Старший бит \mintinline{C}|s| отводится под хранение знака числа.
			                  Следующие 8 бит \mintinline{C}|e| содержат порядок числа, последние 23 бита содержат мантиссу числа \mintinline{C}|m|.
							  Число в памяти можно выразить следующей формулой: $(-1)^s \cdot 1.m \cdot 2^e$
			            \item Характеристика допустимых значений.\\
			                  $E(\text{float}) \in [1.1754943\cdot10^{-38}; 1.1754943\cdot10^{38}]$.
			            \item Тип доступа к элементам - \textbf{прямой}.
		            \end{enumerate}

		      \item Логический уровень представления СД.
		            \begin{enumerate}[label*=\arabic*.]
			            \item Способ описания СД и экземпляра СД на языке программирования.
			                  \begin{minted}{C}
float a;
					  \end{minted}
		            \end{enumerate}
	      \end{enumerate}

	      \textbf{\{red, yellow, green\} colors}
	      \begin{enumerate}[label*=\arabic*.]
		      \item Абстрактный уровень представления СД:

		            \begin{enumerate}[label*=\arabic*.]
			            \item Характер организованности - \textbf{линейный}\\
			                  Характер изменчивости - \textbf{статический}
			            \item Набор допустимых операций - \textbf{объявление, получение значения по идентификатору}
		            \end{enumerate}

		      \item Физический уровень представления СД:

		            \begin{enumerate}[label*=\arabic*.]
			            \item Схема хранения - \textbf{последовательная память}.
			            \item Объем памяти, занимаемый экземпляром СД.\\
			                  enum представляет из себя список констант, каждой из которых присвоено
			                  значение типа int. Размер СД будет равен $N\cdot S$, где N - количество элементов, $S = sizeof(int)$.
			            \item Формат внутреннего представления СД и способ его интерпретации.\\
			                  Последовательность N элементов одного типа int.
			            \item Характеристика допустимых значений.\\
			                  Максимальная мощность равна $2 ^ {sizeof(int) \cdot 8} = 4294967296$.
			            \item Тип доступа к элементам - \textbf{прямой}.
		            \end{enumerate}

		      \item Логический уровень представления СД.
		            \begin{enumerate}[label*=\arabic*.]
			            \item Способ описания СД и экземпляра СД на языке программирования.
			                  \begin{minted}{C}
typedef enum {
    RED, // 0
    GREEN, // 1
    BLUE // 2
} Color;
// Можно также вручную задать значения констант.
typedef enum {
    RED_C = 45, 
    GREEN_C, // 46
    BLUE_C = RED_C + 4 //49
} ColorCustomNumeration;
					  \end{minted}
		            \end{enumerate}
	      \end{enumerate}
	\item Для заданных типов данных определить набор значений, необходимый
	      для изучения физического уровня представления СД.\\
	      \textbf{signed char}
		  \begin{enumerate}[1. ]
			\item -12
			\item 55
		  \end{enumerate}
	      \textbf{float}
		  \begin{enumerate}[1. ]
			\item 12.5
			\item -431.75
		  \end{enumerate}
	      \textbf{\{red, yellow, green\} colors}
		  \begin{enumerate}[1. ]
			\item red
			\item green
		  \end{enumerate}
\item Преобразовать значения в двоичный код.\\
\textbf{signed char}
\begin{enumerate}[1. ]
  \item -12\\
  $-12_{10} = -00001100_2 (\textit{прямой код}) = 1'1110011_2 (\textit{обратный код}) = $\\$ = 1'1110100 (\textit{дополнительный код})$\\
  Запись в памяти:\\
  11110100
  \item 55\\
  $55_{10} = 00110111_2 (\textit{прямой код})$\\
  Запись в памяти:\\
  00110111
\end{enumerate}
\textbf{float}
\begin{enumerate}[1. ]
  \item 12.5\\
  Переведём 12.5 в двоичную систему счисления\\
  $12.5_{10} = 1 \cdot 2 ^ 3 + 1 \cdot 2 ^ 2 + 1 \cdot 2 ^ {-1} = 1100.1_2$\\
  Полученное число приведём к форме $(-1)^s \cdot 1.m \cdot 2^e$\\
  $1100.1_2 = (-1) ^ \textbf{0} \cdot 1.\textbf{1001} \cdot 2 ^ {\textbf{11}}$\\
  $s = 0; m = 1001; e = 11_2 \textit{ или } 3_{10}$\\
  Запись в памяти:\\
  00000001 10000000 00000000 00001001
  \item -431.75\\
  Переведём -431.75 в двоичную систему счисления\\
  $-431.75_{10} = -(1 \cdot 2 ^ 8 + 1 \cdot 2 ^ 7 + 1 \cdot 2 ^ 5 + 1 \cdot 2 ^ 3 + 1 \cdot 2 ^ 2 + 1 \cdot 2 ^ 1 + 1 \cdot 2 ^ 0 + 1 \cdot 2 ^ {-1} + 1 \cdot 2 ^ {-2}) = -110101111.11_2$\\
  Полученное число приведём к форме $(-1)^s \cdot 1.m \cdot 2^e$\\
  $-110101111.11_2 = (-1) ^ \textbf{1} \cdot 1.\textbf{1010111111} \cdot 2 ^ {\textbf{1000}}$\\
  $s = 0; m = 1010111111; e = 1000_2 \textit{ или } 8_{10}$\\
  Запись в памяти:\\
  10000100 00000000 00000010 10111111
\end{enumerate}
\textbf{\{red, yellow, green\} colors}
\begin{enumerate}[1. ]
  \item red\\
  red - первый элемент enum и ему не присвоено значение, а значит его значение будет равно значению по 
  умолчанию - 0. Тип значения - int, значит под хранение элемента на большинстве современных ПК будет выделено 32 бит.
  Запись в памяти:\\
  00000000 00000000 00000000 00000000
  \item green\\
  green - третий элемент enum, ему не присвоено значение и элементам до него не было присвоено значений, в таком случае номера элементов присваиваются по порядку.
  green будет равен 2. Тип значения - int, значит под хранение элемента на большинстве современных ПК будет выделено 32 бит.
  Запись в памяти:\\
  00000000 00000000 00000000 00000010
\end{enumerate}
\item Преобразовать двоичный код в значение.\\
\textbf{signed char}
\begin{enumerate}[1. ]
  \item 11110100\\
  Для перевода из дополнительного кода число необходимо инвертировать биты числа и прибавить 1 (алгоритм перевода числа из прямого кода в дополнительный аналогичен).\\
  $11110100 -> 00001011 -> 00001100 (прямой код)$\\
  $00001100_2 = 1 \cdot 2 ^ 3 + 1 \cdot 2 ^ 2 = \textbf{12}_{10}$.\\
  Значения до и после перевода совпали.
  \item 00110111\\
  Число уже находится в прямом коде, дополнительные действия выполнять не требуется.\\
  $00110111_2 = 1 \cdot 2 ^ 5 + 1 \cdot 2 ^ 4 + 1 \cdot 2 ^ 2 + 1 \cdot 2 ^ 1 + 1 \cdot 2 ^ 0 = \textbf{55}_{10}$.\\
\end{enumerate}
\textbf{float}
\begin{enumerate}[1. ]
  \item 00000001 10000000 00000000 00001001\\
  s: $\textbf{0}0000001 10000000 00000000 00001001 = 0$\\
  e: $0\textbf{0000001 1}0000000 00000000 00001001 = 11_2$\\
  m: $00000001 1\textbf{0000000 00000000 00001001} = 1001_2$\\
  $(-1)^s \cdot 1.m \cdot 2^e = (-1)^0 \cdot 1.1001 \cdot 2^{11_2} = 1 \cdot 1.1001 \cdot 2^{11_2} = 1100.1_2$\\
  $1100.1_2 = 1 \cdot 2 ^ 3 + 1 \cdot 2 ^ 2 + 1 \cdot 2 ^ {-1} = 12.5$\\
  Значения до и после перевода совпали.
  \item 10000100 00000000 00000010 10111111\\
  s: $\textbf{1}0000100 00000000 00000010 10111111 = 1$\\
  e: $1\textbf{0000100 0}0000000 00000010 10111111 = 1000_2$\\
  m: $10000100 0\textbf{0000000 00000010 10111111} = 1010111111_2$\\
  $(-1)^s \cdot 1.m \cdot 2^e = (-1)^1 \cdot 1.1010111111 \cdot 2^{1000_2} = 
  -1 \cdot 1.1010111111 \cdot 2^{1000_2} = 110101111.11_2$\\
  $-110101111.11_2 = -(1 \cdot 2 ^ 8 + 1 \cdot 2 ^ 7 + 1 \cdot 2 ^ 5 + 1 \cdot 2 ^ 3 + 1 \cdot 2 ^ 2 + 
  1 \cdot 2 ^ 1 + 1 \cdot 2 ^ 0 + 1 \cdot 2 ^ {-1} + 1 \cdot 2 ^ {-2}) = -431.75$\\
  Значения до и после перевода совпали.
\end{enumerate}
\textbf{\{red, yellow, green\} colors}\\
Константам из colors будут соответствовать следующие значения:\\
red: 0\\
yellow: 1\\
green: 2\\
\begin{enumerate}[1. ]
  \item 00000000 00000000 00000000 00000000\\
  $00000000 00000000 00000000 00000000_2 = 0_{10}$. Значению 0 соответствует константа red.\\
  Значения до и после перевода совпали.
  \item 00000000 00000000 00000000 00000010\\
  $00000000 00000000 00000000 00000010_2 = 2_{10}$. Значению 2 соответствует константа green.\\
  Значения до и после перевода совпали.
\end{enumerate}
\item Разработать и отладить программу, выдающую двоичное представление значений, заданных СД.\\
\item 
\end{enumerate}
\textbf{Вывод: } в ходе лабораторной работы изучили способы задания отношений, операции над отношениями и свойства отношений, научились программно реализовывать операции и определять свойства отношений.

\end{document}