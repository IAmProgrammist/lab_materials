\documentclass[a4paper,14pt]{extarticle}


\usepackage[english,russian]{babel}
\usepackage[T2A]{fontenc}
\usepackage[utf8]{inputenc}
\usepackage{ragged2e}
\usepackage[utf8]{inputenc}
\usepackage{hyperref}
\usepackage{minted}
\setmintedinline{frame=lines, framesep=2mm, baselinestretch=1.5, bgcolor=LightGray, breaklines,fontsize=\footnotesize}
\setminted{frame=lines, framesep=2mm, baselinestretch=1.5, bgcolor=LightGray, breaklines,fontsize=\footnotesize}
\usepackage{xcolor}
\definecolor{LightGray}{gray}{0.9}
\usepackage{graphicx}
\usepackage[export]{adjustbox}
\usepackage[left=1cm,right=1cm, top=1cm,bottom=1cm,bindingoffset=0cm]{geometry}
\usepackage{fontspec}
\usepackage{ upgreek }
\usepackage[shortlabels]{enumitem}
\usepackage{adjustbox}
\usepackage{multirow}
\usepackage{amsmath}
\usepackage{amssymb}
\usepackage{pifont}
\usepackage{pgfplots}
\graphicspath{ {./images/} }
\makeatletter
\AddEnumerateCounter{\asbuk}{\russian@alph}{щ}
\makeatother
\setmonofont{Consolas}
\setmainfont{Times New Roman}

\newcommand\textbox[1]{
	\parbox{.45\textwidth}{#1}
}

\newcommand{\specialcell}[2][c]{%
	\begin{tabular}[#1]{@{}c@{}}#2\end{tabular}}

\begin{document}
\pagenumbering{gobble}
\begin{center}
	\small{
		МИНИCТЕРCТВО НАУКИ И ВЫCШЕГО ОБРАЗОВАНИЯ \\РОCCИЙCКОЙ ФЕДЕРАЦИИ
		\bigbreak
		ФЕДЕРАЛЬНОЕ ГОCУДАРCТВЕННОЕ БЮДЖЕТНОЕ ОБРАЗОВАТЕЛЬНОЕ УЧРЕЖДЕНИЕ ВЫCШЕГО ОБРАЗОВАНИЯ \\
		\bigbreak
		\textbf{«БЕЛГОРОДCКИЙ ГОCУДАРCТВЕННЫЙ \\ТЕХНОЛОГИЧЕCКИЙ УНИВЕРCИТЕТ им. В. Г. ШУХОВА»\\ (БГТУ им. В.Г. Шухова)} \\
		\bigbreak
		Кафедра программного обеспечения вычислительной техники и автоматизированных систем\\}
\end{center}

\vfill
\begin{center}
	\large{
		\textbf{
			Лабораторная работа №5}}\\
	\normalsize{
		по дисциплине: Алгоритмы и структуры данных \\
		тема: Структуры данных «линейные списки» (Pascal/С)»}
\end{center}
\vfill
\hfill\textbox{
	Выполнил: ст. группы ПВ-223\\Пахомов Владислав Андреевич
	\bigbreak
	Проверили: асс. Солонченко Роман\\Евгеньевич
}
\vfill\begin{center}
	Белгород 2023 г.
\end{center}
\newpage
\begin{center}
	\textbf{Лабораторная работа №5}\\
	Структуры данных «линейные списки» (Pascal/С)»\\
	Вариант 10
\end{center}
\textbf{Цель работы: }изучить СД типа «линейный список», научиться их 
программно реализовывать и использовать.

\begin{enumerate}
	\item Для СД типа «линейный список» определить:
	      \begin{enumerate}[label*=\arabic*.]
		      \item Абстрактный уровень представления СД:

		            \begin{enumerate}[label*=\arabic*.]
			            \item Характер организованности и изменчивости.\\
			            Характер организованности - \textbf{линейный}. Характер изменчивости - \textbf{динамический}.
			            \item Набор допустимых операций.\\
			            Инициализация, включение элемента, исключение элемента,
                        чтение текущего элемента,
                        переход в начало списка,
                        переход в конец списка,
                        переход к следующему элементу,
                        переход к i-му элементу,
                        определение длины списка,
                        уничтожение списка.
		            \end{enumerate}

		      \item Физический уровень представления СД:

		            \begin{enumerate}[label*=\arabic*.]
			            \item Схему хранения.\\
			            Схема хранения - \textbf{связный.}
			            \item Объем памяти, занимаемый экземпляром СД.\\
			            Размер СД состоит из размера дескриптора, размера фиктивного элемента и размера всех элементов.
                        Размер дескриптора: указатель на начало (машинное слово - 4 байт), рабочий указатель (машинное слово - 4 байт) и количество элементов (int - 4 байт).
                        Размер элементов: $1 + N\cdot(sizeof(BaseType) + 4)$ байт.
                        $V = 13 + N\cdot(sizeof(BaseType) + 4)$
			            \item Формат внутреннего представления СД и способ его интерпретации.\\
			            Дескриптор находится в статической памяти, элементы ОЛС находятся в динамической памяти.
			            \item Характеристику допустимых значений.\\
			            $Car(C) = 1 + Car(BaseType) + Car(BaseType) ^ 2 + ... + Car(BaseType) ^ {max}$.
			            \item Тип доступа к элементам.\\
						Тип доступа к элементам - \textbf{последовательный}.
		            \end{enumerate}

		      \item Логический уровень представления СД.
		            \begin{enumerate}[label*=\arabic*.]
			            \item Способ описания СД и экземпляра СД на языке программирования.\\
						\begin{minted}{C}
List l;
InitList(&l);
												  \end{minted}
		            \end{enumerate}
	      \end{enumerate}
		  \item Реализовать СД типа «линейный список» в соответствии с 
          вариантом индивидуального задания (см. табл.14) в виде модуля.\\
main.c (тесты)
		  \begin{minted}{C}
#include <algc.h>

#include <stdio.h>
#include <assert.h>
#include <string.h>

void testPutList() {
    List t1;
    InitList(&t1);
    PutList(&t1, 1);
    int e;
    ReadList(&t1, &e);
    assert(ListError == ListOk && 
    t1.N == 1 && 
    e == 1);

    PutList(&t1, 2);
    ReadList(&t1, &e);
    assert(ListError == ListOk && 
    t1.N == 2 && 
    e == 2);

    MovePtr(&t1);
    MovePtr(&t1);
    PutList(&t1, 3);
    ReadList(&t1, &e);
    assert(ListError == ListOk && 
    t1.N == 3 && 
    e == 3);

    DoneList(&t1);
}

void testGetList() {
    List t1;
    InitList(&t1);
    PutList(&t1, 1);
    MovePtr(&t1);
    PutList(&t1, 2);
    MovePtr(&t1);
    PutList(&t1, 3);
    int e;

    EndPtr(&t1);
    GetList(&t1, &e);
    assert(ListError == ListEnd);

    BeginPtr(&t1);
    MovePtr(&t1);
    MovePtr(&t1);
    GetList(&t1, &e);
    assert(ListError == ListOk && e == 3);

    BeginPtr(&t1);
    MovePtr(&t1);
    GetList(&t1, &e);
    assert(ListError == ListOk && e == 2);

    BeginPtr(&t1);
    GetList(&t1, &e);
    assert(ListError == ListOk && e == 1);

    GetList(&t1, &e);
    assert(ListError == ListUnder || ListError == ListEnd);

    DoneList(&t1);
}

void testReadList() {
    List t1;
    InitList(&t1);
    PutList(&t1, 1);
    MovePtr(&t1);
    PutList(&t1, 2);
    MovePtr(&t1);
    PutList(&t1, 3);
    int e;

    EndPtr(&t1);
    ReadList(&t1, &e);
    assert(ListError == ListEnd);

    BeginPtr(&t1);
    MovePtr(&t1);
    MovePtr(&t1);
    ReadList(&t1, &e);
    assert(ListError == ListOk && e == 3);

    BeginPtr(&t1);
    MovePtr(&t1);
    ReadList(&t1, &e);
    assert(ListError == ListOk && e == 2);

    BeginPtr(&t1);
    ReadList(&t1, &e);
    assert(ListError == ListOk && e == 1);
    
    DoneList(&t1);
}

void testEndList() {
    List t1;
    InitList(&t1);
    EndList(&t1);
    assert(ListError == ListOk);

    PutList(&t1, 1);
    PutList(&t1, 2);
    PutList(&t1, 3);
    BeginPtr(&t1);
    assert(ListError == ListOk && !EndList(&t1));
    MovePtr(&t1);
    assert(ListError == ListOk && !EndList(&t1));
    MovePtr(&t1);
    assert(ListError == ListOk && !EndList(&t1));
    MovePtr(&t1);
    assert(ListError == ListOk && EndList(&t1));
    
    DoneList(&t1);
}

void testCount() {
    List t1;
    InitList(&t1);
    assert(Count(&t1) == 0 && ListError == ListOk);
    PutList(&t1, 1);
    assert(Count(&t1) == 1 && ListError == ListOk);
    PutList(&t1, 2);
    assert(Count(&t1) == 2 && ListError == ListOk);
    PutList(&t1, 3);
    assert(Count(&t1) == 3 && ListError == ListOk);
    
    DoneList(&t1);
}

void testBeginPtr() {
    List t1;
    InitList(&t1);
    PutList(&t1, 1);
    PutList(&t1, 2);
    MovePtr(&t1);
    PutList(&t1, 3);
    BeginPtr(&t1);

    assert(t1.ptr == t1.Start && ListError == ListOk);
    
    DoneList(&t1);
}

void testEndPtr() {
    List t1;
    InitList(&t1);

    PutList(&t1, 1);
    MovePtr(&t1);
    PutList(&t1, 2);
    MovePtr(&t1);
    PutList(&t1, 3);
    EndPtr(&t1);

    assert(t1.ptr->data == 3 && ListError == ListOk);
    
    DoneList(&t1);
}

void testMovePtr() {
    List t1;
    InitList(&t1);
    MovePtr(&t1);
    assert(ListError == ListEnd);

    PutList(&t1, 1);
    MovePtr(&t1);
    assert(ListError == ListOk && t1.ptr->data == 1);
    PutList(&t1, 2);
    MovePtr(&t1);
    assert(ListError == ListOk && t1.ptr->data == 2);

    PutList(&t1, 3);
    MovePtr(&t1);
    assert(ListError == ListOk && t1.ptr->data == 3);

    DoneList(&t1);
}

void testMoveTo() {
    List t1;
    InitList(&t1);
    MoveTo(&t1, 177);
    assert(ListError == ListEnd);

    PutList(&t1, 1);
    MovePtr(&t1);
    PutList(&t1, 2);
    MovePtr(&t1);
    PutList(&t1, 3);

    MoveTo(&t1, 0);
    assert(ListError == ListOk && t1.ptr->data == 1);
    MoveTo(&t1, 2);
    assert(ListError == ListOk && t1.ptr->data == 3);
    MoveTo(&t1, 1);
    assert(ListError == ListOk && t1.ptr->data == 2);

    DoneList(&t1);
}

void test() {
    testPutList();
    testGetList();
    testReadList();
    testEndList();
    testCount();
    testBeginPtr();
    testEndPtr();
    testMovePtr();
    testMoveTo();
}

int main() {
    test();
    
    return 0;
}
\end{minted}
alg.h (заголовки)
\begin{minted}{C}
#ifndef SINGLY_CONNECTED_LIST
#define SINGLY_CONNECTED_LIST

#define ListOk 0
#define ListNotMem 1
#define ListUnder 2
#define ListEnd 3

#ifndef CUSTOM_BASE_TYPE
typedef int BaseType;
#endif

typedef struct element_ {
    BaseType data;
    struct element_ * next;
} element;
typedef element* ptrel;

typedef struct {
    ptrel Start;
    ptrel ptr;
    unsigned int N;
} List;

extern int ListError;

void InitList(List *L);
void PutList(List *L, BaseType E);
void GetList(List *L, BaseType *E);
void ReadList(List *L,BaseType *E);
int FullList(List *L);
int EndList(List *L);
unsigned int Count(List *L);
void BeginPtr(List *L);
void EndPtr(List *L);
void MovePtr(List *L);
void MoveTo(List *L, unsigned int n);
void DoneList(List *L);
void CopyList(List *L1,List *L2);

#endif
\end{minted}
task2.c (реализации функций)
\begin{minted}{C}
#include <lab5/singlyconnectedlist.h>

#include <stddef.h>
#include <malloc.h>

int ListError = ListOk;

void InitList(List *L) {
    ptrel newElement = malloc(sizeof(element));
    if (newElement == NULL) {
        ListError = ListNotMem;
        return;
    }
    newElement->next = NULL;

    L->Start = newElement;
    L->ptr = newElement;
    L->N = 0;
}

void PutList(List *L, BaseType E) {
    ptrel newElement = malloc(sizeof(element));
    if (newElement == NULL) {
        ListError = ListNotMem;
        return;
    }

    newElement->data = E;
    newElement->next = NULL;

    ptrel currentElement = L->ptr;
    ptrel nextElement = currentElement->next;
    currentElement->next = newElement;
    newElement->next = nextElement;
    L->N++;
    ListError = ListOk;
}

void GetList(List *L, BaseType *E) {
    if (Count(L) == 0) {
        ListError = ListUnder;
        return;
    }

    if (EndList(L)) {
        ListError = ListEnd;
        return;
    }

    ptrel currentElement = L->ptr;
    *E = currentElement->next->data;
    L->N--;
    ptrel nextNextElement = currentElement->next->next;
    free(currentElement->next);
    currentElement->next = nextNextElement;
    ListError = ListOk;
}

void ReadList(List *L,BaseType *E) {
    if (Count(L) == 0) {
        ListError = ListUnder;
        return;
    }

    if (EndList(L)) {
        ListError = ListEnd;
        return;
    }

    ptrel currentElement = L->ptr;
    *E = currentElement->next->data;
    ListError = ListOk;
}

// Зачем эта функция???
int FullList(List *L) {
    ListError = ListOk;
    return 0;
}

int EndList(List *L) {
    ListError = ListOk;
    return L->ptr->next == NULL;
}

unsigned int Count(List *L) {
    ListError = ListOk;
    return L->N;
}

void BeginPtr(List *L) {
    ListError = ListOk;
    L->ptr = L->Start;
}

void EndPtr(List *L) {
    L->ptr = L->Start;

    while (L->ptr->next != NULL) {
        L->ptr = L->ptr->next;
    }
    ListError = ListOk;
}

void MovePtr(List *L) {
    if (EndList(L)) {
        ListError = ListEnd;
        return;
    }

    L->ptr = L->ptr->next;
    ListError = ListOk;
}

void MoveTo(List *L, unsigned int n) {
    BeginPtr(L);

    for (int i = 0; i < n + 1; i++) {
        if (L->ptr->next == NULL) {
            ListError = ListEnd;
            return;
        }

        MovePtr(L);
    }
    ListError = ListOk;
}

static void freeElement(ptrel element) {
    if (element == NULL)
        return;
    
    freeElement(element->next);
    free(element);
}

void DoneList(List *L) {
    EndPtr(L);
    if (ListError != ListOk)
        return;
    
    freeElement(L->Start);
    L->Start = NULL;
    L->ptr = NULL;
    L->N = 0;
    ListError = ListOk;
}

void CopyList(List *L1,List *L2) {
    do 
    {
        PutList(L2, L1->ptr->data);
        if (EndList(L1)) break;

        MovePtr(L1);
    } while (ListError == ListOk);    
}
\end{minted}
\item Разработать программу для решения задачи в соответствии с вариантом 
индивидуального задания  с использованием модуля, полученного 
в результате выполнения пункта 2 задания.\\
main.c (основная программа)
\begin{minted}{C}
#include <algc.h>

#include <stdio.h>

int main() {
    List l;
    InitList(&l);
    if (ListError != ListOk) {
        printf("An error occured trying to init a list, code: %d", ListError);
        return ListError;
    }

    int amount;
    printf("Enter elements amount: ");
    scanf("%d", &amount);
    printf("Enter elements: ");
    for (int i = 0; i < amount; i++) {
        int element;
        scanf("%d", &element);

        PutList(&l, element);
        if (ListError != ListOk) {
            printf("An error occured trying to insert an element in list, code: %d", ListError);
            return ListError;
        }
        
        MovePtr(&l);
        if (ListError != ListOk) {
            printf("An error occured trying to insert an element in list, code: %d", ListError);
            return ListError;
        }
    }
    
    int step;
    printf("Enter step: ");
    scanf("%d", &step);

    int previousElement;
    BeginPtr(&l);
    GetList(&l, &previousElement);
    if (ListError != ListOk) {
        printf("An error occured trying to get an element in list, code: %d", ListError);
        return ListError;
    }

    for (int i = 1; i < amount; i++) {
        int currentElement;
        GetList(&l, &currentElement);
        if (ListError != ListOk) {
            printf("An error occured trying to get an element in list, code: %d", ListError);
            return ListError;
        }

        if (currentElement - previousElement != step) {
            printf("A condition is breached with elements %d and %d", previousElement, currentElement);
            return 0;
        }

        previousElement = currentElement;
    }

    
    printf("A condition is fulfilled");

    return 0;
}
\end{minted}
\end{enumerate}
\textbf{Вывод: } в ходе лабораторной работы изучили СД типа «линейный список», 
научились их программно реализовывать и использовать.

\end{document}
