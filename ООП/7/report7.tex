\documentclass[a4paper,14pt]{extarticle}


\usepackage[english,russian]{babel}
\usepackage[T2A]{fontenc}
\usepackage[utf8]{inputenc}
\usepackage{ragged2e}
\usepackage[utf8]{inputenc}
\usepackage{hyperref}
\usepackage{minted}
\setmintedinline{frame=lines, framesep=2mm, baselinestretch=1.5, bgcolor=LightGray, breaklines,fontsize=\scriptsize}
\setminted{frame=lines, framesep=2mm, baselinestretch=1.5, bgcolor=LightGray, breaklines,fontsize=\scriptsize}
\usepackage{xcolor}
\definecolor{LightGray}{gray}{0.9}
\usepackage{graphicx}
\usepackage[export]{adjustbox}
\usepackage[left=1cm,right=1cm, top=1cm,bottom=1cm,bindingoffset=0cm]{geometry}
\usepackage{fontspec}
\usepackage{ upgreek }
\usepackage[shortlabels]{enumitem}
\usepackage{adjustbox}
\usepackage{multirow}
\usepackage{amsmath}
\usepackage{amssymb}
\usepackage{pifont}
\usepackage{pgfplots}
\usepackage{longtable}
\usepackage{array}
\graphicspath{ {./images/} }
\makeatletter
\AddEnumerateCounter{\asbuk}{\russian@alph}{щ}
\makeatother
\setmonofont{Consolas}
\setmainfont{Times New Roman}

\newcommand\textbox[1]{
	\parbox{.45\textwidth}{#1}
} 

\newcommand{\specialcell}[2][c]{%
	\begin{tabular}[#1]{@{}c@{}}#2\end{tabular}}

\begin{document}
\pagenumbering{gobble}
\begin{center}
    \small{
        \textbf{МИНИCТЕРCТВО НАУКИ И ВЫCШЕГО ОБРАЗОВАНИЯ РОCCИЙCКОЙ ФЕДЕРАЦИИ}\\
        ФЕДЕРАЛЬНОЕ ГОCУДАРCТВЕННОЕ БЮДЖЕТНОЕ ОБРАЗОВАТЕЛЬНОЕ УЧРЕЖДЕНИЕ\\ВЫCШЕГО ОБРАЗОВАНИЯ \\
        \textbf{«БЕЛГОРОДCКИЙ ГОCУДАРCТВЕННЫЙ ТЕХНОЛОГИЧЕCКИЙ\\УНИВЕРCИТЕТ им. В. Г. ШУХОВА»\\ (БГТУ им. В.Г. Шухова)} \\
        \bigbreak
        \includegraphics[width=70mm]{log}\\
        ИНСТИТУТ ИНФОРМАЦИОННЫХ ТЕХНОЛОГИЙ И УПРАВЛЯЮЩИХ СИСТЕМ\\}
\end{center}

\vfill
\begin{center}
    \large{
        \textbf{
            Лабораторная работа №7}}\\
    \normalsize{
        по дисциплине: ООП \\
        тема: «Исключительные ситуации в С++»}
\end{center}
\vfill
\hfill\textbox{
    Выполнил: ст. группы ПВ-223\\Пахомов Владислав Андреевич
    \bigbreak
    Проверили: \\пр. Черников Сергей Викторович
}
\vfill\begin{center}
    Белгород 2024 г.
\end{center}
\newpage
\begin{center}
    \textbf{Лабораторная работа №7}\\
    «Исключительные ситуации в С++»\\
    Вариант 10
\end{center}
\textbf{Цель работы: }получение теоретических знаний об исключительных ситуациях
в С++. Получение практических навыков при работе с исключениями в С++.

Разработать программу в соответствии с заданным вариантом задания.\bigbreak
Разработать класс “Калькулятор”. Предусмотреть исключительные ситуации,
деления на ноль, переполнение и др.\\
\textit{main.cpp}
\begin{minted}{C++}
#include "../libs/alg/calculator/calculator.h"

#include <iostream>

int main() {
    std::cout << (Calculator::get() << (Calculator::get() << 4 << Calculator::CalculatorOperation::DIVIDE << 2 << Calculator::CalculatorOperationEnd::END)
                                    << Calculator::CalculatorOperation::PLUS << 7 << Calculator::CalculatorOperationEnd::END) << std::endl;
    std::cout << (Calculator::get() << 3 << Calculator::CalculatorOperation::MULTIPLY << 1.3 << Calculator::CalculatorOperationEnd::END) << std::endl;
    std::cout << (Calculator::get() << 1 << Calculator::CalculatorOperation::PLUS << 0.9 << Calculator::CalculatorOperationEnd::END) << std::endl;
    std::cout << (Calculator::get() << 4 << Calculator::CalculatorOperation::MINUS << 2 << Calculator::CalculatorOperationEnd::END) << std::endl;
}
\end{minted}

\textit{calculator.h}
\begin{minted}{C++}
#ifndef CALCULATOR_H
#define CALCULATOR_H

#include <string>

class Calculator {
public:
    enum CalculatorOperation {
        PLUS,
        MINUS,
        DIVIDE,
        MULTIPLY
    };

    enum CalculatorOperationEnd {
        END
    };

    bool op_left_assigned = false;
    double op_left;
    bool op_assigned = false;
    CalculatorOperation op;
    bool op_right_assigned = false;
    double op_right;

    static Calculator get();
    Calculator& operator<<(double val);
    Calculator& operator<<(CalculatorOperation op);
    double operator<<(CalculatorOperationEnd op);
private:
    Calculator() {

    };
};

#endif // CALCULATOR_H    
\end{minted}

\textit{calculator.cpp}
\begin{minted}{C++}
#include <stdexcept>
#include <limits>

#include "calculator.h"

Calculator Calculator::get() {
    return Calculator();
}

Calculator& Calculator::operator<<(double val) {
    if (!this->op_left_assigned && !this->op_right_assigned && !this->op_assigned) {
        this->op_left = val;
        this->op_left_assigned = true;
    } else if (!this->op_right_assigned && this->op_left_assigned && this->op_assigned) {
        this->op_right = val;
        this->op_right_assigned = true;
    } else {
        throw std::invalid_argument("Invalid operator");
    }
}

Calculator& Calculator::operator<<(CalculatorOperation op) {
    if (this->op_left_assigned && !this->op_assigned && !this->op_right_assigned) {
        this->op = op;
        this->op_assigned = true;
    } else {
        throw std::invalid_argument("Invalid operator");
    }
}

double Calculator::operator<<(CalculatorOperationEnd op) {
    if (op == CalculatorOperationEnd::END) {
        if (!(this->op_assigned && this->op_left_assigned && this->op_right_assigned))
            throw std::invalid_argument("Expression incomplete");

        double res;

        switch (this->op) {
        case CalculatorOperation::PLUS:
            res = this->op_left + this->op_right;
            break;
        case CalculatorOperation::MINUS:
            res = this->op_left - this->op_right;
            break;
        case CalculatorOperation::MULTIPLY:
            res = this->op_left * this->op_right;
            break;
        case CalculatorOperation::DIVIDE:
            if (this->op_right == 0) throw std::domain_error("Can't divide by zero");

            res = this->op_left / this->op_right;
            break;
        default:
            throw std::invalid_argument("Unsupported operation");
        }

        if (res == std::numeric_limits<double>::infinity() || res == -std::numeric_limits<double>::infinity())
            throw std::overflow_error("Number is too big");

        return res;
    } else {
        throw std::invalid_argument("Unsupported operation");
    }
}

\end{minted}

\href{https://github.com/IAmProgrammist/oop/tree/master}{Ссылка на репозиторий}

\textbf{Вывод: } в ходе лабораторной работы получили 
теоретические знания об исключительных ситуациях
в С++. Получение практических навыков при работе 
с исключениями в С++.

\end{document}