\documentclass[a4paper,14pt]{extarticle}


\usepackage[english,russian]{babel}
\usepackage[T2A]{fontenc}
\usepackage[utf8]{inputenc}
\usepackage{ragged2e}
\usepackage[utf8]{inputenc}
\usepackage{hyperref}
\usepackage{minted}
\setmintedinline{frame=lines, framesep=2mm, baselinestretch=1.5, bgcolor=LightGray, breaklines,fontsize=\scriptsize}
\setminted{frame=lines, framesep=2mm, baselinestretch=1.5, bgcolor=LightGray, breaklines,fontsize=\scriptsize}
\usepackage{xcolor}
\definecolor{LightGray}{gray}{0.9}
\usepackage{graphicx}
\usepackage[export]{adjustbox}
\usepackage[left=1cm,right=1cm, top=1cm,bottom=1cm,bindingoffset=0cm]{geometry}
\usepackage{fontspec}
\usepackage{ upgreek }
\usepackage[shortlabels]{enumitem}
\usepackage{adjustbox}
\usepackage{multirow}
\usepackage{amsmath}
\usepackage{amssymb}
\usepackage{pifont}
\usepackage{pgfplots}
\usepackage{longtable}
\usepackage{array}
\graphicspath{ {./images/} }
\makeatletter
\AddEnumerateCounter{\asbuk}{\russian@alph}{щ}
\makeatother
\setmonofont{Consolas}
\setmainfont{Times New Roman}

\newcommand\textbox[1]{
	\parbox{.45\textwidth}{#1}
} 

\newcommand{\specialcell}[2][c]{%
	\begin{tabular}[#1]{@{}c@{}}#2\end{tabular}}

\begin{document}
\pagenumbering{gobble}
\begin{center}
    \small{
        \textbf{МИНИCТЕРCТВО НАУКИ И ВЫCШЕГО ОБРАЗОВАНИЯ РОCCИЙCКОЙ ФЕДЕРАЦИИ}\\
        ФЕДЕРАЛЬНОЕ ГОCУДАРCТВЕННОЕ БЮДЖЕТНОЕ ОБРАЗОВАТЕЛЬНОЕ УЧРЕЖДЕНИЕ\\ВЫCШЕГО ОБРАЗОВАНИЯ \\
        \textbf{«БЕЛГОРОДCКИЙ ГОCУДАРCТВЕННЫЙ ТЕХНОЛОГИЧЕCКИЙ\\УНИВЕРCИТЕТ им. В. Г. ШУХОВА»\\ (БГТУ им. В.Г. Шухова)} \\
        \bigbreak
        \includegraphics[width=70mm]{log}\\
        ИНСТИТУТ ИНФОРМАЦИОННЫХ ТЕХНОЛОГИЙ И УПРАВЛЯЮЩИХ СИСТЕМ\\}
\end{center}

\vfill
\begin{center}
    \large{
        \textbf{
            Лабораторная работа №11}}\\
    \normalsize{
        по дисциплине: ООП \\
        тема: «Знакомство с языком программирования Python. Базовые структуры данных.»}
\end{center}
\vfill
\hfill\textbox{
    Выполнил: ст. группы ПВ-223\\Пахомов Владислав Андреевич
    \bigbreak
    Проверили: \\пр. Черников Сергей Викторович
}
\vfill\begin{center}
    Белгород 2024 г.
\end{center}
\newpage
\begin{center}
    \textbf{Лабораторная работа №11}\\
    «Знакомство с языком программирования Python. Базовые структуры данных.»\\
    Вариант 3
\end{center}
\textbf{Цель работы: }познакомится с базовыми конструкциями языка. Получить навык создания
простых приложений. Изучить базовые типы.\\

В соответствии с вариантом задания требуется выполнить объектную декомпозицию задачи.
В качестве одного из обязательных объектов выделить «матрицу». Для реализации
соблюдения условия задачи требуется использовать возможности перегрузки операторов.
При выводе также требуется выполнить перегрузку соответствующего оператора.\\
На вход подаются данные в форме двумерных «матриц», количество матриц заранее не
определено, разделителем между матрицами являются строки. Для каждой матрицы найти
все, которые удовлетворяют следующему условию: четность/нечетность соответствующих
элементов матриц совпадает. Форма матрицы может быть не полной. Формат вывода
требуется соблюсти.\\
\begin{center}
    \includegraphics[width=140mm]{decomp.png}
\end{center}
\begin{minted}{python3}
from __future__ import print_function
from abc import ABC, abstractmethod


class Matrix:
    def __init__(self):
        self.matrix = []

    def add_line(self, line: list) -> None:
        self.matrix.append(line)

    def get_height(self):
        return len(self.matrix)

    def get_min_width(self):
        return min(map(lambda x: len(x), self.matrix))

    def get_max_width(self):
        return max(map(lambda x: len(x), self.matrix))


class GroupMatricesPredicate(ABC):
    @abstractmethod
    def hash(self, matrix):
        pass


class GroupByEvenOddFactor(GroupMatricesPredicate):
    def hash(self, matrix):
        result = []

        for line in matrix.matrix:
            result.append(tuple(map(lambda x: x % 2 == 0, line)))

        return tuple(result)


class GroupMatrices:
    def __init__(self):
        self.matrices = []

    def add_matrix(self, matrix: Matrix):
        self.matrices.append(matrix)

    def group_by_predicate(self, predicate: GroupMatricesPredicate):
        result: dict = {}

        for matrix in self.matrices:
            hash = predicate.hash(matrix)

            if hash in result.keys():
                result[hash].append(matrix)
            else:
                result[hash] = [matrix]

        return result


def main():
    group_matrices = GroupMatrices()

    with open(input("Введите путь к файлу: "), "r") as f:
        tmp_matrix = Matrix()

        for line in f:
            if not line.strip():
                group_matrices.add_matrix(tmp_matrix)
                tmp_matrix = Matrix()
            else:
                tmp_matrix.add_line(list(map(lambda x: int(x), line.strip().split(" "))))

        if tmp_matrix.get_height() != 0:
            group_matrices.add_matrix(tmp_matrix)

    for group, matrices in group_matrices.group_by_predicate(GroupByEvenOddFactor()).items():
        print("Группа:")
        print(*group, "", sep="\n")

        print("Матрицы:")
        for matrix in matrices:
            print(*matrix.matrix, "", sep="\n")

        print("")


if __name__ == "__main__":
    main()
\end{minted}

\href{https://gitlab.com/vlad4052/2024_pv223_vladislav_10}{Ссылка на репозиторий}

\textbf{Вывод: } в ходе лабораторной работы познакомились с базовыми конструкциями языка. Получили навык создания
простых приложений. Изучили базовые типы.

\end{document}