\documentclass[a4paper,14pt]{extarticle}


\usepackage[english,russian]{babel}
\usepackage[T2A]{fontenc}
\usepackage[utf8]{inputenc}
\usepackage{ragged2e}
\usepackage[utf8]{inputenc}
\usepackage{hyperref}
\usepackage{minted}
\setmintedinline{frame=lines, framesep=2mm, baselinestretch=1.5, bgcolor=LightGray, breaklines,fontsize=\scriptsize}
\setminted{frame=lines, framesep=2mm, baselinestretch=1.5, bgcolor=LightGray, breaklines,fontsize=\scriptsize}
\usepackage{xcolor}
\definecolor{LightGray}{gray}{0.9}
\usepackage{graphicx}
\usepackage[export]{adjustbox}
\usepackage[left=1cm,right=1cm, top=1cm,bottom=1cm,bindingoffset=0cm]{geometry}
\usepackage{fontspec}
\usepackage{ upgreek }
\usepackage[shortlabels]{enumitem}
\usepackage{adjustbox}
\usepackage{multirow}
\usepackage{amsmath}
\usepackage{amssymb}
\usepackage{pifont}
\usepackage{pgfplots}
\usepackage{longtable}
\usepackage{array}
\graphicspath{ {./images/} }
\makeatletter
\AddEnumerateCounter{\asbuk}{\russian@alph}{щ}
\makeatother
\setmonofont{Consolas}
\setmainfont{Times New Roman}

\newcommand\textbox[1]{
	\parbox{.45\textwidth}{#1}
} 

\newcommand{\specialcell}[2][c]{%
	\begin{tabular}[#1]{@{}c@{}}#2\end{tabular}}

\begin{document}
\pagenumbering{gobble}
\begin{center}
    \small{
        \textbf{МИНИCТЕРCТВО НАУКИ И ВЫCШЕГО ОБРАЗОВАНИЯ РОCCИЙCКОЙ ФЕДЕРАЦИИ}\\
        ФЕДЕРАЛЬНОЕ ГОCУДАРCТВЕННОЕ БЮДЖЕТНОЕ ОБРАЗОВАТЕЛЬНОЕ УЧРЕЖДЕНИЕ\\ВЫCШЕГО ОБРАЗОВАНИЯ \\
        \textbf{«БЕЛГОРОДCКИЙ ГОCУДАРCТВЕННЫЙ ТЕХНОЛОГИЧЕCКИЙ\\УНИВЕРCИТЕТ им. В. Г. ШУХОВА»\\ (БГТУ им. В.Г. Шухова)} \\
        \bigbreak
        \includegraphics[width=70mm]{log}\\
        ИНСТИТУТ ИНФОРМАЦИОННЫХ ТЕХНОЛОГИЙ И УПРАВЛЯЮЩИХ СИСТЕМ\\}
\end{center}

\vfill
\begin{center}
    \large{
        \textbf{
            Лабораторная работа №6}}\\
    \normalsize{
        по дисциплине: ООП \\
        тема: «Потоки в С++»}
\end{center}
\vfill
\hfill\textbox{
    Выполнил: ст. группы ПВ-223\\Пахомов Владислав Андреевич
    \bigbreak
    Проверили: \\пр. Черников Сергей Викторович
}
\vfill\begin{center}
    Белгород 2024 г.
\end{center}
\newpage
\begin{center}
    \textbf{Лабораторная работа №6}\\
    «Потоки в С++»\\
    Вариант 3
\end{center}
\textbf{Цель работы: }изучение основных возможностей потоков
управления и потоков ввода-вывода. Получение навыков работы со
стандартными средствами управления потоками в С++11. Знакомство с
классом Thread и стандартными средствами синхронизации потоков.

\textbf{Задание 1}\\
Разработать программу в соответствии с вариантом задания (номер
варианта + 3), используя API CreateThread.\\

Один поток удаляет пробелы в строке и вставляет их в случайное место,
а другой поток выполняет циклический сдвиг текста. Произвести
синхронный вывод при каждой итерации. Показать выполнение работы
программы в синхронном и асинхронном режимах.

Асинхронная программа:
\begin{minted}{C++}
#include <iostream>
#include <mutex>
#include <thread>
#include <vector>
#include <algorithm>
#include <windows.h>

void space_adder(std::vector<char>& val, std::mutex& m) {
    while (1) {
        m.lock();

        auto spaceFind = std::find(val.begin(), val.end(), ' ');
        while (spaceFind != val.end()) {
            val.erase(spaceFind);

            spaceFind = std::find(val.begin(), val.end(), ' ');
        }

        int newSpaces = val.size();
        for (int i = 0; i < newSpaces; i++) {
            int randIndex = val.size() * 1.0 * rand() / RAND_MAX;

            val.insert(val.begin() + randIndex, ' ');
        }

        m.unlock();

        std::this_thread::sleep_for(std::chrono::milliseconds(500));
    }
}

void move_val(std::vector<char>& val, std::mutex& m) {
    while (1) {
        m.lock();

        char b = val[0];
        val.erase(val.begin());
        val.push_back(b);

        m.unlock();

        std::this_thread::sleep_for(std::chrono::milliseconds(500));
    }
}

void outputter(std::vector<char>& val, std::mutex& m) {
    while (1) {
        m.lock();

        for (auto& ch : val) {
            std::cout << ch;
        }
        std::cout << "\n";

        m.unlock();

        std::this_thread::sleep_for(std::chrono::milliseconds(500));
    }
}

int main() {
    SetConsoleOutputCP(CP_UTF8);

    std::mutex m;
    std::vector<char> val = {'H', 'e', 'l', 'l', 'o'};
    std::thread th1(space_adder, std::ref(val), std::ref(m));
    std::thread th2(move_val, std::ref(val), std::ref(m));
    std::thread th3(outputter, std::ref(val), std::ref(m));

    th1.detach();
    th2.detach();
    th3.detach();

    int a;
    std::cin >> a;

    return 0;
}
\end{minted}

Вывод в консоль: 
\begin{minted}{C++}
  He  l lo
He  ll   o
e  ll   oH
     lloHe
    lloHe
  ll oH e
 l    oHel
lo H    el
oHel     l
   H el lo
Hel l    o
 e   ll oH
   e l loH
e  l l o H
l lo H   e
 lo  H e l
lo    H el
  o  He ll
o H  el  l
 He  l  lo
\end{minted}


Синхронная программа:
\begin{minted}{C++}
#include <iostream>
#include <mutex>
#include <thread>
#include <vector>
#include <algorithm>
#include <windows.h>

int main() {
    SetConsoleOutputCP(CP_UTF8);

    std::vector<char> val = {'H', 'e', 'l', 'l', 'o'};

    while (1) {
        auto spaceFind = std::find(val.begin(), val.end(), ' ');
        while (spaceFind != val.end()) {
            val.erase(spaceFind);

            spaceFind = std::find(val.begin(), val.end(), ' ');
        }

        int newSpaces = val.size();
        for (int i = 0; i < newSpaces; i++) {
            int randIndex = val.size() * 1.0 * rand() / RAND_MAX;

            val.insert(val.begin() + randIndex, ' ');
        }

        char b = val[0];
        val.erase(val.begin());
        val.push_back(b);

        for (auto& ch : val) {
            std::cout << ch;
        }
        std::cout << "\n";

        std::this_thread::sleep_for(std::chrono::milliseconds(500));
    }

    return 0;
}
\end{minted}

Вывод в консоль: 
\begin{minted}{C++}
 He  l lo
e  ll   oH
e ll  o H
    elloH
  el lo H
 l    loHe
 l l  oHe
l o    Hel
oHe     ll
  o He ll
He l    lo
H   el lo
  H e llo
  e l l oH
 ll o   He
ll  o H e
l    o Hel
 l  oH el
 o  He  ll
oH  e  ll
\end{minted}

\href{https://github.com/IAmProgrammist/oop/tree/master}{Ссылка на репозиторий}

\textbf{Вывод: } в ходе лабораторной работы изучены основные возможности потоков
управления и потоков ввода-вывода. Получены навыки работы со
стандартными средствами управления потоками в С++11. Познакомились с
классом Thread и стандартными средствами синхронизации потоков.

\end{document}