\documentclass[a4paper,14pt]{extarticle}


\usepackage[english,russian]{babel}
\usepackage[T2A]{fontenc}
\usepackage[utf8]{inputenc}
\usepackage{ragged2e}
\usepackage[utf8]{inputenc}
\usepackage{hyperref}
\usepackage{minted}
\setmintedinline{frame=lines, framesep=2mm, baselinestretch=1.5, bgcolor=LightGray, breaklines,fontsize=\scriptsize}
\setminted{frame=lines, framesep=2mm, baselinestretch=1.5, bgcolor=LightGray, breaklines,fontsize=\scriptsize}
\usepackage{xcolor}
\definecolor{LightGray}{gray}{0.9}
\usepackage{graphicx}
\usepackage[export]{adjustbox}
\usepackage[left=1cm,right=1cm, top=1cm,bottom=1cm,bindingoffset=0cm]{geometry}
\usepackage{fontspec}
\usepackage{ upgreek }
\usepackage[shortlabels]{enumitem}
\usepackage{adjustbox}
\usepackage{multirow}
\usepackage{amsmath}
\usepackage{amssymb}
\usepackage{pifont}
\usepackage{pgfplots}
\usepackage{longtable}
\usepackage{array}
\graphicspath{ {./images/} }
\makeatletter
\AddEnumerateCounter{\asbuk}{\russian@alph}{щ}
\makeatother
\setmonofont{Consolas}
\setmainfont{Times New Roman}

\newcommand\textbox[1]{
	\parbox{.45\textwidth}{#1}
} 

\newcommand{\specialcell}[2][c]{%
	\begin{tabular}[#1]{@{}c@{}}#2\end{tabular}}

\begin{document}
\pagenumbering{gobble}
\begin{center}
    \small{
        \textbf{МИНИCТЕРCТВО НАУКИ И ВЫCШЕГО ОБРАЗОВАНИЯ РОCCИЙCКОЙ ФЕДЕРАЦИИ}\\
        ФЕДЕРАЛЬНОЕ ГОCУДАРCТВЕННОЕ БЮДЖЕТНОЕ ОБРАЗОВАТЕЛЬНОЕ УЧРЕЖДЕНИЕ\\ВЫCШЕГО ОБРАЗОВАНИЯ \\
        \textbf{«БЕЛГОРОДCКИЙ ГОCУДАРCТВЕННЫЙ ТЕХНОЛОГИЧЕCКИЙ\\УНИВЕРCИТЕТ им. В. Г. ШУХОВА»\\ (БГТУ им. В.Г. Шухова)} \\
        \bigbreak
        \includegraphics[width=70mm]{log}\\
        ИНСТИТУТ ИНФОРМАЦИОННЫХ ТЕХНОЛОГИЙ И УПРАВЛЯЮЩИХ СИСТЕМ\\}
\end{center}

\vfill
\begin{center}
    \large{
        \textbf{
            Лабораторная работа №13}}\\
    \normalsize{
        по дисциплине: ООП \\
        тема: «Знакомство с библиотеками языка Python. PyQT.»}
\end{center}
\vfill
\hfill\textbox{
    Выполнил: ст. группы ПВ-223\\Пахомов Владислав Андреевич
    \bigbreak
    Проверили: \\пр. Черников Сергей Викторович
}
\vfill\begin{center}
    Белгород 2024 г.
\end{center}
\newpage
\begin{center}
    \textbf{Лабораторная работа №13}\\
    «Знакомство с библиотеками языка Python. PyQT.»\\
    Вариант 10
\end{center}
\textbf{Цель работы: }приобретение практических навыков создания приложений на
языке Python, QT приложения.\\

QT-Фитнес инструктор (составление тренировок)\\
\begin{center}
    \includegraphics[width=140mm]{init_train.png}
    \includegraphics[width=140mm]{train.png}
\end{center}
\begin{minted}{python3}
import sys

from PySide6.QtWidgets import QApplication

from widgets.init_train_dialog import InitTrainDialog

if __name__ == "__main__":
    app = QApplication(sys.argv)
    dialog = InitTrainDialog()
    dialog.show()
    sys.exit(app.exec())
\end{minted}
\begin{minted}{python3}
class TrainStep:
    def __init__(self, desc: str, time: int):
        self.desc = desc
        self.time = time
\end{minted}

\begin{minted}{python3}
import threading
def debounce(wait_time):
    def decorator(function):
        def debounced(*args, **kwargs):
            def call_function():
                debounced._timer = None
                return function(*args, **kwargs)
            # if we already have a call to the function currently waiting to be executed, reset the timer
            if debounced._timer is not None:
                debounced._timer.cancel()

            # after wait_time, call the function provided to the decorator with its arguments
            debounced._timer = threading.Timer(wait_time, call_function)
            debounced._timer.start()

        debounced._timer = None
        return debounced

    return decorator
\end{minted}

\begin{minted}{python3}
from PySide6.QtWidgets import QDialog, QFileDialog

from src.lab13.models.train_step import TrainStep
from src.lab13.utils.debounce import debounce
from src.lab13.widgets import ui_init_train, train_dialog


class InitTrainDialog(QDialog):
    MESSAGE_NORMAL = "Введите тренировку"
    MESSAGE_INCORRECT = "Формат тренировки неправильный"
    MESSAGE_SAVED = "Тренировка сохранена"
    MESSAGE_YOURE_TYPING = "Вы печатаете..."

    def __init__(self, parent=None):
        super().__init__(parent)
        self.ui = ui_init_train.Ui_Dialog()
        self.ui.setupUi(self)

        self.ui.savetrain.clicked.connect(self.savetrain)
        self.ui.loadtrain.clicked.connect(self.loadtrain)
        self.ui.begintrain.clicked.connect(self.begintrain)
        self.ui.train.textChanged.connect(self.train_text_edit)
        self.ui.message.setText(InitTrainDialog.MESSAGE_NORMAL)

    def train_text_edit(self):
        self.ui.message.setText(InitTrainDialog.MESSAGE_YOURE_TYPING)
        self.train_text_edit_check_syntax()

    @debounce(1)
    def train_text_edit_check_syntax(self):
        if self.check_syntax():
            self.ui.message.setText(InitTrainDialog.MESSAGE_NORMAL)
        else:
            self.ui.message.setText(InitTrainDialog.MESSAGE_INCORRECT)

    def check_syntax(self):
        try:
            train_steps = [TrainStep(" ".join(line.split()[:-1]), int(line.split()[-1])) for line in self.ui.train.toPlainText().strip().splitlines()]

            return train_steps
        except Exception as e:
            return None

    def savetrain(self):
        if not self.check_syntax():
            self.ui.message.setText(InitTrainDialog.MESSAGE_INCORRECT)
            return

        filename = QFileDialog.getSaveFileName(self,
                "Сохранить тренировку", "", "Файлы тренировки (*.train)")

        if len(filename[0]) == 0:
            return

        with open(filename[0], 'w', encoding='utf-8') as f:
            f.write(self.ui.train.toPlainText().strip())

        self.ui.message.setText(InitTrainDialog.MESSAGE_SAVED)

    def loadtrain(self):
        filename = QFileDialog.getOpenFileName(self, "Загрузить тренировку", "", "Файлы тренировки (*.train)")

        if len(filename[0]) == 0:
            return

        with open(filename[0], 'r', encoding='utf-8') as f:
            self.ui.train.setPlainText("".join(list(f.readlines())))

        self.ui.message.setText(InitTrainDialog.MESSAGE_NORMAL)

    def begintrain(self):
        train = self.check_syntax()
        if not train:
            self.ui.message.setText(InitTrainDialog.MESSAGE_INCORRECT)
            return

        self.hide()

        train = train_dialog.TrainDialog(train)
        train.setModal(True)
        train.exec()

        self.show()
\end{minted}

\begin{minted}{python3}
from PySide6.QtCore import QTimer
from PySide6.QtWidgets import QDialog

from src.lab13.models.train_step import TrainStep
from src.lab13.widgets import ui_train

import datetime

class TrainDialog(QDialog):
    MESSAGE_READY = 'Приготовиться!'
    MESSAGE_END = 'Тренировка окончена!'
    MESSAGE_END_SEC = 'Вы - молодец!'

    def __init__(self, train, parent=None):
        super().__init__(parent)
        self.ui = ui_train.Ui_Dialog()
        self.ui.setupUi(self)

        self.train = [TrainStep(TrainDialog.MESSAGE_READY, 5)]
        self.train.extend(train)
        self.ui.canceltrain.clicked.connect(self.cancel_train)
        self.start = datetime.datetime.now()
        self.timer = QTimer(self)
        self.timer.timeout.connect(self.tick)
        self.timer.setInterval(100)
        self.timer.start()

    def __del__(self):
        self.timer.stop()

    def cancel_train(self):
        self.timer.stop()
        self.accept()

    def tick(self):
        nowtime = datetime.datetime.now()
        sum_time = 0
        fitting = None

        for item in self.train:
            if datetime.timedelta(seconds=item.time + sum_time) + self.start > nowtime:
                fitting = item
                break
            sum_time += item.time

        if fitting:
            seconds_left = 1 + (datetime.timedelta(seconds=fitting.time + sum_time) + self.start - nowtime).seconds
            self.ui.trainname.setText(fitting.desc)
            self.ui.traintime.setText(f'{seconds_left} сек.')
        else:
            self.ui.trainname.setText(TrainDialog.MESSAGE_END)
            self.ui.traintime.setText(TrainDialog.MESSAGE_END_SEC)
            self.timer.stop()

\end{minted}

\begin{minted}{python3}
# -*- coding: utf-8 -*-

################################################################################
## Form generated from reading UI file 'init_train.ui'
##
## Created by: Qt User Interface Compiler version 6.7.0
##
## WARNING! All changes made in this file will be lost when recompiling UI file!
################################################################################

from PySide6.QtCore import (QCoreApplication, QDate, QDateTime, QLocale,
    QMetaObject, QObject, QPoint, QRect,
    QSize, QTime, QUrl, Qt)
from PySide6.QtGui import (QBrush, QColor, QConicalGradient, QCursor,
    QFont, QFontDatabase, QGradient, QIcon,
    QImage, QKeySequence, QLinearGradient, QPainter,
    QPalette, QPixmap, QRadialGradient, QTransform)
from PySide6.QtWidgets import (QApplication, QDialog, QHBoxLayout, QLabel,
    QPlainTextEdit, QPushButton, QSizePolicy, QVBoxLayout,
    QWidget)

class Ui_Dialog(object):
    def setupUi(self, Dialog):
        if not Dialog.objectName():
            Dialog.setObjectName(u"Dialog")
        Dialog.resize(480, 640)
        self.verticalLayoutWidget = QWidget(Dialog)
        self.verticalLayoutWidget.setObjectName(u"verticalLayoutWidget")
        self.verticalLayoutWidget.setGeometry(QRect(20, 20, 440, 600))
        self.verticalLayout = QVBoxLayout(self.verticalLayoutWidget)
        self.verticalLayout.setObjectName(u"verticalLayout")
        self.verticalLayout.setContentsMargins(0, 0, 0, 0)
        self.message = QLabel(self.verticalLayoutWidget)
        self.message.setObjectName(u"message")

        self.verticalLayout.addWidget(self.message)

        self.train = QPlainTextEdit(self.verticalLayoutWidget)
        self.train.setObjectName(u"train")

        self.verticalLayout.addWidget(self.train)

        self.begintrain = QPushButton(self.verticalLayoutWidget)
        self.begintrain.setObjectName(u"begintrain")

        self.verticalLayout.addWidget(self.begintrain)

        self.horizontalLayout = QHBoxLayout()
        self.horizontalLayout.setObjectName(u"horizontalLayout")
        self.savetrain = QPushButton(self.verticalLayoutWidget)
        self.savetrain.setObjectName(u"savetrain")

        self.horizontalLayout.addWidget(self.savetrain)

        self.loadtrain = QPushButton(self.verticalLayoutWidget)
        self.loadtrain.setObjectName(u"loadtrain")

        self.horizontalLayout.addWidget(self.loadtrain)


        self.verticalLayout.addLayout(self.horizontalLayout)


        self.retranslateUi(Dialog)

        QMetaObject.connectSlotsByName(Dialog)
    # setupUi

    def retranslateUi(self, Dialog):
        Dialog.setWindowTitle(QCoreApplication.translate("Dialog", u"Dialog", None))
        self.message.setText(QCoreApplication.translate("Dialog", u"TextLabel", None))
        self.begintrain.setText(QCoreApplication.translate("Dialog", u"\u041d\u0430\u0447\u0430\u0442\u044c \u0442\u0440\u0435\u043d\u0438\u0440\u043e\u0432\u043a\u0443", None))
        self.savetrain.setText(QCoreApplication.translate("Dialog", u"\u0421\u043e\u0445\u0440\u0430\u043d\u0438\u0442\u044c \u0442\u0440\u0435\u043d\u0438\u0440\u043e\u0432\u043a\u0443", None))
        self.loadtrain.setText(QCoreApplication.translate("Dialog", u"\u0417\u0430\u0433\u0440\u0443\u0437\u0438\u0442\u044c \u0442\u0440\u0435\u043d\u0438\u0440\u043e\u0432\u043a\u0443", None))
    # retranslateUi
\end{minted}

\begin{minted}{python3}
# -*- coding: utf-8 -*-

################################################################################
## Form generated from reading UI file 'train.ui'
##
## Created by: Qt User Interface Compiler version 6.7.0
##
## WARNING! All changes made in this file will be lost when recompiling UI file!
################################################################################

from PySide6.QtCore import (QCoreApplication, QDate, QDateTime, QLocale,
    QMetaObject, QObject, QPoint, QRect,
    QSize, QTime, QUrl, Qt)
from PySide6.QtGui import (QBrush, QColor, QConicalGradient, QCursor,
    QFont, QFontDatabase, QGradient, QIcon,
    QImage, QKeySequence, QLinearGradient, QPainter,
    QPalette, QPixmap, QRadialGradient, QTransform)
from PySide6.QtWidgets import (QApplication, QDialog, QLabel, QPushButton,
    QSizePolicy, QVBoxLayout, QWidget)

class Ui_Dialog(object):
    def setupUi(self, Dialog):
        if not Dialog.objectName():
            Dialog.setObjectName(u"Dialog")
        Dialog.resize(600, 400)
        self.verticalLayoutWidget = QWidget(Dialog)
        self.verticalLayoutWidget.setObjectName(u"verticalLayoutWidget")
        self.verticalLayoutWidget.setGeometry(QRect(20, 20, 560, 360))
        self.verticalLayout = QVBoxLayout(self.verticalLayoutWidget)
        self.verticalLayout.setObjectName(u"verticalLayout")
        self.verticalLayout.setContentsMargins(0, 0, 0, 0)
        self.trainname = QLabel(self.verticalLayoutWidget)
        self.trainname.setObjectName(u"trainname")
        self.trainname.setLayoutDirection(Qt.LayoutDirection.LeftToRight)
        self.trainname.setStyleSheet(u"font: 700 26pt \"Segoe UI\";")
        self.trainname.setTextFormat(Qt.TextFormat.AutoText)
        self.trainname.setAlignment(Qt.AlignmentFlag.AlignCenter)
        self.trainname.setWordWrap(True)

        self.verticalLayout.addWidget(self.trainname)

        self.traintime = QLabel(self.verticalLayoutWidget)
        self.traintime.setObjectName(u"traintime")
        sizePolicy = QSizePolicy(QSizePolicy.Policy.Preferred, QSizePolicy.Policy.Fixed)
        sizePolicy.setHorizontalStretch(0)
        sizePolicy.setVerticalStretch(0)
        sizePolicy.setHeightForWidth(self.traintime.sizePolicy().hasHeightForWidth())
        self.traintime.setSizePolicy(sizePolicy)
        self.traintime.setStyleSheet(u"font: 16pt \"Segoe UI\";")
        self.traintime.setAlignment(Qt.AlignmentFlag.AlignHCenter|Qt.AlignmentFlag.AlignTop)

        self.verticalLayout.addWidget(self.traintime)

        self.canceltrain = QPushButton(self.verticalLayoutWidget)
        self.canceltrain.setObjectName(u"canceltrain")
        sizePolicy1 = QSizePolicy(QSizePolicy.Policy.Fixed, QSizePolicy.Policy.Fixed)
        sizePolicy1.setHorizontalStretch(0)
        sizePolicy1.setVerticalStretch(0)
        sizePolicy1.setHeightForWidth(self.canceltrain.sizePolicy().hasHeightForWidth())
        self.canceltrain.setSizePolicy(sizePolicy1)
        self.canceltrain.setLayoutDirection(Qt.LayoutDirection.LeftToRight)
        self.canceltrain.setAutoDefault(True)
        self.canceltrain.setFlat(False)

        self.verticalLayout.addWidget(self.canceltrain, 0, Qt.AlignmentFlag.AlignRight)


        self.retranslateUi(Dialog)

        self.canceltrain.setDefault(False)


        QMetaObject.connectSlotsByName(Dialog)
    # setupUi

    def retranslateUi(self, Dialog):
        Dialog.setWindowTitle(QCoreApplication.translate("Dialog", u"Dialog", None))
        self.trainname.setText(QCoreApplication.translate("Dialog", u"\u0422\u0440\u0435\u043d\u0438\u0440\u043e\u0432\u043a\u0430 \u0442\u0440\u0435\u043d\u0438\u0440\u043e\u0432\u043a\u0430 \u0442\u0440\u0435\u043d\u0438\u0440\u043e\u0432\u043a\u0430", None))
        self.traintime.setText(QCoreApplication.translate("Dialog", u"30 \u0441\u0435\u043a.", None))
        self.canceltrain.setText(QCoreApplication.translate("Dialog", u"\u041e\u0442\u043c\u0435\u043d\u0438\u0442\u044c \u0442\u0440\u0435\u043d\u0438\u0440\u043e\u0432\u043a\u0443", None))
    # retranslateUi
\end{minted}

\href{https://gitlab.com/vlad4052/2024_pv223_vladislav_10}{Ссылка на репозиторий}

\textbf{Вывод: } в ходе лабораторной работы приобретели практические навыки создания приложений на
языке Python, QT приложения.

\end{document}