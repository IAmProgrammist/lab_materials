\documentclass[a4paper,14pt]{extarticle}


\usepackage[english,russian]{babel}
\usepackage[T2A]{fontenc}
\usepackage[utf8]{inputenc}
\usepackage{ragged2e}
\usepackage[utf8]{inputenc}
\usepackage{hyperref}
\usepackage{minted}
\setmintedinline{frame=lines, framesep=2mm, baselinestretch=1.5, bgcolor=LightGray, breaklines,fontsize=\scriptsize}
\setminted{frame=lines, framesep=2mm, baselinestretch=1.5, bgcolor=LightGray, breaklines,fontsize=\scriptsize}
\usepackage{xcolor}
\definecolor{LightGray}{gray}{0.9}
\usepackage{graphicx}
\usepackage[export]{adjustbox}
\usepackage[left=1cm,right=1cm, top=1cm,bottom=1cm,bindingoffset=0cm]{geometry}
\usepackage{fontspec}
\usepackage{ upgreek }
\usepackage[shortlabels]{enumitem}
\usepackage{adjustbox}
\usepackage{multirow}
\usepackage{amsmath}
\usepackage{amssymb}
\usepackage{pifont}
\usepackage{pgfplots}
\usepackage{longtable}
\usepackage{array}
\graphicspath{ {./images/} }
\makeatletter
\AddEnumerateCounter{\asbuk}{\russian@alph}{щ}
\makeatother
\setmonofont{Consolas}
\setmainfont{Times New Roman}

\newcommand\textbox[1]{
	\parbox{.45\textwidth}{#1}
} 

\newcommand{\specialcell}[2][c]{%
	\begin{tabular}[#1]{@{}c@{}}#2\end{tabular}}

\begin{document}
\pagenumbering{gobble}
\begin{center}
    \small{
        \textbf{МИНИCТЕРCТВО НАУКИ И ВЫCШЕГО ОБРАЗОВАНИЯ РОCCИЙCКОЙ ФЕДЕРАЦИИ}\\
        ФЕДЕРАЛЬНОЕ ГОCУДАРCТВЕННОЕ БЮДЖЕТНОЕ ОБРАЗОВАТЕЛЬНОЕ УЧРЕЖДЕНИЕ\\ВЫCШЕГО ОБРАЗОВАНИЯ \\
        \textbf{«БЕЛГОРОДCКИЙ ГОCУДАРCТВЕННЫЙ ТЕХНОЛОГИЧЕCКИЙ\\УНИВЕРCИТЕТ им. В. Г. ШУХОВА»\\ (БГТУ им. В.Г. Шухова)} \\
        \bigbreak
        \includegraphics[width=70mm]{log}\\
        ИНСТИТУТ ИНФОРМАЦИОННЫХ ТЕХНОЛОГИЙ И УПРАВЛЯЮЩИХ СИСТЕМ\\}
\end{center}

\vfill
\begin{center}
    \large{
        \textbf{
            Лабораторная работа №1}}\\
    \normalsize{
        по дисциплине: ООП \\
        тема: «Знакомство с интегрированной средой разработки (ИСР) Microsoft Visual
        Studio 2013 или QT»}
\end{center}
\vfill
\hfill\textbox{
    Выполнил: ст. группы ПВ-223\\Пахомов Владислав Андреевич
    \bigbreak
    Проверили: \\пр. Черников Сергей Викторович
}
\vfill\begin{center}
    Белгород 2024 г.
\end{center}
\newpage
\begin{center}
    \textbf{Лабораторная работа №1}\\
    Знакомство с интегрированной средой разработки (ИСР) Microsoft Visual Studio 2013 или QT\\
    Вариант 10
\end{center}
\textbf{Цель работы: }изучение функциональных возможностей интегрированной
среды разработки (ИСР) Visual Studio 2013 или QT.

\begin{enumerate}[1.]
    \item Пошаговое описание способа создания консольного приложения в ИСР
    Visual Studio или QT.\\
    В данной работе был использован QT. Для начала необходимо создать проект:\\
    \includegraphics[width=100mm]{qtS1}\\
    Для создания простого консольного приложения достаточно шаблона из вкладки "Проект без QT" под названием "Приложение на языке C++":\\
    \includegraphics[width=100mm]{qtS2}\\
    Выбираем размещение, систему сборки (можно оставить CMake), любой удобный комплект QT, по желанию можно добавить Git:\\
    \includegraphics[width=100mm]{qtS3}\\
    \includegraphics[width=100mm]{qtS4}\\
    \includegraphics[width=100mm]{qtS5}\\
    
    \item Исходный текст консольного приложения, созданного в соответствии с
    вариантом задания.\\
    B текстовом файле хранятся квадратные вещественные матрицы
    порядка n (n-const). Преобразовать файл, удалив из каждой матрицы
    последнюю строку и последний столбец.\\
    \begin{minted}{C++}
#include <fstream>
#include <sstream>

#define N 3

int main() {
    std::ifstream dataIn("data.txt");
    std::stringstream buf;

    if (!dataIn.is_open()) return 1;

    while (!dataIn.eof()) {
        for (int i = 0; i < N; i++) {
            for (int j = 0; j < N; j++) {
                float num;
                dataIn >> num;

                if (j == N - 1 || i == N - 1) continue;

                buf << num << " ";
            }

            if (i != N - 1)
                buf << "\n";
        }
    }

    dataIn.close();

    std::ofstream dataOut("data.txt");
    dataOut << buf.str();
    dataOut.flush();
    dataOut.close();
}
    \end{minted}
    \href{https://github.com/IAmProgrammist/oop/tree/master/src/lab1/task1}{Ссылка на репозиторий}\bigbreak
    Дан файл целых чисел. Преобразовать этот файл так, чтобы сначала
были числа, кратные трем, затем такие, которые при делении на три
дают B остатке единицу, а все остальные удалить из файла\\
    \begin{minted}{C++}
#include <fstream>
#include <vector>

#define N 3

int main() {
    std::ifstream dataIn("data.txt");
    std::vector<int> divThree;
    std::vector<int> modThreeOne;

    if (!dataIn.is_open()) return 1;

    while (!dataIn.eof()) {
        int num;
        dataIn >> num;

        if (num % 3 == 0) divThree.push_back(num);
        else if (num % 3 == 1) modThreeOne.push_back(num);
    }

    dataIn.close();

    std::ofstream dataOut("data.txt");
    
    for (auto& num : divThree)
        dataOut << num << " ";

    for (auto& num : modThreeOne)
        dataOut << num << " ";

    dataOut.flush();
    dataOut.close();
}
    \end{minted}
    
    \href{https://github.com/IAmProgrammist/oop/tree/master/src/lab1/task2}{Ссылка на репозиторий}\\

    \item Пошаговое описание способа создания экранной формы в ИСР Visual
    Studio или QT.\\
    Для создания приложения с экранной формой достаточно шаблона из вкладки "Приложение (QT)" под названием "Приложение QT Widgets":\\
    \includegraphics[width=100mm]{qtfS1}\\
    Выбираем размещение, систему сборки (можно оставить CMake). В "подробнее" можно выбрать имя класса, модуля и файла графического интерфейса. Также можно выбрать язык, любой удобный комплект QT, по желанию можно добавить Git:\\
    \includegraphics[width=100mm]{qtfS2}\\
    \includegraphics[width=100mm]{qtfS3}\\
    \includegraphics[width=100mm]{qtfS4}\\
    \includegraphics[width=100mm]{qtfS5}\\
    \includegraphics[width=100mm]{qtfS6}\\
    Открываем файл дизайна mainwindow.ui и редактируем его в соответствии с вариантом:\\
    \includegraphics[width=100mm]{qtfS7}\\
    \item Исходный текст модуля экранной формы, созданной в соответствии с
    вариантом задания.\\
    Форма «Свойства: таблицы» MS Word, вкладки «Таблица» и
«Замещающий текст»\\
    \href{https://github.com/IAmProgrammist/oop/tree/master/src/lab1/task3}{Ссылка на репозиторий, вкладка «Таблица»}\\
    \includegraphics[width=100mm]{table}\\
    
    \href{https://github.com/IAmProgrammist/oop/tree/master/src/lab1/task4}{Ссылка на репозиторий, вкладка «Замещающий текст»}\\ 
    \includegraphics[width=100mm]{replace_text}\\
        \end{enumerate}

\textbf{Вывод: } в ходе лабораторной работы изучили функциональные возможности интегрированной
среды разработки QT для создания форм и консольных приложений.

\end{document}