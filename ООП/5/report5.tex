\documentclass[a4paper,14pt]{extarticle}


\usepackage[english,russian]{babel}
\usepackage[T2A]{fontenc}
\usepackage[utf8]{inputenc}
\usepackage{ragged2e}
\usepackage[utf8]{inputenc}
\usepackage{hyperref}
\usepackage{minted}
\setmintedinline{frame=lines, framesep=2mm, baselinestretch=1.5, bgcolor=LightGray, breaklines,fontsize=\scriptsize}
\setminted{frame=lines, framesep=2mm, baselinestretch=1.5, bgcolor=LightGray, breaklines,fontsize=\scriptsize}
\usepackage{xcolor}
\definecolor{LightGray}{gray}{0.9}
\usepackage{graphicx}
\usepackage[export]{adjustbox}
\usepackage[left=1cm,right=1cm, top=1cm,bottom=1cm,bindingoffset=0cm]{geometry}
\usepackage{fontspec}
\usepackage{ upgreek }
\usepackage[shortlabels]{enumitem}
\usepackage{adjustbox}
\usepackage{multirow}
\usepackage{amsmath}
\usepackage{amssymb}
\usepackage{pifont}
\usepackage{pgfplots}
\usepackage{longtable}
\usepackage{array}
\graphicspath{ {./images/} }
\makeatletter
\AddEnumerateCounter{\asbuk}{\russian@alph}{щ}
\makeatother
\setmonofont{Consolas}
\setmainfont{Times New Roman}

\newcommand\textbox[1]{
	\parbox{.45\textwidth}{#1}
} 

\newcommand{\specialcell}[2][c]{%
	\begin{tabular}[#1]{@{}c@{}}#2\end{tabular}}

\begin{document}
\pagenumbering{gobble}
\begin{center}
    \small{
        \textbf{МИНИCТЕРCТВО НАУКИ И ВЫCШЕГО ОБРАЗОВАНИЯ РОCCИЙCКОЙ ФЕДЕРАЦИИ}\\
        ФЕДЕРАЛЬНОЕ ГОCУДАРCТВЕННОЕ БЮДЖЕТНОЕ ОБРАЗОВАТЕЛЬНОЕ УЧРЕЖДЕНИЕ\\ВЫCШЕГО ОБРАЗОВАНИЯ \\
        \textbf{«БЕЛГОРОДCКИЙ ГОCУДАРCТВЕННЫЙ ТЕХНОЛОГИЧЕCКИЙ\\УНИВЕРCИТЕТ им. В. Г. ШУХОВА»\\ (БГТУ им. В.Г. Шухова)} \\
        \bigbreak
        \includegraphics[width=70mm]{log}\\
        ИНСТИТУТ ИНФОРМАЦИОННЫХ ТЕХНОЛОГИЙ И УПРАВЛЯЮЩИХ СИСТЕМ\\}
\end{center}

\vfill
\begin{center}
    \large{
        \textbf{
            Лабораторная работа №5}}\\
    \normalsize{
        по дисциплине: ООП \\
        тема: «Классы, виды отношений. Наследование.»}
\end{center}
\vfill
\hfill\textbox{
    Выполнил: ст. группы ПВ-223\\Пахомов Владислав Андреевич
    \bigbreak
    Проверили: \\пр. Черников Сергей Викторович
}
\vfill\begin{center}
    Белгород 2024 г.
\end{center}
\newpage
\begin{center}
    \textbf{Лабораторная работа №5}\\
    «Классы»\\
    Вариант 6
\end{center}
\textbf{Цель работы: }получение теоретических знаний в области
разработки классов, получение практических навыков реализаций классов
и отношений между ними.

\textbf{Задание 1}\\
Выполнить построение объектной модели заданной предметной области:\\
Система закупок.\\
\includegraphics[width=190mm]{task1}

\textbf{Задание 2}\\
Разработать
диаграмму классов для описанной объектной модели,
и реализовать предложенные классы.

\begin{enumerate}[1. ]
    \item Создать абстрактный класс Товар с методами, позволяющим вывести на
    экран информацию о товаре, а также определить, соответствует ли она
    сроку годности на текущую дату.
    \item Создать производные классы: Продукт (название, цена, дата
    производства, срок годности), Партия (название, цена, количество шт.,
    дата производства, срок годности), Комплект (названия, цена, перечень
    продуктов) со своими методами вывода информации на экран, и
    определения соответствия сроку годности.
    \item Создать базу (массив) из n товаров, вывести полную информацию из
    базы на экран, а также организовать поиск просроченного товара (на
    момент текущей даты).
\end{enumerate}

\includegraphics[width=190mm]{task2}

\textit{goods.h}
\begin{minted}{C++}
#pragma once

#include "../date/date.h"

class Goods {
public:
    virtual bool isExpired() = 0;

    virtual std::ostream& print(std::ostream& out) = 0;
};

class Product : public Goods {
    std::string name;
    int price;
    Date production_date;
    boost::gregorian::date_duration expires_in;
public:
    Product(std::string name,
            int price,
            Date production_date,
            boost::gregorian::date_duration expirates_in) :
        name(name), price(price), production_date(production_date), expires_in(expirates_in) {};

    virtual bool isExpired() override;
    virtual std::ostream& print(std::ostream& out) override;
};

class Shipment : public Goods {
    std::string name;
    int price;
    int amount;
    Date production_date;
    boost::gregorian::date_duration expires_in;
public:
    Shipment(std::string name,
            int price,
            int amount,
            Date production_date,
            boost::gregorian::date_duration expirates_in) :
        name(name), price(price), amount(amount), production_date(production_date), expires_in(expirates_in) {};

    virtual bool isExpired() override;
    virtual std::ostream& print(std::ostream& out) override;
};

class Package : public Goods {
    std::string name;
    int price;
    std::vector<Product> products;
public:
    Package(std::string name,
             int price,
             std::vector<Product> products) :
        name(name), price(price), products(products) {};

    virtual bool isExpired() override;
    virtual std::ostream& print(std::ostream& out) override;
};

class Base {
    std::vector<Goods*> goods;
public:
    bool containsGoods(Goods* goods);
    void addGoods(Goods* goods);
    std::ostream& print(std::ostream& out);
    std::vector<Goods*> getExpired();
};

class Shop {
    Base base;
    std::vector<Goods*> cart;
public:
    void addGoods(Goods* goods);
    void addGoodToCart(Goods* good);
    std::ostream& print(std::ostream& out);
    std::vector<Goods*> getExpired();
};
\end{minted}

\textit{goods.cpp}
\begin{minted}{C++}
#include "goods.h"

bool Product::isExpired() {
    boost::gregorian::date current_date(boost::gregorian::day_clock::local_day());
    Date now(current_date);
    Date exDate = Date(this->production_date.getData() + this->expires_in);
    return now >= exDate;
}

std::ostream& Product::print(std::ostream& out) {
    out << "Name: " << this->name << "\n";
    out << "Price: " << this->price << "\n";
    out << "Production date: " << this->production_date.getData() << "\n";
    out << "Expires at: " << this->production_date.getData() + this->expires_in << "\n";
    out << "Expired: " << (this->isExpired() ? "yup" : "nuh uh") << "\n";

    return out;
}

bool Shipment::isExpired() {
    boost::gregorian::date current_date(boost::gregorian::day_clock::local_day());
    Date now(current_date);
    Date exDate = Date(this->production_date.getData() + this->expires_in);
    return now >= exDate;
}

std::ostream& Shipment::print(std::ostream& out) {
    out << "Name: " << this->name << "\n";
    out << "Price: " << this->price << "\n";
    out << "Amount: " << this->amount << "\n";
    out << "Production date: " << this->production_date.getData() << "\n";
    out << "Expires at: " << this->production_date.getData() + this->expires_in << "\n";
    out << "Expired: " << (this->isExpired() ? "yup" : "nuh uh") << "\n";

    return out;
}

bool Package::isExpired() {
    for (auto& good : this->products)
        if (good.isExpired()) return true;

    return false;
}

std::ostream& Package::print(std::ostream& out) {
    out << "Name: " << this->name << "\n";
    out << "Price: " << this->price << "\n";
    out << "Products:\n";
    out << "=============\n";
    for (auto& prod : this->products) {
        prod.print(out);
    }
    out << "=============\n";

    return out;
}

bool Base::containsGoods(Goods* goods) {
    return std::find(this->goods.begin(), this->goods.end(), goods) != this->goods.end();
}

void Base::addGoods(Goods* goods) {
    if (!containsGoods(goods))
        this->goods.push_back(goods);
}

std::ostream& Base::print(std::ostream& out) {
    std::cout << "*************\n";

    for (auto & good : this->goods) {
        good->print(out);
        std::cout << "*************\n";
    }

    return out;
}

std::vector<Goods*> Base::getExpired() {
    std::vector<Goods*> result;
    for (auto & good : this->goods)
        if (good->isExpired())
            result.push_back(good);

    return result;
}

void Shop::addGoods(Goods* goods) {
    this->base.addGoods(goods);
}

void Shop::addGoodToCart(Goods* good) {
    if (this->base.containsGoods(good))
        this->cart.push_back(good);
}

std::ostream& Shop::print(std::ostream& out) {
    std::cout << "Catalog: \n";
    this->base.print(std::cout);
    std::cout << "Cart: \n";

    std::cout << "*************\n";

    for (auto & good : this->cart) {
        good->print(out);
        std::cout << "*************\n";
    }

    return out;
}

std::vector<Goods*> Shop::getExpired() {
    return this->base.getExpired();
}
\end{minted}

\textit{main.cpp}
\begin{minted}{C++}
#include <iostream>
#include <windows.h>

#include "../libs/alg/goods/goods.h"

int main() {
    Shop shop;
    shop.addGoods(new Product("Monitor", 10000, Date(5, 3, 2024), boost::gregorian::days(3000)));
    shop.addGoods(new Shipment("Sneakers", 3000, 150, Date(3, 3, 2023), boost::gregorian::days(3000)));
    shop.addGoods(new Package("Vegetrables", 10313,
                           {Product("Tomatoes", 100, Date(5, 3, 2024), boost::gregorian::days(45)),
                            Product("Cucumber", 100, Date(5, 3, 2024), boost::gregorian::days(45)),
                            Product("Salad", 300, Date(5, 3, 2024), boost::gregorian::days(45))}));
    shop.addGoods(new Product("Tomatoes", 100, Date(5, 3, 2024), boost::gregorian::days(0)));

    shop.print(std::cout);
}
\end{minted}

Результаты выполнения программы:
\begin{minted}{console}
Catalog: 
*************
Name: Monitor
Price: 10000
Production date: 2024-Mar-05
Expires at: 2032-May-22
Expired: nuh uh
*************
Name: Sneakers
Price: 3000
Amount: 150
Production date: 2023-Mar-03
Expires at: 2031-May-20
Expired: nuh uh
*************
Name: Vegetrables
Price: 10313
Products:
=============
Name: Tomatoes
Price: 100
Production date: 2024-Mar-05
Expires at: 2024-Apr-19
Expired: nuh uh
Name: Cucumber
Price: 100
Production date: 2024-Mar-05
Expires at: 2024-Apr-19
Expired: nuh uh
Name: Salad
Price: 300
Production date: 2024-Mar-05
Expires at: 2024-Apr-19
Expired: nuh uh
=============
*************
Name: Tomatoes
Price: 100
Production date: 2024-Mar-05
Expires at: 2024-Mar-05
Expired: yup
*************
Cart: 
*************
\end{minted}
\href{https://github.com/IAmProgrammist/oop/tree/master}{Ссылка на репозиторий}

\textbf{Вывод: } в ходе лабораторной работы получены теоретические знания в области
разработки классов, получены практические навыки реализаций классов
и отношений между ними.

\end{document}