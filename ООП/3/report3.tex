\documentclass[a4paper,14pt]{extarticle}


\usepackage[english,russian]{babel}
\usepackage[T2A]{fontenc}
\usepackage[utf8]{inputenc}
\usepackage{ragged2e}
\usepackage[utf8]{inputenc}
\usepackage{hyperref}
\usepackage{minted}
\setmintedinline{frame=lines, framesep=2mm, baselinestretch=1.5, bgcolor=LightGray, breaklines,fontsize=\scriptsize}
\setminted{frame=lines, framesep=2mm, baselinestretch=1.5, bgcolor=LightGray, breaklines,fontsize=\scriptsize}
\usepackage{xcolor}
\definecolor{LightGray}{gray}{0.9}
\usepackage{graphicx}
\usepackage[export]{adjustbox}
\usepackage[left=1cm,right=1cm, top=1cm,bottom=1cm,bindingoffset=0cm]{geometry}
\usepackage{fontspec}
\usepackage{ upgreek }
\usepackage[shortlabels]{enumitem}
\usepackage{adjustbox}
\usepackage{multirow}
\usepackage{amsmath}
\usepackage{amssymb}
\usepackage{pifont}
\usepackage{pgfplots}
\usepackage{longtable}
\usepackage{array}
\graphicspath{ {./images/} }
\makeatletter
\AddEnumerateCounter{\asbuk}{\russian@alph}{щ}
\makeatother
\setmonofont{Consolas}
\setmainfont{Times New Roman}

\newcommand\textbox[1]{
	\parbox{.45\textwidth}{#1}
} 

\newcommand{\specialcell}[2][c]{%
	\begin{tabular}[#1]{@{}c@{}}#2\end{tabular}}

\begin{document}
\pagenumbering{gobble}
\begin{center}
    \small{
        \textbf{МИНИCТЕРCТВО НАУКИ И ВЫCШЕГО ОБРАЗОВАНИЯ РОCCИЙCКОЙ ФЕДЕРАЦИИ}\\
        ФЕДЕРАЛЬНОЕ ГОCУДАРCТВЕННОЕ БЮДЖЕТНОЕ ОБРАЗОВАТЕЛЬНОЕ УЧРЕЖДЕНИЕ\\ВЫCШЕГО ОБРАЗОВАНИЯ \\
        \textbf{«БЕЛГОРОДCКИЙ ГОCУДАРCТВЕННЫЙ ТЕХНОЛОГИЧЕCКИЙ\\УНИВЕРCИТЕТ им. В. Г. ШУХОВА»\\ (БГТУ им. В.Г. Шухова)} \\
        \bigbreak
        \includegraphics[width=70mm]{log}\\
        ИНСТИТУТ ИНФОРМАЦИОННЫХ ТЕХНОЛОГИЙ И УПРАВЛЯЮЩИХ СИСТЕМ\\}
\end{center}

\vfill
\begin{center}
    \large{
        \textbf{
            Лабораторная работа №3}}\\
    \normalsize{
        по дисциплине: ООП \\
        тема: «Объектная декомпозиция»}
\end{center}
\vfill
\hfill\textbox{
    Выполнил: ст. группы ПВ-223\\Пахомов Владислав Андреевич
    \bigbreak
    Проверили: \\пр. Черников Сергей Викторович
}
\vfill\begin{center}
    Белгород 2024 г.
\end{center}
\newpage
\begin{center}
    \textbf{Лабораторная работа №3}\\
    «Объектная декомпозиция»\\
    Вариант 10
\end{center}
\textbf{Цель работы: }приобретение навыков выполнения объектной
декомпозиции, выявления объектов и отношений между ними в заданной
предметной области.

для указанных в варианте заданий выполнить
объектную декомпозицию, построить диаграмму взаимодействия объектов
(минимум 7 объектов).\\

\textbf{Задание 1}\\
Программа для воспроизведения музыкальных файлов\\
Выполним объектную декомпозицию программу для воспроизведения музыкальных файлов. 
Пользователь может накладывать различные эффекты, добавлять в плейлист треки, изменять их порядок, 
ставить на паузу, изменять громкость воспроизведения.
Эти процессы можно моделировать при помощи 7 объектов: 
плеер, плейлист, трек, библиотека эффектов, эхо, эквалайзер, компрессор.\\
\includegraphics[width=190mm]{task1}\\
Эффекты получают сообщения добавления, сообщения инициируются библиотекой эффектов и плеером
в соответствии с командой пользователя. Добавление/удаление трека инициируется плейлистом, плеером, 
пользователем. Сообщение об изменении порядка треков в плейлисте и текущем треке инициируется плеером, 
пользователем. Сообщения о паузе, изменении громкости приходят в плеер от пользователя.\\

\textbf{Задание 2}\\
Компьютерная игра в жанре «Стратегия» с произвольным заданием концепции\\
Выполним объектную декомпозицию игры. 
Игрок может создавать город, улучшать его, строить фермы, шахты и получать от них ресурсы. 
Также строить казармы, тренировать воинов и нападать при помощи них на врагов.
Эти процессы можно моделировать при помощи 7 объектов: 
игрок, город, ферма, шахта, казармы, воин, враг.\\
\includegraphics[width=190mm]{task2}\\
Воин может защищаться от врагов и атаковать их, также воин
может получить сообщение о перемещении от казармы при помощи \"Направить\".
Это сообщение в казарму приходит от игрока. Сообщение о начале работы и постройке 
приходит из города в объекты шахты, фермы, казармы. Игрок также может 
улучшить город, это сообщение приходит в город. \\

\textbf{Вывод: } в ходе лабораторной работы приобретены навыки выполнения объектной
декомпозиции, выявления объектов и отношений между ними в заданной
предметной области.

\end{document}