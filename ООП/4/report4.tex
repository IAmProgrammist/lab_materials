\documentclass[a4paper,14pt]{extarticle}


\usepackage[english,russian]{babel}
\usepackage[T2A]{fontenc}
\usepackage[utf8]{inputenc}
\usepackage{ragged2e}
\usepackage[utf8]{inputenc}
\usepackage{hyperref}
\usepackage{minted}
\setmintedinline{frame=lines, framesep=2mm, baselinestretch=1.5, bgcolor=LightGray, breaklines,fontsize=\scriptsize}
\setminted{frame=lines, framesep=2mm, baselinestretch=1.5, bgcolor=LightGray, breaklines,fontsize=\scriptsize}
\usepackage{xcolor}
\definecolor{LightGray}{gray}{0.9}
\usepackage{graphicx}
\usepackage[export]{adjustbox}
\usepackage[left=1cm,right=1cm, top=1cm,bottom=1cm,bindingoffset=0cm]{geometry}
\usepackage{fontspec}
\usepackage{ upgreek }
\usepackage[shortlabels]{enumitem}
\usepackage{adjustbox}
\usepackage{multirow}
\usepackage{amsmath}
\usepackage{amssymb}
\usepackage{pifont}
\usepackage{pgfplots}
\usepackage{longtable}
\usepackage{array}
\graphicspath{ {./images/} }
\makeatletter
\AddEnumerateCounter{\asbuk}{\russian@alph}{щ}
\makeatother
\setmonofont{Consolas}
\setmainfont{Times New Roman}

\newcommand\textbox[1]{
	\parbox{.45\textwidth}{#1}
} 

\newcommand{\specialcell}[2][c]{%
	\begin{tabular}[#1]{@{}c@{}}#2\end{tabular}}

\begin{document}
\pagenumbering{gobble}
\begin{center}
    \small{
        \textbf{МИНИCТЕРCТВО НАУКИ И ВЫCШЕГО ОБРАЗОВАНИЯ РОCCИЙCКОЙ ФЕДЕРАЦИИ}\\
        ФЕДЕРАЛЬНОЕ ГОCУДАРCТВЕННОЕ БЮДЖЕТНОЕ ОБРАЗОВАТЕЛЬНОЕ УЧРЕЖДЕНИЕ\\ВЫCШЕГО ОБРАЗОВАНИЯ \\
        \textbf{«БЕЛГОРОДCКИЙ ГОCУДАРCТВЕННЫЙ ТЕХНОЛОГИЧЕCКИЙ\\УНИВЕРCИТЕТ им. В. Г. ШУХОВА»\\ (БГТУ им. В.Г. Шухова)} \\
        \bigbreak
        \includegraphics[width=70mm]{log}\\
        ИНСТИТУТ ИНФОРМАЦИОННЫХ ТЕХНОЛОГИЙ И УПРАВЛЯЮЩИХ СИСТЕМ\\}
\end{center}

\vfill
\begin{center}
    \large{
        \textbf{
            Лабораторная работа №4}}\\
    \normalsize{
        по дисциплине: ООП \\
        тема: «Классы»}
\end{center}
\vfill
\hfill\textbox{
    Выполнил: ст. группы ПВ-223\\Пахомов Владислав Андреевич
    \bigbreak
    Проверили: \\пр. Черников Сергей Викторович
}
\vfill\begin{center}
    Белгород 2024 г.
\end{center}
\newpage
\begin{center}
    \textbf{Лабораторная работа №4}\\
    «Классы»\\
    Вариант 10
\end{center}
\textbf{Цель работы: }приобретение практических навыков создания класса на
языке С++.

\textbf{Задание 1}\\
Выполнить построение диаграммы объектов в соответствии с заданием варианта.\\
Выполнить
построение объектной
модели следующей
предметной области:
“органайзер”.\\
\includegraphics[width=190mm]{task1}

\textbf{Задание 2}\\
Создать класс для работы с датой.
Разработать следующие элементы
класса:
\begin{enumerate}[label=\asbuk*),ref=\asbuk*]
    \item Поле DataTime data.
    \item Конструкторы, позволяющие
    установить:\\
    заданную дату;\\
    дату 1.01.2009.
    \item Методы, позволяющие:
    вычислить дату предыдущего дня;
    вычислить дату следующего дня;
    определить сколько дней осталось до
    конца месяца.
    \item Перегрузить (переопределить):
    логический сдвиг влево
    операция “больше или равно”.
\end{enumerate}

\textit{date.h}
\begin{minted}{C++}
#pragma once

#include <boost/date_time.hpp>

class Date {
    boost::gregorian::date data;
public:
    Date(boost::gregorian::date newDate) : data(newDate) {};
    Date(int d, int m, int y) : data(y, m, d) {};

    Date getDayAfter();
    Date getDayBefore();
    int getDaysBeforeMonthEnd();
    boost::gregorian::date getData();

    Date& operator<<(int amount);
    Date& operator>>(int amount);
    bool operator>=(Date& another) const;
};

\end{minted}

\textit{date.cpp}
\begin{minted}{C++}
#include "date.h"

Date Date::getDayAfter() {
    return Date(this->data + boost::gregorian::days(1));
}

Date Date::getDayBefore() {
    return Date(this->data - boost::gregorian::days(1));
}

int Date::getDaysBeforeMonthEnd() {
    boost::gregorian::date date_next_month = this->data.end_of_month() + boost::gregorian::days(1);

    return (date_next_month - this->data).days();
}

Date& Date::operator<<(int amount) {
    this->data = this->data - boost::gregorian::days(amount);

    return *this;
}

Date& Date::operator>>(int amount) {
    this->data = this->data + boost::gregorian::days(amount);

    return *this;
}

bool Date::operator>=(Date& another) const {
    if (this->data.year() < another.data.year()) return false;
    if (this->data.year() > another.data.year()) return true;

    if (this->data.month() < another.data.month()) return false;
    if (this->data.month() > another.data.month()) return true;

    return this->data.day() >= another.data.day();
}

boost::gregorian::date Date::getData() {
    return this->data;
}

\end{minted}

\textit{main.cpp}
\begin{minted}{C++}
#include <iostream>

#include "../libs/alg/date/date.h"

int main() {
    Date d(23, 11, 2023);

    Date a_lot_of_days_later = d.getDayAfter().getDayAfter().getDayAfter().getDayAfter().getDayAfter().getDayAfter();
    Date a_lot_of_days_before = d.getDayBefore().getDayBefore().getDayBefore().getDayBefore().getDayBefore();

    Date time_travel = d << 1 >> 312 << 23 >> 121 >> 90 << 80;

    std::cout << a_lot_of_days_before.getData() << " " << a_lot_of_days_before.getDaysBeforeMonthEnd() << std::endl;
    std::cout << a_lot_of_days_later.getData() << " " << a_lot_of_days_later.getDaysBeforeMonthEnd() << std::endl;
    std::cout << "Is a_lot_of_days_before >= a_lot_of_days_later? " << (a_lot_of_days_before >= a_lot_of_days_later) << std::endl;
    std::cout << time_travel.getData() << std::endl;
}

\end{minted}

Результаты выполнения программы:
\begin{minted}{console}
2023-Nov-18 13
2023-Nov-29 2
Is a_lot_of_days_before >= a_lot_of_days_later? 0
2025-Jan-15
\end{minted}

\textbf{Вывод: } в ходе лабораторной работы приобрели практические навыки создания класса на
языке С++.

\end{document}