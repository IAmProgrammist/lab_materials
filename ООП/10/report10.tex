\documentclass[a4paper,14pt]{extarticle}


\usepackage[english,russian]{babel}
\usepackage[T2A]{fontenc}
\usepackage[utf8]{inputenc}
\usepackage{ragged2e}
\usepackage[utf8]{inputenc}
\usepackage{hyperref}
\usepackage{minted}
\setmintedinline{frame=lines, framesep=2mm, baselinestretch=1.5, bgcolor=LightGray, breaklines,fontsize=\scriptsize}
\setminted{frame=lines, framesep=2mm, baselinestretch=1.5, bgcolor=LightGray, breaklines,fontsize=\scriptsize}
\usepackage{xcolor}
\definecolor{LightGray}{gray}{0.9}
\usepackage{graphicx}
\usepackage[export]{adjustbox}
\usepackage[left=1cm,right=1cm, top=1cm,bottom=1cm,bindingoffset=0cm]{geometry}
\usepackage{fontspec}
\usepackage{ upgreek }
\usepackage[shortlabels]{enumitem}
\usepackage{adjustbox}
\usepackage{multirow}
\usepackage{amsmath}
\usepackage{amssymb}
\usepackage{pifont}
\usepackage{pgfplots}
\usepackage{longtable}
\usepackage{array}
\graphicspath{ {./images/} }
\makeatletter
\AddEnumerateCounter{\asbuk}{\russian@alph}{щ}
\makeatother
\setmonofont{Consolas}
\setmainfont{Times New Roman}

\newcommand\textbox[1]{
	\parbox{.45\textwidth}{#1}
} 

\newcommand{\specialcell}[2][c]{%
	\begin{tabular}[#1]{@{}c@{}}#2\end{tabular}}

\begin{document}
\pagenumbering{gobble}
\begin{center}
    \small{
        \textbf{МИНИCТЕРCТВО НАУКИ И ВЫCШЕГО ОБРАЗОВАНИЯ РОCCИЙCКОЙ ФЕДЕРАЦИИ}\\
        ФЕДЕРАЛЬНОЕ ГОCУДАРCТВЕННОЕ БЮДЖЕТНОЕ ОБРАЗОВАТЕЛЬНОЕ УЧРЕЖДЕНИЕ\\ВЫCШЕГО ОБРАЗОВАНИЯ \\
        \textbf{«БЕЛГОРОДCКИЙ ГОCУДАРCТВЕННЫЙ ТЕХНОЛОГИЧЕCКИЙ\\УНИВЕРCИТЕТ им. В. Г. ШУХОВА»\\ (БГТУ им. В.Г. Шухова)} \\
        \bigbreak
        \includegraphics[width=70mm]{log}\\
        ИНСТИТУТ ИНФОРМАЦИОННЫХ ТЕХНОЛОГИЙ И УПРАВЛЯЮЩИХ СИСТЕМ\\}
\end{center}

\vfill
\begin{center}
    \large{
        \textbf{
            Лабораторная работа №10}}\\
    \normalsize{
        по дисциплине: ООП \\
        тема: «Закрепление навыков программирования в объектно-ориентированном стиле. Визуальные компоненты. Знакомство с QT.»}
\end{center}
\vfill
\hfill\textbox{
    Выполнил: ст. группы ПВ-223\\Пахомов Владислав Андреевич
    \bigbreak
    Проверили: \\пр. Черников Сергей Викторович
}
\vfill\begin{center}
    Белгород 2024 г.
\end{center}
\newpage
\begin{center}
    \textbf{Лабораторная работа №10}\\
    «Закрепление навыков программирования в объектно-ориентированном стиле. Визуальные компоненты. Знакомство с QT.»\\
    Вариант 10
\end{center}
\textbf{Цель работы: }приобретение практических навыков создания приложений на
языке С++.

Создать репозиторий под контролем git на открытой площадке gitlab.com
Задать имя репозитория <год>\_<группа>\_<имя студента в транслите>\_<номер
варианта по журналу>. Выполнить проектирование задачи в соответствии с вариантом (табл. 1).
Для реализации поставленной задачи необходимо с проектировать, реализовать
и использовать шаблон «умные указатели». Соответственно это учесть при
проектировании программного обеспечения. Выполнить реализацию в соответствии с вариантом задачи (табл. 1),
использую среду разработки QT. Реализация должна быть кроссплатформенной
и выполнена на основе графических окон.
\textit{main.cpp}
\begin{minted}{C++}
#include "mainwindow.h"
#include "smartptr.h"

int main(int argc, char *argv[])
{
    QApplication a(argc, argv);
    smart_ptr<QMainWindow> w(new QMainWindow());
    Ui_MainWindow b;
    b.setupUi(w.operator->());

    w->show();

    return a.exec();
}
\end{minted}
\textit{mainwindow.h}
\begin{minted}{C++}
#ifndef MAINWINDOW_H
#define MAINWINDOW_H

#include <QVariant>
#include <QAction>
#include <QApplication>
#include <QButtonGroup>
#include <QHeaderView>
#include <QMainWindow>
#include <QMenuBar>
#include <QPushButton>
#include <QStatusBar>
#include <QTextEdit>
#include <QToolBar>
#include <QWidget>
#include <QGridLayout>

#include <queue>

#include "../libs/alg/mathparser/mathparser.h"
#include "smartptr.h"

QT_BEGIN_NAMESPACE
class Ui_MainWindow {
    const static int screenHeight = 400;
    const static int screenWidth = 300;

    smart_ptr<QWidget> centralWidget = new QWidget();
    smart_ptr<QGridLayout> mainLayout = new QGridLayout(centralWidget);

    smart_ptr<QTextEdit> field = new QTextEdit;

    smart_ptr<QPushButton> plusButton = new QPushButton("+");
    smart_ptr<QPushButton> multButton = new QPushButton("*");
    smart_ptr<QPushButton> divButton = new QPushButton("/");
    smart_ptr<QPushButton> powButton = new QPushButton("^");

    smart_ptr<QPushButton> tildaButton = new QPushButton("~");
    smart_ptr<QPushButton> lParButton = new QPushButton("(");
    smart_ptr<QPushButton> rParButton = new QPushButton(")");

    smart_ptr<QPushButton> oneButton = new QPushButton("1");
    smart_ptr<QPushButton> twoButton = new QPushButton("2");
    smart_ptr<QPushButton> threeButton = new QPushButton("3");
    smart_ptr<QPushButton> sinButton = new QPushButton("sin");


    smart_ptr<QPushButton> fourButton = new QPushButton("4");
    smart_ptr<QPushButton> fiveButton = new QPushButton("5");
    smart_ptr<QPushButton> sixButton = new QPushButton("6");
    smart_ptr<QPushButton> cosButton = new QPushButton("cos");

    smart_ptr<QPushButton> sevenButton = new QPushButton("7");
    smart_ptr<QPushButton> eightButton = new QPushButton("8");
    smart_ptr<QPushButton> nineButton = new QPushButton("9");
    smart_ptr<QPushButton> minusButton = new QPushButton("-");

    smart_ptr<QPushButton> zeroButton = new QPushButton("0");
    smart_ptr<QPushButton> dotButton = new QPushButton(".");
    smart_ptr<QPushButton> eqButton = new QPushButton("=");
    smart_ptr<QPushButton> eraseButton = new QPushButton("<-");

    std::deque<MathParser::Token> tokens;
public:
    void setupUi(QMainWindow *mainWindow) {
        mainWindow->setWindowTitle("Инженерный калькулятор");

        centralWidget->setMinimumHeight(screenHeight);
        centralWidget->setFixedHeight(screenHeight);
        centralWidget->setMaximumHeight(screenHeight);

        centralWidget->setMinimumWidth(screenWidth);
        centralWidget->setFixedWidth(screenWidth);
        centralWidget->setMaximumWidth(screenWidth);

        mainWindow->setMinimumHeight(screenHeight);
        mainWindow->setFixedHeight(screenHeight);
        mainWindow->setMaximumHeight(screenHeight);

        mainWindow->setMinimumWidth(screenWidth);
        mainWindow->setFixedWidth(screenWidth);
        mainWindow->setMaximumWidth(screenWidth);

        field->setMaximumHeight(50);
        field->setReadOnly(true);

        mainLayout->addWidget(field, 0, 0, 1, 4);

        mainWindow->setCentralWidget(centralWidget);

        mainLayout->addWidget(plusButton, 1, 0);
        mainLayout->addWidget(multButton, 1, 1);
        mainLayout->addWidget(divButton, 1, 2);
        mainLayout->addWidget(powButton, 1, 3);

        mainLayout->addWidget(lParButton, 2, 0);
        mainLayout->addWidget(rParButton, 2, 1);
        mainLayout->addWidget(tildaButton, 2, 2);
        mainLayout->addWidget(eraseButton, 2, 3);

        mainLayout->addWidget(oneButton, 3, 0);
        mainLayout->addWidget(twoButton, 3, 1);
        mainLayout->addWidget(threeButton, 3, 2);
        mainLayout->addWidget(sinButton, 3, 3);

        mainLayout->addWidget(fourButton, 4, 0);
        mainLayout->addWidget(fiveButton, 4, 1);
        mainLayout->addWidget(sixButton, 4, 2);
        mainLayout->addWidget(cosButton, 4, 3);

        mainLayout->addWidget(sevenButton, 5, 0);
        mainLayout->addWidget(eightButton, 5, 1);
        mainLayout->addWidget(nineButton, 5, 2);
        mainLayout->addWidget(minusButton, 5, 3);

        mainLayout->addWidget(zeroButton, 6, 0);
        mainLayout->addWidget(dotButton, 6, 1);
        mainLayout->addWidget(eqButton, 6, 2, 1, 2);

        std::vector<QPushButton*> numButtons = {oneButton, twoButton, threeButton, fourButton, fiveButton, sixButton, sevenButton, eightButton, nineButton, zeroButton, dotButton};

        for (auto numButton : numButtons) {
            QObject::connect(numButton, &QPushButton::clicked, [this, numButton]() {
                if (!this->tokens.empty() && this->tokens.back().isNumber()) {
                    auto token = this->tokens.back();
                    this->tokens.pop_back();

                    this->tokens.push_back(MathParser::Token(token.val + numButton->text().toStdString()));
                } else {
                    this->tokens.push_back(MathParser::Token(numButton->text().toStdString()));
                }

                this->onTokensUpdated();
            });
        }

        std::vector<QPushButton*> opButtons = {plusButton, minusButton, multButton, divButton, powButton, sinButton, cosButton, lParButton, rParButton, tildaButton};

        for (auto opButton : opButtons) {
            QObject::connect(opButton, &QPushButton::clicked, [this, opButton]() {
                auto searchToken = std::find_if(MathParser::tokens.begin(), MathParser::tokens.end(),
                                                [opButton](MathParser::Token v){return v.val == opButton->text().toStdString();});
                this->tokens.push_back(*searchToken);

                this->onTokensUpdated();
            });
        }

        QObject::connect(eraseButton, &QPushButton::clicked, [this]() {
            if (!this->tokens.empty() && this->tokens.back().isNumber() && this->tokens.back().val.size() > 1) {
                auto token = this->tokens.back();
                this->tokens.pop_back();

                this->tokens.push_back(MathParser::Token(token.val.substr(0, token.val.size() - 1)));
            } else if (tokens.empty()) {
                field->setText("");
            } else {
                this->tokens.pop_back();
            }

            this->onTokensUpdated();
        });

        QObject::connect(eqButton, &QPushButton::clicked, [this]() {
            try {
                std::queue<MathParser::Token> exprQueue(this->tokens);
                exprQueue = MathParser::infixToPostfix(exprQueue);
                auto res = MathParser::evaluate(exprQueue);
                field->setText(QString::fromStdString(res.val));
            } catch(...) {
                field->setText("Syntax error");
            }

            this->tokens.clear();
        });
    }

    void onTokensUpdated() {
        std::queue<MathParser::Token> exprQueue(this->tokens);
        std::string v = printSequence(exprQueue);
        field->setText(QString::fromStdString(v));
    }
};
namespace Ui{
class MainWindow:public Ui_MainWindow{};
}//namespaceUi
QT_END_NAMESPACE


#endif // MAINWINDOW_H
\end{minted}

\href{https://gitlab.com/vlad4052/2024_pv223_vladislav_10}{Ссылка на репозиторий}

\textbf{Вывод: } в ходе лабораторной работы приобрели практические навыки создания приложений на
языке С++.

\end{document}