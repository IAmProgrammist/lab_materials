\documentclass[a4paper,14pt]{extarticle}


\usepackage[english,russian]{babel}
\usepackage[T2A]{fontenc}
\usepackage[utf8]{inputenc}
\usepackage{ragged2e}
\usepackage[utf8]{inputenc}
\usepackage{hyperref}
\usepackage{minted}
\setmintedinline{frame=lines, framesep=2mm, baselinestretch=1.5, bgcolor=LightGray, breaklines,fontsize=\scriptsize}
\setminted{frame=lines, framesep=2mm, baselinestretch=1.5, bgcolor=LightGray, breaklines,fontsize=\scriptsize}
\usepackage[table]{xcolor}
\definecolor{LightGray}{gray}{0.9}
\definecolor{Yellow}{rgb}{1,1,0}
\usepackage{graphicx}
\usepackage[export]{adjustbox}
\usepackage[left=1cm,right=1cm, top=1cm,bottom=1cm,bindingoffset=0cm]{geometry}
\usepackage{fontspec}
\usepackage{ upgreek }
\usepackage[shortlabels]{enumitem}
\usepackage{adjustbox}
\usepackage{multirow}
\usepackage{amsmath}
\usepackage{amssymb}
\usepackage{pifont}
\usepackage{pgfplots}
\usepackage{longtable}
\usepackage{array}

\graphicspath{ {./images/} }
\makeatletter
\AddEnumerateCounter{\asbuk}{\russian@alph}{щ}
\makeatother
\setmonofont{Consolas}
\setmainfont{Times New Roman}
\newcommand{\anonsection}[1]{\section*{#1}\addcontentsline{toc}{section}{#1}}

\newcommand\textbox[1]{
	\parbox{.45\textwidth}{#1}
} 

\newcommand{\specialcell}[2][c]{%
	\begin{tabular}[#1]{@{}c@{}}#2\end{tabular}}

\begin{document}
\pagenumbering{gobble}
\begin{center}
    \small{
        \textbf{МИНИCТЕРCТВО НАУКИ И ВЫCШЕГО ОБРАЗОВАНИЯ РОCCИЙCКОЙ ФЕДЕРАЦИИ}\\
        ФЕДЕРАЛЬНОЕ ГОCУДАРCТВЕННОЕ БЮДЖЕТНОЕ ОБРАЗОВАТЕЛЬНОЕ УЧРЕЖДЕНИЕ\\ВЫCШЕГО ОБРАЗОВАНИЯ \\
        \textbf{«БЕЛГОРОДCКИЙ ГОCУДАРCТВЕННЫЙ ТЕХНОЛОГИЧЕCКИЙ\\УНИВЕРCИТЕТ им. В. Г. ШУХОВА»\\ (БГТУ им. В.Г. Шухова)} \\
        \bigbreak
        \includegraphics[width=70mm]{log}\\
        ИНСТИТУТ ИНФОРМАЦИОННЫХ ТЕХНОЛОГИЙ И УПРАВЛЯЮЩИХ СИСТЕМ\\}
\end{center}

\vfill
\begin{center}
    \large{
        \textbf{
            РГЗ}}\\
    \normalsize{
        по дисциплине: Архитектура вычислительных систем \\
        тема: «Создание программы, выводящей собственный стек вызова»}
\end{center}
\vfill
\hfill\textbox{
    Выполнил: ст. группы ПВ-223\\Пахомов Владислав Андреевич
    \bigbreak
    Проверил: Осипов Олег Васильевич
}
\vfill\begin{center}
    Белгород 2024 г.
\end{center}
\newpage

\renewcommand{\contentsname}{Оглавление}
\tableofcontents\newpage

\section{Формулировка задачи}
Разработать программное обеспечение, выводящее собственный стек вызова
при помощи Assembler. 

\section{Алгоритм работы программы, используемые библиотеки.}
Будем разрабатывать программное обеспечение для операционной системы Windows.
Для анализа данных процесса будем использовать библиотеку dbghelp.h, входящую в 
состав Win32. SymInitialize инициализирует обработчик символов для процесса, после
чего получим адрес функции, выполняющий трейсбек стека из kernel32.dll под
названием RtlCaptureStackBackTrace, и после чего вызываем её, получая стектрейс - 
массив адресов инструкций.

После этого будем анализировать массив адресов инструкций. Можно получить информацию о 
модуле при помощи GetModuleHandleExA. SymGetSymFromAddr позволяет получить
информацию об имени функции, SymGetLineFromAddr позволяет получить номер строки в 
исходном тексте программы.

\section{Исходная программа.}
\textbf{hw.asm}
\begin{minted}{asm}
.686
.model flat, stdcall
option casemap: none

include windows.inc
include user32.inc
include kernel32.inc
include msvcrt.inc
include dbghelp.inc

includelib msvcrt.lib
includelib user32.lib 
includelib kernel32.lib 
; 32-битную версию dbghelp.lib можно взять из x64dbg/pluginsdk
includelib dbghelp.lib

.data
	kernel32dllname db "kernel32.dll", 0
	RtlCaptureStackBackTracename db "RtlCaptureStackBackTrace", 0
	mainprintlevel db "------- Level %d, address 0x%p ------- ", 13, 10, 0
	printtraceheadingprint db "Defining function name and module at address 0x%p", 13, 10, 0
	newline db 13, 10, 0
	callers db 1024 dup(?)
	printtracemodule db "Module begin address: 0x%p", 13, 10
	db "Module name: %s", 13, 10, 0
	symgetsymfromaddr64failedmsg db "SymGetLineFromAddr64 failed. Exit code: %d", 13, 10, 0
	symgetsymfromaddr64subprogram db "Subprogram name: %s, Instruction offset from function beginning: %u", 13, 10, 0
	symgetlinefromaddrloc db "Subprogram '%s' is located at '%s'", 13, 10
	db "Address: 0x%p", 13, 10
	db "Instruction offset from beginning of line: %d", 13, 10
	db "Line: %d", 13, 10, 0

.code

print_trace proc instructionAddress: DWORD
    invoke crt_printf, offset printtraceheadingprint, instructionAddress

	sub esp, 4

	mov eax, GET_MODULE_HANDLE_EX_FLAG_FROM_ADDRESS
	xor eax, GET_MODULE_HANDLE_EX_FLAG_UNCHANGED_REFCOUNT

	invoke GetModuleHandleExA, eax, instructionAddress, esp

	cmp eax, 0
	je print_trace_modulehandlenotexists
		; char module_name[1024];
		sub esp, 261

		; flags
		mov eax, GET_MODULE_HANDLE_EX_FLAG_FROM_ADDRESS
		xor eax, GET_MODULE_HANDLE_EX_FLAG_UNCHANGED_REFCOUNT
		
		mov edx, esp
		add edx, 261
		mov edx, [edx]

		mov ecx, esp

		; GetModuleFileNameA(hModule, module_name, MAX_PATH);
		invoke GetModuleFileNameA, edx, ecx, 260

		mov ecx, esp
		add ecx, 261
		mov ecx, [ecx]

		invoke crt_printf, offset printtracemodule, ecx, esp

		add esp, 261

		; DWORD disp = 0;
		sub esp, 4
		mov dword ptr [esp], 0

		;struct
		;{
		;	IMAGEHLP_SYMBOL64 symbolInfo = { };
		;	char name_buffer[1024];

		;} SYMBOL_DATA;
		sub esp, 1048

		;SYMBOL_DATA.symbolInfo.SizeOfStruct = sizeof(IMAGEHLP_SYMBOL64)
		mov dword ptr [esp], 24
		;SYMBOL_DATA.symbolInfo.MaxNameLength = 1024
		mov dword ptr [esp + 16], 1024

		; eax = CurrentProcess()
		invoke GetCurrentProcess
		; ecx = &disp
		mov ecx, esp
		add ecx, 1048
		
		invoke SymGetSymFromAddr, eax, instructionAddress, ecx, esp

		cmp eax, 0
		je print_trace_symgetsymfromaddr64failed
			; ecx = symbolInfo.name
			mov ecx, esp
			add ecx, 20

			; edx = disp
			mov edx, esp
			add edx, 1048
			mov edx, [edx]

			invoke crt_printf, offset symgetsymfromaddr64subprogram, ecx, edx	

			sub esp, 20
			mov dword ptr [esp], 20
			sub esp, 4

			; ecx - ссылка на displacement
			mov ecx, esp
			
			; edx - ссылка на line
			mov edx, esp
			add edx, 4

			invoke GetCurrentProcess
			invoke SymGetLineFromAddr, eax, instructionAddress, ecx, edx

			cmp eax, 0
			je print_trace_symgetlinefromaddr64failed
				; eax = адрес SYMBOL_DATA.symbolInfo.Name
				mov eax, esp
				add eax, 24
				add eax, 20

				; ecx = ссылка на line.FileName
				mov ecx, esp
				add ecx, 4
				add ecx, 12

				; edx = SYMBOL_DATA.symbolInfo.Address
				mov edx, esp
				add edx, 24
				add edx, 20
				add edx, 4

				; ebx = displacement
				mov ebx, esp
				mov ebx, [ebx]

				; esi = line.LineNumber
				mov esi, esp
				add esi, 4
				add esi, 8
				mov esi, [esi]

				invoke crt_printf, eax, ecx, edx, ebx, esi
				jmp print_trace_symgetlinefromaddr64failedend
print_trace_symgetlinefromaddr64failed:
				invoke GetLastError
				invoke crt_printf, offset symgetsymfromaddr64failedmsg, eax
print_trace_symgetlinefromaddr64failedend:

			add esp, 24

			jmp print_trace_symgetsymfromaddr64failedend

print_trace_symgetsymfromaddr64failed:
			invoke GetLastError
			invoke crt_printf, offset symgetsymfromaddr64failedmsg, eax
print_trace_symgetsymfromaddr64failedend:

		add esp, 1052


print_trace_modulehandlenotexists:
	add esp, 4
	ret
print_trace endp

start: 
	; SymInitialize(GetCurrentProcess(), NULL, TRUE);
	invoke GetCurrentProcess
	invoke SymInitialize, eax, NULL, TRUE

	; CaptureStackBackTraceType pCaptureStackBackTraceType = (CaptureStackBackTraceType)(GetProcAddress(LoadLibraryA("kernel32.dll"), "RtlCaptureStackBackTrace"));
	invoke LoadLibraryA, offset kernel32dllname
	invoke GetProcAddress, eax, offset RtlCaptureStackBackTracename


	cmp eax, 0
	jne pCaptureStackBackTraceTypeNotNull
		; 
		push 1
		call ExitProcess

pCaptureStackBackTraceTypeNotNull:

	; pCaptureStackBackTraceType(0, 1024, callers, NULL);
	push NULL
	push offset callers
	push 1024
	push 0
	call eax
    
	; Сохраняем в стек count
	push eax
	; i = 0
	mov ecx, 0

callersloop:
        ; i >= count? 
		cmp ecx, [esp]
		; Выход из цикла
		jge endcallersloop

		mov ebp, dword ptr callers[ecx * 4]

		; printf("------- Level %d, address 0x%p ------- \n", i + 1, callers[i]); 
		push ecx
		invoke crt_printf, offset mainprintlevel, ecx, ebp
		pop ecx

		mov ebp, dword ptr callers[ecx * 4]
		push ecx
		invoke print_trace, ebp
		pop ecx

		push ecx
		invoke crt_printf, offset newline
		pop ecx

		inc ecx
		jmp callersloop

endcallersloop:

	call crt__getch 	; Задержка ввода, getch()
	; Вызов функции ExitProcess(0)
	push 0		; Поместить аргумент функции в стек
	call ExitProcess 	; Выход из программы
end start
\end{minted}

\section{Результаты выполнения программы.}
\begin{minted}{console}
Insert here please!
\end{minted}

\section{Заключение}
Разработали на Assembler при помощи библиотек Windows
программу, позволяющуую отследить стектрейс вызова текущей программы.

\end{document}