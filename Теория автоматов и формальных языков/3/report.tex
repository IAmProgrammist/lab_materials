\documentclass[a4paper,14pt]{extarticle}


\usepackage[english,russian]{babel}
\usepackage[T2A]{fontenc}
\usepackage[utf8]{inputenc}
\usepackage{ragged2e}
\usepackage[utf8]{inputenc}
\usepackage{hyperref}
\usepackage{minted}
\setmintedinline{frame=lines, framesep=2mm, baselinestretch=1.5, bgcolor=LightGray, breaklines,fontsize=\scriptsize}
\setminted{frame=lines, framesep=2mm, baselinestretch=1.5, bgcolor=LightGray, breaklines,fontsize=\scriptsize}
\usepackage{xcolor}
\definecolor{LightGray}{gray}{0.9}
\usepackage{graphicx}
\usepackage[export]{adjustbox}
\usepackage[left=1cm,right=1cm, top=1cm,bottom=1cm,bindingoffset=0cm]{geometry}
\usepackage{fontspec}
\usepackage{ upgreek }
\usepackage[shortlabels]{enumitem}
\usepackage{adjustbox}
\usepackage{multirow}
\usepackage{amsmath}
\usepackage{amssymb}
\usepackage{pifont}
\usepackage{pgfplots}
\usepackage{longtable}
\usepackage{array}
\graphicspath{ {./images/} }
\makeatletter
\AddEnumerateCounter{\asbuk}{\russian@alph}{щ}
\makeatother
\setmonofont{Ubuntu Mono}
\setmainfont{Times New Roman}

\newcommand\textbox[1]{
	\parbox{.45\textwidth}{#1}
} 

\newcommand{\specialcell}[2][c]{%
	\begin{tabular}[#1]{@{}c@{}}#2\end{tabular}}

\begin{document}
\pagenumbering{gobble}
\begin{center}
    \small{
        \textbf{МИНИCТЕРCТВО НАУКИ И ВЫCШЕГО ОБРАЗОВАНИЯ РОCCИЙCКОЙ ФЕДЕРАЦИИ}\\
        ФЕДЕРАЛЬНОЕ ГОCУДАРCТВЕННОЕ БЮДЖЕТНОЕ ОБРАЗОВАТЕЛЬНОЕ УЧРЕЖДЕНИЕ\\ВЫCШЕГО ОБРАЗОВАНИЯ \\
        \textbf{«БЕЛГОРОДCКИЙ ГОCУДАРCТВЕННЫЙ ТЕХНОЛОГИЧЕCКИЙ\\УНИВЕРCИТЕТ им. В. Г. ШУХОВА»\\ (БГТУ им. В.Г. Шухова)} \\
        \bigbreak
        \includegraphics[width=70mm]{log}\\
        ИНСТИТУТ ИНФОРМАЦИОННЫХ ТЕХНОЛОГИЙ И УПРАВЛЯЮЩИХ СИСТЕМ\\}
\end{center}

\vfill
\begin{center}
    \large{
        \textbf{
            Лабораторная работа №3}}\\
    \normalsize{
        по дисциплине: Теория автоматов и формальных языков \\
        тема: «Регулярные языки и конечные распознаватели»}
\end{center}
\vfill
\hfill\textbox{
    Выполнил: ст. группы ПВ-223\\Пахомов Владислав Андреевич
    \bigbreak
    Проверили: \\ст. пр. Рязанов Юрий Дмитриевич
}
\vfill\begin{center}
    Белгород 2024 г.
\end{center}
\newpage
\begin{center}
    \textbf{Лабораторная работа №3}\\
    Регулярные языки и конечные распознаватели\\
    Вариант 8
\end{center}
\textbf{Цель работы: }изучить основные способы задания регулярных языков, 
способы построения, алгоритмы преобразования, анализа и реализации конечных 
распознавателей.\\

\begin{enumerate}[1.]
    \item Язык $L_1$ в алфавите $\{0,1\}$, представляющий собой множество цепочек, в которых на предпоследнем месте стоит единица, задан грамматикой:\\
    S → A10\\
    S → A11\\
    A → 0A\\
    A → 1A\\
    A → ε\\
    Построить детерминированный конечный распознаватель языка $L_1$.\bigbreak
    Преобразуем заданную грамматику к автоматной правосторонней. Сейчас она является КС-грамматикой.\\
    Приведём грамматику и устраним левую рекурсию.\\
    Лишних символов в грамматике нет.\\
    В грамматике есть ε-правило. Исключим его.\\
    S → A10\\
    S → 10\\
    S → A11\\
    S → 11\\
    A → 0A\\
    A → 0\\
    A → 1A\\
    A → 1\\
    Цепных правил в грамматике нет.\\
    Левой рекурсии в грамматике также нет.\\
    Грамматика приведена, а также в ней нет левой рекурсии.\\ 
    Преобразуем грамматику к такому виду, что каждое 
    правило будет начинаться с терминала:\\ 

    \iffalse

S → A10    S  A  _
S → 10     S  _  _
S → A11    S  A  _
S → 11     S  _  _
A → 0A     A  _  A
A → 0      A  _  _
A → 1A     A  _  A
A → 1      A  _  _

S → A10    S  A  _
S → A11    S  A  _
A → 0A     A  _  A
A → 1A     A  _  A
S → 10     S  _  _
S → 11     S  _  _
A → 0      A  _  _
A → 1      A  _  _

    \fi

S → 0A10\\
S → 1A10\\
S → 010\\
S → 110\\
S → 0A11\\
S → 1A11\\
S → 011\\
S → 111\\
A → 0A\\
A → 1A\\
A → 0\\
A → 1\\
S → 10\\
S → 11\\

    Преобразуем КС-грамматику к правосторонней:\\
    \includegraphics[width=140mm]{task1}\\
S → 0$N_1$\\
S → 1$N_1$\\
S → 010\\
S → 110\\
S → 0$N_2$\\
S → 1$N_2$\\
S → 011\\
S → 111\\
A → 0A\\
A → 1A\\
A → 0\\
A → 1\\
S → 10\\
S → 11\\
$N_1$ → 0$N_1$\\
$N_1$ → 1$N_1$\\
$N_1$ → 010\\
$N_1$ → 110\\
$N_2$ → 0$N_2$\\
$N_2$ → 1$N_2$\\
$N_2$ → 011\\
$N_2$ → 111\\

Исключим лишние символы:\\
S → 0$N_1$\\
S → 1$N_1$\\
S → 010\\
S → 110\\
S → 0$N_2$\\
S → 1$N_2$\\
S → 011\\
S → 111\\
S → 10\\
S → 11\\
$N_1$ → 0$N_1$\\
$N_1$ → 1$N_1$\\
$N_1$ → 010\\
$N_1$ → 110\\
$N_2$ → 0$N_2$\\
$N_2$ → 1$N_2$\\
$N_2$ → 011\\
$N_2$ → 111\\

    Получили правостороннюю грамматику. Теперь преобразуем её к автоматной правосторонней грамматике. 
    Введём нетерминалы: $N_3$ → 1, $N_4$ → 0, $N_5$ → 1$N_3$, $N_6$ → 1$N_4$, $N_7$ → ε и выполним замену там, где это требуется:\\ 

S → 0$N_1$\\
S → 1$N_1$\\
S → 0$N_6$\\
S → 1$N_6$\\
S → 0$N_2$\\
S → 1$N_2$\\
S → 0$N_5$\\
S → 1$N_5$\\
S → 1$N_4$\\
S → 1$N_3$\\
$N_1$ → 0$N_1$\\
$N_1$ → 1$N_1$\\
$N_1$ → 0$N_6$\\
$N_1$ → 1$N_6$\\
$N_2$ → 0$N_2$\\
$N_2$ → 1$N_2$\\
$N_2$ → 0$N_5$\\
$N_2$ → 1$N_5$\\
$N_3$ → 1$N_7$\\
$N_4$ → 0$N_7$\\
$N_5$ → 1$N_3$\\
$N_6$ → 1$N_4$\\
$N_7$ → ε\\

Теперь можем построить распознаватель по КС-грамматике:\\

\begin{tabular}{|c|c|c|c|c|c|c|c|c|}
\hline
& $\downarrow$ &  &&&&&&1\\
\hline
  & S & $N_1$ & $N_2$ & $N_3$ & $N_4$ & $N_5$ & $N_6$ & $N_7$ \\
  \hline
1 & $N_1$, $N_2$, $N_3$, $N_4$, $N_5$, $N_6$ & $N_1$, $N_6$ & $N_2$, $N_5$ & $N_7$ & & $N_3$ & $N_4$ & \\
\hline
0 & $N_1$, $N_2$, $N_5$, $N_6$ & $N_1$, $N_6$ & $N_2$, $N_5$ & & $N_7$ & & & \\
\hline

\end{tabular}\\
\includegraphics[width=100mm]{task1_non_determined}

Распознаватель недетерминированный, преобразуем его к детерминированному.\\
ε-переходов в распознаввателе нет.\\
Преобразуем недетерминированный конечный распознаватель в детерминированный:\\
\begin{tabular}{|c|c|c|c|c|c|}
    \hline
    & & &&&\\
    \hline
    & \{S\} &                                                                           \begin{tabular}{c}\{$N_1$, $N_2$,\\$N_3$, $N_4$,\\$N_5$, $N_6$\}\end{tabular}         & \begin{tabular}{c}\{$N_1$, $N_2$,\\$N_5$, $N_6$\}\end{tabular}                & \begin{tabular}{c}\{$N_1$, $N_2$,\\$N_3$, $N_4$,\\$N_5$, $N_6$,\\$N_7$\}\end{tabular} & \begin{tabular}{c}\{$N_1$, $N_2$,\\$N_5$, $N_6$,\\$N_7$\}\end{tabular} \\
    \hline
    1 & \begin{tabular}{c}\{$N_1$, $N_2$,\\$N_3$, $N_4$,\\$N_5$, $N_6$\}\end{tabular} & \begin{tabular}{c}\{$N_1$, $N_2$,\\$N_3$, $N_4$,\\$N_5$, $N_6$,\\$N_7$\}\end{tabular} & \begin{tabular}{c}\{$N_1$, $N_2$,\\$N_3$, $N_4$,\\$N_5$, $N_6$\}\end{tabular} & \begin{tabular}{c}\{$N_1$, $N_2$,\\$N_3$, $N_4$,\\$N_5$, $N_6$,\\$N_7$\}\end{tabular} & \begin{tabular}{c}\{$N_1$, $N_2$,\\$N_3$, $N_4$,\\$N_5$, $N_6$\}\end{tabular} \\
    \hline
    0 & \begin{tabular}{c}\{$N_1$, $N_2$,\\$N_5$, $N_6$\}\end{tabular} &                \begin{tabular}{c}\{$N_1$, $N_2$,\\$N_5$, $N_6$,\\$N_7$\}\end{tabular}                & \begin{tabular}{c}\{$N_1$, $N_2$,\\$N_5$, $N_6$\}\end{tabular}                & \begin{tabular}{c}\{$N_1$, $N_2$,\\$N_5$, $N_6$,\\$N_7$\}\end{tabular}                & \begin{tabular}{c}\{$N_1$, $N_2$,\\$N_5$, $N_6$\}\end{tabular} \\
    \hline
\end{tabular}

\begin{tabular}{|c|c|c|c|c|c|}
    \hline
    & $\downarrow$ & & & 1 & 1 \\
    \hline
    & S1 & S2 & S3 & S4 & S5 \\
    \hline
1   & S2 & S4 & S2 & S4 & S2  \\
\hline
0   & S3 & S5 & S3 & S5 & S3  \\
\hline
\end{tabular}

\includegraphics[width=100mm]{task1_determined}

Построили детерминированный конечный распознаватель языка $L_1$.

\item Язык $L_2$ в алфавите $\{0,1\}$, представляющий собой множество цепочек, в
которых на последнем месте стоит единица, задан регулярным выражением:\\
(0+1)*1\\
Построить детерминированный конечный распознаватель языка $L_2$.\\
Для начала построим конечный недетерминированный распознаватель языка:\\
\includegraphics[width=70mm]{task2_non_determined}\\
Данный распознаватель языка не является детерминированным, 
так как он содержит ε-переходы. 
Преобразуем данный конечный распознаватель языка в детерминированный:\\
\begin{tabular}{|c|c|c|c|c|c|}
	\hline
	   & $\downarrow$ &    &    & 1  \\
	\hline
	   & S1           & S2 & S3 & S4 \\
	\hline
	1  &              & S2 & S4 &    \\
	\hline
	0  &              & S2 &    &    \\
	\hline
	ε  & S2           & S3 &    &    \\
	\hline
\end{tabular}

Удалим ε-переходы:\\ 
ε-замыкания: ε(S1) = \{S1, S2, S3\}, ε(S2) = \{S2, S3\}, ε(S3) = \{S3\}, ε(S4) = \{S4\}\\
\begin{tabular}{|c|c|c|c|c|}
    \hline
    & $\downarrow$ & $\downarrow$ & & 1\\
    \hline
    & \begin{tabular}{c}ε(S1)\\\{S1, S2, S3\}\end{tabular} & \begin{tabular}{c}ε(S2)\\\{S2, S3\}\end{tabular} & \begin{tabular}{c}ε(S3)\\\{S3\}\end{tabular} & \begin{tabular}{c}ε(S4)\\\{S4\}\end{tabular} \\
    \hline
    1 & ε(S2), ε(S4) & ε(S2), ε(S4) & ε(S4) & \\
    \hline
    0 & ε(S2) & ε(S2) & & \\
    \hline
\end{tabular} 

\begin{tabular}{|c|c|c|c|c|}
    \hline
    & $\downarrow$ & $\downarrow$ & & 1\\
    \hline
    & S1 & S2 & S3 & S4 \\
    \hline
    1 & S2, S3, S4 & S2, S3, S4 & S4 & \\
    \hline
    0 & S2, S3 & S2, S3 & & \\
    \hline
\end{tabular} 

\includegraphics[width=70mm]{task2_no_e_routes}\\

Преобразуем недетерминированный конечный распознаватель в детерминированный:\\
\begin{tabular}{|c|c|c|c|}
    \hline
    & & &\\
    \hline
    & \{S1, S2\} & \{S2, S3\} & \{S2, S3, S4\}\\
    \hline
    1 & \{S2, S3, S4\} & \{S2, S3, S4\} & \{S2, S3, S4\}\\
    \hline
    0 & \{S2, S3\} & \{S2, S3\} & \{S2, S3\}\\
    \hline
\end{tabular} 

Обозначим множества состояний как S'1, S'2, S'3...\\
S'1 обозначим как начальное состояние, солгасно алгоритму, а 
S'3 обозначим как допускающее состояние, так как множество $\{S2, S3, S4\}$
включает в себя допускающее состояние S4.

\begin{tabular}{|c|c|c|c|}
    \hline
    & $\downarrow$ & & 1\\
    \hline
    & S'1 & S'2 & S'3\\
    \hline
    1 & S'3 & S'3 & S'3\\
    \hline
    0 & S'2 & S'2 & S'2\\
    \hline
\end{tabular} 

\includegraphics[width=70mm]{task2_determined}\\

Построили детерминированный конечный распознаватель языка $L_2$.
\item Построить минимальный детерминированный конечный распознаватель
языка $L_3$ в алфавите \{0,1\}, представляющий собой множество цепочек, в
которых хотя бы на одной из последних двух позиций стоит единица.

Можно отметить, что язык $L_1$ в алфавите \{0, 1\} содержит цепочки, которые оканчиваются на 10 или 11.
Язык не эквивалентен $L_3$, так как он не учитывает цепочки, которые содержат только 1 или оканчиваются на 01.  
А язык $L_2$ в алфавите \{0, 1\} содержит цепочки, которые содержат на последней позиции 1 в том числе и 
цепочку 1, а значит учитывает ещё и цепочки, оканчивающиеся на 01. 
Язык не эквивалентен $L_3$, так как он не учитывает цепочки, которые заканчиваются на 10.
Однако если объединим языки, можем получить язык, который содержат все необходимые из обоих языков цепочки, 
а значит и получим искомый язык $L_3$.\\


\begin{tabular}{|c|c|c|c|c|c|c|c|c|}
    \hline
& $\downarrow$ &  &  & 1 & 1 & $\downarrow$ & & 1\\
\hline
& S1 & S2 & S3 & S4 & S5 & S'1 & S'2 & S'3\\
\hline
1 & S2 & S4 & S2 & S4 & S2 & S'3 & S'3 & S'3 \\
\hline
0 & S3 & S5 & S3 & S5 & S3 & S'2 & S'2 & S'2 \\
\hline
\end{tabular}\\
\includegraphics[width=120mm]{task3_new_non_determined}

Преобразуем недетерминированный распознаватель в детерминированный:\\

\begin{tabular}{|c|c|c|c|c|c|}
    \hline
& &  & &&\\
\hline
  & \{S1, S'1\} & \{S2, S'3\} & \{S4, S'3\} & \{S5, S'2\} & \{S3, S'2\}\\
\hline
1 & \{S2, S'3\} & \{S4, S'3\} & \{S4, S'3\} & \{S2, S'3\} & \{S2, S'3\}\\
\hline
0 & \{S3, S'2\} & \{S5, S'2\} & \{S5, S'2\} & \{S3, S'2\} & \{S3, S'2\}\\
\hline
\end{tabular}\\

\begin{tabular}{|c|c|c|c|c|c|}
    \hline
& $\downarrow$ & 1&1&1&\\
\hline
  & S1 & S2 & S3 & S4 & S5\\
\hline
1 & S2 & S3 & S3 & S2 & S2\\
\hline
0 & S5 & S4 & S4 & S5 & S5\\
\hline
\end{tabular}\\
\includegraphics[width=100mm]{task3_new_determined}

Полученный распознаватель является детерминированным, однако является ли он 
минимальным?\\
В распознавателе нет состояний, недостижимых из начального.\\
Перейдём к поиску и исключению эквивалентных состояний:\\
\begin{tabular}{|c|c|c|c|c|c|c|c|}
    \hline
    & $\downarrow$ & 1&1&1&&\\
    \hline
      & S1 & S2 & S3 & S4 & S5&\\
    \hline
    1 & S2 & S3 & S3 & S2 & S2&\\
    \hline
    0 & S5 & S4 & S4 & S5 & S5&\\
    \hline
\end{tabular}

Отвергающие состояния \{S1, S5\} объединим в класс K1 0-эквивалентных состояний, 
а допускающие состояния \{S2, S3, S4\} - в класс K2.

\begin{tabular}{|c|c|c|c|c|c|c|}
    \hline
    & \multicolumn{3}{c|}{K1} & \multicolumn{3}{c|}{K2} \\
    \hline
      & S1 & S5 &    & S2 & S3 & S4 \\
    \hline
    1 & K2 & K2 & K1 & K2 & K2 & K2 \\
    \hline
    0 & K1 & K1 & K1 & K2 & K2 & K1 \\
    \hline
\end{tabular}

Получили таблицу переходов в классы 0-эквивалентных состояний. На основе 
этой таблицы можем построить таблицу переходов в классы 1-эквивалентных 
состояний. 

\begin{tabular}{|c|c|c|c|c|c|c|}
    \hline
    & \multicolumn{2}{c|}{K1} & K2 & \multicolumn{2}{c|}{K3} & K4 \\
    \hline
      & S1 & S5 &    & S2 & S3 & S4 \\
    \hline
    1 & K3 & K3 & K2 & K3 & K3 & K3 \\
    \hline
    0 & K1 & K1 & K2 & K4 & K4 & K1 \\
    \hline
\end{tabular}

По таблице видно, что классы 2-эквивалентных состояний совпадают с
классами 1-эквивалентных состояний, следовательно, классы 1-эквивалентных 
состояний представляют собой классы эквивалентных состояний.\\

Таблица переходов минимального распознавателя:\\
\begin{tabular}{|c|c|c|c|c|}
\hline
  & $\downarrow$ & & 1 & 1 \\
\hline
  & K1 & K2 & K3 & K4\\
\hline
1 & K3 & K2 & K3 & K3\\
\hline
0 & K1 & K2 & K4 & K1\\
\hline

\end{tabular}\\

Переобозначим K1 как S1, K3 как S2, K4 как S3:\\

\begin{tabular}{|c|c|c|c|}
\hline
    & $\downarrow$ & 1 & 1 \\
\hline
  & S1 & S2 & S3\\
\hline
1 & S2 & S2 & S2\\
\hline
0 & S1 & S3 & S1\\
\hline

\end{tabular}\\
\includegraphics[width=70mm]{task3_new_determined_minimal}

Построили минимальный детерминированный конечный распознаватель языка $L_3$.

Сравним полученнный распознаватель с распознавателем, полученным в прошлой версии 
лабораторной работы:\\
\includegraphics[width=70mm]{task3_determined}\\
Во-первых, исходная версия предполагала обязательное наличие хотя бы двух
символов в цепочках, следовательно распознаватели будут формировать 
разные языки, такое различие возникло из-за неверной
интерпретации словесного описания языка.\\
Во-вторых, в новой версии гораздо меньше состояний и переходов, что 
положительно влияет на производительность, понижает сложность отладки и 
сокращает набор тестовых данных.\\

\item Написать программу компиляционного типа для реализации
минимального детерминированного конечного распознавателя языка $L_3$.

\begin{minted}{python3}
MESSAGES = {
    -1: "Отвергнуть, невалидный входной символ",
    0:  "Отвергнуть, цепочка не содержит 1 на одной из двух последних позиций",
    1:  "Допустить",
    2:  "Допустить"
}


def l3_validator(l3_input):
    original_input = l3_input
    s = 0
    while len(l3_input) > 0 and s >= 0:
        current_symbol = l3_input[0]
        if s == 0:
            if current_symbol == "1":
                s = 1
            elif current_symbol == "0":
                s = 0
            else:
                s = -1
                break
        elif s == 1:
            if current_symbol == "1":
                s = 1
            elif current_symbol == "0":
                s = 2
            else:
                s = -1
                break
        elif s == 2:
            if current_symbol == "1":
                s = 1
            elif current_symbol == "0":
                s = 0
            else:
                s = -1
                break

        l3_input = l3_input[1:]

    print(original_input, MESSAGES[s])
    return s
\end{minted}

\item Написать программу интерпретационного типа для реализации
минимального детерминированного конечного распознавателя языка $L_3$.

\begin{minted}{python3}
MESSAGES = {
    -1: "Отвергнуть, невалидный входной символ",
    0: "Отвергнуть, цепочка не содержит 1 на одной из двух последних позиций",
    1: "Допустить",
}

PERMITTING = [1, 2]

MATRIX = {
    "1": [1, 1, 1],
    "0": [0, 2, 0]
}


def l3_validator(l3_input):
    original_input = l3_input
    s = 0
    while len(l3_input) > 0 and s >= 0:
        current_symbol = l3_input[0]
        if current_symbol in MATRIX:
            s = MATRIX[l3_input[0]][s]
        else:
            s = -1
            break

        l3_input = l3_input[1:]

    if s in PERMITTING:
        s = 1

    print(original_input, MESSAGES[s])
    return s
\end{minted}

\item Подобрать наборы тестовых данных так, чтобы в процессе тестирования
сработал каждый переход конечного распознавателя.

\begin{enumerate}
    \item 0110100 - использование всех переходов
    \item 21 - переход в ошибку
\end{enumerate}

Тесты для компиляционного варианта программы:\\
\begin{minted}{python3}
# Тестовые данные для всех переходов
assert l3_validator("0110100") == 0
assert l3_validator("21") == -1
\end{minted}

Тесты для интерпретационного варианта программы:\\
\begin{minted}{python3}
# Тестовые данные для всех переходов
assert l3_validator("0110100") == 0
assert l3_validator("21") == -1
\end{minted}

\item Подобрать наборы тестовых данных так, чтобы в процессе тестирования
распознаватель закончил обработку цепочек в каждом состоянии конечного
распознавателя.

\begin{enumerate}
    \item <пустая строка> - состояние S1
    \item 1 - состояние S2
    \item 10 - состояние S3
    \item 21 - состояние ошибки
\end{enumerate}

Тесты для компиляционного варианта программы:\\
\begin{minted}{python3}
# Тестовые данные для всех состояний
assert l3_validator("") == 0
assert l3_validator("1") == 1
assert l3_validator("10") == 2
assert l3_validator("21") == -1
\end{minted}

Тесты для интерпретационного варианта программы:\\
\begin{minted}{python3}
# Тестовые данные для всех состояний
assert l3_validator("") == 0
assert l3_validator("1") == 1
assert l3_validator("10") == 1
assert l3_validator("21") == -1
\end{minted}

\item Выполнить тестирование программ для реализации минимального
детерминированного конечного распознавателя языка $L_3$.\\

Результаты выполнения компиляционного варианта программы:\\
\begin{minted}{console}
0110100 Отвергнуть, цепочка не содержит 1 на одной из двух последних позиций
21 Отвергнуть, невалидный входной символ
 Отвергнуть, цепочка не содержит 1 на одной из двух последних позиций
1 Допустить
10 Допустить
21 Отвергнуть, невалидный входной символ

Process finished with exit code 0
\end{minted}

Результаты выполнения интерпретационного варианта программы:\\
\begin{minted}{console}
0110100 Отвергнуть, цепочка не содержит 1 на одной из двух последних позиций
21 Отвергнуть, невалидный входной символ
 Отвергнуть, цепочка не содержит 1 на одной из двух последних позиций
1 Допустить
10 Допустить
21 Отвергнуть, невалидный входной символ

Process finished with exit code 0
\end{minted}

Оба варианта программы завершились без ошибок, а значит проверки 
в проверках истинные, следовательно программа написана верно.

\end{enumerate}

\textbf{Вывод: } в ходе лабораторной работы изучили основные способы задания регулярных языков, 
способы построения, алгоритмы преобразования, анализа и реализации конечных 
распознавателей.

\end{document}