\documentclass[a4paper,14pt]{extarticle}


\usepackage[english,russian]{babel}
\usepackage[T2A]{fontenc}
\usepackage[utf8]{inputenc}
\usepackage{ragged2e}
\usepackage[utf8]{inputenc}
\usepackage{hyperref}
\usepackage{minted}
\setmintedinline{frame=lines, framesep=2mm, baselinestretch=1.5, bgcolor=LightGray, breaklines,fontsize=\scriptsize}
\setminted{frame=lines, framesep=2mm, baselinestretch=1.5, bgcolor=LightGray, breaklines,fontsize=\scriptsize}
\usepackage{xcolor}
\definecolor{LightGray}{gray}{0.9}
\usepackage{graphicx}
\usepackage[export]{adjustbox}
\usepackage[left=1cm,right=1cm, top=1cm,bottom=1cm,bindingoffset=0cm]{geometry}
\usepackage{fontspec}
\usepackage{ upgreek }
\usepackage[shortlabels]{enumitem}
\usepackage{adjustbox}
\usepackage{multirow}
\usepackage{amsmath}
\usepackage{amssymb}
\usepackage{pifont}
\usepackage{pgfplots}
\usepackage{longtable}
\usepackage{array}
\graphicspath{ {./images/} }
\makeatletter
\AddEnumerateCounter{\asbuk}{\russian@alph}{щ}
\makeatother
\setmonofont{Consolas}
\setmainfont{Times New Roman}

\newcommand\textbox[1]{
	\parbox{.45\textwidth}{#1}
} 

\newcommand{\specialcell}[2][c]{%
	\begin{tabular}[#1]{@{}c@{}}#2\end{tabular}}

\begin{document}
\pagenumbering{gobble}
\begin{center}
    \small{
        \textbf{МИНИCТЕРCТВО НАУКИ И ВЫCШЕГО ОБРАЗОВАНИЯ РОCCИЙCКОЙ ФЕДЕРАЦИИ}\\
        ФЕДЕРАЛЬНОЕ ГОCУДАРCТВЕННОЕ БЮДЖЕТНОЕ ОБРАЗОВАТЕЛЬНОЕ УЧРЕЖДЕНИЕ\\ВЫCШЕГО ОБРАЗОВАНИЯ \\
        \textbf{«БЕЛГОРОДCКИЙ ГОCУДАРCТВЕННЫЙ ТЕХНОЛОГИЧЕCКИЙ\\УНИВЕРCИТЕТ им. В. Г. ШУХОВА»\\ (БГТУ им. В.Г. Шухова)} \\
        \bigbreak
        \includegraphics[width=70mm]{log}\\
        ИНСТИТУТ ИНФОРМАЦИОННЫХ ТЕХНОЛОГИЙ И УПРАВЛЯЮЩИХ СИСТЕМ\\}
\end{center}

\vfill
\begin{center}
    \large{
        \textbf{
            Лабораторная работа №3}}\\
    \normalsize{
        по дисциплине: Теория автоматов и формальных языков \\
        тема: «Регулярные языки и конечные распознаватели»}
\end{center}
\vfill
\hfill\textbox{
    Выполнил: ст. группы ПВ-223\\Пахомов Владислав Андреевич
    \bigbreak
    Проверили: \\ст. пр. Рязанов Юрий Дмитриевич
}
\vfill\begin{center}
    Белгород 2024 г.
\end{center}
\newpage
\begin{center}
    \textbf{Лабораторная работа №3}\\
    Регулярные языки и конечные распознаватели\\
    Вариант 8
\end{center}
\textbf{Цель работы: }изучить основные способы задания регулярных языков, 
способы построения, алгоритмы преобразования, анализа и реализации конечных 
распознавателей.\\
\textbf{Задание: }язык вещественных констант.
Вещественная константа может состоять из мантиссы и порядка,
разделённых символом E . Мантисса может состоять из целой и дробной 
частей, разделённых точкой. Целая часть мантиссы и порядок —
целые знаковые десятичные числа (знак может отсутствовать). Если
отсутствует порядок и символ E , то могут отсутствовать целая или
дробная часть мантиссы, но не обе сразу. Если присутствует порядок и
символ E , то может отсутствовать дробная часть и точка.

\begin{enumerate}[1.]
    \item Построить минимальный детерминированный конечный распознаватель 
    заданного языка (см.варианты заданий).\\
\end{enumerate}

\textbf{Вывод: } в ходе лабораторной работы изучили основные способы задания регулярных языков, 
способы построения, алгоритмы преобразования, анализа и реализации конечных 
распознавателей.

\end{document}