\documentclass[a4paper,14pt]{extarticle}


\usepackage[english,russian]{babel}
\usepackage[T2A]{fontenc}
\usepackage[utf8]{inputenc}
\usepackage{ragged2e}
\usepackage[utf8]{inputenc}
\usepackage{hyperref}
\usepackage{minted}
\setmintedinline{frame=lines, framesep=2mm, baselinestretch=1.5, bgcolor=LightGray, breaklines,fontsize=\scriptsize}
\setminted{frame=lines, framesep=2mm, baselinestretch=1.5, bgcolor=LightGray, breaklines,fontsize=\scriptsize}
\usepackage{xcolor}
\definecolor{LightGray}{gray}{0.9}
\usepackage{graphicx}
\usepackage[export]{adjustbox}
\usepackage[left=1cm,right=1cm, top=1cm,bottom=1cm,bindingoffset=0cm]{geometry}
\usepackage{fontspec}
\usepackage{ upgreek }
\usepackage[shortlabels]{enumitem}
\usepackage{adjustbox}
\usepackage{multirow}
\usepackage{amsmath}
\usepackage{amssymb}
\usepackage{pifont}
\usepackage{pgfplots}
\usepackage{longtable}
\usepackage{array}
\graphicspath{ {./images/} }
\makeatletter
\AddEnumerateCounter{\asbuk}{\russian@alph}{щ}
\makeatother
\setmonofont{Consolas}
\setmainfont{Times New Roman}

\newcommand\textbox[1]{
	\parbox{.45\textwidth}{#1}
} 

\newcommand{\specialcell}[2][c]{%
	\begin{tabular}[#1]{@{}c@{}}#2\end{tabular}}

\begin{document}
\pagenumbering{gobble}
\begin{center}
    \small{
        \textbf{МИНИCТЕРCТВО НАУКИ И ВЫCШЕГО ОБРАЗОВАНИЯ РОCCИЙCКОЙ ФЕДЕРАЦИИ}\\
        ФЕДЕРАЛЬНОЕ ГОCУДАРCТВЕННОЕ БЮДЖЕТНОЕ ОБРАЗОВАТЕЛЬНОЕ УЧРЕЖДЕНИЕ\\ВЫCШЕГО ОБРАЗОВАНИЯ \\
        \textbf{«БЕЛГОРОДCКИЙ ГОCУДАРCТВЕННЫЙ ТЕХНОЛОГИЧЕCКИЙ\\УНИВЕРCИТЕТ им. В. Г. ШУХОВА»\\ (БГТУ им. В.Г. Шухова)} \\
        \bigbreak
        \includegraphics[width=70mm]{log}\\
        ИНСТИТУТ ИНФОРМАЦИОННЫХ ТЕХНОЛОГИЙ И УПРАВЛЯЮЩИХ СИСТЕМ\\}
\end{center}

\vfill
\begin{center}
    \large{
        \textbf{
            Лабораторная работа №3}}\\
    \normalsize{
        по дисциплине: Теория автоматов и формальных языков \\
        тема: «Регулярные языки и конечные распознаватели»}
\end{center}
\vfill
\hfill\textbox{
    Выполнил: ст. группы ПВ-223\\Пахомов Владислав Андреевич
    \bigbreak
    Проверили: \\ст. пр. Рязанов Юрий Дмитриевич
}
\vfill\begin{center}
    Белгород 2024 г.
\end{center}
\newpage
\begin{center}
    \textbf{Лабораторная работа №3}\\
    Регулярные языки и конечные распознаватели\\
    Вариант 8
\end{center}
\textbf{Цель работы: }изучить основные способы задания регулярных языков, 
способы построения, алгоритмы преобразования, анализа и реализации конечных 
распознавателей.\\

\begin{enumerate}[1.]
    \item Язык $L_1$ в алфавите $\{0,1\}$, представляющий собой множество цепочек, в которых на предпослежнем месте стоит единица, задан грамматикой:\\
    S → A10\\
    A → A011\\
    A → 0A\\
    A → 1A\\
    A → ε\\
    Построить детерминированный конечный распознаватель языка $L_1$.\bigbreak
    Преобразуем заданную грамматику к автоматной правосторонней. Сейчас она является КС-грамматикой.\\
    Приведём грамматику и устраним левую рекурсию.\\
    Лишних символов в грамматике нет.\\
    В грамматике есть ε-правило. Исключим его.\\
    S → A10\\
    S → 10\\
    A → A011\\
    A → 011\\
    A → 0A\\
    A → 0\\
    A → 1A\\
    A → 1\\
    Цепных правил в грамматике нет.\\
    В грамматике есть левая рекурсия. Исключим её.\\
    S → A10\\
    S → 10\\
    A → 011B\\
    A → 0AB\\
    A → 0B\\
    A → 1AB\\
    A → 1B\\
    A → 011\\
    A → 0A\\
    A → 0\\
    A → 1A\\
    A → 1\\
    B → 011B\\
    B → ε\\
    В грамматике есть ε-правило. Исключим его.\\
    S → A10\\
    S → 10\\
    A → 011\\
    A → 011B\\
    A → 0A\\
    A → 0AB\\
    A → 0\\
    A → 0B\\
    A → 1A\\
    A → 1AB\\
    A → 1\\
    A → 1B\\
    A → 011\\
    A → 0A\\
    A → 0\\
    A → 1A\\
    A → 1\\
    B → 011B\\
    B → 011\\
    Исключим правила-дубликаты:\\
    S → A10\\
    S → 10\\
    A → 011\\
    A → 011B\\
    A → 0A\\
    A → 0AB\\
    A → 0\\
    A → 0B\\
    A → 1A\\
    A → 1AB\\
    A → 1\\
    A → 1B\\
    B → 011B\\
    B → 011\\
    Грамматика приведена, а также в ней нет левой рекурсии. 
    Преобразуем грамматику к такому виду, что каждое 
    правило будет начинаться с терминала:\\ 

    S → 01110\\
    S → 011B10\\
    S → 0A10\\
    S → 0AB10\\
    S → 010\\
    S → 0B10\\
    S → 1A10\\
    S → 1AB10\\
    S → 110\\
    S → 1B10\\
    S → 10\\
    A → 011\\
    A → 011B\\
    A → 0A\\
    A → 0AB\\
    A → 0\\
    A → 0B\\
    A → 1A\\
    A → 1AB\\
    A → 1\\
    A → 1B\\
    B → 011B\\
    B → 011\\

    Преобразуем КС-грамматику к правосторонней:\\
    \includegraphics[width=140mm]{task1fail}\\
    Преобразовать грамматику к правосторонней невозможно, так как 
    в ходе преобразований получили правило (подчёркнутое с !!! в вычислениях) $N_4$ → AB$N_1$. 
    С правилом $N_3$ → A$N_1$ они имеют общий префикс, а значит мы получим рекурсию, и 
    следовательно правостороннюю грамматику с конечным числом правил получить нельзя. 
    Задание невыполнимо.
\item Язык $L_2$ в алфавите $\{0,1\}$, представляющий собой множество цепочек, в
которых на последнем месте стоит единица, задан регулярным выражением:\\
(0+1)*1\\
Построить детерминированный конечный распознаватель языка $L_2$.\\
Для начала построим конечный недетерминированный распознаватель языка:\\
\includegraphics[width=70mm]{task2_non_determined}\\
Преобразуем данный конечный распознаватель языка в детерминированный:\\

\end{enumerate}

\textbf{Вывод: } в ходе лабораторной работы изучили основные способы задания регулярных языков, 
способы построения, алгоритмы преобразования, анализа и реализации конечных 
распознавателей.

\end{document}