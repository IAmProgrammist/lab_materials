\documentclass[a4paper,14pt]{extarticle}


\usepackage[english,russian]{babel}
\usepackage[T2A]{fontenc}
\usepackage[utf8]{inputenc}
\usepackage{ragged2e}
\usepackage[utf8]{inputenc}
\usepackage{hyperref}
\usepackage{minted}
\setmintedinline{frame=lines, framesep=2mm, baselinestretch=1.5, bgcolor=LightGray, breaklines,fontsize=\scriptsize}
\setminted{frame=lines, framesep=2mm, baselinestretch=1.5, bgcolor=LightGray, breaklines,fontsize=\scriptsize}
\usepackage{xcolor}
\definecolor{LightGray}{gray}{0.9}
\usepackage{graphicx}
\usepackage[export]{adjustbox}
\usepackage[left=1cm,right=1cm, top=1cm,bottom=1cm,bindingoffset=0cm]{geometry}
\usepackage{fontspec}
\usepackage{ upgreek }
\usepackage[shortlabels]{enumitem}
\usepackage{adjustbox}
\usepackage{multirow}
\usepackage{amsmath}
\usepackage{amssymb}
\usepackage{pifont}
\usepackage{pgfplots}
\usepackage{longtable}
\usepackage{array}
\graphicspath{ {./images/} }
\makeatletter
\AddEnumerateCounter{\asbuk}{\russian@alph}{щ}
\makeatother
\setmonofont{Consolas}
\setmainfont{Times New Roman}

\newcommand\textbox[1]{
	\parbox{.45\textwidth}{#1}
} 

\newcommand{\specialcell}[2][c]{%
	\begin{tabular}[#1]{@{}c@{}}#2\end{tabular}}

\begin{document}
\pagenumbering{gobble}

Преобразовать КС-грамматику, определяющую язык знаковых вещественных
констант, в которых может отсутствовать дробная или цеая часть,
но не обе сразу, в правостороннюю.\\


S → +В\\
S → -В\\
S → В\\
В → Ц.Ц\\
В → Ц.\\
В → .Ц\\
Ц → цЦ\\
Ц → ц\\

Преобразуем грамматику к такому виду, что каждое правило будет начинаться с 
терминала:\\
В грамматике есть цепное правило, исключим его:\\

S → +В\\   
S → -В\\
S → Ц.Ц\\
S → Ц.\\
S → .Ц\\
В → Ц.Ц\\
В → Ц.\\
В → .Ц\\
Ц → цЦ\\
Ц → ц\\

Лишних символов в грамматике нет. e-правил также нет. Грамматика приведена.
Левой рекурсии в грамматике также нет.\\

\iffalse
S → +В\\   S _ В
S → -В\\   S _ В
S → Ц.Ц\\  S Ц Ц
S → Ц.\\   S Ц _
S → .Ц\\   S _ Ц
В → Ц.Ц\\  В Ц Ц
В → Ц.\\   В Ц _
В → .Ц\\   В _ Ц
Ц → цЦ\\   Ц _ Ц
Ц → ц\\    Ц _ _



S _ Ц
В Ц Ц
S Ц Ц
В Ц _
S Ц _
Ц _ Ц
S _ В
S _ В
В _ Ц
Ц _ _


S → .Ц
В → Ц.Ц
S → Ц.Ц
В → Ц.
S → Ц.
Ц → цЦ
S → +В
S → -В
В → .Ц
Ц → ц

S → .Ц
В → цЦ.Ц
В → ц.Ц
S → цЦ.Ц
S → ц.Ц
В → цЦ.
В → ц.
S → цЦ.
S → ц.
Ц → цЦ
S → +В
S → -В
В → .Ц
Ц → ц
\fi

Привели грамматику к виду, где правила начинаются с терминала:\\

S → .Ц\\
В → цЦ.Ц\\
В → ц.Ц\\
S → цЦ.Ц\\
S → ц.Ц\\
В → цЦ.\\
В → ц.\\
S → цЦ.\\
S → ц.\\
Ц → цЦ\\
S → +В\\
S → -В\\
В → .Ц\\
Ц → ц\\

Преобразуем грамматику к правосторонней и выполним замену:\\

\includegraphics[width=140mm]{answer_dec}

Искомая грамматика $G'$:\\
S → .Ц\\
В → ц$N_1$\\
В → ц.Ц\\
S → ц$N_1$\\
S → ц.Ц\\
В → ц$N_2$\\
В → ц.\\
S → ц$N_2$\\
S → ц.\\
Ц → цЦ\\
S → +В\\
S → -В\\
В → .Ц\\
Ц → ц\\
$N_1$ → ц$N_1$\\
$N_1$ → ц.Ц\\
$N_2$ → ц$N_2$\\
$N_2$ → ц.\\



\end{document}