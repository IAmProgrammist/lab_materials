\documentclass[a4paper,14pt]{extarticle}


\usepackage[english,russian]{babel}
\usepackage[T2A]{fontenc}
\usepackage[utf8]{inputenc}
\usepackage{ragged2e}
\usepackage[utf8]{inputenc}
\usepackage{hyperref}
\usepackage{minted}
\setmintedinline{frame=lines, framesep=2mm, baselinestretch=1.5, bgcolor=LightGray, breaklines,fontsize=\scriptsize}
\setminted{frame=lines, framesep=2mm, baselinestretch=1.5, bgcolor=LightGray, breaklines,fontsize=\scriptsize}
\usepackage{xcolor}
\definecolor{LightGray}{gray}{0.9}
\usepackage{graphicx}
\usepackage[export]{adjustbox}
\usepackage[left=1cm,right=1cm, top=1cm,bottom=1cm,bindingoffset=0cm]{geometry}
\usepackage{fontspec}
\usepackage{ upgreek }
\usepackage[shortlabels]{enumitem}
\usepackage{adjustbox}
\usepackage{multirow}
\usepackage{amsmath}
\usepackage{amssymb}
\usepackage{pifont}
\usepackage{pgfplots}
\usepackage{longtable}
\usepackage{array}
\graphicspath{ {./images/} }
\makeatletter
\AddEnumerateCounter{\asbuk}{\russian@alph}{щ}
\makeatother
\setmonofont{Ubuntu Mono}
\setmainfont{Times New Roman}

\newcommand\textbox[1]{
	\parbox{.45\textwidth}{#1}
} 

\newcommand{\specialcell}[2][c]{%
	\begin{tabular}[#1]{@{}c@{}}#2\end{tabular}}

\newcommand\mcc[1]{\multicolumn{1}{c}{#1}} % <--- 

\begin{document}
\pagenumbering{gobble}
\begin{center}
    \small{
        \textbf{МИНИCТЕРCТВО НАУКИ И ВЫCШЕГО ОБРАЗОВАНИЯ РОCCИЙCКОЙ ФЕДЕРАЦИИ}\\
        ФЕДЕРАЛЬНОЕ ГОCУДАРCТВЕННОЕ БЮДЖЕТНОЕ ОБРАЗОВАТЕЛЬНОЕ УЧРЕЖДЕНИЕ\\ВЫCШЕГО ОБРАЗОВАНИЯ \\
        \textbf{«БЕЛГОРОДCКИЙ ГОCУДАРCТВЕННЫЙ ТЕХНОЛОГИЧЕCКИЙ\\УНИВЕРCИТЕТ им. В. Г. ШУХОВА»\\ (БГТУ им. В.Г. Шухова)} \\
        \bigbreak
        \includegraphics[width=70mm]{log}\\
        ИНСТИТУТ ИНФОРМАЦИОННЫХ ТЕХНОЛОГИЙ И УПРАВЛЯЮЩИХ СИСТЕМ\\}
\end{center}

\vfill
\begin{center}
    \large{
        \textbf{
            Лабораторная работа №4}}\\
    \normalsize{
        по дисциплине: Теория автоматов и формальных языков \\
        тема: «Нисходящая обработка контекстно-свободных языков»}
\end{center}
\vfill
\hfill\textbox{
    Выполнил: ст. группы ПВ-223\\Пахомов Владислав Андреевич
    \bigbreak
    Проверили: \\ст. пр. Рязанов Юрий Дмитриевич
}
\vfill\begin{center}
    Белгород 2024 г.
\end{center}
\newpage
\begin{center}
    \textbf{Лабораторная работа №3}\\
    Регулярные языки и конечные распознаватели\\
    Вариант 8
\end{center}
\textbf{Цель работы: }изучить и научиться применять нисходящие методы обработки формальных языков.\\

\begin{enumerate}[1.]
    \item Преобразовать исходную КС-грамматику в LL(1)-грамматику.\\
    1. S→O;S\\
    2. S→O;\\
    3. O→a[S]\\
    4. O→a[S][S]\\
    5. O→a=E\\
    6. E→E+T\\
    7. E→T\\
    8. T→T*P\\
    9. T→P\\
    10. P→(E)\\
    11. P→-(E)\\
    12. P→a\\

    Устраним левую рекурсию в грамматике:\\

    Удалим самолеворекурсивное правило E→E+T\\
    S→O;S\\
    S→O;\\
    O→a[S]\\
    O→a[S][S]\\
    O→a=E\\
    E→T$L_1$\\
    $L_1$→+T$L_1$\\
    $L_1$→ε\\
    T→T*P\\
    T→P\\
    P→(E)\\
    P→-(E)\\
    P→a\\

    Удалим самолеворекурсивное правило T→T*P\\
    S→O;S\\
    S→O;\\
    O→a[S]\\
    O→a[S][S]\\
    O→a=E\\
    E→T$L_1$\\
    T→P$L_2$\\
    P→(E)\\
    P→-(E)\\
    P→a\\
    $L_1$→+T$L_1$\\
    $L_1$→ε\\
    $L_2$→*P$L_2$\\
    $L_2$→ε\\

    В грамматике есть правила с общими префиксами, выполним левую факторизацию.\\
    S→O;$S_1$\\
    O→a$S_2$\\
    E→T$L_1$\\
    T→P$L_2$\\
    P→(E)\\
    P→-(E)\\
    P→a\\
    $L_1$→+T$L_1$\\
    $L_1$→ε\\
    $L_2$→*P$L_2$\\
    $L_2$→ε\\
    $S_1$→S\\
    $S_1$→ε\\
    $S_2$→[S]$S_3$\\
    $S_2$→=E\\
    $S_3$→[S]\\
    $S_3$→ε\\

    В грамматике нет правил с одинаковой левой частью, из правых частей которых выводятся цепочки, имеющие общий префикс.\\

    Определим ВЫБОР для правил кс-грамматики:\\
    \begin{tabular}{|c|c|}
    \hline
    Правило        & ВЫБОР\\
    \hline
    S→O;$S_1$      & a     \\
    \hline
    O→a$S_2$       & a     \\
    \hline
    E→T$L_1$       & ( - a \\
    \hline
    T→P$L_2$       & ( - a \\
    \hline
    P→(E)          & (     \\
    \hline
    P→-(E)         & -     \\
    \hline
    P→a            & a     \\
    \hline
    $L_1$→+T$L_1$  & +     \\
    \hline
    $L_1$→ε        & ) ;   \\
    \hline
    $L_2$→*P$L_2$  & *     \\
    \hline
    $L_2$→ε        & + ; ) \\
    \hline
    $S_1$→S        & a     \\
    \hline
    $S_1$→ε        & ┤ ]   \\
    \hline
    $S_2$→[S]$S_3$ & [     \\
    \hline
    $S_2$→=E       & =     \\
    \hline
    $S_3$→[S]      & [     \\
    \hline
    $S_3$→ε        & ;     \\
    \hline
    \end{tabular}\\

    \begin{tabular}{|c|c|c|}
        \hline
        Символ  & ПЕРВ  & СЛЕД\\
        \hline
        S       & a     & ┤ ] \\
        \hline
        O       & a     & ;\\
        \hline
        E       & ( - a & ) ;\\
        \hline
        T       & ( - a & + ; )\\
        \hline
        P       & ( - a & * + ; )\\
        \hline
        $L_1$   & + ε   & ) ;\\
        \hline
        $L_2$   & * ε   & + ; )\\
        \hline
        $S_1$   & a ε   & ┤ ] \\
        \hline
        $S_2$   & [ =   & ;\\
        \hline
        $S_3$   & [ ε   & ;\\
        \hline
    \end{tabular}\\

    Так как пары правил с одинаковой левой частью и непустым пересечений множеств
    ВЫБОР нет, можно сказать, что грамматика преобразована к LL(1)-грамматике. 
    Искомая грамматика:\\
    1. S→O;$S_1$\\
    2. O→a$S_2$\\
    3. E→T$L_1$\\
    4. T→P$L_2$\\
    5. P→(E)\\
    6. P→-(E)\\
    7. P→a\\
    8. $L_1$→+T$L_1$\\
    9. $L_1$→ε\\
    10. $L_2$→*P$L_2$\\
    11. $L_2$→ε\\
    12. $S_1$→S\\
    13. $S_1$→ε\\
    14. $S_2$→[S]$S_3$\\
    15. $S_2$→=E\\
    16. $S_3$→[S]\\
    17. $S_3$→ε\\

    \item Определить множества ПЕРВЫХ для каждого символа LL(1)-грамматики.\\
    
    \begin{tabular}{|c|c|}
        \hline
        Символ  & ПЕРВ  \\
        \hline
        S       & a     \\
        \hline
        O       & a     \\
        \hline
        E       & ( - a \\
        \hline
        T       & ( - a \\
        \hline
        P       & ( - a \\
        \hline
        $L_1$   & + ε   \\
        \hline
        $L_2$   & * ε   \\
        \hline
        $S_1$   & a ε   \\
        \hline
        $S_2$   & [ =   \\
        \hline
        $S_3$   & [ ε   \\
        \hline
    \end{tabular}

    \item Определить множества СЛЕДУЮЩИХ для каждого символа LL(1)-грамматики.
    \begin{tabular}{|c|c|}
        \hline
        Символ  & СЛЕД  \\
        \hline
        S       & ┤ ] \\
        \hline
        O       & ;\\
        \hline
        E       & ) ;\\
        \hline
        T       & + ; )\\
        \hline
        P       & * + ; )\\
        \hline
        $L_1$   & ) ;\\
        \hline
        $L_2$   & + ; )\\
        \hline
        $S_1$   & ┤ ] \\
        \hline
        $S_2$   & ;\\
        \hline
        $S_3$   & ;\\
        \hline
    \end{tabular}
    \item Определить множество ВЫБОРА для каждого правила LL(1)-грамматики.
    
    \begin{tabular}{|c|c|}
        \hline
        Правило        & ВЫБОР\\
        \hline
        S→O;$S_1$      & a     \\
        \hline
        O→a$S_2$       & a     \\
        \hline
        E→T$L_1$       & ( - a \\
        \hline
        T→P$L_2$       & ( - a \\
        \hline
        P→(E)          & (     \\
        \hline
        P→-(E)         & -     \\
        \hline
        P→a            & a     \\
        \hline
        $L_1$→+T$L_1$  & +     \\
        \hline
        $L_1$→ε        & ) ;   \\
        \hline
        $L_2$→*P$L_2$  & *     \\
        \hline
        $L_2$→ε        & + ; ) \\
        \hline
        $S_1$→S        & a     \\
        \hline
        $S_1$→ε        & ┤ ]   \\
        \hline
        $S_2$→[S]$S_3$ & [     \\
        \hline
        $S_2$→=E       & =     \\
        \hline
        $S_3$→[S]      & [     \\
        \hline
        $S_3$→ε        & ;     \\
        \hline
    \end{tabular}\\

    \item Написать программу-распознаватель методом рекурсивного
    спуска. Программа должна выводить последовательность номеров правил, 
    применяемых при левом выводе обрабатываемой цепочки.\\
    Переобозначим символы в грамматике:\\
    1. S→O;C \{a\}\\
    2. O→aD \{a\}\\
    3. E→TA \{(, -, a\}\\
    4. T→PB \{(, -, a\}\\
    5. P→(E) \{(\}\\
    6. P→-(E) \{-\}\\
    7. P→a \{a\}\\
    8. A→+TA \{+\}\\
    9. A→ε \{), ;\}\\
    10. B→*PB \{*\}\\
    11. B→ε \{+, ;, )\}\\
    12. C→S \{a\}\\
    13. C→ε \{┤, ]\}\\
    14. D→[S]F \{[\}\\
    15. D→=E \{=\}\\
    16. F→[S] \{[\}\\
    17. F→ε \{;\}\\
    \begin{minted}{python3}
InvalidStringError = ValueError("Invalid string detected")


def process_terminal(data, terminal):
    if data[0] == terminal:
        data = data[1:]
    else:
        raise InvalidStringError

    return data


def S(data: str) -> str:
    if data[0] in ["a"]:
        print("    1. S -> O;C")
        data = O(data)
        data = process_terminal(data, ";")
        data = C(data)
    else:
        raise InvalidStringError

    return data


def O(data: str) -> str:
    if data[0] in ["a"]:
        print("    2. O -> aD")
        data = process_terminal(data, "a")
        data = D(data)
    else:
        raise InvalidStringError

    return data


def E(data: str) -> str:
    if data[0] in ["(", "-", "a"]:
        print("    3. E -> TA")
        data = T(data)
        data = A(data)
    else:
        raise InvalidStringError

    return data


def T(data: str) -> str:
    if data[0] in ["(", "-", "a"]:
        print("    4. T -> PB")
        data = P(data)
        data = B(data)
    else:
        raise InvalidStringError

    return data


def P(data: str) -> str:
    if data[0] in ["("]:
        print("    5. P -> (E)")
        data = process_terminal(data, "(")
        data = E(data)
        data = process_terminal(data, ")")
    elif data[0] in ["-"]:
        print("    6. P -> -(E)")
        data = process_terminal(data, "-")
        data = process_terminal(data, "(")
        data = E(data)
        data = process_terminal(data, ")")
    elif data[0] in ["a"]:
        print("    7. P -> a")
        data = process_terminal(data, "a")
    else:
        raise InvalidStringError

    return data


def A(data: str) -> str:
    if data[0] in ["+"]:
        print("    8. A -> +TA")
        data = process_terminal(data, "+")
        data = T(data)
        data = A(data)
    elif data[0] in [")", ";"]:
        print("    9. A -> ε")
        pass
    else:
        raise InvalidStringError

    return data


def B(data: str) -> str:
    if data[0] in ["*"]:
        print("    10. B -> *PB")
        data = process_terminal(data, "*")
        data = P(data)
        data = B(data)
    elif data[0] in ["+", ";", ")"]:
        print("    11. B -> ε")
        pass
    else:
        raise InvalidStringError

    return data


def C(data: str) -> str:
    if data[0] in ["a"]:
        print("    12. C -> S")
        data = S(data)
    elif data[0] in ["┤", "]"]:
        print("    13. C -> ε")
        pass
    else:
        raise InvalidStringError

    return data


def D(data: str) -> str:
    if data[0] in ["["]:
        print("    14. D -> [S]F")
        data = process_terminal(data, "[")
        data = S(data)
        data = process_terminal(data, "]")
        data = F(data)
    elif data[0] in ["="]:
        print("    15. D -> =E")
        data = process_terminal(data, "=")
        data = E(data)
    else:
        raise InvalidStringError

    return data


def F(data: str) -> str:
    if data[0] in ["["]:
        print("    16. F -> [S]")
        data = process_terminal(data, "[")
        data = S(data)
        data = process_terminal(data, "]")
    elif data[0] in [";"]:
        print("    17. F -> ε")
        pass
    else:
        raise InvalidStringError

    return data


def LL1(data):
    print(f"Трейсбек применяемых правил для {data}:")
    data += "┤"
    try:
        result = S(data)
    except ValueError:
        print("Цепочка невалидна")
        raise InvalidStringError

    if result == "┤":
        print("Цепочка валидна")
        return True
    else:
        print("Цепочка невалидна")
        raise InvalidStringError
    \end{minted}
    \item Сформировать наборы тестовых данных. Тестовые данные должны 
    содержать цепочки, принадлежащие языку, заданному грамматикой, 
    (допустимые цепочки) и цепочки, не принадлежащие языку. Для
    каждой допустимой цепочки построить дерево вывода и левый вывод.
    Каждое правило грамматики должно использоваться в выводах допустимых 
    цепочек хотя бы один раз.

    Тестовые данные:
    \begin{itemize}
        \item \textbf{a[a=(a)*-(a)+a+-(a);];}\\
        Валидная цепочка\bigbreak
        Левый вывод:\\
        $S_{1} \Rightarrow O_{2};C \Rightarrow aD_{14};C \Rightarrow a[S_{1}]F;C \Rightarrow a[O_{2};C]F;C \Rightarrow 
        a[aD_{15};C]F;C \Rightarrow a[a=E_{3};C]F;C \Rightarrow a[a=T_{4}A;C]F;C \Rightarrow 
        a[a=P_{5}BA;C]F;C \Rightarrow a[a=(E_{3})BA;C]F;C \Rightarrow a[a=(T_{4}A)BA;C]F;C \Rightarrow 
        a[a=(P_{7}BA)BA;C]F;C \Rightarrow a[a=(aB_{11}A)BA;C]F;C \Rightarrow 
        a[a=(aA_{9})BA;C]F;C \Rightarrow a[a=(a)B_{10}A;C]F;C \Rightarrow 
        a[a=(a)*P_{6}BA;C]F;C \Rightarrow a[a=(a)*-(E_{3})BA;C]F;C \Rightarrow
        a[a=(a)*-(T_{4}A)BA;C]F;C \Rightarrow a[a=(a)*-(P_{7}BA)BA;C]F;C \Rightarrow 
        a[a=(a)*-(aB_{11}A)BA;C]F;C \Rightarrow a[a=(a)*-(aA_{9})BA;C]F;C \Rightarrow
        a[a=(a)*-(a)B_{11}A;C]F;C \Rightarrow a[a=(a)*-(a)A_{8};C]F;C \Rightarrow
        a[a=(a)*-(a)+T_{4}A;C]F;C \Rightarrow a[a=(a)*-(a)+P_{7}BA;C]F;C \Rightarrow
        a[a=(a)*-(a)+aB_{11}A;C]F;C \Rightarrow a[a=(a)*-(a)+aA_{9};C]F;C \Rightarrow
        a[a=(a)*-(a)+a;C_{13}]F;C \Rightarrow a[a=(a)*-(a)+a;]F_{17};C \Rightarrow
        a[a=(a)*-(a)+a;];C_{13} \Rightarrow a[a=(a)*-(a)+a;];$
    \bigbreak
    Дерево вывода:\\
    \includegraphics[width=140mm]{task_6_1}

    \item \textbf{a[a=a;][a=a;];}\\
        Валидная цепочка\bigbreak
        Левый вывод:\\
        $S_{1} \Rightarrow O_{2};C \Rightarrow aD_{14};C \Rightarrow a[S_{1}]F;C \Rightarrow a[O_{2};C]F;C \Rightarrow 
        a[aD_{15};C]F;C \Rightarrow a[a=E_{3};C]F;C \Rightarrow a[a=T_{4}A;C]F;C \Rightarrow 
        a[a=P_{7}BA;C]F;C \Rightarrow a[a=aB_{11}A;C]F;C \Rightarrow a[a=aA_{9};C]F;C \Rightarrow a[a=a;C_{13}]F;C \Rightarrow 
        a[a=a;]F_{16};C \Rightarrow a[a=a;][S_{1}];C \Rightarrow a[a=a;][O_{2};C];C \Rightarrow a[a=a;][aD;C];C \Rightarrow
        a[a=a;][aD_{15};C];C \Rightarrow a[a=a;][a=E_{3};C];C \Rightarrow a[a=a;][a=T_{4}A;C];C \Rightarrow 
        a[a=a;][a=P_{7}BA;C];C \Rightarrow a[a=a;][a=aB_{11}A;C];C \Rightarrow a[a=a;][a=aA_{9};C];C \Rightarrow
        a[a=a;][a=a;C_{13}];C \Rightarrow a[a=a;][a=a;];C_{13} \Rightarrow a[a=a;][a=a;];$
    \bigbreak
    Дерево вывода:\\
    \includegraphics[width=100mm]{task_6_2}
    
    \item \textbf{a=a;a=a;}\\
        Валидная цепочка\bigbreak
        Левый вывод:\\
        $S_{1} \Rightarrow O_{2};C \Rightarrow aD_{15};C \Rightarrow a=E_{3};C \Rightarrow
        a=T_{4}A;C \Rightarrow a=P_{7}BA;C \Rightarrow a=aB_{11}A;C \Rightarrow a=aA_{9};C \Rightarrow
        a=a;C_{12} \Rightarrow a=a;S_{1} \Rightarrow a=a;O_{2};C \Rightarrow a=a;aD_{15};C \Rightarrow
        a=a;a=E;C \Rightarrow a=a;a=E_{3};C \Rightarrow a=a;a=T_{4}A;C \Rightarrow
        a=a;a=P_{7}BA;C \Rightarrow a=a;a=aB_{11}A;C \Rightarrow a=a;a=aA_{9};C \Rightarrow a=a;a=a;C_{13} \Rightarrow a=a;a=a;$
    \bigbreak
    Дерево вывода:\\
    \includegraphics[width=100mm]{task_6_3}
    \item \textbf{a[a=(a)*-(a)+a+-(a);]}\\
    Невалидная цепочка
    \item \textbf{a=[];}\\
    Невалидная цепочка
    \item \textbf{a=[a=a;][a=a;][a=a;][a=a;];}\\
    Невалидная цепочка
    \end{itemize}

    \item Обработать цепочки из набора тестовых данных (см. п.6) программой-распознавателем.
    \begin{minted}{python3}
if __name__ == "__main__":
    assert LL1("a[a=(a)*-(a)+a;];")
    assert LL1("a[a=a;][a=a;];")
    assert LL1("a=a;a=a;")
    try:
        LL1("a[a=(a)*-(a)+a+-(a);]")
        assert False
    except ValueError:
        assert True
    try:
        LL1("a=[];")
        assert False
    except ValueError:
        assert True
    try:
        LL1("a=[a=a;][a=a;][a=a;][a=a;];")
        assert False
    except ValueError:
        assert True

    \end{minted}

Результат выполнения программы:\\
\begin{minted}{console}
Трейсбек применяемых правил для a[a=(a)*-(a)+a;];:
    1. S -> O;C
    2. O -> aD
    14. D -> [S]F
    1. S -> O;C
    2. O -> aD
    15. D -> =E
    3. E -> TA
    4. T -> PB
    5. P -> (E)
    3. E -> TA
    4. T -> PB
    7. P -> a
    11. B -> ε
    9. A -> ε
    10. B -> *PB
    6. P -> -(E)
    3. E -> TA
    4. T -> PB
    7. P -> a
    11. B -> ε
    9. A -> ε
    11. B -> ε
    8. A -> +TA
    4. T -> PB
    7. P -> a
    11. B -> ε
    9. A -> ε
    13. C -> ε
    17. F -> ε
    13. C -> ε
Цепочка валидна
Трейсбек применяемых правил для a[a=a;][a=a;];:
    1. S -> O;C
    2. O -> aD
    14. D -> [S]F
    1. S -> O;C
    2. O -> aD
    15. D -> =E
    3. E -> TA
    4. T -> PB
    7. P -> a
    11. B -> ε
    9. A -> ε
    13. C -> ε
    16. F -> [S]
    1. S -> O;C
    2. O -> aD
    15. D -> =E
    3. E -> TA
    4. T -> PB
    7. P -> a
    11. B -> ε
    9. A -> ε
    13. C -> ε
    13. C -> ε
Цепочка валидна
Трейсбек применяемых правил для a=a;a=a;:
    1. S -> O;C
    2. O -> aD
    15. D -> =E
    3. E -> TA
    4. T -> PB
    7. P -> a
    11. B -> ε
    9. A -> ε
    12. C -> S
    1. S -> O;C
    2. O -> aD
    15. D -> =E
    3. E -> TA
    4. T -> PB
    7. P -> a
    11. B -> ε
    9. A -> ε
    13. C -> ε
Цепочка валидна
Трейсбек применяемых правил для a[a=(a)*-(a)+a+-(a);]:
    1. S -> O;C
    2. O -> aD
    14. D -> [S]F
    1. S -> O;C
    2. O -> aD
    15. D -> =E
    3. E -> TA
    4. T -> PB
    5. P -> (E)
    3. E -> TA
    4. T -> PB
    7. P -> a
    11. B -> ε
    9. A -> ε
    10. B -> *PB
    6. P -> -(E)
    3. E -> TA
    4. T -> PB
    7. P -> a
    11. B -> ε
    9. A -> ε
    11. B -> ε
    8. A -> +TA
    4. T -> PB
    7. P -> a
    11. B -> ε
    8. A -> +TA
    4. T -> PB
    6. P -> -(E)
    3. E -> TA
    4. T -> PB
    7. P -> a
    11. B -> ε
    9. A -> ε
    11. B -> ε
    9. A -> ε
    13. C -> ε
Цепочка невалидна
Трейсбек применяемых правил для a=[];:
    1. S -> O;C
    2. O -> aD
    15. D -> =E
Цепочка невалидна
Трейсбек применяемых правил для a=[a=a;][a=a;][a=a;][a=a;];:
    1. S -> O;C
    2. O -> aD
    15. D -> =E
Цепочка невалидна
\end{minted}
\iffalse ; a ( ) - + [ ] = * ┤ \fi
    \item Построить нисходящий МП-распознаватель по LL(1)-грамматике.\\
    \begin{tabular}{c|c|c|c|c|c|c|c|c|c|c|c|}
        \mcc{}                   & \mcc{;} & \mcc{a} & \mcc{(} & \mcc{)} & \mcc{-} & \mcc{+} & \mcc{[} & \mcc{]} & \mcc{=} & \mcc{*} & \mcc{┤} \\
        \cline{2-12} S           &         & \#1     &         &         &         &         &         &         &         &         &         \\
        \cline{2-12} O           &         & \#2     &         &         &         &         &         &         &         &         &         \\
        \cline{2-12} E           &         & \#3     & \#3     &         & \#3     &         &         &         &         &         &         \\
        \cline{2-12} T           &         & \#4     & \#4     &         & \#4     &         &         &         &         &         &         \\
        \cline{2-12} P           &         & \#7     & \#5     &         & \#6     &         &         &         &         &         &         \\
        \cline{2-12} A           & \#9     &         &         & \#9     &         & \#8     &         &         &         &         &         \\
        \cline{2-12} B           & \#9     &         &         & \#9     &         & \#9     &         &         &         & \#10    &         \\
        \cline{2-12} C           &         & \#11    &         &         &         &         &         & \#9     &         &         & \#9     \\
        \cline{2-12} D           &         &         &         &         &         &         & \#12    &         & \#13    &         &         \\
        \cline{2-12} F           & \#9     &         &         &         &         &         & \#14    &         &         &         &         \\
        \cline{2-12} )           &         &         &         & \#7     &         &         &         &         &         &         &         \\
        \cline{2-12} ]           &         &         &         &         &         &         &         & \#7     &         &         &         \\
        \cline{2-12} $\triangle$ &         &         &         &         &         &         &         &         &         &         & ДОПУСТ. \\
        \cline{2-12}
    \end{tabular}\\
    \#1: ЗАМЕНИТЬ(C;O), ДЕРЖАТЬ\\
    \#2: ЗАМЕНИТЬ(D), СДВИГ\\
    \#3: ЗАМЕНИТЬ(AT), ДЕРЖАТЬ\\
    \#4: ЗАМЕНИТЬ(BP), ДЕРЖАТЬ\\
    \#5: ЗАМЕНИТЬ()E), СДВИГ\\
    \#6: ЗАМЕНИТЬ()E(), СДВИГ\\
    \#7: ВЫТОЛКНУТЬ, СДВИГ\\
    \#8: ЗАМЕНИТЬ(AT), СДВИГ\\
    \#9: ВЫТОЛКНУТЬ, ДЕРЖАТЬ\\
    \#10: ЗАМЕНИТЬ(BP), СДВИГ\\
    \#11: ЗАМЕНИТЬ(S), ДЕРЖАТЬ\\
    \#12: ЗАМЕНИТЬ(F]S), СДВИГ\\
    \#13: ЗАМЕНИТЬ(E), СДВИГ\\
    \#14: ЗАМЕНИТЬ(]S), СДВИГ\\
    
    \item Написать программу-распознаватель, реализующую построенный
    нисходящий МП-распознаватель. Программа должна выводить на каждом 
    шаге номер применяемого правила и промежуточную цепочку левого вывода.
    \begin{minted}{python3}
InvalidStringError = ValueError("Invalid string detected")


def hold(data: str) -> str:
    return data


def next(data: str) -> str:
    return data[1:]


def replace(stack: str, value: str) -> str:
    return value[::-1] + stack[1:]


def pop(stack: str) -> str:
    return stack[1:] if stack != "∆" else stack


def print_step(data, stack, origin_data, rule):
    print(f"*********\n"
          f"Текущая цепочка: {origin_data[0:(len(origin_data) - len(data) + 1)]}{stack[:-1]}\n"
          f"Правило: {rule}\n\n")


def shift_reduce_parser(data):
    print(f"Трейсбек применяемых правил для {data}:")
    try:
        origin_data = data
        data += "┤"
        stack = "S∆"
        should_iter = -1
        while should_iter == -1:
            m = stack[0]
            x = data[0]
            if m == "∆" and data == "┤":
                if data == "┤":
                    should_iter = 1
                else:
                    raise InvalidStringError
            elif m == "S":
                stack, data = S(data, stack, origin_data)
            elif m == "O":
                stack, data = O(data, stack, origin_data)
            elif m == "E":
                stack, data = E(data, stack, origin_data)
            elif m == "T":
                stack, data = T(data, stack, origin_data)
            elif m == "P":
                stack, data = P(data, stack, origin_data)
            elif m == "A":
                stack, data = A(data, stack, origin_data)
            elif m == "B":
                stack, data = B(data, stack, origin_data)
            elif m == "C":
                stack, data = C(data, stack, origin_data)
            elif m == "D":
                stack, data = D(data, stack, origin_data)
            elif m == "F":
                stack, data = F(data, stack, origin_data)
            elif m == x:
                (stack, data) = (pop(stack), next(data))
            else:
                raise InvalidStringError
    except ValueError:
        print("Цепочка невалидна")
        raise InvalidStringError

    print("Цепочка валидна")

    return True


def S(data: str, stack: str, origin_data: str) -> (str, str):
    if data[0] in ["a"]:
        print_step(data, stack, origin_data, "    1. S -> O;C")
        # 1
        return replace(stack, "C;O"), hold(data)

    raise InvalidStringError


def O(data: str, stack: str, origin_data: str) -> (str, str):
    if data[0] in ["a"]:
        print_step(data, stack, origin_data, "    2. O -> aD")
        # 2
        return replace(stack, "D"), next(data)

    raise InvalidStringError


def E(data: str, stack: str, origin_data: str) -> (str, str):
    if data[0] in ["(", "-", "a"]:
        print_step(data, stack, origin_data, "    3. E -> TA")
        # 3
        return replace(stack, "AT"), hold(data)

    raise InvalidStringError


def T(data: str, stack: str, origin_data: str) -> (str, str):
    if data[0] in ["(", "-", "a"]:
        print_step(data, stack, origin_data, "    4. T -> PB")
        # 4
        return replace(stack, "BP"), hold(data)

    raise InvalidStringError


def P(data: str, stack: str, origin_data: str) -> (str, str):
    if data[0] in ["("]:
        print_step(data, stack, origin_data, "    5. P -> (E)")
        # 5
        return replace(stack, ")E"), next(data)
    elif data[0] in ["-"]:
        print_step(data, stack, origin_data, "    6. P -> -(E)")
        # 6
        return replace(stack, ")E("), next(data)
    elif data[0] in ["a"]:
        print_step(data, stack, origin_data, "    7. P -> a")
        # 7
        return pop(stack), next(data)

    raise InvalidStringError


def A(data: str, stack: str, origin_data: str) -> (str, str):
    if data[0] in ["+"]:
        print_step(data, stack, origin_data, "    8. A -> +TA")
        # 8
        return replace(stack, "AT"), next(data)
    elif data[0] in [")", ";"]:
        print_step(data, stack, origin_data, "    9. A -> ε")
        # 9
        return pop(stack), hold(data)

    raise InvalidStringError


def B(data: str, stack: str, origin_data: str) -> (str, str):
    if data[0] in ["*"]:
        print_step(data, stack, origin_data, "    10. B -> *PB")
        # 10
        return replace(stack, "BP"), next(data)
    elif data[0] in ["+", ";", ")"]:
        print_step(data, stack, origin_data, "    11. B -> ε")
        # 9
        return pop(stack), hold(data)

    raise InvalidStringError


def C(data: str, stack: str, origin_data: str) -> (str, str):
    if data[0] in ["a"]:
        print_step(data, stack, origin_data, "    12. C -> S")
        # 11
        return replace(stack, "S"), hold(data)
    elif data[0] in ["┤", "]"]:
        print_step(data, stack, origin_data, "    13. C -> ε")
        # 9
        return pop(stack), hold(data)

    raise InvalidStringError


def D(data: str, stack: str, origin_data: str) -> (str, str):
    if data[0] in ["["]:
        print_step(data, stack, origin_data, "    14. D -> [S]F")
        # 12
        return replace(stack, "F]S"), next(data)
    elif data[0] in ["="]:
        print_step(data, stack, origin_data, "    15. D -> =E")
        # 13
        return replace(stack, "E"), next(data)

    raise InvalidStringError


def F(data: str, stack: str, origin_data: str) -> (str, str):
    if data[0] in ["["]:
        print_step(data, stack, origin_data, "    16. F -> [S]")
        # 14
        return replace(stack, "]S"), next(data)
    elif data[0] in [";"]:
        print_step(data, stack, origin_data, "    17. F -> ε")
        # 9
        return pop(stack), hold(data)

    raise InvalidStringError


if __name__ == "__main__":
    assert shift_reduce_parser("a[a=(a)*-(a)+a;];")
    assert shift_reduce_parser("a[a=a;][a=a;];")
    assert shift_reduce_parser("a=a;a=a;")
    try:
        shift_reduce_parser("a[a=(a)*-(a)+a+-(a);]")
        assert False
    except ValueError:
        assert True
    try:
        shift_reduce_parser("a=[];")
        assert False
    except ValueError:
        assert True
    try:
        shift_reduce_parser("a=[a=a;][a=a;][a=a;][a=a;];")
        assert False
    except ValueError:
        assert True

    \end{minted}
    Результат выполнения программы:\\
    \begin{minted}{console}
Трейсбек применяемых правил для a[a=(a)*-(a)+a;];:
*********
Текущая цепочка: S
Правило:     1. S -> O;C


*********
Текущая цепочка: O;C
Правило:     2. O -> aD


*********
Текущая цепочка: aD;C
Правило:     14. D -> [S]F


*********
Текущая цепочка: a[S]F;C
Правило:     1. S -> O;C


*********
Текущая цепочка: a[O;C]F;C
Правило:     2. O -> aD


*********
Текущая цепочка: a[aD;C]F;C
Правило:     15. D -> =E


*********
Текущая цепочка: a[a=E;C]F;C
Правило:     3. E -> TA


*********
Текущая цепочка: a[a=TA;C]F;C
Правило:     4. T -> PB


*********
Текущая цепочка: a[a=PBA;C]F;C
Правило:     5. P -> (E)


*********
Текущая цепочка: a[a=(E)BA;C]F;C
Правило:     3. E -> TA


*********
Текущая цепочка: a[a=(TA)BA;C]F;C
Правило:     4. T -> PB


*********
Текущая цепочка: a[a=(PBA)BA;C]F;C
Правило:     7. P -> a


*********
Текущая цепочка: a[a=(aBA)BA;C]F;C
Правило:     11. B -> ε


*********
Текущая цепочка: a[a=(aA)BA;C]F;C
Правило:     9. A -> ε


*********
Текущая цепочка: a[a=(a)BA;C]F;C
Правило:     10. B -> *PB


*********
Текущая цепочка: a[a=(a)*PBA;C]F;C
Правило:     6. P -> -(E)


*********
Текущая цепочка: a[a=(a)*-(E)BA;C]F;C
Правило:     3. E -> TA


*********
Текущая цепочка: a[a=(a)*-(TA)BA;C]F;C
Правило:     4. T -> PB


*********
Текущая цепочка: a[a=(a)*-(PBA)BA;C]F;C
Правило:     7. P -> a


*********
Текущая цепочка: a[a=(a)*-(aBA)BA;C]F;C
Правило:     11. B -> ε


*********
Текущая цепочка: a[a=(a)*-(aA)BA;C]F;C
Правило:     9. A -> ε


*********
Текущая цепочка: a[a=(a)*-(a)BA;C]F;C
Правило:     11. B -> ε


*********
Текущая цепочка: a[a=(a)*-(a)A;C]F;C
Правило:     8. A -> +TA


*********
Текущая цепочка: a[a=(a)*-(a)+TA;C]F;C
Правило:     4. T -> PB


*********
Текущая цепочка: a[a=(a)*-(a)+PBA;C]F;C
Правило:     7. P -> a


*********
Текущая цепочка: a[a=(a)*-(a)+aBA;C]F;C
Правило:     11. B -> ε


*********
Текущая цепочка: a[a=(a)*-(a)+aA;C]F;C
Правило:     9. A -> ε


*********
Текущая цепочка: a[a=(a)*-(a)+a;C]F;C
Правило:     13. C -> ε


*********
Текущая цепочка: a[a=(a)*-(a)+a;]F;C
Правило:     17. F -> ε


*********
Текущая цепочка: a[a=(a)*-(a)+a;];C
Правило:     13. C -> ε


Цепочка валидна
Трейсбек применяемых правил для a[a=a;][a=a;];:
*********
Текущая цепочка: S
Правило:     1. S -> O;C


*********
Текущая цепочка: O;C
Правило:     2. O -> aD


*********
Текущая цепочка: aD;C
Правило:     14. D -> [S]F


*********
Текущая цепочка: a[S]F;C
Правило:     1. S -> O;C


*********
Текущая цепочка: a[O;C]F;C
Правило:     2. O -> aD


*********
Текущая цепочка: a[aD;C]F;C
Правило:     15. D -> =E


*********
Текущая цепочка: a[a=E;C]F;C
Правило:     3. E -> TA


*********
Текущая цепочка: a[a=TA;C]F;C
Правило:     4. T -> PB


*********
Текущая цепочка: a[a=PBA;C]F;C
Правило:     7. P -> a


*********
Текущая цепочка: a[a=aBA;C]F;C
Правило:     11. B -> ε


*********
Текущая цепочка: a[a=aA;C]F;C
Правило:     9. A -> ε


*********
Текущая цепочка: a[a=a;C]F;C
Правило:     13. C -> ε


*********
Текущая цепочка: a[a=a;]F;C
Правило:     16. F -> [S]


*********
Текущая цепочка: a[a=a;][S];C
Правило:     1. S -> O;C


*********
Текущая цепочка: a[a=a;][O;C];C
Правило:     2. O -> aD


*********
Текущая цепочка: a[a=a;][aD;C];C
Правило:     15. D -> =E


*********
Текущая цепочка: a[a=a;][a=E;C];C
Правило:     3. E -> TA


*********
Текущая цепочка: a[a=a;][a=TA;C];C
Правило:     4. T -> PB


*********
Текущая цепочка: a[a=a;][a=PBA;C];C
Правило:     7. P -> a


*********
Текущая цепочка: a[a=a;][a=aBA;C];C
Правило:     11. B -> ε


*********
Текущая цепочка: a[a=a;][a=aA;C];C
Правило:     9. A -> ε


*********
Текущая цепочка: a[a=a;][a=a;C];C
Правило:     13. C -> ε


*********
Текущая цепочка: a[a=a;][a=a;];C
Правило:     13. C -> ε


Цепочка валидна
Трейсбек применяемых правил для a=a;a=a;:
*********
Текущая цепочка: S
Правило:     1. S -> O;C


*********
Текущая цепочка: O;C
Правило:     2. O -> aD


*********
Текущая цепочка: aD;C
Правило:     15. D -> =E


*********
Текущая цепочка: a=E;C
Правило:     3. E -> TA


*********
Текущая цепочка: a=TA;C
Правило:     4. T -> PB


*********
Текущая цепочка: a=PBA;C
Правило:     7. P -> a


*********
Текущая цепочка: a=aBA;C
Правило:     11. B -> ε


*********
Текущая цепочка: a=aA;C
Правило:     9. A -> ε


*********
Текущая цепочка: a=a;C
Правило:     12. C -> S


*********
Текущая цепочка: a=a;S
Правило:     1. S -> O;C


*********
Текущая цепочка: a=a;O;C
Правило:     2. O -> aD


*********
Текущая цепочка: a=a;aD;C
Правило:     15. D -> =E


*********
Текущая цепочка: a=a;a=E;C
Правило:     3. E -> TA


*********
Текущая цепочка: a=a;a=TA;C
Правило:     4. T -> PB


*********
Текущая цепочка: a=a;a=PBA;C
Правило:     7. P -> a


*********
Текущая цепочка: a=a;a=aBA;C
Правило:     11. B -> ε


*********
Текущая цепочка: a=a;a=aA;C
Правило:     9. A -> ε


*********
Текущая цепочка: a=a;a=a;C
Правило:     13. C -> ε


Цепочка валидна
Трейсбек применяемых правил для a[a=(a)*-(a)+a+-(a);]:
*********
Текущая цепочка: S
Правило:     1. S -> O;C


*********
Текущая цепочка: O;C
Правило:     2. O -> aD


*********
Текущая цепочка: aD;C
Правило:     14. D -> [S]F


*********
Текущая цепочка: a[S]F;C
Правило:     1. S -> O;C


*********
Текущая цепочка: a[O;C]F;C
Правило:     2. O -> aD


*********
Текущая цепочка: a[aD;C]F;C
Правило:     15. D -> =E


*********
Текущая цепочка: a[a=E;C]F;C
Правило:     3. E -> TA


*********
Текущая цепочка: a[a=TA;C]F;C
Правило:     4. T -> PB


*********
Текущая цепочка: a[a=PBA;C]F;C
Правило:     5. P -> (E)


*********
Текущая цепочка: a[a=(E)BA;C]F;C
Правило:     3. E -> TA


*********
Текущая цепочка: a[a=(TA)BA;C]F;C
Правило:     4. T -> PB


*********
Текущая цепочка: a[a=(PBA)BA;C]F;C
Правило:     7. P -> a


*********
Текущая цепочка: a[a=(aBA)BA;C]F;C
Правило:     11. B -> ε


*********
Текущая цепочка: a[a=(aA)BA;C]F;C
Правило:     9. A -> ε


*********
Текущая цепочка: a[a=(a)BA;C]F;C
Правило:     10. B -> *PB


*********
Текущая цепочка: a[a=(a)*PBA;C]F;C
Правило:     6. P -> -(E)


*********
Текущая цепочка: a[a=(a)*-(E)BA;C]F;C
Правило:     3. E -> TA


*********
Текущая цепочка: a[a=(a)*-(TA)BA;C]F;C
Правило:     4. T -> PB


*********
Текущая цепочка: a[a=(a)*-(PBA)BA;C]F;C
Правило:     7. P -> a


*********
Текущая цепочка: a[a=(a)*-(aBA)BA;C]F;C
Правило:     11. B -> ε


*********
Текущая цепочка: a[a=(a)*-(aA)BA;C]F;C
Правило:     9. A -> ε


*********
Текущая цепочка: a[a=(a)*-(a)BA;C]F;C
Правило:     11. B -> ε


*********
Текущая цепочка: a[a=(a)*-(a)A;C]F;C
Правило:     8. A -> +TA


*********
Текущая цепочка: a[a=(a)*-(a)+TA;C]F;C
Правило:     4. T -> PB


*********
Текущая цепочка: a[a=(a)*-(a)+PBA;C]F;C
Правило:     7. P -> a


*********
Текущая цепочка: a[a=(a)*-(a)+aBA;C]F;C
Правило:     11. B -> ε


*********
Текущая цепочка: a[a=(a)*-(a)+aA;C]F;C
Правило:     8. A -> +TA


*********
Текущая цепочка: a[a=(a)*-(a)+a+TA;C]F;C
Правило:     4. T -> PB


*********
Текущая цепочка: a[a=(a)*-(a)+a+PBA;C]F;C
Правило:     6. P -> -(E)


*********
Текущая цепочка: a[a=(a)*-(a)+a+-(E)BA;C]F;C
Правило:     3. E -> TA


*********
Текущая цепочка: a[a=(a)*-(a)+a+-(TA)BA;C]F;C
Правило:     4. T -> PB


*********
Текущая цепочка: a[a=(a)*-(a)+a+-(PBA)BA;C]F;C
Правило:     7. P -> a


*********
Текущая цепочка: a[a=(a)*-(a)+a+-(aBA)BA;C]F;C
Правило:     11. B -> ε


*********
Текущая цепочка: a[a=(a)*-(a)+a+-(aA)BA;C]F;C
Правило:     9. A -> ε


*********
Текущая цепочка: a[a=(a)*-(a)+a+-(a)BA;C]F;C
Правило:     11. B -> ε


*********
Текущая цепочка: a[a=(a)*-(a)+a+-(a)A;C]F;C
Правило:     9. A -> ε


*********
Текущая цепочка: a[a=(a)*-(a)+a+-(a);C]F;C
Правило:     13. C -> ε


Цепочка невалидна
Трейсбек применяемых правил для a=[];:
*********
Текущая цепочка: S
Правило:     1. S -> O;C


*********
Текущая цепочка: O;C
Правило:     2. O -> aD


*********
Текущая цепочка: aD;C
Правило:     15. D -> =E


Цепочка невалидна
Трейсбек применяемых правил для a=[a=a;][a=a;][a=a;][a=a;];:
*********
Текущая цепочка: S
Правило:     1. S -> O;C


*********
Текущая цепочка: O;C
Правило:     2. O -> aD


*********
Текущая цепочка: aD;C
Правило:     15. D -> =E


Цепочка невалидна
    \end{minted}
\end{enumerate}

\textbf{Вывод: } в ходе лабораторной работы изучили и научились применять нисходящие методы обработки формальных языков.

\end{document}