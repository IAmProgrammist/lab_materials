\documentclass[a4paper,14pt]{extarticle}


\usepackage[english,russian]{babel}
\usepackage[T2A]{fontenc}
\usepackage[utf8]{inputenc}
\usepackage{ragged2e}
\usepackage[utf8]{inputenc}
\usepackage{hyperref}
\usepackage{minted}
\setmintedinline{frame=lines, framesep=2mm, baselinestretch=1.5, bgcolor=LightGray, breaklines,fontsize=\scriptsize}
\setminted{frame=lines, framesep=2mm, baselinestretch=1.5, bgcolor=LightGray, breaklines,fontsize=\scriptsize}
\usepackage{xcolor}
\definecolor{LightGray}{gray}{0.9}
\usepackage{graphicx}
\usepackage[export]{adjustbox}
\usepackage[left=1cm,right=1cm, top=1cm,bottom=1cm,bindingoffset=0cm]{geometry}
\usepackage{fontspec}
\usepackage{ upgreek }
\usepackage[shortlabels]{enumitem}
\usepackage{adjustbox}
\usepackage{multirow}
\usepackage{amsmath}
\usepackage{amssymb}
\usepackage{pifont}
\usepackage{pgfplots}
\usepackage{longtable}
\usepackage{array}
\graphicspath{ {./images/} }
\makeatletter
\AddEnumerateCounter{\asbuk}{\russian@alph}{щ}
\makeatother
\setmonofont{Consolas}
\setmainfont{Times New Roman}

\newcommand\textbox[1]{
	\parbox{.45\textwidth}{#1}
} 

\newcommand{\specialcell}[2][c]{%
	\begin{tabular}[#1]{@{}c@{}}#2\end{tabular}}

\begin{document}
\pagenumbering{gobble}
3. Первый способ - можем выполнить замену всех вхождений 
символа B в каждом правиле, заменив его $2^k$ новыми правилами 
при условии что B входило k раз в исходное правило. 
После этого правило $B \rightarrow D$ можно исключить, если в 
грамматике есть правила $D \rightarrow D$, исключить их.
Если в списке правил есть одинаковые правила, исключить их.
Также важно отметить, что нетерминал B не должен быть начальным
нетерминалом. Если он таковым является, то в правиле 
$B \rightarrow D$ можно выполнить замену края. 
Второй способ - замена края. В правиле B -> D будем заменять D на 
правые части правил, у которых в левой части находится D, 
а само правило B -> D удалим.

4. Наличие $\varepsilon$-правил и цепных правил может привести
к ухудшению времени работы алгоритма (в худшем случае 
экспоненциально) и привести к циклам. 
Так например для следующей грамматики с цепными правилами:\\
1. $A \rightarrow B$\\
2. $B \rightarrow A$\\
3. $B \rightarrow b$\\
После выполнения алгоритма придём к грамматике следующего вида:\\
1. $A \rightarrow B$\\
2. $B \rightarrow B$\\
3. $B \rightarrow b$\\

1. $A \rightarrow B$\\
2. $B \rightarrow bB'$\\
3. $B' \rightarrow B'$\\
4. $B' \rightarrow \varepsilon$\\

В итоге после преобразований получили саморекурсивное правило 3.
При наличии $\varepsilon$-правил также рискуем получить правило, подобное $B \rightarrow B$,
которое приведёт к зацикливанию или несоответствию грамматики условию, например:\\
1. $A \rightarrow \varepsilon$\\
2. $B \rightarrow AB$\\
3. $B \rightarrow b$\\

1. $A \rightarrow \varepsilon$\\
2. $B \rightarrow B$\\
3. $B \rightarrow b$\\

1. $A \rightarrow \varepsilon$\\
2. $B \rightarrow bB'$\\
3. $B' \rightarrow B'$\\
4. $B' \rightarrow \varepsilon$\\

Для грамматики $G_1$ можно применить алгоритм, так как подобных случаев возникнуть не может, 
однако для гарантированной работы алгоритма необходимо избавляться от $\varepsilon$- и цепных 
правил.

5. Грамматика неэквивалентна, так как начального нетерминал T нет. Исправил.

6. Были допущены ошибки при выполнении 1 пункта, также была выполнена замена на неодиночный нетерминал, там где требовался одиночный. Исправил.

7. Правила грамматики в НФГ содержат только правила вида $A \rightarrow t\alpha$, где t - терминал, а 
$\alpha$ - цепочка нетерминалов, возможно пустая. Нет. Нет. Исправил.

8. Для правила вида $A \rightarrow B\alpha$, где B - нетерминал, $\alpha$ - цепочка
терминалов и нетерминалов, возможно пустая, выполним 
замену и левую факторизацию. Если хоть одно из получившихся правил содержит правую часть
$B\alpha$ - алгоритм зациклился. Иначе - продолжаем выполнять замену и левую факторизацию 
для правил и отслеживаем появление $B\alpha$ в правой части новых правил. 

9. Исправил.

10. Символ D в правилах 1, 2 занимает крайнюю правую позицию. В правиле 3 - нет.
Символ b в правилах 3, 4, 5 занимает крайнюю правую позицию. В правиле 5 - нет.
Исправил.

\end{document}