\documentclass[a4paper,14pt]{extarticle}


\usepackage[english,russian]{babel}
\usepackage[T2A]{fontenc}
\usepackage[utf8]{inputenc}
\usepackage{ragged2e}
\usepackage[utf8]{inputenc}
\usepackage{hyperref}
\usepackage{minted}
\setmintedinline{frame=lines, framesep=2mm, baselinestretch=1.5, bgcolor=LightGray, breaklines,fontsize=\scriptsize}
\setminted{frame=lines, framesep=2mm, baselinestretch=1.5, bgcolor=LightGray, breaklines,fontsize=\scriptsize}
\usepackage{xcolor}
\definecolor{LightGray}{gray}{0.9}
\usepackage{graphicx}
\usepackage[export]{adjustbox}
\usepackage[left=1cm,right=1cm, top=1cm,bottom=1cm,bindingoffset=0cm]{geometry}
\usepackage{fontspec}
\usepackage{ upgreek }
\usepackage[shortlabels]{enumitem}
\usepackage{adjustbox}
\usepackage{multirow}
\usepackage{amsmath}
\usepackage{amssymb}
\usepackage{pifont}
\usepackage{pgfplots}
\usepackage{longtable}
\usepackage{array}
\graphicspath{ {./images/} }
\makeatletter
\AddEnumerateCounter{\asbuk}{\russian@alph}{щ}
\makeatother
\setmonofont{Consolas}
\setmainfont{Times New Roman}

\newcommand\textbox[1]{
	\parbox{.45\textwidth}{#1}
} 

\newcommand{\specialcell}[2][c]{%
	\begin{tabular}[#1]{@{}c@{}}#2\end{tabular}}

\begin{document}
\pagenumbering{gobble}
\begin{center}
    \small{
        \textbf{МИНИCТЕРCТВО НАУКИ И ВЫCШЕГО ОБРАЗОВАНИЯ РОCCИЙCКОЙ ФЕДЕРАЦИИ}\\
        ФЕДЕРАЛЬНОЕ ГОCУДАРCТВЕННОЕ БЮДЖЕТНОЕ ОБРАЗОВАТЕЛЬНОЕ УЧРЕЖДЕНИЕ\\ВЫCШЕГО ОБРАЗОВАНИЯ \\
        \textbf{«БЕЛГОРОДCКИЙ ГОCУДАРCТВЕННЫЙ ТЕХНОЛОГИЧЕCКИЙ\\УНИВЕРCИТЕТ им. В. Г. ШУХОВА»\\ (БГТУ им. В.Г. Шухова)} \\
        \bigbreak
        \includegraphics[width=70mm]{log}\\
        ИНСТИТУТ ИНФОРМАЦИОННЫХ ТЕХНОЛОГИЙ И УПРАВЛЯЮЩИХ СИСТЕМ\\}
\end{center}

\vfill
\begin{center}
    \large{
        \textbf{
            Лабораторная работа №2}}\\
    \normalsize{
        по дисциплине: Теория автоматов и формальных языков \\
        тема: «Преобразования КС-грамматик.»}
\end{center}
\vfill
\hfill\textbox{
    Выполнил: ст. группы ПВ-223\\Пахомов Владислав Андреевич
    \bigbreak
    Проверили: \\ст. пр. Рязанов Юрий Дмитриевич
}
\vfill\begin{center}
    Белгород 2024 г.
\end{center}
\newpage
\begin{center}
    \textbf{Лабораторная работа №2}\\
    Преобразования КС-грамматик.\\
    Вариант 8
\end{center}
\textbf{Цель работы: }изучить основные эквивалентные преобразования
КС-грамматик и научиться применять их для получения
КС-грамматик, обладающих заданными свойствами.\\
\textbf{Задание: }\\
    1. $T \rightarrow abETP$\\
    2. $T \rightarrow aDE$\\
    3. $T \rightarrow D$\\
    4. $D \rightarrow DTAb$\\
    5. $D \rightarrow b$\\
    6. $E \rightarrow \varepsilon$\\
    7. $P \rightarrow BCa$\\
    8. $P \rightarrow Cb$\\
    9. $C \rightarrow abC$\\
    10. $A \rightarrow Bbb$\\
    11. $B \rightarrow aECb$\\
    12. $B \rightarrow D$\\

\begin{enumerate}[1.]
    \item Преобразовать исходную грамматику $G$ в
    грамматику $G_1$ без лишних символов.\\
    Найдём в исходной грамматике бесплодные нетерминалы.\\
    Для начала найдём продуктивные нетерминалы.\\
    В множество продуктивных нетерминалов Р включаем нетерминал D
(правило 5) и нетерминал E (правило 6). Получаем $Р=\{D,E\}$.
Повторяем проверку и включаем нетерминал T (правило 2) и нетерминал B (правило 12). 
Получаем $P=\{D,E,T,B\}$\\
Повторяем проверку и включаем A (правило 10). Получаем $P=\{D,E,T,B,A\}$\\
Множество P больше увеличить не можем.\\

Из множества нетерминалов исключаем продуктивные 
нетерминалы и получаем $\{P,C\}$ - множество бесплодных нетерминалов.\\

Исключаем правила 1, 7, 8, 9, 11 так как они содержат 
бесплодные нетерминалы. Получаем грамматику:\\

2. $T \rightarrow aDE$\\
3. $T \rightarrow D$\\
4. $D \rightarrow DTAb$\\
5. $D \rightarrow b$\\
6. $E \rightarrow \varepsilon$\\
10. $A \rightarrow Bbb$\\
12. $B \rightarrow D$\\

Найдём достижимые символы.\\
Положим $P = \{T\}$, где T - начальный нетерминал.\\
Включим в список a, D, E (правило 2). $P = \{T, a, D, E\}$.\\
Включим в список b, A (правило 4), $\varepsilon$. $P = \{T, a, D, E, \varepsilon, b, A\}$.\\
Включим в список B (правило 10). $P = \{T, a, D, E, \varepsilon, b, A, B\}$.\\
Множество P больше увеличить не можем.\\

Из множества терминалов и нетерминалов исключаем достижимые 
нетерминалы и нетерминалы и получаем $\{\}$ - множество недостижимых нетерминалов и терминалов.\\

Недостижимых нетерминалов и терминалов нет.\\
Искомая грамматика $G_1$:\\
1. $T \rightarrow aDE$\\
2. $T \rightarrow D$\\
3. $D \rightarrow DTAb$\\
4. $D \rightarrow b$\\
5. $E \rightarrow \varepsilon$\\
6. $A \rightarrow Bbb$\\
7. $B \rightarrow D$\\


\end{enumerate}

\textbf{Вывод: } в ходе лабораторной работы изучили основные эквивалентные преобразования
КС-грамматик и научились применять их для получения
КС-грамматик, обладающих заданными свойствами.

\end{document}