\documentclass[a4paper,14pt]{extarticle}


\usepackage[english,russian]{babel}
\usepackage[T2A]{fontenc}
\usepackage[utf8]{inputenc}
\usepackage{ragged2e}
\usepackage[utf8]{inputenc}
\usepackage{hyperref}
\usepackage{minted}
\setmintedinline{frame=lines, framesep=2mm, baselinestretch=1.5, bgcolor=LightGray, breaklines,fontsize=\scriptsize}
\setminted{frame=lines, framesep=2mm, baselinestretch=1.5, bgcolor=LightGray, breaklines,fontsize=\scriptsize}
\usepackage{xcolor}
\definecolor{LightGray}{gray}{0.9}
\usepackage{graphicx}
\usepackage[export]{adjustbox}
\usepackage[left=1cm,right=1cm, top=1cm,bottom=1cm,bindingoffset=0cm]{geometry}
\usepackage{fontspec}
\usepackage{ upgreek }
\usepackage[shortlabels]{enumitem}
\usepackage{adjustbox}
\usepackage{multirow}
\usepackage{amsmath}
\usepackage{amssymb}
\usepackage{pifont}
\usepackage{pgfplots}
\usepackage{longtable}
\usepackage{array}
\graphicspath{ {./images/} }
\makeatletter
\AddEnumerateCounter{\asbuk}{\russian@alph}{щ}
\makeatother
\setmonofont{Consolas}
\setmainfont{Times New Roman}

\newcommand\textbox[1]{
	\parbox{.45\textwidth}{#1}
} 

\newcommand{\specialcell}[2][c]{%
	\begin{tabular}[#1]{@{}c@{}}#2\end{tabular}}

\begin{document}
\pagenumbering{gobble}
\begin{center}
    \small{
        \textbf{МИНИCТЕРCТВО НАУКИ И ВЫCШЕГО ОБРАЗОВАНИЯ РОCCИЙCКОЙ ФЕДЕРАЦИИ}\\
        ФЕДЕРАЛЬНОЕ ГОCУДАРCТВЕННОЕ БЮДЖЕТНОЕ ОБРАЗОВАТЕЛЬНОЕ УЧРЕЖДЕНИЕ\\ВЫCШЕГО ОБРАЗОВАНИЯ \\
        \textbf{«БЕЛГОРОДCКИЙ ГОCУДАРCТВЕННЫЙ ТЕХНОЛОГИЧЕCКИЙ\\УНИВЕРCИТЕТ им. В. Г. ШУХОВА»\\ (БГТУ им. В.Г. Шухова)} \\
        \bigbreak
        \includegraphics[width=70mm]{log}\\
        ИНСТИТУТ ИНФОРМАЦИОННЫХ ТЕХНОЛОГИЙ И УПРАВЛЯЮЩИХ СИСТЕМ\\}
\end{center}

\vfill
\begin{center}
    \large{
        \textbf{
            Лабораторная работа №2}}\\
    \normalsize{
        по дисциплине: Теория автоматов и формальных языков \\
        тема: «Преобразования КС-грамматик.»}
\end{center}
\vfill
\hfill\textbox{
    Выполнил: ст. группы ПВ-223\\Пахомов Владислав Андреевич
    \bigbreak
    Проверили: \\ст. пр. Рязанов Юрий Дмитриевич
}
\vfill\begin{center}
    Белгород 2024 г.
\end{center}
\newpage
\begin{center}
    \textbf{Лабораторная работа №2}\\
    Преобразования КС-грамматик.\\
    Вариант 8
\end{center}
\textbf{Цель работы: }изучить основные эквивалентные преобразования
КС-грамматик и научиться применять их для получения
КС-грамматик, обладающих заданными свойствами.\\
\textbf{Задание: }\\
    1. $T \rightarrow abETP$\\
    2. $T \rightarrow aDE$\\
    3. $T \rightarrow D$\\
    4. $D \rightarrow DTAb$\\
    5. $D \rightarrow b$\\
    6. $E \rightarrow \varepsilon$\\
    7. $P \rightarrow BCa$\\
    8. $P \rightarrow Cb$\\
    9. $C \rightarrow abC$\\
    10. $A \rightarrow Bbb$\\
    11. $B \rightarrow aECb$\\
    12. $B \rightarrow D$\\

\begin{enumerate}[1.]
    \item Преобразовать исходную грамматику $G$ в
    грамматику $G_1$ без лишних символов.\\
    \textbf{Модификации: }в ходе выполнения лабораторной работы обнаружено, что в грамматике
    не будет недостижимых символов. Поэтому добавим правило:\\
    
    13. $S \rightarrow ab$


    Найдём в исходной грамматике бесплодные нетерминалы.\\
    Для начала найдём продуктивные нетерминалы.\\
    В множество продуктивных нетерминалов Р включаем нетерминал D
(правило 5) нетерминал E (правило 6) и нетерминал S (правило 13). Получаем $Р=\{D,E,S\}$.
Повторяем проверку и включаем нетерминал T (правило 2) и нетерминал B (правило 12). 
Получаем $P=\{D,E,S,T,B\}$\\
Повторяем проверку и включаем A (правило 10). Получаем $P=\{D,E,S,T,B,A\}$\\
Множество P больше увеличить не можем.\\

Из множества нетерминалов исключаем продуктивные 
нетерминалы и получаем $\{P,C\}$ - множество бесплодных нетерминалов.\\

Исключаем правила 1, 7, 8, 9, 11 так как они содержат 
бесплодные нетерминалы. Получаем грамматику:\\

2. $T \rightarrow aDE$\\
3. $T \rightarrow D$\\
4. $D \rightarrow DTAb$\\
5. $D \rightarrow b$\\
6. $E \rightarrow \varepsilon$\\
10. $A \rightarrow Bbb$\\
12. $B \rightarrow D$\\
13. $S \rightarrow ab$\\

Найдём достижимые символы.\\
Положим $P = \{T\}$, где T - начальный нетерминал.\\
Включим в список a, D, E (правило 2). $P = \{T, a, D, E\}$.\\
Включим в список b, A (правило 4), $\varepsilon$. $P = \{T, a, D, E, \varepsilon, b, A\}$.\\
Включим в список B (правило 10). $P = \{T, a, D, E, \varepsilon, b, A, B\}$.\\
Множество P больше увеличить не можем.\\

Из множества терминалов и нетерминалов исключаем достижимые 
терминалы и нетерминалы и получаем $\{S\}$ - множество недостижимых нетерминалов и терминалов.\\

Исключаем из грамматики правило 13, так как оно содержит недостижимый символ.\\
Искомая грамматика $G_1$:\\
1. $T \rightarrow aDE$\\
2. $T \rightarrow D$\\
3. $D \rightarrow DTAb$\\
4. $D \rightarrow b$\\
5. $E \rightarrow \varepsilon$\\
6. $A \rightarrow Bbb$\\
7. $B \rightarrow D$\\

\item Преобразовать грамматику $G_1$ в грамматику $G_2$ без $\varepsilon$-правил.\\
Выберем правило 5. Иключаем из правой части каждого правила исходной грамматики
всеми возможными способами вхождение нетерминала E. Полученные правила добавляем в множество 
правил грамматики.\\
1\_1. $T \rightarrow aDE$\\
1\_2. $T \rightarrow aD$\\
2. $T \rightarrow D$\\
3. $D \rightarrow DTAb$\\
4. $D \rightarrow b$\\
5. $E \rightarrow \varepsilon$\\
6. $A \rightarrow Bbb$\\
7. $B \rightarrow D$\\
Исключаем из списка правил правило 5.\\
1\_1. $T \rightarrow aDE$\\
1\_2. $T \rightarrow aD$\\
2. $T \rightarrow D$\\
3. $D \rightarrow DTAb$\\
4. $D \rightarrow b$\\
6. $A \rightarrow Bbb$\\
7. $B \rightarrow D$\\
Исключим из правил непродуктивные символы:\\
1\_2. $T \rightarrow aD$\\
2. $T \rightarrow D$\\
3. $D \rightarrow DTAb$\\
4. $D \rightarrow b$\\
6. $A \rightarrow Bbb$\\
7. $B \rightarrow D$\\
В полученной грамматике $G_2$ нет правил вида $A \rightarrow A$, одинаковых правил и $\varepsilon$-правил. Получили искомую грамматику:\\
Искомая грамматика $G_2$:\\
1. $T \rightarrow aD$\\
2. $T \rightarrow D$\\
3. $D \rightarrow DTAb$\\
4. $D \rightarrow b$\\
6. $A \rightarrow Bbb$\\
7. $B \rightarrow D$\\

\item Преобразовать грамматику $G_1$ в грамматику $G_3$ без цепных правил.

Исходная грамматика:\\
1. $T \rightarrow aDE$\\
2. $T \rightarrow D$\\
3. $D \rightarrow DTAb$\\
4. $D \rightarrow b$\\
5. $E \rightarrow \varepsilon$\\
6. $A \rightarrow Bbb$\\
7. $B \rightarrow D$\\

Выполним замену края в правиле 2, так как нетерминал T - начальный:\\
1. $T \rightarrow aDE$\\
2\_1. $T \rightarrow DTAb$\\
2\_2. $T \rightarrow b$\\
3. $D \rightarrow DTAb$\\
4. $D \rightarrow b$\\
5. $E \rightarrow \varepsilon$\\
6. $A \rightarrow Bbb$\\
7. $B \rightarrow D$\\

Заменим символ B в правиле 6 и удалим правило 7:\\
1. $T \rightarrow aDE$\\
2\_1. $T \rightarrow DTAb$\\
2\_2. $T \rightarrow b$\\
3. $D \rightarrow DTAb$\\
4. $D \rightarrow b$\\
5. $E \rightarrow \varepsilon$\\
6\_1. $A \rightarrow Bbb$\\
6\_2. $A \rightarrow Dbb$\\

Исключим правила с бесплодными нетерминалами:\\
1. $T \rightarrow aDE$\\
2\_1. $T \rightarrow DTAb$\\
2\_2. $T \rightarrow b$\\
3. $D \rightarrow DTAb$\\
4. $D \rightarrow b$\\
5. $E \rightarrow \varepsilon$\\
6\_2. $A \rightarrow Dbb$\\

Искомая грамматика $G_3$:\\
1. $T \rightarrow aDE$\\
2. $T \rightarrow DTAb$\\
3. $T \rightarrow b$\\
4. $D \rightarrow DTAb$\\
5. $D \rightarrow b$\\
6. $E \rightarrow \varepsilon$\\
7. $A \rightarrow Dbb$\\

\item Преобразовать грамматику $G_1$ в грамматику $G_4$ без левой рекурсии.\\

Исходная грамматика:\\
1. $T \rightarrow aDE$\\
2. $T \rightarrow D$\\
3. $D \rightarrow DTAb$\\
4. $D \rightarrow b$\\
5. $E \rightarrow \varepsilon$\\
6. $A \rightarrow Bbb$\\
7. $B \rightarrow D$\\

T не содержит левой рекурсии. Выполним устранение самолеворекурсивного правила для D:\\
$T \rightarrow aDE$\\
$T \rightarrow D$\\
$D \rightarrow bD'$\\
$E \rightarrow \varepsilon$\\
$A \rightarrow Bbb$\\
$B \rightarrow D$\\
$D' \rightarrow TAbD'$\\
$D' \rightarrow \varepsilon$\\

Выполним замену края в правиле 6:\\
$T \rightarrow aDE$\\
$T \rightarrow D$\\
$D \rightarrow bD'$\\
$E \rightarrow \varepsilon$\\
$A \rightarrow Bbb$\\
$B \rightarrow bD'$\\
$D' \rightarrow TAbD'$\\
$D' \rightarrow \varepsilon$\\

Искомая грамматика $G_4$:\\
1. $T \rightarrow aDE$\\
2. $T \rightarrow D$\\
3. $D \rightarrow bD'$\\
4. $E \rightarrow \varepsilon$\\
5. $A \rightarrow Bbb$\\
6. $B \rightarrow bD'$\\
7. $D' \rightarrow TAbD'$\\
8. $D' \rightarrow \varepsilon$\\

\item Преобразовать грамматику $G_1$ в грамматику $G_5$ без несаморекурсивных
нетерминалов.

Искходная грамматика:\\
1. $T \rightarrow aDE$\\
2. $T \rightarrow D$\\
3. $D \rightarrow DTAb$\\
4. $D \rightarrow b$\\
5. $E \rightarrow \varepsilon$\\
6. $A \rightarrow Bbb$\\
7. $B \rightarrow D$\\

Нетерминал E несаморекурсивный.\\
Исключаем правило 5:\\
5. $E \rightarrow \varepsilon$\\
Выбираем вхождение символа E в правиле 1 и
выполняем одиночную замену:\\
1\_1. $T \rightarrow aD$\\
2. $T \rightarrow D$\\
3. $D \rightarrow DTAb$\\
4. $D \rightarrow b$\\
6. $A \rightarrow Bbb$\\
7. $B \rightarrow D$\\

Нетерминал B несаморекурсивный.\\
Исключаем правило 7, выполняя одиночную замену:\\
1\_1. $T \rightarrow aD$\\
2. $T \rightarrow D$\\
3. $D \rightarrow DTAb$\\
4. $D \rightarrow b$\\
6. $A \rightarrow Dbb$\\

Нетерминал A несаморекурсивный.\\
Исключаем правило 7, выполняя одиночную замену:\\
1\_1. $T \rightarrow aD$\\
2. $T \rightarrow D$\\
3. $D \rightarrow DTDbbb$\\
4. $D \rightarrow b$\\

Нетерминал D саморекурсивный. Нетерминал T хоть и несаморекурсивный, 
однко является начальным терминалом, а значит выполнять преобразования 
с ним не можем.

Искомая грамматика $G_5$:\\
1. $T \rightarrow aD$\\
2. $T \rightarrow D$\\
3. $D \rightarrow DTDbbb$\\
4. $D \rightarrow b$\\

\item Получить грамматику $G_6$, эквивалентную грамматике $G_1$, в которой
правая часть каждого правила состоит либо из одного терминала, либо двух нетерминалов.

Для получения грамматики $G_6$ необходимо привести грамматику $G_1$ 
к нормальной форме Хомского.\\
Воспользуемся грамматикой $G_3'$, в которой нет цепных правил, $\varepsilon$-правил
и цепных правил.\\
Исходная грамматика:\\
1. $T \rightarrow aD$\\
2. $T \rightarrow DTAb$\\
3. $T \rightarrow b$\\
4. $D \rightarrow DTAb$\\
5. $D \rightarrow b$\\
6. $A \rightarrow Bbb$\\
7. $B \rightarrow DTAb$\\
8. $B \rightarrow b$\\

Выполним пункт 1 алгоритма (преобразование правил вида $A \rightarrow Xa$):\\
$T \rightarrow aD$\\
$T \rightarrow DN_1$\\
$T \rightarrow b$\\
$D \rightarrow DN_1$\\
$D \rightarrow b$\\
$A \rightarrow Bbb$\\
$B \rightarrow DN_1$\\
$B \rightarrow b$\\
$N_1 \rightarrow TAb$\\

$T \rightarrow aD$\\
$T \rightarrow DN_1$\\
$T \rightarrow b$\\
$D \rightarrow DN_1$\\
$D \rightarrow b$\\
$A \rightarrow BN_2$\\
$B \rightarrow DN_1$\\
$B \rightarrow b$\\
$N_1 \rightarrow TAb$\\
$N_2 \rightarrow bb$\\

$T \rightarrow aD$\\
$T \rightarrow DN_1$\\
$T \rightarrow b$\\
$D \rightarrow DN_1$\\
$D \rightarrow b$\\
$A \rightarrow BN_2$\\
$B \rightarrow DN_1$\\
$B \rightarrow b$\\
$N_1 \rightarrow TN_3$\\
$N_2 \rightarrow bb$\\
$N_3 \rightarrow Ab$\\

Выполним пункт 2 алгоритма (преобразование правил вида $A \rightarrow tB$):\\
$T \rightarrow N_4D$\\
$T \rightarrow DN_1$\\
$T \rightarrow b$\\
$D \rightarrow DN_1$\\
$D \rightarrow b$\\
$A \rightarrow BN_2$\\
$B \rightarrow DN_1$\\
$B \rightarrow b$\\
$N_1 \rightarrow TN_3$\\
$N_2 \rightarrow bb$\\
$N_3 \rightarrow Ab$\\
$N_4 \rightarrow a$\\

Выполним пункт 3 алгоритма (преобразование правил вида $A \rightarrow Bt$):\\
$T \rightarrow N_4D$\\
$T \rightarrow DN_1$\\
$T \rightarrow b$\\
$D \rightarrow DN_1$\\
$D \rightarrow b$\\
$A \rightarrow BN_2$\\
$B \rightarrow DN_1$\\
$B \rightarrow b$\\
$N_1 \rightarrow TN_3$\\
$N_2 \rightarrow bb$\\
$N_3 \rightarrow AN_5$\\
$N_4 \rightarrow a$\\
$N_5 \rightarrow b$\\

Выполним пункт 4 алгоритма (преобразование правил вида $A \rightarrow tt$):\\
$T \rightarrow N_4D$\\
$T \rightarrow DN_1$\\
$T \rightarrow b$\\
$D \rightarrow DN_1$\\
$D \rightarrow b$\\
$A \rightarrow BN_2$\\
$B \rightarrow DN_1$\\
$B \rightarrow b$\\
$N_1 \rightarrow TN_3$\\
$N_2 \rightarrow N_5N_5$\\
$N_3 \rightarrow AN_5$\\
$N_4 \rightarrow a$\\
$N_5 \rightarrow b$\\

Искомая грамматика $G_6$:\\
$T \rightarrow N_4D$\\
$T \rightarrow DN_1$\\
$T \rightarrow b$\\
$D \rightarrow DN_1$\\
$D \rightarrow b$\\
$A \rightarrow BN_2$\\
$B \rightarrow DN_1$\\
$B \rightarrow b$\\
$N_1 \rightarrow TN_3$\\
$N_2 \rightarrow N_5N_5$\\
$N_3 \rightarrow AN_5$\\
$N_4 \rightarrow a$\\
$N_5 \rightarrow b$\\

\item Получить грамматику $G_7$, эквивалентную грамматике $G_1$, в которой 
правая часть каждого правила начинается терминалом.

Используем преобразованную грамматику $G_1$ без левой рекурсии $G_4$:\\

В $G_4$ есть $\varepsilon$-правила. Исключим их и получим грамматику $G_4'$:\\
1. $T \rightarrow aD$\\
2. $T \rightarrow DTAb$\\
3. $T \rightarrow b$\\
4. $D \rightarrow bB_1$\\
5. $B_1 \rightarrow TAbB_1$\\
6. $B_1 \rightarrow \varepsilon$\\
7. $A \rightarrow Bbb$\\
8. $B \rightarrow bB_1TAb$\\
9. $B \rightarrow b$\\

1. $T \rightarrow aD$\\
2. $T \rightarrow DTAb$\\
3. $T \rightarrow b$\\
4\_1. $D \rightarrow bB_1$\\
4\_2. $D \rightarrow b$\\
5\_1. $B_1 \rightarrow TAbB_1$\\
5\_2. $B_1 \rightarrow TAb$\\
7. $A \rightarrow Bbb$\\
8\_1. $B \rightarrow bB_1TAb$\\
8\_2. $B \rightarrow bTAb$\\
9. $B \rightarrow b$\\

\iffalse
1. $T \rightarrow aD$\\              T  _  D
2. $T \rightarrow DTAb$\\            T  D  _
3. $T \rightarrow b$\\               T  _  _
4\_1. $D \rightarrow bB_1$\\         D  _  B1
4\_2. $D \rightarrow b$\\            D  _  _
5\_1. $B_1 \rightarrow TAbB_1$\\     B1 T  B1
5\_2. $B_1 \rightarrow TAb$\\        B1 T  _
7. $A \rightarrow Bbb$\\             A  B  _
8\_1. $B \rightarrow bB_1TAb$\\      B  _  _
8\_2. $B \rightarrow bTAb$\\         B  _  _
9. $B \rightarrow b$\\               B  _  _

B1 T  _
B1 T  B1
T  _  D
T  D  _
D  _  B1
A  B  _
B  _  _
B  _  _
B  _  _
T  _  _
D  _  _
\fi

Упорядочим грамматику:\\
1. $B_1 \rightarrow TAb$\\
2. $B_1 \rightarrow TAbB_1$\\
3. $T \rightarrow aD$\\ 
4. $T \rightarrow DTAb$\\
5. $D \rightarrow bB_1$\\
6. $A \rightarrow Bbb$\\
7. $B \rightarrow bB_1TAb$\\ 
8. $B \rightarrow bTAb$\\ 
9. $B \rightarrow b$\\
10. $T \rightarrow b$\\
11. $D \rightarrow b$\\

Выполнение замены края:\\
1. $B_1 \rightarrow TAb$\\
2. $B_1 \rightarrow TAbB_1$\\
3. $T \rightarrow aD$\\ 
4. $T \rightarrow DTAb$\\
5. $D \rightarrow bB_1$\\
6\_1. $A \rightarrow bB_1TAbbb$\\
6\_2. $A \rightarrow bTAbbb$\\
6\_3. $A \rightarrow bbb$\\
7. $B \rightarrow bB_1TAb$\\ 
8. $B \rightarrow bTAb$\\ 
9. $B \rightarrow b$\\
10. $T \rightarrow b$\\
11. $D \rightarrow b$\\

1. $B_1 \rightarrow TAb$\\
2. $B_1 \rightarrow TAbB_1$\\
3. $T \rightarrow aD$\\ 
4. $T \rightarrow bB_1TAb$\\
5. $D \rightarrow bB_1$\\
6\_1. $A \rightarrow bB_1TAbbb$\\
6\_2. $A \rightarrow bTAbbb$\\
6\_3. $A \rightarrow bbb$\\
7. $B \rightarrow bB_1TAb$\\ 
8. $B \rightarrow bTAb$\\ 
9. $B \rightarrow b$\\
10. $T \rightarrow b$\\
11. $D \rightarrow b$\\

1\_1. $B_1 \rightarrow aDAb$\\
1\_2. $B_1 \rightarrow bB_1TAbAb$\\
2\_1. $B_1 \rightarrow aDAbB_1$\\
2\_2. $B_1 \rightarrow bB_1TAbAbB_1$\\
3. $T \rightarrow aD$\\ 
4. $T \rightarrow bB_1TAb$\\
5. $D \rightarrow bB_1$\\
6\_1. $A \rightarrow bB_1TAbbb$\\
6\_2. $A \rightarrow bTAbbb$\\
6\_3. $A \rightarrow bbb$\\
7. $B \rightarrow bB_1TAb$\\ 
8. $B \rightarrow bTAb$\\ 
9. $B \rightarrow b$\\
10. $T \rightarrow b$\\
11. $D \rightarrow b$\\

Искомая грамматика $G_7$:\\
1. $B_1 \rightarrow aDAb$\\
2. $B_1 \rightarrow bB_1TAbAb$\\
3. $B_1 \rightarrow aDAbB_1$\\
4. $B_1 \rightarrow bB_1TAbAbB_1$\\
5. $T \rightarrow aD$\\ 
6. $T \rightarrow bB_1TAb$\\
7. $D \rightarrow bB_1$\\
8. $A \rightarrow bB_1TAbbb$\\
9. $A \rightarrow bTAbbb$\\
10. $A \rightarrow bbb$\\
11. $B \rightarrow bB_1TAb$\\ 
12. $B \rightarrow bTAb$\\ 
13. $B \rightarrow b$\\
14. $T \rightarrow b$\\
15. $D \rightarrow b$\\

\item Получить грамматику $G_8$, эквивалентную грамматике $G_1$, в которой 
правая часть каждого не $\varepsilon$-правила начинается терминалом и любые
два правила с одинаковой левой частью различаются первым символом в правой части.\\
Для получения такой грамматики можем проводить множественную левую факторизацию и замену в грамматике $G_7$.\\

\textbf{Модификации: }в ходе выполнения задания было выявлено, что грамматика $G_7$
преобразовать к искомой невозможно, так как алгоримт зациклился.
Попробуем удалить из грамматики $G_7$ правила 2, 3, 4, 5.\\
$B_1 \rightarrow aDAb$\\
$T \rightarrow bB_1TAb$\\
$D \rightarrow bB_1$\\
$A \rightarrow bB_1TAbbb$\\
$A \rightarrow bTAbbb$\\
$A \rightarrow bbb$\\
$B \rightarrow bB_1TAb$\\ 
$B \rightarrow bTAb$\\ 
$B \rightarrow b$\\
$T \rightarrow b$\\
$D \rightarrow b$\\

Выполним левую факторизацию:\\
$B_1 \rightarrow aDAb$\\
$T \rightarrow bB_1TAb$\\
$D \rightarrow bB_1$\\
$A \rightarrow bE_1$\\
$E_1 \rightarrow B_1TAbbb$\\
$E_1 \rightarrow TAbbb$\\
$E_1 \rightarrow bb$\\ 
$B \rightarrow bE_2$\\
$E_2 \rightarrow B_1TAb$\\
$E_2 \rightarrow TAb$\\
$E_2 \rightarrow \varepsilon$\\
$T \rightarrow b$\\
$D \rightarrow b$\\

Выполним замену:\\
$B_1 \rightarrow aDAb$\\
$T \rightarrow bB_1TAb$\\
$D \rightarrow bB_1$\\
$A \rightarrow bE_1$\\
$E_1 \rightarrow aDAbTAbbb$\\
$E_1 \rightarrow bB_1TAbAbbb$\\
$E_1 \rightarrow bb$\\ 
$B \rightarrow bE_2$\\
$E_2 \rightarrow aDAbTAb$\\
$E_2 \rightarrow bB_1TAbAb$\\
$E_2 \rightarrow \varepsilon$\\
$T \rightarrow b$\\
$D \rightarrow b$\\

Выполним левую факторизацию:\\
$B_1 \rightarrow aDAb$\\
$T \rightarrow bB_1TAb$\\
$D \rightarrow bB_1$\\
$A \rightarrow bE_1$\\
$E_1 \rightarrow aDAbTAbbb$\\
$E_1 \rightarrow bE_3$\\ 
$E_3 \rightarrow B_1TAbAbbb$\\
$E_3 \rightarrow b$\\
$B \rightarrow bE_2$\\
$E_2 \rightarrow aDAbTAb$\\
$E_2 \rightarrow bB_1TAbAb$\\
$E_2 \rightarrow \varepsilon$\\
$T \rightarrow b$\\
$D \rightarrow b$\\

Выполним замену:\\
$B_1 \rightarrow aDAb$\\
$T \rightarrow bB_1TAb$\\
$D \rightarrow bB_1$\\
$A \rightarrow bE_1$\\
$E_1 \rightarrow aDAbTAbbb$\\
$E_1 \rightarrow bE_3$\\ 
$E_3 \rightarrow aDAbTAbAbbb$\\
$E_3 \rightarrow b$\\
$B \rightarrow bE_2$\\
$E_2 \rightarrow aDAbTAb$\\
$E_2 \rightarrow bB_1TAbAb$\\
$E_2 \rightarrow \varepsilon$\\
$T \rightarrow b$\\
$D \rightarrow b$\\

Искомая грамматика $G_8$:\\
$B_1 \rightarrow aDAb$\\
$T \rightarrow bB_1TAb$\\
$D \rightarrow bB_1$\\
$A \rightarrow bE_1$\\
$B \rightarrow bE_2$\\
$T \rightarrow b$\\
$D \rightarrow b$\\
$E_1 \rightarrow aDAbTAbbb$\\
$E_1 \rightarrow bE_3$\\ 
$E_2 \rightarrow aDAbTAb$\\
$E_2 \rightarrow bB_1TAbAb$\\
$E_2 \rightarrow \varepsilon$\\
$E_3 \rightarrow aDAbTAbAbbb$\\
$E_3 \rightarrow b$\\

\iffalse
Исходная грамматика:\\
$B_1 \rightarrow aDAb$\\
$B_1 \rightarrow bB_1TAbAb$\\
$B_1 \rightarrow aDAbB_1$\\
$B_1 \rightarrow bB_1TAbAbB_1$\\
$T \rightarrow aD$\\ 
$T \rightarrow bB_1TAb$\\
$D \rightarrow bB_1$\\
$A \rightarrow bB_1TAbbb$\\
$A \rightarrow bTAbbb$\\
$A \rightarrow bbb$\\
$B \rightarrow bB_1TAb$\\ 
$B \rightarrow bTAb$\\ 
$B \rightarrow b$\\
$T \rightarrow b$\\
$D \rightarrow b$\\

Выполним левую факторизацию для $B_1$:\\
$B_1 \rightarrow aDAb$\\
$B_1 \rightarrow bB_1TAbAb$\\
$B_1 \rightarrow aDAbB_1$\\
$B_1 \rightarrow bB_1TAbAbB_1$\\
$T \rightarrow aD$\\ 
$T \rightarrow bB_1TAb$\\
$D \rightarrow bB_1$\\
$A \rightarrow bB_1TAbbb$\\
$A \rightarrow bTAbbb$\\
$A \rightarrow bbb$\\
$B \rightarrow bB_1TAb$\\ 
$B \rightarrow bTAb$\\ 
$B \rightarrow b$\\
$T \rightarrow b$\\
$D \rightarrow b$\\

Выполним замену:\\
$B_1 \rightarrow bB_1TAbAbE_1$\\
$B_1 \rightarrow aDAbE_1$\\
$E_1 \rightarrow \varepsilon$\\
$E_1 \rightarrow bB_1TAbAbE_1$\\
$E_1 \rightarrow aDAbE_1$\\
$T \rightarrow aD$\\ 
$T \rightarrow bB_1TAb$\\
$D \rightarrow bB_1$\\
$A \rightarrow bB_1TAbbb$\\
$A \rightarrow bTAbbb$\\
$A \rightarrow bbb$\\
$B \rightarrow bB_1TAb$\\ 
$B \rightarrow bTAb$\\ 
$B \rightarrow b$\\
$T \rightarrow b$\\
$D \rightarrow b$\\

Выполним левую факторизацию для A, B:\\
$B_1 \rightarrow bB_1TAbAbE_1$\\
$B_1 \rightarrow aDAbE_1$\\
$E_1 \rightarrow \varepsilon$\\
$E_1 \rightarrow bB_1TAbAbE_1$\\
$E_1 \rightarrow aDAbE_1$\\
$T \rightarrow aD$\\ 
$T \rightarrow bB_1TAb$\\
$D \rightarrow bB_1$\\
$A \rightarrow bE_2$\\
$E_2 \rightarrow B_1TAbbb$\\
$E_2 \rightarrow TAbbb$\\
$E_2 \rightarrow bb$\\
$B \rightarrow bE_3$\\ 
$E_3 \rightarrow B_1TAb$\\ 
$E_3 \rightarrow TAb$\\
$B \rightarrow b$\\
$T \rightarrow b$\\
$D \rightarrow b$\\

Выполним замену:\\
$B_1 \rightarrow bB_1TAbAbE_1$\\
$B_1 \rightarrow aDAbE_1$\\
$E_1 \rightarrow \varepsilon$\\
$E_1 \rightarrow bB_1TAbAbE_1$\\
$E_1 \rightarrow aDAbE_1$\\
$T \rightarrow aD$\\ 
$T \rightarrow bB_1TAb$\\
$D \rightarrow bB_1$\\
$A \rightarrow bE_2$\\
$E_2 \rightarrow bB_1TAbAbE_1TAbbb$\\
$E_2 \rightarrow aDAbE_1TAbbb$\\
$E_2 \rightarrow aDAbbb$\\
$E_2 \rightarrow bB_1TAbAbbb$\\
$E_2 \rightarrow bb$\\
$B \rightarrow bE_3$\\ 
$E_3 \rightarrow bB_1TAbAbE_1TAb$\\ 
$E_3 \rightarrow aDAbE_1TAb$\\ 
$E_3 \rightarrow aDAb$\\
$E_3 \rightarrow bB_1TAbAb$\\
$B \rightarrow b$\\
$T \rightarrow b$\\
$D \rightarrow b$\\

Выполним левую факторизацию для $E_2$, $E_3$:\\
$B_1 \rightarrow bB_1TAbAbE_1$\\
$B_1 \rightarrow aDAbE_1$\\
$E_1 \rightarrow \varepsilon$\\
$E_1 \rightarrow bB_1TAbAbE_1$\\
$E_1 \rightarrow aDAbE_1$\\
$T \rightarrow aD$\\ 
$T \rightarrow bB_1TAb$\\
$D \rightarrow bB_1$\\
$A \rightarrow bE_2$\\
$E_2 \rightarrow aDAbE_4$\\
$E_4 \rightarrow E_1TAbbb$\\
$E_4 \rightarrow bb$\\
$E_2 \rightarrow bE_5$\\
$E_5 \rightarrow B_1TAbAbE_1TAbbb$\\
$E_5 \rightarrow B_1TAbAbbb$\\
$E_5 \rightarrow b$\\
$B \rightarrow bE_3$\\ 
$E_3 \rightarrow aDAbE_6$\\ 
$E_6 \rightarrow E_1TAb$\\ 
$E_6 \rightarrow \varepsilon$\\ 
$E_3 \rightarrow bB_1TAbAbE_7$\\
$E_7 \rightarrow E_1TAb$\\
$E_7 \rightarrow \varepsilon$\\
$B \rightarrow b$\\
$T \rightarrow b$\\
$D \rightarrow b$\\

Выполним замену для $E_4$, $E_5$, $E_6$, $E_7$:\\
$B_1 \rightarrow bB_1TAbAbE_1$\\
$B_1 \rightarrow aDAbE_1$\\
$E_1 \rightarrow \varepsilon$\\
$E_1 \rightarrow bB_1TAbAbE_1$\\
$E_1 \rightarrow aDAbE_1$\\
$T \rightarrow aD$\\ 
$T \rightarrow bB_1TAb$\\
$D \rightarrow bB_1$\\
$A \rightarrow bE_2$\\
$E_2 \rightarrow aDAbE_4$\\
$E_4 \rightarrow aDAbbb$\\
$E_4 \rightarrow bB_1TAbAbbb$\\
$E_4 \rightarrow bB_1TAbAbE_1TAbbb$\\
$E_4 \rightarrow aDAbE_1TAbbb$\\
$E_4 \rightarrow bb$\\
$E_2 \rightarrow bE_5$\\
$E_5 \rightarrow aDAbE_1TAbAbbb$\\
$E_5 \rightarrow aDAbE_1TAbAbE_1TAbbb$\\
$E_5 \rightarrow bB_1TAbAbE_1TAbAbE_1TAbbb$\\
$E_5 \rightarrow bB_1TAbAbE_1TAbAbbb$\\
$E_5 \rightarrow b$\\
$B \rightarrow bE_3$\\ 
$E_3 \rightarrow aDAbE_6$\\
$E_6 \rightarrow aDAb$\\ 
$E_6 \rightarrow bB_1TAbAb$\\ 
$E_6 \rightarrow bB_1TAbAbE_1TAb$\\ 
$E_6 \rightarrow aDAbE_1TAb$\\ 
$E_6 \rightarrow \varepsilon$\\ 
$E_3 \rightarrow bB_1TAbAbE_7$\\
$E_7 \rightarrow aDAb$\\
$E_7 \rightarrow bB_1TAbAb$\\
$E_7 \rightarrow bB_1TAbAbE_1TAb$\\
$E_7 \rightarrow aDAbE_1TAb$\\
$E_7 \rightarrow \varepsilon$\\
$B \rightarrow b$\\
$T \rightarrow b$\\
$D \rightarrow b$\\

Выполним левую факторизацию:\\
$B_1 \rightarrow bB_1TAbAbE_1$\\
$B_1 \rightarrow aDAbE_1$\\
$E_1 \rightarrow \varepsilon$\\
$E_1 \rightarrow bB_1TAbAbE_1$\\
$E_1 \rightarrow aDAbE_1$\\
$T \rightarrow aD$\\ 
$T \rightarrow bB_1TAb$\\
$D \rightarrow bB_1$\\
$A \rightarrow bE_2$\\
$E_2 \rightarrow aDAbE_4$\\
$E_4 \rightarrow aDAbE_8$\\
$E_8 \rightarrow E_1TAbbb$\\
$E_8 \rightarrow bb$\\
$E_4 \rightarrow bE_9$\\
$E_9 \rightarrow B_1TAbAbbb$\\
$E_9 \rightarrow B_1TAbAbE_1TAbbb$\\
$E_9 \rightarrow b$\\
$E_2 \rightarrow bE_5$\\
$E_5 \rightarrow aDAbE_1TAbAbE_{10}$\\
$E_{10} \rightarrow bb$\\
$E_{10} \rightarrow E_1TAbbb$\\
$E_5 \rightarrow bE_{11}$\\
$E_{11} \rightarrow B_1TAbAbE_1TAbAbE_1TAbbb$\\
$E_{11} \rightarrow B_1TAbAbE_1TAbAbbb$\\
$E_{11} \rightarrow \varepsilon$\\
$B \rightarrow bE_3$\\ 
$E_3 \rightarrow aDAbE_6$\\
$E_6 \rightarrow aDAbE_{12}$\\ 
$E_{12} \rightarrow E_1TAb$\\
$E_{12} \rightarrow \varepsilon$\\
$E_6 \rightarrow bB_1TAbAbE_{13}$\\ 
$E_{13} \rightarrow \varepsilon$\\ 
$E_{13} \rightarrow E_1TAb$\\
$E_6 \rightarrow \varepsilon$\\ 
$E_3 \rightarrow bB_1TAbAbE_7$\\
$E_7 \rightarrow aDAbE_{14}$\\
$E_{14} \rightarrow E_1TAb$\\
$E_{14} \rightarrow \varepsilon$\\
$E_7 \rightarrow bB_1TAbAbE_{15}$\\
$E_7 \rightarrow bB_1TAbAb$\\
$E_{15} \rightarrow \varepsilon$\\
$E_{15} \rightarrow E_1TAb$\\
$E_7 \rightarrow \varepsilon$\\
$B \rightarrow b$\\
$T \rightarrow b$\\
$D \rightarrow b$\\

Выполним замену:\\
$B_1 \rightarrow bB_1TAbAbE_1$\\
$B_1 \rightarrow aDAbE_1$\\
$E_1 \rightarrow \varepsilon$\\
$E_1 \rightarrow bB_1TAbAbE_1$\\
$E_1 \rightarrow aDAbE_1$\\
$T \rightarrow aD$\\ 
$T \rightarrow bB_1TAb$\\
$D \rightarrow bB_1$\\
$A \rightarrow bE_2$\\
$E_2 \rightarrow aDAbE_4$\\
$E_4 \rightarrow aDAbE_8$\\
$E_8 \rightarrow aDAbbb$\\
$E_8 \rightarrow bB_1TAbAbbb$\\
$E_8 \rightarrow bB_1TAbAbE_1TAbbb$\\
$E_8 \rightarrow aDAbE_1TAbbb$\\
$E_8 \rightarrow bb$\\
$E_4 \rightarrow bE_9$\\
$E_9 \rightarrow bB_1TAbAbE_1TAbAbbb$\\
$E_9 \rightarrow aDAbE_1TAbAbbb$\\
$E_9 \rightarrow bB_1TAbAbE_1TAbAbE_1TAbbb$\\
$E_9 \rightarrow aDAbE_1TAbAbE_1TAbbb$\\
$E_9 \rightarrow b$\\
$E_2 \rightarrow bE_5$\\
$E_5 \rightarrow aDAbE_1TAbAbE_{10}$\\
$E_{10} \rightarrow bb$\\
$E_{10} \rightarrow aDAbbb$\\
$E_{10} \rightarrow bB_1TAbAbbb$\\
$E_{10} \rightarrow bB_1TAbAbE_1TAbbb$\\
$E_{10} \rightarrow aDAbE_1TAbbb$\\
$E_5 \rightarrow bE_{11}$\
$E_{11} \rightarrow bB_1TAbAbE_1TAbAbE_1TAbAbE_1TAbbb$\\
$E_{11} \rightarrow aDAbE_1TAbAbE_1TAbAbE_1TAbbb$\\
$E_{11} \rightarrow bB_1TAbAbE_1TAbAbE_1TAbAbbb$\\
$E_{11} \rightarrow aDAbE_1TAbAbE_1TAbAbbb$\\
$E_{11} \rightarrow \varepsilon$\\
$B \rightarrow bE_3$\\ 
$E_3 \rightarrow aDAbE_6$\\
$E_6 \rightarrow aDAbE_{12}$\\ 
$E_{12} \rightarrow aDAb$\\
$E_{12} \rightarrow bB_1TAbAb$\\
$E_{12} \rightarrow bB_1TAbAbE_1TAb$\\
$E_{12} \rightarrow aDAbE_1TAb$\\
$E_{12} \rightarrow \varepsilon$\\
$E_6 \rightarrow bB_1TAbAbE_{13}$\\ 
$E_{13} \rightarrow \varepsilon$\\ 
$E_{13} \rightarrow E_1TAb$\\
$E_6 \rightarrow \varepsilon$\\ 
$E_3 \rightarrow bB_1TAbAbE_7$\\
$E_7 \rightarrow aDAbE_{14}$\\
$E_{14} \rightarrow E_1TAb$\\
$E_{14} \rightarrow \varepsilon$\\
$E_7 \rightarrow bB_1TAbAbE_{15}$\\
$E_7 \rightarrow bB_1TAbAb$\\
$E_{15} \rightarrow \varepsilon$\\
$E_{15} \rightarrow E_1TAb$\\
$E_7 \rightarrow \varepsilon$\\
$B \rightarrow b$\\
$T \rightarrow b$\\
$D \rightarrow b$\\

Дальнейшее преобразование не имеет смысла, так как 
повторная факторизация и замена, например, правило\\
$E_{13} \rightarrow E_1TAb$\\
приведёт к правилу с той же правой частью:\\
$E_{13} \rightarrow aDAb$\\
$E_{13} \rightarrow aDAbE_1TAb$\\
$E_{13} \rightarrow bB_1TAbAb$\\
$E_{13} \rightarrow bB_1TAbAbE_1TAb$\\

$E_{13} \rightarrow aDAbE_{14}$\\
$E_{13} \rightarrow bB_1TAbAbE_{14}$\\
$\mathbf{E_{14} \rightarrow E_1TAb}$\\
$E_{14} \rightarrow \varepsilon$\\

Алгоритм зациклился. Искомая грамматика $G_8$, эквивалентная $G_1$ недостижима.
\fi

\item Получить грамматику $G_9$, эквивалентную грамматике $G_1$, в которой
правая часть каждого правила не содержит двух стоящих рядом нетерминала.

Для получения такой грамматики преобразуем грамматику $G_7$ к операторной КС-грамматике.\\
Исходная грамматика:\\
$B_1 \rightarrow aDAb$\\
$B_1 \rightarrow bB_1TAbAb$\\
$B_1 \rightarrow aDAbB_1$\\
$B_1 \rightarrow bB_1TAbAbB_1$\\
$T \rightarrow aD$\\ 
$T \rightarrow bB_1TAb$\\
$D \rightarrow bB_1$\\
$A \rightarrow bB_1TAbbb$\\
$A \rightarrow bTAbbb$\\
$A \rightarrow bbb$\\
$B \rightarrow bB_1TAb$\\ 
$B \rightarrow bTAb$\\ 
$B \rightarrow b$\\
$T \rightarrow b$\\
$D \rightarrow b$\\

Введём операторные правила:\\
$B_1 \rightarrow aN_1b$\\
$B_1 \rightarrow bN_2bAb$\\
$B_1 \rightarrow aN_1bB_1$\\
$B_1 \rightarrow bN_2bAbB_1$\\
$T \rightarrow aD$\\ 
$T \rightarrow bN_2b$\\
$D \rightarrow bB_1$\\
$A \rightarrow bN_2bN_4$\\
$A \rightarrow bN_3bN_4$\\
$A \rightarrow bN_4$\\
$B \rightarrow bN_2b$\\ 
$B \rightarrow bN_3b$\\ 
$B \rightarrow b$\\
$T \rightarrow b$\\
$D \rightarrow b$\\
$N_1 \rightarrow DA$\\
$N_2 \rightarrow B_1TA$\\
$N_3 \rightarrow TA$\\
$N_4 \rightarrow N_5b$\\
$N_5 \rightarrow b$\\

Выполним преобразования операторных правил:\\
$B_1 \rightarrow aN_1b$\\
$B_1 \rightarrow bN_2bAb$\\
$B_1 \rightarrow aN_1bB_1$\\
$B_1 \rightarrow bN_2bAbB_1$\\
$T \rightarrow aD$\\ 
$T \rightarrow bN_2b$\\
$D \rightarrow bB_1$\\
$A \rightarrow bN_2bN_4$\\
$A \rightarrow bN_3bN_4$\\
$A \rightarrow bN_4$\\
$B \rightarrow bN_2b$\\ 
$B \rightarrow bN_3b$\\ 
$B \rightarrow b$\\
$T \rightarrow b$\\
$D \rightarrow b$\\
$N_1 \rightarrow DbN_2bN_4$\\
$N_1 \rightarrow DbN_3bN_4$\\
$N_1 \rightarrow DbN_4$\\
$N_2 \rightarrow B_1aDbN_2bN_4$\\
$N_2 \rightarrow B_1aDbN_3bN_4$\\
$N_2 \rightarrow B_1aDbN_4$\\
$N_2 \rightarrow B_1bN_2bA$\\
$N_3 \rightarrow TbN_2bN_4$\\
$N_3 \rightarrow TbN_3bN_4$\\
$N_3 \rightarrow TbN_4$\\
$N_4 \rightarrow N_5b$\\
$N_5 \rightarrow b$\\

Искомая грамматика $G_9$:\\
1. $B_1 \rightarrow aN_1b$\\
2. $B_1 \rightarrow bN_2bAb$\\
3. $B_1 \rightarrow aN_1bB_1$\\
4. $B_1 \rightarrow bN_2bAbB_1$\\
5. $T \rightarrow aD$\\ 
6. $T \rightarrow bN_2b$\\
7. $D \rightarrow bB_1$\\
8. $A \rightarrow bN_2bN_4$\\
9. $A \rightarrow bN_3bN_4$\\
10. $A \rightarrow bN_4$\\
11. $B \rightarrow bN_2b$\\ 
12. $B \rightarrow bN_3b$\\ 
13. $B \rightarrow b$\\
14. $T \rightarrow b$\\
15. $D \rightarrow b$\\
16. $N_1 \rightarrow DbN_2bN_4$\\
17. $N_1 \rightarrow DbN_3bN_4$\\
18. $N_1 \rightarrow DbN_4$\\
19. $N_2 \rightarrow B_1aDbN_2bN_4$\\
20. $N_2 \rightarrow B_1aDbN_3bN_4$\\
21. $N_2 \rightarrow B_1aDbN_4$\\
22. $N_2 \rightarrow B_1bN_2bA$\\
23. $N_3 \rightarrow TbN_2bN_4$\\
24. $N_3 \rightarrow TbN_3bN_4$\\
25. $N_3 \rightarrow TbN_4$\\
26. $N_4 \rightarrow N_5b$\\
27. $N_5 \rightarrow b$\\

\item Получить грамматику $G_{10}$, эквивалентную грамматике $G_1$, в которой
любой символ занимает либо только крайнюю правую позицию в правых частях правил, 
либо находится левее самого правого символа в правых частях правил.

Возьмём грамматику $G_2$:\\
1. $T \rightarrow aD$\\
2. $T \rightarrow D$\\
3. $D \rightarrow DTAb$\\
4. $D \rightarrow b$\\
5. $A \rightarrow Bbb$\\
6. $B \rightarrow D$\\

Введём одиночные правила $N_1 \rightarrow D$ и $N_2 \rightarrow b$ и выполним замену там где D и b находится не в крайней правой позиции:\\
$T \rightarrow aD$\\
$T \rightarrow D$\\
$D \rightarrow N_1TAb$\\
$D \rightarrow b$\\
$A \rightarrow BN_2b$\\
$B \rightarrow D$\\
$N_1 \rightarrow D$\\
$N_2 \rightarrow b$\\

Искомая грамматика $G_{10}$:\\
1. $T \rightarrow aD$\\
2. $T \rightarrow D$\\
3. $D \rightarrow N_1TAb$\\
4. $D \rightarrow b$\\
5. $A \rightarrow BN_2b$\\
6. $B \rightarrow D$\\
7. $N_1 \rightarrow D$\\
8. $N_2 \rightarrow b$\\

\end{enumerate}

\textbf{Вывод: } в ходе лабораторной работы изучили основные эквивалентные преобразования
КС-грамматик и научились применять их для получения
КС-грамматик, обладающих заданными свойствами.

\end{document}