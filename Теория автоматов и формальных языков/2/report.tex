\documentclass[a4paper,14pt]{extarticle}


\usepackage[english,russian]{babel}
\usepackage[T2A]{fontenc}
\usepackage[utf8]{inputenc}
\usepackage{ragged2e}
\usepackage[utf8]{inputenc}
\usepackage{hyperref}
\usepackage{minted}
\setmintedinline{frame=lines, framesep=2mm, baselinestretch=1.5, bgcolor=LightGray, breaklines,fontsize=\scriptsize}
\setminted{frame=lines, framesep=2mm, baselinestretch=1.5, bgcolor=LightGray, breaklines,fontsize=\scriptsize}
\usepackage{xcolor}
\definecolor{LightGray}{gray}{0.9}
\usepackage{graphicx}
\usepackage[export]{adjustbox}
\usepackage[left=1cm,right=1cm, top=1cm,bottom=1cm,bindingoffset=0cm]{geometry}
\usepackage{fontspec}
\usepackage{ upgreek }
\usepackage[shortlabels]{enumitem}
\usepackage{adjustbox}
\usepackage{multirow}
\usepackage{amsmath}
\usepackage{amssymb}
\usepackage{pifont}
\usepackage{pgfplots}
\usepackage{longtable}
\usepackage{array}
\graphicspath{ {./images/} }
\makeatletter
\AddEnumerateCounter{\asbuk}{\russian@alph}{щ}
\makeatother
\setmonofont{Consolas}
\setmainfont{Times New Roman}

\newcommand\textbox[1]{
	\parbox{.45\textwidth}{#1}
} 

\newcommand{\specialcell}[2][c]{%
	\begin{tabular}[#1]{@{}c@{}}#2\end{tabular}}

\begin{document}
\pagenumbering{gobble}
\begin{center}
    \small{
        \textbf{МИНИCТЕРCТВО НАУКИ И ВЫCШЕГО ОБРАЗОВАНИЯ РОCCИЙCКОЙ ФЕДЕРАЦИИ}\\
        ФЕДЕРАЛЬНОЕ ГОCУДАРCТВЕННОЕ БЮДЖЕТНОЕ ОБРАЗОВАТЕЛЬНОЕ УЧРЕЖДЕНИЕ\\ВЫCШЕГО ОБРАЗОВАНИЯ \\
        \textbf{«БЕЛГОРОДCКИЙ ГОCУДАРCТВЕННЫЙ ТЕХНОЛОГИЧЕCКИЙ\\УНИВЕРCИТЕТ им. В. Г. ШУХОВА»\\ (БГТУ им. В.Г. Шухова)} \\
        \bigbreak
        \includegraphics[width=70mm]{log}\\
        ИНСТИТУТ ИНФОРМАЦИОННЫХ ТЕХНОЛОГИЙ И УПРАВЛЯЮЩИХ СИСТЕМ\\}
\end{center}

\vfill
\begin{center}
    \large{
        \textbf{
            Лабораторная работа №2}}\\
    \normalsize{
        по дисциплине: Теория автоматов и формальных языков \\
        тема: «Преобразования КС-грамматик.»}
\end{center}
\vfill
\hfill\textbox{
    Выполнил: ст. группы ПВ-223\\Пахомов Владислав Андреевич
    \bigbreak
    Проверили: \\ст. пр. Рязанов Юрий Дмитриевич
}
\vfill\begin{center}
    Белгород 2024 г.
\end{center}
\newpage
\begin{center}
    \textbf{Лабораторная работа №2}\\
    Преобразования КС-грамматик.\\
    Вариант 8
\end{center}
\textbf{Цель работы: }изучить основные эквивалентные преобразования
КС-грамматик и научиться применять их для получения
КС-грамматик, обладающих заданными свойствами.\\
\textbf{Задание: }\\
    1. $T \rightarrow abETP$\\
    2. $T \rightarrow aDE$\\
    3. $T \rightarrow D$\\
    4. $D \rightarrow DTAb$\\
    5. $D \rightarrow b$\\
    6. $E \rightarrow \varepsilon$\\
    7. $P \rightarrow BCa$\\
    8. $P \rightarrow Cb$\\
    9. $C \rightarrow abC$\\
    10. $A \rightarrow Bbb$\\
    11. $B \rightarrow aECb$\\
    12. $B \rightarrow D$\\

\begin{enumerate}[1.]
    \item Преобразовать исходную грамматику $G$ в
    грамматику $G_1$ без лишних символов.\\
    \textbf{Модификации: }в ходе выполнения лабораторной работы обнаружено, что в грамматике
    не будет недостижимых символов. Поэтому добавим правило:\\
    
    13. $S \rightarrow ab$


    Найдём в исходной грамматике бесплодные нетерминалы.\\
    Для начала найдём продуктивные нетерминалы.\\
    В множество продуктивных нетерминалов Р включаем нетерминал D
(правило 5) нетерминал E (правило 6) и нетерминал S (правило 13). Получаем $Р=\{D,E,S\}$.
Повторяем проверку и включаем нетерминал T (правило 2) и нетерминал B (правило 12). 
Получаем $P=\{D,E,S,T,B\}$\\
Повторяем проверку и включаем A (правило 10). Получаем $P=\{D,E,S,T,B,A\}$\\
Множество P больше увеличить не можем.\\

Из множества нетерминалов исключаем продуктивные 
нетерминалы и получаем $\{P,C\}$ - множество бесплодных нетерминалов.\\

Исключаем правила 1, 7, 8, 9, 11 так как они содержат 
бесплодные нетерминалы. Получаем грамматику:\\

2. $T \rightarrow aDE$\\
3. $T \rightarrow D$\\
4. $D \rightarrow DTAb$\\
5. $D \rightarrow b$\\
6. $E \rightarrow \varepsilon$\\
10. $A \rightarrow Bbb$\\
12. $B \rightarrow D$\\
13. $S \rightarrow ab$\\

Найдём достижимые символы.\\
Положим $P = \{T\}$, где T - начальный нетерминал.\\
Включим в список a, D, E (правило 2). $P = \{T, a, D, E\}$.\\
Включим в список b, A (правило 4), $\varepsilon$. $P = \{T, a, D, E, \varepsilon, b, A\}$.\\
Включим в список B (правило 10). $P = \{T, a, D, E, \varepsilon, b, A, B\}$.\\
Множество P больше увеличить не можем.\\

Из множества терминалов и нетерминалов исключаем достижимые 
терминалы и нетерминалы и получаем $\{S\}$ - множество недостижимых нетерминалов и терминалов.\\

Исключаем из грамматики правило 13, так как оно содержит недостижимый символ.\\
Искомая грамматика $G_1$:\\
1. $T \rightarrow aDE$\\
2. $T \rightarrow D$\\
3. $D \rightarrow DTAb$\\
4. $D \rightarrow b$\\
5. $E \rightarrow \varepsilon$\\
6. $A \rightarrow Bbb$\\
7. $B \rightarrow D$\\

\item Преобразовать грамматику $G_1$ в грамматику $G_2$ без $\varepsilon$-правил.\\
Выберем правило 5. Иключаем из правой части каждого правила исходной грамматики
всеми возможными способами вхождение нетерминала E. Полученные правила добавляем в множество 
правил грамматики.\\
1\_1. $T \rightarrow aDE$\\
1\_2. $T \rightarrow aD$\\
2. $T \rightarrow D$\\
3. $D \rightarrow DTAb$\\
4. $D \rightarrow b$\\
5. $E \rightarrow \varepsilon$\\
6. $A \rightarrow Bbb$\\
7. $B \rightarrow D$\\
Исключаем из списка правил правило 5.\\
1\_1. $T \rightarrow aDE$\\
1\_2. $T \rightarrow aD$\\
2. $T \rightarrow D$\\
3. $D \rightarrow DTAb$\\
4. $D \rightarrow b$\\
6. $A \rightarrow Bbb$\\
7. $B \rightarrow D$\\
Исключим из правил непродуктивные символы:\\
1\_2. $T \rightarrow aD$\\
2. $T \rightarrow D$\\
3. $D \rightarrow DTAb$\\
4. $D \rightarrow b$\\
6. $A \rightarrow Bbb$\\
7. $B \rightarrow D$\\
В полученной грамматике $G_2$ нет правил вида $A \rightarrow A$, одинаковых правил и $\varepsilon$-правил. Получили искомую грамматику:\\
Искомая грамматика $G_2$:\\
1. $T \rightarrow aD$\\
2. $T \rightarrow D$\\
3. $D \rightarrow DTAb$\\
4. $D \rightarrow b$\\
6. $A \rightarrow Bbb$\\
7. $B \rightarrow D$\\

\item Преобразовать грамматику $G_1$ в грамматику $G_3$ без цепных правил.

Исходная грамматика:\\
1. $T \rightarrow aDE$\\
2. $T \rightarrow D$\\
3. $D \rightarrow DTAb$\\
4. $D \rightarrow b$\\
5. $E \rightarrow \varepsilon$\\
6. $A \rightarrow Bbb$\\
7. $B \rightarrow D$\\

Заменим символ T в правиле 3 символом D согласно правилу 2:\\
1. $T \rightarrow aDE$\\
3\_1. $D \rightarrow DTAb$\\
3\_2. $D \rightarrow DDAb$\\
4. $D \rightarrow b$\\
5. $E \rightarrow \varepsilon$\\
6. $A \rightarrow Bbb$\\
7. $B \rightarrow D$\\

Заменим символ B в правиле 6 символом D согласно правилу 7:\\
1. $T \rightarrow aDE$\\
3\_1. $D \rightarrow DTAb$\\
3\_2. $D \rightarrow DDAb$\\
4. $D \rightarrow b$\\
5. $E \rightarrow \varepsilon$\\
6\_1. $A \rightarrow Bbb$\\
6\_2. $A \rightarrow Dbb$\\

Исключим правила с бесплодными нетерминалами:\\
1. $T \rightarrow aDE$\\
3\_1. $D \rightarrow DTAb$\\
3\_2. $D \rightarrow DDAb$\\
4. $D \rightarrow b$\\
5. $E \rightarrow \varepsilon$\\
6\_2. $A \rightarrow Dbb$\\

Искомая грамматика $G_3$:\\
1. $T \rightarrow aDE$\\
2. $D \rightarrow DTAb$\\
3. $D \rightarrow DDAb$\\
4. $D \rightarrow b$\\
5. $E \rightarrow \varepsilon$\\
6. $A \rightarrow Dbb$\\

\iffalse
4. Применить алгоритм в данном случае к G1 можно. 
В грамматике нет цепных правил таких что A => +A, 
следовательно при выполнении пункта 3 мы не придём 
к тому же нетерминалу слева в правой части, 
а значит алгоритм не зациклится.
Из-за e-правил алгоритм так же может зациклится. 
Например если будем применять замену левого края
для следующий правил по алгоритму:
1. T -> EB
2. E -> e
3. B -> Ba
То в итоге алгоритм зациклится для правила 3. 

\fi

\item Преобразовать грамматику $G_1$ в грамматику $G_4$ без левой рекурсии.\\

Алгоритм применим, если грамматика не имеет циклов (цепных правил) и $\varepsilon$-правил.
Для получения грамматики без $\varepsilon$-правил воспользуемся грамматикой $G_2$.\\
1. $T \rightarrow aD$\\
2. $T \rightarrow D$\\
3. $D \rightarrow DTAb$\\
4. $D \rightarrow b$\\
6. $A \rightarrow Bbb$\\
7. $B \rightarrow D$\\
Преобразуем эту грамматику так, чтобы в ней не было цепных правил.\\

Исходная грамматика:\\
1. $T \rightarrow aD$\\
2. $\mathbf{T \rightarrow D}$\\
3. $D \rightarrow DTAb$\\
4. $D \rightarrow b$\\
6. $A \rightarrow Bbb$\\
7. $B \rightarrow D$\\

Выполним замену края:\\
1. $T \rightarrow aD$\\
2\_1. $T \rightarrow DTAb$\\
2\_2. $T \rightarrow b$\\
3. $D \rightarrow DTAb$\\
4. $D \rightarrow b$\\
6. $A \rightarrow Bbb$\\
7. $\mathbf{B \rightarrow D}$\\

Выполним замену края:\\
1. $T \rightarrow aD$\\
2\_1. $T \rightarrow DTAb$\\
2\_2. $T \rightarrow b$\\
3. $D \rightarrow DTAb$\\
4. $D \rightarrow b$\\
6. $A \rightarrow Bbb$\\
7\_1. $B \rightarrow DTAb$\\
7\_2. $B \rightarrow b$\\

Получили грамматику $G_3'$ без лишних символов, $\varepsilon$-правил и цепных правил:\\
1. $T \rightarrow aD$\\
2. $T \rightarrow DTAb$\\
3. $T \rightarrow b$\\
4. $D \rightarrow DTAb$\\
5. $D \rightarrow b$\\
6. $A \rightarrow Bbb$\\
7. $B \rightarrow DTAb$\\
8. $B \rightarrow b$\\

Обозначим нетерминалы грамматики: T, D, A, B как $A_1, A_2, A_3, A_4$ соответственно.\\
1. $A_1 \rightarrow aA_2$\\
2. $A_1 \rightarrow A_2A_1A_3b$\\
3. $A_1 \rightarrow b$\\
4. $A_2 \rightarrow A_2A_1A_3b$\\
5. $A_2 \rightarrow b$\\
6. $A_3 \rightarrow A_4bb$\\
7. $A_4 \rightarrow A_2A_1A_3b$\\
8. $A_4 \rightarrow b$\\

Рассмотрим нетерминал $A_1$.\\
Правил вида $A_1 \rightarrow A_0a$ не существует, 
следовательно замену края выполнять не будем.\\
Самолеворекурсивных правил для $A_1$ также нет.\\

Рассмотрим нетерминал $A_2$.\\
Правил вида $A_2 \rightarrow A_1a$ не существует, 
следовательно замену края выполнять не будем.\\
Для $A_2$ существует самолеворекурсивное правило 4. 
Также существует несаморекурсивное правило 5. 
Заменим эти правила:\\
1. $A_1 \rightarrow aA_2$\\
2. $A_1 \rightarrow A_2A_1A_3b$\\
3. $A_1 \rightarrow b$\\
9. $A_2 \rightarrow bB_1$\\
10. $B_1 \rightarrow A_1A_3bB_1$\\
11. $B_1 \rightarrow \varepsilon$\\
6. $A_3 \rightarrow A_4bb$\\
7. $A_4 \rightarrow A_2A_1A_3b$\\
8. $A_4 \rightarrow b$\\

Рассмотрим нетерминал $A_3$.\\
Правил вида $A_3 \rightarrow A_2a$ не существует, 
следовательно замену края выполнять не будем.\\
Самолеворекурсивных правил для $A_3$ также нет.\\

Рассмотрим нетерминал $A_4$.\\
Существует правило 7. $A_4 \rightarrow A_2A_1A_3b$, выполним замену края:\\
1. $A_1 \rightarrow aA_2$\\
2. $A_1 \rightarrow A_2A_1A_3b$\\
3. $A_1 \rightarrow b$\\
9. $A_2 \rightarrow bB_1$\\
10. $B_1 \rightarrow A_1A_3bB_1$\\
11. $B_1 \rightarrow \varepsilon$\\
6. $A_3 \rightarrow A_4bb$\\
12. $A_4 \rightarrow bB_1A_1A_3b$\\
8. $A_4 \rightarrow b$\\

Искомая грамматика $G_4$: \\
1. $T \rightarrow aD$\\
2. $T \rightarrow DTAb$\\
3. $T \rightarrow b$\\
4. $D \rightarrow bB_1$\\
5. $B_1 \rightarrow TAbB_1$\\
6. $B_1 \rightarrow \varepsilon$\\
7. $A \rightarrow Bbb$\\
8. $B \rightarrow bB_1TAb$\\
9. $B \rightarrow b$\\

\iffalse
\item Преобразовать грамматику $G_1$ в грамматику $G_5$ без несаморекурсивных
нетерминалов.

Искходная грамматика:\\
1. $T \rightarrow aDE$\\
2. $T \rightarrow D$\\
3. $D \rightarrow DTAb$\\
4. $D \rightarrow b$\\
5. $E \rightarrow \varepsilon$\\
6. $A \rightarrow Bbb$\\
7. $B \rightarrow D$\\

Нетерминал E несаморекурсивный.\\
Исключаем правило 5:\\
5. $E \rightarrow \varepsilon$\\
Выбираем вхождение символа E в правиле 1 и
выполняем замену на правую часть правила 5:\\
1\_1. $T \rightarrow aD$\\
2. $T \rightarrow D$\\
3. $D \rightarrow DTAb$\\
4. $D \rightarrow b$\\
6. $A \rightarrow Bbb$\\
7. $B \rightarrow D$\\

Нетерминал T несаморекурсивный.\\
Исключаем правила 1\_1, 2:\\
1\_1. $T \rightarrow aD$\\
2. $T \rightarrow D$\\
Выбираем вхождение символа T в правиле 3 и
выполняем замену на правую часть правил 1\_1, 2:\\
3\_1. $D \rightarrow DaDAb$\\
3\_2. $D \rightarrow DDAb$\\
4. $D \rightarrow b$\\
6. $A \rightarrow Bbb$\\
7. $B \rightarrow D$\\

Нетерминал B несаморекурсивный.\\
Исключаем правило 7:\\
7. $B \rightarrow D$\\
Выбираем вхождение символа B в правиле 6 и
выполняем замену на правую часть правила 7:\\
3\_1. $D \rightarrow DaDAb$\\
3\_2. $D \rightarrow DDAb$\\
4. $D \rightarrow b$\\
6\_1. $A \rightarrow Dbb$\\

Нетерминал A несаморекурсивный.\\
Исключаем правило 6\_1:\\
6\_1. $A \rightarrow Dbb$\\
Выбираем вхождение символа A в правилах 3\_1, 3\_2 и
выполняем замену на правую часть правила 6\_1:\\
3\_1\_1. $D \rightarrow DaDDbbb$\\
3\_2\_2. $D \rightarrow DDDbbb$\\
4. $D \rightarrow b$\\

Искомая грамматика $G_5$:\\
1. $D \rightarrow DaDDbbb$\\
2. $D \rightarrow DDDbbb$\\
3. $D \rightarrow b$\\

\item Получить грамматику $G_6$, эквивалентную грамматике $G_1$, в которой
правая часть каждого правила состоит либо из одного терминала, либо двух нетерминалов.

Для получения грамматики $G_6$ необходимо привести грамматику $G_1$ 
к нормальной форме Хомского.\\
Воспользуемся грамматикой $G_3'$, в которой нет цепных правил, $\varepsilon$-правил
и цепных правил.\\
Исходная грамматика:\\
1. $T \rightarrow aD$\\
2. $T \rightarrow DTAb$\\
3. $T \rightarrow b$\\
4. $D \rightarrow DTAb$\\
5. $D \rightarrow b$\\
6. $A \rightarrow Bbb$\\
7. $B \rightarrow DTAb$\\
8. $B \rightarrow b$\\

Выполним пункт 1 алгоритма (преобразование правил вида $A \rightarrow Xa$):\\
1. $T \rightarrow aD$\\
2. $T \rightarrow DN_1$\\
3. $T \rightarrow b$\\
4. $D \rightarrow DN_1$\\
5. $D \rightarrow b$\\
6. $A \rightarrow Bbb$\\
7. $B \rightarrow DN_1$\\
8. $B \rightarrow b$\\
9. $N_1 \rightarrow TAb$\\

1. $T \rightarrow aD$\\
2. $T \rightarrow DN_1$\\
3. $T \rightarrow b$\\
4. $D \rightarrow DN_1$\\
5. $D \rightarrow b$\\
6. $A \rightarrow BN_2$\\
7. $B \rightarrow DN_1$\\
8. $B \rightarrow b$\\
9. $N_1 \rightarrow TAb$\\
10. $N_2 \rightarrow bb$\\

1. $T \rightarrow aD$\\
2. $T \rightarrow DN_1$\\
3. $T \rightarrow b$\\
4. $D \rightarrow DN_1$\\
5. $D \rightarrow b$\\
6. $A \rightarrow BN_2$\\
7. $B \rightarrow DN_1$\\
8. $B \rightarrow b$\\
9. $N_1 \rightarrow N_3b$\\
10. $N_2 \rightarrow bb$\\
11. $N_3 \rightarrow TA$\\

Выполним пункт 2 алгоритма (преобразование правил вида $A \rightarrow tB$):\\
1. $T \rightarrow N_4D$\\
2. $T \rightarrow DN_1$\\
3. $T \rightarrow b$\\
4. $D \rightarrow DN_1$\\
5. $D \rightarrow b$\\
6. $A \rightarrow BN_2$\\
7. $B \rightarrow DN_1$\\
8. $B \rightarrow b$\\
9. $N_1 \rightarrow N_3b$\\
10. $N_2 \rightarrow bb$\\
11. $N_3 \rightarrow TA$\\
12. $N_4 \rightarrow a$\\

Выполним пункт 3 алгоритма (преобразование правил вида $A \rightarrow Bt$):\\
1. $T \rightarrow N_4D$\\
2. $T \rightarrow DN_1$\\
3. $T \rightarrow b$\\
4. $D \rightarrow DN_1$\\
5. $D \rightarrow b$\\
6. $A \rightarrow BN_2$\\
7. $B \rightarrow DN_1$\\
8. $B \rightarrow b$\\
9. $N_1 \rightarrow N_3T$\\
10. $N_2 \rightarrow bb$\\
11. $N_3 \rightarrow TA$\\
12. $N_4 \rightarrow a$\\

Выполним пункт 4 алгоритма (преобразование правил вида $A \rightarrow tt$):\\
1. $T \rightarrow N_4D$\\
2. $T \rightarrow DN_1$\\
3. $T \rightarrow b$\\
4. $D \rightarrow DN_1$\\
5. $D \rightarrow b$\\
6. $A \rightarrow BN_2$\\
7. $B \rightarrow DN_1$\\
8. $B \rightarrow b$\\
9. $N_1 \rightarrow N_3T$\\
10. $N_2 \rightarrow TT$\\
11. $N_3 \rightarrow TA$\\
12. $N_4 \rightarrow a$\\

Искомая грамматика $G_6$:\\
1. $T \rightarrow N_4D$\\
2. $T \rightarrow DN_1$\\
3. $T \rightarrow b$\\
4. $D \rightarrow DN_1$\\
5. $D \rightarrow b$\\
6. $A \rightarrow BN_2$\\
7. $B \rightarrow DN_1$\\
8. $B \rightarrow b$\\
9. $N_1 \rightarrow N_3T$\\
10. $N_2 \rightarrow TT$\\
11. $N_3 \rightarrow TA$\\
12. $N_4 \rightarrow a$\\

\item Получить грамматику $G_7$, эквивалентную грамматике $G_1$, в которой 
правая часть каждого правила начинается терминалом.

Для получения грамматики $G_7$ 
необходимо привести грамматику $G_1$ 
к нормальной форме Грейбах.\\

Используем преобразованную грамматику $G_1$ без левой рекурсии $G_4$:\\

В $G_4$ есть $\varepsilon$-правила. Исключим их и получим грамматику $G_4'$:\\
1. $T \rightarrow aD$\\
2. $T \rightarrow DTAb$\\
3. $T \rightarrow b$\\
4. $D \rightarrow bB_1$\\
5. $B_1 \rightarrow TAbB_1$\\
6. $B_1 \rightarrow \varepsilon$\\
7. $A \rightarrow Bbb$\\
8. $B \rightarrow bB_1TAb$\\
9. $B \rightarrow b$\\

1. $T \rightarrow aD$\\
2. $T \rightarrow DTAb$\\
3. $T \rightarrow b$\\
4\_1. $D \rightarrow bB_1$\\
4\_2. $D \rightarrow b$\\
5\_1. $B_1 \rightarrow TAbB_1$\\
5\_2. $B_1 \rightarrow TAb$\\
7. $A \rightarrow Bbb$\\
8\_1. $B \rightarrow bB_1TAb$\\
8\_2. $B \rightarrow bTAb$\\
9. $B \rightarrow b$\\

\iffalse
1. $T \rightarrow aD$\\              T  _  D
2. $T \rightarrow DTAb$\\            T  D  _
3. $T \rightarrow b$\\               T  _  _
4\_1. $D \rightarrow bB_1$\\         D  _  B1
4\_2. $D \rightarrow b$\\            D  _  _
5\_1. $B_1 \rightarrow TAbB_1$\\     B1 T  B1
5\_2. $B_1 \rightarrow TAb$\\        B1 T  _
7. $A \rightarrow Bbb$\\             A  B  _
8\_1. $B \rightarrow bB_1TAb$\\      B  _  _
8\_2. $B \rightarrow bTAb$\\         B  _  _
9. $B \rightarrow b$\\               B  _  _

B1 T  _
B1 T  B1
T  _  D
T  D  _
D  _  B1
A  B  _
B  _  _
B  _  _
B  _  _
T  _  _
D  _  _
\fi

Упорядочим грамматику:\\
1. $B_1 \rightarrow TAb$\\
2. $B_1 \rightarrow TAbB_1$\\
3. $T \rightarrow aD$\\ 
4. $T \rightarrow DTAb$\\
5. $D \rightarrow bB_1$\\
6. $A \rightarrow Bbb$\\
7. $B \rightarrow bB_1TAb$\\ 
8. $B \rightarrow bTAb$\\ 
9. $B \rightarrow b$\\
10. $T \rightarrow b$\\
11. $D \rightarrow b$\\

Выполнение замены края:\\
1. $B_1 \rightarrow TAb$\\
2. $B_1 \rightarrow TAbB_1$\\
3. $T \rightarrow aD$\\ 
4. $T \rightarrow DTAb$\\
5. $D \rightarrow bB_1$\\
6\_1. $A \rightarrow bB_1TAbbb$\\
6\_2. $A \rightarrow bTAbbb$\\
6\_3. $A \rightarrow bbb$\\
7. $B \rightarrow bB_1TAb$\\ 
8. $B \rightarrow bTAb$\\ 
9. $B \rightarrow b$\\
10. $T \rightarrow b$\\
11. $D \rightarrow b$\\

1. $B_1 \rightarrow TAb$\\
2. $B_1 \rightarrow TAbB_1$\\
3. $T \rightarrow aD$\\ 
4. $T \rightarrow bB_1TAb$\\
5. $D \rightarrow bB_1$\\
6\_1. $A \rightarrow bB_1TAbbb$\\
6\_2. $A \rightarrow bTAbbb$\\
6\_3. $A \rightarrow bbb$\\
7. $B \rightarrow bB_1TAb$\\ 
8. $B \rightarrow bTAb$\\ 
9. $B \rightarrow b$\\
10. $T \rightarrow b$\\
11. $D \rightarrow b$\\

1\_1. $B_1 \rightarrow aDAb$\\
1\_2. $B_1 \rightarrow bB_1TAbAb$\\
2\_1. $B_1 \rightarrow aDAbB_1$\\
2\_2. $B_1 \rightarrow bB_1TAbAbB_1$\\
3. $T \rightarrow aD$\\ 
4. $T \rightarrow bB_1TAb$\\
5. $D \rightarrow bB_1$\\
6\_1. $A \rightarrow bB_1TAbbb$\\
6\_2. $A \rightarrow bTAbbb$\\
6\_3. $A \rightarrow bbb$\\
7. $B \rightarrow bB_1TAb$\\ 
8. $B \rightarrow bTAb$\\ 
9. $B \rightarrow b$\\
10. $T \rightarrow b$\\
11. $D \rightarrow b$\\

Искомая грамматика $G_7$:\\
1. $B_1 \rightarrow aDAb$\\
2. $B_1 \rightarrow bB_1TAbAb$\\
3. $B_1 \rightarrow aDAbB_1$\\
4. $B_1 \rightarrow bB_1TAbAbB_1$\\
5. $T \rightarrow aD$\\ 
6. $T \rightarrow bB_1TAb$\\
7. $D \rightarrow bB_1$\\
8. $A \rightarrow bB_1TAbbb$\\
9. $A \rightarrow bTAbbb$\\
10. $A \rightarrow bbb$\\
11. $B \rightarrow bB_1TAb$\\ 
12. $B \rightarrow bTAb$\\ 
13. $B \rightarrow b$\\
14. $T \rightarrow b$\\
15. $D \rightarrow b$\\

\item Получить грамматику $G_8$, эквивалентную грамматике $G_1$, в которой 
правая часть каждого не $\varepsilon$-правила начинается терминалом и любые
два правила с одинаковой левой частью различаются первым символом в правой части.\\
Для получения такой грамматики можем проводить множественную левую факторизацию и замену в грамматике $G_7$.\\

\textbf{Модификации: }в ходе выполнения задания было выявлено, что грамматика $G_7$
преобразовать к искомой невозможно, так как алгоримт зациклился.
Попробуем удалить из грамматики $G_7$ правила 2, 3, 4, 5.\\
$B_1 \rightarrow aDAb$\\
$T \rightarrow bB_1TAb$\\
$D \rightarrow bB_1$\\
$A \rightarrow bB_1TAbbb$\\
$A \rightarrow bTAbbb$\\
$A \rightarrow bbb$\\
$B \rightarrow bB_1TAb$\\ 
$B \rightarrow bTAb$\\ 
$B \rightarrow b$\\
$T \rightarrow b$\\
$D \rightarrow b$\\

Выполним левую факторизацию:\\
$B_1 \rightarrow aDAb$\\
$T \rightarrow bB_1TAb$\\
$D \rightarrow bB_1$\\
$A \rightarrow bE_1$\\
$E_1 \rightarrow B_1TAbbb$\\
$E_1 \rightarrow TAbbb$\\
$E_1 \rightarrow bb$\\ 
$B \rightarrow bE_2$\\
$E_2 \rightarrow B_1TAb$\\
$E_2 \rightarrow TAb$\\
$E_2 \rightarrow \varepsilon$\\
$T \rightarrow b$\\
$D \rightarrow b$\\

Выполним замену:\\
$B_1 \rightarrow aDAb$\\
$T \rightarrow bB_1TAb$\\
$D \rightarrow bB_1$\\
$A \rightarrow bE_1$\\
$E_1 \rightarrow aDAbTAbbb$\\
$E_1 \rightarrow bB_1TAbAbbb$\\
$E_1 \rightarrow bb$\\ 
$B \rightarrow bE_2$\\
$E_2 \rightarrow aDAbTAb$\\
$E_2 \rightarrow bB_1TAbAb$\\
$E_2 \rightarrow \varepsilon$\\
$T \rightarrow b$\\
$D \rightarrow b$\\

Выполним левую факторизацию:\\
$B_1 \rightarrow aDAb$\\
$T \rightarrow bB_1TAb$\\
$D \rightarrow bB_1$\\
$A \rightarrow bE_1$\\
$E_1 \rightarrow aDAbTAbbb$\\
$E_1 \rightarrow bE_3$\\ 
$E_3 \rightarrow B_1TAbAbbb$\\
$E_3 \rightarrow b$\\
$B \rightarrow bE_2$\\
$E_2 \rightarrow aDAbTAb$\\
$E_2 \rightarrow bB_1TAbAb$\\
$E_2 \rightarrow \varepsilon$\\
$T \rightarrow b$\\
$D \rightarrow b$\\

Выполним замену:\\
$B_1 \rightarrow aDAb$\\
$T \rightarrow bB_1TAb$\\
$D \rightarrow bB_1$\\
$A \rightarrow bE_1$\\
$E_1 \rightarrow aDAbTAbbb$\\
$E_1 \rightarrow bE_3$\\ 
$E_3 \rightarrow aDAbTAbAbbb$\\
$E_3 \rightarrow b$\\
$B \rightarrow bE_2$\\
$E_2 \rightarrow aDAbTAb$\\
$E_2 \rightarrow bB_1TAbAb$\\
$E_2 \rightarrow \varepsilon$\\
$T \rightarrow b$\\
$D \rightarrow b$\\

Искомая грамматика $G_8$:\\
$B_1 \rightarrow aDAb$\\
$T \rightarrow bB_1TAb$\\
$D \rightarrow bB_1$\\
$A \rightarrow bE_1$\\
$B \rightarrow bE_2$\\
$T \rightarrow b$\\
$D \rightarrow b$\\
$E_1 \rightarrow aDAbTAbbb$\\
$E_1 \rightarrow bE_3$\\ 
$E_2 \rightarrow aDAbTAb$\\
$E_2 \rightarrow bB_1TAbAb$\\
$E_2 \rightarrow \varepsilon$\\
$E_3 \rightarrow aDAbTAbAbbb$\\
$E_3 \rightarrow b$\\

\iffalse
Исходная грамматика:\\
$B_1 \rightarrow aDAb$\\
$B_1 \rightarrow bB_1TAbAb$\\
$B_1 \rightarrow aDAbB_1$\\
$B_1 \rightarrow bB_1TAbAbB_1$\\
$T \rightarrow aD$\\ 
$T \rightarrow bB_1TAb$\\
$D \rightarrow bB_1$\\
$A \rightarrow bB_1TAbbb$\\
$A \rightarrow bTAbbb$\\
$A \rightarrow bbb$\\
$B \rightarrow bB_1TAb$\\ 
$B \rightarrow bTAb$\\ 
$B \rightarrow b$\\
$T \rightarrow b$\\
$D \rightarrow b$\\

Выполним левую факторизацию для $B_1$:\\
$B_1 \rightarrow aDAb$\\
$B_1 \rightarrow bB_1TAbAb$\\
$B_1 \rightarrow aDAbB_1$\\
$B_1 \rightarrow bB_1TAbAbB_1$\\
$T \rightarrow aD$\\ 
$T \rightarrow bB_1TAb$\\
$D \rightarrow bB_1$\\
$A \rightarrow bB_1TAbbb$\\
$A \rightarrow bTAbbb$\\
$A \rightarrow bbb$\\
$B \rightarrow bB_1TAb$\\ 
$B \rightarrow bTAb$\\ 
$B \rightarrow b$\\
$T \rightarrow b$\\
$D \rightarrow b$\\

Выполним замену:\\
$B_1 \rightarrow bB_1TAbAbE_1$\\
$B_1 \rightarrow aDAbE_1$\\
$E_1 \rightarrow \varepsilon$\\
$E_1 \rightarrow bB_1TAbAbE_1$\\
$E_1 \rightarrow aDAbE_1$\\
$T \rightarrow aD$\\ 
$T \rightarrow bB_1TAb$\\
$D \rightarrow bB_1$\\
$A \rightarrow bB_1TAbbb$\\
$A \rightarrow bTAbbb$\\
$A \rightarrow bbb$\\
$B \rightarrow bB_1TAb$\\ 
$B \rightarrow bTAb$\\ 
$B \rightarrow b$\\
$T \rightarrow b$\\
$D \rightarrow b$\\

Выполним левую факторизацию для A, B:\\
$B_1 \rightarrow bB_1TAbAbE_1$\\
$B_1 \rightarrow aDAbE_1$\\
$E_1 \rightarrow \varepsilon$\\
$E_1 \rightarrow bB_1TAbAbE_1$\\
$E_1 \rightarrow aDAbE_1$\\
$T \rightarrow aD$\\ 
$T \rightarrow bB_1TAb$\\
$D \rightarrow bB_1$\\
$A \rightarrow bE_2$\\
$E_2 \rightarrow B_1TAbbb$\\
$E_2 \rightarrow TAbbb$\\
$E_2 \rightarrow bb$\\
$B \rightarrow bE_3$\\ 
$E_3 \rightarrow B_1TAb$\\ 
$E_3 \rightarrow TAb$\\
$B \rightarrow b$\\
$T \rightarrow b$\\
$D \rightarrow b$\\

Выполним замену:\\
$B_1 \rightarrow bB_1TAbAbE_1$\\
$B_1 \rightarrow aDAbE_1$\\
$E_1 \rightarrow \varepsilon$\\
$E_1 \rightarrow bB_1TAbAbE_1$\\
$E_1 \rightarrow aDAbE_1$\\
$T \rightarrow aD$\\ 
$T \rightarrow bB_1TAb$\\
$D \rightarrow bB_1$\\
$A \rightarrow bE_2$\\
$E_2 \rightarrow bB_1TAbAbE_1TAbbb$\\
$E_2 \rightarrow aDAbE_1TAbbb$\\
$E_2 \rightarrow aDAbbb$\\
$E_2 \rightarrow bB_1TAbAbbb$\\
$E_2 \rightarrow bb$\\
$B \rightarrow bE_3$\\ 
$E_3 \rightarrow bB_1TAbAbE_1TAb$\\ 
$E_3 \rightarrow aDAbE_1TAb$\\ 
$E_3 \rightarrow aDAb$\\
$E_3 \rightarrow bB_1TAbAb$\\
$B \rightarrow b$\\
$T \rightarrow b$\\
$D \rightarrow b$\\

Выполним левую факторизацию для $E_2$, $E_3$:\\
$B_1 \rightarrow bB_1TAbAbE_1$\\
$B_1 \rightarrow aDAbE_1$\\
$E_1 \rightarrow \varepsilon$\\
$E_1 \rightarrow bB_1TAbAbE_1$\\
$E_1 \rightarrow aDAbE_1$\\
$T \rightarrow aD$\\ 
$T \rightarrow bB_1TAb$\\
$D \rightarrow bB_1$\\
$A \rightarrow bE_2$\\
$E_2 \rightarrow aDAbE_4$\\
$E_4 \rightarrow E_1TAbbb$\\
$E_4 \rightarrow bb$\\
$E_2 \rightarrow bE_5$\\
$E_5 \rightarrow B_1TAbAbE_1TAbbb$\\
$E_5 \rightarrow B_1TAbAbbb$\\
$E_5 \rightarrow b$\\
$B \rightarrow bE_3$\\ 
$E_3 \rightarrow aDAbE_6$\\ 
$E_6 \rightarrow E_1TAb$\\ 
$E_6 \rightarrow \varepsilon$\\ 
$E_3 \rightarrow bB_1TAbAbE_7$\\
$E_7 \rightarrow E_1TAb$\\
$E_7 \rightarrow \varepsilon$\\
$B \rightarrow b$\\
$T \rightarrow b$\\
$D \rightarrow b$\\

Выполним замену для $E_4$, $E_5$, $E_6$, $E_7$:\\
$B_1 \rightarrow bB_1TAbAbE_1$\\
$B_1 \rightarrow aDAbE_1$\\
$E_1 \rightarrow \varepsilon$\\
$E_1 \rightarrow bB_1TAbAbE_1$\\
$E_1 \rightarrow aDAbE_1$\\
$T \rightarrow aD$\\ 
$T \rightarrow bB_1TAb$\\
$D \rightarrow bB_1$\\
$A \rightarrow bE_2$\\
$E_2 \rightarrow aDAbE_4$\\
$E_4 \rightarrow aDAbbb$\\
$E_4 \rightarrow bB_1TAbAbbb$\\
$E_4 \rightarrow bB_1TAbAbE_1TAbbb$\\
$E_4 \rightarrow aDAbE_1TAbbb$\\
$E_4 \rightarrow bb$\\
$E_2 \rightarrow bE_5$\\
$E_5 \rightarrow aDAbE_1TAbAbbb$\\
$E_5 \rightarrow aDAbE_1TAbAbE_1TAbbb$\\
$E_5 \rightarrow bB_1TAbAbE_1TAbAbE_1TAbbb$\\
$E_5 \rightarrow bB_1TAbAbE_1TAbAbbb$\\
$E_5 \rightarrow b$\\
$B \rightarrow bE_3$\\ 
$E_3 \rightarrow aDAbE_6$\\
$E_6 \rightarrow aDAb$\\ 
$E_6 \rightarrow bB_1TAbAb$\\ 
$E_6 \rightarrow bB_1TAbAbE_1TAb$\\ 
$E_6 \rightarrow aDAbE_1TAb$\\ 
$E_6 \rightarrow \varepsilon$\\ 
$E_3 \rightarrow bB_1TAbAbE_7$\\
$E_7 \rightarrow aDAb$\\
$E_7 \rightarrow bB_1TAbAb$\\
$E_7 \rightarrow bB_1TAbAbE_1TAb$\\
$E_7 \rightarrow aDAbE_1TAb$\\
$E_7 \rightarrow \varepsilon$\\
$B \rightarrow b$\\
$T \rightarrow b$\\
$D \rightarrow b$\\

Выполним левую факторизацию:\\
$B_1 \rightarrow bB_1TAbAbE_1$\\
$B_1 \rightarrow aDAbE_1$\\
$E_1 \rightarrow \varepsilon$\\
$E_1 \rightarrow bB_1TAbAbE_1$\\
$E_1 \rightarrow aDAbE_1$\\
$T \rightarrow aD$\\ 
$T \rightarrow bB_1TAb$\\
$D \rightarrow bB_1$\\
$A \rightarrow bE_2$\\
$E_2 \rightarrow aDAbE_4$\\
$E_4 \rightarrow aDAbE_8$\\
$E_8 \rightarrow E_1TAbbb$\\
$E_8 \rightarrow bb$\\
$E_4 \rightarrow bE_9$\\
$E_9 \rightarrow B_1TAbAbbb$\\
$E_9 \rightarrow B_1TAbAbE_1TAbbb$\\
$E_9 \rightarrow b$\\
$E_2 \rightarrow bE_5$\\
$E_5 \rightarrow aDAbE_1TAbAbE_{10}$\\
$E_{10} \rightarrow bb$\\
$E_{10} \rightarrow E_1TAbbb$\\
$E_5 \rightarrow bE_{11}$\\
$E_{11} \rightarrow B_1TAbAbE_1TAbAbE_1TAbbb$\\
$E_{11} \rightarrow B_1TAbAbE_1TAbAbbb$\\
$E_{11} \rightarrow \varepsilon$\\
$B \rightarrow bE_3$\\ 
$E_3 \rightarrow aDAbE_6$\\
$E_6 \rightarrow aDAbE_{12}$\\ 
$E_{12} \rightarrow E_1TAb$\\
$E_{12} \rightarrow \varepsilon$\\
$E_6 \rightarrow bB_1TAbAbE_{13}$\\ 
$E_{13} \rightarrow \varepsilon$\\ 
$E_{13} \rightarrow E_1TAb$\\
$E_6 \rightarrow \varepsilon$\\ 
$E_3 \rightarrow bB_1TAbAbE_7$\\
$E_7 \rightarrow aDAbE_{14}$\\
$E_{14} \rightarrow E_1TAb$\\
$E_{14} \rightarrow \varepsilon$\\
$E_7 \rightarrow bB_1TAbAbE_{15}$\\
$E_7 \rightarrow bB_1TAbAb$\\
$E_{15} \rightarrow \varepsilon$\\
$E_{15} \rightarrow E_1TAb$\\
$E_7 \rightarrow \varepsilon$\\
$B \rightarrow b$\\
$T \rightarrow b$\\
$D \rightarrow b$\\

Выполним замену:\\
$B_1 \rightarrow bB_1TAbAbE_1$\\
$B_1 \rightarrow aDAbE_1$\\
$E_1 \rightarrow \varepsilon$\\
$E_1 \rightarrow bB_1TAbAbE_1$\\
$E_1 \rightarrow aDAbE_1$\\
$T \rightarrow aD$\\ 
$T \rightarrow bB_1TAb$\\
$D \rightarrow bB_1$\\
$A \rightarrow bE_2$\\
$E_2 \rightarrow aDAbE_4$\\
$E_4 \rightarrow aDAbE_8$\\
$E_8 \rightarrow aDAbbb$\\
$E_8 \rightarrow bB_1TAbAbbb$\\
$E_8 \rightarrow bB_1TAbAbE_1TAbbb$\\
$E_8 \rightarrow aDAbE_1TAbbb$\\
$E_8 \rightarrow bb$\\
$E_4 \rightarrow bE_9$\\
$E_9 \rightarrow bB_1TAbAbE_1TAbAbbb$\\
$E_9 \rightarrow aDAbE_1TAbAbbb$\\
$E_9 \rightarrow bB_1TAbAbE_1TAbAbE_1TAbbb$\\
$E_9 \rightarrow aDAbE_1TAbAbE_1TAbbb$\\
$E_9 \rightarrow b$\\
$E_2 \rightarrow bE_5$\\
$E_5 \rightarrow aDAbE_1TAbAbE_{10}$\\
$E_{10} \rightarrow bb$\\
$E_{10} \rightarrow aDAbbb$\\
$E_{10} \rightarrow bB_1TAbAbbb$\\
$E_{10} \rightarrow bB_1TAbAbE_1TAbbb$\\
$E_{10} \rightarrow aDAbE_1TAbbb$\\
$E_5 \rightarrow bE_{11}$\
$E_{11} \rightarrow bB_1TAbAbE_1TAbAbE_1TAbAbE_1TAbbb$\\
$E_{11} \rightarrow aDAbE_1TAbAbE_1TAbAbE_1TAbbb$\\
$E_{11} \rightarrow bB_1TAbAbE_1TAbAbE_1TAbAbbb$\\
$E_{11} \rightarrow aDAbE_1TAbAbE_1TAbAbbb$\\
$E_{11} \rightarrow \varepsilon$\\
$B \rightarrow bE_3$\\ 
$E_3 \rightarrow aDAbE_6$\\
$E_6 \rightarrow aDAbE_{12}$\\ 
$E_{12} \rightarrow aDAb$\\
$E_{12} \rightarrow bB_1TAbAb$\\
$E_{12} \rightarrow bB_1TAbAbE_1TAb$\\
$E_{12} \rightarrow aDAbE_1TAb$\\
$E_{12} \rightarrow \varepsilon$\\
$E_6 \rightarrow bB_1TAbAbE_{13}$\\ 
$E_{13} \rightarrow \varepsilon$\\ 
$E_{13} \rightarrow E_1TAb$\\
$E_6 \rightarrow \varepsilon$\\ 
$E_3 \rightarrow bB_1TAbAbE_7$\\
$E_7 \rightarrow aDAbE_{14}$\\
$E_{14} \rightarrow E_1TAb$\\
$E_{14} \rightarrow \varepsilon$\\
$E_7 \rightarrow bB_1TAbAbE_{15}$\\
$E_7 \rightarrow bB_1TAbAb$\\
$E_{15} \rightarrow \varepsilon$\\
$E_{15} \rightarrow E_1TAb$\\
$E_7 \rightarrow \varepsilon$\\
$B \rightarrow b$\\
$T \rightarrow b$\\
$D \rightarrow b$\\

Дальнейшее преобразование не имеет смысла, так как 
повторная факторизация и замена, например, правило\\
$E_{13} \rightarrow E_1TAb$\\
приведёт к правилу с той же правой частью:\\
$E_{13} \rightarrow aDAb$\\
$E_{13} \rightarrow aDAbE_1TAb$\\
$E_{13} \rightarrow bB_1TAbAb$\\
$E_{13} \rightarrow bB_1TAbAbE_1TAb$\\

$E_{13} \rightarrow aDAbE_{14}$\\
$E_{13} \rightarrow bB_1TAbAbE_{14}$\\
$\mathbf{E_{14} \rightarrow E_1TAb}$\\
$E_{14} \rightarrow \varepsilon$\\

Алгоритм зациклился. Искомая грамматика $G_8$, эквивалентная $G_1$ недостижима.
\fi

\item Получить грамматику $G_9$, эквивалентную грамматике $G_1$, в которой
правая часть каждого правила не содержит двух стоящих рядом нетерминала.

Для получения такой грамматики преобразуем грамматику $G_7$ к операторной КС-грамматике.\\
Для приведения грамматики $G_7$ к форме Грейбах введём правило: $G \rightarrow b$:\\
Исходная грамматика:\\
$B_1 \rightarrow aDAG$\\
$B_1 \rightarrow bB_1TAGAG$\\
$B_1 \rightarrow aDAGB_1$\\
$B_1 \rightarrow bB_1TAGAGB_1$\\
$T \rightarrow aD$\\ 
$T \rightarrow bB_1TAG$\\
$D \rightarrow bB_1$\\
$A \rightarrow bB_1TAGGG$\\
$A \rightarrow bTAGGG$\\
$A \rightarrow bGG$\\
$B \rightarrow bB_1TAG$\\ 
$B \rightarrow bTAG$\\ 
$B \rightarrow b$\\
$T \rightarrow b$\\
$D \rightarrow b$\\
$G \rightarrow b$\\

Введём операторные правила:\\
$B_1 \rightarrow aN_1$\\
$B_1 \rightarrow bN_2$\\
$B_1 \rightarrow aN_3$\\
$B_1 \rightarrow bN_4$\\
$T \rightarrow aD$\\ 
$T \rightarrow bN_5$\\
$D \rightarrow bB_1$\\
$A \rightarrow bN_6$\\
$A \rightarrow bN_7$\\
$A \rightarrow bN_8$\\
$B \rightarrow bN_5$\\ 
$B \rightarrow bN_9$\\ 
$B \rightarrow b$\\
$T \rightarrow b$\\
$D \rightarrow b$\\
$G \rightarrow b$\\
$N_1 \rightarrow DAG$\\
$N_2 \rightarrow B_1TAGAG$\\
$N_3 \rightarrow DAGB_1$\\
$N_4 \rightarrow B_1TAGAGB_1$\\
$N_5 \rightarrow B_1TAG$\\
$N_6 \rightarrow B_1TAGGG$\\
$N_7 \rightarrow TAGGG$\\
$N_8 \rightarrow GG$\\
$N_9 \rightarrow TAG$\\

Выполним замену:\\
$B_1 \rightarrow aN_1$\\
$B_1 \rightarrow bN_2$\\
$B_1 \rightarrow aN_3$\\
$B_1 \rightarrow bN_4$\\
$T \rightarrow aD$\\ 
$T \rightarrow bN_5$\\
$D \rightarrow bB_1$\\
$A \rightarrow bN_6$\\
$A \rightarrow bN_7$\\
$A \rightarrow bN_8$\\
$B \rightarrow bN_5$\\ 
$B \rightarrow bN_9$\\ 
$B \rightarrow b$\\
$T \rightarrow b$\\
$D \rightarrow b$\\
$G \rightarrow b$\\
$N_1 \rightarrow DbN_6b$\\
$N_1 \rightarrow DbN_7b$\\
$N_1 \rightarrow DbN_8b$\\
$N_2 \rightarrow B_1aDbN_6bAb$\\
$N_2 \rightarrow B_1aDbN_7bAb$\\
$N_2 \rightarrow B_1aDbN_8bAb$\\
$N_2 \rightarrow B_1bN_5bN_6bAb$\\
$N_2 \rightarrow B_1bN_5bN_7bAb$\\
$N_2 \rightarrow B_1bN_5bN_8bAb$\\
$N_3 \rightarrow DbN_6bB_1$\\
$N_3 \rightarrow DbN_7bB_1$\\
$N_3 \rightarrow DbN_8bB_1$\\
$N_4 \rightarrow B_1aDbN_6bAbB_1$\\
$N_4 \rightarrow B_1aDbN_7bAbB_1$\\
$N_4 \rightarrow B_1aDbN_8bAbB_1$\\
$N_4 \rightarrow B_1bN_5bN_6bAbB_1$\\
$N_4 \rightarrow B_1bN_5bN_7bAbB_1$\\
$N_4 \rightarrow B_1bN_5bN_8bAbB_1$\\
$N_5 \rightarrow B_1aDbN_6b$\\
$N_5 \rightarrow B_1aDbN_7b$\\
$N_5 \rightarrow B_1aDbN_8b$\\
$N_5 \rightarrow B_1bN_5bN_6b$\\
$N_5 \rightarrow B_1bN_5bN_7b$\\
$N_5 \rightarrow B_1bN_5bN_8b$\\
$N_6 \rightarrow B_1aDbN_6bGb$\\
$N_6 \rightarrow B_1aDbN_7bGb$\\
$N_6 \rightarrow B_1aDbN_8bGb$\\
$N_6 \rightarrow B_1bN_5bN_6bGb$\\
$N_6 \rightarrow B_1bN_5bN_7bGb$\\
$N_6 \rightarrow B_1bN_5bN_8bGb$\\
$N_7 \rightarrow TbN_6bGb$\\
$N_7 \rightarrow TbN_7bGb$\\
$N_7 \rightarrow TbN_8bGb$\\
$N_8 \rightarrow Gb$\\
$N_9 \rightarrow TbN_6b$\\
$N_9 \rightarrow TbN_7b$\\
$N_9 \rightarrow TbN_8b$\\

Получена искомая грамматика $G_9$:\\
1. $B_1 \rightarrow aN_1$\\
2. $B_1 \rightarrow bN_2$\\
3. $B_1 \rightarrow aN_3$\\
4. $B_1 \rightarrow bN_4$\\
5. $T \rightarrow aD$\\ 
6. $T \rightarrow bN_5$\\
7. $D \rightarrow bB_1$\\
8. $A \rightarrow bN_6$\\
9. $A \rightarrow bN_7$\\
11. $A \rightarrow bN_8$\\
12. $B \rightarrow bN_5$\\ 
13. $B \rightarrow bN_9$\\ 
14. $B \rightarrow b$\\
15. $T \rightarrow b$\\
16. $D \rightarrow b$\\
17. $G \rightarrow b$\\
18. $N_1 \rightarrow DbN_6b$\\
19. $N_1 \rightarrow DbN_7b$\\
20. $N_1 \rightarrow DbN_8b$\\
21. $N_2 \rightarrow B_1aDbN_6bAb$\\
22. $N_2 \rightarrow B_1aDbN_7bAb$\\
23. $N_2 \rightarrow B_1aDbN_8bAb$\\
24. $N_2 \rightarrow B_1bN_5bN_6bAb$\\
25. $N_2 \rightarrow B_1bN_5bN_7bAb$\\
26. $N_2 \rightarrow B_1bN_5bN_8bAb$\\
27. $N_3 \rightarrow DbN_6bB_1$\\
28. $N_3 \rightarrow DbN_7bB_1$\\
29. $N_3 \rightarrow DbN_8bB_1$\\
30. $N_4 \rightarrow B_1aDbN_6bAbB_1$\\
31. $N_4 \rightarrow B_1aDbN_7bAbB_1$\\
32. $N_4 \rightarrow B_1aDbN_8bAbB_1$\\
33. $N_4 \rightarrow B_1bN_5bN_6bAbB_1$\\
34. $N_4 \rightarrow B_1bN_5bN_7bAbB_1$\\
35. $N_4 \rightarrow B_1bN_5bN_8bAbB_1$\\
36. $N_5 \rightarrow B_1aDbN_6b$\\
37. $N_5 \rightarrow B_1aDbN_7b$\\
38. $N_5 \rightarrow B_1aDbN_8b$\\
39. $N_5 \rightarrow B_1bN_5bN_6b$\\
40. $N_5 \rightarrow B_1bN_5bN_7b$\\
41. $N_5 \rightarrow B_1bN_5bN_8b$\\
42. $N_6 \rightarrow B_1aDbN_6bGb$\\
43. $N_6 \rightarrow B_1aDbN_7bGb$\\
44. $N_6 \rightarrow B_1aDbN_8bGb$\\
45. $N_6 \rightarrow B_1bN_5bN_6bGb$\\
46. $N_6 \rightarrow B_1bN_5bN_7bGb$\\
47. $N_6 \rightarrow B_1bN_5bN_8bGb$\\
48. $N_7 \rightarrow TbN_6bGb$\\
49. $N_7 \rightarrow TbN_7bGb$\\
50. $N_7 \rightarrow TbN_8bGb$\\
51. $N_8 \rightarrow Gb$\\
52. $N_9 \rightarrow TbN_6b$\\
53. $N_9 \rightarrow TbN_7b$\\
54. $N_9 \rightarrow TbN_8b$\\

\item Получить грамматику $G_{10}$, эквивалентную грамматике $G_1$, в которой
любой символ занимает либо только крайнюю правую позицию в правых частях правил, 
либо находится левее самого правого символа в правых частях правил.

Возьмём грамматику $G_2$:\\
1. $T \rightarrow aD$\\
2. $T \rightarrow D$\\
3. $D \rightarrow DTAb$\\
4. $D \rightarrow b$\\
6. $A \rightarrow Bbb$\\
7. $B \rightarrow D$\\

Введём правило $N \rightarrow \varepsilon$. Добавим символ N к концу всех правил:\\
1. $T \rightarrow aDN$\\
2. $T \rightarrow DN$\\
3. $D \rightarrow DTAbN$\\
4. $D \rightarrow bN$\\
6. $A \rightarrow BbbN$\\
7. $B \rightarrow DN$\\
8. $N \rightarrow \varepsilon$\\
Если выполним замену символа N во всех правилах, получим эквивалентную $G_2$, 
а значит и $G_1$ грамматику. Символ N занимает только крайнюю правую позицию в правилах.
Остальные символы находятся левее него.\\
Получили искомую грамматику $G_{10}$:\\
1. $T \rightarrow aDN$\\
2. $T \rightarrow DN$\\
3. $D \rightarrow DTAbN$\\
4. $D \rightarrow bN$\\
6. $A \rightarrow BbbN$\\
7. $B \rightarrow DN$\\
8. $N \rightarrow \varepsilon$\\
\fi

\end{enumerate}

\textbf{Вывод: } в ходе лабораторной работы изучили основные эквивалентные преобразования
КС-грамматик и научились применять их для получения
КС-грамматик, обладающих заданными свойствами.

\end{document}