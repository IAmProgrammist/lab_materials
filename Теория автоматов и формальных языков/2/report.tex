\documentclass[a4paper,14pt]{extarticle}


\usepackage[english,russian]{babel}
\usepackage[T2A]{fontenc}
\usepackage[utf8]{inputenc}
\usepackage{ragged2e}
\usepackage[utf8]{inputenc}
\usepackage{hyperref}
\usepackage{minted}
\setmintedinline{frame=lines, framesep=2mm, baselinestretch=1.5, bgcolor=LightGray, breaklines,fontsize=\scriptsize}
\setminted{frame=lines, framesep=2mm, baselinestretch=1.5, bgcolor=LightGray, breaklines,fontsize=\scriptsize}
\usepackage{xcolor}
\definecolor{LightGray}{gray}{0.9}
\usepackage{graphicx}
\usepackage[export]{adjustbox}
\usepackage[left=1cm,right=1cm, top=1cm,bottom=1cm,bindingoffset=0cm]{geometry}
\usepackage{fontspec}
\usepackage{ upgreek }
\usepackage[shortlabels]{enumitem}
\usepackage{adjustbox}
\usepackage{multirow}
\usepackage{amsmath}
\usepackage{amssymb}
\usepackage{pifont}
\usepackage{pgfplots}
\usepackage{longtable}
\usepackage{array}
\graphicspath{ {./images/} }
\makeatletter
\AddEnumerateCounter{\asbuk}{\russian@alph}{щ}
\makeatother
\setmonofont{Consolas}
\setmainfont{Times New Roman}

\newcommand\textbox[1]{
	\parbox{.45\textwidth}{#1}
} 

\newcommand{\specialcell}[2][c]{%
	\begin{tabular}[#1]{@{}c@{}}#2\end{tabular}}

\begin{document}
\pagenumbering{gobble}
\begin{center}
    \small{
        \textbf{МИНИCТЕРCТВО НАУКИ И ВЫCШЕГО ОБРАЗОВАНИЯ РОCCИЙCКОЙ ФЕДЕРАЦИИ}\\
        ФЕДЕРАЛЬНОЕ ГОCУДАРCТВЕННОЕ БЮДЖЕТНОЕ ОБРАЗОВАТЕЛЬНОЕ УЧРЕЖДЕНИЕ\\ВЫCШЕГО ОБРАЗОВАНИЯ \\
        \textbf{«БЕЛГОРОДCКИЙ ГОCУДАРCТВЕННЫЙ ТЕХНОЛОГИЧЕCКИЙ\\УНИВЕРCИТЕТ им. В. Г. ШУХОВА»\\ (БГТУ им. В.Г. Шухова)} \\
        \bigbreak
        \includegraphics[width=70mm]{log}\\
        ИНСТИТУТ ИНФОРМАЦИОННЫХ ТЕХНОЛОГИЙ И УПРАВЛЯЮЩИХ СИСТЕМ\\}
\end{center}

\vfill
\begin{center}
    \large{
        \textbf{
            Лабораторная работа №2}}\\
    \normalsize{
        по дисциплине: Теория автоматов и формальных языков \\
        тема: «Преобразования КС-грамматик.»}
\end{center}
\vfill
\hfill\textbox{
    Выполнил: ст. группы ПВ-223\\Пахомов Владислав Андреевич
    \bigbreak
    Проверили: \\ст. пр. Рязанов Юрий Дмитриевич
}
\vfill\begin{center}
    Белгород 2024 г.
\end{center}
\newpage
\begin{center}
    \textbf{Лабораторная работа №2}\\
    Преобразования КС-грамматик.\\
    Вариант 8
\end{center}
\textbf{Цель работы: }изучить основные эквивалентные преобразования
КС-грамматик и научиться применять их для получения
КС-грамматик, обладающих заданными свойствами.\\
\textbf{Задание: }\\
    1. $T \rightarrow abETP$\\
    2. $T \rightarrow aDE$\\
    3. $T \rightarrow D$\\
    4. $D \rightarrow DTAb$\\
    5. $D \rightarrow b$\\
    6. $E \rightarrow \varepsilon$\\
    7. $P \rightarrow BCa$\\
    8. $P \rightarrow Cb$\\
    9. $C \rightarrow abC$\\
    10. $A \rightarrow Bbb$\\
    11. $B \rightarrow aECb$\\
    12. $B \rightarrow D$\\

\begin{enumerate}[1.]
    \item Преобразовать исходную грамматику $G$ в
    грамматику $G_1$ без лишних символов.\\
    \textbf{Модификации: }в ходе выполнения лабораторной работы обнаружено, что в грамматике
    не будет недостижимых символов. Поэтому добавим правило:\\
    
    13. $S \rightarrow ab$


    Найдём в исходной грамматике бесплодные нетерминалы.\\
    Для начала найдём продуктивные нетерминалы.\\
    В множество продуктивных нетерминалов Р включаем нетерминал D
(правило 5) нетерминал E (правило 6) и нетерминал S (правило 13). Получаем $Р=\{D,E,S\}$.
Повторяем проверку и включаем нетерминал T (правило 2) и нетерминал B (правило 12). 
Получаем $P=\{D,E,S,T,B\}$\\
Повторяем проверку и включаем A (правило 10). Получаем $P=\{D,E,S,T,B,A\}$\\
Множество P больше увеличить не можем.\\

Из множества нетерминалов исключаем продуктивные 
нетерминалы и получаем $\{P,C\}$ - множество бесплодных нетерминалов.\\

Исключаем правила 1, 7, 8, 9, 11 так как они содержат 
бесплодные нетерминалы. Получаем грамматику:\\

2. $T \rightarrow aDE$\\
3. $T \rightarrow D$\\
4. $D \rightarrow DTAb$\\
5. $D \rightarrow b$\\
6. $E \rightarrow \varepsilon$\\
10. $A \rightarrow Bbb$\\
12. $B \rightarrow D$\\
13. $S \rightarrow ab$\\

Найдём достижимые символы.\\
Положим $P = \{T\}$, где T - начальный нетерминал.\\
Включим в список a, D, E (правило 2). $P = \{T, a, D, E\}$.\\
Включим в список b, A (правило 4), $\varepsilon$. $P = \{T, a, D, E, \varepsilon, b, A\}$.\\
Включим в список B (правило 10). $P = \{T, a, D, E, \varepsilon, b, A, B\}$.\\
Множество P больше увеличить не можем.\\

Из множества терминалов и нетерминалов исключаем достижимые 
терминалы и нетерминалы и получаем $\{S\}$ - множество недостижимых нетерминалов и терминалов.\\

Исключаем из грамматики правило 13, так как оно содержит недостижимый символ.\\
Искомая грамматика $G_1$:\\
1. $T \rightarrow aDE$\\
2. $T \rightarrow D$\\
3. $D \rightarrow DTAb$\\
4. $D \rightarrow b$\\
5. $E \rightarrow \varepsilon$\\
6. $A \rightarrow Bbb$\\
7. $B \rightarrow D$\\

\item Преобразовать грамматику $G_1$ в грамматику $G_2$ без $\varepsilon$-правил.\\
Выберем правило 5. Иключаем из правой части каждого правила исходной грамматики
всеми возможными способами вхождение нетерминала E. Полученные правила добавляем в множество 
правил грамматики.\\
1\_1. $T \rightarrow aDE$\\
1\_2. $T \rightarrow aD$\\
2. $T \rightarrow D$\\
3. $D \rightarrow DTAb$\\
4. $D \rightarrow b$\\
5. $E \rightarrow \varepsilon$\\
6. $A \rightarrow Bbb$\\
7. $B \rightarrow D$\\
Исключаем из списка правил правило 5.\\
1\_1. $T \rightarrow aDE$\\
1\_2. $T \rightarrow aD$\\
2. $T \rightarrow D$\\
3. $D \rightarrow DTAb$\\
4. $D \rightarrow b$\\
6. $A \rightarrow Bbb$\\
7. $B \rightarrow D$\\
Исключим из правил непродуктивные символы:\\
1\_2. $T \rightarrow aD$\\
2. $T \rightarrow D$\\
3. $D \rightarrow DTAb$\\
4. $D \rightarrow b$\\
6. $A \rightarrow Bbb$\\
7. $B \rightarrow D$\\
В полученной грамматике $G_2$ нет правил вида $A \rightarrow A$, одинаковых правил и $\varepsilon$-правил.

\item Преобразовать грамматику $G_1$ в грамматику $G_3$ без цепных правил.

Применим замену края:\\

Исходная грамматика:\\
1. $T \rightarrow aDE$\\
2. $\mathbf{T \rightarrow D}$\\
3. $D \rightarrow DTAb$\\
4. $D \rightarrow b$\\
5. $E \rightarrow \varepsilon$\\
6. $A \rightarrow Bbb$\\
7. $B \rightarrow D$\\

Шаг 1: \\
1. $T \rightarrow aDE$\\
2\_1. $T \rightarrow DTAb$\\
2\_2. $T \rightarrow b$\\
3. $D \rightarrow DTAb$\\
4. $D \rightarrow b$\\
5. $E \rightarrow \varepsilon$\\
6. $A \rightarrow Bbb$\\
7. $\mathbf{B \rightarrow D}$\\

Шаг 2:\\
1. $T \rightarrow aDE$\\
2\_1. $T \rightarrow DTAb$\\
2\_2. $T \rightarrow b$\\
3. $D \rightarrow DTAb$\\
4. $D \rightarrow b$\\
5. $E \rightarrow \varepsilon$\\
6. $A \rightarrow Bbb$\\
7\_1. $B \rightarrow DTAb$\\
7\_2. $B \rightarrow b$\\

Цепных правил не осталось.
Получили искомую грамматику $G_3$.

Альтернативный вариант:

Исключим из грамматики все нецепные правила. Это правила 1, 3, 4, 5, 6.\\
2. $T \rightarrow D$\\
7. $B \rightarrow D$\\

Примем множества $M^T = \{T\}$.
Включим нетерминал D в множество $M^T$, так как есть правило 2 $T \rightarrow D$.
$M^T = \{T, D\}$.
Больше в $M^T$ ничего добавить не можем. Исключаем T:
$M^T = \{D\}$.


Примем множества $M^B = \{B\}$.
Включим нетерминал D в множество $M^T$, так как есть правило 7 $B \rightarrow D$.
$M^B = \{B, D\}$.
Больше в $M^T$ ничего добавить не можем. Исключаем B:
$M^B = \{D\}$.


Исключаем из грамматики $G_1$ все цепные правила:\\
1. $T \rightarrow aDE$\\
3. $D \rightarrow DTAb$\\
4. $D \rightarrow b$\\
5. $E \rightarrow \varepsilon$\\
6. $A \rightarrow Bbb$\\

Для правила 3 добавим правило 3\_1. $T \rightarrow DTAb$, так как D принадлежит $M^T = \{D\}$.\\
Для правила 3 добавим правило 3\_2. $B \rightarrow DTAb$, так как D принадлежит $M^B = \{D\}$.\\
Для правила 4 добавим правило 4\_1. $T \rightarrow b$, так как D принадлежит $M^T = \{D\}$.\\
Для правила 4 добавим правило 4\_2. $B \rightarrow b$, так как D принадлежит $M^B = \{D\}$.\\

Искомая грамматика $G_3$:\\
1. $T \rightarrow aDE$\\
3. $D \rightarrow DTAb$\\
3\_1. $T \rightarrow DTAb$\\
3\_2. $B \rightarrow DTAb$\\
4. $D \rightarrow b$\\
4\_1. $T \rightarrow b$\\
4\_2. $B \rightarrow b$\\
5. $E \rightarrow \varepsilon$\\
6. $A \rightarrow Bbb$\\

\item Преобразовать грамматику $G_1$ в грамматику $G_4$ без левой рекурсии.\\

Алгоритм применим, если грамматика не имеет циклов (цепных правил) и $\varepsilon$-правил.
Для получения грамматики без $\varepsilon$-правил воспользуемся грамматикой $G_2$.\\
1. $T \rightarrow aD$\\
2. $T \rightarrow D$\\
3. $D \rightarrow DTAb$\\
4. $D \rightarrow b$\\
6. $A \rightarrow Bbb$\\
7. $B \rightarrow D$\\
Преобразуем эту грамматику так, чтобы в ней не было цепных правил.\\

Исходная грамматика:\\
1. $T \rightarrow aD$\\
2. $\mathbf{T \rightarrow D}$\\
3. $D \rightarrow DTAb$\\
4. $D \rightarrow b$\\
6. $A \rightarrow Bbb$\\
7. $B \rightarrow D$\\

Выполним замену края:\\
1. $T \rightarrow aD$\\
2\_1. $T \rightarrow DTAb$\\
2\_2. $T \rightarrow b$\\
3. $D \rightarrow DTAb$\\
4. $D \rightarrow b$\\
6. $A \rightarrow Bbb$\\
7. $\mathbf{B \rightarrow D}$\\

Выполним замену края:\\
1. $T \rightarrow aD$\\
2\_1. $T \rightarrow DTAb$\\
2\_2. $T \rightarrow b$\\
3. $D \rightarrow DTAb$\\
4. $D \rightarrow b$\\
6. $A \rightarrow Bbb$\\
7\_1. $B \rightarrow DTAb$\\
7\_2. $B \rightarrow b$\\

Получили грамматику $G_3'$ без лишних символов, $\varepsilon$-правил и цепных правил:\\
1. $T \rightarrow aD$\\
2. $T \rightarrow DTAb$\\
3. $T \rightarrow b$\\
4. $D \rightarrow DTAb$\\
5. $D \rightarrow b$\\
6. $A \rightarrow Bbb$\\
7. $B \rightarrow DTAb$\\
8. $B \rightarrow b$\\

Обозначим нетерминалы грамматики: T, D, A, B как $A_1, A_2, A_3, A_4$ соответственно.\\
1. $A_1 \rightarrow aA_2$\\
2. $A_1 \rightarrow A_2A_1A_3b$\\
3. $A_1 \rightarrow b$\\
4. $A_2 \rightarrow A_2A_1A_3b$\\
5. $A_2 \rightarrow b$\\
6. $A_3 \rightarrow A_4bb$\\
7. $A_4 \rightarrow A_2A_1A_3b$\\
8. $A_4 \rightarrow b$\\

Рассмотрим нетерминал $A_1$.\\
Правил вида $A_1 \rightarrow A_0a$ не существует, 
следовательно замену края выполнять не будем.\\
Самолеворекурсивных правил для $A_1$ также нет.\\

Рассмотрим нетерминал $A_2$.\\
Правил вида $A_2 \rightarrow A_1a$ не существует, 
следовательно замену края выполнять не будем.\\
Для $A_2$ существует самолеворекурсивное правило 4. 
Также существует несаморекурсивное правило 5. 
Заменим эти правила:\\
1. $A_1 \rightarrow aA_2$\\
2. $A_1 \rightarrow A_2A_1A_3b$\\
3. $A_1 \rightarrow b$\\
9. $A_2 \rightarrow bB_1$\\
10. $B_1 \rightarrow A_1A_3bB_1$\\
11. $B_1 \rightarrow \varepsilon$\\
6. $A_3 \rightarrow A_4bb$\\
7. $A_4 \rightarrow A_2A_1A_3b$\\
8. $A_4 \rightarrow b$\\

Рассмотрим нетерминал $A_3$.\\
Правил вида $A_3 \rightarrow A_2a$ не существует, 
следовательно замену края выполнять не будем.\\
Самолеворекурсивных правил для $A_3$ также нет.\\

Рассмотрим нетерминал $A_4$.\\
Существует правило 7. $A_4 \rightarrow A_2A_1A_3b$, выполним замену края:\\
1. $A_1 \rightarrow aA_2$\\
2. $A_1 \rightarrow A_2A_1A_3b$\\
3. $A_1 \rightarrow b$\\
9. $A_2 \rightarrow bB_1$\\
10. $B_1 \rightarrow A_1A_3bB_1$\\
11. $B_1 \rightarrow \varepsilon$\\
6. $A_3 \rightarrow A_4bb$\\
12. $A_4 \rightarrow bB_1A_1A_3b$\\
8. $A_4 \rightarrow b$\\

Искомая грамматика $G_4$: \\
1. $T \rightarrow aD$\\
2. $T \rightarrow DTAb$\\
3. $T \rightarrow b$\\
4. $D \rightarrow bB_1$\\
5. $B_1 \rightarrow TAbB_1$\\
6. $B_1 \rightarrow \varepsilon$\\
7. $A \rightarrow Bbb$\\
8. $B \rightarrow bB_1TAb$\\
9. $B \rightarrow b$\\

\item Преобразовать грамматику $G_1$ в грамматику $G_5$ без несаморекурсивных
нетерминалов.

Искходная грамматика:\\
1. $T \rightarrow aDE$\\
2. $T \rightarrow D$\\
3. $D \rightarrow DTAb$\\
4. $D \rightarrow b$\\
5. $E \rightarrow \varepsilon$\\
6. $A \rightarrow Bbb$\\
7. $B \rightarrow D$\\

Нетерминал E несаморекурсивный.\\
Исключаем правило 5:\\
5. $E \rightarrow \varepsilon$\\
Выбираем вхождение символа E в правиле 1 и
выполняем замену на правую часть правила 5:\\
1\_1. $T \rightarrow aD$\\
2. $T \rightarrow D$\\
3. $D \rightarrow DTAb$\\
4. $D \rightarrow b$\\
6. $A \rightarrow Bbb$\\
7. $B \rightarrow D$\\

Нетерминал T несаморекурсивный.\\
Исключаем правила 1\_1, 2:\\
1\_1. $T \rightarrow aD$\\
2. $T \rightarrow D$\\
Выбираем вхождение символа T в правиле 3 и
выполняем замену на правую часть правил 1\_1, 2:\\
3\_1. $D \rightarrow DaDAb$\\
3\_2. $D \rightarrow DDAb$\\
4. $D \rightarrow b$\\
6. $A \rightarrow Bbb$\\
7. $B \rightarrow D$\\

Нетерминал B несаморекурсивный.\\
Исключаем правило 7:\\
7. $B \rightarrow D$\\
Выбираем вхождение символа B в правиле 6 и
выполняем замену на правую часть правила 7:\\
3\_1. $D \rightarrow DaDAb$\\
3\_2. $D \rightarrow DDAb$\\
4. $D \rightarrow b$\\
6\_1. $A \rightarrow Dbb$\\

Нетерминал A несаморекурсивный.\\
Исключаем правило 6\_1:\\
6\_1. $A \rightarrow Dbb$\\
Выбираем вхождение символа A в правилах 3\_1, 3\_2 и
выполняем замену на правую часть правила 6\_1:\\
3\_1\_1. $D \rightarrow DaDDbbb$\\
3\_2\_2. $D \rightarrow DDDbbb$\\
4. $D \rightarrow b$\\

Искомая грамматика $G_5$:\\
1. $D \rightarrow DaDDbbb$\\
2. $D \rightarrow DDDbbb$\\
3. $D \rightarrow b$\\

\item Получить грамматику $G_6$, эквивалентную грамматике $G_1$, в которой
правая часть каждого правила состоит либо из одного терминала, либо двух нетерминалов.

Для получения грамматики $G_6$ необходимо привести грамматику $G_1$ 
к нормальной форме Хомского.\\
Воспользуемся грамматикой $G_3'$, в которой нет цепных правил, $\varepsilon$-правил
и цепных правил.\\
Исходная грамматика:\\
1. $T \rightarrow aD$\\
2. $T \rightarrow DTAb$\\
3. $T \rightarrow b$\\
4. $D \rightarrow DTAb$\\
5. $D \rightarrow b$\\
6. $A \rightarrow Bbb$\\
7. $B \rightarrow DTAb$\\
8. $B \rightarrow b$\\

Выполним пункт 1 алгоритма (преобразование правил вида $A \rightarrow Xa$):\\
1. $T \rightarrow aD$\\
2. $T \rightarrow DN_1$\\
3. $T \rightarrow b$\\
4. $D \rightarrow DN_1$\\
5. $D \rightarrow b$\\
6. $A \rightarrow Bbb$\\
7. $B \rightarrow DN_1$\\
8. $B \rightarrow b$\\
9. $N_1 \rightarrow TAb$\\

1. $T \rightarrow aD$\\
2. $T \rightarrow DN_1$\\
3. $T \rightarrow b$\\
4. $D \rightarrow DN_1$\\
5. $D \rightarrow b$\\
6. $A \rightarrow BN_2$\\
7. $B \rightarrow DN_1$\\
8. $B \rightarrow b$\\
9. $N_1 \rightarrow TAb$\\
10. $N_2 \rightarrow bb$\\

1. $T \rightarrow aD$\\
2. $T \rightarrow DN_1$\\
3. $T \rightarrow b$\\
4. $D \rightarrow DN_1$\\
5. $D \rightarrow b$\\
6. $A \rightarrow BN_2$\\
7. $B \rightarrow DN_1$\\
8. $B \rightarrow b$\\
9. $N_1 \rightarrow N_3b$\\
10. $N_2 \rightarrow bb$\\
11. $N_3 \rightarrow TA$\\

Выполним пункт 2 алгоритма (преобразование правил вида $A \rightarrow tB$):\\
1. $T \rightarrow N_4D$\\
2. $T \rightarrow DN_1$\\
3. $T \rightarrow b$\\
4. $D \rightarrow DN_1$\\
5. $D \rightarrow b$\\
6. $A \rightarrow BN_2$\\
7. $B \rightarrow DN_1$\\
8. $B \rightarrow b$\\
9. $N_1 \rightarrow N_3b$\\
10. $N_2 \rightarrow bb$\\
11. $N_3 \rightarrow TA$\\
12. $N_4 \rightarrow a$\\

Выполним пункт 3 алгоритма (преобразование правил вида $A \rightarrow Bt$):\\
1. $T \rightarrow N_4D$\\
2. $T \rightarrow DN_1$\\
3. $T \rightarrow b$\\
4. $D \rightarrow DN_1$\\
5. $D \rightarrow b$\\
6. $A \rightarrow BN_2$\\
7. $B \rightarrow DN_1$\\
8. $B \rightarrow b$\\
9. $N_1 \rightarrow N_3T$\\
10. $N_2 \rightarrow bb$\\
11. $N_3 \rightarrow TA$\\
12. $N_4 \rightarrow a$\\

Выполним пункт 4 алгоритма (преобразование правил вида $A \rightarrow tt$):\\
1. $T \rightarrow N_4D$\\
2. $T \rightarrow DN_1$\\
3. $T \rightarrow b$\\
4. $D \rightarrow DN_1$\\
5. $D \rightarrow b$\\
6. $A \rightarrow BN_2$\\
7. $B \rightarrow DN_1$\\
8. $B \rightarrow b$\\
9. $N_1 \rightarrow N_3T$\\
10. $N_2 \rightarrow TT$\\
11. $N_3 \rightarrow TA$\\
12. $N_4 \rightarrow a$\\

Искомая грамматика $G_6$:\\
1. $T \rightarrow N_4D$\\
2. $T \rightarrow DN_1$\\
3. $T \rightarrow b$\\
4. $D \rightarrow DN_1$\\
5. $D \rightarrow b$\\
6. $A \rightarrow BN_2$\\
7. $B \rightarrow DN_1$\\
8. $B \rightarrow b$\\
9. $N_1 \rightarrow N_3T$\\
10. $N_2 \rightarrow TT$\\
11. $N_3 \rightarrow TA$\\
12. $N_4 \rightarrow a$\\

\item Получить грамматику $G_7$, эквивалентную грамматике $G_1$, в которой 
правая часть каждого правила начинается терминалом.

Для получения грамматики $G_7$ 
необходимо привести грамматику $G_1$ 
к нормальной форме Грейбах.\\

Используем преобразованную грамматику $G_1$ без левой рекурсии $G_4$:\\
\iffalse
1. $T \rightarrow aD$\\              T  _  D
2. $T \rightarrow DTAb$\\            T  D  _
3. $T \rightarrow b$\\               T  _  _
4. $D \rightarrow bB_1$\\            D  _  B1
5. $B_1 \rightarrow TAbB_1$\\        B1 T  B1
6. $B_1 \rightarrow \varepsilon$\\   B1 _  _
7. $A \rightarrow Bbb$\\             A  B  _
8. $B \rightarrow bB_1TAb$\\         B  _  _
9. $B \rightarrow b$\\               B  _  _

B1 T  B1
T  _  D
T  D  _
D  _  B1
A  B  _
B1 _  _
B  _  _
B  _  _
T  _  _
\fi

1. $T \rightarrow aD$\\
2. $T \rightarrow DTAb$\\
3. $T \rightarrow b$\\
4. $D \rightarrow bB_1$\\
5. $B_1 \rightarrow TAbB_1$\\
6. $B_1 \rightarrow \varepsilon$\\
7. $A \rightarrow Bbb$\\
8. $B \rightarrow bB_1TAb$\\
9. $B \rightarrow b$\\

Упорядочим грамматику:\\
1. $T \rightarrow aD$\\
2. $T \rightarrow DTAb$\\
4. $D \rightarrow bB_1$\\
5. $B_1 \rightarrow TAbB_1$\\
7. $A \rightarrow Bbb$\\
6. $B_1 \rightarrow \varepsilon$\\
8. $B \rightarrow bB_1TAb$\\
9. $B \rightarrow b$\\
3. $T \rightarrow b$\\

Выполнение замены края:\\
1. $T \rightarrow aD$\\
2. $T \rightarrow DTAb$\\
4. $D \rightarrow bB_1$\\
5. $B_1 \rightarrow TAbB_1$\\
7\_1. $A \rightarrow bB_1TAbbb$\\
7\_2. $A \rightarrow bbb$\\
6. $B_1 \rightarrow \varepsilon$\\
8. $B \rightarrow bB_1TAb$\\
9. $B \rightarrow b$\\
3. $T \rightarrow b$\\

1. $T \rightarrow aD$\\
2. $T \rightarrow DTAb$\\
4. $D \rightarrow bB_1$\\
5\_1. $B_1 \rightarrow bAbB_1$\\
7\_1. $A \rightarrow bB_1TAbbb$\\
7\_2. $A \rightarrow bbb$\\
6. $B_1 \rightarrow \varepsilon$\\
8. $B \rightarrow bB_1TAb$\\
9. $B \rightarrow b$\\
3. $T \rightarrow b$\\

1. $T \rightarrow aD$\\
2\_1. $T \rightarrow bB_1TAb$\\
4. $D \rightarrow bB_1$\\
5\_1. $B_1 \rightarrow bAbB_1$\\
7\_1. $A \rightarrow bB_1TAbbb$\\
7\_2. $A \rightarrow bbb$\\
6. $B_1 \rightarrow \varepsilon$\\
8. $B \rightarrow bB_1TAb$\\
9. $B \rightarrow b$\\
3. $T \rightarrow b$\\

Введём правило $N \rightarrow b$ и выполним замену, где необходимо:\\
1. $T \rightarrow aD$\\
2\_1. $T \rightarrow bB_1TAN$\\
4. $D \rightarrow bB_1$\\
5\_1. $B_1 \rightarrow bANB_1$\\
7\_1. $A \rightarrow bB_1TANNN$\\
7\_2. $A \rightarrow bNN$\\
6. $B_1 \rightarrow \varepsilon$\\
8. $B \rightarrow bB_1TAN$\\
9. $B \rightarrow b$\\
3. $T \rightarrow b$\\
10. $N \rightarrow b$\\

Искомая грамматика $G_7$:\\
1. $T \rightarrow aD$\\
2. $T \rightarrow bB_1TAN$\\
3. $D \rightarrow bB_1$\\
4. $B_1 \rightarrow bANB_1$\\
5. $A \rightarrow bB_1TANNN$\\
6. $A \rightarrow bNN$\\
7. $B_1 \rightarrow \varepsilon$\\
8. $B \rightarrow bB_1TAN$\\
9. $B \rightarrow b$\\
10. $T \rightarrow b$\\
11. $N \rightarrow b$\\

\item Получить грамматику $G_8$, эквивалентную грамматике $G_1$, в которой 
правая часть каждого не $\varepsilon$-правила начинается терминалом и любые
два правила с одинаковой левой частью различаются первым символом в правой части.\\
Для получения такой грамматики можем проводить множественную левую факторизацию и замену в грамматике $G_7$.\\

\textbf{Модификации: }в ходе выполнения задания было получено крайне большое количество вычислений, поэтому
из грамматики $G_7$ были удалены правила: $T \rightarrow b$, $T \rightarrow bB_1TAN$.\\

Исходная грамматика:\\
$T \rightarrow aD$\\
$A \rightarrow bNN$\\
$A \rightarrow bB_1TANNN$\\
$B_1 \rightarrow bANB_1$\\
$B_1 \rightarrow \varepsilon$\\
$B \rightarrow b$\\
$B \rightarrow bB_1TAN$\\
$D \rightarrow bB_1$\\
$N \rightarrow b$\\

Выполним левую факторизацию для B:\\
$T \rightarrow aD$\\
$A \rightarrow bNN$\\
$A \rightarrow bB_1TANNN$\\
$B_1 \rightarrow bANB_1$\\
$B_1 \rightarrow \varepsilon$\\
$B \rightarrow BE_1$\\
$D \rightarrow bB_1$\\
$N \rightarrow b$\\
$E_1 \rightarrow \varepsilon$\\
$E_1 \rightarrow B_1TAN$\\

Выполним замену для $E_1$:\\
$T \rightarrow aD$\\
$A \rightarrow bNN$\\
$A \rightarrow bB_1TANNN$\\
$B_1 \rightarrow bANB_1$\\
$B_1 \rightarrow \varepsilon$\\
$B \rightarrow BE_1$\\
$D \rightarrow bB_1$\\
$N \rightarrow b$\\
$E_1 \rightarrow \varepsilon$\\
$E_1 \rightarrow bANB_1TAN$\\
$E_1 \rightarrow aDAN$\\

Искомая грамматика $G_8$:\\
1. $T \rightarrow aD$\\
2. $A \rightarrow bNN$\\
3. $A \rightarrow bB_1TANNN$\\
4. $B_1 \rightarrow bANB_1$\\
5. $B_1 \rightarrow \varepsilon$\\
6. $B \rightarrow BE_1$\\
7. $D \rightarrow bB_1$\\
8. $N \rightarrow b$\\
9. $E_1 \rightarrow \varepsilon$\\
10. $E_1 \rightarrow bANB_1TAN$\\
11. $E_1 \rightarrow aDAN$\\

\item Получить грамматику $G_9$, эквивалентную грамматике $G_1$, в которой
правая часть каждого правила не содержит двух стоящих рядом нетерминала.

Для получения такой грамматики преобразуем грамматику $G_7$ к операторной КС-грамматике.\\
Исходная грамматика:\\
1. $T \rightarrow aD$\\
2. $T \rightarrow bB_1TAN$\\
3. $D \rightarrow bB_1$\\
4. $B_1 \rightarrow bANB_1$\\
5. $A \rightarrow bB_1TANNN$\\
6. $A \rightarrow bNN$\\
7. $B_1 \rightarrow \varepsilon$\\
8. $B \rightarrow bB_1TAN$\\
9. $B \rightarrow b$\\
10. $T \rightarrow b$\\
11. $N \rightarrow b$\\

Выполним замену символа N во всех правилах, при этом N станет недостижимым и его можно будет удалить.
1. $T \rightarrow aD$\\
2. $T \rightarrow bB_1TAb$\\
3. $D \rightarrow bB_1$\\
4. $B_1 \rightarrow bAbB_1$\\
5. $A \rightarrow bB_1TAbbb$\\
6. $A \rightarrow bbb$\\
7. $B_1 \rightarrow \varepsilon$\\
8. $B \rightarrow bB_1TAb$\\
9. $B \rightarrow b$\\
10. $T \rightarrow b$\\

Рассмотрим правило 2. Добавим правило $N_1 \rightarrow B_1TA$ и выполним замену:\\
1. $T \rightarrow aD$\\
2. $T \rightarrow bN_1b$\\
3. $D \rightarrow bB_1$\\
4. $B_1 \rightarrow bAbB_1$\\
5. $A \rightarrow bB_1TAbbb$\\
6. $A \rightarrow bbb$\\
7. $B_1 \rightarrow \varepsilon$\\
8. $B \rightarrow bB_1TAb$\\
9. $B \rightarrow b$\\
10. $T \rightarrow b$\\
11. $N_1 \rightarrow B_1TA$\\

Рассмотрим правило 8. Заменим $B_1TA$ на $N_1$:\\
1. $T \rightarrow aD$\\
2. $T \rightarrow bN_1b$\\
3. $D \rightarrow bB_1$\\
4. $B_1 \rightarrow bAbB_1$\\
5. $A \rightarrow bB_1TAbbb$\\
6. $A \rightarrow bbb$\\
7. $B_1 \rightarrow \varepsilon$\\
8. $B \rightarrow bN_1b$\\
9. $B \rightarrow b$\\
10. $T \rightarrow b$\\
11. $N_1 \rightarrow B_1TA$\\



\end{enumerate}

\textbf{Вывод: } в ходе лабораторной работы изучили основные эквивалентные преобразования
КС-грамматик и научились применять их для получения
КС-грамматик, обладающих заданными свойствами.

\end{document}