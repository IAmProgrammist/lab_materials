\documentclass[a4paper,14pt]{extarticle}


\usepackage[english,russian]{babel}
\usepackage[T2A]{fontenc}
\usepackage[utf8]{inputenc}
\usepackage{ragged2e}
\usepackage[utf8]{inputenc}
\usepackage{hyperref}
\usepackage{minted}
\setmintedinline{frame=lines, framesep=2mm, baselinestretch=1.5, bgcolor=LightGray, breaklines,fontsize=\scriptsize}
\setminted{frame=lines, framesep=2mm, baselinestretch=1.5, bgcolor=LightGray, breaklines,fontsize=\scriptsize}
\usepackage{xcolor}
\definecolor{LightGray}{gray}{0.9}
\usepackage{graphicx}
\usepackage[export]{adjustbox}
\usepackage[left=1cm,right=1cm, top=1cm,bottom=1cm,bindingoffset=0cm]{geometry}
\usepackage{fontspec}
\usepackage{ upgreek }
\usepackage[shortlabels]{enumitem}
\usepackage{adjustbox}
\usepackage{multirow}
\usepackage{amsmath}
\usepackage{amssymb}
\usepackage{pifont}
\usepackage{pgfplots}
\usepackage{longtable}
\usepackage{array}
\graphicspath{ {./images/} }
\makeatletter
\AddEnumerateCounter{\asbuk}{\russian@alph}{щ}
\makeatother
\setmonofont{Consolas}
\setmainfont{Times New Roman}

\newcommand\textbox[1]{
	\parbox{.45\textwidth}{#1}
} 

\newcommand{\specialcell}[2][c]{%
	\begin{tabular}[#1]{@{}c@{}}#2\end{tabular}}

\begin{document}
\pagenumbering{gobble}
\begin{center}
    \small{
        \textbf{МИНИCТЕРCТВО НАУКИ И ВЫCШЕГО ОБРАЗОВАНИЯ РОCCИЙCКОЙ ФЕДЕРАЦИИ}\\
        ФЕДЕРАЛЬНОЕ ГОCУДАРCТВЕННОЕ БЮДЖЕТНОЕ ОБРАЗОВАТЕЛЬНОЕ УЧРЕЖДЕНИЕ\\ВЫCШЕГО ОБРАЗОВАНИЯ \\
        \textbf{«БЕЛГОРОДCКИЙ ГОCУДАРCТВЕННЫЙ ТЕХНОЛОГИЧЕCКИЙ\\УНИВЕРCИТЕТ им. В. Г. ШУХОВА»\\ (БГТУ им. В.Г. Шухова)} \\
        \bigbreak
        \includegraphics[width=70mm]{log}\\
        ИНСТИТУТ ИНФОРМАЦИОННЫХ ТЕХНОЛОГИЙ И УПРАВЛЯЮЩИХ СИСТЕМ\\}
\end{center}

\vfill
\begin{center}
    \large{
        \textbf{
            Лабораторная работа №4.2}}\\
    \normalsize{
        по дисциплине: Дискретная математика \\
        тема: «Циклы»}
\end{center}
\vfill
\hfill\textbox{
    Выполнил: ст. группы ПВ-223\\Пахомов Владислав Андреевич
    \bigbreak
    Проверили: \\ст. пр. Рязанов Юрий Дмитриевич\\
    ст. пр. Бондаренко Татьяна Владимировна
}
\vfill\begin{center}
    Белгород 2023 г.
\end{center}
\newpage
\begin{center}
    \textbf{Лабораторная работа №4.2}\\
    Циклы
\end{center}
\textbf{Цель работы: }изучить разновидности циклов в графах, научиться генерировать случайные графы,
определять их принадлежность к множеству эйлеровых и гамильтоновых графов, находить все эйлеровы и гамильтоновы циклы в графах.

\begin{enumerate}[1.]
    \item Разработать и реализовать алгоритм генерации случайного графа,
          содержащего $n$ вершин и $m$ ребер.\\
          \begin{minted}{C++}
#include "../../libs/alg/alg.h"

#include <random>

typedef Node<int> IntNode;

Graph<Edge<IntNode>>* generateRandomGraph(int nodesCount, int edgeCount) {
    auto graph = new AdjacencyMatrixGraph<Edge<IntNode>>();

    for (int i = 1; i <= nodesCount; i++) {
        Node<int> n(i);

        graph->addNode(n);
    }

    std::vector<std::pair<int, int>> edges;
    for (int i = 1; i <= nodesCount; i++) {
        for (int j = i + 1; j <= nodesCount; j++) {
            if (i == j) continue;

            edges.push_back({i, j});
        }
    }

    std::random_device rd;
    std::mt19937 g(rd());

    std::shuffle(edges.begin(), edges.end(), g);

    for (int i = 0; i < edgeCount && i < edges.size(); i++) 
        graph->addEdge({{(*graph)[edges[i].first], 
        (*graph)[edges[i].second]}, false});

    return graph;
}
    \end{minted}
    \item Написать программу, которая:\\
          \begin{enumerate}[label=\asbuk*),ref=\asbuk*]
              \item в течение десяти секунд генерирует случайные графы, содержащие $n$ вершин и $m$ ребер;
              \item для каждого полученного графа определяет, является ли он эйлеровым или гамильтоновым;
              \item подсчитывает общее количество сгенерированных графов и количество графов каждого типа.
          \end{enumerate}
          Результат работы программы представить в виде таблицы. Величину h подобрать такой, чтобы в таблице количество
          строк было в диапазоне от 20 до 30.
          \textit{task1.tpp}
          \begin{minted}{C++}
#include "../lab11/graph.tpp"

template <typename E, typename N>
bool AdjacencyMatrixGraph<E, N>::isHamiltonian() {
    if (this->nodes.size() < 3) return false;

    std::vector<bool> cache(this->nodes.size(), false);

    return hasHamiltonianCycle(0, 0, cache);
}

template <typename E, typename N>
bool AdjacencyMatrixGraph<E, N>::hasHamiltonianCycle(int originIndex, int startIndex, std::vector<bool>& takenNodes) {
    takenNodes[startIndex] = true;

    bool anyElementFound = false;
    for (int i = 0; i < this->nodes.size(); i++) {
        if (takenNodes[i] || this->edges[startIndex][i] == nullptr) continue;

        anyElementFound = true;
        
        if (hasHamiltonianCycle(originIndex, i, takenNodes)) return true;
    }

    if (!anyElementFound) {
        if (this->edges[startIndex][originIndex] == nullptr) {
            takenNodes[startIndex] = false;

            return false;
        }

        for (auto takenFlag : takenNodes) 
            if (!takenFlag) {
                takenNodes[startIndex] = false;

                return false;
            }

        return true;
    }

    takenNodes[startIndex] = false;
    return false;
}

template <typename E, typename N>
bool AdjacencyMatrixGraph<E, N>::isEuler() {
    if (this->nodes.size() < 3) return false;

    // Строим матрицу M 
    BoolMatrixRelation linkMatrix(this->nodes.size(), [this](int x, int y) {
        return this->edges[x - 1][y - 1] != nullptr;
    });

    // Получаем матрицу C = I + M+ и формируем из C фактормножество
    auto factorSet = BoolMatrixRelation::getIdentity(this->nodes.size())
                                        .unite(linkMatrix.transitiveClosureWarshall(nullptr))
                                        .getPackedFactorSet();
    
    // Если в полученном фактормножестве несколько классов эквивалентности, то множество несвязное.
    for (int i = 1; i < factorSet.size(); i++) 
        if (factorSet[0] != factorSet[i]) return false;

    for (int i = 0; i < this->nodes.size(); i++) {
        int nodePow = 0;
        for (int j = 0; j < this->nodes.size(); j++) 
            nodePow += (this->edges[i][j] != nullptr);

        // Степень каждой вершины должна быть чётна.
        if (nodePow % 2 != 0) return false;
    }

    return true;
}
\end{minted}

          \textit{main.cpp}
          \begin{minted}{C++}
#include "task1.h"

#include <thread>
#include <future>
#include <unistd.h>

struct Report {
    int n;
    int edgesCount;
    int hamilthonCount;
    int eulerCount;
    int totalCount;
};

#define LOWER_N 8
#define UPPER_N 10
#define ELEMENTS_FOR_N 10

#define CORES_AMOUNT 10
std::atomic<int> takenCores;

int main() {
    std::vector<std::pair<std::chrono::_V2::system_clock::time_point, std::future<Report>>> pool;
    for (int n = LOWER_N; n <= UPPER_N; n++) {
        for (int i = 0; i < ELEMENTS_FOR_N; i++) {
            while (takenCores >= CORES_AMOUNT) {
                sleep(1);
                std::cout << "Working on it..." << "\n";
            }

            int e = n;
            if (i != 0)
                e += (i * ((((n * (n - 1)) / 2 - n))))  / (ELEMENTS_FOR_N - 1);

            takenCores++;
            pool.push_back({std::chrono::system_clock::now(), std::async(std::launch::async, [n, e] {
                int totalGraphsGenerated = 0;
                int hamilthonGraphs = 0;
                int eulerGraphs = 0;
                auto start = std::chrono::system_clock::now();
                while (std::chrono::system_clock::now() - start < std::chrono::seconds(10)) {
                    auto graph = generateRandomGraph(n, e);
                    totalGraphsGenerated++;

                    if (graph->isHamiltonian()) hamilthonGraphs++;
                    if (graph->isEuler()) eulerGraphs++;

                    delete graph;
                }

                takenCores--;

                return (Report){n, e, hamilthonGraphs, eulerGraphs, totalGraphsGenerated};
            })});
        }
    }



    std::cout << "Waiting for results...\n\n";
    for (auto& future : pool) {
        if (future.second.wait_until(future.first + std::chrono::seconds(60)) != std::future_status::ready) {
            std::cout << "====================\n";
            std::cout << "Thread is not responing, consider no result is present.\n";

            continue;
        }

        auto t = future.second.get();

        std::cout << "\\hline\n";
        std::cout << t.n << "&"<<t.edgesCount << "&" << t.eulerCount << "&" << t.hamilthonCount << "&" << t.totalCount << "\\\\\n";
    }

    getchar();

    return 0;
}
\end{minted}
          \begin{center}\textbf{Результаты работы программы}\end{center}
          \begin{tabular}{|c|c|c|c|c|}
              \hline
              \multirow{2}{*}{Количество вершин} & \multirow{2}{*}{Количество рёбер} & \multicolumn{3}{|c|}{Количество графов}                          \\
              \cline{3-5}
                                                 &                                   & эйлеровых                               & гамильтоновых & всех   \\
              \hline
              8                                  & 8                                 & 102                                     & 102           & 138589 \\
              \hline
              8                                  & 10                                & 647                                     & 4777          & 139096 \\
              \hline
              8                                  & 12                                & 834                                     & 30119         & 121321 \\
              \hline
              8                                  & 14                                & 897                                     & 75835         & 121929 \\
              \hline
              8                                  & 16                                & 1766                                    & 195543        & 223236 \\
              \hline
              8                                  & 19                                & 918                                     & 113577        & 114899 \\
              \hline
              8                                  & 21                                & 845                                     & 103548        & 103659 \\
              \hline
              8                                  & 23                                & 898                                     & 102016        & 102016 \\
              \hline
              8                                  & 25                                & 0                                       & 94443         & 94443  \\
              \hline
              8                                  & 28                                & 0                                       & 92151         & 92151  \\
              \hline
              9                                  & 9                                 & 82                                      & 82            & 325377 \\
              \hline
              9                                  & 12                                & 366                                     & 6651          & 164324 \\
              \hline
              9                                  & 15                                & 532                                     & 58577         & 157293 \\
              \hline
              9                                  & 18                                & 491                                     & 105194        & 133994 \\
              \hline
              9                                  & 21                                & 526                                     & 126056        & 131892 \\
              \hline
              9                                  & 24                                & 476                                     & 124748        & 125456 \\
              \hline
              9                                  & 27                                & 415                                     & 109102        & 109134 \\
              \hline
              9                                  & 30                                & 433                                     & 110253        & 110253 \\
              \hline
              9                                  & 33                                & 1174                                    & 101690        & 101690 \\
              \hline
              9                                  & 36                                & 98078                                   & 98078         & 98078  \\
              \hline
              10                                 & 10                                & 6                                       & 6             & 126670 \\
              \hline
              10                                 & 13                                & 175                                     & 2991          & 209716 \\
              \hline
              10                                 & 17                                & 184                                     & 37283         & 115934 \\
              \hline
              10                                 & 21                                & 169                                     & 75209         & 95293  \\
              \hline
              10                                 & 25                                & 193                                     & 94958         & 98776  \\
              \hline
              10                                 & 29                                & 249                                     & 145783        & 146400 \\
              \hline
              10                                 & 33                                & 182                                     & 96687         & 96711  \\
              \hline
              10                                 & 37                                & 160                                     & 86866         & 86866  \\
              \hline
              10                                 & 41                                & 0                                       & 85518         & 85518  \\
              \hline
              10                                 & 45                                & 0                                       & 79227         & 79227  \\
              \hline
          \end{tabular}

    \item Выполнить программу при n = 8,9,10 и сделать выводы.\\
          Чем больше рёбер в графе, тем больше вероятность того, что он окажется
          гамильтоновым.
          Рост вероятности того, что граф окажется эйлеровым с ростом количества рёбер сначала растёт, потом - падает.
    \item Привести пример диаграммы графа, который является эйлеровым,
          но не гамильтоновым. Найти в нем все эйлеровы циклы.\\
          \includegraphics[width=100mm]{4}\\
          Эйлеровы циклы:\\
          $\{1, 2, 3, 5, 2, 4, 1\}$\\
          $\{2, 3, 5, 2, 4, 1, 2\}$\\
          $\{3, 5, 2, 4, 1, 2, 3\}$\\
          $\{5, 2, 4, 1, 2, 3, 5\}$\\
          $\{2, 4, 1, 2, 3, 5, 2\}$\\
          $\{4, 1, 2, 3, 5, 2, 4\}$\\

          $\{1, 4, 2, 5, 3, 2, 1\}$\\
          $\{4, 2, 5, 3, 2, 1, 4\}$\\
          $\{2, 5, 3, 2, 1, 4, 2\}$\\
          $\{5, 3, 2, 1, 4, 2, 5\}$\\
          $\{3, 2, 1, 4, 2, 5, 3\}$\\
          $\{2, 1, 4, 2, 5, 3, 2\}$\\

          $\{1, 2, 5, 3, 2, 4, 1\}$\\
          $\{2, 5, 3, 2, 4, 1, 2\}$\\
          $\{5, 3, 2, 4, 1, 2, 5\}$\\
          $\{3, 2, 4, 1, 2, 5, 3\}$\\
          $\{2, 4, 1, 2, 5, 3, 2\}$\\
          $\{4, 1, 2, 5, 3, 2, 4\}$\\

          $\{1, 4, 2, 3, 5, 2, 1\}$\\
          $\{4, 2, 3, 5, 2, 1, 4\}$\\
          $\{2, 3, 5, 2, 1, 4, 2\}$\\
          $\{3, 5, 2, 1, 4, 2, 3\}$\\
          $\{5, 2, 1, 4, 2, 3, 5\}$\\
          $\{2, 1, 4, 2, 3, 5, 2\}$\\

    \item Привести пример диаграммы графа, который является
          гамильтоновым, но не эйлеровым. Найти в нем все гамильтоновы циклы.\\
          \includegraphics[width=100mm]{5}\\
          Гамильтоновы циклы:\\
          $\{1, 2, 4, 3, 1\}$\\
          $\{2, 4, 3, 1, 2\}$\\
          $\{4, 3, 1, 2, 4\}$\\
          $\{3, 1, 2, 4, 3\}$\\
          $\{1, 3, 4, 2, 1\}$\\
          $\{3, 4, 2, 1, 3\}$\\
          $\{4, 2, 1, 3, 4\}$\\
          $\{2, 1, 3, 4, 2\}$

    \item Привести пример диаграммы графа, который является эйлеровым
          и гамильтоновым. Найти в нем все эйлеровы и гамильтоновы циклы.\\
          \includegraphics[width=100mm]{6}\\
          Эйлеровы циклы:\\
          $\{1, 2, 3, 1\}$\\
          $\{2, 3, 1, 2\}$\\
          $\{3, 1, 2, 3\}$\\
          $\{1, 3, 2, 1\}$\\
          $\{3, 2, 1, 3\}$\\
          $\{2, 1, 3, 2\}$\\
          Гамильтоновы циклы:\\
          $\{1, 2, 3, 1\}$\\
          $\{2, 3, 1, 2\}$\\
          $\{3, 1, 2, 3\}$\\
          $\{1, 3, 2, 1\}$\\
          $\{3, 2, 1, 3\}$\\
          $\{2, 1, 3, 2\}$

    \item Привести пример диаграммы графа, который не является ни
          эйлеровым, ни гамильтоновым.\\
          \includegraphics[width=100mm]{7}

\end{enumerate}
\textbf{Вывод: } в ходе лабораторной работы изучили разновидности циклов в графах, научились генерировать случайные графы,
определять их принадлежность к множеству эйлеровых и гамильтоновых графов, находить все эйлеровы и гамильтоновы циклы в графах.
\end{document}