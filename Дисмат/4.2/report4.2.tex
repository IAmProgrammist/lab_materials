\documentclass[a4paper,14pt]{extarticle}


\usepackage[english,russian]{babel}
\usepackage[T2A]{fontenc}
\usepackage[utf8]{inputenc}
\usepackage{ragged2e}
\usepackage[utf8]{inputenc}
\usepackage{hyperref}
\usepackage{minted}
\setmintedinline{frame=lines, framesep=2mm, baselinestretch=1.5, bgcolor=LightGray, breaklines,fontsize=\scriptsize}
\setminted{frame=lines, framesep=2mm, baselinestretch=1.5, bgcolor=LightGray, breaklines,fontsize=\scriptsize}
\usepackage{xcolor}
\definecolor{LightGray}{gray}{0.9}
\usepackage{graphicx}
\usepackage[export]{adjustbox}
\usepackage[left=1cm,right=1cm, top=1cm,bottom=1cm,bindingoffset=0cm]{geometry}
\usepackage{fontspec}
\usepackage{ upgreek }
\usepackage[shortlabels]{enumitem}
\usepackage{adjustbox}
\usepackage{multirow}
\usepackage{amsmath}
\usepackage{amssymb}
\usepackage{pifont}
\usepackage{pgfplots}
\usepackage{longtable}
\usepackage{array}
\graphicspath{ {./images/} }
\makeatletter
\AddEnumerateCounter{\asbuk}{\russian@alph}{щ}
\makeatother
\setmonofont{Consolas}
\setmainfont{Times New Roman}

\newcommand\textbox[1]{
	\parbox{.45\textwidth}{#1}
} 

\newcommand{\specialcell}[2][c]{%
	\begin{tabular}[#1]{@{}c@{}}#2\end{tabular}}

\begin{document}
\pagenumbering{gobble}
\begin{center}
    \small{
        \textbf{МИНИCТЕРCТВО НАУКИ И ВЫCШЕГО ОБРАЗОВАНИЯ РОCCИЙCКОЙ ФЕДЕРАЦИИ}\\
        ФЕДЕРАЛЬНОЕ ГОCУДАРCТВЕННОЕ БЮДЖЕТНОЕ ОБРАЗОВАТЕЛЬНОЕ УЧРЕЖДЕНИЕ\\ВЫCШЕГО ОБРАЗОВАНИЯ \\
        \textbf{«БЕЛГОРОДCКИЙ ГОCУДАРCТВЕННЫЙ ТЕХНОЛОГИЧЕCКИЙ\\УНИВЕРCИТЕТ им. В. Г. ШУХОВА»\\ (БГТУ им. В.Г. Шухова)} \\
        \bigbreak
        \includegraphics[width=70mm]{log}\\
        ИНСТИТУТ ИНФОРМАЦИОННЫХ ТЕХНОЛОГИЙ И УПРАВЛЯЮЩИХ СИСТЕМ\\}
\end{center}

\vfill
\begin{center}
    \large{
        \textbf{
            Лабораторная работа №4.2}}\\
    \normalsize{
        по дисциплине: Дискретная математика \\
        тема: «Циклы»}
\end{center}
\vfill
\hfill\textbox{
    Выполнил: ст. группы ПВ-223\\Пахомов Владислав Андреевич
    \bigbreak
    Проверили: \\ст. пр. Рязанов Юрий Дмитриевич\\
    ст. пр. Бондаренко Татьяна Владимировна
}
\vfill\begin{center}
    Белгород 2023 г.
\end{center}
\newpage
\begin{center}
    \textbf{Лабораторная работа №4.2}\\
    Циклы
\end{center}
\textbf{Цель работы: }изучить разновидности циклов в графах, научиться генерировать случайные графы,
определять их принадлежность к множеству эйлеровых и гамильтоновых графов, находить все эйлеровы и гамильтоновы циклы в графах.

\begin{enumerate}[1.]
    \item Разработать и реализовать алгоритм генерации случайного графа,
          содержащего $n$ вершин и $m$ ребер.\\
          \begin{minted}{C++}
#include "../../libs/alg/alg.h"

#include <random>

typedef Node<int> IntNode;

Graph<Edge<IntNode>>* generateRandomGraph(int nodesCount, int edgeCount) {
    auto graph = new AdjacencyMatrixGraph<Edge<IntNode>>();

    for (int i = 1; i <= nodesCount; i++) {
        Node<int> n(i);

        graph->addNode(n);
    }

    std::vector<std::pair<int, int>> edges;
    for (int i = 1; i <= nodesCount; i++) {
        for (int j = 1; j <= nodesCount; j++) {
            if (i == j) continue;

            edges.push_back({i, j});
        }
    }

    std::random_device rd;
    std::mt19937 g(rd());

    std::shuffle(edges.begin(), edges.end(), g);

    for (int i = 0; i < edgeCount && i < edges.size(); i++) 
        graph->addEdge({{(*graph)[edges[i].first], 
        (*graph)[edges[i].second]}, false});

    return graph;
}
    \end{minted}
    \item Написать программу, которая:\\
          \begin{enumerate}[label=\asbuk*),ref=\asbuk*]
              \item в течение десяти секунд генерирует случайные графы, содержащие $n$ вершин и $m$ ребер;
              \item для каждого полученного графа определяет, является ли он эйлеровым или гамильтоновым;
              \item подсчитывает общее количество сгенерированных графов и количество графов каждого типа.
          \end{enumerate}
          Результат работы программы представить в виде таблицы. Величину h подобрать такой, чтобы в таблице количество
          строк было в диапазоне от 20 до 30.
          \textit{task1.tpp}
          \begin{minted}{C++}
#include "../lab11/graph.tpp"

template <typename E, typename N>
bool AdjacencyMatrixGraph<E, N>::isHamiltonian() {
    if (this->nodes.size() < 3) return false;

    std::vector<bool> cache(this->nodes.size(), false);

    return hasHamiltonianCycle(0, 0, cache);
}

template <typename E, typename N>
bool AdjacencyMatrixGraph<E, N>::hasHamiltonianCycle(int originIndex, int startIndex, std::vector<bool>& takenNodes) {
    takenNodes[startIndex] = true;

    bool anyElementFound = false;
    for (int i = 0; i < this->nodes.size(); i++) {
        if (takenNodes[i] || this->edges[startIndex][i] == nullptr) continue;

        anyElementFound = true;
        
        if (hasHamiltonianCycle(originIndex, i, takenNodes)) return true;
    }

    if (!anyElementFound) {
        if (this->edges[startIndex][originIndex] == nullptr) {
            takenNodes[startIndex] = false;

            return false;
        }

        for (auto takenFlag : takenNodes) 
            if (!takenFlag) {
                takenNodes[startIndex] = false;

                return false;
            }

        return true;
    }

    takenNodes[startIndex] = false;
    return false;
}

template <typename E, typename N>
bool AdjacencyMatrixGraph<E, N>::isEuler() {
    if (this->nodes.size() < 3) return false;

    // Строим матрицу M 
    BoolMatrixRelation linkMatrix(this->nodes.size(), [this](int x, int y) {
        return this->edges[x - 1][y - 1] != nullptr;
    });

    // Получаем матрицу C = I + M+ и формируем из C фактормножество
    auto factorSet = BoolMatrixRelation::getIdentity(this->nodes.size())
                                        .unite(linkMatrix.transitiveClosureWarshall(nullptr))
                                        .getPackedFactorSet();
    
    // Если в полученном фактормножестве несколько классов эквивалентности, то множество несвязное.
    for (int i = 1; i < factorSet.size(); i++) 
        if (factorSet[0] != factorSet[i]) return false;

    for (int i = 0; i < this->nodes.size(); i++) {
        int nodePow = 0;
        for (int j = 0; j < this->nodes.size(); j++) 
            nodePow += (this->edges[i][j] != nullptr);

        // Степень каждой вершины должна быть чётна.
        if (nodePow % 2 != 0) return false;
    }

    return true;
}
\end{minted}

          \textit{main.cpp}
          \begin{minted}{C++}
#include "task1.h"

#include <thread>
#include <future>
#include <unistd.h>

struct Report {
    int n;
    double h;
    int edgesCount;
    int hamilthonCount;
    int eulerCount;
    int totalCount;
};

#define LOWER_N 8
#define UPPER_N 10
#define ELEMENTS_FOR_N 20

#define CORES_AMOUNT 10
std::atomic<int> takenCores;

int main() {
    std::vector<std::pair<std::chrono::_V2::system_clock::time_point, std::future<Report>>> pool;
    for (int n = LOWER_N; n <= UPPER_N; n++) {
        double h = ((n * (n - 1)) / 2 - 1.0 * n) / (ELEMENTS_FOR_N);
        
        for (double e = n; e <= ((n * (n - 1)) / 2); e += h) {
            while (takenCores >= CORES_AMOUNT) {
                sleep(1);
                std::cout << "Working on it..." << "\n";
            }

            takenCores++;
            pool.push_back({std::chrono::system_clock::now(), std::async(std::launch::async, [n, h, e] {
                int totalGraphsGenerated = 0;
                int hamilthonGraphs = 0;
                int eulerGraphs = 0;
                auto start = std::chrono::system_clock::now();
                while (std::chrono::system_clock::now() - start < std::chrono::seconds(10)) {
                    auto graph = generateRandomGraph(n, static_cast<int>(e));
                    totalGraphsGenerated++;

                    if (graph->isHamiltonian()) hamilthonGraphs++;
                    if (graph->isEuler()) eulerGraphs++;

                    delete graph;
                }

                takenCores--;

                return (Report){n, h, static_cast<int>(e), hamilthonGraphs, eulerGraphs, totalGraphsGenerated};
            })});
        }
    }



    std::cout << "Waiting for results...\n\n";
    for (auto& future : pool) {
        if (future.second.wait_until(future.first + std::chrono::seconds(60)) != std::future_status::ready) {
            std::cout << "====================\n";
            std::cout << "Thread is not responing, consider no result is present.\n";

            continue;
        }

        auto t = future.second.get();

        std::cout << "====================\n";
        std::cout << "n = " << t.n << "\n";
        std::cout << "h = " << t.h << "\n";
        std::cout << "edges = " << t.edgesCount << "\n";
        std::cout << "hamilthon = " << t.hamilthonCount << "\n";
        std::cout << "euler = " << t.eulerCount << "\n";
        std::cout << "total = " << t.totalCount << "\n\n";
    }

    getchar();

    return 0;
}
\end{minted}
          \begin{center}\textbf{Результаты работы программы (n = 8, h = 1)}\end{center}
          \begin{tabular}{|c|c|c|c|c|}
              \hline
              \multirow{2}{*}{Количество вершин} & \multirow{2}{*}{Количество рёбер} & \multicolumn{3}{|c|}{Количество графов}                          \\
              \cline{3-5}
                                                 &                                   & эйлеровых                               & гамильтоновых & всех   \\
              \hline
              8                                  & 8                                 & 103                                     & 103           & 200820 \\
              \hline
              8                                  & 10                                & 540                                     & 2897          & 183051 \\
              \hline
              8                                  & 12                                & 866                                     & 18442         & 164452 \\
              \hline
              8                                  & 14                                & 1172                                    & 54080         & 166376 \\
              \hline
              8                                  & 16                                & 1023                                    & 84642         & 147675 \\
              \hline
              8                                  & 18                                & 1132                                    & 112615        & 146409 \\
              \hline
              8                                  & 20                                & 1080                                    & 128453        & 144606 \\
              \hline
              8                                  & 22                                & 1007                                    & 121600        & 127881 \\
              \hline
              8                                  & 24                                & 981                                     & 128928        & 131627 \\
              \hline
              8                                  & 26                                & 991                                     & 129678        & 130709 \\
              \hline
              8                                  & 28                                & 1039                                    & 131660        & 132050 \\
              \hline
              9                                  & 9                                 & 32                                      & 32            & 256243 \\
              \hline
              9                                  & 11                                & 258                                     & 1303          & 233387 \\
              \hline
              9                                  & 14                                & 584                                     & 22228         & 215795 \\
              \hline
              9                                  & 17                                & 565                                     & 66885         & 171215 \\
              \hline
              9                                  & 19                                & 581                                     & 100800        & 166941 \\
              \hline
              9                                  & 22                                & 618                                     & 133776        & 161496 \\
              \hline
              9                                  & 25                                & 572                                     & 135365        & 144932 \\
              \hline
              9                                  & 27                                & 556                                     & 146455        & 151423 \\
              \hline
              9                                  & 30                                & 554                                     & 133836        & 135290 \\
              \hline
              9                                  & 33                                & 496                                     & 128841        & 129238 \\
              \hline
              9                                  & 36                                & 446                                     & 124971        & 125083 \\
              \hline
              10                                 & 10                                & 8                                       & 8             & 222246 \\
              \hline
              10                                 & 13                                & 111                                     & 1475          & 211436 \\
              \hline
              10                                 & 17                                & 202                                     & 23527         & 146719 \\
              \hline
              10                                 & 20                                & 248                                     & 61836         & 141350 \\
              \hline
              10                                 & 24                                & 247                                     & 102376        & 134103 \\
              \hline
              10                                 & 27                                & 248                                     & 110193        & 123522 \\
              \hline
              10                                 & 31                                & 238                                     & 117785        & 121892 \\
              \hline
              10                                 & 34                                & 274                                     & 123179        & 124767 \\
              \hline
              10                                 & 38                                & 357                                     & 170638        & 171228 \\
              \hline
              10                                 & 41                                & 325                                     & 167164        & 167356 \\
              \hline
              10                                 & 45                                & 306                                     & 159020        & 159064 \\
              \hline
          \end{tabular}

    \item Выполнить программу при n = 8,9,10 и сделать выводы.\\
          Чем больше рёбер в графе, тем больше вероятность того, что он окажется
          гамильтоновым.
          Рост вероятности того, что граф окажется эйлеровым с ростом количества рёбер не такой значительный.
    \item Привести пример диаграммы графа, который является эйлеровым,
          но не гамильтоновым. Найти в нем все эйлеровы циклы.\\
          \includegraphics[width=100mm]{4}\\
          Эйлеровы циклы:\\
          $\{1, 2, 3, 1\}$\\
          $\{2, 3, 1, 2\}$\\
          $\{3, 1, 2, 3\}$\\
          $\{1, 3, 2, 1\}$\\
          $\{3, 2, 1, 3\}$\\
          $\{2, 1, 3, 2\}$

    \item Привести пример диаграммы графа, который является
          гамильтоновым, но не эйлеровым. Найти в нем все гамильтоновы циклы.\\
          \includegraphics[width=100mm]{5}\\
          Гамильтоновы циклы:\\
          $\{1, 2, 4, 3, 1\}$\\
          $\{2, 4, 3, 1, 2\}$\\
          $\{4, 3, 1, 2, 4\}$\\
          $\{3, 1, 2, 4, 3\}$\\
          $\{1, 3, 4, 2, 1\}$\\
          $\{3, 4, 2, 1, 3\}$\\
          $\{4, 2, 1, 3, 4\}$\\
          $\{2, 1, 3, 4, 2\}$

    \item Привести пример диаграммы графа, который является эйлеровым
          и гамильтоновым. Найти в нем все эйлеровы и гамильтоновы циклы.\\
          \includegraphics[width=100mm]{6}\\
          Эйлеровы циклы:\\
          $\{1, 2, 3, 1\}$\\
          $\{2, 3, 1, 2\}$\\
          $\{3, 1, 2, 3\}$\\
          $\{1, 3, 2, 1\}$\\
          $\{3, 2, 1, 3\}$\\
          $\{2, 1, 3, 2\}$\\
          Гамильтоновы циклы:\\
          $\{1, 2, 3, 1\}$\\
          $\{2, 3, 1, 2\}$\\
          $\{3, 1, 2, 3\}$\\
          $\{1, 3, 2, 1\}$\\
          $\{3, 2, 1, 3\}$\\
          $\{2, 1, 3, 2\}$

    \item Привести пример диаграммы графа, который не является ни
          эйлеровым, ни гамильтоновым.\\
          \includegraphics[width=100mm]{7}

\end{enumerate}
\textbf{Вывод: } в ходе лабораторной работы изучили разновидности циклов в графах, научились генерировать случайные графы,
определять их принадлежность к множеству эйлеровых и гамильтоновых графов, находить все эйлеровы и гамильтоновы циклы в графах.
\end{document}