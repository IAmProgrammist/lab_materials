\documentclass[a4paper,14pt]{extarticle}


\usepackage[english,russian]{babel}
\usepackage[T2A]{fontenc}
\usepackage[utf8]{inputenc}
\usepackage{ragged2e}
\usepackage[utf8]{inputenc}
\usepackage{hyperref}
\usepackage{minted}
\setmintedinline{frame=lines, framesep=2mm, baselinestretch=1.5, bgcolor=LightGray, breaklines,fontsize=\scriptsize}
\setminted{frame=lines, framesep=2mm, baselinestretch=1.5, bgcolor=LightGray, breaklines,fontsize=\scriptsize}
\usepackage{xcolor}
\definecolor{LightGray}{gray}{0.9}
\usepackage{graphicx}
\usepackage[export]{adjustbox}
\usepackage[left=1cm,right=1cm, top=1cm,bottom=1cm,bindingoffset=0cm]{geometry}
\usepackage{fontspec}
\usepackage{ upgreek }
\usepackage[shortlabels]{enumitem}
\usepackage{adjustbox}
\usepackage{multirow}
\usepackage{amsmath}
\usepackage{amssymb}
\usepackage{pifont}
\usepackage{pgfplots}
\usepackage{longtable}
\usepackage{array}
\graphicspath{ {./images/} }
\makeatletter
\AddEnumerateCounter{\asbuk}{\russian@alph}{щ}
\makeatother
\setmonofont{Consolas}
\setmainfont{Times New Roman}

\newcommand\textbox[1]{
	\parbox{.45\textwidth}{#1}
} 

\newcommand{\specialcell}[2][c]{%
	\begin{tabular}[#1]{@{}c@{}}#2\end{tabular}}

\begin{document}
\pagenumbering{gobble}
\begin{center}
    \small{
        \textbf{МИНИCТЕРCТВО НАУКИ И ВЫCШЕГО ОБРАЗОВАНИЯ РОCCИЙCКОЙ ФЕДЕРАЦИИ}\\
        ФЕДЕРАЛЬНОЕ ГОCУДАРCТВЕННОЕ БЮДЖЕТНОЕ ОБРАЗОВАТЕЛЬНОЕ УЧРЕЖДЕНИЕ\\ВЫCШЕГО ОБРАЗОВАНИЯ \\
        \textbf{«БЕЛГОРОДCКИЙ ГОCУДАРCТВЕННЫЙ ТЕХНОЛОГИЧЕCКИЙ\\УНИВЕРCИТЕТ им. В. Г. ШУХОВА»\\ (БГТУ им. В.Г. Шухова)} \\
        \bigbreak
        \includegraphics[width=70mm]{log}\\
        ИНСТИТУТ ИНФОРМАЦИОННЫХ ТЕХНОЛОГИЙ И УПРАВЛЯЮЩИХ СИСТЕМ\\}
\end{center}

\vfill
\begin{center}
    \large{
        \textbf{
            Лабораторная работа №4.1}}\\
    \normalsize{
        по дисциплине: Дискретная математика \\
        тема: «Маршруты»}
\end{center}
\vfill
\hfill\textbox{
    Выполнил: ст. группы ПВ-223\\Пахомов Владислав Андреевич
    \bigbreak
    Проверили: \\ст. пр. Рязанов Юрий Дмитриевич\\
    ст. пр. Бондаренко Татьяна Владимировна
}
\vfill\begin{center}
    Белгород 2023 г.
\end{center}
\newpage
\begin{center}
    \textbf{Лабораторная работа №4.1}\\
    Маршруты\\
    Вариант 10
\end{center}
\textbf{Цель работы: }изучить основные понятия теории графов, способы
задания графов, научиться программно реализовывать
алгоритмы получения и анализа маршрутов в графах.

\begin{enumerate}[1.]
    \item Представить графы $G_1$ и $G_2$ матрицей смежности,
          матрицей инцидентности, диаграммой.\\
          $G_1$:\\
          Матрица смежности:\\
          \begin{center}
              $\begin{pmatrix}
                      0 & 1 & 0 & 1 & 1 & 1 & 1 \\
                      1 & 0 & 1 & 0 & 1 & 1 & 0 \\
                      0 & 1 & 0 & 1 & 0 & 0 & 0 \\
                      1 & 0 & 1 & 0 & 0 & 0 & 0 \\
                      1 & 1 & 0 & 0 & 0 & 0 & 0 \\
                      1 & 1 & 0 & 0 & 0 & 0 & 1 \\
                      1 & 0 & 0 & 0 & 0 & 1 & 0 \\
                  \end{pmatrix}$
          \end{center}
          \bigbreak
          Матрица инцидентности:\\
          \begin{center}
              $\begin{pmatrix}
                      1 & 1 & 1 & 1 & 1 & 0 & 0 & 0 & 0 & 0 \\
                      1 & 0 & 0 & 0 & 0 & 1 & 1 & 1 & 0 & 0 \\
                      0 & 0 & 0 & 0 & 0 & 1 & 0 & 0 & 1 & 0 \\
                      0 & 1 & 0 & 0 & 0 & 0 & 0 & 0 & 1 & 0 \\
                      0 & 0 & 1 & 0 & 0 & 0 & 1 & 0 & 0 & 0 \\
                      0 & 0 & 0 & 1 & 0 & 0 & 0 & 1 & 0 & 1 \\
                      0 & 0 & 0 & 0 & 1 & 0 & 0 & 0 & 0 & 1 \\
                  \end{pmatrix}$
          \end{center}\bigbreak
          Диаграмма:\\
          \begin{center}
              \includegraphics[width=100mm]{G1diagram}
          \end{center}\bigbreak

          $G_2$:\\
          Матрица смежности:\\
          \begin{center}
              $\begin{pmatrix}
                      0 & 1 & 0 & 1 & 0 & 1 & 1 \\
                      1 & 0 & 1 & 1 & 0 & 0 & 0 \\
                      0 & 1 & 0 & 1 & 1 & 1 & 0 \\
                      1 & 1 & 1 & 0 & 0 & 0 & 0 \\
                      0 & 0 & 1 & 0 & 0 & 1 & 0 \\
                      1 & 0 & 1 & 0 & 1 & 0 & 1 \\
                      1 & 0 & 0 & 0 & 0 & 1 & 0 \\
                  \end{pmatrix}$
          \end{center}
          \bigbreak
          \setcounter{MaxMatrixCols}{20}
          Матрица инцидентности:\\
          \begin{center}
              $\begin{pmatrix}
                      1 & 0 & 0 & 1 & 0 & 1 & 1 & 0 & 0 & 0 & 0 \\
                      1 & 1 & 0 & 0 & 0 & 0 & 0 & 1 & 0 & 0 & 0 \\
                      0 & 1 & 1 & 0 & 0 & 0 & 0 & 0 & 1 & 1 & 0 \\
                      0 & 0 & 0 & 0 & 0 & 0 & 1 & 1 & 1 & 0 & 0 \\
                      0 & 0 & 1 & 0 & 0 & 0 & 0 & 0 & 0 & 0 & 1 \\
                      0 & 0 & 0 & 0 & 1 & 1 & 0 & 0 & 0 & 1 & 1 \\
                      0 & 0 & 0 & 1 & 1 & 0 & 0 & 0 & 0 & 0 & 0 \\
                  \end{pmatrix}$
          \end{center}\bigbreak
          Диаграмма:\\
          \begin{center}
              \includegraphics[width=100mm]{G2diagram}
          \end{center}\bigbreak
          \item Определить, являются ли последовательности вершин маршрутом, цепью, 
          простой цепью, циклом, простым циклом в графах $G_1$ и $G_2$.\\

          $(2, 3, 4, 1, 7, 6)$\\
          В $G_1$ последовательность вершин является маршрутом, цепью, простой цепью. 
          Не является циклом, следовательно, и простым циклом тоже не является.
          В $G_2$ последовательность вершин является маршрутом, цепью, простой цепью. 
          Не является циклом, следовательно, и простым циклом тоже не является.\bigbreak
          $(4, 1, 6, 7, 1, 2)$\\
          В $G_1$ последовательность вершин является маршрутом, цепью, не является простой цепью (вершина 1 повторяется дважды). 
          Не является циклом, следовательно, и простым циклом тоже не является.
          В $G_2$ последовательность вершин является маршрутом, цепью, не является простой цепью (вершина 1 повторяется дважды). 
          Не является циклом, следовательно, и простым циклом тоже не является.\bigbreak
          $(1, 4, 3, 2, 1)$\\
          В $G_1$ последовательность вершин является маршрутом, цепью, не является простой цепью (вершина 1 повторяется дважды). 
          Является циклом и простым циклом.
          В $G_2$ последовательность вершин является маршрутом, цепью, не является простой цепью (вершина 1 повторяется дважды). 
          Является циклом и простым циклом.\bigbreak
          $(1, 2, 4, 3, 2, 1)$\\
          В $G_1$ последовательность вершин не является маршрутом (2 и 4 не смежны, следовательно не найдётся такого инцидентного общего ребра для обоих вершин), следовательно не обладает другими свойствами.
          В $G_2$ последовательность вершин является маршрутом, не является цепью (ребро 1-2 повторяется дважды), следовательно, не является и простой цепью, циклом и простым циклом. \bigbreak
          $(2, 4, 1, 7, 6, 1, 2)$\\
          В $G_1$ последовательность вершин не является маршрутом (2 и 4 не смежны, следовательно не найдётся такого инцидентного общего ребра для обоих вершин), следовательно не обладает другими свойствами.
          В $G_2$ последовательность вершин является маршрутом, цепью, не является простой цепью (1, 2 повторяются дважды). Является циклом, не является простым циклом (1 повторяется дважды). \bigbreak

        \item Написать программу, определяющую, является ли заданная последовательность
        вершин маршрутом, цепью, простой цепью, циклом, простым циклом в графах $G_1$ и $G_2$.\\
        Напишем вспомогательный классы, определяющие объекты графа, вершины, ребра, точечного маршрута. В дальнейшем будем использовать его.\\
        \textit{node.tpp}
        \begin{minted}{C++}
#ifndef GRAPH_NODE
#define GRAPH_NODE

template <typename T, typename V>
class NamedNode {
protected:    
    T value;

public:
    using NodeValueType = T;

    V name;

    NamedNode(T value, V name) {
        this->value = value;
        this->name  = name;
    };

    NamedNode<T, V>* clone() {
        return new NamedNode<T, V>(value, name);
    }

    virtual bool equals(const NamedNode<T, V>& b) const {
        return this->value == b.value;
    }

    const T& getValue() const {
        return value;
    }
};

template <typename T>
class Node : public NamedNode<T, T> {
public:
    Node(T value) : NamedNode<T, T>(value, value) {};

    Node<T>* clone() {
        return new Node<T>(this->value);
    }
};

#endif
        \end{minted} 
        
        \textit{edge.tpp}
        \begin{minted}{C++}
#ifndef GRAPH_EDGE
#define GRAPH_EDGE

template <typename T, typename V>
class NamedEdge {
public:
    using NodeType = T;
    using NameType = V;

    std::vector<T*> nodes;
    V name;
    bool isDirected;

    NamedEdge(std::initializer_list<T*> nodes, V name, bool isDirected) {
        if (nodes.size() < 2) throw std::invalid_argument("You can't create edge with less than two nodes");
        this->nodes.insert(this->nodes.begin(), nodes.begin(), nodes.end());
        this->name = name;
        this->isDirected = isDirected;
    }

    virtual bool equals(const NamedEdge<T, V>& b) const {
        if (isDirected != b.isDirected || nodes.size() != b.nodes.size() || name != b.name) return false;

        for (int i = 0; i < nodes.size(); i++)
            if (!(*this->nodes[i]).equals(*b.nodes[i])) return false;
        
        return true;
    }
};

template <typename T>
class Edge : public NamedEdge<T, std::string> {
public:
    Edge(std::initializer_list<T*> nodes, bool isDirected) : NamedEdge<T, std::string>(nodes, "", isDirected) {};
};

#endif
        \end{minted} 

        \textit{graph.tpp}
        \begin{minted}{C++}
#ifndef GRAPH
#define GRAPH

// Abandon hope, all ye who enter here

#include "node.tpp"
#include "edge.tpp"
#include "routes.tpp"

#include <set>

// Абстрактный класс граф, содержит виртуальные методы
template <typename E, typename N = typename E::NodeType>
class Graph {

public:
    using NodeValueType = typename N::NodeValueType;

    virtual void addNode(N& node) = 0;
    virtual void addEdge(E edge) = 0;
    virtual N* operator[](const NodeValueType value) = 0;
    virtual void deleteEdge(E edge) = 0;
    virtual void deleteNode(const N& node) = 0;

    virtual bool isRoute(std::vector<N*> route) = 0;
    virtual bool isChain(std::vector<N*> chain) = 0;
    virtual bool isSimpleChain(std::vector<N*> simpleChain) = 0;
    virtual bool isCycle(std::vector<N*> cycle) = 0;
    virtual bool isSimpleCycle(std::vector<N*> simpleCycle) = 0;

    virtual const std::vector<N*> getAdjacentNodes(N* node) = 0;
    virtual std::vector<PointRoute<N*>> getAllRoutes(N* start, int steps) = 0;

    virtual int countAllRoutesBetweenAllNodes(int steps) = 0;

    virtual std::vector<PointRoute<N*>> getAllRoutesBetweenTwoNodes(N* start, N* end, int steps) = 0;

    virtual std::vector<PointRoute<N*>> getAllSimpleMaximumChains(N* start) = 0;
};

template <typename E, typename N = typename E::NodeType>
class AdjacencyMatrixGraph : public Graph<E, N> {
public:
    class EdgeContainer {
        public:
        E current;
        EdgeContainer* linked = nullptr;

        EdgeContainer(E curr) : current(curr) {};

        EdgeContainer(E current, EdgeContainer* linked) {
            this->current = current;
            this->linked = linked;
        }

        bool operator==(const EdgeContainer& b) {
            return current.equals(b.current) && this->linked == b.linked;
        }
    };

    using NodeValueType = typename N::NodeValueType;

    std::vector<N*> nodes;
    std::vector<std::vector<std::vector<EdgeContainer>*>> edges;

    void addNode(N& node) override {
        for (auto &pNode : nodes) 
            if (pNode->equals(node)) 
                throw std::invalid_argument("Node is already present");

        nodes.push_back(node.clone());

        for (int i = 0; i < edges.size(); i++) {
            edges[i].resize(edges.size() + 1);
            edges[i][edges[i].size() - 1] = nullptr;
        }

        edges.push_back(std::vector<std::vector<EdgeContainer>*>(edges.size() + 1, nullptr));
    }

    void addEdge(E edge) {
        reinterpretNodes(edge.nodes);

        std::vector<EdgeContainer>*& edgePos = this->at(edge.nodes.front(), edge.nodes.back());
        if (!edgePos) edgePos = new std::vector<EdgeContainer>();

        (*edgePos).push_back(EdgeContainer(edge));
        EdgeContainer* added = &(*edgePos).back();
        if (!edge.isDirected) {
            std::reverse(edge.nodes.begin(), edge.nodes.end());

            std::vector<EdgeContainer>*& revEdgePos = this->at(edge.nodes.front(), edge.nodes.back());
            if (!revEdgePos) revEdgePos = new std::vector<EdgeContainer>();

            (*revEdgePos).push_back(EdgeContainer(edge));

            EdgeContainer* nonDirectAdded = &(*revEdgePos).back();

            added->linked = nonDirectAdded;
            nonDirectAdded->linked = added;
        }
    }

    void deleteEdge(E edge) {
        deleteEdge(edge, false);
    }

    void deleteNode(const N& node) {
        int deleteIndex = this->nodes.size();
        for (int i = 0; i < this->nodes.size() && deleteIndex == this->nodes.size(); i++) {
            if ((*this->nodes[i]).getValue() == node.getValue())
                deleteIndex = i;
        }

        if (deleteIndex == this->nodes.size()) throw std::invalid_argument("Node does not belongs to graph");

        for (auto &colEdge : this->edges[deleteIndex]) {
            delete colEdge;
        }
        this->nodes.erase(this->nodes.begin() + deleteIndex);

        for (auto &rowEdge : this->edges) {
            delete rowEdge[deleteIndex];
            rowEdge.erase(rowEdge.begin() + deleteIndex);
        }
    }

    bool isRoute(std::vector<N*> route) {
        reinterpretNodes(route);

        // Если две вершины смежны, то они инцидентны одному ребру.
        // Проверкой на смежность будем сразу проверять, что вершина i
        // инцидентна какому-то ребру, и вершина i + 1 тоже инцидентна
        // тому же ребру.
        for (int i = 0; i < route.size() - 1; i++) {
            if (!isNodesAdjacent(*route[i], *route[i + 1])) return false;
        }

        return true;
    }

    bool isChain(std::vector<N*> chain) {
        if (!isRoute(chain)) return false;
        reinterpretNodes(chain);
        
        std::vector<EdgeContainer> visitedEdges;
        for (int i = 0; i < chain.size() - 1; i++) {
            auto edges = *findByBeginEndPoint((*chain[i]).getValue(), (*chain[i + 1]).getValue());

            // Отдаём приоритет направленным рёбрам
            // Можем обратиться к 0 элементу безопасно из-за проверки на маршрут выше.
            EdgeContainer searchEdge = edges[0];
            bool found = false;
            bool foundDirected = false;
            for (int j = 0; j < edges.size(); j++) {
                if (edges[j].linked == nullptr && std::find(visitedEdges.begin(), visitedEdges.end(), edges[j]) == visitedEdges.end()) {
                    foundDirected = true;
                    found = true;
                    visitedEdges.push_back(edges[j]);
                    break;
                } else if (edges[j].linked != nullptr && std::find(visitedEdges.begin(), visitedEdges.end(), edges[j]) == visitedEdges.end()) {
                    found = true;
                    searchEdge = edges[j];
                }
            }
            
            if (!found) 
                return false;
            else if (found && !foundDirected) {
                visitedEdges.push_back(searchEdge);
                visitedEdges.push_back(*searchEdge.linked);
            } 
        }

        return true;
    }

    bool isSimpleChain(std::vector<N*> simpleChain) {
        // Проверяем, что рёбра не повторяются
        if (!isRoute(simpleChain)) return false;

        for (int i = 0; i < simpleChain.size() - 1; i++) {
            for (int j = i + 1; j < simpleChain.size(); j++) {
                if ((simpleChain[i])->equals(*(simpleChain[j]))) return false;
            }
        }

        return true;
    }

    bool isCycle(std::vector<N*> cycle) {
        if (!isChain(cycle)) return false;

        return (*(cycle.front())).equals(*(cycle.back()));
    }

    bool isSimpleCycle(std::vector<N*> cycle) {
        if (!isCycle(cycle)) return false;

        for (int i = 0; i < cycle.size() - 2; i++) {
            for (int j = i + 1; j < cycle.size() - 1; j++) {
                if ((cycle[i])->equals(*(cycle[j]))) return false;
            }
        }

        return true;
    }

    // Ищет среди вершин нужную с value. Если вершина не найдена - возвращает null. Иначе - возвращает ссылку на него. 
    N* operator[](const NodeValueType value) {
        for (auto &n : this->nodes) 
            if (n->getValue() == value)
                return n;
        
        return nullptr;
    }

    // Ищет и возвращает указатель на массив, все грани в котором начинаются с точки с значением a и заканчивается b
    const std::vector<EdgeContainer>* findByBeginEndPoint(const NodeValueType begin, const NodeValueType end) {
        return at((*this)[begin], (*this)[end]);
    }

    ~AdjacencyMatrixGraph() {
        // Удаляем рёбра
        for (auto row : this->edges) {
            for (auto element : row) {
                delete element;
            }
        }

        // Удаляем вершины
        for (auto node: this->nodes) {
            delete node;
        }
    }

    const std::vector<N*> getAdjacentNodes(N* node) {
        int nodeIndex = -1;
        for (int i = 0; i < this->nodes.size(); i++) {
            if ((*(this->nodes[i])).equals(*node)) { 
                nodeIndex = i;
            
                break;
            }
        }

        if (nodeIndex == -1) throw std::invalid_argument("Node does not belongs to graph");

        std::vector<N*> result;
        for (int i = 0; i < this->nodes.size(); i++) {
            if (this->edges[nodeIndex][i] != nullptr)
                result.push_back(this->nodes[i]);
        }

        return result;
    }

    std::vector<PointRoute<N*>> getAllRoutes(N* start, int steps) {
        std::vector<PointRoute<N*>> result;
        if (steps <= 1) {
            result.push_back(PointRoute<N*>({start}));
            return result;
        }

        for (auto &adjNode : getAdjacentNodes(start)) {
            auto routes = getAllRoutes(adjNode, steps - 1);
            for (auto &route : routes) {
                route.route.insert(route.route.begin(), start);

                result.push_back(route);
            }
        }

        return result;
    }

    int countAllRoutesBetweenAllNodes(int steps) {
        int count = 0;
        for (int i = 0; i < nodes.size(); i++)
            for (int j = 0; j < nodes.size(); j++) 
                count += countAllRoutesBetweenTwoNodes(i, j, steps);


        return count;
    }

    std::vector<PointRoute<N*>> getAllRoutesBetweenTwoNodes(N* start, N* end, int steps) {
        std::vector<PointRoute<N*>> result;
        if (steps < 1) {
            if (start->equals(*end))
                result.push_back(PointRoute<N*>({start}));
            
            return result;
        }

        for (auto &adjNode : getAdjacentNodes(start)) {
            auto routes = getAllRoutesBetweenTwoNodes(adjNode, end, steps - 1);
            if (routes.size() == 0) continue;
            
            for (auto &route : routes) {
                route.route.insert(route.route.begin(), start);

                result.push_back(route);
            }
        }

        return result;
    }

    std::vector<PointRoute<N*>> getAllSimpleMaximumChains(N* start) {
        return getAllSimpleMaximumChains(start, {});
    }

private:
    int countAllRoutesBetweenTwoNodes(int start, int end, int steps) {
        if (steps <= 1) {
            return edges[start][end] != nullptr;
        }

        int count = 0;
        for (int i = 0; i < edges.size(); i++) {
            count += countAllRoutesBetweenTwoNodes(start, i, steps - 1) * (edges[i][end] != nullptr);
        }

        return count;
    }

    std::vector<PointRoute<N*>> getAllSimpleMaximumChains(N* start, std::vector<N*> takenNodes) {
        std::vector<PointRoute<N*>> result;
        takenNodes.push_back(start);

        bool anyElementFound = false;
        for (auto &adjNode : getAdjacentNodes(start)) {
            if (std::find(takenNodes.begin(), takenNodes.end(), adjNode) != takenNodes.end()) continue;
 
            anyElementFound = true;
            auto adjacentNodesForCurrentNode = getAdjacentNodes(adjNode);
            
            auto routes = getAllSimpleMaximumChains(adjNode, takenNodes);
            for (auto &route : routes) {
                route.route.insert(route.route.begin(), start);

                result.push_back(route);
            }
        }

        if (!anyElementFound) {
            result.push_back(PointRoute<N*>({start}));
        }

        return result;
    }

    void deleteEdge(E edge, bool nonDirectionalDelete) {
        reinterpretNodes(edge.nodes);

        std::vector<EdgeContainer>*& edgePos = this->at(edge.nodes.front(), edge.nodes.back());
        if (!edgePos) throw std::invalid_argument("Edge doesn't exist in graph");
        
        
        for (int i = 0; i < (*edgePos).size(); i++) {
            EdgeContainer& currentEdge = (*edgePos)[i];

            if (!currentEdge.current.equals(edge)) continue;

            if (!edge.isDirected && !nonDirectionalDelete) {
                deleteEdge(currentEdge.linked->current, true);
            }

            (*edgePos).erase((*edgePos).begin() + i);
            if ((*edgePos).size() == 0) {
                delete edgePos;
                edgePos = nullptr;
            }

            return;
        }

        throw std::invalid_argument("Edge doesn't exist in graph");
    }

    bool isNodesAdjacent(const N& a, const N& b) {
        try {
            return findByBeginEndPoint(a.getValue(), b.getValue()) != nullptr;
        } catch (std::invalid_argument& ignore) {}

        return false;
    }

    // Обновляет вершины в edge, делая так, чтобы они принадлежали graph. Выбрасывает ошибку, если вершины не существует.
    void reinterpretNodes(std::vector<N*>& nodes) {
        for (int i = 0; i < nodes.size(); i++) {
            auto pNode = this->operator[](nodes[i]->getValue());

            if (pNode == nullptr)
                throw std::invalid_argument("Node does not belongs to graph");

            nodes[i] = pNode;
        }
    }

    std::vector<EdgeContainer>*& at(N* from, N* to) {
        if (from == nullptr || to == nullptr) throw std::invalid_argument("Node does not belongs to graph");

        int fromIndex = -1;
        int toIndex = -1;
        for (int i = 0; i < nodes.size() && (fromIndex == -1 || toIndex == -1); i++) {
            if (nodes[i] == from) fromIndex = i;
            if (nodes[i] == to)   toIndex = i;
        }

        if (fromIndex == -1 || toIndex == -1) {
            throw std::invalid_argument("Node is not present");
        }

        return edges[fromIndex][toIndex];
    }
};

#endif
        \end{minted} 

        В данном задании были использованы методы 
        isRoute, isChain, isSimpleChain, isCycle, isSimpleCycle.

        \textit{util.h}
        \begin{minted}{C++}
#pragma once

#include "../libs/alg/alg.h"

typedef Node<int> IntNode;

Graph<Edge<IntNode>>* constructGraph1() {
    auto G1 = new AdjacencyMatrixGraph<Edge<IntNode>>();

    IntNode N1(1);
    IntNode N2(2);
    IntNode N3(3);
    IntNode N4(4);
    IntNode N5(5);
    IntNode N6(6);
    IntNode N7(7);

    G1->addNode(N1);
    G1->addNode(N2);
    G1->addNode(N3);
    G1->addNode(N4);
    G1->addNode(N5);
    G1->addNode(N6);
    G1->addNode(N7);

    G1->addEdge({{&N1, &N2}, false});
    G1->addEdge({{&N1, &N4}, false});
    G1->addEdge({{&N1, &N5}, false});
    G1->addEdge({{&N1, &N6}, false});
    G1->addEdge({{&N1, &N7}, false});
    G1->addEdge({{&N2, &N3}, false});
    G1->addEdge({{&N2, &N5}, false});
    G1->addEdge({{&N2, &N6}, false});
    G1->addEdge({{&N3, &N4}, false});
    G1->addEdge({{&N6, &N7}, false});

    return G1;
}

Graph<Edge<IntNode>>* constructGraph2() {
    auto G2 = new AdjacencyMatrixGraph<Edge<IntNode>>();

    IntNode N1(1);
    IntNode N2(2);
    IntNode N3(3);
    IntNode N4(4);
    IntNode N5(5);
    IntNode N6(6);
    IntNode N7(7);

    G2->addNode(N1);
    G2->addNode(N2);
    G2->addNode(N3);
    G2->addNode(N4);
    G2->addNode(N5);
    G2->addNode(N6);
    G2->addNode(N7);

    G2->addEdge({{&N7, &N6}, false});
    G2->addEdge({{&N1, &N6}, false});
    G2->addEdge({{&N4, &N1}, false});
    G2->addEdge({{&N4, &N2}, false});
    G2->addEdge({{&N4, &N3}, false});
    G2->addEdge({{&N3, &N6}, false});
    G2->addEdge({{&N5, &N6}, false});
    G2->addEdge({{&N7, &N1}, false});
    G2->addEdge({{&N2, &N1}, false});
    G2->addEdge({{&N2, &N3}, false});
    G2->addEdge({{&N5, &N3}, false});

    return G2;
}
        \end{minted}

        \begin{minted}{C++}
#include "../util.h"

void testRoute(Graph<Edge<IntNode>>* G, std::string graphName, std::vector<int>& route) {
    std::cout << "Testing route {";
    for (int i = 0; i < route.size(); i++) {
        if (i == route.size() - 1) {
            std::cout << route[i];
        } else {
            std::cout << route[i] << ", ";
        }
    }
    std::cout << "} for graph " << graphName << ":\n\n";
    std::vector<IntNode*> nodes;
    for (auto &element : route) {
        nodes.push_back((*G)[element]);
    }

    std::cout << (G->isRoute(nodes) ? "Is route\n" : "Is not route\n");
    std::cout << (G->isChain(nodes) ? "Is chain\n" : "Is not chain\n");
    std::cout << (G->isSimpleChain(nodes) ? "Is simple chain\n" : "Is not simple chain\n");
    std::cout << (G->isCycle(nodes) ? "Is cycle\n" : "Is not cycle\n");
    std::cout << (G->isSimpleCycle(nodes) ? "Is simple cycle\n" : "Is not simple cycle\n\n");
}

int main() {
    Graph<Edge<IntNode>> *G1 = constructGraph1();
    Graph<Edge<IntNode>> *G2 = constructGraph2();

    std::vector<int> route1 {2, 3, 4, 1, 7, 6};
    std::vector<int> route2 {4, 1, 6, 7, 1, 2};
    std::vector<int> route3 {1, 4, 3, 2, 1};
    std::vector<int> route4 {1, 2, 4, 3, 2, 1};
    std::vector<int> route5 {2, 4, 1, 7, 6, 1, 2};

    testRoute(G1, "G1", route1);
    testRoute(G1, "G1", route2);
    testRoute(G1, "G1", route3);
    testRoute(G1, "G1", route4);
    testRoute(G1, "G1", route5);

    testRoute(G2, "G2", route1);
    testRoute(G2, "G2", route2);
    testRoute(G2, "G2", route3);
    testRoute(G2, "G2", route4);
    testRoute(G2, "G2", route5);

    return 0;
}
        \end{minted}

        Результат работы программы:
        \begin{minted}{console}
Testing route {2, 3, 4, 1, 7, 6} for graph G1:

Is route
Is chain
Is simple chain
Is not cycle
Is not simple cycle

Testing route {4, 1, 6, 7, 1, 2} for graph G1:

Is route
Is chain
Is not simple chain
Is not cycle
Is not simple cycle

Testing route {1, 4, 3, 2, 1} for graph G1:

Is route
Is chain
Is not simple chain
Is cycle
Is simple cycle
Testing route {1, 2, 4, 3, 2, 1} for graph G1:

Is not route
Is not chain
Is not simple chain
Is not cycle
Is not simple cycle

Testing route {2, 4, 1, 7, 6, 1, 2} for graph G1:

Is not route
Is not chain
Is not simple chain
Is not cycle
Is not simple cycle

Testing route {2, 3, 4, 1, 7, 6} for graph G2:

Is route
Is chain
Is simple chain
Is not cycle
Is not simple cycle

Testing route {4, 1, 6, 7, 1, 2} for graph G2:

Is route
Is chain
Is not simple chain
Is not cycle
Is not simple cycle

Testing route {1, 4, 3, 2, 1} for graph G2:

Is route
Is chain
Is not simple chain
Is cycle
Is simple cycle
Testing route {1, 2, 4, 3, 2, 1} for graph G2:

Is route
Is not chain
Is not simple chain
Is not cycle
Is not simple cycle

Testing route {2, 4, 1, 7, 6, 1, 2} for graph G2:

Is route
Is chain
Is not simple chain
Is cycle
Is not simple cycle
        \end{minted}
        Результат выполнения программы совпали с ручными вычислениями.

        % 4 задание

        \item Написать программу, получающую все маршруты заданной 
        длины, выходящие из заданной вершины. Использовать программу для
        
        получения всех маршрутов заданной длины в графах $G_1$ и $G_2$.\\
        В данном задании использовался метод getAllRoutes.\\
        \textit{main.cpp}
        \begin{minted}{C++}
#include "../util.h"

int main() {
    Graph<Edge<IntNode>> *G1 = constructGraph1();
    Graph<Edge<IntNode>> *G2 = constructGraph2();

    std::cout << "All routes that start in 5 in G1 that has 3 edges: \n";
    auto r = G1->getAllRoutes((*G1)[5], 3);

    for (auto &route : r) {
        for (int i = 0; i < route.route.size(); i++) {
            if (i == route.route.size() - 1) {
                std::cout << route.route[i]->name;
            } else {
                std::cout << route.route[i]->name << ", ";
            }
        }

        std::cout << std::endl;
    }

    std::cout << "All routes that start in 2 in G2 that has 4 edges: \n";
    r = G2->getAllRoutes((*G1)[2], 4);

    for (auto &route : r) {
        for (int i = 0; i < route.route.size(); i++) {
            if (i == route.route.size() - 1) {
                std::cout << route.route[i]->name;
            } else {
                std::cout << route.route[i]->name << ", ";
            }
        }

        std::cout << std::endl;
    }

    return 0;
}
        \end{minted}
        Результат выполнения программы:
        \begin{minted}{console}
All routes that start in 5 in G1 that has 3 edges: 
5, 1, 2
5, 1, 4
5, 1, 5
5, 1, 6
5, 1, 7
5, 2, 1
5, 2, 3
5, 2, 5
5, 2, 6
All routes that start in 2 in G2 that has 4 edges: 
2, 1, 2, 1
2, 1, 2, 3
2, 1, 2, 4
2, 1, 4, 1
2, 1, 4, 2
2, 1, 4, 3
2, 1, 6, 1
2, 1, 6, 3
2, 1, 6, 5
2, 1, 6, 7
2, 1, 7, 1
2, 1, 7, 6
2, 3, 2, 1
2, 3, 2, 3
2, 3, 2, 4
2, 3, 4, 1
2, 3, 4, 2
2, 3, 4, 3
2, 3, 5, 3
2, 3, 5, 6
2, 3, 6, 1
2, 3, 6, 3
2, 3, 6, 5
2, 3, 6, 7
2, 4, 1, 2
2, 4, 1, 4
2, 4, 1, 6
2, 4, 1, 7
2, 4, 2, 1
2, 4, 2, 3
2, 4, 2, 4
2, 4, 3, 2
2, 4, 3, 4
2, 4, 3, 5
2, 4, 3, 6
        \end{minted}

        % 5 задание
        \item Написать программу, определяющую количество маршрутов
заданной длины между каждой парой вершин графа. Использовать
программу для определения количества маршрутов заданной длины

между каждой парой вершин в графах $G_1$ и $G_2$.\\
        В данном задании использовался метод countAllRoutesBetweenAllNodes.\\
        \textit{main.cpp}
        \begin{minted}{C++}
#include "../util.h"

int main() {
    Graph<Edge<IntNode>> *G1 = constructGraph1();
    Graph<Edge<IntNode>> *G2 = constructGraph2();

    std::cout << "All routes between pairs in 1 step in G1: " << G1->countAllRoutesBetweenAllNodes(1) << std::endl;
    std::cout << "All routes between pairs in 3 steps in G1: " << G1->countAllRoutesBetweenAllNodes(3) << std::endl;
    std::cout << "All routes between pairs in 2 steps in G2: " << G2->countAllRoutesBetweenAllNodes(2) << std::endl;

    return 0;
}
        \end{minted}
        Результат выполнения программы:
        \begin{minted}{console}
All routes between pairs in 1 step in G1: 20
All routes between pairs in 3 steps in G1: 206
All routes between pairs in 2 steps in G2: 74
        \end{minted}

        % 6 задание

        \item Написать программу, определяющую все маршруты заданной
        длины между заданной парой вершин графа. Использовать программу
        для определения всех маршрутов заданной длины между заданной 
        парой вершин в графах $G_1$ и $G_2$.\\
        В данном задании использовался метод getAllRoutesBetweenTwoNodes.\\
        \textit{main.cpp}
        \begin{minted}{C++}
#include "../util.h"

int main() {
    Graph<Edge<IntNode>> *G1 = constructGraph1();
    Graph<Edge<IntNode>> *G2 = constructGraph2();

    std::cout << "All routes between adjacent elements 1 and 7 in 1 step in G1: \n";
    
    auto r = G1->getAllRoutesBetweenTwoNodes((*G1)[1], (*G1)[7], 1);
    for (auto &route : r) {
        for (int i = 0; i < route.route.size(); i++) {
            if (i == route.route.size() - 1) {
                std::cout << route.route[i]->name;
            } else {
                std::cout << route.route[i]->name << ", ";
            }
        }

        std::cout << std::endl;
    }

    std::cout << "All routes between non-adjacent elements 3 and 4 in 4 steps in G1: \n";
    r = G1->getAllRoutesBetweenTwoNodes((*G1)[3], (*G1)[7], 4);
    for (auto &route : r) {
        for (int i = 0; i < route.route.size(); i++) {
            if (i == route.route.size() - 1) {
                std::cout << route.route[i]->name;
            } else {
                std::cout << route.route[i]->name << ", ";
            }
        }

        std::cout << std::endl;
    }

    std::cout << "All routes between non-adjacent elements 7 and 5 in 4 steps in G2: \n";
    r = G2->getAllRoutesBetweenTwoNodes((*G2)[7], (*G2)[5], 4);
    for (auto &route : r) {
        for (int i = 0; i < route.route.size(); i++) {
            if (i == route.route.size() - 1) {
                std::cout << route.route[i]->name;
            } else {
                std::cout << route.route[i]->name << ", ";
            }
        }

        std::cout << std::endl;
    }

    return 0;
}
        \end{minted}
        Результат выполнения программы:
        \begin{minted}{console}
All routes between adjacent elements 1 and 7 in 1 step in G1: 
1, 7
All routes between non-adjacent elements 3 and 4 in 4 steps in G1: 
3, 2, 1, 6, 7
3, 2, 5, 1, 7
3, 2, 6, 1, 7
3, 4, 1, 6, 7
All routes between non-adjacent elements 7 and 5 in 4 steps in G2: 
7, 1, 2, 3, 5
7, 1, 4, 3, 5
7, 1, 6, 3, 5
7, 1, 7, 6, 5
7, 6, 1, 6, 5
7, 6, 3, 6, 5
7, 6, 5, 3, 5
7, 6, 5, 6, 5
7, 6, 7, 6, 5
        \end{minted}

% 7 задание

        \item Написать программу, получающую все простые максимальные

        цепи, выходящие из заданной вершины графа. Использовать 
        программу для получения всех простые максимальных цепей, выходящих из
        
        заданной вершины в графах $G_1$ и $G_2$.\\
        В данном задании использовался метод getAllSimpleMaximumChains.\\
        \textit{main.cpp}
        \begin{minted}{C++}
#include "../util.h"

int main() {
    Graph<Edge<IntNode>> *G1 = constructGraph1();
    Graph<Edge<IntNode>> *G2 = constructGraph2();


    std::cout << "Simple maximum chains, starting in 4 in G1:\n"; 
    auto r = G1->getAllSimpleMaximumChains((*G1)[4]);
    for (auto &route : r) {
        for (int i = 0; i < route.route.size(); i++) {
            if (i == route.route.size() - 1) {
                std::cout << route.route[i]->name;
            } else {
                std::cout << route.route[i]->name << ", ";
            }
        }

        std::cout << std::endl;
    }

    std::cout << "Simple maximum chains, starting in 2 in G2:\n";
    r = G1->getAllSimpleMaximumChains((*G1)[2]);
    for (auto &route : r) {
        for (int i = 0; i < route.route.size(); i++) {
            if (i == route.route.size() - 1) {
                std::cout << route.route[i]->name;
            } else {
                std::cout << route.route[i]->name << ", ";
            }
        }

        std::cout << std::endl;
    }

    return 0;
}
        \end{minted}
        Результат выполнения программы:
        \begin{minted}{console}
Simple maximum chains, starting in 4 in G1:
4, 1, 2, 3
4, 1, 2, 5
4, 1, 2, 6, 7
4, 1, 5, 2, 3
4, 1, 5, 2, 6, 7
4, 1, 6, 2, 3
4, 1, 6, 2, 5
4, 1, 6, 7
4, 1, 7, 6, 2, 3
4, 1, 7, 6, 2, 5
4, 3, 2, 1, 5
4, 3, 2, 1, 6, 7
4, 3, 2, 1, 7, 6
4, 3, 2, 5, 1, 6, 7
4, 3, 2, 5, 1, 7, 6
4, 3, 2, 6, 1, 5
4, 3, 2, 6, 1, 7
4, 3, 2, 6, 7, 1, 5
Simple maximum chains, starting in 2 in G2:
2, 1, 4, 3
2, 1, 5
2, 1, 6, 7
2, 1, 7, 6
2, 3, 4, 1, 5
2, 3, 4, 1, 6, 7
2, 3, 4, 1, 7, 6
2, 5, 1, 4, 3
2, 5, 1, 6, 7
2, 5, 1, 7, 6
2, 6, 1, 4, 3
2, 6, 1, 5
2, 6, 1, 7
2, 6, 7, 1, 4, 3
2, 6, 7, 1, 5
        \end{minted}


        \end{enumerate}

\textbf{Вывод: } в ходе лабораторной работы изучили основные понятия теории графов, способы
задания графов, научиться программно реализовывать
алгоритмы получения и анализа маршрутов в графах.

\end{document}