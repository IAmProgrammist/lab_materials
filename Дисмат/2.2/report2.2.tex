\documentclass[a4paper,14pt]{extarticle}


\usepackage[english,russian]{babel}
\usepackage[T2A]{fontenc}
\usepackage[utf8]{inputenc}
\usepackage{ragged2e}
\usepackage[utf8]{inputenc}
\usepackage{hyperref}
\usepackage{minted}
\usepackage{xcolor}
\definecolor{LightGray}{gray}{0.9}
\usepackage{graphicx}
\usepackage[export]{adjustbox}
\usepackage[left=1cm,right=1cm, top=1cm,bottom=1cm,bindingoffset=0cm]{geometry}
\usepackage{fontspec}
\usepackage{ upgreek }
\usepackage[shortlabels]{enumitem}
\usepackage[mathletters]{ucs}
\usepackage{adjustbox}
\usepackage{multirow}
\usepackage{amsmath}
\usepackage{amssymb}
\usepackage{pifont}
\graphicspath{ {./images/} }
\makeatletter
\AddEnumerateCounter{\asbuk}{\russian@alph}{щ}
\makeatother
\setmonofont{Consolas}
\setmainfont{Times New Roman}

\newcommand\textbox[1]{
	\parbox{.45\textwidth}{#1}
}

\newcommand{\specialcell}[2][c]{%
	\begin{tabular}[#1]{@{}c@{}}#2\end{tabular}}

\begin{document}	
	\pagenumbering{gobble}
	\begin{center}
		\small{
			МИНИCТЕРCТВО НАУКИ И ВЫCШЕГО ОБРАЗОВАНИЯ \\РОCCИЙCКОЙ ФЕДЕРАЦИИ
			\bigbreak
			ФЕДЕРАЛЬНОЕ ГОCУДАРCТВЕННОЕ БЮДЖЕТНОЕ ОБРАЗОВАТЕЛЬНОЕ УЧРЕЖДЕНИЕ ВЫCШЕГО ОБРАЗОВАНИЯ \\
			\bigbreak
			\textbf{«БЕЛГОРОДCКИЙ ГОCУДАРCТВЕННЫЙ \\ТЕХНОЛОГИЧЕCКИЙ УНИВЕРCИТЕТ им. В. Г. ШУХОВА»\\ (БГТУ им. В.Г. Шухова)} \\
			\bigbreak
			Кафедра программного обеспечения вычислительной техники и автоматизированных систем\\}
	\end{center}
	
	\vfill
	\begin{center}
		\large{
			\textbf{
				Лабораторная работа №2.2}}\\
		\normalsize{
			по дисциплине: Дискретная математика \\
			тема: «Задачи выбора»}
	\end{center}
	\vfill
	\hfill\textbox{
		Выполнил: ст. группы ПВ-223\\Пахомов Владислав Андреевич
		\bigbreak
		Проверили: \\ст. пр. Рязанов Юрий Дмитриевич\\
		ст. пр. Бондаренко Татьяна Владимировна
	}
	\vfill\begin{center}
		Белгород 2023 г.
	\end{center}
	\newpage
	\begin{center}
		\textbf{Лабораторная работа №2.2}\\
		Задачи выбора\\
		Вариант 10
	\end{center}
	\textbf{Цель работы: }приобрести практические навыки в использовании алгоритмов порождения комбинаторных объектов при проектировании алгоритмов решения задач выбора.
	\begin{enumerate}[№1. ]
		\item Ознакомиться с задачей.\bigbreak
		Задана точка в узле целочисленной решетке размером $n \times n$. Расположить всеми возможными способами k точек в узлах решетки так, чтобы расстояние от каждой до заданной было одинаковым.
		\item Определить класс комбинаторных объектов, содержащих решение задачи (траекторию задачи).\\
		Сформируем массив points (размер - $n \times n$), который содержит все точки решётки. По условию необходимо взять только k точек. Множество точек фиксированно и их порядок важен (они должны быть упорядочены и не должны повторяться), траектории задачи получим в результате формирования \textbf{сочетаний} из множества points по k.\\
		\item Определить, что в задаче является функционалом и способ его вычисления.\bigbreak
		Функционал - массив расстояний от точки в траектории до точки A.\\
		Рассмотрим пример. Пусть $k=3; A=(3, 4); path=\{(1, 2), (6, 7), (3, 3)\}$\\
		Создадим множество distances.\\
		$distances_i=DISTANCE(A, path_i)$, где DISTANCE - функция вычисления квадрата расстояния между двумя точками. \\
		$distances_0=(3 - 1)^2 + (4 - 2)^2 = 8$\\
		$distances_1=(3 - 6)^2 + (4 - 7)^2 = 18$\\
		$distances_2=(3 - 3)^2 + (4 - 3)^2 = 1$\\
		$distances = \{8, 18, 1\}$ - получили функционал.
		\item Определить способ распознавания решения по значению функционала.\bigbreak
		Если все элементы функционала $distances$ равны между собой, значит решение найдено.
		\item Реализовать алгоритм решения задачи.\bigbreak
		\begin{minted}[frame=lines, framesep=2mm, baselinestretch=1.2, bgcolor=LightGray, breaklines,fontsize=\scriptsize]{C++}
#include "../alg.h"

#define DISTANCE(a, b) \
(pow2(max(b.x, a.x) - min(b.x, a.x)) +\
pow2(max(b.y, a.y) - min(b.y, a.y)))

typedef std::vector<Point> Path;
typedef std::vector<unsigned long long> Functional;

static unsigned long long max(unsigned long long a, unsigned long long b) {
	return a > b ? a : b;
}

static unsigned long long min(unsigned long long a, unsigned long long b) {
	return a < b ? a : b;
}

static unsigned long long pow2(unsigned long long a) {
	return a * a;
}

// Генерация траекторий задачи
static std::vector<Path> getPaths(size_t n, size_t k) {
	// Массив точек решётки
	std::vector<Point> points;
	
	// Заполняем его всеми точками решётки.
	for (long long x = 0; x < n; x++)
		for (long long y = 0; y < n; y++)
			points.push_back({x, y});
	
	// Такие сочетания получить невозможно - возвращаем пустой массив траекторий
	if (points.size() < k || k == 0) return {};
	
	return getCombinations(points, k);
}

// Генерация функционалов задачи
static std::vector<std::pair<Functional, Path>> getFunctionals(std::vector<Path> paths, Point a) {
	std::vector<std::pair<Functional, Path>> answer;
	
	// Каждой траектории задачи поставим в соответствие функционал
	// Функционал - массив расстояний от точек в траектории до исходной точки a
	for (auto &path: paths) {
		Functional distances;
		
		// Определяем расстояние от каждой точки в траектории до точки a
		for (auto &point : path)
			distances.push_back(DISTANCE(a, point));
		
		// Сохраняем полученный функционал
		answer.push_back({distances, path});
	}
	
	return answer;
}

// Определяем траектории
size_t getDecision(size_t n, Point a, size_t k) {
	// Генерация траекторий
	auto paths = getPaths(n, k);
	
	// Генерация функционала
	auto functionals = getFunctionals(paths, a);
	
	size_t decisions = 0;
	// Если все расстояния в функционале одинаковы, решение найдено
	for (auto &pair : functionals) {
		auto distances = pair.first;
		bool areEqual = true;
		
		for (auto &dist : distances)
			areEqual &= dist == distances[0];
		
		decisions += areEqual;
	}
	
	return decisions;
}
		\end{minted}
	\item Подготовить тестовые данные и решить задачу.
			\begin{minted}[frame=lines, framesep=2mm, baselinestretch=1.2, bgcolor=LightGray, breaklines,fontsize=\scriptsize]{C++}
#include "../../libs/alg/alg.h"

int main() {
	assert(getDecision(1, (Point){0, 0}, 1) == 1);
	assert(getDecision(1, (Point){0, 0}, 12) == 0);
	assert(getDecision(5, (Point){3, 2}, 1) == 25);
	assert(getDecision(5, (Point){3, 3}, 2) == 27);
	assert(getDecision(5, (Point){3, 3}, 4) == 4);
	assert(getDecision(0, (Point){0, 0}, 0) == 0);
}
	\end{minted}
\begin{center}
	\includegraphics[width=120mm]{/task6}
\end{center}
\end{enumerate}
			\textbf{Вывод: } в ходе лабораторной работы приобрели практические навыки в использовании алгоритмов порождения комбинаторных объектов при проектировании алгоритмов решения задач выбора.
			
\end{document}