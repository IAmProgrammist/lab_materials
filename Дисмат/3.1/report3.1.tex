\documentclass[a4paper,14pt]{extarticle}


\usepackage[english,russian]{babel}
\usepackage[T2A]{fontenc}
\usepackage[utf8]{inputenc}
\usepackage{ragged2e}
\usepackage[utf8]{inputenc}
\usepackage{hyperref}
\usepackage{minted}
\setmintedinline{frame=lines, framesep=2mm, baselinestretch=1.5, bgcolor=LightGray, breaklines,fontsize=\scriptsize}
\setminted{frame=lines, framesep=2mm, baselinestretch=1.5, bgcolor=LightGray, breaklines,fontsize=\scriptsize}
\usepackage{xcolor}
\definecolor{LightGray}{gray}{0.9}
\usepackage{graphicx}
\usepackage[export]{adjustbox}
\usepackage[left=1cm,right=1cm, top=1cm,bottom=1cm,bindingoffset=0cm]{geometry}
\usepackage{fontspec}
\usepackage{ upgreek }
\usepackage[shortlabels]{enumitem}
\usepackage{adjustbox}
\usepackage{multirow}
\usepackage{amsmath}
\usepackage{amssymb}
\usepackage{pifont}
\usepackage{pgfplots}
\graphicspath{ {./images/} }
\makeatletter
\AddEnumerateCounter{\asbuk}{\russian@alph}{щ}
\makeatother
\setmonofont{Consolas}
\setmainfont{Times New Roman}

\newcommand\textbox[1]{
	\parbox{.45\textwidth}{#1}
} 

\newcommand{\specialcell}[2][c]{%
	\begin{tabular}[#1]{@{}c@{}}#2\end{tabular}}

\begin{document}
\pagenumbering{gobble}
\begin{center}
	\small{
		МИНИCТЕРCТВО НАУКИ И ВЫCШЕГО ОБРАЗОВАНИЯ \\РОCCИЙCКОЙ ФЕДЕРАЦИИ
		\bigbreak
		ФЕДЕРАЛЬНОЕ ГОCУДАРCТВЕННОЕ БЮДЖЕТНОЕ ОБРАЗОВАТЕЛЬНОЕ УЧРЕЖДЕНИЕ ВЫCШЕГО ОБРАЗОВАНИЯ \\
		\bigbreak
		\textbf{«БЕЛГОРОДCКИЙ ГОCУДАРCТВЕННЫЙ \\ТЕХНОЛОГИЧЕCКИЙ УНИВЕРCИТЕТ им. В. Г. ШУХОВА»\\ (БГТУ им. В.Г. Шухова)} \\
		\bigbreak
		Кафедра программного обеспечения вычислительной техники и автоматизированных систем\\}
\end{center}

\vfill
\begin{center}
	\large{
		\textbf{
			Лабораторная работа №3.1}}\\
	\normalsize{
		по дисциплине: Дискретная математика \\
		тема: «Отношения и их свойства»}
\end{center}
\vfill
\hfill\textbox{
	Выполнил: ст. группы ПВ-223\\Пахомов Владислав Андреевич
	\bigbreak
	Проверили: \\ст. пр. Рязанов Юрий Дмитриевич\\
	ст. пр. Бондаренко Татьяна Владимировна
}
\vfill\begin{center}
	Белгород 2023 г.
\end{center}
\newpage
\begin{center}
	\textbf{Лабораторная работа №3.1}\\
	Отношения и их свойства\\
	Вариант 10
\end{center}
\textbf{Цель работы: }изучить способы задания отношений, операции над отношениями и свойства отношений, научиться программно реализовывать операции и определять свойства отношений.
\begin{center} \textbf{Часть 1. Операции над отношениями} \end{center}

\begin{enumerate}[label=1.\arabic*.]
	\item Представить отношения (см. ”Варианты заданий”, п.а) графиком, графом и матрицей.\\
	      $A=\{(x, y) | x \in N \textit{ и } y \in N \textit{ и } x < 11 \textit{ и } y < 11 \textit{ и } (x < y < (9 - x) \textit{ или } (9 - x) < y < x) \}$\\
	      \begin{center}
		      \begin{tikzpicture}
			      \begin{axis}[
					      xtick={0,1,2,3,4,5,6,7,8,9,10,11,12},
					      ytick={0,1,2,3,4,5,6,7,8,9,10,11,12},
					      axis lines = left,
					      xlabel = \(x\),
					      ylabel = {\(y\)},
					      ymax=13,
					      xmax=13
				      ]
				      \addplot[
					      mark size=2pt,
					      only marks,
				      ]
				      table {
						      x  y
						      1  2
						      1  3
						      1  4
						      1  5
						      1  6
						      1  7
						      2  3
						      2  4
						      2  5
						      2  6
						      3  4
						      3  5
						      6  4
						      6  5
						      7  3
						      7  4
						      7  5
						      7  6
						      8  2
						      8  3
						      8  4
						      8  5
						      8  6
						      8  7
						      9  1
						      9  2
						      9  3
						      9  4
						      9  5
						      9  6
						      9  7
						      9  8
						      10  1
						      10  2
						      10  3
						      10  4
						      10  5
						      10  6
						      10  7
						      10  8
						      10  9
					      };
			      \end{axis}
		      \end{tikzpicture}\\
		      \includegraphics[width=140mm]{1.1.A}\\
		      \begin{tabular}{|c|c|c|c|c|c|c|c|c|c|c|}
			      \hline
			                        & \textbf{1} & \textbf{2} & \textbf{3} & \textbf{4} & \textbf{5} & \textbf{6} & \textbf{7} & \textbf{8} & \textbf{9} & \textbf{10} \\
			      \hline\textbf{1}  & 0          & 1          & 1          & 1          & 1          & 1          & 1          & 0          & 0          & 0           \\
			      \hline\textbf{2}  & 0          & 0          & 1          & 1          & 1          & 1          & 0          & 0          & 0          & 0           \\
			      \hline\textbf{3}  & 0          & 0          & 0          & 1          & 1          & 0          & 0          & 0          & 0          & 0           \\
			      \hline\textbf{4}  & 0          & 0          & 0          & 0          & 0          & 0          & 0          & 0          & 0          & 0           \\
			      \hline\textbf{5}  & 0          & 0          & 0          & 0          & 0          & 0          & 0          & 0          & 0          & 0           \\
			      \hline\textbf{6}  & 0          & 0          & 0          & 1          & 1          & 0          & 0          & 0          & 0          & 0           \\
			      \hline\textbf{7}  & 0          & 0          & 1          & 1          & 1          & 1          & 0          & 0          & 0          & 0           \\
			      \hline\textbf{8}  & 0          & 1          & 1          & 1          & 1          & 1          & 1          & 0          & 0          & 0           \\
			      \hline\textbf{9}  & 1          & 1          & 1          & 1          & 1          & 1          & 1          & 1          & 0          & 0           \\
			      \hline\textbf{10} & 1          & 1          & 1          & 1          & 1          & 1          & 1          & 1          & 1          & 0           \\
			      \hline
		      \end{tabular}
	      \end{center}
	      \bigbreak
	      $B=\{(x, y) | x \in N \textit{ и } y \in N \textit{ и } x < 11 \textit{ и } y < 11 \textit{ и } x \textit{ - чётно и } y \textit{ - нечётно}  \}$\\
	      \begin{center}
		      \begin{tikzpicture}
			      \begin{axis}[
					      xtick={0,1,2,3,4,5,6,7,8,9,10,11,12},
					      ytick={0,1,2,3,4,5,6,7,8,9,10,11,12},
					      axis lines = left,
					      xlabel = \(x\),
					      ylabel = {\(y\)},
					      ymax=13,
					      xmax=13
				      ]
				      \addplot[
					      mark size=2pt,
					      only marks,
				      ]
				      table {
						      x  y
						      2  1
						      2  3
						      2  5
						      2  7
						      2  9
						      4  1
						      4  3
						      4  5
						      4  7
						      4  9
						      6  1
						      6  3
						      6  5
						      6  7
						      6  9
						      8  1
						      8  3
						      8  5
						      8  7
						      8  9
						      10  1
						      10  3
						      10  5
						      10  7
						      10  9
					      };
			      \end{axis}
		      \end{tikzpicture}\\
		      \includegraphics[width=140mm]{1.1.B}\\
		      \begin{tabular}{|c|c|c|c|c|c|c|c|c|c|c|}
			      \hline
			                        & \textbf{1} & \textbf{2} & \textbf{3} & \textbf{4} & \textbf{5} & \textbf{6} & \textbf{7} & \textbf{8} & \textbf{9} & \textbf{10} \\
			      \hline\textbf{1}  & 0          & 0          & 0          & 0          & 0          & 0          & 0          & 0          & 0          & 0           \\
			      \hline\textbf{2}  & 1          & 0          & 1          & 0          & 1          & 0          & 1          & 0          & 1          & 0           \\
			      \hline\textbf{3}  & 0          & 0          & 0          & 0          & 0          & 0          & 0          & 0          & 0          & 0           \\
			      \hline\textbf{4}  & 1          & 0          & 1          & 0          & 1          & 0          & 1          & 0          & 1          & 0           \\
			      \hline\textbf{5}  & 0          & 0          & 0          & 0          & 0          & 0          & 0          & 0          & 0          & 0           \\
			      \hline\textbf{6}  & 1          & 0          & 1          & 0          & 1          & 0          & 1          & 0          & 1          & 0           \\
			      \hline\textbf{7}  & 0          & 0          & 0          & 0          & 0          & 0          & 0          & 0          & 0          & 0           \\
			      \hline\textbf{8}  & 1          & 0          & 1          & 0          & 1          & 0          & 1          & 0          & 1          & 0           \\
			      \hline\textbf{9}  & 0          & 0          & 0          & 0          & 0          & 0          & 0          & 0          & 0          & 0           \\
			      \hline\textbf{10} & 1          & 0          & 1          & 0          & 1          & 0          & 1          & 0          & 1          & 0           \\
			      \hline
		      \end{tabular}
	      \end{center}
	      \bigbreak
	      $C=\{(x, y) | x \in N \textit{ и } y \in N \textit{ и } x < 11 \textit{ и } y < 11 \textit{ и } x\cdot y \text{ кратно трём}  \}$\\
	      \begin{center}
		      \begin{tikzpicture}
			      \begin{axis}[
					      xtick={0,1,2,3,4,5,6,7,8,9,10,11,12},
					      ytick={0,1,2,3,4,5,6,7,8,9,10,11,12},
					      axis lines = left,
					      xlabel = \(x\),
					      ylabel = {\(y\)},
					      ymax=13,
					      xmax=13
				      ]
				      \addplot[
					      mark size=2pt,
					      only marks,
				      ]
				      table {
						      x  y
						      1  3
						      1  6
						      1  9
						      2  3
						      2  6
						      2  9
						      3  1
						      3  2
						      3  3
						      3  4
						      3  5
						      3  6
						      3  7
						      3  8
						      3  9
						      3  10
						      4  3
						      4  6
						      4  9
						      5  3
						      5  6
						      5  9
						      6  1
						      6  2
						      6  3
						      6  4
						      6  5
						      6  6
						      6  7
						      6  8
						      6  9
						      6  10
						      7  3
						      7  6
						      7  9
						      8  3
						      8  6
						      8  9
						      9  1
						      9  2
						      9  3
						      9  4
						      9  5
						      9  6
						      9  7
						      9  8
						      9  9
						      9  10
						      10  3
						      10  6
						      10  9
					      };
			      \end{axis}
		      \end{tikzpicture}\\
		      \includegraphics[width=140mm]{1.1.C}\\
		      \begin{tabular}{|c|c|c|c|c|c|c|c|c|c|c|}
			      \hline
			                        & \textbf{1} & \textbf{2} & \textbf{3} & \textbf{4} & \textbf{5} & \textbf{6} & \textbf{7} & \textbf{8} & \textbf{9} & \textbf{10} \\
			      \hline\textbf{1}  & 0          & 0          & 1          & 0          & 0          & 1          & 0          & 0          & 1          & 0           \\
			      \hline\textbf{2}  & 0          & 0          & 1          & 0          & 0          & 1          & 0          & 0          & 1          & 0           \\
			      \hline\textbf{3}  & 1          & 1          & 1          & 1          & 1          & 1          & 1          & 1          & 1          & 1           \\
			      \hline\textbf{4}  & 0          & 0          & 1          & 0          & 0          & 1          & 0          & 0          & 1          & 0           \\
			      \hline\textbf{5}  & 0          & 0          & 1          & 0          & 0          & 1          & 0          & 0          & 1          & 0           \\
			      \hline\textbf{6}  & 1          & 1          & 1          & 1          & 1          & 1          & 1          & 1          & 1          & 1           \\
			      \hline\textbf{7}  & 0          & 0          & 1          & 0          & 0          & 1          & 0          & 0          & 1          & 0           \\
			      \hline\textbf{8}  & 0          & 0          & 1          & 0          & 0          & 1          & 0          & 0          & 1          & 0           \\
			      \hline\textbf{9}  & 1          & 1          & 1          & 1          & 1          & 1          & 1          & 1          & 1          & 1           \\
			      \hline\textbf{10} & 0          & 0          & 1          & 0          & 0          & 1          & 0          & 0          & 1          & 0           \\
			      \hline
		      \end{tabular}
	      \end{center}
	      \bigbreak
	\item Вычислить значение выражения (см. ”Варианты заданий”, п.б) при заданных отношениях (см. ”Варианты заданий”, п.а).\\
	      $D=A\circ B^2 - \overline{C} \cup C^{-1}$\\
	      \includegraphics[width=70mm]{1.2}\\
	      \begin{enumerate}[1) ]\item
		            $A\circ B = $  \begin{tabular}{|c|c|c|c|c|c|c|c|c|c|c|}
			            \hline
			                              & \textbf{1} & \textbf{2} & \textbf{3} & \textbf{4} & \textbf{5} & \textbf{6} & \textbf{7} & \textbf{8} & \textbf{9} & \textbf{10} \\
			            \hline\textbf{1}  & 1          & 0          & 1          & 0          & 1          & 0          & 1          & 0          & 1          & 0           \\
			            \hline\textbf{2}  & 1          & 0          & 1          & 0          & 1          & 0          & 1          & 0          & 1          & 0           \\
			            \hline\textbf{3}  & 1          & 0          & 1          & 0          & 1          & 0          & 1          & 0          & 1          & 0           \\
			            \hline\textbf{4}  & 0          & 0          & 0          & 0          & 0          & 0          & 0          & 0          & 0          & 0           \\
			            \hline\textbf{5}  & 0          & 0          & 0          & 0          & 0          & 0          & 0          & 0          & 0          & 0           \\
			            \hline\textbf{6}  & 1          & 0          & 1          & 0          & 1          & 0          & 1          & 0          & 1          & 0           \\
			            \hline\textbf{7}  & 1          & 0          & 1          & 0          & 1          & 0          & 1          & 0          & 1          & 0           \\
			            \hline\textbf{8}  & 1          & 0          & 1          & 0          & 1          & 0          & 1          & 0          & 1          & 0           \\
			            \hline\textbf{9}  & 1          & 0          & 1          & 0          & 1          & 0          & 1          & 0          & 1          & 0           \\
			            \hline\textbf{10} & 1          & 0          & 1          & 0          & 1          & 0          & 1          & 0          & 1          & 0           \\
			            \hline
		            \end{tabular}
		      \item $\_1 \circ B = $  \begin{tabular}{|c|c|c|c|c|c|c|c|c|c|c|}
			            \hline
			                              & \textbf{1} & \textbf{2} & \textbf{3} & \textbf{4} & \textbf{5} & \textbf{6} & \textbf{7} & \textbf{8} & \textbf{9} & \textbf{10} \\
			            \hline\textbf{1}  & 0          & 0          & 0          & 0          & 0          & 0          & 0          & 0          & 0          & 0           \\
			            \hline\textbf{2}  & 0          & 0          & 0          & 0          & 0          & 0          & 0          & 0          & 0          & 0           \\
			            \hline\textbf{3}  & 0          & 0          & 0          & 0          & 0          & 0          & 0          & 0          & 0          & 0           \\
			            \hline\textbf{4}  & 0          & 0          & 0          & 0          & 0          & 0          & 0          & 0          & 0          & 0           \\
			            \hline\textbf{5}  & 0          & 0          & 0          & 0          & 0          & 0          & 0          & 0          & 0          & 0           \\
			            \hline\textbf{6}  & 0          & 0          & 0          & 0          & 0          & 0          & 0          & 0          & 0          & 0           \\
			            \hline\textbf{7}  & 0          & 0          & 0          & 0          & 0          & 0          & 0          & 0          & 0          & 0           \\
			            \hline\textbf{8}  & 0          & 0          & 0          & 0          & 0          & 0          & 0          & 0          & 0          & 0           \\
			            \hline\textbf{9}  & 0          & 0          & 0          & 0          & 0          & 0          & 0          & 0          & 0          & 0           \\
			            \hline\textbf{10} & 0          & 0          & 0          & 0          & 0          & 0          & 0          & 0          & 0          & 0           \\
			            \hline
		            \end{tabular}
		      \item $\overline{C} = $  \begin{tabular}{|c|c|c|c|c|c|c|c|c|c|c|}
			            \hline
			                              & \textbf{1} & \textbf{2} & \textbf{3} & \textbf{4} & \textbf{5} & \textbf{6} & \textbf{7} & \textbf{8} & \textbf{9} & \textbf{10} \\
			            \hline\textbf{1}  & 1          & 1          & 0          & 1          & 1          & 0          & 1          & 1          & 0          & 1           \\
			            \hline\textbf{2}  & 1          & 1          & 0          & 1          & 1          & 0          & 1          & 1          & 0          & 1           \\
			            \hline\textbf{3}  & 0          & 0          & 0          & 0          & 0          & 0          & 0          & 0          & 0          & 0           \\
			            \hline\textbf{4}  & 1          & 1          & 0          & 1          & 1          & 0          & 1          & 1          & 0          & 1           \\
			            \hline\textbf{5}  & 1          & 1          & 0          & 1          & 1          & 0          & 1          & 1          & 0          & 1           \\
			            \hline\textbf{6}  & 0          & 0          & 0          & 0          & 0          & 0          & 0          & 0          & 0          & 0           \\
			            \hline\textbf{7}  & 1          & 1          & 0          & 1          & 1          & 0          & 1          & 1          & 0          & 1           \\
			            \hline\textbf{8}  & 1          & 1          & 0          & 1          & 1          & 0          & 1          & 1          & 0          & 1           \\
			            \hline\textbf{9}  & 0          & 0          & 0          & 0          & 0          & 0          & 0          & 0          & 0          & 0           \\
			            \hline\textbf{10} & 1          & 1          & 0          & 1          & 1          & 0          & 1          & 1          & 0          & 1           \\
			            \hline
		            \end{tabular}
		      \item $C^{-1} = $  \begin{tabular}{|c|c|c|c|c|c|c|c|c|c|c|}
			            \hline
			                              & \textbf{1} & \textbf{2} & \textbf{3} & \textbf{4} & \textbf{5} & \textbf{6} & \textbf{7} & \textbf{8} & \textbf{9} & \textbf{10} \\
			            \hline\textbf{1}  & 0          & 0          & 1          & 0          & 0          & 1          & 0          & 0          & 1          & 0           \\
			            \hline\textbf{2}  & 0          & 0          & 1          & 0          & 0          & 1          & 0          & 0          & 1          & 0           \\
			            \hline\textbf{3}  & 1          & 1          & 1          & 1          & 1          & 1          & 1          & 1          & 1          & 1           \\
			            \hline\textbf{4}  & 0          & 0          & 1          & 0          & 0          & 1          & 0          & 0          & 1          & 0           \\
			            \hline\textbf{5}  & 0          & 0          & 1          & 0          & 0          & 1          & 0          & 0          & 1          & 0           \\
			            \hline\textbf{6}  & 1          & 1          & 1          & 1          & 1          & 1          & 1          & 1          & 1          & 1           \\
			            \hline\textbf{7}  & 0          & 0          & 1          & 0          & 0          & 1          & 0          & 0          & 1          & 0           \\
			            \hline\textbf{8}  & 0          & 0          & 1          & 0          & 0          & 1          & 0          & 0          & 1          & 0           \\
			            \hline\textbf{9}  & 1          & 1          & 1          & 1          & 1          & 1          & 1          & 1          & 1          & 1           \\
			            \hline\textbf{10} & 0          & 0          & 1          & 0          & 0          & 1          & 0          & 0          & 1          & 0           \\
			            \hline
		            \end{tabular}
		      \item $\_2 - \_3 = $  \begin{tabular}{|c|c|c|c|c|c|c|c|c|c|c|}
			            \hline
			                              & \textbf{1} & \textbf{2} & \textbf{3} & \textbf{4} & \textbf{5} & \textbf{6} & \textbf{7} & \textbf{8} & \textbf{9} & \textbf{10} \\
			            \hline\textbf{1}  & 0          & 0          & 0          & 0          & 0          & 0          & 0          & 0          & 0          & 0           \\
			            \hline\textbf{2}  & 0          & 0          & 0          & 0          & 0          & 0          & 0          & 0          & 0          & 0           \\
			            \hline\textbf{3}  & 0          & 0          & 0          & 0          & 0          & 0          & 0          & 0          & 0          & 0           \\
			            \hline\textbf{4}  & 0          & 0          & 0          & 0          & 0          & 0          & 0          & 0          & 0          & 0           \\
			            \hline\textbf{5}  & 0          & 0          & 0          & 0          & 0          & 0          & 0          & 0          & 0          & 0           \\
			            \hline\textbf{6}  & 0          & 0          & 0          & 0          & 0          & 0          & 0          & 0          & 0          & 0           \\
			            \hline\textbf{7}  & 0          & 0          & 0          & 0          & 0          & 0          & 0          & 0          & 0          & 0           \\
			            \hline\textbf{8}  & 0          & 0          & 0          & 0          & 0          & 0          & 0          & 0          & 0          & 0           \\
			            \hline\textbf{9}  & 0          & 0          & 0          & 0          & 0          & 0          & 0          & 0          & 0          & 0           \\
			            \hline\textbf{10} & 0          & 0          & 0          & 0          & 0          & 0          & 0          & 0          & 0          & 0           \\
			            \hline
		            \end{tabular}

		      \item $\_5 \cup \_4 = $  \begin{tabular}{|c|c|c|c|c|c|c|c|c|c|c|}
			            \hline
			                              & \textbf{1} & \textbf{2} & \textbf{3} & \textbf{4} & \textbf{5} & \textbf{6} & \textbf{7} & \textbf{8} & \textbf{9} & \textbf{10} \\
			            \hline\textbf{1}  & 0          & 0          & 1          & 0          & 0          & 1          & 0          & 0          & 1          & 0           \\
			            \hline\textbf{2}  & 0          & 0          & 1          & 0          & 0          & 1          & 0          & 0          & 1          & 0           \\
			            \hline\textbf{3}  & 1          & 1          & 1          & 1          & 1          & 1          & 1          & 1          & 1          & 1           \\
			            \hline\textbf{4}  & 0          & 0          & 1          & 0          & 0          & 1          & 0          & 0          & 1          & 0           \\
			            \hline\textbf{5}  & 0          & 0          & 1          & 0          & 0          & 1          & 0          & 0          & 1          & 0           \\
			            \hline\textbf{6}  & 1          & 1          & 1          & 1          & 1          & 1          & 1          & 1          & 1          & 1           \\
			            \hline\textbf{7}  & 0          & 0          & 1          & 0          & 0          & 1          & 0          & 0          & 1          & 0           \\
			            \hline\textbf{8}  & 0          & 0          & 1          & 0          & 0          & 1          & 0          & 0          & 1          & 0           \\
			            \hline\textbf{9}  & 1          & 1          & 1          & 1          & 1          & 1          & 1          & 1          & 1          & 1           \\
			            \hline\textbf{10} & 0          & 0          & 1          & 0          & 0          & 1          & 0          & 0          & 1          & 0           \\
			            \hline
		            \end{tabular}\bigbreak
		            $D = $ \begin{tabular}{|c|c|c|c|c|c|c|c|c|c|c|}
			            \hline
			                              & \textbf{1} & \textbf{2} & \textbf{3} & \textbf{4} & \textbf{5} & \textbf{6} & \textbf{7} & \textbf{8} & \textbf{9} & \textbf{10} \\
			            \hline\textbf{1}  & 0          & 0          & 1          & 0          & 0          & 1          & 0          & 0          & 1          & 0           \\
			            \hline\textbf{2}  & 0          & 0          & 1          & 0          & 0          & 1          & 0          & 0          & 1          & 0           \\
			            \hline\textbf{3}  & 1          & 1          & 1          & 1          & 1          & 1          & 1          & 1          & 1          & 1           \\
			            \hline\textbf{4}  & 0          & 0          & 1          & 0          & 0          & 1          & 0          & 0          & 1          & 0           \\
			            \hline\textbf{5}  & 0          & 0          & 1          & 0          & 0          & 1          & 0          & 0          & 1          & 0           \\
			            \hline\textbf{6}  & 1          & 1          & 1          & 1          & 1          & 1          & 1          & 1          & 1          & 1           \\
			            \hline\textbf{7}  & 0          & 0          & 1          & 0          & 0          & 1          & 0          & 0          & 1          & 0           \\
			            \hline\textbf{8}  & 0          & 0          & 1          & 0          & 0          & 1          & 0          & 0          & 1          & 0           \\
			            \hline\textbf{9}  & 1          & 1          & 1          & 1          & 1          & 1          & 1          & 1          & 1          & 1           \\
			            \hline\textbf{10} & 0          & 0          & 1          & 0          & 0          & 1          & 0          & 0          & 1          & 0           \\
			            \hline
		            \end{tabular}
	      \end{enumerate}
	\item Написать программы, формирующие матрицы заданных отношений (см. ”Варианты заданий”, п.а).\\
	      \textit{main.cpp}
	      \begin{minted}{C++}
#include "../../libs/alg/alg.h"

bool predA(int x, int y) {
	return (x < y && y < (9 - x)) || ((9 - x) < y && y < x);
}

bool predB(int x, int y) {
	return x % 2 == 0 && y % 2 != 0;
}

bool predC(int x, int y) {
	return (x * y) % 3 == 0 ;
}

int main() {
	BoolMatrixRelation a(10, predA);
	std::cout << a << std::endl;
	
	BoolMatrixRelation b(10, predB);
	std::cout << b << std::endl;
	
	BoolMatrixRelation c(10, predC);
	std::cout << c << std::endl;
}
    	\end{minted}
	      \textit{alg.h} (объявление методов класса)
	      \begin{minted}{C++}
class BoolMatrixRelation
{
	private:
	std::vector<std::vector<bool>> data;
	int size;
	
	public:
	BoolMatrixRelation(const int size, bool (*pred)(int, int));
	~BoolMatrixRelation();
	friend std::ostream& operator<<(std::ostream& out, BoolMatrixRelation &val) {
		out << std::setw(3) << "" << " ";
		for (int i = 1; i <= val.size; i++) {
			out << std::setw(3) << i << " ";
		}
		out << "\n";
		
		for (int x = 0; x < val.size; x++) {
			out << std::setw(3) << x + 1 << " ";
			for (int y = 0; y < val.size; y++) {
				out << std::setw(3) << val.data[x][y] << " ";
			}
			
			out << "\n";
		}
		
		return out;
	}
};
\end{minted}
	      \textit{task13.cpp} (реализация методов класса)
	      \begin{minted}{C++}
#include "../alg.h"

BoolMatrixRelation::BoolMatrixRelation(const int size, bool (*pred)(int, int)) {
	this->size = size;
	
	for (int x = 1; x <= size; x++) {
		std::vector<bool> val;
		
		for (int y = 1; y <= size; y++) {
			val.push_back(pred(x, y));
		}
		
		this->data.push_back(val);
	}
}

BoolMatrixRelation::~BoolMatrixRelation() {}
\end{minted}
	      Результат выполнения программы:\\
	      \includegraphics[width=110mm]{1.3}
	\item Программно реализовать операции над отношениями.\\
	      Немного модифицируем класс BoolMatrixRelation.
	      \textit{alg.h} (объявление методов класса)
	      \begin{minted}{C++}
class BoolMatrixRelation
{
	private:
	std::vector<std::vector<bool>> data;
	int size;
	
	static BoolMatrixRelation getDefault() {
		return BoolMatrixRelation();
	}
	
	public:
	BoolMatrixRelation(const int size, std::function<bool (int, int)> pred);
	BoolMatrixRelation() {
		this->size = 0;
	}
	~BoolMatrixRelation();
	
	bool includes(BoolMatrixRelation b);
	bool equals(BoolMatrixRelation b);
	bool includesStrict(BoolMatrixRelation b);
	BoolMatrixRelation unite(BoolMatrixRelation b);
	BoolMatrixRelation intersect(BoolMatrixRelation b);
	BoolMatrixRelation diff(BoolMatrixRelation b);
	BoolMatrixRelation symDiff(BoolMatrixRelation b);
	BoolMatrixRelation non();
	BoolMatrixRelation transpose();
	BoolMatrixRelation compose(BoolMatrixRelation b);
	BoolMatrixRelation pow(int p);
	
	friend std::ostream& operator<<(std::ostream& out, BoolMatrixRelation &val) {
		out << std::setw(3) << "" << " ";
		for (int i = 1; i <= val.size; i++) {
			out << std::setw(3) << i << " ";
		}
		out << "\n";
		
		for (int x = 0; x < val.size; x++) {
			out << std::setw(3) << x + 1 << " ";
			for (int y = 0; y < val.size; y++) {
				out << std::setw(3) << val.data[x][y] << " ";
			}
			
			out << "\n";
		}
		
		return out;
	}
};
    \end{minted}
	      \textit{task13.cpp} (реализация методов класса)
	      \begin{minted}{C++}
#include "../alg.h"

BoolMatrixRelation::BoolMatrixRelation(const int size, std::function<bool (int, int)> pred)
{
	this->size = size;
	
	for (int x = 1; x <= size; x++)
	{
		std::vector<bool> val;
		
		for (int y = 1; y <= size; y++)
		{
			val.push_back(pred(x, y));
		}
		
		this->data.push_back(val);
	}
}

BoolMatrixRelation::~BoolMatrixRelation() {}
    \end{minted}
	      \textit{task14.cpp} (реализация методов класса)
	      \begin{minted}{C++}
#include "../alg.h"

bool BoolMatrixRelation::includes(BoolMatrixRelation b)
{
	if (this->size != b.size) return false;
	
	for (int i = 0; i < size; i++) {
		for (int j = 0; j < size; j++) {
			if (data[i][j] && !b.data[i][j])
			return false;
		}
	}
	return true;
}
bool BoolMatrixRelation::equals(BoolMatrixRelation b)
{
	if (this->size != b.size) return false;
	
	for (int i = 0; i < size; i++) {
		for (int j = 0; j < size; j++) {
			if (data[i][j] != b.data[i][j])
			return false;
		}
	}
	return true;
}
bool BoolMatrixRelation::includesStrict(BoolMatrixRelation b)
{
	return (*this).includes(b) && !(*this).equals(b);
}
BoolMatrixRelation BoolMatrixRelation::unite(BoolMatrixRelation b)
{
	if (this->size != b.size) return BoolMatrixRelation::getDefault();
	
	return BoolMatrixRelation(size, [this, &b](int x, int y) {
		return data[x - 1][y - 1] || b.data[x - 1][y - 1];
	});
}
BoolMatrixRelation BoolMatrixRelation::intersect(BoolMatrixRelation b)
{
	if (this->size != b.size) return BoolMatrixRelation::getDefault();
	
	return BoolMatrixRelation(size, [this, &b](int x, int y) {
		return data[x - 1][y - 1] && b.data[x - 1][y - 1];
	});
}
BoolMatrixRelation BoolMatrixRelation::diff(BoolMatrixRelation b)
{
	if (this->size != b.size) return BoolMatrixRelation::getDefault();
	
	return BoolMatrixRelation(size, [this, &b](int x, int y) {
		return data[x - 1][y - 1] && !b.data[x - 1][y - 1];
	});
}
BoolMatrixRelation BoolMatrixRelation::symDiff(BoolMatrixRelation b)
{
	if (this->size != b.size) return BoolMatrixRelation::getDefault();
	
	return BoolMatrixRelation(size, [this, &b](int x, int y) {
		return data[x - 1][y - 1] ^ b.data[x - 1][y - 1];
	});
}
BoolMatrixRelation BoolMatrixRelation::non()
{
	return BoolMatrixRelation(size, [this](int x, int y) {
		return !data[x - 1][y - 1];
	});
}
BoolMatrixRelation BoolMatrixRelation::transpose()
{
	return BoolMatrixRelation(size, [this](int x, int y) {
		return data[y - 1][x - 1];
	});
}
BoolMatrixRelation BoolMatrixRelation::compose(BoolMatrixRelation b)
{
	if (this->size != b.size) return BoolMatrixRelation::getDefault();
	
	return BoolMatrixRelation(size, [this, &b](int x, int y) {
		for (int z = 0; z < size; z++) {
			if (data[x - 1][z] && b.data[z][y - 1])
			return true;
		}
		
		return false;
	});
}
BoolMatrixRelation BoolMatrixRelation::pow(int p)
{
	if (p < 0) return transpose();
	if (p == 0) return BoolMatrixRelation(size, [](int x, int y){return x == y;});
	if (p == 1) return *this;
	
	BoolMatrixRelation lowP = pow(p - 1);
	return compose(lowP);
}
\end{minted}
	\item Написать программу, вычисляющую значение выражения (см. “Варианты заданий”, п.б) и вычислить его при заданных отношениях (см.”Варианты заданий”, п.а).
	      \\
	      \textit{main.cpp}
	      \begin{minted}{C++}
#include "../../libs/alg/alg.h"

bool predA(int x, int y) {
	return (x < y && y < (9 - x)) || ((9 - x) < y && y < x);
}

bool predB(int x, int y) {
	return x % 2 == 0 && y % 2 != 0;
}

bool predC(int x, int y) {
	return (x * y) % 3 == 0 ;
}

int main() {
	BoolMatrixRelation a(10, predA);
	BoolMatrixRelation b(10, predB);
	BoolMatrixRelation c(10, predC);
	BoolMatrixRelation d = (((a.compose(b)).compose(b)).diff(c.non())).unite(c.pow(-1));
	
	std::cout << d << std::endl;
}
    \end{minted}
	      Результат выполнения программы:\\
	      \includegraphics[width=110mm]{1.5}\\
	      Значение формулы D, вычисленное вручную, и результат выполненния программы совпали.
\end{enumerate}

\begin{center} \textbf{Часть 2. Свойства отношений} \end{center}
\begin{enumerate}[label=2.\arabic*.]
	\item Определить основные свойства отношений (см. ”Варианты заданий”, п.а).\\
	      \begin{tabular}{|c|c|c|c|}
		      \hline
		                         & A & B & C \\
		      \hline
		      Рефлексивность     &   &   &   \\
		      \hline
		      Антирефлексивность & + & + &   \\
		      \hline
		      Симметричность     &   &   & + \\
		      \hline
		      Антисимметричность & + & + &   \\
		      \hline
		      Транзитивность     & + & + &   \\
		      \hline
		      Антитранзитивность &   & + &   \\
		      \hline
		      Полнота            &   &   &   \\
		      \hline
	      \end{tabular}
	\item Определить, являются ли заданные отношения отношениями толерантности, эквивалентности и порядка.\\
	      \begin{tabular}{|c|c|c|c|}
		      \hline
		                                   & A & B & C \\
		      \hline
		      Толерантно                   &   &   &   \\
		      \hline
		      Эквивалентно                 &   &   &   \\
		      \hline
		      Порядка                      & + & + &   \\
		      \hline
		      Нестрогого порядка           &   &   &   \\
		      \hline
		      Строгого порядка             & + & + &   \\
		      \hline
		      Линейного порядка            &   &   &   \\
		      \hline
		      Нестрогого линейного порядка &   &   &   \\
		      \hline
		      Строгого линейного порядка   &   &   &   \\

		      \hline
	      \end{tabular}
	\item Написать программу, определяющую свойства отношения, в том числе толерантности, эквивалентности и порядка, и определить свойства отношений (см. ”Варианты заданий”, п.а).
	      \textit{main.cpp}
	      \begin{minted}{C++}
#include "../../libs/alg/alg.h"

bool predA(int x, int y) {
    return (x < y && y < (9 - x)) || ((9 - x) < y && y < x);
}

bool predB(int x, int y) {
    return x % 2 == 0 && y % 2 != 0;
}

bool predC(int x, int y) {
    return (x * y) % 3 == 0 ;
}

void outputProperties(std::string name, BoolMatrixRelation a) {
    std::pair<int, int> failedAt;
    std::cout << "Properties for " << name << "\n";
    if (a.isReflexive(failedAt)) {
        std::cout << "Reflexive\n";
    } else {
        std::cout << "Non reflexive, failed pair: (" << failedAt.first << ", " << failedAt.second << ")\n";
    }
    if (a.isAntiReflexive(failedAt)) {
        std::cout << "Antireflexive\n";
    } else {
        std::cout << "Non antireflexive, failed pair: (" << failedAt.first << ", " << failedAt.second << ")\n";
    }
    
    if (a.isSymmetric(failedAt)) {
        std::cout << "Symmetric\n";
    } else {
        std::cout << "Non symmetric, failed pair: (" << failedAt.first << ", " << failedAt.second << ")\n";
    }
    
    if (a.isAntiSymmetric(failedAt)) {
        std::cout << "AntiSymmetric\n";
    } else {
        std::cout << "Non antisymmetric, failed pair: (" << failedAt.first << ", " << failedAt.second << ")\n";
    }
    
    if (a.isTransitive(failedAt)) {
        std::cout << "Transitive\n";
    } else {
        std::cout << "Non transitive, failed pair: (" << failedAt.first << ", " << failedAt.second << ")\n";
    }
    
    if (a.isAntiTransitive(failedAt)) {
        std::cout << "AntiTransitive\n";
    } else {
        std::cout << "Non antitransitive, failed pair: (" << failedAt.first << ", " << failedAt.second << ")\n";
    }
    
    if (a.isFull(failedAt)) {
        std::cout << "Full\n";
    } else {
        std::cout << "Non full, failed pair: (" << failedAt.first << ", " << failedAt.second << ")\n";
    }
    std::cout << (a.isTolerant() ? "Tolerant" : "Non tolerant") << "\n";
    std::cout << (a.isEquivalent() ? "Equivalent" : "Non equivalent") << "\n";
    std::cout << (a.isOrdered() ? "Ordered" : "Non ordered") << "\n";
    std::cout << (a.isOrderedNonStrict() ? "Ordered non strict" : "Non ordered non strict") << "\n";
    std::cout << (a.isOrderedStrict() ? "Ordered strict" : "Non ordered strict") << "\n";
    std::cout << (a.isOrderedLinear() ? "Ordered linear" : "Non ordered linear") << "\n";
    std::cout << (a.isOrderedLinearNonStrict() ? "Ordered linear non strict" : "Non ordered linear non strict") << "\n";
    std::cout << (a.isOrderedLinearStrict() ? "Ordered linear strict" : "Non ordered linear strict") << "\n" << std::endl;
}

int main() {
    BoolMatrixRelation a(10, predA);
    BoolMatrixRelation b(10, predB);
    BoolMatrixRelation c(10, predC);

    outputProperties("A", a);
    outputProperties("B", b);
    outputProperties("C", c);
}
    	\end{minted}
	      \textit{alg.h}
	      \begin{minted}{C++}
class BoolMatrixRelation
{
private:
    std::vector<std::vector<bool>> data;
    int size;

    static BoolMatrixRelation getDefault() {
        return BoolMatrixRelation();
    }

public:
    BoolMatrixRelation(const int size, std::function<bool (int, int)> pred);
    BoolMatrixRelation() {
        this->size = 0;
    }
    ~BoolMatrixRelation();

    bool includes(BoolMatrixRelation b);
    bool equals(BoolMatrixRelation b);
    bool includesStrict(BoolMatrixRelation b);
    BoolMatrixRelation unite(BoolMatrixRelation b);
    BoolMatrixRelation intersect(BoolMatrixRelation b);
    BoolMatrixRelation diff(BoolMatrixRelation b);
    BoolMatrixRelation symDiff(BoolMatrixRelation b);
    BoolMatrixRelation non();
    BoolMatrixRelation transpose();
    BoolMatrixRelation compose(BoolMatrixRelation b);
    BoolMatrixRelation pow(int p);

    static BoolMatrixRelation getIdentity(int size);
    static BoolMatrixRelation getUniversum(int size);

    bool isEmpty();

    bool isReflexive(std::pair<int, int> &failed);
    bool isAntiReflexive(std::pair<int, int> &failed);
    bool isSymmetric(std::pair<int, int> &failed);
    bool isAntiSymmetric(std::pair<int, int> &failed);
    bool isTransitive(std::pair<int, int> &failed);
    bool isAntiTransitive(std::pair<int, int> &failed);
    bool isFull(std::pair<int, int> &failed);
    bool isTolerant();
    bool isEquivalent();
    bool isOrdered();
    bool isOrderedNonStrict();
    bool isOrderedStrict();
    bool isOrderedLinear();
    bool isOrderedLinearNonStrict();
    bool isOrderedLinearStrict();

    friend std::ostream& operator<<(std::ostream& out, BoolMatrixRelation &val) {
        out << std::setw(3) << "" << " ";
        for (int i = 1; i <= val.size; i++) {
            out << std::setw(3) << i << " ";
        }
        out << "\n";

        for (int x = 0; x < val.size; x++) {
            out << std::setw(3) << x + 1 << " ";
            for (int y = 0; y < val.size; y++) {
                out << std::setw(3) << val.data[x][y] << " ";
            }

            out << "\n";
        }

        return out;
    }
};
    	\end{minted}
	      \textit{task23.cpp}
	      \begin{minted}{C++}
#include "../alg.h"

bool BoolMatrixRelation::isEmpty() {
    for (int i = 0; i < size; i++) {
        for (int j = 0; j < size; j++) {
            if (data[i][j]) return false;
        }
    }

    return true;
}

BoolMatrixRelation BoolMatrixRelation::getIdentity(int size) {
    return BoolMatrixRelation(size, [](int x, int y) {
        return x == y;
    });
}
BoolMatrixRelation BoolMatrixRelation::getUniversum(int size) {
return BoolMatrixRelation(size, [](int x, int y) {
        return true;
    });
}

bool BoolMatrixRelation::isReflexive(std::pair<int, int> &failed)
{
    for (int i = 0; i < size; i++) {
        if (!data[i][i]) {
            failed = {i + 1, i + 1};
            return false;
        }
    }

    return true;
}
bool BoolMatrixRelation::isAntiReflexive(std::pair<int, int> &failed)
{
    for (int i = 0; i < size; i++) {
        if (data[i][i]) {
            failed = {i + 1, i + 1};
            return false;
        }
    }

    return true;
}
bool BoolMatrixRelation::isSymmetric(std::pair<int, int> &failed)
{
    for (int i = 0; i < size; i++) {
        for (int j = i + 1; j < size; j++) {
            if (data[i][j] != data[j][i]) {
                failed = {i + 1, j + 1};
                return false;
            }
        }
    }

    return true;
}
bool BoolMatrixRelation::isAntiSymmetric(std::pair<int, int> &failed)
{
    for (int i = 0; i < size; i++) {
        for (int j = i + 1; j < size; j++) {
            if (data[i][j] && data[j][i]) {
                failed = {i + 1, j + 1};
                return false;
            }
        }
    }

    return true;
}
bool BoolMatrixRelation::isTransitive(std::pair<int, int> &failed)
{
    //return ((*this).pow(2)).includes((*this));
    for (int i = 0; i < size; i++) {
        for (int j = 0; j < size; j++) {
            for (int z = 0; z < size; z++) {
                if ((data[i][z] && data[z][j]) && !data[i][j]) {
                    failed = {i + 1, j + 1};
                    return false;
                }
            }
        }
    }

    return true;
}
bool BoolMatrixRelation::isAntiTransitive(std::pair<int, int> &failed)
{
    for (int i = 0; i < size; i++) {
        for (int j = 0; j < size; j++) {
            for (int z = 0; z < size; z++) {
                if ((data[i][z] && data[z][j]) && data[i][j]) {
                    failed = {i + 1, j + 1};
                    return false;
                }
            }
        }
    }

    return true;
}
bool BoolMatrixRelation::isFull(std::pair<int, int> &failed)
{
    for (int i = 0; i < size; i++) {
        for (int j = i + 1; j < size; j++) {
            if (!(data[i][j] || data[j][i])) {
                failed = {i + 1, j + 1};
                return false;
            }
        }
    }

    return true;
}
bool BoolMatrixRelation::isTolerant()
{
    std::pair<int, int> ignored;
    return isReflexive(ignored) && isSymmetric(ignored);
}
bool BoolMatrixRelation::isEquivalent()
{
    std::pair<int, int> ignored;
    return isReflexive(ignored) && isSymmetric(ignored) && isTransitive(ignored);
}
bool BoolMatrixRelation::isOrdered()
{
    std::pair<int, int> ignored;
    return isAntiSymmetric(ignored) && isTransitive(ignored);
}
bool BoolMatrixRelation::isOrderedNonStrict()
{
    std::pair<int, int> ignored;
    return isOrdered() && isReflexive(ignored);
}
bool BoolMatrixRelation::isOrderedStrict()
{
    std::pair<int, int> ignored;
    return isOrdered() && isAntiReflexive(ignored);
}
bool BoolMatrixRelation::isOrderedLinear()
{
    std::pair<int, int> ignored;
    return isOrdered() && isFull(ignored);
}
bool BoolMatrixRelation::isOrderedLinearNonStrict()
{
    std::pair<int, int> ignored;
    return isOrderedNonStrict() && isFull(ignored);
}
bool BoolMatrixRelation::isOrderedLinearStrict()
{
    std::pair<int, int> ignored;
    return isOrderedStrict() && isFull(ignored);
}
    	\end{minted}
	      Результат выполнения программы:\\
	      \includegraphics[width=110mm]{2.3}
\end{enumerate}

\textbf{Вывод: } в ходе лабораторной работы изучили способы задания отношений, операции над отношениями и свойства отношений, научились программно реализовывать операции и определять свойства отношений.

\end{document}