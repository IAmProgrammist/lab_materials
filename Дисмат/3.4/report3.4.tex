\documentclass[a4paper,14pt]{extarticle}


\usepackage[english,russian]{babel}
\usepackage[T2A]{fontenc}
\usepackage[utf8]{inputenc}
\usepackage{ragged2e}
\usepackage[utf8]{inputenc}
\usepackage{hyperref}
\usepackage{minted}
\setmintedinline{frame=lines, framesep=2mm, baselinestretch=1.5, bgcolor=LightGray, breaklines,fontsize=\scriptsize}
\setminted{frame=lines, framesep=2mm, baselinestretch=1.5, bgcolor=LightGray, breaklines,fontsize=\scriptsize}
\usepackage{xcolor}
\definecolor{LightGray}{gray}{0.9}
\usepackage{graphicx}
\usepackage[export]{adjustbox}
\usepackage[left=1cm,right=1cm, top=1cm,bottom=1cm,bindingoffset=0cm]{geometry}
\usepackage{fontspec}
\usepackage{ upgreek }
\usepackage[shortlabels]{enumitem}
\usepackage{adjustbox}
\usepackage{multirow}
\usepackage{amsmath}
\usepackage{amssymb}
\usepackage{pifont}
\usepackage{pgfplots}
\usepackage{longtable}
\usepackage{array}
\graphicspath{ {./images/} }
\makeatletter
\AddEnumerateCounter{\asbuk}{\russian@alph}{щ}
\makeatother
\setmonofont{Consolas}
\setmainfont{Times New Roman}

\newcommand\textbox[1]{
	\parbox{.45\textwidth}{#1}
} 

\newcommand{\specialcell}[2][c]{%
	\begin{tabular}[#1]{@{}c@{}}#2\end{tabular}}

\begin{document}
\pagenumbering{gobble}
\begin{center}
	\small{
		\textbf{МИНИCТЕРCТВО НАУКИ И ВЫCШЕГО ОБРАЗОВАНИЯ РОCCИЙCКОЙ ФЕДЕРАЦИИ}\\
		ФЕДЕРАЛЬНОЕ ГОCУДАРCТВЕННОЕ БЮДЖЕТНОЕ ОБРАЗОВАТЕЛЬНОЕ УЧРЕЖДЕНИЕ\\ВЫCШЕГО ОБРАЗОВАНИЯ \\
		\textbf{«БЕЛГОРОДCКИЙ ГОCУДАРCТВЕННЫЙ ТЕХНОЛОГИЧЕCКИЙ\\УНИВЕРCИТЕТ им. В. Г. ШУХОВА»\\ (БГТУ им. В.Г. Шухова)} \\
		\bigbreak
		\includegraphics[width=70mm]{log}\\
		ИНСТИТУТ ИНФОРМАЦИОННЫХ ТЕХНОЛОГИЙ И УПРАВЛЯЮЩИХ СИСТЕМ\\}
\end{center}

\vfill
\begin{center}
	\large{
		\textbf{
			Лабораторная работа №3.4}}\\
	\normalsize{
		по дисциплине: Дискретная математика \\
		тема: «Упорядоченные множества»}
\end{center}
\vfill
\hfill\textbox{
	Выполнил: ст. группы ПВ-223\\Пахомов Владислав Андреевич
	\bigbreak
	Проверили: \\ст. пр. Рязанов Юрий Дмитриевич\\
	ст. пр. Бондаренко Татьяна Владимировна
}
\vfill\begin{center}
	Белгород 2023 г.
\end{center}
\newpage
\begin{center}
	\textbf{Лабораторная работа №3.4}\\
	Упорядоченные множества
\end{center}
\textbf{Цель работы: }изучить упорядоченные множества, алгоритм 
топологической сортировки, научиться представлять 
множества диаграммами Хассе, находить минимальные
(максимальные) и наименьшие (наибольшие) элементы 
упорядоченного множества.

\begin{enumerate}[1.]
	\item Написать программы, формирующие матрицы отношений в 
	соответствии с вариантом задания, на множествах $М_1$ и $М_2$.\\
	\textit{alg.h}
	\begin{minted}{C++}
template <typename T>
class Relation : public BoolMatrixRelation
{
protected:
    std::vector<T> origin;

public:
    Relation(std::vector<T> origin, std::function<bool(T, T)> pred) : 
	BoolMatrixRelation(origin.size(), 
	[&origin, &pred](int x, int y) { 
		return pred(origin[x - 1], origin[y - 1]); 
	})
    {
        this->origin = origin;
    };

    friend std::ostream& operator<<(std::ostream& out, Relation<T>& val) {
        int blockSize = 10;
        out << std::setw(blockSize) << ""
            << " ";
        for (int i = 0; i < val.size; i++)
        {
            std::stringstream ss;
            ss << val.origin[i];
            std::string buf = ss.str();

            out << std::setw(blockSize) << buf << " ";
        }
        out << "\n";

        for (int x = 0; x < val.size; x++)
        {
            std::stringstream ss;
            ss << val.origin[x];
            std::string buf = ss.str();

            out << std::setw(blockSize) << buf << " ";
            for (int y = 0; y < val.size; y++)
            {
                out << std::setw(blockSize) << val.data[x][y] << " ";
            }

            out << "\n";
        }

        return out;
    }
};
	\end{minted}
	\textit{main.cpp}
	\begin{minted}{C++}
#include <vector>
#include <iostream>

typedef std::pair<int, int> Vec2;

std::ostream& operator<<(std::ostream& out, Vec2& o) {
    out << "(" << o.first << ", " << o.second << ")";

    return out;
}

#include "../../libs/alg/alg.h"

bool predicate(Vec2 a, Vec2 b) {
    return (a.first - b.first) < (b.second - a.second);
}

int main() {
    std::vector<Vec2> M1 = {
    {-1,  1}, {0,  1}, {1,  1},
    {-1,  0}, {0,  0}, {1,  0},
    {-1, -1}, {0, -1}, {1, -1}
    };

    Relation<Vec2> m1Ordered(M1, predicate);


    std::vector<Vec2> M2 = {
                        { 0,  2},
                {-1,  1}, { 0,  1}, { 1,  1},
    {-2,  0}, {-1,  0}, { 0,  0}, { 1,  0}, { 2,  0},
                {-1, -1}, { 0, -1}, { 1, -1},
                        { 0, -2}
    };

    Relation<Vec2> m2Ordered(M2, predicate);

    std::cout << m1Ordered << "\n" << m2Ordered;

    return 0;
}
			\end{minted}
			Результат выполнения программы:\\
			\includegraphics[width=140mm]{task1}
			\item Написать программы, формирующие матрицы отношения 
			доминирования по матрицам отношения порядка.\\
	\textit{alg.h}
	\begin{minted}{C++}
template <typename T>
class DominantRelation;

template <typename T>
class Relation : public BoolMatrixRelation
{
protected:
    std::vector<T> origin;

public:
    Relation(std::vector<T> origin, std::function<bool(T, T)> pred) : BoolMatrixRelation(origin.size(), [&origin, &pred](int x, int y)
                                                                                         { return pred(origin[x - 1], origin[y - 1]); })
    {
        this->origin = origin;
    };

    DominantRelation<T> getDominantRelation();

    friend std::ostream& operator<<(std::ostream& out, Relation<T>& val) {
        int blockSize = 10;
        out << std::setw(blockSize) << ""
            << " ";
        for (int i = 0; i < val.size; i++)
        {
            std::stringstream ss;
            ss << val.origin[i];
            std::string buf = ss.str();

            out << std::setw(blockSize) << buf << " ";
        }
        out << "\n";

        for (int x = 0; x < val.size; x++)
        {
            std::stringstream ss;
            ss << val.origin[x];
            std::string buf = ss.str();

            out << std::setw(blockSize) << buf << " ";
            for (int y = 0; y < val.size; y++)
            {
                out << std::setw(blockSize) << val.data[x][y] << " ";
            }

            out << "\n";
        }

        return out;
    }
};

template <typename T>
class DominantRelation : public Relation<T>
{
public:
    DominantRelation(std::vector<T> origin, std::function<bool(T, T)> pred) : Relation<T>(origin, pred) {};
};

#include "lab10/task2.tpp"
	\end{minted}
	
	\textit{task2.tpp}
	\begin{minted}{C++}
template <typename T>
DominantRelation<T> Relation<T>::getDominantRelation()
{
    if (!isOrdered())
        throw std::invalid_argument("Original relation should be relation of order");

    DominantRelation<T> result(origin, [](T, T)
                               { return false; });
    for (int x = 0; x < size; x++)
    {
        for (int y = 0; y < size; y++)
        {
            if (!data[x][y])
                continue;
            bool anyZ = false;

            for (int z = 0; z < size && !anyZ; z++)
            {
                if (data[x][z] && data[z][y])
                    anyZ = true;
            }

            result.data[x][y] = !anyZ;
        }
    }

    return result;
}
	\end{minted}
	\textit{main.cpp}
	\begin{minted}{C++}
#include <vector>
#include <iostream>

typedef std::pair<int, int> Vec2;

std::ostream& operator<<(std::ostream& out, Vec2& o) {
    out << "(" << o.first << ", " << o.second << ")";

    return out;
}

#include "../../libs/alg/alg.h"

bool predicate(Vec2 a, Vec2 b) {
    return (a.first - b.first) < (b.second - a.second);
}

int main() {
    std::vector<Vec2> M1 = {{-1,  1}, {0,  1}, {1,  1},
                                           {-1,  0}, {0,  0}, {1,  0},
                                           {-1, -1}, {0, -1}, {1, -1}};

    Relation<Vec2> m1Ordered(M1, predicate);
    auto dominationRelM1 = m1Ordered.getDominantRelation();

    std::cout << dominationRelM1 << std::endl;


    std::vector<Vec2> M2 = {
                                               { 0,  2},
                                     {-1,  1}, { 0,  1}, { 1,  1},
                           {-2,  0}, {-1,  0}, { 0,  0}, { 1,  0}, { 2,  0},
                                     {-1, -1}, { 0, -1}, { 1, -1},
                                               { 0, -2}};

    Relation<Vec2> m2Ordered(M2, predicate);
    auto dominationRelM2 = m2Ordered.getDominantRelation();

    std::cout << dominationRelM2 << std::endl;

    return 0;
}
			\end{minted}
			Результат выполнения программы:\\
			\includegraphics[width=140mm]{task2}
			\item Написать программу, реализующую алгоритм топологической
			сортировки по матрице отношения доминирования.\\
			\textit{alg.h}
			\begin{minted}{C++}
template <typename T>
class DominantRelation;

template <typename T>
class Relation : public BoolMatrixRelation
{
protected:
    std::vector<T> origin;

public:
    Relation(std::vector<T> origin, std::function<bool(T, T)> pred) : BoolMatrixRelation(origin.size(), [&origin, &pred](int x, int y)
                                                                                         { return pred(origin[x - 1], origin[y - 1]); })
    {
        this->origin = origin;
    };

    DominantRelation<T> getDominantRelation();

    friend std::ostream& operator<<(std::ostream& out, Relation<T>& val) {
        int blockSize = 10;
        out << std::setw(blockSize) << ""
            << " ";
        for (int i = 0; i < val.size; i++)
        {
            std::stringstream ss;
            ss << val.origin[i];
            std::string buf = ss.str();

            out << std::setw(blockSize) << buf << " ";
        }
        out << "\n";

        for (int x = 0; x < val.size; x++)
        {
            std::stringstream ss;
            ss << val.origin[x];
            std::string buf = ss.str();

            out << std::setw(blockSize) << buf << " ";
            for (int y = 0; y < val.size; y++)
            {
                out << std::setw(blockSize) << val.data[x][y] << " ";
            }

            out << "\n";
        }

        return out;
    }
};

template <typename T>
class DominantRelation : public Relation<T>
{
public:
    DominantRelation(std::vector<T> origin, std::function<bool(T, T)> pred) : Relation<T>(origin, pred) {};

    std::vector<std::vector<T>> getTopologicalSort();
};

#include "lab10/task2.tpp"
#include "lab10/task3.tpp"
			\end{minted}
			\textit{task3.tpp}
			\begin{minted}{C++}
template <typename T>
std::vector<std::vector<T>> DominantRelation<T>::getTopologicalSort() {
    std::vector<int> W(this->size, 0);

    for (int x = 0; x < this->size; x++) {
        for (int y = 0; y < this->size; y++) {
            W[y] += this->data[x][y];
        }
    }

    int ind = 0;

    while (true)
    {
        std::vector<int> indices;
        
        // Поиск нулевых элементов и сохранение их индексов
        auto iter = W.begin();
        while ((iter = std::find(iter, W.end(), 0)) != W.end()) {
            indices.push_back(iter - W.begin());

            iter++;
        }

        if (indices.size() == 0) break;

        ind--;
        for (auto index : indices) {
            // Заменяем нулевой элемент 
            W[index] = ind;

            for (int elementIndex = 0; elementIndex < this->data[index].size(); elementIndex++) { 
                auto element = this->data[index][elementIndex];
                W[elementIndex] -= element;
            }
        }
    }

    std::vector<std::vector<T>> levels;
    for (int i = 0; i < this->size; i++) {
        std::vector<T> level;

        for (int j = 0; j < this->size; j++) {
            if (W[j] == -(1 + i)) 
                level.push_back(this->origin[j]);
        }

        if (level.empty()) break;

        levels.push_back(level);
    }
    
    return levels;
}
					\end{minted}
					\textit{main.cpp}
					\begin{minted}{C++}
#include <vector>
#include <iostream>

typedef std::pair<int, int> Vec2;

std::ostream& operator<<(std::ostream& out, Vec2& o) {
    out << "(" << o.first << ", " << o.second << ")";

    return out;
}

#include "../../libs/alg/alg.h"

bool predicate(Vec2 a, Vec2 b) {
    return (a.first - b.first) < (b.second - a.second);
}

int main() {
    std::vector<Vec2> M1 = {{-1,  1}, {0,  1}, {1,  1},
                                           {-1,  0}, {0,  0}, {1,  0},
                                           {-1, -1}, {0, -1}, {1, -1}};

    Relation<Vec2> m1Ordered(M1, predicate);
    auto dominationRelM1 = m1Ordered.getDominantRelation();
    auto m1levels = dominationRelM1.getTopologicalSort();
    for (auto it = m1levels.end() - 1; it >= m1levels.begin(); it--) {
        for (auto element : *it) {
            std::cout << "(" << element.first << ", " << element.second << ") ";
        }

        std::cout << "\n";
    }

    std::cout << std::endl;


    std::vector<Vec2> M2 = {
                                               { 0,  2},
                                     {-1,  1}, { 0,  1}, { 1,  1},
                           {-2,  0}, {-1,  0}, { 0,  0}, { 1,  0}, { 2,  0},
                                     {-1, -1}, { 0, -1}, { 1, -1},
                                               { 0, -2}};

    Relation<Vec2> m2Ordered(M2, predicate);
    auto dominationRelM2 = m2Ordered.getDominantRelation();
    auto m2levels = dominationRelM2.getTopologicalSort();
    for (auto it = m2levels.end() - 1; it >= m2levels.begin(); it--) {
        for (auto element : *it) {
            std::cout << "(" << element.first << ", " << element.second << ") ";
        }

        std::cout << "\n";
    }

    return 0;
}
							\end{minted}
					Результат выполнения программы:\\
					\includegraphics[width=140mm]{task3}
\end{enumerate}

\textbf{Вывод: } в ходе лабораторной работы научились формировать фактормножество для заданного
отношения эквивалентности на ЭВМ.

\end{document}