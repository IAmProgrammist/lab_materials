\documentclass[a4paper,14pt]{extarticle}



\usepackage[english,russian]{babel}
\usepackage[T2A]{fontenc}
\usepackage[utf8]{inputenc}
\usepackage{ragged2e}
\usepackage[utf8]{inputenc}
\usepackage{hyperref}
\usepackage{minted}
\usepackage{xcolor}
\definecolor{LightGray}{gray}{0.9}
\usepackage{graphicx}
\usepackage[export]{adjustbox}
\usepackage[left=1cm,right=1cm, top=1cm,bottom=1cm,bindingoffset=0cm]{geometry}
\usepackage{fontspec}
\usepackage{ upgreek }
\usepackage[shortlabels]{enumitem}
\usepackage{adjustbox}

\usepackage{multirow}
\makeatletter
\AddEnumerateCounter{\asbuk}{\russian@alph}{щ}
\makeatother
\setmonofont{Consolas}
\setmainfont{Times New Roman}

\newcommand\textbox[1]{
	\parbox{.45\textwidth}{#1}
}

\newcommand{\specialcell}[2][c]{%
	\begin{tabular}[#1]{@{}c@{}}#2\end{tabular}}
\usepackage{color,xcolor,colortbl}
\begin{document}	
	
Ядро Квайна – это импликанты, не подлежащие исключению.

\begin{tabular}{ |c|c|c|c|c|c| } 
	\hline
	\multirow{2}{*}{\specialcell{Простые\\ импликанты}} & \multicolumn{5}{|c|}{Конституенты}\\
	\cline{2-6}
	& 1000	&1001	&1010	&1101	&1111 \\ 
	\hline
	10-0&+&&+&&\\
	\hline
	100-&+&+&&&\\
	\hline
	1-01&&+&&+&\\
	\hline
	11-1&&&&+&+\\
	\hline
\end{tabular}\bigbreak

Найдём в импликантной таблице столбцы, которые имеют один плюс. Соответствующие им простые импликанты называются \textit{базисными}, и они составляют \textit{ядро Квайна}. Ядро Квайна будет входить в минимальную тупиковую НФК. Отметим в таблице ядро, закрасив соответствующие столбцы и ряды.\bigbreak

\begin{tabular}{ |c|c|c|c|c|c| } 
	\hline
	\multirow{2}{*}{\specialcell{Простые\\ импликанты}} & \multicolumn{5}{|c|}{Конституенты}\\
	\cline{2-6}
	& 1000	&1001	& \cellcolor{red}1010	&1101	&\cellcolor{red}1111 \\ 
	\hline
	\cellcolor{red} 10-0&\cellcolor{red}+&\cellcolor{red}&\cellcolor{red}+&\cellcolor{red}&\cellcolor{red}\\
	\hline
	100-&+&+&\cellcolor{red}&&\cellcolor{red}\\
	\hline
	1-01&&+&\cellcolor{red}&+&\cellcolor{red}\\
	\hline
	\cellcolor{red}11-1&\cellcolor{red}&\cellcolor{red}&\cellcolor{red}&\cellcolor{red}+&\cellcolor{red}+\\
	\hline
\end{tabular}\bigbreak

Теперь осталось выбрать простую импликанту, покрывающую оставшиеся столбцы. В данном случае это может быть импликанта 100- или 1-01. Для вычисления второй требуется меньше операций, поэтому возьмём её. Таким образом нам пришлось выбирать не из 4, а из 2 простых импликант. Кроме того, мы сразу получили готовую комбинацию импликант.\\

$a \cap c \cap d = A\cap\bar{B}\cap\bar{D}\cup A\cap\bar{C}\cap D\cup A\cap B\cap D$\bigbreak

\begin{tabular}{ |c|c|c|c|c|c| } 
	\hline
	\multirow{2}{*}{\specialcell{Простые\\ импликанты}} & \multicolumn{5}{|c|}{Конституенты}\\
	\cline{2-6}
	& \cellcolor{yellow} 1000	&\cellcolor{yellow}1001	& \cellcolor{red}1010	&\cellcolor{yellow}1101	&\cellcolor{red}1111 \\ 
	\hline
	\cellcolor{red} 10-0&\cellcolor{red}+&\cellcolor{red}&\cellcolor{red}+&\cellcolor{red}&\cellcolor{red}\\
	\hline
	100-&+&+&\cellcolor{red}&&\cellcolor{red}\\
	\hline
	\cellcolor{yellow}1-01&\cellcolor{yellow}&\cellcolor{yellow}+&\cellcolor{red}&\cellcolor{yellow}+&\cellcolor{red}\\
	\hline
	\cellcolor{red}11-1&\cellcolor{red}&\cellcolor{red}&\cellcolor{red}&\cellcolor{red}+&\cellcolor{red}+\\
	\hline
\end{tabular}\bigbreak

\end{document}