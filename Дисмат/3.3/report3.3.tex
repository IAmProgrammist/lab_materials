\documentclass[a4paper,14pt]{extarticle}


\usepackage[english,russian]{babel}
\usepackage[T2A]{fontenc}
\usepackage[utf8]{inputenc}
\usepackage{ragged2e}
\usepackage[utf8]{inputenc}
\usepackage{hyperref}
\usepackage{minted}
\setmintedinline{frame=lines, framesep=2mm, baselinestretch=1.5, bgcolor=LightGray, breaklines,fontsize=\scriptsize}
\setminted{frame=lines, framesep=2mm, baselinestretch=1.5, bgcolor=LightGray, breaklines,fontsize=\scriptsize}
\usepackage{xcolor}
\definecolor{LightGray}{gray}{0.9}
\usepackage{graphicx}
\usepackage[export]{adjustbox}
\usepackage[left=1cm,right=1cm, top=1cm,bottom=1cm,bindingoffset=0cm]{geometry}
\usepackage{fontspec}
\usepackage{ upgreek }
\usepackage[shortlabels]{enumitem}
\usepackage{adjustbox}
\usepackage{multirow}
\usepackage{amsmath}
\usepackage{amssymb}
\usepackage{pifont}
\usepackage{pgfplots}
\usepackage{longtable}
\usepackage{array}
\graphicspath{ {./images/} }
\makeatletter
\AddEnumerateCounter{\asbuk}{\russian@alph}{щ}
\makeatother
\setmonofont{Consolas}
\setmainfont{Times New Roman}

\newcommand\textbox[1]{
	\parbox{.45\textwidth}{#1}
} 

\newcommand{\specialcell}[2][c]{%
	\begin{tabular}[#1]{@{}c@{}}#2\end{tabular}}

\begin{document}
\pagenumbering{gobble}
\begin{center}
	\small{
		МИНИCТЕРCТВО НАУКИ И ВЫCШЕГО ОБРАЗОВАНИЯ \\РОCCИЙCКОЙ ФЕДЕРАЦИИ
		\bigbreak
		ФЕДЕРАЛЬНОЕ ГОCУДАРCТВЕННОЕ БЮДЖЕТНОЕ ОБРАЗОВАТЕЛЬНОЕ УЧРЕЖДЕНИЕ ВЫCШЕГО ОБРАЗОВАНИЯ \\
		\bigbreak
		\textbf{«БЕЛГОРОДCКИЙ ГОCУДАРCТВЕННЫЙ \\ТЕХНОЛОГИЧЕCКИЙ УНИВЕРCИТЕТ им. В. Г. ШУХОВА»\\ (БГТУ им. В.Г. Шухова)} \\
		\bigbreak
		Кафедра программного обеспечения вычислительной техники и автоматизированных систем\\}
\end{center}

\vfill
\begin{center}
	\large{
		\textbf{
			Лабораторная работа №3.3}}\\
	\normalsize{
		по дисциплине: Дискретная математика \\
		тема: «Фактормножества»}
\end{center}
\vfill
\hfill\textbox{
	Выполнил: ст. группы ПВ-223\\Пахомов Владислав Андреевич
	\bigbreak
	Проверили: \\ст. пр. Рязанов Юрий Дмитриевич\\
	ст. пр. Бондаренко Татьяна Владимировна
}
\vfill\begin{center}
	Белгород 2023 г.
\end{center}
\newpage
\begin{center}
	\textbf{Лабораторная работа №3.3}\\
	Фактормножества
\end{center}
\textbf{Цель работы: }научиться формировать фактормножество для заданного
отношения эквивалентности на ЭВМ.

\begin{enumerate}[1.]
	\item Отношение представить графом и характеристической функцией в матричной форме.
	      Найти разбиение Ф, определяемое заданным отношением эквивалентности.\\
	      $\textit{A }=\{(x,y) | x\in N \textit{ и } y\in N \textit{ и } x<11 \textit{ и } y<11 $\\$\textit{ и (x и y кратно 3 или x и y кратно 5 или x=y)}\}$\\
		      \includegraphics[width=100mm]{task1}\\
	      $A = \begin{pmatrix}
		      1 & 0 & 0 & 0 & 0 & 0 & 0 & 0 & 0 & 0 \\
		      0 & 1 & 0 & 0 & 0 & 0 & 0 & 0 & 0 & 0 \\
		      0 & 0 & 1 & 0 & 0 & 1 & 0 & 0 & 1 & 0 \\
		      0 & 0 & 0 & 1 & 0 & 0 & 0 & 0 & 0 & 0 \\
		      0 & 0 & 0 & 0 & 1 & 0 & 0 & 0 & 0 & 1 \\
		      0 & 0 & 1 & 0 & 0 & 1 & 0 & 0 & 1 & 0 \\
		      0 & 0 & 0 & 0 & 0 & 0 & 1 & 0 & 0 & 0 \\
		      0 & 0 & 0 & 0 & 0 & 0 & 0 & 1 & 0 & 0 \\
		      0 & 0 & 1 & 0 & 0 & 1 & 0 & 0 & 1 & 0 \\
		      0 & 0 & 0 & 0 & 1 & 0 & 0 & 0 & 0 & 1
	      \end{pmatrix}$\bigbreak
		      Рассмотрим классы эвкивалентности для $x < 11$ и $x \in N$.\\
	      $[1] = \{1\}$\\
	      $[2] = \{2\}$\\
	      $[3] = \{3, 6, 9\}$\\
	      $[4] = \{4\}$\\
	      $[5] = \{5, 10\}$\\
	      $[6] = \{3, 6, 9\}$\\
	      $[7] = \{7\}$\\
	      $[8] = \{8\}$\\
	      $[9] = \{3, 6, 9\}$\\
	      $[10] = \{5, 10\}$\bigbreak
	      $\textit{Ф }=\{\{1\}, \{2\}, \{3, 6, 9\}, \{4\}, \{5, 10\}, \{7\}, \{8\}\}$
	\item Написать программу, которая формирует разбиение, определяемое заданным отношением эквивалентности.
	      Определим тип данных для компактного представления фактормножества и метод BoolMatrixRelation, возвращающий компактное фактормножество.\\
	      \textit{alg.h}
	      \begin{minted}{C++}
typedef std::vector<int> FactorSet;
FactorSet getPackedFactorSet();
	\end{minted}
	      \textit{task2.cpp}
	      \begin{minted}{C++}
#include "../alg.h"

FactorSet BoolMatrixRelation::getPackedFactorSet() {
    if (!isEquivalent()) throw std::invalid_argument("The relation is not equivalence relation");

    FactorSet result(size, -1);
    for (int x = 0; x < size; x++) {
        bool generatingEquivalenceClass = false;

        for (int y = 0; y < size; y++) {
            if (generatingEquivalenceClass && data[x][y])  {
                result[y] = x + 1;
                continue;
            } else if (result[y] == -1 && data[x][y]) {
                result[y] = x + 1;
                generatingEquivalenceClass = true;
            } else if (result[y] != -1 && data[x][y]) break;
        }
    }

    return result;
}
			\end{minted}
	      \textit{main.cpp}
	      \begin{minted}{C++}
#include "../../libs/alg/alg.h"

std::ostream& operator<<(std::ostream& out, FactorSet& factorSet) {
    for (int i = 0; i < factorSet.size(); i++) {
        bool anyOutput = false;
        for (int j = 0; j < factorSet.size(); j++) {
            if (j < i && factorSet[i] == factorSet[j]) break;
            if (factorSet[i] == factorSet[j]) { 
                out << j + 1 << " "; 
                anyOutput = true;
            }
        }
        if (anyOutput) 
            out << "\b\n";
    }

    return out;
}

int main() {
    BoolMatrixRelation rel(10, [](int x, int y) {
        return (x % 3 == 0 && y % 3 == 0) || (x % 5 == 0 && y % 5 == 0) || x == y;
    });

    FactorSet set = rel.getPackedFactorSet();
    std::cout << set;
    
    return 0;
}
			\end{minted}
	      Результат работы программы\\
	      \includegraphics[width=50mm]{task2}\\
	      Фактормножества, полученное при помощи программы и при помощи ручных вычислений, совпали. Вычисления и программа верны.
\end{enumerate}

\textbf{Вывод: } в ходе лабораторной работы научились формировать фактормножество для заданного
отношения эквивалентности на ЭВМ.

\end{document}