\documentclass[a4paper,14pt]{extarticle}


\usepackage[english,russian]{babel}
\usepackage[T2A]{fontenc}
\usepackage[utf8]{inputenc}
\usepackage{ragged2e}
\usepackage[utf8]{inputenc}
\usepackage{hyperref}
\usepackage{minted}
\usepackage{xcolor}
\definecolor{LightGray}{gray}{0.9}
\usepackage{graphicx}
\usepackage[export]{adjustbox}
\usepackage[left=1cm,right=1cm, top=1cm,bottom=1cm,bindingoffset=0cm]{geometry}
\usepackage{fontspec}
\usepackage{ upgreek }
\usepackage[shortlabels]{enumitem}
\usepackage[mathletters]{ucs}
\usepackage{adjustbox}
\usepackage{multirow}
\usepackage{amsmath}
\usepackage{pifont}
\graphicspath{ {./images/} }
\makeatletter
\AddEnumerateCounter{\asbuk}{\russian@alph}{щ}
\makeatother
\setmonofont{Consolas}
\setmainfont{Times New Roman}

\newcommand\textbox[1]{
	\parbox{.45\textwidth}{#1}
}

\newcommand{\xmark}{\ding{55}}

\newcommand{\specialcell}[2][c]{%
	\begin{tabular}[#1]{@{}c@{}}#2\end{tabular}}

\begin{document}	
	\pagenumbering{gobble}
	\begin{center}
		\small{
			МИНИCТЕРCТВО НАУКИ И ВЫCШЕГО ОБРАЗОВАНИЯ \\РОCCИЙCКОЙ ФЕДЕРАЦИИ
			\bigbreak
			ФЕДЕРАЛЬНОЕ ГОCУДАРCТВЕННОЕ БЮДЖЕТНОЕ ОБРАЗОВАТЕЛЬНОЕ УЧРЕЖДЕНИЕ ВЫCШЕГО ОБРАЗОВАНИЯ \\
			\bigbreak
			\textbf{«БЕЛГОРОДCКИЙ ГОCУДАРCТВЕННЫЙ \\ТЕХНОЛОГИЧЕCКИЙ УНИВЕРCИТЕТ им. В. Г. ШУХОВА»\\ (БГТУ им. В.Г. Шухова)} \\
			\bigbreak
			Кафедра программного обеспечения вычислительной техники и автоматизированных систем\\}
	\end{center}
	
	\vfill
	\begin{center}
		\large{
			\textbf{
				Лабораторная работа №1.3 }}\\
		\normalsize{
			по дисциплине: Дискретная математика \\
			тема: «Теоретико-множественные тождества»}
	\end{center}
	\vfill
	\hfill\textbox{
		Выполнил: ст. группы ПВ-223\\Пахомов Владислав Андреевич
		\bigbreak
		Проверили: \\ст. пр. Рязанов Юрий Дмитриевич\\
		ст. пр. Бондаренко Татьяна Владимировна
	}
	\vfill\begin{center}
		Белгород 2023 г.
	\end{center}
	\newpage
	\begin{center}
		\textbf{Лабораторная работа №1.3}\\
		Теоретико-множественные тождества\\
		Вариант 10
	\end{center}
	\textbf{Цель работы: }изучить методы доказательства теоретико-множественных тождеств.
	\begin{enumerate}[№1. ]
		\item На рис. \ref{ris:image} изображены круги Эйлера, соответствующие множествам А, В и С, с пронумерованными элементарными областями (не	содержащими внутри себя других областей). Заштриховать элементарные области в соответствии с вариантом задания (в соответствии с вариантом №10, необходимо заштриховать области 2, 3, 4).\\
		 \begin{figure}[h]
		 	\center{\includegraphics[width=70mm]{/task1}}
		    \caption{Круги Эйлера, соответствующие множествам А, В и С
		 		с пронумерованными элементарными областями}
		 	\label{ris:image}
		 \end{figure}
		 \item Написать выражение 1 над множествами А, В и С, определяющее заштрихованную область, используя операции пересечения, объединения и дополнения.\\
		 \begin{equation}F=\overline{A}\cap B\cap\overline{C}\cup A\cap B\cap\overline{C}\cup\overline{A}\cap\overline{B}\cap C\end{equation}
		 \item Используя свойства операций над множествами, преобразовать выражение 1 в выражение 2, не содержащее операции дополнения множества.\\
		 $F=\overline{A}\cap B\cap \overline{C}\cup A\cap B\cap \overline{C}\cup \overline{A}\cap \overline{B}\cap C=B\cap \overline{C}\cup \overline{A}\cap \overline{B}\cap C=(B-C)\cup(C-\overline{\overline{A}\cap \overline{B}})=(B-C)\cup(C-(A\cup B))$

		 
		 \begin{equation}F=(B-C)\cup(C-(A\cup B))\end{equation}
		 \item Используя свойства операций над множествами, преобразовать выражение 2 в выражение 3, не содержащее операции объединения множеств.\\
		 $F=(B-C)\cup(C-(A\cup B))=(B-C) \cup (C - A) \cap (C - B) = ((B - C) \cup (C - B)) \cap ((B - C) \cup (C - A)) = (B ∆ C)\cap ((B - C) \cup (C - A)) = (B ∆ C)\cap (((B - C) - (C - A))  ∆ (C - A))$
		 \begin{equation}F=(B ∆ C)\cap (((B - C) - (C - A))  ∆ (C - A))\end{equation}
		 \item Используя свойства операций над множествами, преобразовать выражение 3 в выражение 4, не содержащее операции пересечения множеств.\\
		 $F=(B ∆ C)\cap (((B - C) - (C - A))  ∆ (C - A))=(B ∆ C) - ((B ∆ C) - (((B - C) - (C - A))  ∆ (C - A)))$
		 \begin{equation}F=(B ∆ C) - ((B ∆ C) - (((B - C) - (C - A))  ∆ (C - A)))\end{equation}
	 		
	 		\item Доказать тождественность выражений 2 и 3 методом характеристических функций.\bigbreak
		 Используем метод арифметических характеристических функций\\
		 $(B-C)\cup(C-(A\cup B)) =(B ∆ C) - ((B ∆ C) - (((B - C) - (C - A))  ∆ (C - A))$\\
		 В дальнейших вычислениях для упрощения введём обозначения:\\
		 $a = X_A(x)$\\
		 $b = X_B(x)$\\
		 $c = X_C(x)$\bigbreak
		 Преобразуем левую часть выражения\\
		 $X_{(B-C)\cup(C-(A\cup B)}(x)=X_{B-C}(x)+X_{C-(A\cup B)}(x)-X_{B-C}(x)\cdot X_{C-(A\cup B)}(x)=
		 (X_B(x) - X_B(x) \cdot X_C(x)) +
		 (X_C(x) - X_C(x) \cdot X_{A\cup B}(x)) -
		 (X_B(x) - X_B(x) \cdot X_C(x)) \cdot
		 (X_C(x) - X_C(x) \cdot X_{A\cup B}(x))=
		 (X_B(x) - X_B(x) \cdot X_C(x)) + (X_C(x) - X_C(x) \cdot (X_A(x) + X_B(x) - X_A(x) \cdot X_B(x))) - (X_B(x) - X_B(x) \cdot X_C(x)) \cdot (X_C(x) - X_C(x) \cdot (X_A(x) + X_B(x) - X_A(x) \cdot X_B(x)))=
		 (b - b  c) + (c - c  (a + b - a  b)) - (b - b  c)  (c - c  (a + b - a  b)) = 
		 b + c + bc^2 + cb^2 - ac - b^2c^2 - 3bc + ab^2c^2 - abc^2 - acb^2 + 2abc =
		 b + c + bc + cb - ac - bc - 3bc + abc - abc - acb + 2abc=
		 b + c - ac - 2bc + abc=X_B(x)+X_C(x)-X_A(x) \cdot X_C(x) - 2 \cdot X_B(x) \cdot X_C(x) + X_A(x) \cdot X_B(x) \cdot X_C(x)
		 $\\
		 Преобразуем правую часть выражения\\
		 $ X_{(B ∆ C)\cap (((B - C) - (C - A))  ∆ (C - A))}(x) = 
		 X_{B ∆ C}(x) \cdot X_{((B - C) - (C - A))  ∆ (C - A)}(x)=
		 (X_B(x) + X_C(x) - 2 \cdot X_B(x) \cdot X_C(x)) \cdot  (X_{(B - C) - (C - A)}(x) + X_{C - A}(x) - 2 \cdot X_{(B - C) - (C - A)}(x) \cdot X_{C - A}(x))=
		 (X_B(x) + X_C(x) - 2 \cdot X_B(x) \cdot X_C(x)) \cdot  (    X_{B-C}(x) - X_{B-C}(x) \cdot X_{C-A}   +        X_C(x) -  X_C(x) \cdot X_A(x)              - 2 \cdot (X_{B-C}(x) - X_{B-C}(x) \cdot X_{C-A}) \cdot (X_C(x) -  X_C(x) \cdot X_A(x)))=
		 (X_B(x) + X_C(x) - 2 \cdot X_B(x) \cdot X_C(x)) \cdot  (    X_B(x) - X_B(x) \cdot X_C(x)  - (X_B(x) - X_B(x) \cdot X_C(x)) \cdot (X_C(x) - X_C(x) \cdot X_A(x))   +        X_C(x) -  X_C(x) \cdot X_A(x)              - 2 \cdot (X_B(x) - X_B(x) \cdot X_C(x)  - (X_B(x) - X_B(x) \cdot X_C(x)) \cdot (X_C(x) - X_C(x) \cdot X_A(x)) ) \cdot (X_C(x) -  X_C(x) \cdot X_A(x)))=(b + c - 2  b  c)   (    b - b  c  - (b - b  c)  (c - c  a)   +        c -  c  a              - 2  (b - b  c  - (b - b  c)  (c - c  a) )  (c -  c  a))=(b + c - 2  b  c)   (    b - b  c    +  c -  c  a ))=b-bc+bc-abc+bc-bc+c-ac-2bc+2bc-2bc+2abc=b+c-ac+abc-2bc=X_B(x)+X_C(x)-X_A(x) \cdot X_C(x) - 2 \cdot X_B(x) \cdot X_C(x) + X_A(x) \cdot X_B(x) \cdot X_C(x)$\bigbreak
		 АХФ совпали, значит $F_2$ и $F_3$ тождестенны
		 \newpage
\item Доказать тождественность выражений 2 и 4 методом  логических функций. Для автоматизации доказательства написать программу, которая получает и сравнивает таблицы истинности логических функций.\\

$(B-C)\cup(C-(A\cup B)) = (B ∆ C)\cap (((B - C) - (C - A))  ∆ (C - A))$\\
Преобразуем левую часть выражения\\
$\lambda_{(B-C)\cup(C-(A\cup B)}(x)=(\lambda_B(x)\wedge \overline{\lambda_C(x)})\vee(\lambda_C(x) \wedge \overline{\lambda_A(x) \vee \lambda_B(x)})
$\\
Преобразуем правую часть выражения\\
$\lambda_{(B ∆ C)\cap (((B - C) - (C - A))  ∆ (C - A))}(x) = (\lambda_B(x) \bigoplus \lambda_C(x)) \wedge (((\lambda_B(x)\wedge \overline{\lambda_C(x)}) \wedge \overline{\lambda_C(x)\wedge \overline{\lambda_A(x)}})$\\ 
$\bigoplus (\lambda_C(x)\wedge \overline{\lambda_A(x)}))$\bigbreak

Получили ЛХФ, на их основе составили две функции на языке программирования C++, функция F2 соответствует функции $F_2$, функция F4 соответствует функции $F_4$. Параметры $a, b, c$ в обоих функциях соответственно равны $\lambda_A(x), \lambda_B(x), \lambda_C(x)$\\
Операторам $\wedge, \vee, \bigoplus, \overline{A}$ в формуле в программе соответствуют логическое "И" (\&\&), логическое "ИЛИ" (||), исключающее "ИЛИ" (\^) и "НЕ" (!)\\
Полный код программы:\\
\begin{minted}[frame=lines, framesep=2mm, baselinestretch=1.2, bgcolor=LightGray, breaklines, fontsize=\footnotesize]{C}
#include <iostream>

bool F2(bool a, bool b, bool c) {
	return (b && !c) || (c && !(a || b));
}

bool F4(bool a, bool b, bool c) {
	return (b ^ c) && (((b && !c) && !(c && !a)) ^ (c && !a));
}

int main() {
	std::cout << "A B C F2 F4" << std::endl;
	for (int bA = 0; bA <= 1; bA++)
	for (int bB = 0; bB <= 1; bB++)
	for (int bC = 0; bC <= 1; bC++) {
		std::cout << bA << " " << bB << " " << bC << " " << F2(bA, bB, bC) << "  " << F4(bA, bB, bC) << std::endl;
	}
	return 0;
}
\end{minted}

Результат выполнения программы:\\
\includegraphics[width=140mm]{/wr1}\\
Таблицы истинности совпали, следовательно $F_2$ и $F_4$ тождественны.

\item Доказать тождественность выражений 3 и 4 теоретико-множественным методом. Для автоматизации доказательства написать программу, в которой вычисляются и сравниваются значения выражений 3 и 4 при А=\{1,3,5,7\}, B=\{2,3,6,7\} и C=\{4,5,6,7\}.\\
Используя функции, полученные в л.р. № 1.1 получим класс Sett:
\begin{minted}[frame=lines, framesep=2mm, baselinestretch=1.2, bgcolor=LightGray, breaklines, fontsize=\footnotesize]{C}
class Sett {
	public:
	std::vector<int> elements;
	
	Sett(std::vector<int> elms) {
		elements = elms;
		std::sort(elements.begin(), elements.end());
	}
	
	~Sett() {
	}
	
	Sett operator*(Sett anotherSet) {
		std::vector<int> arrayC(0, 0);
		int arrayASize = elements.size();
		int arrayBSize = anotherSet.elements.size();
		size_t i = 0, j = 0;
		
		while (i < arrayASize && j < arrayBSize)
		if (elements[i] < anotherSet.elements[j])
		i++;
		else if (elements[i] > anotherSet.elements[j])
		j++;
		else {
			arrayC.push_back(elements[i]);
			i++;
			j++;
		}
		
		return Sett(arrayC);
	}
	
	Sett operator-(Sett anotherSet) {
		std::vector<int> arrayC(0, 0);
		int arrayASize = elements.size();
		int arrayBSize = anotherSet.elements.size();
		size_t i = 0, j = 0;
		
		while (i < arrayASize && j < arrayBSize)
		if (elements[i] < anotherSet.elements[j])
		arrayC.push_back(elements[i++]);
		else if (elements[i] > anotherSet.elements[j])
		j++;
		else {
			i++;
			j++;
		}
		
		while (i < arrayASize)
		arrayC.push_back(elements[i++]);
		
		return Sett(arrayC);
	}
	
	Sett operator+(Sett anotherSet) {
		std::vector<int> arrayC(0, 0);
		int arrayASize = elements.size();
		int arrayBSize = anotherSet.elements.size();
		size_t i = 0, j = 0;
		
		while (i < arrayASize && j < arrayBSize)
		if (elements[i] < anotherSet.elements[j])
		arrayC.push_back(elements[i++]);
		else if (elements[i] > anotherSet.elements[j])
		arrayC.push_back(anotherSet.elements[j++]);
		else {
			arrayC.push_back(elements[i]);
			i++;
			j++;
		}
		
		while (i < arrayASize)
		arrayC.push_back(elements[i++]);
		
		while (j < arrayBSize)
		arrayC.push_back(anotherSet.elements[j++]);
		
		return Sett(arrayC);
	}
	
	Sett non(Sett universum) {
		std::vector<int> arrayC(0, 0);
		int arraySize = elements.size();
		int universumSize = universum.elements.size();
		size_t i = 0, j = 0;
		// Проверяем, что универсум действительно универсум
		assert(elements[arraySize - 1] <= universum.elements[universumSize - 1]);
		
		while (i < universumSize && j < arraySize) {
			if (universum.elements[i] < elements[j])
			arrayC.push_back(universum.elements[i++]);
			else if (universum.elements[i] == elements[j]) {
				i++;
				j++;
				// вторым его отличием будет то, что если элемент есть в A и его нет в universum, программа будет падать
			} else
			assert(elements[j] >= universum.elements[i]);
		}
		
		while (i < universumSize)
		arrayC.push_back(universum.elements[i++]);
		
		return Sett(arrayC);
	}
	
	Sett operator^(Sett anotherSet) {
		std::vector<int> arrayC(0, 0);
		int arrayASize = elements.size();
		int arrayBSize = anotherSet.elements.size();
		size_t i = 0, j = 0;
		
		while (i < arrayASize && j < arrayBSize)
		if (elements[i] < anotherSet.elements[j])
		arrayC.push_back(elements[i++]);
		else if (elements[i] > anotherSet.elements[j])
		arrayC.push_back(anotherSet.elements[j++]);
		else {
			j++;
			i++;
		}
		
		while (i < arrayASize)
		arrayC.push_back(elements[i++]);
		
		while (j < arrayBSize)
		arrayC.push_back(anotherSet.elements[j++]);
		
		return Sett(arrayC);
	}
	
	void print() {
		for (int i = 0; i < elements.size(); i++) {
			std::cout << elements[i] << " ";
		}
		
		std::cout << std::endl;
	}
};
\end{minted}
В котором переопределены следующие операторы и находятся следующие методы:\\
$\^$ – симметрическая разность\\
$*$ - пересечение\\
$+$ - объединение\\
$non$ – дополнение\\
$-$ - разница\\
$print$ – метод, выводящий множество\\
Итоговая программа выглядит так:\\
\begin{minted}[frame=lines, framesep=2mm, baselinestretch=1.2, bgcolor=LightGray, breaklines, fontsize=\footnotesize]{C}
int main() {
	Sett A(std::vector<int>({1, 3, 5, 7}));
	Sett B(std::vector<int>({2, 3, 6, 7}));
	Sett C(std::vector<int>({4, 5, 6, 7}));
	Sett U(std::vector<int>({1, 2, 3, 4, 5, 6, 7}));
	
	Sett F3 = ((B - C) ^ (C - ((A ^ B) ^ (A * B)))) ^ ((B - C) * (C - ((A ^ B) ^ (A * B))));
	Sett F4 = ((B - C) ^ (C - ((A ^ B) ^ (A - (A - B))))) ^ ((B - C) - ((B - C) - (C - ((A ^ B) ^ (A - (A - B))))));
	
	F3.print();
	F4.print();
	
	return 0;
}
\end{minted}

Результат выполнения программы:\\
\includegraphics[width=140mm]{/wr2}\\
Результаты вычисления выражений равны, следовательно $F_3$ и $F_4$ тождественны.
	\end{enumerate}
			\textbf{Вывод: } в ходе лабораторной работы изучили методы доказательства \\теоретико-множественных тождеств.
\end{document}