\documentclass[a4paper,14pt]{extarticle}


\usepackage[english,russian]{babel}
\usepackage[T2A]{fontenc}
\usepackage[utf8]{inputenc}
\usepackage{ragged2e}
\usepackage[utf8]{inputenc}
\usepackage{hyperref}
\usepackage{minted}
\usepackage{xcolor}
\definecolor{LightGray}{gray}{0.9}
\usepackage{graphicx}
\usepackage[export]{adjustbox}
\usepackage[left=1cm,right=1cm, top=1cm,bottom=1cm,bindingoffset=0cm]{geometry}
\usepackage{fontspec}
\usepackage{ upgreek }
\usepackage[shortlabels]{enumitem}
\usepackage[mathletters]{ucs}
\usepackage{adjustbox}
\usepackage{multirow}
\usepackage{amsmath}
\usepackage{pifont}
\graphicspath{ {./images/} }
\makeatletter
\AddEnumerateCounter{\asbuk}{\russian@alph}{щ}
\makeatother
\setmonofont{Consolas}
\setmainfont{Times New Roman}

\newcommand\textbox[1]{
	\parbox{.45\textwidth}{#1}
}

\newcommand{\xmark}{\ding{55}}

\newcommand{\specialcell}[2][c]{%
	\begin{tabular}[#1]{@{}c@{}}#2\end{tabular}}

\begin{document}	
	\pagenumbering{gobble}
		$\overline{A}-C=B-A-C$\\
		\includegraphics[width=70mm]{/q11}\\
		$A = \{1,3,5,7\}, B = \{2, 3, 6, 7\}, C = \{4, 5, 6, 7\}, U = \{0, 1, 2, 3, 4, 5, 6, 7\}$\\
		Вычислим левую часть тожедства\\
		$\overline{A}-C = \{0, 2, 4, 6\} - \{4, 5, 6, 7\}=\{0, 2\}$\\
		Вычислим правую часть тожедства\\
		$B-A-C = \{2, 3, 6, 7\} - \{1,3,5,7\}-\{4, 5, 6, 7\}=\{2, 6\} - \{4, 5, 6, 7\} = \{2\}$\\
		Воспользуемся программой, полученной в 8 пункте л.р. 1.3, для вычисления частей тождества:\\
	\begin{minted}[frame=lines, framesep=2mm, baselinestretch=1.2, bgcolor=LightGray, breaklines, fontsize=\footnotesize]{C}
int main() {
	Sett A(std::vector<int>({1, 3, 5, 7}));
	Sett B(std::vector<int>({2, 3, 6, 7}));
	Sett C(std::vector<int>({4, 5, 6, 7}));
	Sett U(std::vector<int>({0, 1, 2, 3, 4, 5, 6, 7}));
	
	Sett left = A.non(U) - C;
	Sett right = B - A - C;
	
	left.print();
	right.print();
	
	return 0;
}
	\end{minted}
Результат работы программы:\\
\includegraphics[width=140mm]{/q12}\\
	Вывод: результаты вычисления правой и левой части тождества не совпали при автоматизированном и ручном способе вычисления, тождество не верно.
\end{document}