\documentclass[a4paper,14pt]{extarticle}


\usepackage[english,russian]{babel}
\usepackage[T2A]{fontenc}
\usepackage[utf8]{inputenc}
\usepackage{ragged2e}
\usepackage[utf8]{inputenc}
\usepackage{hyperref}
\usepackage{minted}
\setmintedinline{frame=lines, framesep=2mm, baselinestretch=1.5, bgcolor=LightGray, breaklines,fontsize=\scriptsize}
\setminted{frame=lines, framesep=2mm, baselinestretch=1.5, bgcolor=LightGray, breaklines,fontsize=\scriptsize}
\usepackage{xcolor}
\definecolor{LightGray}{gray}{0.9}
\usepackage{graphicx}
\usepackage[export]{adjustbox}
\usepackage[left=1cm,right=1cm, top=1cm,bottom=1cm,bindingoffset=0cm]{geometry}
\usepackage{fontspec}
\usepackage{ upgreek }
\usepackage[shortlabels]{enumitem}
\usepackage{adjustbox}
\usepackage{multirow}
\usepackage{amsmath}
\usepackage{amssymb}
\usepackage{pifont}
\usepackage{pgfplots}
\usepackage{longtable}
\usepackage{array}
\graphicspath{ {./images/} }
\makeatletter
\AddEnumerateCounter{\asbuk}{\russian@alph}{щ}
\makeatother
\setmonofont{Consolas}
\setmainfont{Times New Roman}

\newcommand\textbox[1]{
	\parbox{.45\textwidth}{#1}
} 

\newcommand{\specialcell}[2][c]{%
	\begin{tabular}[#1]{@{}c@{}}#2\end{tabular}}

\begin{document}
\pagenumbering{gobble}
\begin{center}
    \small{
        \textbf{МИНИCТЕРCТВО НАУКИ И ВЫCШЕГО ОБРАЗОВАНИЯ РОCCИЙCКОЙ ФЕДЕРАЦИИ}\\
        ФЕДЕРАЛЬНОЕ ГОCУДАРCТВЕННОЕ БЮДЖЕТНОЕ ОБРАЗОВАТЕЛЬНОЕ УЧРЕЖДЕНИЕ\\ВЫCШЕГО ОБРАЗОВАНИЯ \\
        \textbf{«БЕЛГОРОДCКИЙ ГОCУДАРCТВЕННЫЙ ТЕХНОЛОГИЧЕCКИЙ\\УНИВЕРCИТЕТ им. В. Г. ШУХОВА»\\ (БГТУ им. В.Г. Шухова)} \\
        \bigbreak
        \includegraphics[width=70mm]{log}\\
        ИНСТИТУТ ИНФОРМАЦИОННЫХ ТЕХНОЛОГИЙ И УПРАВЛЯЮЩИХ СИСТЕМ\\}
\end{center}

\vfill
\begin{center}
    \large{
        \textbf{
            Лабораторная работа №4.5}}\\
    \normalsize{
        по дисциплине: Дискретная математика \\
        тема: «Кратчайшие пути между каждой парой вершин
        во взвешенном орграфе»}
\end{center}
\vfill
\hfill\textbox{
    Выполнил: ст. группы ПВ-223\\Пахомов Владислав Андреевич
    \bigbreak
    Проверили: \\ст. пр. Рязанов Юрий Дмитриевич\\
    ст. пр. Бондаренко Татьяна Владимировна
}
\vfill\begin{center}
    Белгород 2023 г.
\end{center}
\newpage
\begin{center}
    \textbf{Лабораторная работа №4.5}\\
    Кратчайшие пути между каждой парой вершин во взвешенном орграфе\\
    Вариант 10
\end{center}
\textbf{Цель работы: }изучить алгоритмы нахождения кратчайших путей
между каждой парой вершин во взвешенном орграфе,
научиться использовать их при решении различных
задач.

\begin{enumerate}[1.]
    \item Изучить алгоритмы нахождения кратчайших путей между 
    каждой парой вершин во взвешенном орграфе.\\
    В данной работе для реализации был выбран алгоритм Флойда.
        \begin{minted}{C++}
template <typename E, typename EdgeValueType = typename E::NameType>
using ShortWay = std::pair<EdgeValueType, int>;
        \end{minted}
        \begin{minted}{C++}
// Алгоритм Флойда
std::vector<std::vector<ShortWay<E>>> getShortestWayMatrix() {
    std::vector<std::vector<ShortWay<E>>> W(this->nodes.size(), 
        std::vector<ShortWay<E>>(this->nodes.size()));

    // Формирование M
    for (int i = 0; i < this->nodes.size(); i++) 
        for (int j = 0; j < this->nodes.size(); j++) {
            auto edge = getShortestEdge(i, j);
            if (edge == nullptr) 
                W[i][j] = {std::numeric_limits<EdgeValueType>::max(), -1};
            else 
                W[i][j] = {edge->name, (i == j ? 0 : i)};
        } 

    for (int z = 0; z < this->nodes.size(); z++) 
        for (int x = 0; x < this->nodes.size(); x++) {
            if (W[x][z].second == -1) continue;

            for (int y = 0; y < this->nodes.size(); y++) {
                if (W[z][y].second == -1) continue;

                EdgeValueType product = W[x][z].first + W[z][y].first;
                if (product < W[x][y].first) {
                    W[x][y].first = product;
                    W[x][y].second = W[z][y].second;
                }
            }
        }
    
    return W;
}

E* getShortestEdge(int i, int j) {
    if (this->edges[i][j] == nullptr) return nullptr;

    int minEdgeIndex = 0;
    for (int k = 0; k < (*this->edges[i][j]).size(); k++) 
        if ((*this->edges[i][j])[minEdgeIndex]->current.name > (*this->edges[i][j])[k]->current.name) minEdgeIndex = k;

    return &(*this->edges[i][j])[minEdgeIndex]->current;
}
        \end{minted}
        \item Разработать и реализовать алгоритм решения задачи.\bigbreak
        Во взвешенном орграфе найти все пары вершин $v_i$ и $v_j$, такие,
что кратчайшее расстояние от $v_i$ до $v_j$ меньше кратчайшего расстояния
от $v_j$ до $v_i$. Вывести кратчайшие пути между найденными парами вершин.\\
        \begin{minted}{C++}
template<typename G, typename W>
void printPath( G& graph, W shortestWayMatrix, 
int begin, int end, bool isEnd) {
    if (begin == -1 || end == -1) return;

    if (begin != end)
        printPath(graph, shortestWayMatrix, begin, shortestWayMatrix[begin][end].second, false);

    std::cout << "(" << graph.nodes[end]->name << ")";
    if (!isEnd)
        std::cout << " ----> ";
}

template<typename T>
void analyzeTree(T& g, std::string graphName) {
    std::cout << "Graph " << graphName << ":\n";
    auto shortestWayMatrix = g.getShortestWayMatrix();
    bool anyFound = false;
    
    for (int i = 0; i < g.nodes.size(); i++) {
        for (int j = i + 1; j < g.nodes.size(); j++) {
            if (j == i) continue;

            // Пути должны существовать
            if (shortestWayMatrix[i][j].second == -1 || shortestWayMatrix[j][i].second == -1) continue;

            // Длина путей должна быть равна
            if (shortestWayMatrix[i][j].first == shortestWayMatrix[j][i].first) continue;

            bool shouldSwap = false;
            if (shortestWayMatrix[i][j].first > shortestWayMatrix[j][i].first)  {
                shouldSwap = true;
                std::swap(i, j);
            }
        
            anyFound = true;
            std::cout << "Found paths with weights of " << shortestWayMatrix[i][j].first << " and " << shortestWayMatrix[j][i].first << ":\n";
            printPath(g, shortestWayMatrix, i, j, true);
            std::cout << "\n";
            printPath(g, shortestWayMatrix, j, i, true);
            std::cout << "\n" << std::endl;

            if (shouldSwap)
                std::swap(i, j);      
        }
    }

    if (!anyFound) 
        std::cout << "No paths found" << std::endl;
}
                \end{minted}
        \item Подобрать тестовые данные. Результат представить в виде диаграммы графа.\\
        Вывод в консоль:
                    \begin{minted}{console}
Graph Worm:
Found paths with weights of 1 and 3:
(1) ----> (2)
(2) ----> (1)

Graph Chain:
No paths found
Graph Hard chain:
Found paths with weights of 3 and 5:
(1) ----> (2) ----> (3) ----> (4)  
(4) ----> (3) ----> (2) ----> (1)  

Found paths with weights of 2 and 4:
(2) ----> (3) ----> (4)
(4) ----> (3) ----> (2)

Found paths with weights of 1 and 3:
(3) ----> (4)
(4) ----> (3)

Graph Four sides:
Found paths with weights of 1 and 4:
(1) ----> (2)
(2) ----> (3) ----> (4) ----> (1)  

Found paths with weights of 2 and 3:
(1) ----> (2) ----> (3)
(3) ----> (4) ----> (1)

Found paths with weights of 2 and 3:
(4) ----> (1)
(1) ----> (2) ----> (3) ----> (4)  

Found paths with weights of 1 and 4:
(2) ----> (3)
(3) ----> (4) ----> (1) ----> (2)  

Found paths with weights of 2 and 3:
(2) ----> (3) ----> (4)
(4) ----> (1) ----> (2)

Found paths with weights of 1 and 4:
(3) ----> (4)
(4) ----> (1) ----> (2) ----> (3)
                    \end{minted}
        
        \textbf{Червяк}\\
        \includegraphics[width=100mm]{testWorm}\\
        В данном графе будет только один путь, удовлетворящий условию:\\
        1 -> 2 | длина = 1\\
        2 -> 1 | длина = 3\\

        Результат выполнения программы совпал с предположениями.\\

        \textbf{Цепочка}\\
        \includegraphics[width=100mm]{testChain}\\
        В данном графе нет таких путей.\\

        Результат выполнения программы совпал с предположениями.\\

        \textbf{Тяжёлая цепочка}\\
        \includegraphics[width=100mm]{testHardChain}\\
        В данном графе такие пути уже составить можно:\\
        1 -> 2 -> 3 -> 4 | длина = 3\\
        4 -> 3 -> 2 -> 1 | длина = 5\bigbreak

        2 -> 3 -> 4 | длина = 2\\
        4 -> 3 -> 2 | длина = 4\bigbreak

        3 -> 4 | длина = 1\\
        4 -> 3 | длина = 3\\

        Результат выполнения программы совпал с предположениями.\\

        \textbf{Четырёхугольник}\\
        \includegraphics[width=100mm]{testFourSides}\\
        В данном графе можно составить следующие пути, удовлетворящие условию:\\
        1 -> 2 | длина = 1\\
        2 -> 3 -> 4 -> 1 | длина = 4\bigbreak

        1 -> 2 -> 3 | длина = 1\\
        3 -> 4 -> 1 | длина = 2\bigbreak

        4 -> 1 | длина = 2\\
        1 -> 2 -> 3 -> 4 | длина = 3\bigbreak

        2 -> 3 | длина = 1\\
        3 -> 4 -> 1 -> 2 | длина = 4\bigbreak

        2 -> 3 -> 4 | длина = 2\\
        4 -> 1 -> 2 | длина = 3\bigbreak

        3 -> 4 | длина = 1\\
        4 -> 1 -> 2 -> 3 | длина = 4\bigbreak

        Результат выполнения программы совпал с предположениями.\\

        Программа:
        \begin{minted}{C++}
#include "../../libs/alg/alg.h"

template<typename G, typename W>
void printPath( G& graph, W shortestWayMatrix, 
int begin, int end, bool isEnd) {
    if (begin == -1 || end == -1) return;

    if (begin != end)
        printPath(graph, shortestWayMatrix, begin, shortestWayMatrix[begin][end].second, false);

    std::cout << "(" << graph.nodes[end]->name << ")";
    if (!isEnd)
        std::cout << " ----> ";
}

template<typename T>
void analyzeTree(T& g, std::string graphName) {
    std::cout << "Graph " << graphName << ":\n";
    auto shortestWayMatrix = g.getShortestWayMatrix();
    bool anyFound = false;
    
    for (int i = 0; i < g.nodes.size(); i++) {
        for (int j = i + 1; j < g.nodes.size(); j++) {
            if (j == i) continue;

            // Пути должны существовать
            if (shortestWayMatrix[i][j].second == -1 || shortestWayMatrix[j][i].second == -1) continue;

            // Длина путей должна быть равна
            if (shortestWayMatrix[i][j].first == shortestWayMatrix[j][i].first) continue;

            bool shouldSwap = false;
            if (shortestWayMatrix[i][j].first > shortestWayMatrix[j][i].first)  {
                shouldSwap = true;
                std::swap(i, j);
            }
        
            anyFound = true;
            std::cout << "Found paths with weights of " << shortestWayMatrix[i][j].first << " and " << shortestWayMatrix[j][i].first << ":\n";
            printPath(g, shortestWayMatrix, i, j, true);
            std::cout << "\n";
            printPath(g, shortestWayMatrix, j, i, true);
            std::cout << "\n" << std::endl;

            if (shouldSwap)
                std::swap(i, j);      
        }
    }

    if (!anyFound) 
        std::cout << "No paths found" << std::endl;
}

// Червяк
void testWorm() {
    AdjacencyMatrixGraph<NamedEdge<Node<int>, unsigned long long>> g;

    Node<int> N1(1);
    Node<int> N2(2);

    g.addNode(N1);
    g.addNode(N2);

    g.addEdge({{&N1, &N2}, 1, true});
    g.addEdge({{&N2, &N1}, 3, true});

    analyzeTree(g, "Worm");
}

// Цепочка
void testChain() {
    AdjacencyMatrixGraph<NamedEdge<Node<int>, unsigned long long>> g;

    Node<int> N1(1);
    Node<int> N2(2);
    Node<int> N3(3);
    Node<int> N4(4);

    g.addNode(N1);
    g.addNode(N2);
    g.addNode(N3);
    g.addNode(N4);

    g.addEdge({{&N1, &N2}, 1, true});
    g.addEdge({{&N2, &N1}, 1, true});
    g.addEdge({{&N3, &N2}, 1, true});
    g.addEdge({{&N2, &N3}, 1, true});
    g.addEdge({{&N3, &N4}, 1, true});
    g.addEdge({{&N4, &N3}, 1, true});
    analyzeTree(g, "Chain");
}

// Сложная цепочка
void testHardChain() {
    AdjacencyMatrixGraph<NamedEdge<Node<int>, unsigned long long>> g;

    Node<int> N1(1);
    Node<int> N2(2);
    Node<int> N3(3);
    Node<int> N4(4);

    g.addNode(N1);
    g.addNode(N2);
    g.addNode(N3);
    g.addNode(N4);

    g.addEdge({{&N1, &N2}, 1, true});
    g.addEdge({{&N2, &N1}, 1, true});
    g.addEdge({{&N3, &N2}, 1, true});
    g.addEdge({{&N2, &N3}, 1, true});
    g.addEdge({{&N3, &N4}, 1, true});
    g.addEdge({{&N4, &N3}, 3, true});
    analyzeTree(g, "Hard chain");
}

// Четырёхугольник
void testFourSides() {
    AdjacencyMatrixGraph<NamedEdge<Node<int>, unsigned long long>> g;

    Node<int> N1(1);
    Node<int> N2(2);
    Node<int> N3(3);
    Node<int> N4(4);

    g.addNode(N1);
    g.addNode(N2);
    g.addNode(N3);
    g.addNode(N4);
    
    g.addEdge({{&N1, &N2}, 1, true});
    g.addEdge({{&N2, &N3}, 1, true});
    g.addEdge({{&N3, &N4}, 1, true});
    g.addEdge({{&N4, &N1}, 2, true});

    analyzeTree(g, "Four sides");
}

void test() {
    testWorm();
    testChain();
    testHardChain();
    testFourSides();
}

int main() {
    test();
}
                    \end{minted}
                    
        \end{enumerate}

\textbf{Вывод: } в ходе лабораторной работы изучили алгоритмы нахождения кратчайших путей
между каждой парой вершин во взвешенном орграфе,
научиться использовать их при решении различных
задач.

\end{document}