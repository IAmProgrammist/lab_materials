\documentclass[a4paper,14pt]{extarticle}


\usepackage[english,russian]{babel}
\usepackage[T2A]{fontenc}
\usepackage[utf8]{inputenc}
\usepackage{ragged2e}
\usepackage[utf8]{inputenc}
\usepackage{hyperref}
\usepackage{minted}
\usepackage{xcolor}
\definecolor{LightGray}{gray}{0.9}
\usepackage{graphicx}
\usepackage[export]{adjustbox}
\usepackage[left=1cm,right=1cm, top=1cm,bottom=1cm,bindingoffset=0cm]{geometry}
\usepackage{fontspec}
\usepackage{ upgreek }
\usepackage[shortlabels]{enumitem}
\usepackage[mathletters]{ucs}
\usepackage{adjustbox}
\usepackage{multirow}
\usepackage{amsmath}
\usepackage{amssymb}
\usepackage{pifont}
\graphicspath{ {./images/} }
\makeatletter
\AddEnumerateCounter{\asbuk}{\russian@alph}{щ}
\makeatother
\setmonofont{Consolas}
\setmainfont{Times New Roman}

\newcommand\textbox[1]{
	\parbox{.45\textwidth}{#1}
}

\newcommand{\specialcell}[2][c]{%
	\begin{tabular}[#1]{@{}c@{}}#2\end{tabular}}

\begin{document}	
	\pagenumbering{gobble}
	\noindent Мощность множества всех трёхэлементных подмножеств множества А равно 10.
	
	\begin{enumerate}[№1. ]
		\item Чему равна мощность множества А?\bigbreak
		"Множества всех трёхэлементных подмножеств множества" описывает комбинаторный объект "сочетания" \, где k = 3. Количество элементов в сочетании определяется по следующей формуле:\\
		$C^k_n=\frac{n!}{(n-k)!k!}$\\
		Подставим в формулу условие задания $k=3; C^k_n = 10$. Получим уравнение, где n - искомая мощность исходного множества A\\
		$10=\frac{n!}{(n-3)!3!}$\\
		$\frac{n!}{(n-3)!}=60$\\
		$\frac{1\cdot2\cdot3\cdot...\cdot(n-4)\cdot(n-3)\cdot(n-2)\cdot(n-1)\cdot n}{1\cdot2\cdot3\cdot...\cdot(n-4)\cdot(n-3)}=60$\\
		$(n-2)\cdot(n-1)\cdot n = 60$\\
		При n = 5\\
		$3\cdot4\cdot5=60$\\
		$12\cdot5=60$\\
		$60=60$\\
		Равенство доказано, следовательно мощность множества A n = 5
		
		\textbf{Ответ:} 5
		
		\item Чему равна мощность множества всех размещений множества А по 3-м местам?
		
		Формула для вычисления количества размещений\\
		$A^k_n=\frac{n!}{(n-k)!}$\\
		$A^3_5=\frac{5!}{(5-3)!}=\frac{1\cdot2\cdot3\cdot4\cdot5}{1\cdot2}=3\cdot4\cdot5=60$\\
		
		\textbf{Ответ:} 60
		
	\end{enumerate}
	
	
\end{document}