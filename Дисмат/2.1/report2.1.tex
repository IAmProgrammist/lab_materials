\documentclass[a4paper,14pt]{extarticle}


\usepackage[english,russian]{babel}
\usepackage[T2A]{fontenc}
\usepackage[utf8]{inputenc}
\usepackage{ragged2e}
\usepackage[utf8]{inputenc}
\usepackage{hyperref}
\usepackage{minted}
\usepackage{xcolor}
\definecolor{LightGray}{gray}{0.9}
\usepackage{graphicx}
\usepackage[export]{adjustbox}
\usepackage[left=1cm,right=1cm, top=1cm,bottom=1cm,bindingoffset=0cm]{geometry}
\usepackage{fontspec}
\usepackage{ upgreek }
\usepackage[shortlabels]{enumitem}
\usepackage[mathletters]{ucs}
\usepackage{adjustbox}
\usepackage{multirow}
\usepackage{amsmath}
\usepackage{amssymb}
\usepackage{pifont}
\graphicspath{ {./images/} }
\makeatletter
\AddEnumerateCounter{\asbuk}{\russian@alph}{щ}
\makeatother
\setmonofont{Consolas}
\setmainfont{Times New Roman}

\newcommand\textbox[1]{
	\parbox{.45\textwidth}{#1}
}

\newcommand{\specialcell}[2][c]{%
	\begin{tabular}[#1]{@{}c@{}}#2\end{tabular}}

\begin{document}	
	\pagenumbering{gobble}
	\begin{center}
		\small{
			МИНИCТЕРCТВО НАУКИ И ВЫCШЕГО ОБРАЗОВАНИЯ \\РОCCИЙCКОЙ ФЕДЕРАЦИИ
			\bigbreak
			ФЕДЕРАЛЬНОЕ ГОCУДАРCТВЕННОЕ БЮДЖЕТНОЕ ОБРАЗОВАТЕЛЬНОЕ УЧРЕЖДЕНИЕ ВЫCШЕГО ОБРАЗОВАНИЯ \\
			\bigbreak
			\textbf{«БЕЛГОРОДCКИЙ ГОCУДАРCТВЕННЫЙ \\ТЕХНОЛОГИЧЕCКИЙ УНИВЕРCИТЕТ им. В. Г. ШУХОВА»\\ (БГТУ им. В.Г. Шухова)} \\
			\bigbreak
			Кафедра программного обеспечения вычислительной техники и автоматизированных систем\\}
	\end{center}
	
	\vfill
	\begin{center}
		\large{
			\textbf{
				Лабораторная работа №2.1}}\\
		\normalsize{
			по дисциплине: Дискретная математика \\
			тема: «Алгоритмы порождения комбинаторных объектов»}
	\end{center}
	\vfill
	\hfill\textbox{
		Выполнил: ст. группы ПВ-223\\Пахомов Владислав Андреевич
		\bigbreak
		Проверили: \\ст. пр. Рязанов Юрий Дмитриевич\\
		ст. пр. Бондаренко Татьяна Владимировна
	}
	\vfill\begin{center}
		Белгород 2023 г.
	\end{center}
	\newpage
	\begin{center}
		\textbf{Лабораторная работа №2.1}\\
		Алгоритмы порождения комбинаторных объектов
	\end{center}
	\textbf{Цель работы: }изучить основные комбинаторные объекты, алгоритмы их порождения, программно реализовать и оценить временную сложность алгоритмов.
	\begin{enumerate}[№1. ]
		\item Реализовать алгоритм порождения подмножеств.\bigbreak
\begin{minted}[frame=lines, framesep=2mm, baselinestretch=1.2, bgcolor=LightGray, breaklines,fontsize=\footnotesize]{C++}
template <typename T>
std::vector<std::vector<T>> getSubsets(std::vector<T>& baseSet, std::vector<T> currentSet,
size_t count) {
	// Если мы достигли конца исходного множества, возвращаем текущее множество
	if (count == baseSet.size())
		return {currentSet};
	
	// Добавляем в массив подмножеств шаг, если мы не добавляем на текущем шаге новый элемент
	std::vector<std::vector<T>> resultSubsets = getSubsets(baseSet, currentSet, count + 1);
	
	// Добавляем в текущее подмножество элемент массива
	currentSet.push_back(baseSet[count]);
	
	// Добавляем в массив подмножеств шаг, если мы добавляем на текущем шаге новый элемент
	std::vector<std::vector<T>> branch = getSubsets(baseSet, currentSet, count + 1);
	resultSubsets.insert(std::end(resultSubsets), std::begin(branch), std::end(branch));
	
	return resultSubsets;
}
\end{minted}
Мы реализовали рекуррентный алгоритм порождения подмножеств. Функция getSubsets возвращает массив всех подмножеств исходного множества baseSet
\item Построить график зависимости количества всех подмножеств от мощности множества.\bigbreak
\begin{center}
\begin{tabular}{ cc} 
	\hline
	0&1\\
	\hline
	1&2\\
	\hline
	2&4\\
	\hline
	3&8\\
	\hline
	4&16\\
	\hline
	5&32\\
	\hline
	6&64\\
	\hline
	7&128\\
	\hline
	8&256\\
	\hline
	9&512\\
	\hline
	10&1024\\
	\hline
	11&2048\\
	\hline
	12&4096\\
	\hline
	13&8192 \\
	\hline
	14&16384\\
	\hline
	15&32768\\
	\hline
	16&65536\\
	\hline
	17&131072\\
	\hline
	18&262144\\
	\hline
	19&524288\\
	\hline
	20&1048576\\
	\hline
	21&2097152\\
	\hline
	22&4194304\\
	\hline
	23&8388608\\
	\hline
	24&16777216\\
	\hline
	25&33554432\\
	\hline
	26&67108864\\
	\hline
\end{tabular}\\
\includegraphics[width=140mm]{/task2}\\
\end{center}
\item Построить графики зависимости времени выполнения алгоритмов п.1 на вашей ЭВМ от мощности множества.\bigbreak
\begin{center}
\begin{tabular}{ cc} 
	\hline
	0&0\\
	\hline
	1&0\\
	\hline
	2&0\\
	\hline
	3&0\\
	\hline
	4&0\\
	\hline
	5&0\\
	\hline
	6&0\\
	\hline
	7&0\\
	\hline
	8&0\\
	\hline
	9&0\\
	\hline
	10&0,001\\
	\hline
	11&0,005\\
	\hline
	12&0,003\\
	\hline
	13&0,007 \\
	\hline
	14&0,01\\
	\hline
	15&0,02\\
	\hline
	16&0,038\\
	\hline
	17&0,079\\
	\hline
	18&0,164\\
	\hline
	19&0,348\\
	\hline
	20&0,705\\
	\hline
	21&1,494\\
	\hline
	22&3,121\\
	\hline
	23&6,5\\
	\hline
	24&13,623\\
	\hline
	25&29,204\\
	\hline
	26&64,215\\
	\hline
\end{tabular}\\
\includegraphics[width=140mm]{/task3}\\
\end{center}
\item Определить максимальную мощность множества, для которого можно получить все подмножества не более чем за час, сутки, месяц, год на вашей ЭВМ.\bigbreak
Воспользовавшись методом наименьших квадратов для функции $t = a \cdot 2^N + b$ на основе полученных точек получили обобщенную аппроксимацию $a = 9.35885430168442\cdot10^{-7}, b = -0.225015460816618$\\
$t = 9.35885430168442\cdot10^{-7} \cdot 2^N - 0.225015460816618$\bigbreak
\begin{tabular}{ c|c} 
	
	\hline
	t&N\\
	\hline
	3600 с. = 1 час&28.519 = 28\\
	\hline
	86400 с. = 1 сутки&33.104 = 33\\
	\hline
	2678400 с. = 1 месяц (31 день)&38.058 = 38\\
	\hline
	31536000 с. = 1 год (365 дней)&41.616 = 41\\
	\hline
\end{tabular}

\item Определить максимальную мощность множества, для которого можно получить все подмножества не более чем за час, сутки, месяц, год на ЭВМ, в 10 и в 100 раз быстрее вашей.\bigbreak
$t_\textup{(ЭВМ мощнее в 10 раз)} = \frac{9.35885430168442\cdot10^{-7} \cdot 2^N - 0.225015460816618}{10}$\\
$t_\textup{(ЭВМ мощнее в 100 раз)} = \frac{9.35885430168442\cdot10^{-7} \cdot 2^N - 0.225015460816618}{100}$\bigbreak

\begin{tabular}{ c|c|c} 
	
	\hline
	t&$N_\textup{(ЭВМ мощнее в 10 раз)}$&$N_\textup{(ЭВМ мощнее в 100 раз)}$\\
	\hline
	3600 с. = 1 час&31.841 = 31 & 35.163 = 35\\
	\hline
	86400 с. = 1 сутки&36.426 = 36 & 39.748 = 39\\
	\hline
	2678400 с. = 1 месяц (31 день)&41.38 = 31 & 44.702 = 44\\
	\hline
	31536000 с. = 1 год (365 дней)&44.938 = 44 & 48.26 = 48\\
	\hline
\end{tabular}

\item Реализовать алгоритм порождения сочетаний.\bigbreak
\begin{minted}[frame=lines, framesep=2mm, baselinestretch=1.2, bgcolor=LightGray, breaklines,fontsize=\footnotesize]{C++}
template<typename T>
std::vector<std::vector<T>> getCombinations(std::vector<T> &baseSet, std::vector<T> currentSet, size_t minIndex, size_t k, size_t count) {
	std::vector<std::vector<T>> resultCombs;
	
	// Если количество перестановок равно необходимому, мы достигли искомого множества, возвращаем его
	if (count >= k)
		return {currentSet};
	
	for (size_t i = minIndex; i <= baseSet.size() - k + count; i++) {
		// Добавляем в текущее множество новый элемент, Ci = x
		std::vector<T> newCurrentSet(currentSet);
		newCurrentSet.push_back(baseSet[i]);
		
		// Вызываем следующий шаг итерации, сохраняем его результат в общий массив множеств
		std::vector<std::vector<T>> combinations = getCombinations(baseSet, newCurrentSet, i + 1, k, count + 1);
		resultCombs.insert(std::begin(resultCombs), std::begin(combinations), std::end(combinations));
	}
	
	// Возвращаем массив множеств
	return resultCombs;
}
\end{minted}
Мы реализовали рекуррентный алгоритм порождения сочетаний. Функция getCombinations возвращает массив всех сочетаний исходного множества baseSet
\item Построить графики зависимости количества всех сочетаний из n по k от k при n= (5, 7, 9).\bigbreak
\textbf{n = 5}\\
\begin{center}
\begin{tabular}{c|c} 
	
	\hline
	k&Кол-во сочетаний\\
	\hline
	0&1\\
	\hline
	1&5\\
	\hline
	2&10\\
	\hline
	3&10\\
	\hline
	4&5\\
	\hline
	5&1\\
	\hline
\end{tabular}\\
\includegraphics[width=120mm]{/task6a}\bigbreak
\end{center}

\textbf{n = 7}\\
\begin{center}
\begin{tabular}{c|c} 
	
	\hline
	k&Кол-во сочетаний\\
	\hline
	0&1\\
	\hline
	1&7\\
	\hline
	2&21\\
	\hline
	3&35\\
	\hline
	4&35\\
	\hline
	5&21\\
	\hline
	6&7\\
	\hline
	7&1\\
	\hline
\end{tabular}\\
\includegraphics[width=120mm]{/task6b}\bigbreak
\end{center}

\textbf{n = 9}\\
\begin{center}
\begin{tabular}{c|c} 
	
	\hline
	k&Кол-во сочетаний\\
	\hline
	0&1\\
	\hline
	1&9\\
	\hline
	2&36\\
	\hline
	3&84\\
	\hline
	4&126\\
	\hline
	5&126\\
	\hline
	6&84\\
	\hline
	7&36\\
	\hline
	8&9\\
	\hline
	9&1\\
	\hline
\end{tabular}\\
\includegraphics[width=120mm]{/task6c}
\end{center}
\bigbreak
\item Реализовать алгоритм порождения перестановок.\bigbreak
\begin{minted}[frame=lines, framesep=2mm, baselinestretch=1.2, bgcolor=LightGray, breaklines,fontsize=\footnotesize]{C++}
template<typename T>
std::vector<std::vector<T>> getPermutations(std::vector<T> baseSet, std::vector<T> currentSet) {
	std::vector<std::vector<T>> resultPerms;
	
	// Если элементов в изначальном множестве не осталось, получено искомое множество
	if (baseSet.size() == 0) return {currentSet};
	
	for (size_t i = 0; i < baseSet.size(); i++) {
		// Удаляем из исходного массива x
		std::vector<T> newBaseSet(baseSet);
		newBaseSet.erase(std::begin(newBaseSet) + i);
		
		// Добавляем в текущее множество новый элемент
		std::vector<T> newCurrentSet(currentSet);
		newCurrentSet.push_back(baseSet[i]);
		
		// Выполняем следующий шаг итерации, сохраняем в итоговый массив множеств
		auto permutations = getPermutations(newBaseSet, newCurrentSet);
		resultPerms.insert(std::begin(resultPerms), std::begin(permutations), std::end(permutations));
	}
	
	return resultPerms;
}
\end{minted}
Мы реализовали рекуррентный алгоритм порождения перестановок. Функция getPermutations возвращает массив всех перестановок исходного множества baseSet
\item Построить график зависимости количества всех перестановок от мощности множества.\bigbreak
\begin{center}
\begin{tabular}{ cc} 
	\hline
	0&1\\
	\hline
	1&1\\
	\hline
	2&2\\
	\hline
	3&6\\
	\hline
	4&24\\
	\hline
	5&120\\
	\hline
	6&720\\
	\hline
	7&5040\\
	\hline
	8&40320\\
	\hline
	9&362880\\
	\hline
	10&3628800\\
	\hline
	11&39916800\\
	\hline
\end{tabular}\\
\includegraphics[width=140mm]{/task9}\\
\end{center}
\item Построить графики зависимости времени выполнения алгоритма п.8 на вашей ЭВМ от мощности множества.\bigbreak
\begin{center}
	\begin{tabular}{ cc} 
		\hline
		0&0\\
		\hline
		1&0\\
		\hline
		2&0\\
		\hline
		3&0\\
		\hline
		4&0\\
		\hline
		5&0\\
		\hline
		6&0,003\\
		\hline
		7&0,016\\
		\hline
		8&0,111\\
		\hline
		9&1,043\\
		\hline
		10&11,39\\
		\hline
		11&137,71\\
		\hline
	\end{tabular}\\
	\includegraphics[width=140mm]{/task10}\\
\end{center}
\item Определить максимальную мощность множества, для которого можно получить все перестановки не более чем за час, сутки, месяц, год на вашей ЭВМ.\bigbreak
Воспользовавшись методом наименьших квадратов для функции $t = a \cdot N!$ на основе полученных точек получили обобщенную аппроксимацию $a = 3.4473277768732325\cdot10^{-6}$\\
$t = 3.4473277768732325\cdot10^{-6} \cdot N!$\bigbreak
\begin{tabular}{ c|c} 
	
	\hline
	t&N\\
	\hline
	3600 с. = 1 час&12.307 = 12\\
	\hline
	86400 с. = 1 сутки&13.531 = 13\\
	\hline
	2678400 с. = 1 месяц (31 день)&14.81 = 14\\
	\hline
	31536000 с. = 1 год (365 дней)&15.704 = 15\\
	\hline
\end{tabular}

\item Определить максимальную мощность множества, для которого можно получить все подмножества не более чем за час, сутки, месяц, год на ЭВМ, в 10 и в 100 раз быстрее вашей.\bigbreak
$t_\textup{(ЭВМ мощнее в 10 раз)} = \frac{3.4473277768732325\cdot10^{-6} \cdot N!}{10}$\\
$t_\textup{(ЭВМ мощнее в 100 раз)} = \frac{3.4473277768732325\cdot10^{-6} \cdot N!}{100}$\bigbreak

\begin{tabular}{ c|c|c} 
	
	\hline
	t&$N_\textup{(ЭВМ мощнее в 10 раз)}$&$N_\textup{(ЭВМ мощнее в 100 раз)}$\\
	\hline
	3600 с. = 1 час&13.198 = 13 & 14.067 = 14\\
	\hline
	86400 с. = 1 сутки&14.393 = 14 & 15.237 = 15\\
	\hline
	2678400 с. = 1 месяц (31 день)&15.645 = 15 & 16.466 = 16\\
	\hline
	31536000 с. = 1 год (365 дней)&16.523 = 16 & 17.329 = 17\\
	\hline
\end{tabular}
\item Реализовать алгоритм порождения размещений.\bigbreak
\begin{minted}[frame=lines, framesep=2mm, baselinestretch=1.2, bgcolor=LightGray, breaklines,fontsize=\footnotesize]{C++}
template<typename T>
std::vector<std::vector<T>> getPlacements(std::vector<T> baseSet, std::vector<T> currentSet, size_t k) {
	// k не может быть больше размера исходного множества
	k = std::min(baseSet.size(), k);
	
	// Если k = 0, то количество перестановок закончилось или закончился исходный массив. В обоих случаях мы достигли
	// искомого множества, возвращаем его.
	if (k == 0) return {currentSet};
	
	std::vector<std::vector<T>> resultPerms;
	
	for (size_t i = 0; i < baseSet.size(); i++) {
		// Удаляем из исходного массива x
		std::vector<T> newBaseSet(baseSet);
		newBaseSet.erase(std::begin(newBaseSet) + i);
		
		// Добавляем в текущее множество новый элемент
		std::vector<T> newCurrentSet(currentSet);
		newCurrentSet.push_back(baseSet[i]);
		
		// Выполняем следующий шаг итерации, сохраняем в итоговый массив множеств
		auto permutations = getPlacements(newBaseSet, newCurrentSet, k - 1);
		resultPerms.insert(std::begin(resultPerms), std::begin(permutations), std::end(permutations));
	}
	
	return resultPerms;
}
\end{minted}
Мы реализовали рекуррентный алгоритм порождения размещекний. Функция getPlacements возвращает массив всех размещений исходного множества baseSet
\item Построить графики зависимости количества всех сочетаний из n по k от k при n= (5, 7, 9).\bigbreak
\textbf{n = 5}\\
\begin{center}
	\begin{tabular}{c|c} 
		
		\hline
		k&Кол-во сочетаний\\
		\hline
		0&1\\
		\hline
		1&5\\
		\hline
		2&20\\
		\hline
		3&60\\
		\hline
		4&120\\
		\hline
		5&120\\
		\hline
	\end{tabular}\\
	\includegraphics[width=120mm]{/task14a}\bigbreak
\end{center}

\textbf{n = 7}\\
\begin{center}
	\begin{tabular}{c|c} 
		
		\hline
		k&Кол-во сочетаний\\
		\hline
		0&1\\
		\hline
		1&7\\
		\hline
		2&42\\
		\hline
		3&210\\
		\hline
		4&840\\
		\hline
		5&2520\\
		\hline
		6&5040\\
		\hline
		7&5040\\
		\hline
	\end{tabular}\\
	\includegraphics[width=120mm]{/task14b}\bigbreak
\end{center}

\textbf{n = 9}\\
\begin{center}
	\begin{tabular}{c|c} 
		
		\hline
		k&Кол-во сочетаний\\
		\hline
		0&1\\
		\hline
		1&9\\
		\hline
		2&72\\
		\hline
		3&504\\
		\hline
		4&3024\\
		\hline
		5&15120\\
		\hline
		6&60480\\
		\hline
		7&181440\\
		\hline
		8&362880\\
		\hline
		9&362880\\
		\hline
	\end{tabular}\\
	\includegraphics[width=120mm]{/task14c}
\end{center}
\bigbreak
\end{enumerate}
			\textbf{Вывод: } в ходе лабораторной работы изучили основные комбинаторные объекты, алгоритмы их порождения, программно реализовали и оценили временную сложность алгоритмов.
			
\end{document}