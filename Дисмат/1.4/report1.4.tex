\documentclass[a4paper,14pt]{extarticle}


\usepackage[english,russian]{babel}
\usepackage[T2A]{fontenc}
\usepackage[utf8]{inputenc}
\usepackage{ragged2e}
\usepackage[utf8]{inputenc}
\usepackage{hyperref}
\usepackage{minted}
\usepackage{xcolor}
\definecolor{LightGray}{gray}{0.9}
\usepackage{graphicx}
\usepackage[export]{adjustbox}
\usepackage[left=1cm,right=1cm, top=1cm,bottom=1cm,bindingoffset=0cm]{geometry}
\usepackage{fontspec}
\usepackage{ upgreek }
\usepackage[shortlabels]{enumitem}
\usepackage[mathletters]{ucs}
\usepackage{adjustbox}
\usepackage{multirow}
\usepackage{amsmath}
\usepackage{amssymb}
\usepackage{pifont}
\graphicspath{ {./images/} }
\makeatletter
\AddEnumerateCounter{\asbuk}{\russian@alph}{щ}
\makeatother
\setmonofont{Consolas}
\setmainfont{Times New Roman}

\newcommand\textbox[1]{
	\parbox{.45\textwidth}{#1}
}

\newcommand{\specialcell}[2][c]{%
	\begin{tabular}[#1]{@{}c@{}}#2\end{tabular}}

\begin{document}	
	\pagenumbering{gobble}
	\begin{center}
		\small{
			МИНИCТЕРCТВО НАУКИ И ВЫCШЕГО ОБРАЗОВАНИЯ \\РОCCИЙCКОЙ ФЕДЕРАЦИИ
			\bigbreak
			ФЕДЕРАЛЬНОЕ ГОCУДАРCТВЕННОЕ БЮДЖЕТНОЕ ОБРАЗОВАТЕЛЬНОЕ УЧРЕЖДЕНИЕ ВЫCШЕГО ОБРАЗОВАНИЯ \\
			\bigbreak
			\textbf{«БЕЛГОРОДCКИЙ ГОCУДАРCТВЕННЫЙ \\ТЕХНОЛОГИЧЕCКИЙ УНИВЕРCИТЕТ им. В. Г. ШУХОВА»\\ (БГТУ им. В.Г. Шухова)} \\
			\bigbreak
			Кафедра программного обеспечения вычислительной техники и автоматизированных систем\\}
	\end{center}
	
	\vfill
	\begin{center}
		\large{
			\textbf{
				Лабораторная работа №1.4}}\\
		\normalsize{
			по дисциплине: Дискретная математика \\
			тема: «Теоретико-множественные тождества»}
	\end{center}
	\vfill
	\hfill\textbox{
		Выполнил: ст. группы ПВ-223\\Пахомов Владислав Андреевич
		\bigbreak
		Проверили: \\ст. пр. Рязанов Юрий Дмитриевич\\
		ст. пр. Бондаренко Татьяна Владимировна
	}
	\vfill\begin{center}
		Белгород 2023 г.
	\end{center}
	\newpage
	\begin{center}
		\textbf{Лабораторная работа №1.4}\\
		Теоретико-множественные тождества\\
		Вариант 10
	\end{center}
	\textbf{Цель работы: }научиться решать теоретико-множественные уравнения с применением ЭВМ.
	\begin{enumerate}[№1. ]
		\item Преобразовать исходное уравнение $A\cap(B \cup X)∆C = A- (B\cap \overline{X})∆C \cap X$ в уравнение с пустой правой частью.\bigbreak
		$(A\cap(B \cup X)∆C) ∆ (A- (B\cap \overline{X})∆C \cap X) =\varnothing$
		\item Преобразовать левую часть уравнения к виду $\overline{X}\cap \varphi ^ \varnothing \cup X \cap \varphi ^ U$, используя разложение Шеннона по неизвестному множеству X.\bigbreak
		$\varphi ^ \varnothing = (A\cap(B \cup \varnothing)∆C) ∆ (A- (B\cap \overline{\varnothing})∆C \cap \varnothing)$\\
		$\varphi ^ U = (A\cap(B \cup U)∆C) ∆ (A- (B\cap \overline{U})∆C \cap U)$\\
		$\overline{X}\cap ((A\cap(B \cup \varnothing)∆C) ∆ (A- (B\cap \overline{\varnothing})∆C \cap \varnothing)) \cup X \cap ((A\cap(B \cup U)∆C) ∆ (A- (B\cap \overline{U})∆C \cap U)) = \varnothing$
		\item Написать программу, вычисляющую значения множеств $\varphi ^ \varnothing$ и $\overline{\varphi ^ U}$ при заданных исходных множествах. $U = \{1, 2, 3, 4, 5, 6, 7, 8, 9, 10\}, A = \{4, 5, 7, 8, 9, 10\}, B = \{2, 3, 4, 5, 6, 10\}, C = \{4, 5, 7, 8, 10\}$\bigbreak
		Для вычисления значений воспользуемся классом, полученным в л.р. №3, внеcя в него некоторые правки
\begin{minted}[frame=lines, framesep=2mm, baselinestretch=1.2, bgcolor=LightGray, breaklines,fontsize=\footnotesize]{C++}
#include <vector>
#include <iostream>

template<class T>
class Sett {
	public:
	std::vector<T> elements;
	
	Sett<T>(std::vector<T> elms) {
		elements = elms;
		std::sort(elements.begin(), elements.end());
	}
	
	~Sett() {
		
	}
	
	Sett<T> operator*(Sett<T> anotherSet) {
		std::vector<T> arrayC(0, 0);
		int arrayASize = elements.size();
		int arrayBSize = anotherSet.elements.size();
		size_t i = 0, j = 0;
		
		while (i < arrayASize && j < arrayBSize)
			if (elements[i] < anotherSet.elements[j])
				i++;
			else if (elements[i] > anotherSet.elements[j])
				j++;
			else {
				arrayC.push_back(elements[i]);
				i++;
				j++;
			}
		
		return Sett<T>(arrayC);
	}
	
	Sett<T> operator-(Sett<T> anotherSet) {
		std::vector<T> arrayC(0, 0);
		int arrayASize = elements.size();
		int arrayBSize = anotherSet.elements.size();
		size_t i = 0, j = 0;
		
		while (i < arrayASize && j < arrayBSize)
			if (elements[i] < anotherSet.elements[j])
				arrayC.push_back(elements[i++]);
			else if (elements[i] > anotherSet.elements[j])
				j++;
			else {
				i++;
				j++;
			}
		
		while (i < arrayASize)
			arrayC.push_back(elements[i++]);
		
		return Sett<T>(arrayC);
	}
	
	Sett<T> operator+(Sett<T> anotherSet) {
		std::vector<int> arrayC(0, 0);
		int arrayASize = elements.size();
		int arrayBSize = anotherSet.elements.size();
		size_t i = 0, j = 0;
		
		while (i < arrayASize && j < arrayBSize)
			if (elements[i] < anotherSet.elements[j])
				arrayC.push_back(elements[i++]);
			else if (elements[i] > anotherSet.elements[j])
				arrayC.push_back(anotherSet.elements[j++]);
			else {
				arrayC.push_back(elements[i]);
				i++;
				j++;
			}
		
		while (i < arrayASize)
			arrayC.push_back(elements[i++]);
		
		while (j < arrayBSize)
			arrayC.push_back(anotherSet.elements[j++]);
		
		return Sett<T>(arrayC);
	}
	
	Sett<T> non(Sett<T> universum) {
		std::vector<T> arrayC(0, 0);
		int arraySize = elements.size();
		int universumSize = universum.elements.size();
		size_t i = 0, j = 0;
		// Проверяем, что универсум действительно универсум
		assert(arraySize == 0 || elements[arraySize - 1] <= universum.elements[universumSize - 1]);
		
		while (i < universumSize && j < arraySize) {
			if (universum.elements[i] < elements[j])
				arrayC.push_back(universum.elements[i++]);
			else if (universum.elements[i] == elements[j]) {
				i++;
				j++;
				// вторым его отличием будет то, что если элемент есть в A и его нет в universum, программа будет падать
			} else
				assert(elements[j] >= universum.elements[i]);
		}
		
		while (i < universumSize)
			arrayC.push_back(universum.elements[i++]);
		
		return Sett<T>(arrayC);
	}
	
	Sett<T> operator^(Sett<T> anotherSet) {
		std::vector<T> arrayC(0, 0);
		int arrayASize = elements.size();
		int arrayBSize = anotherSet.elements.size();
		size_t i = 0, j = 0;
		
		while (i < arrayASize && j < arrayBSize)
			if (elements[i] < anotherSet.elements[j])
				arrayC.push_back(elements[i++]);
			else if (elements[i] > anotherSet.elements[j])
				arrayC.push_back(anotherSet.elements[j++]);
			else {
				j++;
				i++;
			}
		
		while (i < arrayASize)
			arrayC.push_back(elements[i++]);
		
		while (j < arrayBSize)
			arrayC.push_back(anotherSet.elements[j++]);
		
		return Sett<T>(arrayC);
	}
	
	bool operator==(Sett<T> other) {
		return elements == other.elements;
	}
	
	bool operator!=(Sett<T> other) {
		return !(*this == other);
	}
	
	void print() {
		for (int i = 0; i < elements.size(); i++) {
			std::cout << elements[i] << " ";
		}
		
		std::cout << std::endl;
	}
	
	void print(std::ostream &out) {
		for (int i = 0; i < elements.size(); i++) {
			out << elements[i] << " ";
		}
		
		out << std::endl;
	}
};
\end{minted}
Получили программу:\\
\begin{minted}[frame=lines, framesep=2mm, baselinestretch=1.2, bgcolor=LightGray, breaklines,fontsize=\footnotesize]{C++}
#include "../../libs/alg/alg.h"
	
int main() {
	Sett<int> nothing(std::vector<int>({}));
	Sett<int> U(std::vector<int>({1, 2, 3, 4, 5, 6, 7, 8, 9, 10}));
	Sett<int> A(std::vector<int>({4, 5, 7, 8, 9, 10}));
	Sett<int> B(std::vector<int>({2, 3, 4, 5, 6, 10}));
	Sett<int> C(std::vector<int>({4, 5, 7, 8, 10}));
		
	Sett<int> phiNothing = ((A * (B + nothing)) ^ C) ^ (A - ((B * nothing.non(U)) ^ (C * nothing)));
	Sett<int> nonPhiU = (((A * (B + U)) ^ C) ^ (A - ((B * U.non(U)) ^ (C * U)))).non(U);
	
	phiNothing.print();
	nonPhiU.print();
}
\end{minted}
\item Вычислить значения множеств $\varphi ^ \varnothing$ и $\overline{\varphi ^ U}$ и сделать вывод о существовании решения уравнения. Если решения уравнения не существует, то выполнить п.п. 1—4 для следующего (предыдущего) варианта.\bigbreak
Результат выполнения программы:\\
\includegraphics[width=140mm]{/task3}\\
Множество $\varphi ^ \varnothing = \{9\}$ является подмножеством множества\\ $\overline{\varphi ^ U} = \{1, 2, 3, 4, 5, 6, 7, 8, 9, 10\}$, следовательно, уравнение имеет решения.
\item Определить мощность общего решения, найти некоторые (или все) частные решения, в том числе частные решения наименьшей и наибольшей мощности.\bigbreak
$\overline{\varphi ^ U} - \varphi ^ \varnothing  = \{1, 2, 3, 4, 5, 6, 7, 8, 10\}$\\
Количество возможных подмножеств слишком велико, мощность общего решения будет равна количеству подмножеств множества $\overline{\varphi ^ U} - \varphi ^ \varnothing$, вычислим его по формуле $2^{|\overline{\varphi ^ U} - \varphi ^ \varnothing|}$\\
$2^{|\overline{\varphi ^ U} - \varphi ^ \varnothing|} = 2^9 = 512$\\
Приведём 8 возможных решений уравнения:\\
$X=\{9\}$ - решение наименьшей мощности\\
$X=\{1, 2, 3, 4, 5, 6, 7, 8, 9, 10\}$ - решение наибольшей мощности\\
$X=\{1, 5, 6, 9\}$\\
$X=\{1, 2, 3, 4, 5, 9, 10\}$\\
$X=\{7, 8, 9, 10\}$\\
$X=\{1, 2, 5, 6, 7, 9\}$\\
$X=\{2, 3, 7, 8, 9, 10\}$\\
$X=\{3, 4, 5, 6, 8, 9, 10\}$\\
\item Написать программу для проверки найденных решений.
\begin{minted}[frame=lines, framesep=2mm, baselinestretch=1.2, bgcolor=LightGray, breaklines,fontsize=\footnotesize]{C++}
#include "../../libs/alg/alg.h"

Sett<int> U(std::vector<int>({1, 2, 3, 4, 5, 6, 7, 8, 9, 10}));
Sett<int> A(std::vector<int>({4, 5, 7, 8, 9, 10}));
Sett<int> B(std::vector<int>({2, 3, 4, 5, 6, 10}));
Sett<int> C(std::vector<int>({4, 5, 7, 8, 10}));

// Функция для вычисления левой части уравнения
template<typename T>
Sett<T> left(Sett<T> X) {
	return A * (B + X) ^ C;
}

// Функция для вычисления правой части уравнения
template<typename T>
Sett<T> right(Sett<T> X) {
	return A - (B * X.non(U)) ^ (C * X);
}

int main() {
	// Решения уравнения
	std::vector<Sett<int>> decisions = {
		Sett<int>({9}),
		Sett<int>({1, 2, 3, 4, 5, 6, 7, 8, 9, 10}),
		Sett<int>({1, 5, 6, 9}),
		Sett<int>({1, 2, 3, 4, 5, 9, 10}),
		Sett<int>({7, 8, 9, 10}),
		Sett<int>({1, 2, 5, 6, 7, 9}),
		Sett<int>({2, 3, 7, 8, 9, 10}),
		Sett<int>({3, 4, 5, 6, 8, 9, 10})
	};
	
	for (auto &x: decisions) 
		// Проверяем, что левая и правая части уравнений равны
		if (left(x) != right(x)) {
			std::cerr << "Множество" << std::endl;
			x.print(std::cerr);
			std::cerr << "не является решением уравнения" << std::endl;
			return 1;
		}
	
	std::cout << "Решения подходят для уравнения" << std::endl;
	
	return 0;
}
\end{minted}
Результат выполнения программы:\\
\includegraphics[width=140mm]{/task6}\\
\end{enumerate}
			\textbf{Вывод: } в ходе лабораторной работы научились решать теоретико-множественные уравнения с применением ЭВМ.
\end{document}