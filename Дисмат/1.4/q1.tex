\documentclass[a4paper,14pt]{extarticle}


\usepackage[english,russian]{babel}
\usepackage[T2A]{fontenc}
\usepackage[utf8]{inputenc}
\usepackage{ragged2e}
\usepackage[utf8]{inputenc}
\usepackage{hyperref}
\usepackage{minted}
\usepackage{xcolor}
\definecolor{LightGray}{gray}{0.9}
\usepackage{graphicx}
\usepackage[export]{adjustbox}
\usepackage[left=1cm,right=1cm, top=1cm,bottom=1cm,bindingoffset=0cm]{geometry}
\usepackage{fontspec}
\usepackage{ upgreek }
\usepackage[shortlabels]{enumitem}
\usepackage[mathletters]{ucs}
\usepackage{adjustbox}
\usepackage{multirow}
\usepackage{amsmath}
\usepackage{amssymb}
\usepackage{pifont}
\graphicspath{ {./images/} }
\makeatletter
\AddEnumerateCounter{\asbuk}{\russian@alph}{щ}
\makeatother
\setmonofont{Consolas}
\setmainfont{Times New Roman}

\newcommand\textbox[1]{
	\parbox{.45\textwidth}{#1}
}

\newcommand{\specialcell}[2][c]{%
	\begin{tabular}[#1]{@{}c@{}}#2\end{tabular}}

\begin{document}	
	\pagenumbering{gobble}
	\noindent$A – X = X– A$\\
	$U=\{1,2,3,4,5\}$\\
	$A=\{1,2\}$\bigbreak
	\noindent Преобразуем уравнение в уравнение с пустой правой частью\\
	$(A-X)∆(X-A)=\varnothing$\bigbreak
	\noindent Преобразуем левую часть уравнения к виду $\overline{X}\cap \varphi ^ \varnothing \cup X \cap \varphi ^ U$, используя разложение Шеннона по неизвестному множеству X и упростим полученные $\varphi ^ \varnothing$ и $\varphi ^ U$\bigbreak
	\noindent $\varphi ^ \varnothing = (A-\varnothing)∆(\varnothing - A) = A∆\varnothing = A$\\
	$\varphi ^ U = (A-U)∆(U - A) = \varnothing ∆ (U - A) = U - A$\bigbreak
\noindent	$\varphi ^ \varnothing = A = \{1, 2\}$\\
	$\overline{\varphi ^ U} = \overline{\{3, 4, 5\}} = \{1, 2\}$\bigbreak
	
	\noindent $\varphi ^ \varnothing$ является подмножеством множества $\overline{\varphi ^ U}$, следовательно решение уравнения существует.\\
	$\overline{\varphi ^ U} - \varphi ^ \varnothing = \{\}$\\
	Так как мощность выражения $\overline{\varphi ^ U} - \varphi ^ \varnothing$ равна нулю, общее решение будет состоять из одного множества: $\{\{1, 2\}\}$\bigbreak
	
	Ответ: $\{\{1, 2\}\}$
	
	
\end{document}