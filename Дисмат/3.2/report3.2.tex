\documentclass[a4paper,14pt]{extarticle}


\usepackage[english,russian]{babel}
\usepackage[T2A]{fontenc}
\usepackage[utf8]{inputenc}
\usepackage{ragged2e}
\usepackage[utf8]{inputenc}
\usepackage{hyperref}
\usepackage{minted}
\setmintedinline{frame=lines, framesep=2mm, baselinestretch=1.5, bgcolor=LightGray, breaklines,fontsize=\scriptsize}
\setminted{frame=lines, framesep=2mm, baselinestretch=1.5, bgcolor=LightGray, breaklines,fontsize=\scriptsize}
\usepackage{xcolor}
\definecolor{LightGray}{gray}{0.9}
\usepackage{graphicx}
\usepackage[export]{adjustbox}
\usepackage[left=1cm,right=1cm, top=1cm,bottom=1cm,bindingoffset=0cm]{geometry}
\usepackage{fontspec}
\usepackage{ upgreek }
\usepackage[shortlabels]{enumitem}
\usepackage{adjustbox}
\usepackage{multirow}
\usepackage{amsmath}
\usepackage{amssymb}
\usepackage{pifont}
\usepackage{pgfplots}
\usepackage{longtable}
\usepackage{array}
\graphicspath{ {./images/} }
\makeatletter
\AddEnumerateCounter{\asbuk}{\russian@alph}{щ}
\makeatother
\setmonofont{Consolas}
\setmainfont{Times New Roman}

\newcommand\textbox[1]{
	\parbox{.45\textwidth}{#1}
} 

\newcommand{\specialcell}[2][c]{%
	\begin{tabular}[#1]{@{}c@{}}#2\end{tabular}}

\begin{document}
\pagenumbering{gobble}
\begin{center}
	\small{
		МИНИCТЕРCТВО НАУКИ И ВЫCШЕГО ОБРАЗОВАНИЯ \\РОCCИЙCКОЙ ФЕДЕРАЦИИ
		\bigbreak
		ФЕДЕРАЛЬНОЕ ГОCУДАРCТВЕННОЕ БЮДЖЕТНОЕ ОБРАЗОВАТЕЛЬНОЕ УЧРЕЖДЕНИЕ ВЫCШЕГО ОБРАЗОВАНИЯ \\
		\bigbreak
		\textbf{«БЕЛГОРОДCКИЙ ГОCУДАРCТВЕННЫЙ \\ТЕХНОЛОГИЧЕCКИЙ УНИВЕРCИТЕТ им. В. Г. ШУХОВА»\\ (БГТУ им. В.Г. Шухова)} \\
		\bigbreak
		Кафедра программного обеспечения вычислительной техники и автоматизированных систем\\}
\end{center}

\vfill
\begin{center}
	\large{
		\textbf{
			Лабораторная работа №3.2}}\\
	\normalsize{
		по дисциплине: Дискретная математика \\
		тема: «Транзитивное замыкание отношения»}
\end{center}
\vfill
\hfill\textbox{
	Выполнил: ст. группы ПВ-223\\Пахомов Владислав Андреевич
	\bigbreak
	Проверили: \\ст. пр. Рязанов Юрий Дмитриевич\\
	ст. пр. Бондаренко Татьяна Владимировна
}
\vfill\begin{center}
	Белгород 2023 г.
\end{center}
\newpage
\begin{center}
	\textbf{Лабораторная работа №3.2}\\
	Транзитивное замыкание отношения
\end{center}
\textbf{Цель работы: }изучить и выполнить сравнительный анализ алгоритмов
вычисления транзитивного замыкания отношения.

\begin{enumerate}[1.]
	\item Изучить и программно реализовать алгоритмы объединения степеней
	      и Уоршалла для вычисления транзитивного замыкания отношения.\\
	      Используем раннее разработанный класс BoolMatrixRelation, добавим в него новые методы\\
	      \textit{alg.h}
	      \begin{minted}{C++}
BoolMatrixRelation transitiveClosurePowUnite(unsigned long long *steps = NULL);
BoolMatrixRelation transitiveClosureWarshall(unsigned long long *steps = NULL);
	\end{minted}
	      \textit{task1.cpp}
	      \begin{minted}{C++}
#include "../alg.h"

BoolMatrixRelation BoolMatrixRelation::transitiveClosurePowUnite(unsigned long long *steps)
{
    BoolMatrixRelation ctran = *this;
    BoolMatrixRelation c2 = ctran;

    for (int i = 1; i <= size; i++)
    {
        bool shouldRun = false;
        ctran = ctran.unite(c2);
        
        // Самым вложенным является цикл композиции, поэтому
        // вычислять композицию будем вручную, а не при помощи 
        // реализованного ранее метода pow
        c2 = BoolMatrixRelation(size, [&shouldRun, steps, &ctran](int x, int y)
                                {
        for (int z = 0; z < ctran.size; z++) {
            (*steps)++;
            if (ctran.data[x - 1][z] && ctran.data[z][y - 1]) {
                shouldRun = true;
                return true;
            }
        }

        return false; 
        });

        if (!shouldRun)
            break;
    }

    return ctran;
}

BoolMatrixRelation BoolMatrixRelation::transitiveClosureWarshall(unsigned long long *steps)
{
    BoolMatrixRelation res(size, [this](int x, int y)
                           { return data[x - 1][y - 1]; });

    for (int z = 0; z < size; z++)
    {
        for (int x = 0; x < size; x++)
        {
            if (!res.data[x][z]) continue;

            for (int y = 0; y < size; y++)
            {
                if (steps != NULL)
                    (*steps)++;

                res.data[x][y] = res.data[x][y] || res.data[x][z] && res.data[z][y];
            }
        }
    }

    return res;
}
	\end{minted}
	\item Разработать и программно реализовать генератор отношений на
	      множестве мощности N и содержащих заданное число пар.
	      \textit{alg.h}
	      \begin{minted}{C++}
static BoolMatrixRelation getRandom(int size, int pairs);
	\end{minted}
	      \textit{task2.cpp}
	      \begin{minted}{C++}
#include "../alg.h"

BoolMatrixRelation BoolMatrixRelation::getRandom(int size, int pairs) {
    if (pairs >= size * size) return BoolMatrixRelation (size, [](int x, int y){return true;});
    
    BoolMatrixRelation res(size, [](int x, int y){return false;});
    while (pairs > 0) {
        int rX = rand() % size, rY = rand() % size;

        if (!res.data[rX][rY]) {
            res.data[rX][rY] = true;
            pairs--;
        }
    }

    return res;
}
	\end{minted}
	\item Разработать и написать программу, которая генерирует
	      1000 отношений на множестве мощности N с заданным числом пар,
	      для каждого отношения вычисляет транзитивное замыкание двумя
	      алгоритмами и подсчитывает количество k выполнений тела самого
	      вложенного цикла. Значение k при обработке различных отношений на
	      множестве мощности N с заданным числом пар может быть разным,
	      поэтому программа должна определять минимальное и максимальное

	      значение k. Отношение, при обработке которого получено минимальное
	      (максимальное) k, и его транзитивное замыкание, сохранить в

	      файле. Выполнить программу при N = 5, 10 и 15. Результат для каждого
	      N представить в виде таблицы, а сохраненные отношения — в виде графа.\\
	      \textit{alg.h}
	      \begin{minted}{C++}
std::ostream& printAsAdjMatrix(std::ostream &out) {
    for (int i = 0; i < size; i++) {
        for (int j = 0; j < size; j++) {
            out << data[i][j] << ",";
        }

        out << "\n";
    }

    return out;
}
\end{minted}
	      \textit{main.cpp}
	      \begin{minted}{C++}
#include "../../libs/alg/alg.h"

#include <fstream>
#include <dir.h>
#include <filesystem>

void experiment(int N, int pairs) {
    mkdir("output");
    srand(time(nullptr));
    
    unsigned long long minKUP = ULLONG_MAX;
    BoolMatrixRelation minUPM = BoolMatrixRelation::getIdentity(1);
    BoolMatrixRelation minUPO = BoolMatrixRelation::getIdentity(1);
    unsigned long long maxKUP = 0;
    BoolMatrixRelation maxUPM = BoolMatrixRelation::getIdentity(1);
    BoolMatrixRelation maxUPO = BoolMatrixRelation::getIdentity(1);
    unsigned long long minKWarshall = ULLONG_MAX;
    BoolMatrixRelation minWarshallM = BoolMatrixRelation::getIdentity(1);
    BoolMatrixRelation minWarshallO = BoolMatrixRelation::getIdentity(1);
    unsigned long long maxKWarshall = 0;
    BoolMatrixRelation maxWarshallM = BoolMatrixRelation::getIdentity(1);
    BoolMatrixRelation maxWarshallO = BoolMatrixRelation::getIdentity(1);
    for (int i = 0; i < 1000; i++) {
        BoolMatrixRelation rel = BoolMatrixRelation::getRandom(N, pairs);

        unsigned long long kPow = 0;
        BoolMatrixRelation powUnite = rel.transitiveClosurePowUnite(&kPow);
        unsigned long long kWarshall = 0;
        BoolMatrixRelation warshall = rel.transitiveClosureWarshall(&kWarshall);

        std::pair<int, int> ignored;
        assert(powUnite.isTransitive(ignored));
        assert(warshall.isTransitive(ignored));
        assert(powUnite.equals(warshall));

        if (kPow >= maxKUP) {
            maxKUP = kPow;
            maxUPM = powUnite;
            maxUPO = rel;
        }
        if (kPow <= minKUP) {
            minKUP = kPow;
            minUPM = powUnite;
            minUPO = rel;
        }

        if (kWarshall >= maxKWarshall) {
            maxKWarshall = kWarshall;
            maxWarshallM = warshall;
            maxWarshallO = rel;
        }
        if (kWarshall <= minKWarshall) {
            minKWarshall = kWarshall;
            minWarshallM = warshall;
            minWarshallO = rel;
        }
    }

    std::string prefix = "output/";
    // U  - объединение степенией / W - воршэлл
    // M  - транзитивное замыкание / O - оригинал
    // Mi - минимальное кол-во выполнений тела самого вложенного цикла / Ma - максимальное кол-во выполнений тела самого вложенного цикла
    // N  - Мощность
    // P  - Кол-во сгенерированных пар 
    std::ofstream wrMinUPM(prefix + "N" + std::to_string(N) + "U" + "M" + "Mi" + std::to_string(minKUP)       + "P" + std::to_string(pairs) + ".txt");
    std::ofstream wrMinUPO(prefix + "N" + std::to_string(N) + "U" + "O" + "Mi" + std::to_string(minKUP)       + "P" + std::to_string(pairs) + ".txt");
    std::ofstream wrMaxUPM(prefix + "N" + std::to_string(N) + "U" + "M" + "Ma" + std::to_string(maxKUP)       + "P" + std::to_string(pairs) + ".txt");
    std::ofstream wrMaxUPO(prefix + "N" + std::to_string(N) + "U" + "O" + "Ma" + std::to_string(maxKUP)       + "P" + std::to_string(pairs) + ".txt");
    std::ofstream wrMinWPM(prefix + "N" + std::to_string(N) + "W" + "M" + "Mi" + std::to_string(minKWarshall) + "P" + std::to_string(pairs) + ".txt");
    std::ofstream wrMinWPO(prefix + "N" + std::to_string(N) + "W" + "O" + "Mi" + std::to_string(minKWarshall) + "P" + std::to_string(pairs) + ".txt");
    std::ofstream wrMaxWPM(prefix + "N" + std::to_string(N) + "W" + "M" + "Ma" + std::to_string(maxKWarshall) + "P" + std::to_string(pairs) + ".txt");
    std::ofstream wrMaxWPO(prefix + "N" + std::to_string(N) + "W" + "O" + "Ma" + std::to_string(maxKWarshall) + "P" + std::to_string(pairs) + ".txt");
    
    minUPM.printAsAdjMatrix(wrMinUPM);
    minUPO.printAsAdjMatrix(wrMinUPO);
    maxUPM.printAsAdjMatrix(wrMaxUPM);
    maxUPO.printAsAdjMatrix(wrMaxUPO);
    minWarshallM.printAsAdjMatrix(wrMinWPM);
    minWarshallO.printAsAdjMatrix(wrMinWPO);
    maxWarshallM.printAsAdjMatrix(wrMaxWPM);
    maxWarshallO.printAsAdjMatrix(wrMaxWPO);
}

int main() {
    std::filesystem::remove_all("output/");

    for (int i = 5; i <= 15; i += 5) {
        experiment(i, 1);
        experiment(i, (i * i) / 4);
        experiment(i, (i * i) / 2);
        experiment(i, (i * i * 2) / 3);
        experiment(i, i * i);
    }

    return 0;
}
\end{minted}
	      \begin{center}
		      \setlength\tabcolsep{1.5pt}
		      \textbf{Количество выполнений тела самого вложенного цикла (N = 5)}
		      \begin{tabular}{|c|c|c|c|c|c|c|c|c|c|c|}
			      \hline
			      \multirow{2}{6em}{Число пар в отношении}                      & \multicolumn{2}{|c|}{1} &
			      \multicolumn{2}{|c|}{$\frac{N^2}{4}$}                         &
			      \multicolumn{2}{|c|}{$\frac{N^2}{2}$}                         &
			      \multicolumn{2}{|c|}{$\frac{2 \cdot N^2}{3}$}                 &
			      \multicolumn{2}{|c|}{$N^2$}                                                                                                                   \\
			      \cline{2-11}
			                                                                    & min                     & max & min & max & min & max & min & max & min & max \\
			      \hline
			      \begin{tabular}{c}Алгоритм объединения\\степеней (до \\оптимизации)\end{tabular} & 481                     & 500 & 127 & 500 & 100 & 400 & 100 & 268 & 100 & 100 \\
			      \hline
			      \begin{tabular}{c}Алгоритм объединения\\степеней\end{tabular} & 125                     & 500 & 125 & 500 & 100 & 372 & 100 & 260 & 100 & 100 \\
				  \hline 
				  Алгоритм Уоршалла                                             & 5                     & 5 & 30 & 75 & 60 & 125 & 80 & 125 & 125 & 125 \\
			      \hline
		      \end{tabular}\\
			  \begin{longtable}{>{\centering\arraybackslash}p{0.5\textwidth}|>{\centering\arraybackslash}p{0.5\textwidth}}
				Исходное отношение & Транзитивное замыкание\\
				\hline
				\multicolumn{2}{c}{Алгоритм объединения степеней, минимум повторений цикла, 1 пара}\\
				\includegraphics[width=70mm]{N5UOMiP1} & \includegraphics[width=70mm]{N5UMMiP1}\\
				\hline
				\multicolumn{2}{c}{Алгоритм объединения степеней, максимум повторений цикла, 1 пара}\\
				\includegraphics[width=70mm]{N5UOMaP1} & \includegraphics[width=70mm]{N5UMMaP1}\\
				\hline
				\multicolumn{2}{c}{Алгоритм объединения степеней, минимум повторений цикла, 6 пар}\\
				\includegraphics[width=70mm]{N5UOMiP6} & \includegraphics[width=70mm]{N5UMMiP6}\\
				\hline
				\multicolumn{2}{c}{Алгоритм объединения степеней, максимум повторений цикла, 6 пар}\\
				\includegraphics[width=70mm]{N5UOMaP6} & \includegraphics[width=70mm]{N5UMMaP6}\\
				\hline
				\multicolumn{2}{c}{Алгоритм объединения степеней, минимум повторений цикла, 12 пар}\\
				\includegraphics[width=70mm]{N5UOMiP12} & \includegraphics[width=70mm]{N5UMMiP12}\\
				\hline
				\multicolumn{2}{c}{Алгоритм объединения степеней, максимум повторений цикла, 12 пар}\\
				\includegraphics[width=70mm]{N5UOMaP12} & \includegraphics[width=70mm]{N5UMMaP12}\\
				\hline
				\multicolumn{2}{c}{Алгоритм объединения степеней, минимум повторений цикла, 16 пар}\\
				\includegraphics[width=70mm]{N5UOMiP16} & \includegraphics[width=70mm]{N5UMMiP16}\\
				\hline
				\multicolumn{2}{c}{Алгоритм объединения степеней, максимум повторений цикла, 16 пар}\\
				\includegraphics[width=70mm]{N5UOMaP16} & \includegraphics[width=70mm]{N5UMMaP16}\\
				\hline
				\multicolumn{2}{c}{Алгоритм объединения степеней, минимум повторений цикла, 25 пар}\\
				\includegraphics[width=70mm]{N5UOMiP25} & \includegraphics[width=70mm]{N5UMMiP25}\\
				\hline
				\multicolumn{2}{c}{Алгоритм объединения степеней, максимум повторений цикла, 25 пар}\\
				\includegraphics[width=70mm]{N5UOMaP25} & \includegraphics[width=70mm]{N5UMMaP25}\\
				\hline
				\multicolumn{2}{c}{Алгоритм Уоршалла, минимум повторений цикла, 1 пара}\\
				\includegraphics[width=70mm]{N5WOMiP1} & \includegraphics[width=70mm]{N5UMMiP1}\\
				\hline
				\multicolumn{2}{c}{Алгоритм Уоршалла, максимум повторений цикла, 1 пара}\\
				\includegraphics[width=70mm]{N5WOMaP1} & \includegraphics[width=70mm]{N5WMMaP1}\\
				\hline
				\multicolumn{2}{c}{Алгоритм Уоршалла, минимум повторений цикла, 6 пар}\\
				\includegraphics[width=70mm]{N5WOMiP6} & \includegraphics[width=70mm]{N5WMMiP6}\\
				\hline
				\multicolumn{2}{c}{Алгоритм Уоршалла, максимум повторений цикла, 6 пар}\\
				\includegraphics[width=70mm]{N5WOMaP6} & \includegraphics[width=70mm]{N5WMMaP6}\\
				\hline
				\multicolumn{2}{c}{Алгоритм Уоршалла, минимум повторений цикла, 12 пар}\\
				\includegraphics[width=70mm]{N5WOMiP12} & \includegraphics[width=70mm]{N5WMMiP12}\\
				\hline
				\multicolumn{2}{c}{Алгоритм Уоршалла, максимум повторений цикла, 12 пар}\\
				\includegraphics[width=70mm]{N5WOMaP12} & \includegraphics[width=70mm]{N5WMMaP12}\\
				\hline
				\multicolumn{2}{c}{Алгоритм Уоршалла, минимум повторений цикла, 16 пар}\\
				\includegraphics[width=70mm]{N5WOMiP16} & \includegraphics[width=70mm]{N5WMMiP16}\\
				\hline
				\multicolumn{2}{c}{Алгоритм Уоршалла, максимум повторений цикла, 16 пар}\\
				\includegraphics[width=70mm]{N5WOMaP16} & \includegraphics[width=70mm]{N5WMMaP16}\\
				\hline
				\multicolumn{2}{c}{Алгоритм Уоршалла, минимум повторений цикла, 25 пар}\\
				\includegraphics[width=70mm]{N5WOMiP25} & \includegraphics[width=70mm]{N5WMMiP25}\\
				\hline
				\multicolumn{2}{c}{Алгоритм Уоршалла, максимум повторений цикла, 25 пар}\\
				\includegraphics[width=70mm]{N5WOMaP25} & \includegraphics[width=70mm]{N5WMMaP25}\\
				\hline
			  \end{longtable}
		      \setlength\tabcolsep{1.5pt}
		      \textbf{Количество выполнений тела самого вложенного цикла (N = 10)}
		      \begin{tabular}{|c|c|c|c|c|c|c|c|c|c|c|}
			      \hline
			      \multirow{2}{6em}{Число пар в отношении}                      & \multicolumn{2}{|c|}{1} &
			      \multicolumn{2}{|c|}{$\frac{N^2}{4}$}                         &
			      \multicolumn{2}{|c|}{$\frac{N^2}{2}$}                         &
			      \multicolumn{2}{|c|}{$\frac{2 \cdot N^2}{3}$}                 &
			      \multicolumn{2}{|c|}{$N^2$}                                                                                                                            \\
			      \cline{2-11}
			                                                                    & min                     & max  & min  & max  & min  & max  & min  & max  & min  & max  \\
			      \hline
			      \begin{tabular}{c}Алгоритм объединения\\степеней (до \\оптимизации)\end{tabular} & 8919                    & 9000 & 920  & 7067 & 900  & 2391 & 900  & 928  & 900  & 900  \\
			      \hline
				  \begin{tabular}{c}Алгоритм объединения\\степеней\end{tabular} & 1000                    & 9000 & 934  & 7179 & 900  & 2540 & 900  & 2358  & 900  & 900  \\
			      \hline
			      Алгоритм Уоршалла                                             & 10                    & 10 & 310 & 830 & 630 & 990 & 810 & 1000 & 1000 & 1000 \\
			      \hline
		      \end{tabular}\\
			  \begin{longtable}{>{\centering\arraybackslash}p{0.5\textwidth}|>{\centering\arraybackslash}p{0.5\textwidth}}
				Исходное отношение & Транзитивное замыкание\\
				\hline
				\multicolumn{2}{c}{Алгоритм объединения степеней, минимум повторений цикла, 1 пара}\\
				\includegraphics[width=70mm]{N10UOMiP1} & \includegraphics[width=70mm]{N10UMMiP1}\\
				\hline
				\multicolumn{2}{c}{Алгоритм объединения степеней, максимум повторений цикла, 1 пара}\\
				\includegraphics[width=70mm]{N10UOMaP1} & \includegraphics[width=70mm]{N10UMMaP1}\\
				\hline
				\multicolumn{2}{c}{Алгоритм объединения степеней, минимум повторений цикла, 25 пар}\\
				\includegraphics[width=70mm]{N10UOMiP25} & \includegraphics[width=70mm]{N10UMMiP25}\\
				\hline
				\multicolumn{2}{c}{Алгоритм объединения степеней, максимум повторений цикла, 25 пар}\\
				\includegraphics[width=70mm]{N10UOMaP25} & \includegraphics[width=70mm]{N10UMMaP25}\\
				\hline
				\multicolumn{2}{c}{Алгоритм объединения степеней, минимум повторений цикла, 50 пар}\\
				\includegraphics[width=70mm]{N10UOMiP50} & \includegraphics[width=70mm]{N10UMMiP50}\\
				\hline
				\multicolumn{2}{c}{Алгоритм объединения степеней, максимум повторений цикла, 50 пар}\\
				\includegraphics[width=70mm]{N10UOMaP50} & \includegraphics[width=70mm]{N10UMMaP50}\\
				\hline
				\multicolumn{2}{c}{Алгоритм объединения степеней, минимум повторений цикла, 66 пар}\\
				\includegraphics[width=70mm]{N10UOMiP66} & \includegraphics[width=70mm]{N10UMMiP66}\\
				\hline
				\multicolumn{2}{c}{Алгоритм объединения степеней, максимум повторений цикла, 66 пар}\\
				\includegraphics[width=70mm]{N10UOMaP66} & \includegraphics[width=70mm]{N10UMMaP66}\\
				\hline
				\multicolumn{2}{c}{Алгоритм объединения степеней, минимум повторений цикла, 100 пар}\\
				\includegraphics[width=70mm]{N10UOMiP100} & \includegraphics[width=70mm]{N10UMMiP100}\\
				\hline
				\multicolumn{2}{c}{Алгоритм объединения степеней, максимум повторений цикла, 100 пар}\\
				\includegraphics[width=70mm]{N10UOMaP100} & \includegraphics[width=70mm]{N10UMMaP100}\\
				\hline
				\multicolumn{2}{c}{Алгоритм Уоршалла, минимум повторений цикла, 1 пара}\\
				\includegraphics[width=70mm]{N10WOMiP1} & \includegraphics[width=70mm]{N10UMMiP1}\\
				\hline
				\multicolumn{2}{c}{Алгоритм Уоршалла, максимум повторений цикла, 1 пара}\\
				\includegraphics[width=70mm]{N10WOMaP1} & \includegraphics[width=70mm]{N10WMMaP1}\\
				\hline
				\multicolumn{2}{c}{Алгоритм Уоршалла, минимум повторений цикла, 25 пар}\\
				\includegraphics[width=70mm]{N10WOMiP25} & \includegraphics[width=70mm]{N10WMMiP25}\\
				\hline
				\multicolumn{2}{c}{Алгоритм Уоршалла, максимум повторений цикла, 25 пар}\\
				\includegraphics[width=70mm]{N10WOMaP25} & \includegraphics[width=70mm]{N10WMMaP25}\\
				\hline
				\multicolumn{2}{c}{Алгоритм Уоршалла, минимум повторений цикла, 50 пар}\\
				\includegraphics[width=70mm]{N10WOMiP50} & \includegraphics[width=70mm]{N10WMMiP50}\\
				\hline
				\multicolumn{2}{c}{Алгоритм Уоршалла, максимум повторений цикла, 50 пар}\\
				\includegraphics[width=70mm]{N10WOMaP50} & \includegraphics[width=70mm]{N10WMMaP50}\\
				\hline
				\multicolumn{2}{c}{Алгоритм Уоршалла, минимум повторений цикла, 66 пар}\\
				\includegraphics[width=70mm]{N10WOMiP66} & \includegraphics[width=70mm]{N10WMMiP66}\\
				\hline
				\multicolumn{2}{c}{Алгоритм Уоршалла, максимум повторений цикла, 66 пар}\\
				\includegraphics[width=70mm]{N10WOMaP66} & \includegraphics[width=70mm]{N10WMMaP66}\\
				\hline
				\multicolumn{2}{c}{Алгоритм Уоршалла, минимум повторений цикла, 100 пар}\\
				\includegraphics[width=70mm]{N10WOMiP100} & \includegraphics[width=70mm]{N10WMMiP100}\\
				\hline
				\multicolumn{2}{c}{Алгоритм Уоршалла, максимум повторений цикла, 100 пар}\\
				\includegraphics[width=70mm]{N10WOMaP100} & \includegraphics[width=70mm]{N10WMMaP100}\\
				\hline
			  \end{longtable}
		      \setlength\tabcolsep{1.5pt}
		      \textbf{Количество выполнений тела самого вложенного цикла (N = 15)}
		      \begin{tabular}{|c|c|c|c|c|c|c|c|c|c|c|}
			      \hline
			      \multirow{2}{6em}{Число пар в отношении}                      & \multicolumn{2}{|c|}{1} &
			      \multicolumn{2}{|c|}{$\frac{N^2}{4}$}                         &
			      \multicolumn{2}{|c|}{$\frac{N^2}{2}$}                         &
			      \multicolumn{2}{|c|}{$\frac{2 \cdot N^2}{3}$}                 &
			      \multicolumn{2}{|c|}{$N^2$}                                                                                                                              \\
			      \cline{2-11}
			                                                                    & min                     & max   & min  & max   & min  & max  & min  & max  & min  & max  \\
			      \hline
			      \begin{tabular}{c}Алгоритм объединения\\степеней (до \\оптимизации)\end{tabular} & 47074                   & 47250 & 3165 & 17100 & 3150 & 6077 & 3150 & 3165 & 3150 & 3150 \\
			      \hline
				  \begin{tabular}{c}Алгоритм объединения\\степеней\end{tabular} & 3375                   & 47250 & 3150 & 19278 & 3150 & 6090 & 3150 & 3165 & 3150 & 3150 \\
			      \hline
			      Алгоритм Уоршалла                                             & 15                    & 15  & 1470 & 3000  & 2640 & 3345 & 2910 & 3375 & 3375 & 3375 \\
			      \hline
		      \end{tabular}\\
			  \begin{longtable}{>{\centering\arraybackslash}p{0.5\textwidth}|>{\centering\arraybackslash}p{0.5\textwidth}}
				Исходное отношение & Транзитивное замыкание\\
				\hline
				\multicolumn{2}{c}{Алгоритм объединения степеней, минимум повторений цикла, 1 пара}\\
				\includegraphics[width=70mm]{N15UOMiP1} & \includegraphics[width=70mm]{N15UMMiP1}\\
				\hline
				\multicolumn{2}{c}{Алгоритм объединения степеней, максимум повторений цикла, 1 пара}\\
				\includegraphics[width=70mm]{N15UOMaP1} & \includegraphics[width=70mm]{N15UMMaP1}\\
				\hline
				\multicolumn{2}{c}{Алгоритм объединения степеней, минимум повторений цикла, 25 пар}\\
				\includegraphics[width=70mm]{N15UOMiP56} & \includegraphics[width=70mm]{N15UMMiP56}\\
				\hline
				\multicolumn{2}{c}{Алгоритм объединения степеней, максимум повторений цикла, 25 пар}\\
				\includegraphics[width=70mm]{N15UOMaP56} & \includegraphics[width=70mm]{N15UMMaP56}\\
				\hline
				\multicolumn{2}{c}{Алгоритм объединения степеней, минимум повторений цикла, 50 пар}\\
				\includegraphics[width=70mm]{N15UOMiP112} & \includegraphics[width=70mm]{N15UMMiP112}\\
				\hline
				\multicolumn{2}{c}{Алгоритм объединения степеней, максимум повторений цикла, 50 пар}\\
				\includegraphics[width=70mm]{N15UOMaP112} & \includegraphics[width=70mm]{N15UMMaP112}\\
				\hline
				\multicolumn{2}{c}{Алгоритм объединения степеней, минимум повторений цикла, 66 пар}\\
				\includegraphics[width=70mm]{N15UOMiP150} & \includegraphics[width=70mm]{N15UMMiP150}\\
				\hline
				\multicolumn{2}{c}{Алгоритм объединения степеней, максимум повторений цикла, 66 пар}\\
				\includegraphics[width=70mm]{N15UOMaP150} & \includegraphics[width=70mm]{N15UMMaP150}\\
				\hline
				\multicolumn{2}{c}{Алгоритм объединения степеней, минимум повторений цикла, 100 пар}\\
				\includegraphics[width=70mm]{N15UOMiP225} & \includegraphics[width=70mm]{N15UMMiP225}\\
				\hline
				\multicolumn{2}{c}{Алгоритм объединения степеней, максимум повторений цикла, 100 пар}\\
				\includegraphics[width=70mm]{N15UOMaP225} & \includegraphics[width=70mm]{N15UMMaP225}\\
				\hline
				\multicolumn{2}{c}{Алгоритм Уоршалла, минимум повторений цикла, 1 пара}\\
				\includegraphics[width=70mm]{N15WOMiP1} & \includegraphics[width=70mm]{N15UMMiP1}\\
				\hline
				\multicolumn{2}{c}{Алгоритм Уоршалла, максимум повторений цикла, 1 пара}\\
				\includegraphics[width=70mm]{N15WOMaP1} & \includegraphics[width=70mm]{N15WMMaP1}\\
				\hline
				\multicolumn{2}{c}{Алгоритм Уоршалла, минимум повторений цикла, 25 пар}\\
				\includegraphics[width=70mm]{N15WOMiP56} & \includegraphics[width=70mm]{N15WMMiP56}\\
				\hline
				\multicolumn{2}{c}{Алгоритм Уоршалла, максимум повторений цикла, 25 пар}\\
				\includegraphics[width=70mm]{N15WOMaP56} & \includegraphics[width=70mm]{N15WMMaP56}\\
				\hline
				\multicolumn{2}{c}{Алгоритм Уоршалла, минимум повторений цикла, 50 пар}\\
				\includegraphics[width=70mm]{N15WOMiP112} & \includegraphics[width=70mm]{N15WMMiP112}\\
				\hline
				\multicolumn{2}{c}{Алгоритм Уоршалла, максимум повторений цикла, 50 пар}\\
				\includegraphics[width=70mm]{N15WOMaP112} & \includegraphics[width=70mm]{N15WMMaP112}\\
				\hline
				\multicolumn{2}{c}{Алгоритм Уоршалла, минимум повторений цикла, 66 пар}\\
				\includegraphics[width=70mm]{N15WOMiP150} & \includegraphics[width=70mm]{N15WMMiP150}\\
				\hline
				\multicolumn{2}{c}{Алгоритм Уоршалла, максимум повторений цикла, 66 пар}\\
				\includegraphics[width=70mm]{N15WOMaP150} & \includegraphics[width=70mm]{N15WMMaP150}\\
				\hline
				\multicolumn{2}{c}{Алгоритм Уоршалла, минимум повторений цикла, 100 пар}\\
				\includegraphics[width=70mm]{N15WOMiP225} & \includegraphics[width=70mm]{N15WMMiP225}\\
				\hline
				\multicolumn{2}{c}{Алгоритм Уоршалла, максимум повторений цикла, 100 пар}\\
				\includegraphics[width=70mm]{N15WOMaP225} & \includegraphics[width=70mm]{N15WMMaP225}\\
				\hline
			  \end{longtable}
	      \end{center}

	\item Определить порядок функции временной сложности алгоритмов вычисления транзитивного замыкания.\\
	      \textbf{Алгоритм объединения степеней} (после оптимизации)\\
	      Порядок ФВС зависит от количества недостающих пар. Так, если отношение уже транзитивное, основной цикл выполнится всего 1 раз. То есть \\
		  $\Omega(1)$\\
		  В худшем случае выше упомянутый цикл будет выполняться все size раз, например если отношение пустое (матрица состоит из 0). Следовательно\\
	      $O(N^4)$\\
	      \textbf{Алгоритм Уоршалла}\\
		  Порядок временной функции временной сложности также зависит от количества пар в отношении. Однако в отличие от алгоритма объединения степеней, чем меньше отношений, тем меньше время выполнения. Так, например, для пустого отношения (матрица состоит из нулей) самый вложенный цикл выполняться не будет
	      \begin{minted}{C++}
if (!res.data[x][z]) continue;
            
for (int y = 0; y < size; y++) {
    if (steps != NULL)
        (*steps)++;

    res.data[x][y] = res.data[x][y] || res.data[x][z] && res.data[z][y];
}
\end{minted}
	      Степень вложенности в лучшем случае - 2, следовательно\\
	      $\Omega(N ^ 2)$  \\
	      В худшем случае выше упомянутый цикл будет выполняться все size раз, например если отношение можно представить в виде полного неориентированного графа (матрица состоит из 1). Следовательно\\
	      $O(N^3)$\\
\end{enumerate}

\textbf{Вывод: } в ходе лабораторной работы изучили и выполнили сравнительный анализ алгоритмов вычисления транзитивного замыкания отношения.

\end{document}