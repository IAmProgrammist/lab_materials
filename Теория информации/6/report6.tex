\documentclass[a4paper,14pt]{extarticle}


\usepackage[english,russian]{babel}
\usepackage[T2A]{fontenc}
\usepackage[utf8]{inputenc}
\usepackage{ragged2e}
\usepackage[utf8]{inputenc}
\usepackage{hyperref}
\usepackage{minted}
\setmintedinline{frame=lines, framesep=2mm, baselinestretch=1.5, bgcolor=LightGray, breaklines,fontsize=\scriptsize}
\setminted{frame=lines, framesep=2mm, baselinestretch=1.5, bgcolor=LightGray, breaklines,fontsize=\scriptsize}
\usepackage{xcolor}
\definecolor{LightGray}{gray}{0.9}
\usepackage{graphicx}
\usepackage[export]{adjustbox}
\usepackage[left=1cm,right=1cm, top=1cm,bottom=1cm,bindingoffset=0cm]{geometry}
\usepackage{fontspec}
\usepackage{ upgreek }
\usepackage[shortlabels]{enumitem}
\usepackage{adjustbox}
\usepackage{multirow}
\usepackage{amsmath}
\usepackage{amssymb}
\usepackage{pifont}
\usepackage{pgfplots}
\usepackage{longtable}
\usepackage{array}
\graphicspath{ {./images/} }
\makeatletter
\AddEnumerateCounter{\asbuk}{\russian@alph}{щ}
\makeatother
\setmonofont{Consolas}
\setmainfont{Times New Roman}

\newcommand\textbox[1]{
	\parbox{.45\textwidth}{#1}
} 

\newcommand{\specialcell}[2][c]{%
	\begin{tabular}[#1]{@{}c@{}}#2\end{tabular}}

\begin{document}
\pagenumbering{gobble}
\begin{center}
    \small{
        \textbf{МИНИCТЕРCТВО НАУКИ И ВЫCШЕГО ОБРАЗОВАНИЯ РОCCИЙCКОЙ ФЕДЕРАЦИИ}\\
        ФЕДЕРАЛЬНОЕ ГОCУДАРCТВЕННОЕ БЮДЖЕТНОЕ ОБРАЗОВАТЕЛЬНОЕ УЧРЕЖДЕНИЕ\\ВЫCШЕГО ОБРАЗОВАНИЯ \\
        \textbf{«БЕЛГОРОДCКИЙ ГОCУДАРCТВЕННЫЙ ТЕХНОЛОГИЧЕCКИЙ\\УНИВЕРCИТЕТ им. В. Г. ШУХОВА»\\ (БГТУ им. В.Г. Шухова)} \\
        \bigbreak
        \includegraphics[width=70mm]{log}\\
        ИНСТИТУТ ИНФОРМАЦИОННЫХ ТЕХНОЛОГИЙ И УПРАВЛЯЮЩИХ СИСТЕМ\\}
\end{center}

\vfill
\begin{center}
    \large{
        \textbf{
            Лабораторная работа №6}}\\
    \normalsize{
        по дисциплине: Теория информации \\
        тема: «LZW»}
\end{center}
\vfill
\hfill\textbox{
    Выполнил: ст. группы ПВ-223\\Пахомов Владислав Андреевич
    \bigbreak
    Проверили: \\пр. Твердохлеб Виталий Викторович
}
\vfill\begin{center}
    Белгород 2024 г.
\end{center}
\newpage
\begin{center}
    \textbf{Лабораторная работа №6}\\
    LZW\\
\end{center}
\textbf{Цель работы: }исследовать особенности метода LZW.\\
Задание 1. Подготовить предварительно сообщения:\\
- длиной 1000 символов;\\
- длиной порядка 10000 символов.\\
Используем источник Хартли.\\
Задание 2.
Подготовить сообщение длиной 10000 символов, отличающееся по природе (тип источника) от ранее подготовленного такой же длины. (используем источник Бернулли).\\
Задание 3. Построить обработчик LZW
\textit{Coder.java}
\begin{minted}{Java}
public String LZWCompress(String input) {
    HashMap<String, Integer> dictionary = new LinkedHashMap<>();
    String[] data = (input + "").split("");
    String out = "";
    ArrayList<String> tempOut = new ArrayList<>();
    String currentChar;
    String phrase = data[0];
    int code = 256;
    for (int i = 1; i < data.length; i++) {
        currentChar = data[i];
        if (dictionary.get(phrase + currentChar) != null) {
            phrase += currentChar;
        } else {
            if (phrase.length() > 1) {
                tempOut.add(Character.toString((char) dictionary.get(phrase).intValue()));
            } else {
                tempOut.add(Character.toString((char) Character.codePointAt(phrase, 0)));
            }

            dictionary.put(phrase + currentChar, code);
            code++;
            phrase = currentChar;
        }
    }

    if (phrase.length() > 1) {
        tempOut.add(Character.toString((char) dictionary.get(phrase).intValue()));
    } else {
        tempOut.add(Character.toString((char) Character.codePointAt(phrase, 0)));
    }

    for (String outchar : tempOut) {
        out += outchar;
    }
    return out;
}
\end{minted}

Задание 4. Исследовать зависимость коэффициента сжатия от длины сообщения и его принадлежности к различным источникам.\\

Коэффициент сжатия для 1000 символов Хартли: 1.165841\\
Коэффициент сжатия для 10000 символов Хартли: 1.34551\\
Коэффициент сжатия для 10000 символов Бернулли: 1.931807\\

\textbf{Вывод: } в ходе лабораторной работы исследовали особенности метода LZW. 

\end{document}