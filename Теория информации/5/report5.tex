\documentclass[a4paper,14pt]{extarticle}


\usepackage[english,russian]{babel}
\usepackage[T2A]{fontenc}
\usepackage[utf8]{inputenc}
\usepackage{ragged2e}
\usepackage[utf8]{inputenc}
\usepackage{hyperref}
\usepackage{minted}
\setmintedinline{frame=lines, framesep=2mm, baselinestretch=1.5, bgcolor=LightGray, breaklines,fontsize=\scriptsize}
\setminted{frame=lines, framesep=2mm, baselinestretch=1.5, bgcolor=LightGray, breaklines,fontsize=\scriptsize}
\usepackage{xcolor}
\definecolor{LightGray}{gray}{0.9}
\usepackage{graphicx}
\usepackage[export]{adjustbox}
\usepackage[left=1cm,right=1cm, top=1cm,bottom=1cm,bindingoffset=0cm]{geometry}
\usepackage{fontspec}
\usepackage{ upgreek }
\usepackage[shortlabels]{enumitem}
\usepackage{adjustbox}
\usepackage{multirow}
\usepackage{amsmath}
\usepackage{amssymb}
\usepackage{pifont}
\usepackage{pgfplots}
\usepackage{longtable}
\usepackage{array}
\graphicspath{ {./images/} }
\makeatletter
\AddEnumerateCounter{\asbuk}{\russian@alph}{щ}
\makeatother
\setmonofont{Consolas}
\setmainfont{Times New Roman}

\newcommand\textbox[1]{
	\parbox{.45\textwidth}{#1}
} 

\newcommand{\specialcell}[2][c]{%
	\begin{tabular}[#1]{@{}c@{}}#2\end{tabular}}

\begin{document}
\pagenumbering{gobble}
\begin{center}
    \small{
        \textbf{МИНИCТЕРCТВО НАУКИ И ВЫCШЕГО ОБРАЗОВАНИЯ РОCCИЙCКОЙ ФЕДЕРАЦИИ}\\
        ФЕДЕРАЛЬНОЕ ГОCУДАРCТВЕННОЕ БЮДЖЕТНОЕ ОБРАЗОВАТЕЛЬНОЕ УЧРЕЖДЕНИЕ\\ВЫCШЕГО ОБРАЗОВАНИЯ \\
        \textbf{«БЕЛГОРОДCКИЙ ГОCУДАРCТВЕННЫЙ ТЕХНОЛОГИЧЕCКИЙ\\УНИВЕРCИТЕТ им. В. Г. ШУХОВА»\\ (БГТУ им. В.Г. Шухова)} \\
        \bigbreak
        \includegraphics[width=70mm]{log}\\
        ИНСТИТУТ ИНФОРМАЦИОННЫХ ТЕХНОЛОГИЙ И УПРАВЛЯЮЩИХ СИСТЕМ\\}
\end{center}

\vfill
\begin{center}
    \large{
        \textbf{
            Лабораторная работа №5}}\\
    \normalsize{
        по дисциплине: Теория информации \\
        тема: «Арифметическое кодирование»}
\end{center}
\vfill
\hfill\textbox{
    Выполнил: ст. группы ПВ-223\\Пахомов Владислав Андреевич
    \bigbreak
    Проверили: \\пр. Твердохлеб Виталий Викторович
}
\vfill\begin{center}
    Белгород 2024 г.
\end{center}
\newpage
\begin{center}
    \textbf{Лабораторная работа №5}\\
    Арифметическое кодирование\\
\end{center}
\textbf{Цель работы: }Исследовать особенности метода арифметического кодирования.\\
Задание 1. Построить обработчик, реализующий данный алгоритм арифметического кодирования.
\textit{Coder.java}
\begin{minted}{Java}
public static <T> ArithmeticEncodingTable<T> getArithmeticEncodingTable(List<T> input) {
    ArithmeticEncodingTable<T> aTable = new ArithmeticEncodingTable<>();
    List<TableElement<T>> segTable = getSegmentisedTable(input);

    for (TableElement<T> element : segTable) {
        aTable.addNewSymbol(element, (1.0 * element.amount) / input.size());
    }

    return aTable;
}

public static <T> AbstractMap.SimpleEntry<Double, Double> encodeArithmeticEncodingTable(List<T> input) {
    if (input.isEmpty()) return new AbstractMap.SimpleEntry<>(0., 0.);

    ArithmeticEncodingTable<T> table = getArithmeticEncodingTable(input);
    AbstractMap.SimpleEntry<Double, Double> answer =
            table.getProbabilitySpanBySymbol(input.get(0));

    for (int i = 1; i < input.size(); i++) {
        AbstractMap.SimpleEntry<Double, Double> currSymbol = table.getProbabilitySpanBySymbol(input.get(i));

        answer = new AbstractMap.SimpleEntry<>(
                answer.getKey() + (answer.getValue() - answer.getKey()) * currSymbol.getKey(),
                answer.getKey() + (answer.getValue() - answer.getKey()) * currSymbol.getValue());
    }

    return answer;
}
\end{minted}

\textit{Main.java}
\begin{minted}{Java}
package rchat.info.lab5;

import dnl.utils.text.table.TextTable;
import rchat.info.libs.Coder;

import javax.swing.event.TableModelListener;
import javax.swing.table.TableModel;
import java.io.BufferedReader;
import java.io.IOException;
import java.io.InputStreamReader;
import java.nio.file.Files;
import java.nio.file.Path;
import java.util.AbstractMap;
import java.util.ArrayList;
import java.util.Arrays;
import java.util.List;
import java.util.stream.Collectors;

public class Main {
    public static void main(String[] args) throws IOException {
        System.out.println("Введите сообщение: ");
        String finput;
        BufferedReader r = new BufferedReader(new InputStreamReader(System.in));
        finput = r.readLine();

        AbstractMap.SimpleEntry<Double, Double> ans = Coder.encodeArithmeticEncodingTable(Arrays.asList(finput.split("")));

        double left = 0.0;
        double right = 1.0;
        double middle;
        int it = 0;

        while (true) {
            it++;
            middle = (right + left) / 2;

            if (middle > ans.getValue()) {
                right = middle;
            } else if (middle < ans.getKey()) {
                left = middle;
            } else break;
        }

        System.out.println("Интервал: " + ans);
        System.out.println("Число, характеризующее интервал: " + middle);
        System.out.println("Количество бит: " + it);
        System.out.println("Код: ");
        while (middle != 0) {
            System.out.print((int)(middle * 2) == 1 ? "1" : "0");
            middle *= 2;
            if (middle >= 1)
                middle -= 1;
        }
    }
}
\end{minted}

Задание 2. Построить код для сообщения, содержащего строку панграммы
«в чащах юга жил бы цитрус? Да, но фальшивый экземпляр!».

\begin{minted}{console}
Введите сообщение: 
в чащах юга жил бы цитрус? Да, но фальшивый экземпляр!
Интервал: 0.4505322663570715=0.4505322663570715
Число, характеризующее интервал: 0.4505322663570715
Количество бит: 54
Код: 
011100110101011000010101001001011100101111011111110001
\end{minted}

Количество данных по сравнению с кодированием по методу Шеннона Фано уменьшилось, 54 < 258.\\
Коэффициент сжатия: $8$

Задание 3. Построить код для сообщения, содержащего строку «Victoria nulla est, Quam quae confessos animo quoque subjugat hostes»

\begin{minted}{console}
Введите сообщение: 
Victoria nulla est, Quam quae confessos animo quoque subjugat hostes
Интервал: 0.9966629048585032=0.9966629048585032
Число, характеризующее интервал: 0.9966629048585032
Количество бит: 52
Код: 
1111111100100101010011001101010110000000111011011111
\end{minted}

Количество данных по сравнению с кодированием по методу Шеннона Фано уменьшилось, 52 < 278.\\
Коэффициент сжатия: $10\frac{6}{13}$

\textbf{Вывод: } в ходе лабораторной работы исследовали особенности метода арифметического кодирования. Эффективность
сжатия алгоритма сильно превосходит алгоритм Шеннона Фано.

\end{document}