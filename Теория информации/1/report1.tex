\documentclass[a4paper,14pt]{extarticle}


\usepackage[english,russian]{babel}
\usepackage[T2A]{fontenc}
\usepackage[utf8]{inputenc}
\usepackage{ragged2e}
\usepackage[utf8]{inputenc}
\usepackage{hyperref}
\usepackage{minted}
\setmintedinline{frame=lines, framesep=2mm, baselinestretch=1.5, bgcolor=LightGray, breaklines,fontsize=\scriptsize}
\setminted{frame=lines, framesep=2mm, baselinestretch=1.5, bgcolor=LightGray, breaklines,fontsize=\scriptsize}
\usepackage{xcolor}
\definecolor{LightGray}{gray}{0.9}
\usepackage{graphicx}
\usepackage[export]{adjustbox}
\usepackage[left=1cm,right=1cm, top=1cm,bottom=1cm,bindingoffset=0cm]{geometry}
\usepackage{fontspec}
\usepackage{ upgreek }
\usepackage[shortlabels]{enumitem}
\usepackage{adjustbox}
\usepackage{multirow}
\usepackage{amsmath}
\usepackage{amssymb}
\usepackage{pifont}
\usepackage{pgfplots}
\usepackage{longtable}
\usepackage{array}
\graphicspath{ {./images/} }
\makeatletter
\AddEnumerateCounter{\asbuk}{\russian@alph}{щ}
\makeatother
\setmonofont{Consolas}
\setmainfont{Times New Roman}

\newcommand\textbox[1]{
	\parbox{.45\textwidth}{#1}
} 

\newcommand{\specialcell}[2][c]{%
	\begin{tabular}[#1]{@{}c@{}}#2\end{tabular}}

\begin{document}
\pagenumbering{gobble}
\begin{center}
    \small{
        \textbf{МИНИCТЕРCТВО НАУКИ И ВЫCШЕГО ОБРАЗОВАНИЯ РОCCИЙCКОЙ ФЕДЕРАЦИИ}\\
        ФЕДЕРАЛЬНОЕ ГОCУДАРCТВЕННОЕ БЮДЖЕТНОЕ ОБРАЗОВАТЕЛЬНОЕ УЧРЕЖДЕНИЕ\\ВЫCШЕГО ОБРАЗОВАНИЯ \\
        \textbf{«БЕЛГОРОДCКИЙ ГОCУДАРCТВЕННЫЙ ТЕХНОЛОГИЧЕCКИЙ\\УНИВЕРCИТЕТ им. В. Г. ШУХОВА»\\ (БГТУ им. В.Г. Шухова)} \\
        \bigbreak
        \includegraphics[width=70mm]{log}\\
        ИНСТИТУТ ИНФОРМАЦИОННЫХ ТЕХНОЛОГИЙ И УПРАВЛЯЮЩИХ СИСТЕМ\\}
\end{center}

\vfill
\begin{center}
    \large{
        \textbf{
            Лабораторная работа №1}}\\
    \normalsize{
        по дисциплине: ООП \\
        тема: «Исследование кодирования по методу Хаффмана. Оценка эффективности кода.»}
\end{center}
\vfill
\hfill\textbox{
    Выполнил: ст. группы ПВ-223\\Пахомов Владислав Андреевич
    \bigbreak
    Проверили: \\пр. Твердохлеб Виталий Викторович
}
\vfill\begin{center}
    Белгород 2024 г.
\end{center}
\newpage
\begin{center}
    \textbf{Лабораторная работа №1}\\
    Исследование кодирования по методу Хаффмана. Оценка эффективности кода.\\
\end{center}
\textbf{Цель работы: }исследовать кодирование по методу Хаффмана. Научиться оценивать эффективности кода.\\
Задание 1. Построить кодовое представление сообщения, вероятности
появления символов в пределах алфавита.\\
\begin{center}
    \begin{tabular}{|c|c|c|c|c|c|c|c|c|}
        \hline
        Символ      & s1   & s2   & s3   & s4   & s5   & s6   & s7   & s8   \\
        \hline
        Вероятность & 0.23 & 0.19 & 0.16 & 0.16 & 0.10 & 0.10 & 0.05 & 0.01 \\
        \hline
    \end{tabular}
    \includegraphics[width=120mm]{task1.jpg}\\
\end{center}
Задание 2. Построить кодовое представление сообщения, вероятности
появления символов в пределах алфавита.\\
\begin{center}
    \begin{tabular}{|c|c|c|c|c|c|c|c|c|}
        \hline
        Символ      & s1   & s2   & s3   & s4   & s5  & s6   & s7   & s8   \\
        \hline
        Вероятность & 0.25 & 0.22 & 0.13 & 0.11 & 0.1 & 0.09 & 0.07 & 0.03 \\
        \hline
    \end{tabular}
    \includegraphics[width=120mm]{task2.jpg}\\
\end{center}
Задание 3. Построить кодовое представление сообщения:\\
\textbf{оитомии о ими оооитми о о о ооиимтомиимотоим оои тоо и и м оио и омтоо тоимо т и}\\
\begin{center}
    \begin{tabular}{|c|c|c|c|c|c|c|c|c|}
        \hline
        Символ      & о      & и      & т   & м     & пробел \\
        \hline
        Количество  & 25     & 19     & 8   & 10    & 18     \\
        \hline
        Вероятность & 0.3125 & 0.2375 & 0.1 & 0.125 & 0.225  \\
        \hline
    \end{tabular}
    \includegraphics[width=120mm]{task3.jpg}\\
\end{center}
о: 11\\
и: 10\\
пробел: 01\\
т: 000\\
м: 001\\
Кодовое представление: 111000011001101001110110001100111111110000001100111\\
01110111011111101000100011001101000111000111000101111110010001111011001100\\
10010111101101100111001000111101000111000111010000110\\
Задание 4. Для условий, приведенных в заданиях 1 и 2 и 3, выявить
возможность построения альтернативных кодовых моделей сообщения. В
случае обнаружения таковых, выявить наиболее эффективные из них по
критериям $K_{comp}$ и $\delta$.\\
Задание 1:\\
Пусть n = 100\\
$B = 100 \cdot 8 = 800$\\
$B' = 23 \cdot 2 + 19 \cdot 2 + 16 \cdot 3 + 16 \cdot 3 + 10 \cdot 3 + 10 \cdot 4 + 5 \cdot 5 + 5 = 280$\\
$K_{comp} = \frac{B}{B'} = \frac{800}{280} = 2\frac{6}{7}$\\
$l_{ср} = 0.23 \cdot 2 + 0.19 \cdot 2 + 0.16 \cdot 3 + 0.16 \cdot 3 + 0.10 \cdot 3 + 0.10 \cdot 4 + 0.05 \cdot 5 + 0.01 \cdot 5 = 2.8$\\
$\delta = 0.23 \cdot (2 - 2.8)^2 + 0.19 \cdot (2 - 2.8)^2 + 0.16 \cdot (3 - 2.8)^2 + 0.16 \cdot (3 - 2.8)^2 + 0.10 \cdot (3 - 2.8)^2 + 0.10 \cdot (4 - 2.8)^2 + 0.05 \cdot (5 - 2.8)^2 + 0.01 \cdot (5 - 2.8)^2 = 0.72$\\

Варианты появляются на 3 шаге формирования дерева:
\begin{center}
    \includegraphics[width=120mm]{task1_var1.jpg}\\
\end{center}
Длина символов не поменялась, следовательно дисперсия не изменится.\\

Задание 2:\\
Пусть n = 100\\
$B = 100 \cdot 8 = 800$\\
$B' = 25 \cdot 2 + 22 \cdot 2 + 13 \cdot 3 + 11 \cdot 3 + 10 \cdot 4 + 9 \cdot 4 + 7 \cdot 4 + 3 \cdot 4 = 282$\\
$K_{comp} = \frac{B}{B'} = \frac{800}{282} = 2\frac{118}{141}$\\
$l_{ср} = 0.25 \cdot 2 + 0.22 \cdot 2 + 0.13 \cdot 3 + 0.11 \cdot 3 + 0.1 \cdot 4 + 0.09 \cdot 4 + 0.07 \cdot 4 + 0.03 \cdot 4 = 2.82$\\
$\delta = 0.25 \cdot (2 - 2.82)^2 + 0.22 \cdot (2 - 2.82)^2 + 0.13 \cdot (3 - 2.82)^2 + 0.11 \cdot (3 - 2.82)^2 + 0.1 \cdot (4 - 2.82)^2 + 0.09 \cdot (4 - 2.82)^2 + 0.07 \cdot (4 - 2.82)^2 + 0.03 \cdot (4 - 2.82)^2 = 0.7276$\\

Варианты появляются на 2 шаге формирования дерева:
\begin{center}
    \includegraphics[width=120mm]{task2_var1.jpg}\\
\end{center}

$B' = 25 \cdot 2 + 22 \cdot 2 + 13 \cdot 3 + 11 \cdot 3 + 10 \cdot 3 + 9 \cdot 4 + 7 \cdot 5 + 3 \cdot 5 = 282$\\
$K_{comp} = \frac{B}{B'} = \frac{800}{282} = 2\frac{118}{141}$\\
$l_{ср} = 0.25 \cdot 2 + 0.22 \cdot 2 + 0.13 \cdot 3 + 0.11 \cdot 3 + 0.1 \cdot 3 + 0.09 \cdot 4 + 0.07 \cdot 5 + 0.03 \cdot 5 = 2.82$\\
$\delta = 0.25 \cdot (2 - 2.82)^2 + 0.22 \cdot (2 - 2.82)^2 + 0.13 \cdot (3 - 2.82)^2 + 0.11 \cdot (3 - 2.82)^2 + 0.1 \cdot (3 - 2.82)^2 + 0.09 \cdot (4 - 2.82)^2 + 0.07 \cdot (5 - 2.82)^2 + 0.03 \cdot (5 - 2.82)^2 = 0.9276$\\

У первого варианта дисперсия меньше.\\

Задание 3:\\
$B = 80 \cdot 8 = 640$\\
$B' = 8 \cdot 3 + 10 \cdot 3 + 18 \cdot 2 + 19 \cdot 2 + 25 \cdot 2 = 178$\\
$K_{comp} = \frac{B}{B'} = \frac{640}{178} = 3\frac{53}{89}$\\
$l_{ср} = \frac{8}{80} \cdot 3 + \frac{10}{80} \cdot 3 + \frac{18}{80} \cdot 2 + \frac{19}{80} \cdot 2 + \frac{25}{80} \cdot 2 = 2.225$\\
$\delta = \frac{8}{80} \cdot (3 - 2.225)^2 + \frac{10}{80} \cdot (3 - 2.225)^2 + \frac{18}{80} \cdot (2 - 2.225)^2 + \frac{19}{80} \cdot (2 - 2.225)^2 + \frac{25}{80} \cdot (2 - 2.225)^2 = 0.174375$\\

\begin{center}
    \includegraphics[width=100mm]{huf1}\\
    \includegraphics[width=100mm]{huf2}\\
    \includegraphics[width=100mm]{huf3}\\
\end{center}
\textbf{Вывод: } в ходе лабораторной работы исследовали кодирование по методу Хаффмана. Научились оценивать эффективности кода.

\end{document}