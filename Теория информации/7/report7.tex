\documentclass[a4paper,14pt]{extarticle}


\usepackage[english,russian]{babel}
\usepackage[T2A]{fontenc}
\usepackage[utf8]{inputenc}
\usepackage{ragged2e}
\usepackage[utf8]{inputenc}
\usepackage{hyperref}
\usepackage{minted}
\setmintedinline{frame=lines, framesep=2mm, baselinestretch=1.5, bgcolor=LightGray, breaklines,fontsize=\scriptsize}
\setminted{frame=lines, framesep=2mm, baselinestretch=1.5, bgcolor=LightGray, breaklines,fontsize=\scriptsize}
\usepackage{xcolor}
\definecolor{LightGray}{gray}{0.9}
\usepackage{graphicx}
\usepackage[export]{adjustbox}
\usepackage[left=1cm,right=1cm, top=1cm,bottom=1cm,bindingoffset=0cm]{geometry}
\usepackage{fontspec}
\usepackage{ upgreek }
\usepackage[shortlabels]{enumitem}
\usepackage{adjustbox}
\usepackage{multirow}
\usepackage{amsmath}
\usepackage{amssymb}
\usepackage{pifont}
\usepackage{pgfplots}
\usepackage{longtable}
\usepackage{array}
\graphicspath{ {./images/} }
\makeatletter
\AddEnumerateCounter{\asbuk}{\russian@alph}{щ}
\makeatother
\setmonofont{Consolas}
\setmainfont{Times New Roman}

\newcommand\textbox[1]{
	\parbox{.45\textwidth}{#1}
} 

\newcommand{\specialcell}[2][c]{%
	\begin{tabular}[#1]{@{}c@{}}#2\end{tabular}}

\begin{document}
\pagenumbering{gobble}
\begin{center}
    \small{
        \textbf{МИНИCТЕРCТВО НАУКИ И ВЫCШЕГО ОБРАЗОВАНИЯ РОCCИЙCКОЙ ФЕДЕРАЦИИ}\\
        ФЕДЕРАЛЬНОЕ ГОCУДАРCТВЕННОЕ БЮДЖЕТНОЕ ОБРАЗОВАТЕЛЬНОЕ УЧРЕЖДЕНИЕ\\ВЫCШЕГО ОБРАЗОВАНИЯ \\
        \textbf{«БЕЛГОРОДCКИЙ ГОCУДАРCТВЕННЫЙ ТЕХНОЛОГИЧЕCКИЙ\\УНИВЕРCИТЕТ им. В. Г. ШУХОВА»\\ (БГТУ им. В.Г. Шухова)} \\
        \bigbreak
        \includegraphics[width=70mm]{log}\\
        ИНСТИТУТ ИНФОРМАЦИОННЫХ ТЕХНОЛОГИЙ И УПРАВЛЯЮЩИХ СИСТЕМ\\}
\end{center}

\vfill
\begin{center}
    \large{
        \textbf{
            Лабораторная работа №7}}\\
    \normalsize{
        по дисциплине: Теория информации \\
        тема: «Исследование помехоустойчивых кодов на примере алгоритма Хэмминга»}
\end{center}
\vfill
\hfill\textbox{
    Выполнил: ст. группы ПВ-223\\Пахомов Владислав Андреевич
    \bigbreak
    Проверили: \\пр. Твердохлеб Виталий Викторович
}
\vfill\begin{center}
    Белгород 2024 г.
\end{center}
\newpage
\begin{center}
    \textbf{Лабораторная работа №7}\\
    Исследование помехоустойчивых кодов на примере алгоритма Хэмминга\\
\end{center}
\textbf{Цель работы: }исследовать помехоустойчивые коды на примере алгоритма Хэмминга.\\
Задание 1. Закодировать по Хэммингу произвольно сформированные
последовательности двоичных символов длиной 18 бит, 48 бит.\\
Сообщение длиной 18 бит: 100101010100101001\\
Сообщение длиной 48 бит: 100010100100100101001010001010010101001010001001\\

\begin{center}
\includegraphics[width=140mm]{1_16}\\
\includegraphics[width=200mm]{1_48}
\end{center}
Задание 2. Внести одиночную ошибку и устранить ее, используя механизм
восстановления.
\begin{center}
\includegraphics[width=140mm]{f1_16}\\
\includegraphics[width=200mm]{f1_48}
\end{center}
Задание 3. Рассмотреть вариант кодирования сообщения из 48 бит с предварительным
сегментированием на блоки (размерность блоков выбрать самостоятельно).
\begin{center}
    \includegraphics[width=140mm]{3}
\end{center}
Задание 4. Сравнить режимы кодирования с сегментацией и без.\\
Режим кодирования с сегментированием позволяет обнаружить большее количество ошибок, однако содержит больше 
избыточной информации.

\textbf{Вывод: } в ходе лабораторной работы исследовали помехоустойчивые коды на примере алгоритма Хэмминга.

\end{document}