\documentclass[a4paper,14pt]{extarticle}


\usepackage[english,russian]{babel}
\usepackage[T2A]{fontenc}
\usepackage[utf8]{inputenc}
\usepackage{ragged2e}
\usepackage[utf8]{inputenc}
\usepackage{hyperref}
\usepackage{minted}
\setmintedinline{frame=lines, framesep=2mm, baselinestretch=1.5, bgcolor=LightGray, breaklines,fontsize=\scriptsize}
\setminted{frame=lines, framesep=2mm, baselinestretch=1.5, bgcolor=LightGray, breaklines,fontsize=\scriptsize}
\usepackage{xcolor}
\definecolor{LightGray}{gray}{0.9}
\usepackage{graphicx}
\usepackage[export]{adjustbox}
\usepackage[left=1cm,right=1cm, top=1cm,bottom=1cm,bindingoffset=0cm]{geometry}
\usepackage{fontspec}
\usepackage{ upgreek }
\usepackage[shortlabels]{enumitem}
\usepackage{adjustbox}
\usepackage{multirow}
\usepackage{amsmath}
\usepackage{amssymb}
\usepackage{pifont}
\usepackage{pgfplots}
\usepackage{longtable}
\usepackage{array}
\graphicspath{ {./images/} }
\makeatletter
\AddEnumerateCounter{\asbuk}{\russian@alph}{щ}
\makeatother
\setmonofont{Consolas}
\setmainfont{Times New Roman}

\newcommand\textbox[1]{
	\parbox{.45\textwidth}{#1}
} 

\newcommand{\specialcell}[2][c]{%
	\begin{tabular}[#1]{@{}c@{}}#2\end{tabular}}

\begin{document}
\pagenumbering{gobble}
\begin{center}
    \small{
        \textbf{МИНИCТЕРCТВО НАУКИ И ВЫCШЕГО ОБРАЗОВАНИЯ РОCCИЙCКОЙ ФЕДЕРАЦИИ}\\
        ФЕДЕРАЛЬНОЕ ГОCУДАРCТВЕННОЕ БЮДЖЕТНОЕ ОБРАЗОВАТЕЛЬНОЕ УЧРЕЖДЕНИЕ\\ВЫCШЕГО ОБРАЗОВАНИЯ \\
        \textbf{«БЕЛГОРОДCКИЙ ГОCУДАРCТВЕННЫЙ ТЕХНОЛОГИЧЕCКИЙ\\УНИВЕРCИТЕТ им. В. Г. ШУХОВА»\\ (БГТУ им. В.Г. Шухова)} \\
        \bigbreak
        \includegraphics[width=70mm]{log}\\
        ИНСТИТУТ ИНФОРМАЦИОННЫХ ТЕХНОЛОГИЙ И УПРАВЛЯЮЩИХ СИСТЕМ\\}
\end{center}

\vfill
\begin{center}
    \large{
        \textbf{
            Лабораторная работа №1}}\\
    \normalsize{
        по дисциплине: Базы данных \\
        тема: «Разработка структуры базы данных»}
\end{center}
\vfill
\hfill\textbox{
    Выполнил: ст. группы ПВ-223\\Пахомов Владислав Андреевич
    \bigbreak
    Проверили: \\ст. пр. Панченко Максим Владимирович
}
\vfill\begin{center}
    Белгород 2024 г.
\end{center}
\newpage
\begin{center}
    \textbf{Лабораторная работа №1}\\
    Разработка структуры базы данных\\
    Вариант 8
\end{center}
\textbf{Цель работы: }изучение способов задания инфологической модели данных и создания структуры базы данных в заданной предметной области.
\textbf{Задание: }Жилищная управляющая компания. База данных должна содержать следующие данные: информацию об исполнителях работ и выполненных работах, жильцах, выставленных им счетах и выполненных ими платежах. Предусмотреть возможность анализа следующих показателей: составить рейтинг злостных неплательщиков, рейтинг исполнителей работ с указанием их доли в статье расходов.

\begin{enumerate}[1.]
    \item Выполнить анализ предметной области, выделить основные сущности, атрибуты и связи.\\
          
    \textbf{Дом} - для идентификации дома достаточно его адреса 
    (улица, город, область, страна, номер дома). 
    Также можем внести доп. информацию о доме, 
    например его этажность, индекс, дату введения в эксплуатацию.\\
    \textbf{Задача} - для введения дом в эксплуатацию и его 
    обслуживания необходимо оформить задачу. У дома может быть 
    от 0 до N задач. У задачи может быть уникальный номер, к ней 
    прикреплён дом, рабочие. У каждой хорошей задачи должен быть 
    дедлайн и срок окончания работы в случае выполнения. 
    А у не просто хорошей а замечательной задачи ещё должна быть оплата.\\
    \textbf{Рабочий} - может быть как человеком, так и организацией, 
    поэтому в качестве ключа можем хранить ИНН а также контактные 
    сведения – номер телефона, email. У задачи может быть от 0 до N рабочих.\\
    \textbf{Договор} - договор содержит условия аренды жилья: 
    кол-во ежемесячной платы, срок аренды и т.д. 
    В нашем случае будем ещё и держать номер договора. 
    Договор прикрепляется к дому, так как дом может быть многоквартирным, 
    у дома может быть от 0 до N договоров.\\
    \textbf{Жильцы} - содержат контактные данные (номер телефона, email), 
    ФИО, идентификационные данные – серия и номер паспорта. Может быть 
    прикреплён к нескольким квартирам посредством наличия 
    от 0 до N договоров.\\
    \textbf{Выплаты} - выплата формируется из договора, идентифицируется при 
    помощи УИП, имеет срок оплаты и дату оплаты договора. У договора может быть
    от 0 до N выплат.\\
    Рейтинг жильцов = количество своевременных выплат / общее количество выплат\\
    Рейтинг рабочих = количество вовремя выполненных работ / общее количество работ\\
    Доля в статье расходов = для каждой выполненной работы (оплата / количество участников в работе)\\

    \item Создать диаграмму «сущность — связь» в нотации Чена.\\
    \includegraphics[width=180mm]{db_task2}\\

    \item Самостоятельно изучить нотацию IDEF1X для представления диаграммы «сущность-связь». Создать схему базы данных в нотации IDEF1X. \\
    \includegraphics[width=180mm]{db_task3}\\

    \item Разработать структуру базы данных и составить описание 
    столбцов таблиц базы данных, включающее: имя столбца, 
    назначение (какие данные хранятся), тип данных, допускает ли столбец пустые значения.

    \textbf{home}
    \begin{itemize}
        \item address: string, адрес, первичный ключ
        \item commisioning: date, дата введения, необязательное
        \item floors: number, этажность, обязательное
        \item index: number, индекс, обязательное
    \end{itemize}
    \textbf{contract}
    \begin{itemize}
        \item id: number, номер договора, первичный ключ
        \item transaction\_date: date, дата заключения, обязательное
        \item until\_date: date, договор до, обязательное
        \item payment: number, ежемесячная плата, обязательное
        \item home: string, внешний ключ к home, обязательное
    \end{itemize}
    \textbf{resident}
    \begin{itemize}
        \item passport\_data: string, серия и номер паспорта, первичный ключ
        \item snp: string, ФИО
        \item email: string, эл. почта, необязательное
        \item phone: string, номер телефона, необязательное
    \end{itemize}
    \textbf{residents\_contracts}
    \begin{itemize}
        \item resident\_passport\_data: string, серия и номер паспорта жильца, внешний ключ
        \item contract\_id: number, номер договора, внешний ключ
    \end{itemize}
    \textbf{payment}
    \begin{itemize}
        \item id: string, уип, первичный ключ
        \item paid\_date: date, дата оплаты (если пустой, неоплачен), необязательное
        \item until\_date: date, дедлайн оплаты, обязательное
        \item contract\_id: number, номер договора, внешний ключ
    \end{itemize}
    \textbf{task}
    \begin{itemize}
        \item id: number, номер дела, первичный ключ
        \item payment: number, оплата за задачу, обязательное
        \item until\_date: date, дедлайн задачи, обязательное
        \item completed\_date: date, время выполнения задачи 
        (если пустое, не выполнено), необязательное
        \item home\_address: string, адрес дома, внешний ключ
    \end{itemize}
    \textbf{worker}
    \begin{itemize}
        \item inn: string, ИНН, первичный ключ
        \item email: string, эл. почта, необязательное
        \item phone: string, номер телефона, необязательное
    \end{itemize}
    \textbf{workers\_tasks}
    \begin{itemize}
        \item worker\_inn: string, ИНН, внешний ключ
        \item task\_id: number, номер дела, внешний ключ
    \end{itemize}
\end{enumerate}

\textbf{Вывод: } в ходе лабораторной работы изучили способы задания инфологической модели данных и создания структуры базы данных в заданной предметной области.

\end{document}