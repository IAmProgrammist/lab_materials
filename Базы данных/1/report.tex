\documentclass[a4paper,14pt]{extarticle}


\usepackage[english,russian]{babel}
\usepackage[T2A]{fontenc}
\usepackage[utf8]{inputenc}
\usepackage{ragged2e}
\usepackage[utf8]{inputenc}
\usepackage{hyperref}
\usepackage{minted}
\setmintedinline{frame=lines, framesep=2mm, baselinestretch=1.5, bgcolor=LightGray, breaklines,fontsize=\scriptsize}
\setminted{frame=lines, framesep=2mm, baselinestretch=1.5, bgcolor=LightGray, breaklines,fontsize=\scriptsize}
\usepackage{xcolor}
\definecolor{LightGray}{gray}{0.9}
\usepackage{graphicx}
\usepackage[export]{adjustbox}
\usepackage[left=1cm,right=1cm, top=1cm,bottom=1cm,bindingoffset=0cm]{geometry}
\usepackage{fontspec}
\usepackage{ upgreek }
\usepackage[shortlabels]{enumitem}
\usepackage{adjustbox}
\usepackage{multirow}
\usepackage{amsmath}
\usepackage{amssymb}
\usepackage{pifont}
\usepackage{pgfplots}
\usepackage{longtable}
\usepackage{array}
\graphicspath{ {./images/} }
\makeatletter
\AddEnumerateCounter{\asbuk}{\russian@alph}{щ}
\makeatother
\setmonofont{Consolas}
\setmainfont{Times New Roman}

\newcommand\textbox[1]{
	\parbox{.45\textwidth}{#1}
} 

\newcommand{\specialcell}[2][c]{%
	\begin{tabular}[#1]{@{}c@{}}#2\end{tabular}}

\begin{document}
\pagenumbering{gobble}
\begin{center}
    \small{
        \textbf{МИНИCТЕРCТВО НАУКИ И ВЫCШЕГО ОБРАЗОВАНИЯ РОCCИЙCКОЙ ФЕДЕРАЦИИ}\\
        ФЕДЕРАЛЬНОЕ ГОCУДАРCТВЕННОЕ БЮДЖЕТНОЕ ОБРАЗОВАТЕЛЬНОЕ УЧРЕЖДЕНИЕ\\ВЫCШЕГО ОБРАЗОВАНИЯ \\
        \textbf{«БЕЛГОРОДCКИЙ ГОCУДАРCТВЕННЫЙ ТЕХНОЛОГИЧЕCКИЙ\\УНИВЕРCИТЕТ им. В. Г. ШУХОВА»\\ (БГТУ им. В.Г. Шухова)} \\
        \bigbreak
        \includegraphics[width=70mm]{log}\\
        ИНСТИТУТ ИНФОРМАЦИОННЫХ ТЕХНОЛОГИЙ И УПРАВЛЯЮЩИХ СИСТЕМ\\}
\end{center}

\vfill
\begin{center}
    \large{
        \textbf{
            Лабораторная работа №1}}\\
    \normalsize{
        по дисциплине: Теория автоматов и формальных языков \\
        тема: «Формальные грамматики. Выводы.»}
\end{center}
\vfill
\hfill\textbox{
    Выполнил: ст. группы ПВ-223\\Пахомов Владислав Андреевич
    \bigbreak
    Проверили: \\ст. пр. Рязанов Юрий Дмитриевич
}
\vfill\begin{center}
    Белгород 2024 г.
\end{center}
\newpage
\begin{center}
    \textbf{Лабораторная работа №1}\\
    Формальные грамматики. Выводы.\\
    Вариант 8
\end{center}
\textbf{Цель работы: }изучить основные понятия теории формальных языков и грамматик.

\begin{enumerate}[1.]
    \item Написать программу, выполняющую левый вывод в заданной
          КС-грамматике.\\
          Исходный код:
          \begin{minted}{python3}
from libs.lab1 import solve_context_free_grammar_left, get_available_grammars_left

if __name__ == '__main__':
    grammar_dict = []
    print("Введите КС-грамматику:")
    while len(grammar := input()):
        splitted_grammar = grammar.split(" -> ")
        grammar_dict.append((splitted_grammar[0], splitted_grammar[1]))
    print("Левый вывод: ")
    i = 1
    commands = []
    result = (False, "S", "S")
    while any([v.isupper() for v in result[1]]):
        print(f"Шаг {i}")
        print(f"Промежуточная цепочка: {result[1]}")
        print(f"Можно применить:")
        available_grammars = get_available_grammars_left(grammar_dict, result[1])
        for available_grammar in available_grammars:
            print(f"{available_grammar[0] + 1}. {available_grammar[1][0]} -> {available_grammar[1][1]}")

        while True:
            apply_index = int(input("Введите правило: "))
            result = solve_context_free_grammar_left(grammar_dict, [apply_index - 1], result[1], result[2])
            if not result[0]:
                print("Правило не подходит")
            else:
                break

    print(f"\nТерминальная цепочка: {result[1]}")
    print(f"Последовательность правил: {' '.join((str(command) for command in commands))}")
    print(f"ЛСФ ДВ: {result[2]}")
\end{minted}
          \begin{minted}{python3}
def get_available_grammars_left(grammar_dict, chain):
    available_grammars = []
    min_index = len(chain)
    for i in range(0, len(grammar_dict)):
        grammar = grammar_dict[i]
        if (index := chain.find(grammar[0])) != -1 and index < min_index:
            available_grammars = [(i, grammar)]
            min_index = index
        elif index == min_index:
            available_grammars.append((i, grammar))
    return available_grammars

def solve_context_free_grammar_left(grammar_dict, commands, chain=None, tree=None):
    # Инициализация цепочки, дерева
    if not chain:
        chain = "S"
        tree = "S"
    step = 0

    # Пока в цепочке есть нетерминальные символы и пул правил непустой
    while any([v.isupper() for v in chain]) and step < len(commands):
        # Определяем доступные грамматики
        available_grammars = get_available_grammars_left(grammar_dict, chain)

        # Если правила нет в доступных грамматиках, возвращаем False
        if commands[step] not in [available_grammar[0] for available_grammar in available_grammars]:
            return (False, chain, tree, step)

        # Иначе производим замену для цепочки
        apply_index = commands[step]
        grammar = grammar_dict[apply_index]
        splitted_chain = chain.split(grammar[0])
        chain = splitted_chain[0] + grammar[1] + grammar[0].join(splitted_chain[1:])

        # Производим замену для дерева
        splitted_tree = tree.replace(f"{grammar[0]}(", "*").split(grammar[0])
        tree = (splitted_tree[0] + grammar[0] + "(" + grammar[1] + ")" + grammar[0].join(splitted_tree[1:])).replace(
            "*", f"{grammar[0]}(")

        step += 1

    # Если выполнены не все правила, выходим с ошибкой
    if len(commands) != step:
        return (False, chain, tree, step)

    return (True, chain, tree)
\end{minted}
          Результат выполнения программы:
          \begin{minted}{console}
Введите КС-грамматику:
S -> aAbS
S -> b
A -> SAc
A -> 

Левый вывод: 
Шаг 1
Промежуточная цепочка: S
Можно применить:
1. S -> aAbS
2. S -> b
Введите правило: 1
Шаг 1
Промежуточная цепочка: aAbS
Можно применить:
3. A -> SAc
4. A -> 
Введите правило: 3
Шаг 1
Промежуточная цепочка: aSAcbS
Можно применить:
1. S -> aAbS
2. S -> b
Введите правило: 2
Шаг 1
Промежуточная цепочка: abAcbS
Можно применить:
3. A -> SAc
4. A -> 
Введите правило: 4
Шаг 1
Промежуточная цепочка: abcbS
Можно применить:
1. S -> aAbS
2. S -> b
Введите правило: 2

Терминальная цепочка: abcbb
Последовательность правил: 1 3 2 4 2
ЛСФ ДВ: S(aA(S(b)A()c)bS(b))

Process finished with exit code 0
    \end{minted}





    \item Выполнить левый вывод терминальной цепочки в заданной грамматике, построить дерево вывода.
          Определить, существует ли неэквивалентный вывод полученной цепочки и, если существует, представить его деревом вывода.\bigbreak
          Исходный вывод и дерево:\\
          \includegraphics[width=140mm]{task2_origin}\\
          \includegraphics[width=140mm]{task2_origin_tree}\\
          Неэквивалентный вывод и дерево:\\
          \includegraphics[width=140mm]{task2_alternative}\\
          \includegraphics[width=140mm]{task2_alternative_tree}\\




    \item Написать программу, определяющую, можно ли применить
          заданную последовательность правил при левом выводе
          цепочки в заданной КС-грамматике.
          \begin{minted}{python3}
from libs.lab1 import solve_context_free_grammar_left

if __name__ == '__main__':
    grammar_dict = []
    print("Введите КС-грамматику:")
    while len(grammar := input()):
        splitted_grammar = grammar.split(" -> ")
        grammar_dict.append((splitted_grammar[0], splitted_grammar[1]))
    commands = [int(v) - 1 for v in input("Введите последовательность правил: ").split()]

    result = solve_context_free_grammar_left(grammar_dict, commands)
    if result[0]:
        print("Заданную последовательность правил при левом выводе применить можно.")
    else:
        print("Заданную последовательность правил при левом выводе применить нелзя.")
        print(f"В промежуточной цепочке {result[1]} нельзя применить "
              f"правило {commands[result[3]] + 1}. {grammar_dict[commands[result[3]]][0]} -> {grammar_dict[commands[result[3]]][1]} "
              f"под шагом {result[3] + 1}.")

          \end{minted}
          Результат выполнения программы:
          \begin{minted}{console}
Введите КС-грамматику:
S -> aAbS
S -> b
A -> SAc
A -> 

Введите последовательность правил: 1 3 2 1 4
Заданную последовательность правил при левом выводе применить нелзя.
В промежуточной цепочке abAcbS нельзя применить правило 1. S -> aAbS под шагом 4.

Process finished with exit code 0
  \end{minted}





    \item Для каждой последовательности правил определить,
          можно ли её применить при левом (правом) выводе терминальной цепочки
          в заданной КС-грамматике, и, если можно, построить дерево вывода.\\
          \includegraphics[width=140mm]{task4_1}\\
          \includegraphics[width=140mm]{task4_2}\\
          \includegraphics[width=140mm]{task4_3}\\
          \includegraphics[width=140mm]{task4_4}\\

    \item Написать программу, определяющую, можно ли применить заданную
          последовательность правил при выводе цепочки в заданной
          КС-грамматике.
          Исходный код:
          \begin{minted}{python3}
from libs.lab1 import solve_context_free_grammar

if __name__ == '__main__':
    grammar_dict = []
    print("Введите КС-грамматику:")
    while len(grammar := input()):
        splitted_grammar = grammar.split(" -> ")
        grammar_dict.append((splitted_grammar[0], splitted_grammar[1]))
    commands = [int(v) - 1 for v in input("Введите команды: ").split()]

    print(f"Заданную последовательность правил применить {'можно' if any(solve_context_free_grammar(grammar_dict, commands)) else 'нельзя'}")
\end{minted}
          \begin{minted}{python3}
def solve_context_free_grammar(grammar_dict, rules, chain=None, tree=None, step=0):
    """
    Рекурсивная проверка последовательности правил в грамматике
    
    1. Если в последовательности правил больше не осталось шагов, выходим из рекурсии.
    2. Для всех нетерминальных символов в промежуточной цепочке находим правила, которые можно применить.
    3. Если правила на текущем шаге не окажется в полученном списке правил, значит выходим из рекурсии.
    4. Иначе получим непосредственный вывод для правила для каждого нетерминала в промежуточной цепочке и перейдём к сдедующему шагу рекурсии.
    Например для промежуточной цепочки aSbScSi и правила S -> h будут получены следующие непосредтвенные выводы:
    * aSbScSi => ahbScSi => (рекурсивный вызов)
    * aSbScSi => aSbhcSi => (рекурсивный вызов)
    * aSbScSi => aSbSchi => (рекурсивный вызов)
    """

    # Если в последовательности правил больше не осталось шагов, выходим из рекурсии
    if step >= len(rules):
        yield (True, chain, tree)
    else:
        # Инициализация цепочки, дерева
        if not chain:
            chain = "S"
            tree = "S"

        # Для всех нетерминальных символов в промежуточной цепочке находим правила, которые можно применить
        available_rules = []
        for i in range(0, len(grammar_dict)):
            grammar = grammar_dict[i]
            if grammar[0] in chain:
                available_rules.append((i, grammar))

        # Если правила на текущем шаге не окажется в полученном списке правил, значит выходим из рекурсии
        if rules[step] not in [available_grammar[0] for available_grammar in available_rules]:
            pass
        else:
            # Иначе получим непосредственный вывод для правила для каждого нетерминала в промежуточной цепочке и перейдём к сдедующему шагу рекурсии
            grammar = grammar_dict[rules[step]]
            splitted_chain = chain.split(grammar[0])
            splitted_tree = tree.replace(f"{grammar[0]}(", "*").split(grammar[0])

            for i in range(0, len(splitted_chain) - 1):
                # Применяем правило для цепочки
                new_chain = grammar[0].join(splitted_chain[:i+1]) + grammar[1] + grammar[0].join(splitted_chain[i+1:])

                # Применяем правило для дерева
                new_tree = (grammar[0].join(splitted_tree[:i+1]) +
                            grammar[0] + "(" + grammar[1] + ")" +
                            grammar[0].join(splitted_tree[i+1:])).replace("*", f"{grammar[0]}(")

                # Вызываем рекурентно функцию для обработки следующего шага
                yield from solve_context_free_grammar(grammar_dict, rules, new_chain, new_tree, step + 1)

\end{minted}
          Результат выполнения программы:
          \begin{minted}{console}
Введите КС-грамматику:
S -> aAbS
S -> b
A -> SAc
A -> 

Введите команды: 1 3 2 1 4
Заданную последовательность правил применить можно

Process finished with exit code 0
    \end{minted}
    \item Для каждой последовательности правил (см. варианты заданий
          п.2) определить, можно ли её применить при выводе терминальной
          цепочки в заданной КС-грамматике, и, если можно, построить дерево
          вывода и записать эквивалентные левый и правый вывод.\\

          Пункт 1:\\
          \includegraphics[width=140mm]{task6_1_1}\\
          \includegraphics[width=140mm]{task6_1_2}\bigbreak
          Пункт 2:\\
          \includegraphics[width=140mm]{task6_2}\bigbreak
          Пункт 3:\\
          \includegraphics[width=140mm]{task6_3}\bigbreak
          Пункт 4:\\
          \includegraphics[width=140mm]{task6_4}\bigbreak
\end{enumerate}

\textbf{Вывод: } в ходе лабораторной работы изучили основные понятия теории формальных языков и грамматик.

\end{document}