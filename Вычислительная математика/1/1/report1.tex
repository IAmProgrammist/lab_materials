\documentclass[a4paper,14pt]{extarticle}


\usepackage[english,russian]{babel}
\usepackage[T2A]{fontenc}
\usepackage[utf8]{inputenc}
\usepackage{ragged2e}
\usepackage[utf8]{inputenc}
\usepackage{hyperref}
\usepackage{minted}
\setmintedinline{frame=lines, framesep=2mm, baselinestretch=1.5, bgcolor=LightGray, breaklines,fontsize=\scriptsize}
\setminted{frame=lines, framesep=2mm, baselinestretch=1.5, bgcolor=LightGray, breaklines,fontsize=\scriptsize}
\usepackage{xcolor}
\definecolor{LightGray}{gray}{0.9}
\usepackage{graphicx}
\usepackage[export]{adjustbox}
\usepackage[left=1cm,right=1cm, top=1cm,bottom=1cm,bindingoffset=0cm]{geometry}
\usepackage{fontspec}
\usepackage{ upgreek }
\usepackage[shortlabels]{enumitem}
\usepackage{adjustbox}
\usepackage{multirow}
\usepackage{amsmath}
\usepackage{amssymb}
\usepackage{pifont}
\usepackage{pgfplots}
\usepackage{longtable}
\usepackage{array}
\graphicspath{ {./images/} }
\makeatletter
\AddEnumerateCounter{\asbuk}{\russian@alph}{щ}
\makeatother
\setmonofont{Consolas}
\setmainfont{Times New Roman}

\newcommand\textbox[1]{
	\parbox{.45\textwidth}{#1}
} 

\newcommand{\specialcell}[2][c]{%
	\begin{tabular}[#1]{@{}c@{}}#2\end{tabular}}

\begin{document}
\pagenumbering{gobble}
\begin{center}
    \small{
        \textbf{МИНИCТЕРCТВО НАУКИ И ВЫCШЕГО ОБРАЗОВАНИЯ РОCCИЙCКОЙ ФЕДЕРАЦИИ}\\
        ФЕДЕРАЛЬНОЕ ГОCУДАРCТВЕННОЕ БЮДЖЕТНОЕ ОБРАЗОВАТЕЛЬНОЕ УЧРЕЖДЕНИЕ\\ВЫCШЕГО ОБРАЗОВАНИЯ \\
        \textbf{«БЕЛГОРОДCКИЙ ГОCУДАРCТВЕННЫЙ ТЕХНОЛОГИЧЕCКИЙ\\УНИВЕРCИТЕТ им. В. Г. ШУХОВА»\\ (БГТУ им. В.Г. Шухова)} \\
        \bigbreak
        \includegraphics[width=70mm]{log}\\
        ИНСТИТУТ ИНФОРМАЦИОННЫХ ТЕХНОЛОГИЙ И УПРАВЛЯЮЩИХ СИСТЕМ\\}
\end{center}

\vfill
\begin{center}
    \large{
        \textbf{
            Лабораторная работа №1}}\\
    \normalsize{
        по дисциплине: Вычислительная математика \\
        тема: «Погрешности. Приближенные вычисления. Вычислительная устойчивость.»}
\end{center}
\vfill
\hfill\textbox{
    Выполнил: ст. группы ПВ-223\\Пахомов Владислав Андреевич
    \bigbreak
    Проверили: \\пр. Четвертухин Виктор Романович
}
\vfill\begin{center}
    Белгород 2024 г.
\end{center}
\newpage
\begin{center}
    \textbf{Лабораторная работа №1}\\
    «Погрешности. Приближенные вычисления. Вычислительная устойчивость.\\
    Вариант 10
\end{center}
\textbf{Цель работы: }Изучить особенности организации вычислительных процессов, связанные с
погрешностями, приближенным характером вычислений на компьютерах современного типа,
вычислительной устойчивостью.

\begin{enumerate}[1.]
    \item Запустить и проинтерпретировать результаты работы разных вычислительных схем для
    простого арифметического выражения на языке Rust.\\
    \begin{minted}{Rust}
fn main() {
    let num1: f32 = 0.23456789;
    let num2: f32 = 1.5678e+20;
    let num3: f32 = 1.2345e+10;
    let result1 = (num1 * num2) / num3;
    let result2 = (num1 / num3) * num2;
    let result3: f64 = num1 as f64 * num2 as f64 / num3 as f64;
    println!("({} * {}) / {} = {}", num1, num2, num3, result1);
    println!("({} / {}) * {} = {}", num1, num3, num2, result2);
    println!(" {} * {} / {} = {}", num1, num2, num3, result3);
}
    \end{minted}
    Результаты работы:
    \begin{minted}{console}
(0.2345679 * 156780000000000000000) / 12345000000 = 2978983700
(0.2345679 / 12345000000) * 156780000000000000000 = 2978984000
 0.2345679 * 156780000000000000000 / 12345000000 = 2978983717.267449
    \end{minted} 
    
    \item Запустить и проинтерпретировать результаты работы разных вычислительных схем для
    интерационного и неитерационного вычисления на языке Rust.\\
    \begin{minted}{Rust}
// демонстрация накопления погрешности для итерационного процесса
// версия для одинарной точности
fn iter(numbers: &[f32], iterations: i32) {
    println!("Итерационный метод: ");
    for &number in numbers {
        let mut result = number;
        for _ in 0..iterations {
            result = result.sqrt(); // послед. извлечение квадратного корня
        }
        for _ in 0..iterations {
            result = result * result; // послед. возведение числа в квадрат
        }
        let error = (number - result).abs();
        println!(
            "Исх-е значение: {:e}, результат: {:e}, абс-ая погрешность:
    {:e}, отн-ая погрешность: {:e} (%)",
            number,
            result,
            error,
            error * 100. / number
        );
    }
}

// замена итерации функцией
// версия для одинарной точности c powf
fn non_iter(numbers: &[f32], iterations: i32) {
    println!("Безытерационный метод: ");
    for &number in numbers {
        // извлекаем корень
        let intermediate = number.powf(1.0f32 / (1 << iterations) as f32);
        // восстанавливаем значение
        let result = intermediate.powf((1 << iterations) as f32);
        let error = (number - result).abs();
        println!(
            "Исх-е значение: {:e}, результат: {:e}, абс-ая погрешность:
    {:e}, отн-ая погрешность: {:e} (%)",
            number,
            result,
            error,
            error * 100. / number
        );
    }
}

fn main() {
    let numbers = [
        1.0f32,
        20.,
        300.,
        4000.,
        5e6,
        f32::MIN_POSITIVE,
        f32::MAX * 0.99,
    ]; // вектор с числами одинарной точности

    let iterations = 10; // число итераций
    
    iter(&numbers, iterations);
    non_iter(&numbers, iterations)
}
    \end{minted}
    Результаты работы:
    \begin{minted}{console}
Итерационный метод: 
Исх-е значение: 1e0, результат: 1e0, абс-ая погрешность:
    0e0, отн-ая погрешность: 0e0 (%)
Исх-е значение: 2e1, результат: 2.000009e1, абс-ая погрешность:
    8.9645386e-5, отн-ая погрешность: 4.4822693e-4 (%)
Исх-е значение: 3e2, результат: 3.0001422e2, абс-ая погрешность:
    1.4221191e-2, отн-ая погрешность: 4.740397e-3 (%)
Исх-е значение: 4e3, результат: 4.0001064e3, абс-ая погрешность:
    1.0644531e-1, отн-ая погрешность: 2.6611327e-3 (%)
Исх-е значение: 5e6, результат: 4.9994865e6, абс-ая погрешность:
    5.135e2, отн-ая погрешность: 1.027e-2 (%)
Исх-е значение: 1.1754944e-38, результат: 1.17548e-38, абс-ая погрешность:
    1.43e-43, отн-ая погрешность: 1.2159348e-3 (%)
Исх-е значение: 3.3687953e38, результат: 3.3686973e38, абс-ая погрешность:
    9.796404e33, отн-ая погрешность: 2.9079844e-3 (%)
Безытерационный метод: 
Исх-е значение: 1e0, результат: 1e0, абс-ая погрешность:
    0e0, отн-ая погрешность: 0e0 (%)
Исх-е значение: 2e1, результат: 2.0000069e1, абс-ая погрешность:
    6.866455e-5, отн-ая погрешность: 3.4332275e-4 (%)
Исх-е значение: 3e2, результат: 3.0000873e2, абс-ая погрешность:
    8.728027e-3, отн-ая погрешность: 2.9093425e-3 (%)
Исх-е значение: 4e3, результат: 4.0001143e3, абс-ая погрешность:
    1.1425781e-1, отн-ая погрешность: 2.8564453e-3 (%)
Исх-е значение: 5e6, результат: 5.000186e6, абс-ая погрешность:
    1.86e2, отн-ая погрешность: 3.72e-3 (%)
Исх-е значение: 1.1754944e-38, результат: 1.175497e-38, абс-ая погрешность:
    2.7e-44, отн-ая погрешность: 2.2649765e-4 (%)
Исх-е значение: 3.3687953e38, результат: 3.3687553e38, абс-ая погрешность:
    3.9956347e33, отн-ая погрешность: 1.1860722e-3 (%)
    \end{minted} 
    

    \item С помощью программы на языке Rust вывести на экран двоичное представление
    машинных чисел одинарной точности стандарта IEEE 754 для записи: числа π,
    бесконечности, нечисла (NaN), наименьшего положительного числа, наибольшего
    положительного числа, наименьшего отрицательного числа. Сформулировать обоснование
    полученных результатов в пунктах 1 и 2, опираясь на двоичное представление машинных
    чисел.\\

    \begin{minted}{Rust}
fn main() {
    let pi = std::f32::consts::PI;
    let infinity = std::f32::INFINITY;
    let nan = std::f32::NAN;
    let smallest_positive = std::f32::MIN_POSITIVE;
    let largest_positive = std::f32::MAX;
    let smallest_negative = -std::f32::MIN_POSITIVE;
    println!("π: {}", float_to_binary_string(pi));
    println!("Infinity: {}", float_to_binary_string(infinity));
    println!("NaN: {}", float_to_binary_string(nan));
    println!(
        "Smallest Positive Number: {}",
        float_to_binary_string(smallest_positive)
    );
    println!(
        "Largest Positive Number: {}",
        float_to_binary_string(largest_positive)
    );
    println!(
        "Smallest Negative Number: {}",
        float_to_binary_string(smallest_negative)
    );
}
fn float_to_binary_string(num: f32) -> String {
    let bits = num.to_bits();
    format!("{:032b}", bits)
}
            \end{minted}
            Результаты работы:
            \begin{minted}{console}
π: 01000000010010010000111111011011
Infinity: 01111111100000000000000000000000
NaN: 01111111110000000000000000000000
Smallest Positive Number: 00000000100000000000000000000000
Largest Positive Number: 01111111011111111111111111111111
Smallest Negative Number: 10000000100000000000000000000000
            \end{minted} 
            Первое задание: \\
            Точность вычислений зависит от порядка операций. Мантисса числа ограничена, и из-за этого последние цифры вещественных чисел
    могут округляться некорректно. В первом и втором примере мы получили различные варианты ответов в зависимости от порядка выполнения операций. 
    На порядок и размер чисел стоит обращать внимание, когда порядок может превысить размеры порядка в памяти, тогда мы получим бесконечность
    или минус бесконечность. В этом случае первый вариант намного логичней, так как порядок числа после умножения числа с отрицательной
    экспонентой и положительной не будет давать экспоненты больше, чем может содержать память. В третьем примере вещественные числа, 
    хранящиеся в 32-битной схеме расширяются до 64-битной, что увеличивает размер мантиссы и, следовательно, точность вычислений.\\
    Второе задание: \\
    Итерационный метод в целом имеет большую относительную погрешность из-за постепенного накопления ошибок в младших битах при выполнении множественных действий, 
    безытерационный метод лишён таких недостатков, и выполняет действия над числами намного реже, что сокращает кол-во ошибок при сокращении.\\
\item Подобрать такие входные данные, что в первом случае схема
демонстрировала бы заметную потерю в точности, а вторая на тех же
входных данных — улучшала бы результат.\\
\begin{minted}{Rust}
pub fn make_computation(principal: f32, rate: f32, periods: f32) -> f32 {
    principal * (1.0 + rate).powf(periods)
}

pub fn make_improved_computation(principal: f32, rate: f32, periods: f32) -> f32 {
    principal * (periods * (1.0 + rate).ln()).exp()
}
            \end{minted}

\begin{minted}{Rust}
extern crate algr;

use algr::lab1::task4::{make_computation, make_improved_computation};

fn main() {
    let pri = 1.0f32;
    let rat = -0.9999f32;
    let per = 4.0f32;

    println!("{:.32}", make_computation(pri, rat, per));
    println!("{:.32}", make_improved_computation(pri, rat, per));
    println!("{:.32}", make_computation(pri, rat, per) - make_improved_computation(pri, rat, per));
}
\end{minted}
            Результаты работы:
            \begin{minted}{console}
0.00000000000000010006639451092431
0.00000000000000010006634818881001
0.00000000000000000000004632211430
            \end{minted} 
\end{enumerate}

\textbf{Вывод: } в ходе лабораторной работы изучили особенности организации вычислительных процессов, связанные с
погрешностями, приближенным характером вычислений на компьютерах современного типа,
вычислительной устойчивостью. Предложенный улушченный метод вычисления выражения в большинстве случаев 
оказывается менее точным, чем его обычная версия, так как содержит множество дополнительных действий, 
что увеличивает количество ошибок при сокращении.

\end{document}