\documentclass[a4paper,14pt]{extarticle}


\usepackage[english,russian]{babel}
\usepackage[T2A]{fontenc}
\usepackage[utf8]{inputenc}
\usepackage{ragged2e}
\usepackage[utf8]{inputenc}
\usepackage{hyperref}
\usepackage{minted}
\usepackage{xcolor}
\definecolor{LightGray}{gray}{0.9}
\usepackage{graphicx}
\usepackage[export]{adjustbox}
\graphicspath{ {./images/} }
\usepackage[left=1cm,right=1cm, top=1cm,bottom=1cm,bindingoffset=0cm]{geometry}
\usepackage{fontspec}


\newcommand\textbox[1]{
    \parbox{.45\textwidth}{#1}
}

\newcommand{\header}[7]{
    \begin{center}
        \small{
            МИНИСТЕРСТВО НАУКИ И ВЫСШЕГО ОБРАЗОВАНИЯ \\РОССИЙСКОЙ ФЕДЕРАЦИИ
            \bigbreak
            ФЕДЕРАЛЬНОЕ ГОСУДАРСТВЕННОЕ БЮДЖЕТНОЕ ОБРАЗОВАТЕЛЬНОЕ УЧРЕЖДЕНИЕ ВЫСШЕГО ОБРАЗОВАНИЯ \\
            \bigbreak
            \textbf{«БЕЛГОРОДСКИЙ ГОСУДАРСТВЕННЫЙ \\ТЕХНОЛОГИЧЕСКИЙ УНИВЕРСИТЕТ им. В. Г. ШУХОВА»\\ (БГТУ им. В.Г. Шухова)} \\
            \bigbreak
            Кафедра программного обеспечения вычислительной техники и автоматизированных систем\\}
    \end{center}

    \vfill
    \begin{center}
        \large{
            \textbf{
                Лабораторная работа №#1 }}\\
        \normalsize{
            по дисциплине: #2 \\
            тема: «#3»}
    \end{center}
    \vfill
    \hfill\textbox{
        Выполнил: #4
        \bigbreak
        Проверили: #5
        \bigbreak
        Код-ревьер: #6
    }
    \vfill\begin{center}
              Белгород #7 г.
    \end{center}
    \newpage
}
\newcommand{\content}[4]{
    \justifying
    \begin{center}
        \large{
            \textbf{
                Лабораторная работа №#1
            }
        }\\
    \end{center}
    \textbf {
        Содержание отчёта:
    }
    #2
    \textbf{Тема лабораторной работы: }#3\\
    \textbf{Цель лабораторной работы: }#4
    \newpage
}
\newcommand\codeforcesresult[1]{
    \begin{center}
        \includegraphics[width=\linewidth]{#1}
    \end{center}
}

\setmonofont{Consolas}
\setmainfont{Times New Roman}

\begin{document}

    \pagenumbering{gobble}

    \header{10}{Основы программирования}{Бинарный поиск}{ст. группы ПВ-223\\Пахомов Владислав Андреевич}{Притчин Иван Сергеевич\\ Черников Сергей Викторович}{ст. группы ПВ-223\\ Голуцкий Георгий Юрьевич}{2023}
\content{10\\Вариант 1}{\begin{itemize}
    \item Тема лабораторной работы
\item Цель лабораторной работы
\item Решения задач
\begin{itemize}
    \item Название задачи.
\item Исходный код.
\item Вердикт тестирующей системы.
\item Задачи с двумя звёздочками не являются обязательными.

\end{itemize}
\item Работа над ошибками (код-ревью)
\item Вывод по работе

\end{itemize}}{Бинарный поиск}{получение навыков использования алгоритмов бинарного поиска для решения задач оптимизации.}
\begin{enumerate}
    \item \href{https://codeforces.com/edu/course/2/lesson/6/1/practice/contest/283911/problem/A}{\textbf{Двоичный поиск}}
\begin{minted}[frame=lines, framesep=2mm, baselinestretch=1.2, bgcolor=LightGray, breaklines, fontsize=\small]{C}
#include "stdio.h"
#include "stdlib.h"

int main() {
    int arraySize, requestAmount;
    scanf("%d %d", &arraySize, &requestAmount);

    int *array = (int *) malloc(sizeof(int) * arraySize);
    for (int i = 0; i < arraySize; i++)
        scanf("%d", array + i);

    for (int i = 0; i < requestAmount; i++) {
        int requestedNumber;
        scanf("%d", &requestedNumber);

        int left = -1;
        int right = arraySize;

        while (right - left > 1) {
            int middleIndex = left + (right - left) / 2;
            if (array[middleIndex] < requestedNumber)
                left = middleIndex;
            else
                right = middleIndex;
        }

        if (array[right] == requestedNumber)
            printf("YES\n");
        else
            printf("NO\n");
    }

    free(array);

    return 0;
}
\end{minted}
\codeforcesresult{/codeforceresults/283911A}
\href{https://github.com/IAmProgrammist/programming-and-algorithmization-basics/blob/c/lab10/1.c}{\underline{Ссылка на репозиторий}}

\newpage
\item \href{https://codeforces.com/edu/course/2/lesson/6/1/practice/contest/283911/problem/D}{\textbf{Быстрый поиск в массиве}}
\begin{minted}[frame=lines, framesep=2mm, baselinestretch=1.2, bgcolor=LightGray, breaklines, fontsize=\small]{C}
#include "stdio.h"
#include "stdlib.h"

int basicNumComparator(const void *a, const void *b) {
    if (*((int *) a) > *((int *) b)) return 1;
    if (*((int *) a) < *((int *) b)) return -1;
    return 0;
}

int main() {
    int arraySize;
    scanf("%d", &arraySize);

    int *array = (int *) malloc(sizeof(int) * arraySize);
    for (int i = 0; i < arraySize; i++)
        scanf("%d", array + i);

    qsort(array, arraySize, sizeof(array[0]), basicNumComparator);

    int requestAmount;
    scanf("%d", &requestAmount);

    for (int i = 0; i < requestAmount; i++) {
        int requestedLeftBorder, requestedRightBorder;
        scanf("%d %d", &requestedLeftBorder, &requestedRightBorder);

        int left = -1;
        int right = arraySize;

        while (right - left > 1) {
            int middleIndex = left + (right - left) / 2;
            if (array[middleIndex] < requestedLeftBorder)
                left = middleIndex;
            else
                right = middleIndex;
        }

        int leftBorder = right;
        right = arraySize;

        while (right - left > 1) {
            int middleIndex = left + (right - left) / 2;
            if (array[middleIndex] <= requestedRightBorder)
                left = middleIndex;
            else
                right = middleIndex;
        }

        printf("%d\n", right - leftBorder);
    }

    free(array);

    return 0;
}
\end{minted}
\codeforcesresult{/codeforceresults/283911D}
\href{https://github.com/IAmProgrammist/programming-and-algorithmization-basics/blob/c/lab10/4.c}{\underline{Ссылка на репозиторий}}

\newpage
\item \href{https://codeforces.com/edu/course/2/lesson/6/2/practice/contest/283932/problem/C}{\textbf{Очень Легкая Задача}}
\begin{minted}[frame=lines, framesep=2mm, baselinestretch=1.2, bgcolor=LightGray, breaklines, fontsize=\small]{C}
#include "stdio.h"

int printedPages(int x, int y, int time) {
    return time / x + time / y;
}

int main() {
    int copiesAmount, xVelocity, yVelocity;
    scanf("%d %d %d", &copiesAmount, &xVelocity, &yVelocity);

    int left = -1;
    int right = xVelocity * copiesAmount;
    copiesAmount--;

    while (right - left > 1) {
        int middleTime = left + (right - left) / 2;
        if (printedPages(xVelocity, yVelocity, middleTime) < copiesAmount)
            left = middleTime;
        else
            right = middleTime;
    }

    right += xVelocity < yVelocity ? xVelocity : yVelocity;

    printf("%d", right);

    return 0;
}
\end{minted}
\codeforcesresult{/codeforceresults/283932A}
\href{https://github.com/IAmProgrammist/programming-and-algorithmization-basics/blob/c/lab10/6.c}{\underline{Ссылка на репозиторий}}

\newpage
\item \href{https://codeforces.com/problemset/problem/192/A}{\textbf{Модные числа}}
\begin{minted}[frame=lines, framesep=2mm, baselinestretch=1.2, bgcolor=LightGray, breaklines, fontsize=\small]{C}
#include <stdbool.h>
#include "stdio.h"

long long triangleNum(long long num) {
    return num * (num + 1) / 2;
}

int main() {
    long long num;
    scanf("%lld", &num);

    bool result = false;
    for (long long k = 1; (triangleNum(k) < num) && !result; k++) {
        long long secondTriangleNum = num - triangleNum(k);

        long long left = k - 1;
        long long right = num;

        while (right - left > 1) {
            long long middleNum = left + (right - left) / 2;

            if (triangleNum(middleNum) < secondTriangleNum)
                left = middleNum;
            else
                right = middleNum;
        }

        if (secondTriangleNum - triangleNum(right) == 0)
            result = true;
    }

    if (result)
        printf("YES");
    else
        printf("NO");

    return 0;
}
\end{minted}
\codeforcesresult{/codeforceresults/192A}
\href{https://github.com/IAmProgrammist/programming-and-algorithmization-basics/blob/c/lab10/12.c}{\underline{Ссылка на репозиторий}}

\newpage
\item \href{https://codeforces.com/contest/996/problem/B}{\textbf{** Чемпионат мира}}
\begin{minted}[frame=lines, framesep=2mm, baselinestretch=1.2, bgcolor=LightGray, breaklines, fontsize=\small]{C}
#include <stdio.h>

int timePassed(int gate, int lapsDone, int fanZonesTotal) {
    return gate + lapsDone * fanZonesTotal;
}

int main() {
    int fanZones;
    scanf("%d", &fanZones);

    int minimalTime = -1;
    int gate = 0;
    for (int i = 0; i < fanZones; i++) {
        int order;
        scanf("%d", &order);

        int left = -1;
        int right = (order - i) / fanZones + 1;

        while (right - left > 1) {
            int middle = left + (right - left) / 2;

            if (timePassed(i, middle, fanZones) < order)
                left = middle;
            else
                right = middle;
        }

        int time = timePassed(i, right, fanZones);

        if (time < minimalTime || gate == 0) {
            minimalTime = time;
            gate = i + 1;
        }
    }

    printf("%d", gate);
}
\end{minted}
\codeforcesresult{/codeforceresults/996B}
\href{https://github.com/IAmProgrammist/programming-and-algorithmization-basics/blob/c/lab10/14.c}{\underline{Ссылка на репозиторий}}

\newpage
\item \href{https://codeforces.com/problemset/problem/1201/C}{\textbf{** Максимальная медиана}}
\begin{minted}[frame=lines, framesep=2mm, baselinestretch=1.2, bgcolor=LightGray, breaklines, fontsize=\small]{C}
#include <stdio.h>
#include <stdlib.h>

int numComparator(const void *a, const void *b) {
    if (*((long long *) a) > *((long long *) b)) return 1;
    if (*((long long *) a) < *((long long *) b)) return -1;
    return 0;
}

long long max(long long a, long long b) {
    return a > b ? a : b;
}

long long countRequiredSum(long long newNumber, long long const *const numberArray, int arraySize) {
    long long requiredSum = 0;
    for (int i = arraySize / 2; i < arraySize; i++)
        requiredSum += max(0, newNumber - numberArray[i]);

    return requiredSum;
}

int main() {
    int n;
    long long k;
    scanf("%d %lld", &n, &k);

    long long *array = (long long *) malloc(sizeof(long long) * n);
    for (int i = 0; i < n; i++)
        scanf("%lld", array + i);

    qsort(array, n, sizeof(array[0]), numComparator);

    long long left = -1;
    long long right = 2 * 1000000001;

    while (right - left > 1) {
        long long middleNum = left + (right - left) / 2;

        if (countRequiredSum(middleNum, array, n) <= k)
            left = middleNum;
        else
            right = middleNum;
    }

    free(array);

    printf("%lld", left);
}
\end{minted}
\codeforcesresult{/codeforceresults/1201С}
\href{https://github.com/IAmProgrammist/programming-and-algorithmization-basics/blob/c/lab10/15.c}{\underline{Ссылка на репозиторий}}

\newpage
\item \href{https://codeforces.com/edu/course/2/lesson/6/3/practice/contest/285083/problem/B}{\textbf{** Разделение массива}}
\begin{minted}[frame=lines, framesep=2mm, baselinestretch=1.2, bgcolor=LightGray, breaklines, fontsize=\small]{C}
#include <stdio.h>
#include <stdlib.h>
#include <stdbool.h>

long long max(long long a, long long b) {
    return a > b ? a : b;
}

long long countCuts(long long sectionMaxSum, long long const *const numberArray, int arraySize) {
    long long tempSum = 0, cuts = 0;

    for (int i = 0; i < arraySize; i++) {
        if (tempSum + numberArray[i] > sectionMaxSum) {
            cuts++;
            tempSum = 0;
        }

        if (numberArray[i] > sectionMaxSum) {
            cuts = -1;
            break;
        }

        tempSum += numberArray[i];
    }

    return cuts;
}

int main() {
    int n;
    long long k;
    scanf("%d %lld", &n, &k);

    long long left = -1;
    long long right = 0;

    long long *array = (long long *) malloc(sizeof(long long) * n);
    for (int i = 0; i < n; i++) {
        scanf("%lld", array + i);
        right += array[i];
    }

    while (right - left > 1) {
        long long middleNum = left + (right - left) / 2;

        long long cuts = countCuts(middleNum, array, n);
        if (cuts < k && cuts != -1)
            right = middleNum;
        else
            left = middleNum;
    }

    free(array);

    printf("%lld", right);
}
\end{minted}
\codeforcesresult{/codeforceresults/285083B}
\href{https://github.com/IAmProgrammist/programming-and-algorithmization-basics/blob/c/lab10/16.c}{\underline{Ссылка на репозиторий}}

\newpage
\item \href{https://codeforces.com/edu/course/2/lesson/6/2/practice/contest/283932/problem/H}{\textbf{** Гамбургеры}}
\begin{minted}[frame=lines, framesep=2mm, baselinestretch=1.2, bgcolor=LightGray, breaklines, fontsize=\small]{C}
#include <stdio.h>
#include <string.h>

#define MAX_RECIPE_LENGTH 100

long long max(long long a, long long b) {
    return a > b ? a : b;
}

long long moneyRequire(long long int hamburgerAmount, int breadInHamburger, int sausageInHamburger, int cheeseInHamburger,
             int breadAmount, int sausageAmount, int cheeseAmount,
             int breadPrice, int sausagePrice, int cheesePrice) {
    return max(hamburgerAmount * breadInHamburger - breadAmount, 0) * breadPrice +
           max(hamburgerAmount * sausageInHamburger - sausageAmount, 0) * sausagePrice +
           max(hamburgerAmount * cheeseInHamburger - cheeseAmount, 0) * cheesePrice;
}

void getRecipe(char *recipe, int *nBread, int *nSausage, int *nCheese) {
    // strlen - функция, определенная в string.h
    // для вычисления длины строки
    int nIngredients = strlen(recipe);
    *nBread = 0;
    *nSausage = 0;
    *nCheese = 0;
    for (int ingredientIndex = 0; ingredientIndex < nIngredients;
         ingredientIndex++) {
        switch (recipe[ingredientIndex]) {
            case 'B':
                (*nBread)++;
                break;
            case 'S':
                (*nSausage)++;
                break;
            case 'C':
                (*nCheese)++;
                break;
        }
    }
}

int main() {
    char recipe[MAX_RECIPE_LENGTH + 1]; // +1 - под ноль-символ
    int nBread, nSausage, nCheese;
    gets(recipe);
    getRecipe(recipe, &nBread, &nSausage, &nCheese);

    int bAmount, sAmount, cAmount;
    scanf("%d %d %d", &bAmount, &sAmount, &cAmount);

    int bPrice, sPrice, cPrice;
    scanf("%d %d %d", &bPrice, &sPrice, &cPrice);

    long long money;
    scanf("%lld", &money);

    long long left = -1;
    long long right = 100000000000000;

    while (right - left > 1) {
        long long middle = left + (right - left) / 2;

        if (moneyRequire(middle, nBread, nSausage, nCheese,
                         bAmount, sAmount, cAmount,
                         bPrice, sPrice, cPrice) <= money)
            left = middle;
        else
            right = middle;
    }

    printf("%lld", left);

    return 0;
}


\end{minted}
\codeforcesresult{/codeforceresults/283932H}
\href{https://github.com/IAmProgrammist/programming-and-algorithmization-basics/blob/c/lab10/17.c}{\underline{Ссылка на репозиторий}}

\newpage
\item \href{https://codeforces.com/problemset/problem/1574/C}{\textbf{** Slay the Dragon}}
\begin{minted}[frame=lines, framesep=2mm, baselinestretch=1.2, bgcolor=LightGray, breaklines, fontsize=\small]{C}
#include <stdio.h>
#include <stdlib.h>

int numComparator(const void *a, const void *b) {
    if (*((long long *) a) > *((long long *) b)) return 1;
    if (*((long long *) a) < *((long long *) b)) return -1;
    return 0;
}

long long max(long long a, long long b) {
    return a > b ? a : b;
}

long long min(long long a, long long b) {
    return a < b ? a : b;
}

int main() {
    long long heroesAmount;
    scanf("%lld", &heroesAmount);

    long long *heroesPower = (long long *) malloc(sizeof(long long) * heroesAmount);
    long long totalHeroPower = 0;
    for (long long i = 0; i < heroesAmount; i++) {
        scanf("%lld", heroesPower + i);

        totalHeroPower += heroesPower[i];
    }

    qsort(heroesPower, heroesAmount, sizeof(heroesPower[0]), numComparator);

    int dragonsAmount;
    scanf("%d", &dragonsAmount);

    for (int i = 0; i < dragonsAmount; i++) {
        long long defendingDragonPower, attackingDragonPower;
        scanf("%lld %lld", &defendingDragonPower, &attackingDragonPower);

        long long left = -1;
        long long right = heroesAmount;
        while (right - left > 1) {
            long long middle = left + (right - left) / 2;

            if (heroesPower[middle] < defendingDragonPower)
                left = middle;
            else
                right = middle;
        }

        long long totalMoneySpent;
        if (left == -1)
            totalMoneySpent = max(0, defendingDragonPower - heroesPower[right]) +
                              max(0, attackingDragonPower - (totalHeroPower - heroesPower[right]));
        else if (right == heroesAmount)
            totalMoneySpent = max(0, defendingDragonPower - heroesPower[left]) +
                              max(0, attackingDragonPower - (totalHeroPower - heroesPower[left]));
        else
            totalMoneySpent = min(max(0, defendingDragonPower - heroesPower[right]) +
                                  max(0, attackingDragonPower - (totalHeroPower - heroesPower[right])),
                                  max(0, defendingDragonPower - heroesPower[left]) +
                                  max(0, attackingDragonPower - (totalHeroPower - heroesPower[left])));

        printf("%lld\n", totalMoneySpent);
    }

    free(heroesPower);

    return 0;
}


\end{minted}
\codeforcesresult{/codeforceresults/1574C}
\href{https://github.com/IAmProgrammist/programming-and-algorithmization-basics/blob/c/lab10/18.c}{\underline{Ссылка на репозиторий}}

\newpage
\item \href{https://codeforces.com/edu/course/2/lesson/6/2/practice/contest/283932/problem/D}{\textbf{** Детский праздник}}
\begin{minted}[frame=lines, framesep=2mm, baselinestretch=1.2, bgcolor=LightGray, breaklines, fontsize=\small]{C}
#include <stdio.h>
#include <stdlib.h>

long long min(long long a, long long b) {
    return a < b ? a : b;
}

long long max(long long a, long long b) {
    return a > b ? a : b;
}

typedef struct Worker {
    long long balloonMakingTime;
    long long balloonAmount;
    long long relaxTime;
} Worker;

typedef struct Report {
    long long *workersReport;
    long long workersAmount;
    long long ballonsTotal;
} Report;

Report balloons(long long time, Worker *workers, int workersAmount) {
    Report report = {(long long *) malloc(sizeof(long long) * workersAmount), workersAmount, 0};

    for (int i = 0; i < workersAmount; i++) {
        Worker worker = workers[i];
        long long fullCycleTime = worker.balloonMakingTime * worker.balloonAmount + worker.relaxTime;
        long long balloonsFullCycle = (time / fullCycleTime) * worker.balloonAmount;
        long long timeLeft = max(0, time - (fullCycleTime * balloonsFullCycle / worker.balloonAmount));
        balloonsFullCycle += min(timeLeft, worker.balloonMakingTime * worker.balloonAmount) /
                worker.balloonMakingTime;
        report.ballonsTotal += balloonsFullCycle;
        report.workersReport[i] = balloonsFullCycle;
    }

    return report;
}

int main() {
    long long balloonsAmount;
    int workersAmount;
    scanf("%lld %d", &balloonsAmount, &workersAmount);

    Worker *workers = (Worker *) malloc(sizeof(Worker) * workersAmount);
    for (int i = 0; i < workersAmount; i++) {
        long long balloonMakingTime, balloonAmount, relaxTime;
        scanf("%lld %lld %lld", &balloonMakingTime, &balloonAmount, &relaxTime);
        workers[i] = (Worker) {balloonMakingTime, balloonAmount, relaxTime};
    }

    long long left = -1;
    long long right = 17000000000;
    Report fReport = {NULL, 0, 0};


    while (right - left > 1) {
        long long middle = left + (right - left) / 2;
        Report cReport = balloons(middle, workers, workersAmount);
        if (cReport.ballonsTotal < balloonsAmount) {
            left = middle;
            free(cReport.workersReport);
        } else {
            free(fReport.workersReport);
            fReport = cReport;
            right = middle;
        }
    }

    printf("%lld\n", right);

    for (int i = 0; i < workersAmount; i++) {
        printf("%lld ", min(balloonsAmount, fReport.workersReport[i]));

        balloonsAmount -= fReport.workersReport[i];
        balloonsAmount = balloonsAmount < 0 ? 0 : balloonsAmount;
    }

    free(workers);

    return 0;
}


\end{minted}
\codeforcesresult{/codeforceresults/283932D}
\href{https://github.com/IAmProgrammist/programming-and-algorithmization-basics/blob/c/lab10/19.c}{\underline{Ссылка на репозиторий}}

\newpage

\end{enumerate}
\textbf{Вывод: }
в ходе лабораторной работы получили навыки использования алгоритмов бинарного поиска для решения задач оптимизации.


    % Здесь можно редактировать Tex-файл вручную
\end{document}
