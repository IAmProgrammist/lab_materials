\documentclass[a4paper,14pt]{extarticle}

\usepackage[english,russian]{babel}
\usepackage[T2A]{fontenc}
\usepackage[utf8]{inputenc}
\usepackage{ragged2e}
\usepackage[utf8]{inputenc}
\usepackage{hyperref}
\usepackage{minted}
\setmintedinline{frame=lines, framesep=2mm, baselinestretch=1.5, bgcolor=LightGray, breaklines,fontsize=\scriptsize, tabsize=4}
\setminted{frame=lines, framesep=2mm, baselinestretch=1.5, bgcolor=LightGray, breaklines,fontsize=\scriptsize, tabsize=4}
\usepackage{xcolor}
\definecolor{LightGray}{gray}{0.9}
\usepackage{graphicx}
\usepackage[export]{adjustbox}
\usepackage[left=3cm,right=1.5cm,
top=2cm,bottom=2cm,bindingoffset=0cm]{geometry}
\usepackage{fontspec}
\usepackage{ upgreek }
\usepackage[shortlabels]{enumitem}
\usepackage{adjustbox}
\usepackage{multirow}
\usepackage{amsmath}
\usepackage{amssymb}
\usepackage{pifont}
\usepackage{pgfplots}
\usepackage{longtable}
\usepackage{array}
\usepackage{titlesec}
\usepackage{capt-of}
\usepackage{caption} %заголовки плавающих объектов

\captionsetup[figure]{name=Рисунок}

\DeclareCaptionFormat{tablecaption}
{
    \begin{flushright}
        \textit{#1}
    \end{flushright}
    \begin{center}
        \textbf{#3}
    \end{center}
}

\DeclareCaptionFormat{imagecaption}
{%
    #1#2#3
}

\newcommand\makenewfig[3] {
    \captionsetup{format=imagecaption}
    \begin{center}
        #1
        \nopagebreak
        \captionof{figure}{#2}
        \nopagebreak
        \label{#3}
    \end{center}
}

\renewcommand{\baselinestretch}{1.3}

\makeatletter
\AddEnumerateCounter{\asbuk}{\russian@alph}{щ}
\makeatother
\setmonofont{Consolas}
\setmainfont{Times New Roman}

\titleformat*{\section}{\centering\bfseries}
\titleformat*{\subsection}{\centering\bfseries}
\newcommand{\anonsection}[1]{\section*{#1}\addcontentsline{toc}{section}{#1}}

\newcommand\textbox[1]{
	\parbox{.45\textwidth}{#1}
} 

\newcommand{\specialcell}[2][c]{%
	\begin{tabular}[#1]{@{}c@{}}#2\end{tabular}}

\newbox\namebox
\newdimen\signboxdim

\def\signature#1{%
    \setbox\namebox=\hbox{#1}
    \signboxdim=\dimexpr(\wd\namebox+1cm)
    \parbox[t]{\signboxdim}{%
        \centering
%           \mbox{}\leaders\hbox to .4em{\hss.\hss}\hskip\nameboxdim\mbox{}\\   % for dots
            \hrulefill\\    % for a line
            #1
        \par}%
    }

\begin{document}
\linespread{1.3}
	\setlength{\parskip}{0cm}

Слайд 1 [Inspiration *stars*]

Большую часть информации человек воспринимает глазами, 
именно поэтому одним из самых популярных видов контента на сегодняшний день 
является визуальный контент.

Красочная, яркая и пёстрая или серая, драматичная. За множество лет человечество успело
отобразить реальность в графическом формате множество раз в виде картин, фильмов, фотографии.

Компьютеры стали незамениными помощниками в создании графического контента.
Вычислительные мощности компьютеров, растущие ежегодно, уже позволяют создавать картинку, неотличимую от реальности.
Это стало возможно благодаря развитию графических процессоров - отдельному устройству ПК.

Слайд 2 [Что мы сделали]

Целью данного проекта стала разработка графического движка, использующего техники реалистичного рендера и видеокарту.

Предметами исследования стали библиотеки для работы с видеокартами от AMD HIP и HIP RT, 
техники реалистичного рендера: трассировка лучей и физически корректный рендеринг.

Для достижения цели необходимо изучить техники реалистичного рендера и документацию для работы с видеокартами, 
выбрать подходящий формат хранения сцены и реализовать на основе полученных данных программу.

Слайд 3 [рейтрейсинг]

В основе трассировки лучей лежит довольно простая идея. Предположим, нам нужно нарисовать картину, но 
всё что мы можем сделать - это ставить точки и безошибочно определять цвет, куда мы смотрим.
Можно разбить холст на квадраты и методично просматривать каждый из них, определяя цвет и ставя точку соответствующего цвета.
Таким образом можно получить картину. 

Трассировка лучей работает схожим образом. Из точки наблюдения мы будем испускать луч в соответствующем направлении и определять, 
в какой цвет окрашивать текущий пиксель.  

Точка наблюдения - это координаты камеры, а направление луча определяется поворотом камеры и вертикальным углом обзора. 
Будем отрисовывать цвет ближайшего объекта, в который мы попали.

Выглядит неплохо, однако чего-то не хватает. Давайте введём свет. В движке доступно три источника света: 
солнечный, точечный и лампа. Солнечный свет равномерен и падает на любой объект под одинаковым углом. 
Точечный свет испускается из одной точки во все другие, свет лампы отличается от точечного направлением, 
радиусом внутреннего конуса (в нём свет будет максимальный) и внешнего конуса (вне его света не будет). 
Освещённость будем высчитывать из параметров света, его цвета, затухания, угла падения. Получим следующую картину

Стало лучше, но не хватает теней. В случае света от лампы и точечного света будем определять, находится ли 
другой объект между точкой света и текущим рассматриваемым объектом. Объект есть - есть тень. Нет - тени нет.
В случае солнечного света будем испускать луч наоборот из рассматриваемой точки и проверять, пересекается ли он с 
другими объектами.

Получили следующую картину:

Слайд 4 [PBR]
До сих пор мы рассматривали взаимодействие света и 
объекта упрощённо, однако в реальности такого не бывает.

Сборник техник Physically Based Rendering (в дальнейшем физически корректный рендеринг) 
позволяет определить, как свет будет взаимодействовать с объектом в зависимости 
от его физических свойств - шероховатости и металличности.

Согласно физически корректному рендерингу, чтобы модель освещения могла быть реалистичной,
она должна удовлетворять трём условиям: Основываться на модели отражащих микрограней, Подчиняться закону сохранения энергии, Использовать двулучевую функцию отражательной способности. 

Декомпозируем формулу освещения и сразу приведём её к удобному виду:

Здесь $f_r$ - BRDF, 
$L_i$ - интенсивность света, $w_i$ - вектор к источнику света, N - нормаль поверхности, p - рассматриваемая точка объекта, $w_0$ - вектор к наблюдателю, n - количество источников света.
Условие три выполнено - мы использовали BRDF.

BRDF в свою очередь состоит из двух частей: диффузного света и блика.
$k_d$ и $k_f$ - коэффициенты, которые влияют на то, насколько сильным будет тот или иной вид освещения, при этом их сумма 
всегда меньше или равна 1 - здесь выполняется второе условие.

Для получения диффузного света будем используется метод Ламберта.
Перед погружением в бликовую часть света, необходимо сначала разобраться с отражающими микрогранями.

Поверхность любого объекта, какой бы она ни казалась идеальной, на самом деле таковой не является. 
Она состоит из маленьких микроподповерхностей, которые отражают свет в другом направлении, заводят его
в "ловушку" из других граней. Модель Кука-Торренса учитывает эти особенности микрограней:  

D - функция нормальной дистрибуции. Она описывает, как много микрограней будут повернуты к наблюдателю. Для её вычисления будем использовать модель GGX/Trowbridge-Reitz.
G - функция геометрической затенённости, описывает, как много поверхности не самозатенено и не скрыто другими микрогранями. Для её вычисления будем использовать модель Schlick-GGX.
F - функция Френеля, описывает, какая часть света была отражена, а какая - преломлена (поглощена самим объектом). 

В итоге получим картинку:

Слайд 5 [Как работает видеокарта]

Графическому процессору, как и любому другому процессору, для выполнения задачи необходимо указать, с какими данными нужно работать и каким образом.
Библиотека HIP RT предоставляет специальные акселерационные структуры данных и классы, ускоряющие нахождение пересечения луча и примитива. 
Эти структуры данных содержат информацию
о геометрии сцены, о трансформации объектов и времени трансформации. Дополнительно видеокарте будем передавать информацию о нормалях вершин, 
материалы. 

Особенностью видеокарты является то, что она может выполнять один процесс многократно и параллельно, 
такой процесс описывается кернелом - функцией, которая будет запускаться на видеокарте.
После того, как мы загрузили сцену в акселирирующие структуры данных, можно приступать к запуску кернела. 
Процессы в видеокарте выполняются в 3-мерном пространстве, каждому процессу присваивается координата x, y и z. 
Так как наша задача - получить двухмерную картинку, вычисления будут двухмерные, а 
процессу будут присвоены координаты x и y, соответствующие координате отрисовываемого пикселя на холсте.

В результате получим изображение сцены:

Слайд 6 [Скрины и выводы]

В ходе курсовой работы были изучены основы техник генерации реалистичной компьютерной графики, 
инструменты для добавления аппаратной поддержки в видеокарты.
Полученный движок способен отрисовывать простые сцены, содержащие примитивы, свет.
Полученный движок реализует базовые техники реалистичного рендера и для дальнейшего применения
в определённой сфере требует соответствующих доработок. 

\end{document}