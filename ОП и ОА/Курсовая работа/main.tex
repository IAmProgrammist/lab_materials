\documentclass[a4paper,14pt]{extarticle}

\usepackage[english,russian]{babel}
\usepackage[T2A]{fontenc}
\usepackage[utf8]{inputenc}
\usepackage{ragged2e}
\usepackage[utf8]{inputenc}
\usepackage{hyperref}
\usepackage{minted}
\setmintedinline{frame=lines, framesep=2mm, baselinestretch=1.5, bgcolor=LightGray, breaklines,fontsize=\scriptsize}
\setminted{frame=lines, framesep=2mm, baselinestretch=1.5, bgcolor=LightGray, breaklines,fontsize=\scriptsize}
\usepackage{xcolor}
\definecolor{LightGray}{gray}{0.9}
\usepackage{graphicx}
\usepackage[export]{adjustbox}
\usepackage[left=3cm,right=1.5cm,
top=2cm,bottom=2cm,bindingoffset=0cm]{geometry}
\usepackage{fontspec}
\usepackage{ upgreek }
\usepackage[shortlabels]{enumitem}
\usepackage{adjustbox}
\usepackage{multirow}
\usepackage{amsmath}
\usepackage{amssymb}
\usepackage{pifont}
\usepackage{pgfplots}
\usepackage{longtable}
\usepackage{array}
\usepackage{titlesec}
\usepackage{capt-of}
\usepackage{caption} %заголовки плавающих объектов

\captionsetup[figure]{name=Рисунок}

\DeclareCaptionFormat{tablecaption}
{
    \begin{flushright}
        \textit{#1}
    \end{flushright}
    \begin{center}
        \textbf{#3}
    \end{center}
}

\DeclareCaptionFormat{imagecaption}
{%
    #1#2#3
}

\newcommand\makenewfig[3] {
    \captionsetup{format=imagecaption}
    \begin{center}
        #1
        \nopagebreak
        \captionof{figure}{#2}
        \nopagebreak
        \label{#3}
    \end{center}
}

\renewcommand{\baselinestretch}{1.3}

\graphicspath{ {./images/} }
\makeatletter
\AddEnumerateCounter{\asbuk}{\russian@alph}{щ}
\makeatother
\setmonofont{Consolas}
\setmainfont{Times New Roman}

\titleformat*{\section}{\centering\bfseries}
\titleformat*{\subsection}{\centering\bfseries}
\newcommand{\anonsection}[1]{\section*{#1}\addcontentsline{toc}{section}{#1}}

\newcommand\textbox[1]{
	\parbox{.45\textwidth}{#1}
} 

\newcommand{\specialcell}[2][c]{%
	\begin{tabular}[#1]{@{}c@{}}#2\end{tabular}}

\newbox\namebox
\newdimen\signboxdim

\def\signature#1{%
    \setbox\namebox=\hbox{#1}
    \signboxdim=\dimexpr(\wd\namebox+1cm)
    \parbox[t]{\signboxdim}{%
        \centering
%           \mbox{}\leaders\hbox to .4em{\hss.\hss}\hskip\nameboxdim\mbox{}\\   % for dots
            \hrulefill\\    % for a line
            #1
        \par}%
    }

\begin{document}
\linespread{1.3}
	\setlength{\parskip}{0cm}
\pagenumbering{gobble}
\justifying
\begin{center}
    \small{
        \textbf{МИНИCТЕРCТВО НАУКИ И ВЫCШЕГО ОБРАЗОВАНИЯ РОCCИЙCКОЙ ФЕДЕРАЦИИ}\\
        ФЕДЕРАЛЬНОЕ ГОCУДАРCТВЕННОЕ БЮДЖЕТНОЕ ОБРАЗОВАТЕЛЬНОЕ УЧРЕЖДЕНИЕ\\ВЫCШЕГО ОБРАЗОВАНИЯ \\
        \textbf{«БЕЛГОРОДCКИЙ ГОCУДАРCТВЕННЫЙ ТЕХНОЛОГИЧЕCКИЙ\\УНИВЕРCИТЕТ им. В. Г. ШУХОВА»\\ (БГТУ им. В.Г. Шухова)} \\
        \bigbreak
        \includegraphics[width=70mm]{log}\\
        ИНСТИТУТ ИНФОРМАЦИОННЫХ ТЕХНОЛОГИЙ И УПРАВЛЯЮЩИХ СИСТЕМ\\}
\end{center}

\vfill
\begin{center}
    \large{
        \textbf{
            КУРСОВОЙ ПРОЕКТ}}\\
    \normalsize{
        по дисциплине: Основы программирования \\
        тема: «Разработка графического движка»}
\end{center}
\vfill
\begin{center}
    Автор работы \signature{\small{(подпись)}} Пахомов Владислав Андреевич ПВ-223\bigbreak
    Руководитель проекта \signature{\small{(подпись)}} Черников Сергей Викторович
\end{center}
\vfill
\begin{center}
    Оценка \signature{}
\end{center}
\vfill\begin{center}
    Белгород 2023 г.
\end{center}
\newpage

\renewcommand{\contentsname}{Оглавление}
\tableofcontents\newpage

\anonsection{Введение}


Большую часть информации человек воспринимает глазами, 
именно поэтому одним из самых популярных видов контента на сегодняшний день 
является визуальный контент.

Красочная, яркая и пёстрая или серая, драматичная. За множество лет человечество успело
отобразить реальность в графическом формате множество раз в виде картин, фильмов, фотографии.

Компьютеры стали незамениными помощниками в создании графического контента.
Вычислительные мощности компьютеров, растущие ежегодно, уже позволяют создавать картинку, неотличимую от реальности.
Это стало возможно благодаря развитию графических процессоров - отдельному устройству ПК.

Более мощные компьютеры позволяют сегодня использовать более сложные алгоритмы для получения 
реалистичной картинки, например трассировка лучей и Physically Based Rendering. В последние модели 
видеокарт добавляются дополнительные ядра, которые способны решать задачи, направленные на рендеринг при помощи 
данных техник.

Производитель видеокарт поставляет библиотеки, позволяющие работать с этими ядрами. Для видеокарт от
компании AMD такой библиотекой является HIP RT, расширяющая библиотеку для работы с видеокартой HIP.

Объект исследования - разработка графического движка.

Предмет исследования - библиотеки для работы с видеокартами от AMD HIP и HIP RT, 
техники реалистичного рендера трассировка лучей и Physically Based Rendering.

Цель - разработать графический движок, использующий техники трассировка лучей и Physically Based Rendering и аппаратное обеспечение (видеокарту).

Для достижения поставленной цели необходимо решить следующие задачи:
\begin{itemize}
    \item Изучить техники реалистичного рендера трассировка лучей и Physically Based Rendering.
    \item Изучить и применить библиотеки для аппаратного ускорения при использовании техник реалистичного рендера.
    \item Подобрать удобный формат хранения информации о 3D-сцене.
    \item Разработать программу, генерирующую изображение на основе данных о 3D-сцене.
\end{itemize}
\section{Техники реалистичного рендера}
\subsection{Трассировка лучей}
В основе трассировки лучей лежит довольно простая идея. Предположим, нам нужно нарисовать картину, но 
всё что мы можем сделать - это ставить точки и безошибочно определять цвет, куда мы смотрим.
Можно разбить холст на квадраты и методично просматривать каждый из них, определяя цвет и ставя точку соответствующего цвета.
Таким образом можно получить картину. 

Трассировка лучей работает схожим образом. Из точки наблюдения мы будем испускать луч в соответствующем направлении и определять, 
в какой цвет окрашивать текущий пиксель.  

Точка наблюдения - это координаты камеры. 
Направление испускаемого луча можно определить по следующей формуле:\\
$x, y$ - координаты текущего обрабатываемого пикселя,\\
$AR = \frac{Res_W}{Res_H}$ - соотношение сторон, $Res_W$ - ширина холста, $Res_H$ - высота холста.\\
$S_H = \frac{2}{1 + AR}$, $S_W = 2 - S_H$ - стороны прямоугольника, подобного холсту, причём $S_H + S_W = 2$.\\
$D = ((x / Res_W) \cdot S_W - \frac{S_W}{2}, (y / Res_H) \cdot S_H - \frac{S_H}{2}, ( -S_W / 2 ) / tan( FOV / 2 ))$, где FOV - вертикальный обзор камеры.
Дополнительно вектор D можно умножить на матрицу вращения для того, чтобы повернуть обозревателя.\\

Будем находить, пересёкся ли луч с каким-либо объектом, и если пересёкся, ставить точку его цвета. Иначе - чёрную.
\makenewfig{\includegraphics[width=90mm]{raytrace_1}}{Пересечение луча и фигуры}{ris:raytrace_1}

Хотелсь бы немного разнообразить сцену - да будет свет! Введём направленные источники света, иначе говоря - солнца. 
Солнечный свет имеет направление, и следовательно освещать объекты будет по-разному. Чем больше угол между солнечным лучом и нормалью вершины, тем меньше солнца он будет получать.

$Color = BaseColor \cdot L_C \cdot L_I \cdot cos(-N, L_D)$, где BaseColor - цвет объекта, $L_C$ - цвет света, $L_I$ - интенсивность света, $N$ - нормаль объекта, $L_D$ - направление света.

\makenewfig{\includegraphics[width=90mm]{raytrace_2}}{Сцена с источником света}{ris:raytrace_2}

Сцена стала немного интересней, однако она всё ещё довольно странная - предметы не отбрасывают тень. 
Для того, чтобы понять, отбрасывает ли объект тень, можем испустить луч от точки пересечения в противоположном направлении свету. 
Если мы найдём хоть один объект, котиорый пересекает луч, то значит в данной точке будет тень. Иначе - точка освещена.
\makenewfig{\includegraphics[width=90mm]{raytrace_3}}{Добавление тени}{ris:raytrace_3}

Тетраэдр начал отбрасывать тень на сферу. 
Можно также добавить другие источники света - точечный свет и лампы.
Основным отличием от солнца у этих источников света является затухание. 
С расстоянием сила света будет становиться меньше. 
Затухание можно расчитать по следующей формуле:

$Att = max(min(1 - \frac{Distance}{L_R} ^ 4, 1), Distance ^ 2)$, где Distance - 
расстояние между точкой на объекте и источником света, $L_R$ - радиус света.

Точечный свет находится в одной точке и испускает свет во все стороны, формула для получения света будет следующей:\\
$Color = BaseColor \cdot L_C \cdot L_I \cdot cos(-N, L_D) \cdot Att$

\makenewfig{\includegraphics[width=90mm]{raytrace_4}}{Точечный свет}{ris:raytrace_4}

Для определения тени будем испускать луч из источника света в направлении к рассматриваемой в данный момент точке.
Если ближайшее пересечение с объектом - пересечение с искомым объектом, то он освещён. Иначе - оставляем тень.

\makenewfig{\includegraphics[width=90mm]{raytrace_5}}{Точечный свет с тенью}{ris:raytrace_5}

В источнике света "лампа" появляются углы внутреннего и внешнего конусов. Свет, находящийся во внутреннем конусе, имеет максимальную интенсивность, 
между внешним и внутренним конусом затихает, и вне внешнего конуса света нет. 

TODO: Ты закончил здесь.

\section{Используемые библиотеки и технологии}
\subsection{Формат хранения сцены}
Перед тем как начать рендер 3D-сцены, необходимо получить информацию о ней.
Можно было "хардкодить" информацию о ней прямо в коде, однако полученная программа 
будет крайне негибка: требовать постоянной рекомпиляции, работы с программистом.

Информацию о сцене можно получать из файла. Выбор пал на несколько форматов хранения информации о 3D-сценах:
\begin{itemize}
    \item STL - простой формат файла, однако не подходящий для реалистичного рендера, он содержит только информацию о форме объектов.
    \item FBX - популярный формат файла, разрабатываемый в Autodesk. Формат тоже не подходит, 
    так как не содержит информацию для выбранной техники Physically Based Rendering.
    \item COLLADA - формат, основанный на XML и разработанный для передачи информации между 3D приложениями. Управляется Khronos Group. 
    Формат также не поддерживает Physically Based Rendering.
    \item glTF - формат, основанный на JSON, расширяемый, легкопонимаемый. Используется в веб-технологиях. 
    Поддерживает все необходимые данные для выбранных техник. 
\end{itemize}

Для хранения сцены был выбран формат glTF. Для обработки glTF-файла была использована библиотека tinygltf. 
\subsection{HIP и его расширение HIP RT}
Основное преимущество графического процессора, или видеокарты, заключается в 
возможности выполнять множество процессов одновременно.  
\section{Заключение}
\section{Список источников и литературы}

https://gpuopen.com/hiprt/
https://github.com/syoyo/tinygltf
https://habr.com/en/articles/342510/

\end{document}