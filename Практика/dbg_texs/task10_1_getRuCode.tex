% Созданная блок схема работает только с компиляторами XeTeX и LuaTeX.
\documentclass{article}

% Необходимые зависимости
\usepackage[utf8]{inputenc}
\usepackage[english,russian]{babel}
\usepackage{pgfplots}
\usepackage{verbatim}
\usetikzlibrary{positioning}
\usetikzlibrary{shapes.geometric}
\usetikzlibrary{shapes.misc}
\usetikzlibrary{calc}
\usetikzlibrary{chains}
\usetikzlibrary{matrix}
\usetikzlibrary{decorations.text}
\usepackage{fontspec}
\usetikzlibrary{backgrounds}

\begin{document}
% Блок-схема
% Если Вы хотите добавить блок схему в свой документ, скопируйте код между комментариями
% С линияи и вставьте в документ.

% --------------------------
\begin{tikzpicture}[every node/.style={inner sep=0,outer sep=0}, background rectangle/.style={opacity=1.000000, fill=colorLBEILTTGFOLGCP}, show background rectangle]
\makeatletter
\newcommand{\verbatimfont}[1]{\def\verbatim@font{#1}}
\makeatother
% Шрифты
\newfontfamily\fontLBEILTTGADCCDAB[Scale=0.641323, SizeFeatures={Size=10.000000}]{Consolas}

% Цвета
\definecolor{colorLBEILTTGFOLGCP}{rgb}{1.000000,1.000000,1.000000}
\definecolor{colorLBEILTTGHEISADPG}{rgb}{0.949020,0.949020,0.949020}
\definecolor{colorLBEILTTGKLJKFHQI}{rgb}{0.000000,0.000000,0.000000}
\draw[opacity=1.000000, rounded corners=0.248869 cm, fill=colorLBEILTTGHEISADPG, draw=none] (0.877828,0.004525) -- (3.117647,0.004525) -- (3.117647,-0.493213) -- (0.877828,-0.493213) -- cycle;
\draw[opacity=1.000000, rounded corners=0.248869 cm, fill=none, line width=0.022624cm, colorLBEILTTGKLJKFHQI] (0.877828,0.004525) -- (3.117647,0.004525) -- (3.117647,-0.493213) -- (0.877828,-0.493213) -- cycle;
\verbatimfont{\normalsize\fontLBEILTTGADCCDAB}
\node[opacity=1.000000, above right, colorLBEILTTGKLJKFHQI] at(1.114795, -0.364871) {\verb|getRuCode(beg)|};
\draw[fill=colorLBEILTTGHEISADPG, draw=none, opacity=1.000000] (-0.004525,-0.809955) -- (4.000000,-0.809955) -- (4.000000,-1.307692) -- (-0.004525,-1.307692) -- cycle;
\draw[opacity=1.000000, fill=none, line width=0.022624cm, colorLBEILTTGKLJKFHQI] (-0.004525,-0.809955) -- (4.000000,-0.809955) -- (4.000000,-1.307692) -- (-0.004525,-1.307692) -- cycle;
\verbatimfont{\normalsize\fontLBEILTTGADCCDAB}
\node[opacity=1.000000, above right, colorLBEILTTGKLJKFHQI] at(0.125900, -1.179350) {\verb|code := (beg[0] << 8) + beg[1]|};
\draw[opacity=1.000000, fill=none, line width=0.022624 cm, colorLBEILTTGKLJKFHQI] (1.997738,-0.493213) -- (1.997738,-0.809955);
\draw[opacity=1.000000, rounded corners=0.248869 cm, fill=colorLBEILTTGHEISADPG, draw=none] (1.002262,-1.624434) -- (2.993213,-1.624434) -- (2.993213,-2.122172) -- (1.002262,-2.122172) -- cycle;
\draw[opacity=1.000000, rounded corners=0.248869 cm, fill=none, line width=0.022624cm, colorLBEILTTGKLJKFHQI] (1.002262,-1.624434) -- (2.993213,-1.624434) -- (2.993213,-2.122172) -- (1.002262,-2.122172) -- cycle;
\verbatimfont{\normalsize\fontLBEILTTGADCCDAB}
\node[opacity=1.000000, above right, colorLBEILTTGKLJKFHQI] at(1.238407, -1.993830) {\verb|Вернуть code|};
\draw[opacity=1.000000, fill=none, line width=0.022624 cm, colorLBEILTTGKLJKFHQI] (1.997738,-1.307692) -- (1.997738,-1.624434);
\end{tikzpicture}
% --------------------------

% Конец блок схемы
\end{document}