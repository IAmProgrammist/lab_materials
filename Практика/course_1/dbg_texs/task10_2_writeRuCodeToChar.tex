% Созданная блок схема работает только с компиляторами XeTeX и LuaTeX.
\documentclass{article}

% Необходимые зависимости
\usepackage[utf8]{inputenc}
\usepackage[english,russian]{babel}
\usepackage{pgfplots}
\usepackage{verbatim}
\usetikzlibrary{positioning}
\usetikzlibrary{shapes.geometric}
\usetikzlibrary{shapes.misc}
\usetikzlibrary{calc}
\usetikzlibrary{chains}
\usetikzlibrary{matrix}
\usetikzlibrary{decorations.text}
\usepackage{fontspec}
\usetikzlibrary{backgrounds}

\begin{document}
% Блок-схема
% Если Вы хотите добавить блок схему в свой документ, скопируйте код между комментариями
% С линияи и вставьте в документ.

% --------------------------
\begin{tikzpicture}[every node/.style={inner sep=0,outer sep=0}, background rectangle/.style={opacity=1.000000, fill=colorSNRCFQQENPREA}, show background rectangle]
\makeatletter
\newcommand{\verbatimfont}[1]{\def\verbatim@font{#1}}
\makeatother
% Шрифты
\newfontfamily\fontSNRCFQTOAPGES[Scale=0.801653, SizeFeatures={Size=10.000000}]{Consolas}

% Цвета
\definecolor{colorSNRCFQQENPREA}{rgb}{1.000000,1.000000,1.000000}
\definecolor{colorSNRCFQDDRINGEM}{rgb}{0.949020,0.949020,0.949020}
\definecolor{colorSNRCFQGNHRTICT}{rgb}{0.000000,0.000000,0.000000}
\draw[opacity=1.000000, rounded corners=0.311086 cm, fill=colorSNRCFQDDRINGEM, draw=none] (0.008484,0.005656) -- (4.985860,0.005656) -- (4.985860,-0.616516) -- (0.008484,-0.616516) -- cycle;
\draw[opacity=1.000000, rounded corners=0.311086 cm, fill=none, line width=0.028281cm, colorSNRCFQGNHRTICT] (0.008484,0.005656) -- (4.985860,0.005656) -- (4.985860,-0.616516) -- (0.008484,-0.616516) -- cycle;
\verbatimfont{\normalsize\fontSNRCFQTOAPGES}
\node[opacity=1.000000, above right, colorSNRCFQGNHRTICT] at(0.307475, -0.464918) {\verb|writeRuCodeToChar(code, beg)|};
\draw[fill=colorSNRCFQDDRINGEM, draw=none, opacity=1.000000] (-0.005656,-1.012443) -- (5.000000,-1.012443) -- (5.000000,-1.634615) -- (-0.005656,-1.634615) -- cycle;
\draw[opacity=1.000000, fill=none, line width=0.028281cm, colorSNRCFQGNHRTICT] (-0.005656,-1.012443) -- (5.000000,-1.012443) -- (5.000000,-1.634615) -- (-0.005656,-1.634615) -- cycle;
\verbatimfont{\normalsize\fontSNRCFQTOAPGES}
\node[opacity=1.000000, above right, colorSNRCFQGNHRTICT] at(0.148546, -1.483018) {\verb|beg[0] := (code & 0xFF00) >> 8|};
\draw[opacity=1.000000, fill=none, line width=0.028281 cm, colorSNRCFQGNHRTICT] (2.497172,-0.616516) -- (2.497172,-1.012443);
\draw[fill=colorSNRCFQDDRINGEM, draw=none, opacity=1.000000] (0.537330,-2.030543) -- (4.457014,-2.030543) -- (4.457014,-2.652715) -- (0.537330,-2.652715) -- cycle;
\draw[opacity=1.000000, fill=none, line width=0.028281cm, colorSNRCFQGNHRTICT] (0.537330,-2.030543) -- (4.457014,-2.030543) -- (4.457014,-2.652715) -- (0.537330,-2.652715) -- cycle;
\verbatimfont{\normalsize\fontSNRCFQTOAPGES}
\node[opacity=1.000000, above right, colorSNRCFQGNHRTICT] at(0.695970, -2.501117) {\verb|beg[1] := code & 0x00FF|};
\draw[opacity=1.000000, fill=none, line width=0.028281 cm, colorSNRCFQGNHRTICT] (2.497172,-1.634615) -- (2.497172,-2.030543);
\draw[opacity=1.000000, rounded corners=0.311086 cm, fill=colorSNRCFQDDRINGEM, draw=none] (1.252828,-3.048643) -- (3.741516,-3.048643) -- (3.741516,-3.670814) -- (1.252828,-3.670814) -- cycle;
\draw[opacity=1.000000, rounded corners=0.311086 cm, fill=none, line width=0.028281cm, colorSNRCFQGNHRTICT] (1.252828,-3.048643) -- (3.741516,-3.048643) -- (3.741516,-3.670814) -- (1.252828,-3.670814) -- cycle;
\verbatimfont{\normalsize\fontSNRCFQTOAPGES}
\node[opacity=1.000000, above right, colorSNRCFQGNHRTICT] at(1.631889, -3.519217) {\verb|Выход (beg)|};
\draw[opacity=1.000000, fill=none, line width=0.028281 cm, colorSNRCFQGNHRTICT] (2.497172,-2.652715) -- (2.497172,-3.048643);
\end{tikzpicture}
% --------------------------

% Конец блок схемы
\end{document}