\documentclass[a4paper, 12pt, oneside]{article}

\setlength{\parindent}{0}
\setlength{\parskip}{0}

\usepackage[english,russian]{babel}
\usepackage{titlesec}
\usepackage[T2A]{fontenc}
\usepackage[utf8]{inputenc}
\usepackage{ragged2e}
\usepackage{hyperref}
\usepackage{standalone}
\usepackage{minted}
\setmintedinline{frame=lines, framesep=2mm, baselinestretch=1.5, bgcolor=LightGray, breaklines,fontsize=\scriptsize}
\setminted{frame=lines, framesep=2mm, baselinestretch=1.5, bgcolor=LightGray, breaklines,fontsize=\scriptsize}
\usepackage{xcolor}
\definecolor{LightGray}{gray}{0.9}
\usepackage{graphicx}
\usepackage[export]{adjustbox}
\usepackage[left=3cm,right=1.5cm,
    top=2cm,bottom=2cm,bindingoffset=0cm]{geometry}
\usepackage{ upgreek }
\usepackage[shortlabels]{enumitem}
\usepackage[mathletters]{ucs}
\usepackage{multirow}
\usepackage{amsmath}
\usepackage{amssymb}
\usepackage{pifont}
\usepackage{titlesec}

\usepackage{pgfplots}
\usepackage{verbatim}
\usetikzlibrary{positioning}
\usetikzlibrary{shapes.geometric}
\usetikzlibrary{shapes.misc}
\usetikzlibrary{calc}
\usetikzlibrary{chains}
\usetikzlibrary{matrix}
\usetikzlibrary{decorations.text}
\usepackage{fontspec}
\usepackage{booktabs}
\usetikzlibrary{backgrounds}
\usepackage{capt-of}
\usepackage{caption} %заголовки плавающих объектов

\captionsetup[figure]{name=Рисунок}

\newcounter{labeltablecounter}
\renewcommand{\thelabeltablecounter}{
    \arabic{labeltablecounter}
}

\DeclareCaptionFormat{tablecaption}
{
    \begin{flushright}
        \textit{#1}
    \end{flushright}
    \begin{center}
        \textbf{#3}
    \end{center}
}

\DeclareCaptionFormat{imagecaption}
{%
    #1#2#3
}


\captionsetup{format=custom}

\renewcommand{\baselinestretch}{1.5}

\graphicspath{ {./images/} }
\makeatletter
\AddEnumerateCounter{\asbuk}{\russian@alph}{щ}
\makeatother
\setmonofont{Consolas}
\setmainfont{Times New Roman}

\newcommand\textbox[1]{
    \parbox{.45\textwidth}{#1}
}

\newcommand\makenewfig[3] {
    \captionsetup{format=imagecaption}
    \begin{center}
        #1
        \nopagebreak
        \captionof{figure}{#2}
        \nopagebreak
        \label{#3}
    \end{center}
}

\newcommand\labeltable[3] {
    \captionsetup{format=tablecaption}
    \begin{table}[H]
        \caption{#2}
        \begin{center}
            #1
        \end{center}
        \label{#3}
    \end{table}
}

\titleformat*{\section}{\Large\bfseries}
\titleformat*{\subsection}{\large\bfseries}
\newcommand{\anonsection}[1]{\section*{#1}\addcontentsline{toc}{section}{#1}}

\newcommand{\specialcell}[2][c]{%
    \begin{tabular}[#1]{@{}c@{}}
        #2
    \end{tabular}}

\begin{document}
    \linespread{1.5}
    \setlength{\parskip}{0cm}

    \pagenumbering{gobble}

    \justifying
    \begin{center}
        \small{
            МИНИCТЕРCТВО НАУКИ И ВЫCШЕГО ОБРАЗОВАНИЯ \\РОCCИЙCКОЙ ФЕДЕРАЦИИ
            \bigbreak
            ФЕДЕРАЛЬНОЕ ГОCУДАРCТВЕННОЕ БЮДЖЕТНОЕ ОБРАЗОВАТЕЛЬНОЕ УЧРЕЖДЕНИЕ ВЫCШЕГО ОБРАЗОВАНИЯ \\
            \bigbreak
            \textbf{«БЕЛГОРОДCКИЙ ГОCУДАРCТВЕННЫЙ \\ТЕХНОЛОГИЧЕCКИЙ УНИВЕРCИТЕТ им. В. Г. ШУХОВА»\\ (БГТУ им. В.Г. Шухова)} \\
            \bigbreak
            Кафедра программного обеспечения вычислительной техники и автоматизированных систем\\}
    \end{center}

    \vfill
    \begin{center}
        \large{
            \textbf{
                Учебно-ознакомительная практика}}\\
        \normalsize{
            по дисциплине: Основы программирования
        }
    \end{center}
    \vfill
    \hfill\textbox{
        Выполнил: ст. группы ПВ-223\\Пахомов Владислав Андреевич
        \bigbreak
        Проверили: Новожен Никита Викторович\\
        Черников Сергей Викторович
    }
    \vfill\begin{center}
              Белгород 2023 г.
    \end{center}
    \newpage
    \renewcommand{\contentsname}{Оглавление}
    \tableofcontents\newpage
    \begin{center}
        Вариант 10\\
    \end{center}


    \section{Линейные алгоритмы}

    \subsection{Задание варианта}
    Найти корни квадратного уравнения, заданного своими коэффициентами, с положительным дискриминантом; подстановкой в уравнение убедиться в погрешности вычислений.

    \subsection{Обоснование}
    Корни уравнения находим по формуле $x_1 = \frac{-b + \sqrt{D}}{2 \cdot a}$ и $x_2 = \frac{-b - \sqrt{D}}{2 \cdot a}$, где $D = b^2 - 4 \cdot a \cdot c$

    \subsection{Спецификация}
    \begin{enumerate}
        \item Заголовок:  \mintinline{C}|void solveSquareEquation(float a, float b, float c, float *x1, float *x2)|
        \item Назначение: принимает коэффициенты уравнения вида \mintinline{C}|a| $\cdot x^2 + $\mintinline{C}|b|$ \cdot x + $\mintinline{C}|c|$ = 0$, сохраняет его корни в \mintinline{C}|x1|, \mintinline{C}|x2|.
    \end{enumerate}

    \subsection{Блок-схема}
    \makenewfig{\documentclass{article}

\usepackage[utf8]{inputenc}
\usepackage[english,russian]{babel}
\usepackage{pgfplots}
\usepackage{verbatim}
\usetikzlibrary{positioning}
\usetikzlibrary{shapes.geometric}
\usetikzlibrary{shapes.misc}
\usetikzlibrary{calc}
\usetikzlibrary{chains}
\usetikzlibrary{matrix}
\usetikzlibrary{decorations.text}
\usepackage{fontspec}
\usetikzlibrary{backgrounds}



\begin{document}
    \begin{tikzpicture}[every node/.style={inner sep=0,outer sep=0}, background rectangle/.style={opacity=1.000000, fill=colorKLTKDRBQBFIJJS}, show background rectangle]
        \makeatletter
        \newcommand{\verbatimfont}[1]{\def\verbatim@font{#1}}%
        \makeatother
        \newfontfamily\fontKLTKDRBQLLTNPJ[Scale=1.289648, SizeFeatures={Size=10.000000}]{Consolas}
        \definecolor{colorKLTKDRBQBFIJJS}{rgb}{1.000000,1.000000,1.000000}
        \definecolor{colorKLTKDRBQONNQLCA}{rgb}{0.949020,0.949020,0.949020}
        \definecolor{colorKLTKDRBQHJGDDKLQ}{rgb}{0.000000,0.000000,0.000000}
        \draw[opacity=1.000000, rounded corners=0.500455 cm, fill=colorKLTKDRBQONNQLCA, draw=none] (-0.009099,0.009099) -- (10.000000,0.009099) -- (10.000000,-0.991811) -- (-0.009099,-0.991811) -- cycle;
        \draw[opacity=1.000000, rounded corners=0.500455 cm, fill=none, line width=0.045496cm, colorKLTKDRBQHJGDDKLQ] (-0.009099,0.009099) -- (10.000000,0.009099) -- (10.000000,-0.991811) -- (-0.009099,-0.991811) -- cycle;
        \verbatimfont{\normalsize\fontKLTKDRBQLLTNPJ}
        \node[opacity=1.000000, above right, colorKLTKDRBQHJGDDKLQ] at(0.475786, -0.695607) {\verb|solveSquareEquation(a, b, c, x1, x2)|};
        \draw[fill=colorKLTKDRBQONNQLCA, draw=none, opacity=1.000000] (0.441310,-1.628753) -- (9.549591,-1.628753) -- (9.549591,-2.629663) -- (0.441310,-2.629663) -- cycle;
        \draw[opacity=1.000000, fill=none, line width=0.045496cm, colorKLTKDRBQHJGDDKLQ] (0.441310,-1.628753) -- (9.549591,-1.628753) -- (9.549591,-2.629663) -- (0.441310,-2.629663) -- cycle;
        \draw[opacity=1.000000, fill=none, line width=0.045496cm, colorKLTKDRBQHJGDDKLQ] (0.591447,-1.628753) -- (9.399454,-1.628753) -- (9.399454,-2.629663) -- (0.591447,-2.629663) -- cycle;
        \verbatimfont{\normalsize\fontKLTKDRBQLLTNPJ}
        \node[opacity=1.000000, above right, colorKLTKDRBQHJGDDKLQ] at(0.864195, -2.333460) {\verb|sqrtD := sqrtf(b * b - 4 * a * c)|};
        \draw[opacity=1.000000, fill=none, line width=0.045496 cm, colorKLTKDRBQHJGDDKLQ] (4.995450,-0.991811) -- (4.995450,-1.628753);
        \draw[fill=colorKLTKDRBQONNQLCA, draw=none, opacity=1.000000] (1.219290,-3.266606) -- (8.771611,-3.266606) -- (8.771611,-4.267516) -- (1.219290,-4.267516) -- cycle;
        \draw[opacity=1.000000, fill=none, line width=0.045496cm, colorKLTKDRBQHJGDDKLQ] (1.219290,-3.266606) -- (8.771611,-3.266606) -- (8.771611,-4.267516) -- (1.219290,-4.267516) -- cycle;
        \verbatimfont{\normalsize\fontKLTKDRBQLLTNPJ}
        \node[opacity=1.000000, above right, colorKLTKDRBQHJGDDKLQ] at(1.482118, -3.971312) {\verb|x1 := (-b + sqrtD) / (2 * a)|};
        \draw[opacity=1.000000, fill=none, line width=0.045496 cm, colorKLTKDRBQHJGDDKLQ] (4.995450,-2.629663) -- (4.995450,-3.266606);
        \draw[fill=colorKLTKDRBQONNQLCA, draw=none, opacity=1.000000] (1.219290,-4.904459) -- (8.771611,-4.904459) -- (8.771611,-5.905369) -- (1.219290,-5.905369) -- cycle;
        \draw[opacity=1.000000, fill=none, line width=0.045496cm, colorKLTKDRBQHJGDDKLQ] (1.219290,-4.904459) -- (8.771611,-4.904459) -- (8.771611,-5.905369) -- (1.219290,-5.905369) -- cycle;
        \verbatimfont{\normalsize\fontKLTKDRBQLLTNPJ}
        \node[opacity=1.000000, above right, colorKLTKDRBQHJGDDKLQ] at(1.482118, -5.609165) {\verb|x2 := (-b - sqrtD) / (2 * a)|};
        \draw[opacity=1.000000, fill=none, line width=0.045496 cm, colorKLTKDRBQHJGDDKLQ] (4.995450,-4.267516) -- (4.995450,-4.904459);
        \draw[opacity=1.000000, rounded corners=0.500455 cm, fill=colorKLTKDRBQONNQLCA, draw=none] (2.743403,-6.542311) -- (7.247498,-6.542311) -- (7.247498,-7.543221) -- (2.743403,-7.543221) -- cycle;
        \draw[opacity=1.000000, rounded corners=0.500455 cm, fill=none, line width=0.045496cm, colorKLTKDRBQHJGDDKLQ] (2.743403,-6.542311) -- (7.247498,-6.542311) -- (7.247498,-7.543221) -- (2.743403,-7.543221) -- cycle;
        \verbatimfont{\normalsize\fontKLTKDRBQLLTNPJ}
        \node[opacity=1.000000, above right, colorKLTKDRBQHJGDDKLQ] at(3.229957, -7.247017) {\verb|Выход (x1, x2)|};
        \draw[opacity=1.000000, fill=none, line width=0.045496 cm, colorKLTKDRBQHJGDDKLQ] (4.995450,-5.905369) -- (4.995450,-6.542311);
    \end{tikzpicture}
\end{document}}{Функция solveSquareEquation}{ris:task1}

    \subsection{Код программы}
    \begin{minted}{C}
#include <math.h>

void solveSquareEquation(float a, float b, float c, float *x1, float *x2) {
    float sqrtD = sqrtf(b * b - 4 * a * c);
    *x1 = (-b + sqrtD) / (2 * a);
    *x2 = (-b - sqrtD) / (2 * a);
}
    \end{minted}
    \bigbreak
    \labeltable{
        \begin{tabular}{|c|c|c|c|c|c|c|}
            \hline
            a    & b            & c               & $x_1$  (факт.) & $x_2$  (факт.) & $|F(x_1)|$             & $|F(x_2)|$             \\
            \hline
            2    & -12          & 16              & 2.000000     & 4.000000     & 0                      & 0                      \\
            \hline
            2    & $-6\sqrt{2}$ & 16              & 1.414214     & 2.828427     &                        &                        \\
            \hline
            2    & 4            & -16             & 2.000000     & -4.000000    & 0                      & 0                      \\
            \hline
            1.5  & -6           & $-\frac{14}{3}$ & 4.666667     & -0.666667    & $2.666667\cdot10^{-6}$ & $2.666667\cdot10^{-6}$ \\
            \hline
            2    & 6            & 0               & -3.000000    & 0.000000     & 0                      & 0                      \\
            \hline
            3    & 1            & 0               & -0.333333    & 0.000000     & $3.333333\cdot10^{-7}$ & 0                      \\
            \hline
            2    & -6           & 0               & 3.000000     & 0.000000     & 0                      & 0                      \\
            \hline
            3    & -1           & 0               & 0.333333     & 0.000000     & $3.333333\cdot10^{-7}$ & 0                      \\
            \hline
            2    & 12           & 16              & -2.000000    & -4.000000    & 0                      & 0                      \\
            \hline
            2    & $6\sqrt{2}$  & 8               & -1.414214    & -2.828427    & $1.237795\cdot10^{-6}$ & $3.528354\cdot10^{-6}$ \\
            \hline
            -2   & -12          & -16             & -2.000000    & -4.000000    & 0                      & 0                      \\
            \hline
            -2   & $-6\sqrt{2}$ & -8              & 1.414214     & 2.828427     & $1.237795\cdot10^{-6}$ & $3.528354\cdot10^{-6}$ \\
            \hline
            -2   & 4            & 16              & -2.000000    & 4.000000     & 0                      & 0                      \\
            \hline
            -1.5 & -6           & $\frac{14}{3}$  & -4.666667    & 0.666667     & $2.666667\cdot10^{-6}$ & $2.666667\cdot10^{-6}$ \\
            \hline
            -2   & -6           & 0               & -3.000000    & 0.000000     & 0                      & 0                      \\
            \hline
            -3   & -1           & 0               & -0.333333    & 0.000000     & $3.333333\cdot10^{-7}$ & 0                      \\
            \hline
            -2   & 6            & 0               & 3.000000     & 0.000000     & 0                      & 0                      \\
            \hline
            -3   & 1            & 0               & 0.333333     & -0.000000    & $3.333333\cdot10^{-7}$ & 0                      \\
            \hline
            -2   & 12           & -16             & 2.000000     & 4.000000     & 0                      & 0                      \\
            \hline
            2    & $-6\sqrt{2}$ & 8               & 1.414214     & 2.828427     & $1.237795\cdot10^{-6}$ & $3.528354\cdot10^{-6}$ \\
            \hline
        \end{tabular}
    }{Результаты выполнения функции}{tab:task1-res-table}

    \subsection{Тестовые данные}
    \labeltable{
        \begin{tabular}{|c|c|c|c|c|}
            \hline
            a    & b            & c               & $x_1$           & $x_2$          \\
            \hline
            2    & -12          & 16              & 2               & 4              \\
            \hline
            2    & $-6\sqrt{2}$ & 16              & $\sqrt{2}$      & $2\sqrt{2}$    \\
            \hline
            2    & 4            & -16             & 2               & -4             \\
            \hline
            1.5  & -6           & $-\frac{14}{3}$ & $4\frac{2}{3}$  & $-\frac{2}{3}$ \\
            \hline
            2    & 6            & 0               & -3              & 0              \\
            \hline
            3    & 1            & 0               & $-\frac{1}{3}$  & 0              \\
            \hline
            2    & -6           & 0               & 3               & 0              \\
            \hline
            3    & -1           & 0               & $\frac{1}{3}$   & 0              \\
            \hline
            2    & 12           & 16              & -2              & -4             \\
            \hline
            2    & $6\sqrt{2}$  & 8               & $-\sqrt{2}$     & $-2\sqrt{2}$   \\
            \hline
            -2   & -12          & -16             & -2              & -4             \\
            \hline
            -2   & $-6\sqrt{2}$ & -8              & $\sqrt{2}$      & $2\sqrt{2}$    \\
            \hline
            -2   & 4            & 16              & -2              & 4              \\
            \hline
            -1.5 & -6           & $\frac{14}{3}$  & $-4\frac{2}{3}$ & $\frac{2}{3}$  \\
            \hline
            -2   & -6           & 0               & -3              & 0              \\
            \hline
            -3   & -1           & 0               & $-\frac{1}{3}$  & 0              \\
            \hline
            -2   & 6            & 0               & 3               & 0              \\
            \hline
            -3   & 1            & 0               & $\frac{1}{3}$   & 0              \\
            \hline
            -2   & 12           & -16             & 2               & 4              \\
            \hline
            2    & $-6\sqrt{2}$ & 8               & $\sqrt{2}$      & $2\sqrt{2}$    \\
            \hline
        \end{tabular}
    }{Тестовые данные solveSquareEquation}{tab:task1-test-table}

    \subsection{Результат выполнения тестов}
    \makenewfig{
        \includegraphics[width=120mm]{/task1tests}}{Результат выполнения тестов функции solveSquareEquation}


    \section{Разветвляющиеся алгоритмы}

    \subsection{Задание варианта}
    Можно ли на прямоугольном участке застройки размером a на b метров разместить два дома размером в планер q на p и r на s метров? Дома можно располагать только параллельно сторонам участка.

    \subsection{Обоснование}
    \textbf{Обоснование:}

    Легче всего данную задачу представить в графическом формате. Возможно 8 способов расстановки домов:\\
    \makenewfig{\includegraphics[width=80mm]{/task2}}{Все возможные перестановки домов}
    Если одна из расстановок помещается в a на b - значит дома разместить можно, иначе - нет.

    \subsection{Спецификация}
    \begin{enumerate}
        \item Заголовок:  \mintinline{C}|static int max(int a, int b)|
        \item Назначение: Возвращает максимальное из \mintinline{C}|a| и \mintinline{C}|b|
    \end{enumerate}
    \hline
    \begin{enumerate}
        \item Заголовок:  \mintinline{C}|void solveSquareEquation(float a, float b, float c, float *x1, float *x2)|
        \item Назначение: принимает коэффициенты уравнения вида \mintinline{C}|a| $\cdot x^2 + $\mintinline{C}|b|$ \cdot x + $\mintinline{C}|c|$ = 0$, сохраняет его корни в \mintinline{C}|x1|, \mintinline{C}|x2|.
    \end{enumerate}

    \subsection{Блок-схема}
    \makenewfig{\documentclass{article}

\usepackage[utf8]{inputenc}
\usepackage[english,russian]{babel}
\usepackage{pgfplots}
\usepackage{verbatim}
\usetikzlibrary{positioning}
\usetikzlibrary{shapes.geometric}
\usetikzlibrary{shapes.misc}
\usetikzlibrary{calc}
\usetikzlibrary{chains}
\usetikzlibrary{matrix}
\usetikzlibrary{decorations.text}
\usepackage{fontspec}
\usetikzlibrary{backgrounds}

\begin{document}
    \begin{tikzpicture}[every node/.style={inner sep=0,outer sep=0}, background rectangle/.style={opacity=1.000000, fill=colorODFMNARNCAEQSIMT}, show background rectangle]
        \makeatletter
        \newcommand{\verbatimfont}[1]{\def\verbatim@font{#1}}%
        \makeatother
        \newfontfamily\fontODFMNARNBEOOANAH[Scale=0.858984, SizeFeatures={Size=10.000000}]{Consolas}
        \definecolor{colorODFMNARNCAEQSIMT}{rgb}{1.000000,1.000000,1.000000}
        \definecolor{colorODFMNARNHOEDKRCG}{rgb}{0.949020,0.949020,0.949020}
        \definecolor{colorODFMNARNTGBHTD}{rgb}{0.000000,0.000000,0.000000}
        \draw[opacity=1.000000, rounded corners=0.333333 cm, fill=colorODFMNARNHOEDKRCG, draw=none] (1.163636,0.006061) -- (3.830303,0.006061) -- (3.830303,-0.660606) -- (1.163636,-0.660606) -- cycle;
        \draw[opacity=1.000000, rounded corners=0.333333 cm, fill=none, line width=0.030303cm, colorODFMNARNTGBHTD] (1.163636,0.006061) -- (3.830303,0.006061) -- (3.830303,-0.660606) -- (1.163636,-0.660606) -- cycle;
        \verbatimfont{\normalsize\fontODFMNARNBEOOANAH}
        \node[opacity=1.000000, above right, colorODFMNARNTGBHTD] at(1.737578, -0.475461) {\verb|max(a, b)|};
        \draw[opacity=1.000000, fill=colorODFMNARNHOEDKRCG, draw=none] (1.593939, -1.536364) -- (2.496970, -1.987879) -- (3.400000, -1.536364) -- (2.496970, -1.084848) --cycle;
        \draw[opacity=1.000000, fill=none, line width=0.030303cm, colorODFMNARNTGBHTD] (1.593939, -1.536364) -- (2.496970, -1.987879) -- (3.400000, -1.536364) -- (2.496970, -1.084848) --cycle;
        \verbatimfont{\normalsize\fontODFMNARNBEOOANAH}
        \node[opacity=1.000000, above right, colorODFMNARNTGBHTD] at(2.073123, -1.648188) {\verb|a > b|};
        \draw[opacity=1.000000, fill=none, line width=0.030303cm, colorODFMNARNTGBHTD] (1.593939, -1.536364) -- (1.093939, -1.536364) -- (1.093939, -2.412121);
        \draw[fill=colorODFMNARNHOEDKRCG, draw=none, opacity=1.000000] (-0.006061,-2.412121) -- (2.193939,-2.412121) -- (2.193939,-3.078788) -- (-0.006061,-3.078788) -- cycle;
        \draw[opacity=1.000000, fill=none, line width=0.030303cm, colorODFMNARNTGBHTD] (-0.006061,-2.412121) -- (2.193939,-2.412121) -- (2.193939,-3.078788) -- (-0.006061,-3.078788) -- cycle;
        \verbatimfont{\normalsize\fontODFMNARNBEOOANAH}
        \node[opacity=1.000000, above right, colorODFMNARNTGBHTD] at(0.175605, -2.893642) {\verb|result := a|};
        \verbatimfont{\normalsize\fontODFMNARNBEOOANAH}
        \node[opacity=1.000000, above right, colorODFMNARNTGBHTD] at(1.093939, -1.354545) {\verb|+|};
        \draw[opacity=1.000000, fill=none, line width=0.030303cm, colorODFMNARNTGBHTD] (3.400000, -1.536364) -- (3.900000, -1.536364) -- (3.900000, -2.412121);
        \verbatimfont{\normalsize\fontODFMNARNBEOOANAH}
        \node[opacity=1.000000, above right, colorODFMNARNTGBHTD] at(3.723397, -1.354545) {\verb|-|};
        \draw[fill=colorODFMNARNHOEDKRCG, draw=none, opacity=1.000000] (2.800000,-2.412121) -- (5.000000,-2.412121) -- (5.000000,-3.078788) -- (2.800000,-3.078788) -- cycle;
        \draw[opacity=1.000000, fill=none, line width=0.030303cm, colorODFMNARNTGBHTD] (2.800000,-2.412121) -- (5.000000,-2.412121) -- (5.000000,-3.078788) -- (2.800000,-3.078788) -- cycle;
        \verbatimfont{\normalsize\fontODFMNARNBEOOANAH}
        \node[opacity=1.000000, above right, colorODFMNARNTGBHTD] at(2.981666, -2.893642) {\verb|result := b|};
        \draw[opacity=1.000000, fill=none, line width=0.030303cm, colorODFMNARNTGBHTD] (3.900000, -3.078788) -- (3.900000, -3.503030) -- (2.496970, -3.503030);
        \draw[opacity=1.000000, fill=none, line width=0.030303cm, colorODFMNARNTGBHTD] (1.093939, -3.078788) -- (1.093939, -3.503030) -- (2.496970, -3.503030);
        \draw[opacity=1.000000, fill=none, line width=0.030303 cm, colorODFMNARNTGBHTD] (2.496970,-0.660606) -- (2.496970,-1.084848);
        \draw[opacity=1.000000, rounded corners=0.333333 cm, fill=colorODFMNARNHOEDKRCG, draw=none] (0.996970,-3.927273) -- (3.996970,-3.927273) -- (3.996970,-4.593939) -- (0.996970,-4.593939) -- cycle;
        \draw[opacity=1.000000, rounded corners=0.333333 cm, fill=none, line width=0.030303cm, colorODFMNARNTGBHTD] (0.996970,-3.927273) -- (3.996970,-3.927273) -- (3.996970,-4.593939) -- (0.996970,-4.593939) -- cycle;
        \verbatimfont{\normalsize\fontODFMNARNBEOOANAH}
        \node[opacity=1.000000, above right, colorODFMNARNTGBHTD] at(1.313732, -4.408794) {\verb|Вернуть result|};
        \draw[opacity=1.000000, fill=none, line width=0.030303 cm, colorODFMNARNTGBHTD] (2.496970,-3.503030) -- (2.496970,-3.927273);
    \end{tikzpicture}
\end{document}}{Функция max}{ris:task2_max}
    \makenewfig{\documentclass{article}

\usepackage[utf8]{inputenc}
\usepackage[english,russian]{babel}
\usepackage{pgfplots}
\usepackage{verbatim}
\usetikzlibrary{positioning}
\usetikzlibrary{shapes.geometric}
\usetikzlibrary{shapes.misc}
\usetikzlibrary{calc}
\usetikzlibrary{chains}
\usetikzlibrary{matrix}
\usetikzlibrary{decorations.text}
\usepackage{fontspec}
\usetikzlibrary{backgrounds}

\begin{document}
    \begin{tikzpicture}[every node/.style={inner sep=0,outer sep=0}, background rectangle/.style={opacity=1.000000, fill=colorIQMGDBHODPQINRE}, show background rectangle]
        \makeatletter
        \newcommand{\verbatimfont}[1]{\def\verbatim@font{#1}}%
        \makeatother
        \newfontfamily\fontIQMGDBHIEJNDTIF[Scale=0.635571, SizeFeatures={Size=10.000000}]{Consolas}
        \definecolor{colorIQMGDBHODPQINRE}{rgb}{1.000000,1.000000,1.000000}
        \definecolor{colorIQMGDBHHGJIEPC}{rgb}{0.949020,0.949020,0.949020}
        \definecolor{colorIQMGDBHOOITRR}{rgb}{0.000000,0.000000,0.000000}
        \draw[opacity=1.000000, rounded corners=0.246637 cm, fill=colorIQMGDBHHGJIEPC, draw=none] (2.067265,0.004484) -- (7.865471,0.004484) -- (7.865471,-0.488789) -- (2.067265,-0.488789) -- cycle;
        \draw[opacity=1.000000, rounded corners=0.246637 cm, fill=none, line width=0.022422cm, colorIQMGDBHOOITRR] (2.067265,0.004484) -- (7.865471,0.004484) -- (7.865471,-0.488789) -- (2.067265,-0.488789) -- cycle;
        \verbatimfont{\normalsize\fontIQMGDBHIEJNDTIF}
        \node[opacity=1.000000, above right, colorIQMGDBHOOITRR] at(2.301701, -0.363647) {\verb|canPlaceTwoHousesInSquare(a, b, p, q, r, s)|};
        \draw[fill=colorIQMGDBHHGJIEPC, draw=none, opacity=1.000000] (3.647982,-0.802691) -- (6.284753,-0.802691) -- (6.284753,-1.295964) -- (3.647982,-1.295964) -- cycle;
        \draw[opacity=1.000000, fill=none, line width=0.022422cm, colorIQMGDBHOOITRR] (3.647982,-0.802691) -- (6.284753,-0.802691) -- (6.284753,-1.295964) -- (3.647982,-1.295964) -- cycle;
        \draw[opacity=1.000000, fill=none, line width=0.022422cm, colorIQMGDBHOOITRR] (3.721973,-0.802691) -- (6.210762,-0.802691) -- (6.210762,-1.295964) -- (3.721973,-1.295964) -- cycle;
        \verbatimfont{\normalsize\fontIQMGDBHIEJNDTIF}
        \node[opacity=1.000000, above right, colorIQMGDBHOOITRR] at(3.854619, -1.170822) {\verb|maxPS := max(p, s)|};
        \draw[opacity=1.000000, fill=none, line width=0.022422 cm, colorIQMGDBHOOITRR] (4.966368,-0.488789) -- (4.966368,-0.802691);
        \draw[fill=colorIQMGDBHHGJIEPC, draw=none, opacity=1.000000] (3.647982,-1.609865) -- (6.284753,-1.609865) -- (6.284753,-2.103139) -- (3.647982,-2.103139) -- cycle;
        \draw[opacity=1.000000, fill=none, line width=0.022422cm, colorIQMGDBHOOITRR] (3.647982,-1.609865) -- (6.284753,-1.609865) -- (6.284753,-2.103139) -- (3.647982,-2.103139) -- cycle;
        \draw[opacity=1.000000, fill=none, line width=0.022422cm, colorIQMGDBHOOITRR] (3.721973,-1.609865) -- (6.210762,-1.609865) -- (6.210762,-2.103139) -- (3.721973,-2.103139) -- cycle;
        \verbatimfont{\normalsize\fontIQMGDBHIEJNDTIF}
        \node[opacity=1.000000, above right, colorIQMGDBHOOITRR] at(3.854619, -1.977997) {\verb|maxPR := max(p, r)|};
        \draw[opacity=1.000000, fill=none, line width=0.022422 cm, colorIQMGDBHOOITRR] (4.966368,-1.295964) -- (4.966368,-1.609865);
        \draw[fill=colorIQMGDBHHGJIEPC, draw=none, opacity=1.000000] (3.647982,-2.417040) -- (6.284753,-2.417040) -- (6.284753,-2.910314) -- (3.647982,-2.910314) -- cycle;
        \draw[opacity=1.000000, fill=none, line width=0.022422cm, colorIQMGDBHOOITRR] (3.647982,-2.417040) -- (6.284753,-2.417040) -- (6.284753,-2.910314) -- (3.647982,-2.910314) -- cycle;
        \draw[opacity=1.000000, fill=none, line width=0.022422cm, colorIQMGDBHOOITRR] (3.721973,-2.417040) -- (6.210762,-2.417040) -- (6.210762,-2.910314) -- (3.721973,-2.910314) -- cycle;
        \verbatimfont{\normalsize\fontIQMGDBHIEJNDTIF}
        \node[opacity=1.000000, above right, colorIQMGDBHOOITRR] at(3.854619, -2.785172) {\verb|maxQS := max(q, s)|};
        \draw[opacity=1.000000, fill=none, line width=0.022422 cm, colorIQMGDBHOOITRR] (4.966368,-2.103139) -- (4.966368,-2.417040);
        \draw[fill=colorIQMGDBHHGJIEPC, draw=none, opacity=1.000000] (3.647982,-3.224215) -- (6.284753,-3.224215) -- (6.284753,-3.717489) -- (3.647982,-3.717489) -- cycle;
        \draw[opacity=1.000000, fill=none, line width=0.022422cm, colorIQMGDBHOOITRR] (3.647982,-3.224215) -- (6.284753,-3.224215) -- (6.284753,-3.717489) -- (3.647982,-3.717489) -- cycle;
        \draw[opacity=1.000000, fill=none, line width=0.022422cm, colorIQMGDBHOOITRR] (3.721973,-3.224215) -- (6.210762,-3.224215) -- (6.210762,-3.717489) -- (3.721973,-3.717489) -- cycle;
        \verbatimfont{\normalsize\fontIQMGDBHIEJNDTIF}
        \node[opacity=1.000000, above right, colorIQMGDBHOOITRR] at(3.854619, -3.592347) {\verb|maxQR := max(q, r)|};
        \draw[opacity=1.000000, fill=none, line width=0.022422 cm, colorIQMGDBHOOITRR] (4.966368,-2.910314) -- (4.966368,-3.224215);
        \draw[opacity=1.000000, fill=colorIQMGDBHHGJIEPC, draw=none] (1.340807, -5.844170) -- (4.966368, -7.656951) -- (8.591928, -5.844170) -- (4.966368, -4.031390) --cycle;
        \draw[opacity=1.000000, fill=none, line width=0.022422cm, colorIQMGDBHOOITRR] (1.340807, -5.844170) -- (4.966368, -7.656951) -- (8.591928, -5.844170) -- (4.966368, -4.031390) --cycle;
        \verbatimfont{\normalsize\fontIQMGDBHIEJNDTIF}
        \node[opacity=1.000000, above right, colorIQMGDBHOOITRR] at(3.360509, -5.106921) {\verb|maxPS ≤ a and q + r ≤ b or|};
        \verbatimfont{\normalsize\fontIQMGDBHIEJNDTIF}
        \node[opacity=1.000000, above right, colorIQMGDBHOOITRR] at(3.360509, -5.367429) {\verb|maxPR ≤ a and q + s ≤ b or|};
        \verbatimfont{\normalsize\fontIQMGDBHIEJNDTIF}
        \node[opacity=1.000000, above right, colorIQMGDBHOOITRR] at(3.360509, -5.627937) {\verb|maxQS ≤ a and p + r ≤ b or|};
        \verbatimfont{\normalsize\fontIQMGDBHIEJNDTIF}
        \node[opacity=1.000000, above right, colorIQMGDBHOOITRR] at(3.360509, -5.888445) {\verb|maxQR ≤ a and p + s ≤ b or|};
        \verbatimfont{\normalsize\fontIQMGDBHIEJNDTIF}
        \node[opacity=1.000000, above right, colorIQMGDBHOOITRR] at(3.360509, -6.148953) {\verb|maxQR ≤ b and p + s ≤ a or|};
        \verbatimfont{\normalsize\fontIQMGDBHIEJNDTIF}
        \node[opacity=1.000000, above right, colorIQMGDBHOOITRR] at(3.360509, -6.409461) {\verb|maxQS ≤ b and p + r ≤ a or|};
        \verbatimfont{\normalsize\fontIQMGDBHIEJNDTIF}
        \node[opacity=1.000000, above right, colorIQMGDBHOOITRR] at(3.360509, -6.669969) {\verb|maxPR ≤ b and q + s ≤ a or|};
        \verbatimfont{\normalsize\fontIQMGDBHIEJNDTIF}
        \node[opacity=1.000000, above right, colorIQMGDBHOOITRR] at(3.536977, -6.930477) {\verb|maxPS ≤ b and q + r ≤ a|};
        \draw[opacity=1.000000, fill=none, line width=0.022422cm, colorIQMGDBHOOITRR] (1.340807, -5.844170) -- (1.116592, -5.844170) -- (1.116592, -7.970852);
        \draw[fill=colorIQMGDBHHGJIEPC, draw=none, opacity=1.000000] (-0.004484,-7.970852) -- (2.237668,-7.970852) -- (2.237668,-8.464126) -- (-0.004484,-8.464126) -- cycle;
        \draw[opacity=1.000000, fill=none, line width=0.022422cm, colorIQMGDBHOOITRR] (-0.004484,-7.970852) -- (2.237668,-7.970852) -- (2.237668,-8.464126) -- (-0.004484,-8.464126) -- cycle;
        \verbatimfont{\normalsize\fontIQMGDBHIEJNDTIF}
        \node[opacity=1.000000, above right, colorIQMGDBHOOITRR] at(0.128371, -8.338983) {\verb|canPlace := true|};
        \verbatimfont{\normalsize\fontIQMGDBHIEJNDTIF}
        \node[opacity=1.000000, above right, colorIQMGDBHOOITRR] at(1.116592, -5.709641) {\verb|+|};
        \draw[opacity=1.000000, fill=none, line width=0.022422cm, colorIQMGDBHOOITRR] (8.591928, -5.844170) -- (8.816143, -5.844170) -- (8.816143, -7.970852);
        \verbatimfont{\normalsize\fontIQMGDBHIEJNDTIF}
        \node[opacity=1.000000, above right, colorIQMGDBHOOITRR] at(8.674969, -5.709641) {\verb|-|};
        \draw[fill=colorIQMGDBHHGJIEPC, draw=none, opacity=1.000000] (7.632287,-7.970852) -- (10.000000,-7.970852) -- (10.000000,-8.464126) -- (7.632287,-8.464126) -- cycle;
        \draw[opacity=1.000000, fill=none, line width=0.022422cm, colorIQMGDBHOOITRR] (7.632287,-7.970852) -- (10.000000,-7.970852) -- (10.000000,-8.464126) -- (7.632287,-8.464126) -- cycle;
        \verbatimfont{\normalsize\fontIQMGDBHIEJNDTIF}
        \node[opacity=1.000000, above right, colorIQMGDBHOOITRR] at(7.757336, -8.338983) {\verb|canPlace := false|};
        \draw[opacity=1.000000, fill=none, line width=0.022422cm, colorIQMGDBHOOITRR] (8.816143, -8.464126) -- (8.816143, -8.778027) -- (4.966368, -8.778027);
        \draw[opacity=1.000000, fill=none, line width=0.022422cm, colorIQMGDBHOOITRR] (1.116592, -8.464126) -- (1.116592, -8.778027) -- (4.966368, -8.778027);
        \draw[opacity=1.000000, fill=none, line width=0.022422 cm, colorIQMGDBHOOITRR] (4.966368,-3.717489) -- (4.966368,-4.031390);
        \draw[opacity=1.000000, rounded corners=0.246637 cm, fill=colorIQMGDBHHGJIEPC, draw=none] (3.733184,-9.091928) -- (6.199552,-9.091928) -- (6.199552,-9.585202) -- (3.733184,-9.585202) -- cycle;
        \draw[opacity=1.000000, rounded corners=0.246637 cm, fill=none, line width=0.022422cm, colorIQMGDBHOOITRR] (3.733184,-9.091928) -- (6.199552,-9.091928) -- (6.199552,-9.585202) -- (3.733184,-9.585202) -- cycle;
        \verbatimfont{\normalsize\fontIQMGDBHIEJNDTIF}
        \node[opacity=1.000000, above right, colorIQMGDBHOOITRR] at(3.978147, -9.460060) {\verb|Вернуть canPlace|};
        \draw[opacity=1.000000, fill=none, line width=0.022422 cm, colorIQMGDBHOOITRR] (4.966368,-8.778027) -- (4.966368,-9.091928);
    \end{tikzpicture}
\end{document}}{Функция canPlaceTwoHousesInSquare}{ris:task2}

    \subsection{Код программы}
    \begin{minted}{C}
#include <stdbool.h>

static int max(int a, int b) {
    return a > b ? a : b;
}

bool canPlaceTwoHousesInSquare(int a, int b, int p, int q, int r, int s) {
    return max(p, s) <= a && q + r <= b ||
           max(p, r) <= a && q + s <= b ||
           max(q, s) <= a && p + r <= b ||
           max(q, r) <= a && p + s <= b ||
           p + s <= a && max(q, r) <= b ||
           p + r <= a && max(q, s) <= b ||
           q + s <= a && max(p, r) <= b ||
           q + r <= a && max(p, s) <= b;
}
    \end{minted}

    \subsection{Тестовые данные}

    \labeltable{
        \begin{tabular}{|c|c|}
            \hline
            Тестовые данные & Ожидаемый результат \\
            \hline
            a = 0, b = 0,
            p = 2, q = 3,
            r = 1, s = 5    & false               \\
            \hline
            a = 2, b = 5,
            p = 2, q = 3,
            r = 1, s = 5    & false               \\
            \hline
            a = 1, b = 500,
            p = 2, q = 3,
            r = 1, s = 5    & false               \\
            \hline
            a = 4, b = 6,
            p = 4, q = 2,
            r = 4, s = 4    & true                \\
            \hline
            a = 4, b = 8,
            p = 2, q = 5,
            r = 3, s = 3    & true                \\
            \hline
        \end{tabular}
    }{Тестовые данные canPlaceTwoHousesInSquare}{tab:task2-test-table}

    \subsection{Результат выполнения тестов}
    \makenewfig{
        \includegraphics[width=120mm]{/task2tests}}{Результат выполнения тестов функции canPlaceTwoHousesInSquare}


    \section{Циклические и итерационные алгоритмы}

    \subsection{Задание варианта}
    Известно время начала и окончания (например, 6:00 и 24:00) работы некоторого пригородного автобусного маршрута с одним автобусом на линии, а также протяжённость маршрута в минутах (в один конец) и время отдыха на конечных остановках. Составить суточное расписание этого маршрута (моменты отправления с конечных пунктов) без учета времени на обед и пересменку.

    \subsection{Обоснование}
    Будем определять время начала работы в минутах и общее время работы.
    В цикле инициаилизирующее выражение - инициализируем счётчик *scheduleSize = 0,
    на каждой итерации цикла будет его постепенно увеличивать.
    В условии возобновлении цикла будем проверять, что время старта +
    время работы (работник вряд-ли будет счастлив, если выедет ровно во время окончания
    рабочего дня и будет работать дополнительно, например, 2 часа) не будет
    превышать общее время работы.
    Пока это условие выполняется, будем сохранять время выезда в массив.
    Также в условии возобновления цикла проверим, что мы не превысим максимальное количество
    записей в расписании, иначе можем получить ошибку во время выполнения программы.

    \subsection{Спецификация}
    \begin{enumerate}
        \item Заголовок:  \mintinline{C}|getBusSchedule(Clock start, Clock end, int restTime, int travelTime, Clock *schedule,|\\
        \mintinline{C}|int *scheduleSize, int maxScheduleSize)|
        \item Назначение: сохраняет в массив \mintinline{C}|schedule| максимального размера\\ \mintinline{C}|maxScheduleSize| расписание автобуса при условии, что он начинает работу в
        \mintinline{C}|start|, оканчивает в \mintinline{C}|end|, отдыхает на остановках \mintinline{C}|restTime| имеет длительность маршрута \mintinline{C}|travelTime|. Сохраняет количество записей в \mintinline{C}|scheduleSize|.
    \end{enumerate}

    \subsection{Блок-схема}
    \makenewfig{\documentclass{article}

\usepackage[utf8]{inputenc}
\usepackage[english,russian]{babel}
\usepackage{pgfplots}
\usepackage{verbatim}
\usetikzlibrary{positioning}
\usetikzlibrary{shapes.geometric}
\usetikzlibrary{shapes.misc}
\usetikzlibrary{calc}
\usetikzlibrary{chains}
\usetikzlibrary{matrix}
\usetikzlibrary{decorations.text}
\usepackage{fontspec}
\usetikzlibrary{backgrounds}

\begin{document}
    \begin{tikzpicture}[every node/.style={inner sep=0,outer sep=0}, background rectangle/.style={opacity=1.000000, fill=colorBBMJKKLACMIT}, show background rectangle]
        \makeatletter
        \newcommand{\verbatimfont}[1]{\def\verbatim@font{#1}}%
        \makeatother
        \newfontfamily\fontBBMJKKAKHFFHGA[Scale=0.668233, SizeFeatures={Size=10.000000}]{Consolas}
        \definecolor{colorBBMJKKLACMIT}{rgb}{1.000000,1.000000,1.000000}
        \definecolor{colorBBMJKKBKDMIDR}{rgb}{0.949020,0.949020,0.949020}
        \definecolor{colorBBMJKKHRDOOTSO}{rgb}{0.000000,0.000000,0.000000}
        \draw[opacity=1.000000, rounded corners=0.509194 cm, fill=colorBBMJKKBKDMIDR, draw=none] (2.088637,0.004715) -- (7.906648,0.004715) -- (7.906648,-1.013673) -- (2.088637,-1.013673) -- cycle;
        \draw[opacity=1.000000, rounded corners=0.509194 cm, fill=none, line width=0.023574cm, colorBBMJKKHRDOOTSO] (2.088637,0.004715) -- (7.906648,0.004715) -- (7.906648,-1.013673) -- (2.088637,-1.013673) -- cycle;
        \verbatimfont{\normalsize\fontBBMJKKAKHFFHGA}
        \node[opacity=1.000000, above right, colorBBMJKKHRDOOTSO] at(2.597623, -0.369645) {\verb|getBusSchedule(start, end, restTime, |};
        \verbatimfont{\normalsize\fontBBMJKKAKHFFHGA}
        \node[opacity=1.000000, above right, colorBBMJKKHRDOOTSO] at(2.650564, -0.630850) {\verb|travelTime, schedule, scheduleSize, |};
        \verbatimfont{\normalsize\fontBBMJKKAKHFFHGA}
        \node[opacity=1.000000, above right, colorBBMJKKHRDOOTSO] at(3.956458, -0.892055) {\verb|maxScheduleSize)|};
        \draw[fill=colorBBMJKKBKDMIDR, draw=none, opacity=1.000000] (3.429986,-1.343706) -- (6.565299,-1.343706) -- (6.565299,-1.862329) -- (3.429986,-1.862329) -- cycle;
        \draw[opacity=1.000000, fill=none, line width=0.023574cm, colorBBMJKKHRDOOTSO] (3.429986,-1.343706) -- (6.565299,-1.343706) -- (6.565299,-1.862329) -- (3.429986,-1.862329) -- cycle;
        \verbatimfont{\normalsize\fontBBMJKKAKHFFHGA}
        \node[opacity=1.000000, above right, colorBBMJKKHRDOOTSO] at(3.568219, -1.718065) {\verb|MINUTES_IN_HOURS := 60|};
        \draw[opacity=1.000000, fill=none, line width=0.023574 cm, colorBBMJKKHRDOOTSO] (4.997643,-1.013673) -- (4.997643,-1.343706);
        \draw[fill=colorBBMJKKBKDMIDR, draw=none, opacity=1.000000] (0.773220,-2.192362) -- (9.222065,-2.192362) -- (9.222065,-2.710985) -- (0.773220,-2.710985) -- cycle;
        \draw[opacity=1.000000, fill=none, line width=0.023574cm, colorBBMJKKHRDOOTSO] (0.773220,-2.192362) -- (9.222065,-2.192362) -- (9.222065,-2.710985) -- (0.773220,-2.710985) -- cycle;
        \verbatimfont{\normalsize\fontBBMJKKAKHFFHGA}
        \node[opacity=1.000000, above right, colorBBMJKKHRDOOTSO] at(0.903491, -2.566721) {\verb|workTimeStart := start.hours * MINUTES_IN_HOURS + start.minutes|};
        \draw[opacity=1.000000, fill=none, line width=0.023574 cm, colorBBMJKKHRDOOTSO] (4.997643,-1.862329) -- (4.997643,-2.192362);
        \draw[fill=colorBBMJKKBKDMIDR, draw=none, opacity=1.000000] (-0.004715,-3.041018) -- (10.000000,-3.041018) -- (10.000000,-3.559642) -- (-0.004715,-3.559642) -- cycle;
        \draw[opacity=1.000000, fill=none, line width=0.023574cm, colorBBMJKKHRDOOTSO] (-0.004715,-3.041018) -- (10.000000,-3.041018) -- (10.000000,-3.559642) -- (-0.004715,-3.559642) -- cycle;
        \verbatimfont{\normalsize\fontBBMJKKAKHFFHGA}
        \node[opacity=1.000000, above right, colorBBMJKKHRDOOTSO] at(0.127014, -3.415378) {\verb|workTimeTotal := end.hours * MINUTES_IN_HOURS + end.minutes - workTimeStart|};
        \draw[opacity=1.000000, fill=none, line width=0.023574 cm, colorBBMJKKHRDOOTSO] (4.997643,-2.710985) -- (4.997643,-3.041018);
        \draw[fill=colorBBMJKKBKDMIDR, draw=none, opacity=1.000000] (3.752947,-3.889675) -- (6.242339,-3.889675) -- (6.242339,-4.408298) -- (3.752947,-4.408298) -- cycle;
        \draw[opacity=1.000000, fill=none, line width=0.023574cm, colorBBMJKKHRDOOTSO] (3.752947,-3.889675) -- (6.242339,-3.889675) -- (6.242339,-4.408298) -- (3.752947,-4.408298) -- cycle;
        \verbatimfont{\normalsize\fontBBMJKKAKHFFHGA}
        \node[opacity=1.000000, above right, colorBBMJKKHRDOOTSO] at(3.885869, -4.264034) {\verb|*scheduleSize = 0|};
        \draw[opacity=1.000000, fill=none, line width=0.023574 cm, colorBBMJKKHRDOOTSO] (4.997643,-3.559642) -- (4.997643,-3.889675);
        \draw[opacity=1.000000, fill=colorBBMJKKBKDMIDR, draw=none] (0.619991, -6.927157) -- (4.997643, -9.115983) -- (9.375295, -6.927157) -- (4.997643, -4.738331) --cycle;
        \draw[opacity=1.000000, fill=none, line width=0.023574cm, colorBBMJKKHRDOOTSO] (0.619991, -6.927157) -- (4.997643, -9.115983) -- (9.375295, -6.927157) -- (4.997643, -4.738331) --cycle;
        \verbatimfont{\normalsize\fontBBMJKKAKHFFHGA}
        \node[opacity=1.000000, above right, colorBBMJKKHRDOOTSO] at(2.791742, -6.764034) {\verb|scheduleSize < maxScheduleSize and|};
        \verbatimfont{\normalsize\fontBBMJKKAKHFFHGA}
        \node[opacity=1.000000, above right, colorBBMJKKHRDOOTSO] at(1.485849, -7.025239) {\verb|scheduleSize * (restTime + travelTime) + travelTime ≤ |};
        \verbatimfont{\normalsize\fontBBMJKKAKHFFHGA}
        \node[opacity=1.000000, above right, colorBBMJKKHRDOOTSO] at(4.150577, -7.286444) {\verb|workTimeTotal|};
        \draw[fill=colorBBMJKKBKDMIDR, draw=none, opacity=1.000000] (0.318246,-9.446016) -- (9.677039,-9.446016) -- (9.677039,-9.964639) -- (0.318246,-9.964639) -- cycle;
        \draw[opacity=1.000000, fill=none, line width=0.023574cm, colorBBMJKKHRDOOTSO] (0.318246,-9.446016) -- (9.677039,-9.446016) -- (9.677039,-9.964639) -- (0.318246,-9.964639) -- cycle;
        \verbatimfont{\normalsize\fontBBMJKKAKHFFHGA}
        \node[opacity=1.000000, above right, colorBBMJKKHRDOOTSO] at(0.444664, -9.820375) {\verb|dispatchTime := workTimeStart + scheduleSize * (restTime + travelTime)|};
        \draw[fill=colorBBMJKKBKDMIDR, draw=none, opacity=1.000000] (0.902876,-10.294672) -- (9.092409,-10.294672) -- (9.092409,-11.063178) -- (0.902876,-11.063178) -- cycle;
        \draw[opacity=1.000000, fill=none, line width=0.023574cm, colorBBMJKKHRDOOTSO] (0.902876,-10.294672) -- (9.092409,-10.294672) -- (9.092409,-11.063178) -- (0.902876,-11.063178) -- cycle;
        \verbatimfont{\normalsize\fontBBMJKKAKHFFHGA}
        \node[opacity=1.000000, above right, colorBBMJKKHRDOOTSO] at(1.027021, -10.669032) {\verb|schedule[scheduleSize] := {dispatchTime div MINUTES_IN_HOURS,|};
        \verbatimfont{\normalsize\fontBBMJKKAKHFFHGA}
        \node[opacity=1.000000, above right, colorBBMJKKHRDOOTSO] at(2.791742, -10.930237) {\verb|dispatchTime mod MINUTES_IN_HOURS}|};
        \draw[opacity=1.000000, fill=none, line width=0.023574 cm, colorBBMJKKHRDOOTSO] (4.997643,-9.964639) -- (4.997643,-10.294672);
        \draw[opacity=1.000000, fill=none, line width=0.023574 cm, colorBBMJKKHRDOOTSO] (4.997643,-9.115983) -- (4.997643,-9.446016);
        \verbatimfont{\normalsize\fontBBMJKKAKHFFHGA}
        \node[opacity=1.000000, above right, colorBBMJKKHRDOOTSO] at(4.842321, -9.363044) {\verb|+|};
        \draw[opacity=1.000000, fill=none, line width=0.023574cm, colorBBMJKKHRDOOTSO] (4.997643, -11.063178) -- (4.997643, -11.393211) -- (0.082508, -11.393211) -- (0.082508, -4.573314) -- (4.997643, -4.573314);
        \verbatimfont{\normalsize\fontBBMJKKAKHFFHGA}
        \node[opacity=1.000000, above right, colorBBMJKKHRDOOTSO] at(9.771599, -6.785714) {\verb|-|};
        \draw[opacity=1.000000, fill=none, line width=0.023574cm, colorBBMJKKHRDOOTSO] (9.375295, -6.927157) -- (9.912777, -6.927157) -- (9.912777, -11.628949) -- (4.997643, -11.628949);
        \draw[opacity=1.000000, fill=none, line width=0.023574 cm, colorBBMJKKHRDOOTSO] (4.997643,-4.408298) -- (4.997643,-4.738331);
        \draw[opacity=1.000000, rounded corners=0.259312 cm, fill=colorBBMJKKBKDMIDR, draw=none] (2.793494,-11.958982) -- (7.201792,-11.958982) -- (7.201792,-12.477605) -- (2.793494,-12.477605) -- cycle;
        \draw[opacity=1.000000, rounded corners=0.259312 cm, fill=none, line width=0.023574cm, colorBBMJKKHRDOOTSO] (2.793494,-11.958982) -- (7.201792,-11.958982) -- (7.201792,-12.477605) -- (2.793494,-12.477605) -- cycle;
        \verbatimfont{\normalsize\fontBBMJKKAKHFFHGA}
        \node[opacity=1.000000, above right, colorBBMJKKHRDOOTSO] at(3.038803, -12.333341) {\verb|Выход (schedule, scheduleSize)|};
        \draw[opacity=1.000000, fill=none, line width=0.023574 cm, colorBBMJKKHRDOOTSO] (4.997643,-11.628949) -- (4.997643,-11.958982);
    \end{tikzpicture}
\end{document}}{Функция getBusSchedule}{ris:task3}

    \subsection{Код программы}
    \begin{minted}{C++}
#include <stddef.h>

#include "Tasks.h"

#define MINUTES_IN_HOURS 60

// Содержит время - hours (часы) и minutes (минуты)
typedef struct Clock {
    int hours;
    int minutes;
} Clock;

void getBusSchedule(Clock start, Clock end, int restTime, int travelTime, Clock *schedule, int *scheduleSize, int maxScheduleSize) {
    // Определяем время начала работы и её окончания
    int workTimeStart = start.hours * MINUTES_IN_HOURS + start.minutes;
    int workTimeTotal = end.hours * MINUTES_IN_HOURS + end.minutes - workTimeStart;

    for (*scheduleSize = 0; *scheduleSize < maxScheduleSize &&
    *scheduleSize * (restTime + travelTime) + travelTime <= workTimeTotal;
    (*scheduleSize)++) {
        // Определяем время отправки в минутах
        int dispatchTime = workTimeStart + (*scheduleSize) * (restTime + travelTime);
        // И сохраняем это время в расписание
        schedule[*scheduleSize] = (Clock) {dispatchTime / MINUTES_IN_HOURS,
                                           dispatchTime % MINUTES_IN_HOURS};
    }
}
    \end{minted}

    \subsection{Тестовые данные}

    \labeltable{
        \begin{tabular}{|c|c|}
            \hline
            Тестовые данные & Ожидаемый результат  \\
            \hline
            \begin{tabular}{c}
                begin = 12:00,
                end = 12:00,     \\
                restTime = 12,
                travelTime = 12, \\
                MAX\_SCHEDULE\_SIZE = 1000
            \end{tabular}
            & schedule = \{\}      \\
            \hline
            \begin{tabular}{c}
                begin = 12:00,
                end = 10:00,     \\
                restTime = 12,
                travelTime = 12, \\MAX\_SCHEDULE\_SIZE = 1000
            \end{tabular}
            & schedule = \{\}      \\
            \hline
            \begin{tabular}{c}
                begin = 12:00,
                end = 12:11,     \\
                restTime = 12,
                travelTime = 12, \\MAX\_SCHEDULE\_SIZE = 1000
            \end{tabular}
            & schedule = \{\}      \\
            \hline
            \begin{tabular}{c}
                begin = 12:00,
                end = 12:12,     \\
                restTime = 12,
                travelTime = 12, \\MAX\_SCHEDULE\_SIZE = 1000
            \end{tabular}
            & schedule = \{12:00\} \\
            \hline
            \begin{tabular}{c}
                begin = 10:00,
                end = 12:12,     \\
                restTime = 10,
                travelTime = 15, \\MAX\_SCHEDULE\_SIZE = 1000
            \end{tabular}
            &
            \begin{tabular}{c}
                schedule = \{10:00, 10:25, 10:50,\\ 11:15, 11:40\}
            \end{tabular} \\
            \hline
            \begin{tabular}{c}
                begin = 10:00,
                end = 11:00,    \\
                restTime = 10,
                travelTime = 0, \\MAX\_SCHEDULE\_SIZE = 1000
            \end{tabular}
            & \begin{tabular}{c}
                  schedule = \{10:00, 10:10, 10:20, \\10:30, 10:40, 10:50, 11:00\}
            \end{tabular} \\
            \hline
            \begin{tabular}{c}
                begin = 10:00,
                end = 11:00,     \\
                restTime = 0,
                travelTime = 10, \\MAX\_SCHEDULE\_SIZE = 1000
            \end{tabular}
            & \begin{tabular}{c}
                  schedule = \{10:00, 10:10, 10:20, \\10:30, 10:40, 10:50\}
            \end{tabular} \\
            \hline
            \begin{tabular}{c}
                begin = 10:00,
                end = 11:00,    \\
                restTime = 0,
                travelTime = 0, \\MAX\_SCHEDULE\_SIZE = 1000
            \end{tabular}
            & \begin{tabular}{c}
                  scheduleSize = 1000
            \end{tabular} \\
            \hline
        \end{tabular}
    }{Тестовые данные getBusSchedule}{tab:task3-test-table}

    \subsection{Результат выполнения тестов}
    \makenewfig{
        \includegraphics[width=120mm]{/task3tests}}{Результат выполнения тестов функции getBusSchedule}


    \section{Простейшие операции над массивами}

    \subsection{Задание варианта}
    Найти среднее арифметическое элементов каждой строки матрицы $Q(l, m)$ и вычесть его из элементов этой строки.

    \subsection{Обоснование}
    Сначала вычислим среднее для каждой строчки матрицы, потом вычислим среднее строчки из каждого элемента строчки.

    \subsection{Спецификация}
    \begin{enumerate}
        \item Заголовок:  \mintinline{C}|void subtractMiddleOfLineFromEveryLine(double **matrix, int l, int m)|
        \item Назначение: из каждого элемента строчки матрицы \mintinline{C}|matrix| размером \mintinline{C}|l|$\times$\mintinline{C}|m| вычитает среднее строчки.
    \end{enumerate}

    \subsection{Блок-схема}
    \makenewfig{\documentclass{article}

\usepackage[utf8]{inputenc}
\usepackage[english,russian]{babel}
\usepackage{pgfplots}
\usepackage{verbatim}
\usetikzlibrary{positioning}
\usetikzlibrary{shapes.geometric}
\usetikzlibrary{shapes.misc}
\usetikzlibrary{calc}
\usetikzlibrary{chains}
\usetikzlibrary{matrix}
\usetikzlibrary{decorations.text}
\usepackage{fontspec}
\usetikzlibrary{backgrounds}

\begin{document}
    \begin{tikzpicture}[every node/.style={inner sep=0,outer sep=0}, background rectangle/.style={opacity=1.000000, fill=colorGSMKNAQNASCMEP}, show background rectangle]
        \makeatletter
        \newcommand{\verbatimfont}[1]{\def\verbatim@font{#1}}%
        \makeatother
        \newfontfamily\fontGSMKNAQJNFBQFR[Scale=0.991828, SizeFeatures={Size=10.000000}]{Consolas}
        \definecolor{colorGSMKNAQNASCMEP}{rgb}{1.000000,1.000000,1.000000}
        \definecolor{colorGSMKNAQOBPJFBMR}{rgb}{0.949020,0.949020,0.949020}
        \definecolor{colorGSMKNAQNOJKQL}{rgb}{0.000000,0.000000,0.000000}
        \draw[opacity=1.000000, rounded corners=0.384885 cm, fill=colorGSMKNAQOBPJFBMR, draw=none] (-0.006998,0.006998) -- (10.000000,0.006998) -- (10.000000,-0.762771) -- (-0.006998,-0.762771) -- cycle;
        \draw[opacity=1.000000, rounded corners=0.384885 cm, fill=none, line width=0.034990cm, colorGSMKNAQNOJKQL] (-0.006998,0.006998) -- (10.000000,0.006998) -- (10.000000,-0.762771) -- (-0.006998,-0.762771) -- cycle;
        \verbatimfont{\normalsize\fontGSMKNAQJNFBQFR}
        \node[opacity=1.000000, above right, colorGSMKNAQNOJKQL] at(0.371878, -0.534970) {\verb|subtractMiddleOfLineFromEveryLine(matrix, l, m) |};
        \draw[opacity=1.000000, fill=colorGSMKNAQOBPJFBMR, draw=none] (2.071379, -2.715185) -- (3.533940, -1.252624) -- (6.459062, -1.252624) -- (7.921624, -2.715185) -- (6.459062, -4.177747) -- (3.533940, -4.177747) --cycle;
        \draw[opacity=1.000000, fill=none, line width=0.034990cm, colorGSMKNAQNOJKQL] (2.071379, -2.715185) -- (3.533940, -1.252624) -- (6.459062, -1.252624) -- (7.921624, -2.715185) -- (6.459062, -4.177747) -- (3.533940, -4.177747) --cycle;
        \verbatimfont{\normalsize\fontGSMKNAQJNFBQFR}
        \node[opacity=1.000000, above right, colorGSMKNAQNOJKQL] at(3.743264, -2.788294) {\verb|i := 0, l - 1|};
        \draw[fill=colorGSMKNAQOBPJFBMR, draw=none, opacity=1.000000] (3.726382,-4.667600) -- (6.266620,-4.667600) -- (6.266620,-5.437369) -- (3.726382,-5.437369) -- cycle;
        \draw[opacity=1.000000, fill=none, line width=0.034990cm, colorGSMKNAQNOJKQL] (3.726382,-4.667600) -- (6.266620,-4.667600) -- (6.266620,-5.437369) -- (3.726382,-5.437369) -- cycle;
        \verbatimfont{\normalsize\fontGSMKNAQJNFBQFR}
        \node[opacity=1.000000, above right, colorGSMKNAQNOJKQL] at(3.937427, -5.209568) {\verb|middle := 0|};
        \draw[opacity=1.000000, fill=colorGSMKNAQOBPJFBMR, draw=none] (2.071379, -7.389783) -- (3.533940, -5.927222) -- (6.459062, -5.927222) -- (7.921624, -7.389783) -- (6.459062, -8.852344) -- (3.533940, -8.852344) --cycle;
        \draw[opacity=1.000000, fill=none, line width=0.034990cm, colorGSMKNAQNOJKQL] (2.071379, -7.389783) -- (3.533940, -5.927222) -- (6.459062, -5.927222) -- (7.921624, -7.389783) -- (6.459062, -8.852344) -- (3.533940, -8.852344) --cycle;
        \verbatimfont{\normalsize\fontGSMKNAQJNFBQFR}
        \node[opacity=1.000000, above right, colorGSMKNAQNOJKQL] at(3.743264, -7.462892) {\verb|j := 0, m - 1|};
        \draw[fill=colorGSMKNAQOBPJFBMR, draw=none, opacity=1.000000] (1.801959,-9.342197) -- (8.191043,-9.342197) -- (8.191043,-10.111966) -- (1.801959,-10.111966) -- cycle;
        \draw[opacity=1.000000, fill=none, line width=0.034990cm, colorGSMKNAQNOJKQL] (1.801959,-9.342197) -- (8.191043,-9.342197) -- (8.191043,-10.111966) -- (1.801959,-10.111966) -- cycle;
        \verbatimfont{\normalsize\fontGSMKNAQJNFBQFR}
        \node[opacity=1.000000, above right, colorGSMKNAQNOJKQL] at(2.013443, -9.884165) {\verb|middle := middle + matrix[i][j]|};
        \draw[opacity=1.000000, fill=none, line width=0.034990 cm, colorGSMKNAQNOJKQL] (4.996501,-8.852344) -- (4.996501,-9.342197);
        \verbatimfont{\normalsize\fontGSMKNAQJNFBQFR}
        \node[opacity=1.000000, above right, colorGSMKNAQNOJKQL] at(4.763693, -9.205369) {\verb|+|};
        \draw[opacity=1.000000, fill=none, line width=0.034990cm, colorGSMKNAQNOJKQL] (4.996501, -10.111966) -- (4.996501, -10.601819) -- (1.452064, -10.601819) -- (1.452064, -7.389783) -- (2.071379, -7.389783);
        \verbatimfont{\normalsize\fontGSMKNAQJNFBQFR}
        \node[opacity=1.000000, above right, colorGSMKNAQNOJKQL] at(8.329123, -7.179846) {\verb|-|};
        \draw[opacity=1.000000, fill=none, line width=0.034990cm, colorGSMKNAQNOJKQL] (7.921624, -7.389783) -- (8.540938, -7.389783) -- (8.540938, -10.951714) -- (4.996501, -10.951714);
        \draw[opacity=1.000000, fill=none, line width=0.034990 cm, colorGSMKNAQNOJKQL] (4.996501,-5.437369) -- (4.996501,-5.927222);
        \draw[fill=colorGSMKNAQOBPJFBMR, draw=none, opacity=1.000000] (2.862141,-11.441568) -- (7.130861,-11.441568) -- (7.130861,-12.211337) -- (2.862141,-12.211337) -- cycle;
        \draw[opacity=1.000000, fill=none, line width=0.034990cm, colorGSMKNAQNOJKQL] (2.862141,-11.441568) -- (7.130861,-11.441568) -- (7.130861,-12.211337) -- (2.862141,-12.211337) -- cycle;
        \verbatimfont{\normalsize\fontGSMKNAQJNFBQFR}
        \node[opacity=1.000000, above right, colorGSMKNAQNOJKQL] at(3.072517, -11.983535) {\verb|middle := middle / m|};
        \draw[opacity=1.000000, fill=none, line width=0.034990 cm, colorGSMKNAQNOJKQL] (4.996501,-10.951714) -- (4.996501,-11.441568);
        \draw[opacity=1.000000, fill=colorGSMKNAQOBPJFBMR, draw=none] (2.071379, -14.163751) -- (3.533940, -12.701190) -- (6.459062, -12.701190) -- (7.921624, -14.163751) -- (6.459062, -15.626312) -- (3.533940, -15.626312) --cycle;
        \draw[opacity=1.000000, fill=none, line width=0.034990cm, colorGSMKNAQNOJKQL] (2.071379, -14.163751) -- (3.533940, -12.701190) -- (6.459062, -12.701190) -- (7.921624, -14.163751) -- (6.459062, -15.626312) -- (3.533940, -15.626312) --cycle;
        \verbatimfont{\normalsize\fontGSMKNAQJNFBQFR}
        \node[opacity=1.000000, above right, colorGSMKNAQNOJKQL] at(3.743264, -14.236859) {\verb|j := 0, m - 1|};
        \draw[fill=colorGSMKNAQOBPJFBMR, draw=none, opacity=1.000000] (1.224633,-16.116165) -- (8.768369,-16.116165) -- (8.768369,-16.885934) -- (1.224633,-16.885934) -- cycle;
        \draw[opacity=1.000000, fill=none, line width=0.034990cm, colorGSMKNAQNOJKQL] (1.224633,-16.116165) -- (8.768369,-16.116165) -- (8.768369,-16.885934) -- (1.224633,-16.885934) -- cycle;
        \verbatimfont{\normalsize\fontGSMKNAQJNFBQFR}
        \node[opacity=1.000000, above right, colorGSMKNAQNOJKQL] at(1.430952, -16.658133) {\verb|matrix[i][j] := matrix[i][j] - middle|};
        \draw[opacity=1.000000, fill=none, line width=0.034990 cm, colorGSMKNAQNOJKQL] (4.996501,-15.626312) -- (4.996501,-16.116165);
        \verbatimfont{\normalsize\fontGSMKNAQJNFBQFR}
        \node[opacity=1.000000, above right, colorGSMKNAQNOJKQL] at(4.763693, -15.979337) {\verb|+|};
        \draw[opacity=1.000000, fill=none, line width=0.034990cm, colorGSMKNAQNOJKQL] (4.996501, -16.885934) -- (4.996501, -17.375787) -- (0.874738, -17.375787) -- (0.874738, -14.163751) -- (2.071379, -14.163751);
        \verbatimfont{\normalsize\fontGSMKNAQJNFBQFR}
        \node[opacity=1.000000, above right, colorGSMKNAQNOJKQL] at(8.906450, -13.953814) {\verb|-|};
        \draw[opacity=1.000000, fill=none, line width=0.034990cm, colorGSMKNAQNOJKQL] (7.921624, -14.163751) -- (9.118265, -14.163751) -- (9.118265, -17.725682) -- (4.996501, -17.725682);
        \draw[opacity=1.000000, fill=none, line width=0.034990 cm, colorGSMKNAQNOJKQL] (4.996501,-12.211337) -- (4.996501,-12.701190);
        \draw[opacity=1.000000, fill=none, line width=0.034990 cm, colorGSMKNAQNOJKQL] (4.996501,-4.177747) -- (4.996501,-4.667600);
        \verbatimfont{\normalsize\fontGSMKNAQJNFBQFR}
        \node[opacity=1.000000, above right, colorGSMKNAQNOJKQL] at(4.763693, -4.530771) {\verb|+|};
        \draw[opacity=1.000000, fill=none, line width=0.034990cm, colorGSMKNAQNOJKQL] (4.996501, -17.725682) -- (4.996501, -18.215535) -- (0.524843, -18.215535) -- (0.524843, -2.715185) -- (2.071379, -2.715185);
        \verbatimfont{\normalsize\fontGSMKNAQJNFBQFR}
        \node[opacity=1.000000, above right, colorGSMKNAQNOJKQL] at(9.256345, -2.505248) {\verb|-|};
        \draw[opacity=1.000000, fill=none, line width=0.034990cm, colorGSMKNAQNOJKQL] (7.921624, -2.715185) -- (9.468160, -2.715185) -- (9.468160, -18.565430) -- (4.996501, -18.565430);
        \draw[opacity=1.000000, fill=none, line width=0.034990 cm, colorGSMKNAQNOJKQL] (4.996501,-0.762771) -- (4.996501,-1.252624);
        \draw[opacity=1.000000, rounded corners=0.384885 cm, fill=colorGSMKNAQOBPJFBMR, draw=none] (3.264521,-19.055283) -- (6.728481,-19.055283) -- (6.728481,-19.825052) -- (3.264521,-19.825052) -- cycle;
        \draw[opacity=1.000000, rounded corners=0.384885 cm, fill=none, line width=0.034990cm, colorGSMKNAQNOJKQL] (3.264521,-19.055283) -- (6.728481,-19.055283) -- (6.728481,-19.825052) -- (3.264521,-19.825052) -- cycle;
        \verbatimfont{\normalsize\fontGSMKNAQJNFBQFR}
        \node[opacity=1.000000, above right, colorGSMKNAQNOJKQL] at(3.637356, -19.597251) {\verb|Выход (matrix)|};
        \draw[opacity=1.000000, fill=none, line width=0.034990 cm, colorGSMKNAQNOJKQL] (4.996501,-18.565430) -- (4.996501,-19.055283);
    \end{tikzpicture}
\end{document}}{Функция subtractMiddleOfLineFromEveryLine}{ris:task4}

    \subsection{Код программы}
    \begin{minted}{C}
#include "Tasks.h"

void subtractMiddleOfLineFromEveryLine(double **matrix, int l, int m) {
    for (int i = 0; i < l; i++) {
        // Для каждой строчки высчитаем среднее.
        // Сначала высчитаем сумму
        double middle = 0;

        for (int j = 0; j < m; j++)
            middle += matrix[i][j];

        // После этого можем поделить сумму на m - кол-во элементов - и получить среднее
        middle /= m;

        // Остаётся только вычесть её из каждого элемента в строке
        for (int j = 0; j < m; j++)
            matrix[i][j] -= middle;
    }
}
    \end{minted}

    \subsection{Тестовые данные}

    \labeltable{
        \begin{tabular}{|c|c|}
            \hline
            Тестовые данные & Ожидаемый результат \\
            \hline Пустая матрица
            & Пустая матрица \\
            \hline
            \begin{pmatrix}
                5 & 1 & 4 & 2 & 3
            \end{pmatrix} & \begin{pmatrix}
                                2 & -2 & 1 & -1& 0
            \end{pmatrix}
            \\
            \hline
            \begin{pmatrix}
                5 \\ 1 \\ 4 \\ 2 \\ 3
            \end{pmatrix} & \begin{pmatrix}
                                2 \\ -2 \\ 1 \\  -1\\ 0
            \end{pmatrix} \\
            \hline
            \begin{pmatrix}
                5 & 1 & 4 & 2 & 3 \\
                5 & 5 & 5 & 5 & 5 \\
                3 & 1 & 1 & 1 & 3 \\
                0 & 0 & 5 & 0 & 0 \\
                5 & 4 & 3 & 4 & 5 \\
            \end{pmatrix} & \begin{pmatrix}
                                2   & -2   & 1    & -1   & 0   \\
                                0   & 0    & 0    & 0    & 0   \\
                                1.2 & -0.8 & -0.8 & -0.8 & 1.2 \\
                                -1  & -1   & 4    & -1   & -1  \\
                                0.8 & -0.2 & -1.2 & -0.2 & 0.8 \\
            \end{pmatrix}
            \\
            \hline
        \end{tabular}
    }{Тестовые данные subtractMiddleOfLineFromEveryLine}{tab:task4-test-table}

    \subsection{Результат выполнения тестов}
    \makenewfig{
        \includegraphics[width=120mm]{/task4tests}}{Результат выполнения тестов функции subtractMiddleOfLineFromEveryLine}


    \section{Векторы и матрицы}

    \subsection{Задание варианта}
    По кругу располагаются $n$ человек. Ведущий считает по кругу, начиная с первого, и выводит («казнит») $m$-го человека. Круг смыкается, счёт возобновляется со следующего после «казнённого»; так продолжается, пока «в живых» останется только один человек. Найти номер оставшегося «в живых» человека, а также для заданного найти такое $m > 1$, при котором «в живых» останется первый.

    \subsection{Обоснование}
    Логично предположить что в круге из одного человека при любом шаге он будет победителем.
    Теперь рассмотрим задачу из двух людей.
    Вычеркнем из этого круга человека в зависимости от m - мы получим оставшегося победителя.
    Рассмотрим задачу с большим количество людей, например 5.
    Будем проходиться по кругу с шагом m, вычеркнем из этого круга m человека.
    У нас останется 4 человека, мы уже знаем решение этой задачи, поэтому на основе этих данных мы можем определить текущего победителя для n и m по формуле\\
    $res[n][m] = (m + res[n - 1][m] - 1) \% n + 1$\\
    на основе этой формулы можем составить матрицу, где элемент с индексом n и m содержит ответ на задачу.

    \subsection{Спецификация}

    \begin{enumerate}
        \item Заголовок:  \mintinline{C}|static JosephProblemMatrix createJosephProblemMatrix(int nn)|
        \item Назначение: возвращает структуру, содержащую указатель на матрицу шириной и длиной \mintinline{C}|nn| и её длину.
        Ячейка матрицы в i ряду и j столбце является ответом задача Иосифа Флавия, где (i + 1) - количество людей в круге,
        (j + 1) - шаг с которым казнят людей в кругу.
    \end{enumerate}
    \hline
    \begin{enumerate}
        \item Заголовок:  \mintinline{C}|void freeJosephProblemMatrix(JosephProblemMatrix matrix)|
        \item Назначение: освобождает память выделенную на объект \mintinline{C}|matrix|.
    \end{enumerate}
    \hline
    \begin{enumerate}
        \item Заголовок:  \mintinline{C}|int josephProblemFindSurvivor(int n, int m)|
        \item Назначение: возвращает номер выжившего в задаче Иосифа Флавия в кругу из \mintinline{C}|n| человек с шагом \mintinline{C}|m|.
    \end{enumerate}
    \hline
    \begin{enumerate}
        \item Заголовок:  \mintinline{C}|int josephProblemFindMForSurvivorOne(int n)|
        \item Назначение: возвращает шаг при котором в задаче Иосифа Флавия в кругу из \mintinline{C}|n| людей выжившим является человек с 1 номером.
    \end{enumerate}

    \subsection{Блок-схема}
    \makenewfig{\documentclass{article}

\usepackage[utf8]{inputenc}
\usepackage[english,russian]{babel}
\usepackage{pgfplots}
\usepackage{verbatim}
\usetikzlibrary{positioning}
\usetikzlibrary{shapes.geometric}
\usetikzlibrary{shapes.misc}
\usetikzlibrary{calc}
\usetikzlibrary{chains}
\usetikzlibrary{matrix}
\usetikzlibrary{decorations.text}
\usepackage{fontspec}
\usetikzlibrary{backgrounds}


\begin{document}
    \begin{tikzpicture}[every node/.style={inner sep=0,outer sep=0}, background rectangle/.style={opacity=1.000000, fill=colorJLDJAMBQBNNEPTTT}, show background rectangle]
        \makeatletter
        \newcommand{\verbatimfont}[1]{\def\verbatim@font{#1}}%
        \makeatother
        \newfontfamily\fontJLDJAMBQAHQNFMGN[Scale=0.881968, SizeFeatures={Size=10.000000}]{Consolas}
        \definecolor{colorJLDJAMBQBNNEPTTT}{rgb}{1.000000,1.000000,1.000000}
        \definecolor{colorJLDJAMBQGLJFIB}{rgb}{0.949020,0.949020,0.949020}
        \definecolor{colorJLDJAMBQBLOIIMS}{rgb}{0.000000,0.000000,0.000000}
        \draw[opacity=1.000000, rounded corners=0.342253 cm, fill=colorJLDJAMBQGLJFIB, draw=none] (2.171749,0.006223) -- (7.822029,0.006223) -- (7.822029,-0.678283) -- (2.171749,-0.678283) -- cycle;
        \draw[opacity=1.000000, rounded corners=0.342253 cm, fill=none, line width=0.031114cm, colorJLDJAMBQBLOIIMS] (2.171749,0.006223) -- (7.822029,0.006223) -- (7.822029,-0.678283) -- (2.171749,-0.678283) -- cycle;
        \verbatimfont{\normalsize\fontJLDJAMBQAHQNFMGN}
        \node[opacity=1.000000, above right, colorJLDJAMBQBLOIIMS] at(2.508258, -0.479490) {\verb|createJosephProblemMatrix(nn)|};
        \draw[fill=colorJLDJAMBQGLJFIB, draw=none, opacity=1.000000] (1.798382,-1.113877) -- (8.195395,-1.113877) -- (8.195395,-1.798382) -- (1.798382,-1.798382) -- cycle;
        \draw[opacity=1.000000, fill=none, line width=0.031114cm, colorJLDJAMBQBLOIIMS] (1.798382,-1.113877) -- (8.195395,-1.113877) -- (8.195395,-1.798382) -- (1.798382,-1.798382) -- cycle;
        \draw[opacity=1.000000, fill=none, line width=0.031114cm, colorJLDJAMBQBLOIIMS] (1.901058,-1.113877) -- (8.092719,-1.113877) -- (8.092719,-1.798382) -- (1.901058,-1.798382) -- cycle;
        \verbatimfont{\normalsize\fontJLDJAMBQAHQNFMGN}
        \node[opacity=1.000000, above right, colorJLDJAMBQBLOIIMS] at(2.084661, -1.599589) {\verb|res := malloc(nn * sizeof(int *));|};
        \draw[opacity=1.000000, fill=none, line width=0.031114 cm, colorJLDJAMBQBLOIIMS] (4.996889,-0.678283) -- (4.996889,-1.113877);
        \draw[opacity=1.000000, fill=colorJLDJAMBQGLJFIB, draw=none] (2.227754, -3.618544) -- (3.612321, -2.233976) -- (6.381456, -2.233976) -- (7.766024, -3.618544) -- (6.381456, -5.003111) -- (3.612321, -5.003111) --cycle;
        \draw[opacity=1.000000, fill=none, line width=0.031114cm, colorJLDJAMBQBLOIIMS] (2.227754, -3.618544) -- (3.612321, -2.233976) -- (6.381456, -2.233976) -- (7.766024, -3.618544) -- (6.381456, -5.003111) -- (3.612321, -5.003111) --cycle;
        \verbatimfont{\normalsize\fontJLDJAMBQAHQNFMGN}
        \node[opacity=1.000000, above right, colorJLDJAMBQBLOIIMS] at(3.796698, -3.687330) {\verb|i := 0, nn - 1|};
        \draw[fill=colorJLDJAMBQGLJFIB, draw=none, opacity=1.000000] (1.711263,-5.438706) -- (8.282514,-5.438706) -- (8.282514,-6.123211) -- (1.711263,-6.123211) -- cycle;
        \draw[opacity=1.000000, fill=none, line width=0.031114cm, colorJLDJAMBQBLOIIMS] (1.711263,-5.438706) -- (8.282514,-5.438706) -- (8.282514,-6.123211) -- (1.711263,-6.123211) -- cycle;
        \draw[opacity=1.000000, fill=none, line width=0.031114cm, colorJLDJAMBQBLOIIMS] (1.813939,-5.438706) -- (8.179838,-5.438706) -- (8.179838,-6.123211) -- (1.813939,-6.123211) -- cycle;
        \verbatimfont{\normalsize\fontJLDJAMBQAHQNFMGN}
        \node[opacity=1.000000, above right, colorJLDJAMBQBLOIIMS] at(1.996412, -5.924418) {\verb|res[i] := malloc(sizeof(int) * nn);|};
        \draw[opacity=1.000000, fill=none, line width=0.031114 cm, colorJLDJAMBQBLOIIMS] (4.996889,-5.003111) -- (4.996889,-5.438706);
        \verbatimfont{\normalsize\fontJLDJAMBQAHQNFMGN}
        \node[opacity=1.000000, above right, colorJLDJAMBQBLOIIMS] at(4.801722, -5.320809) {\verb|+|};
        \draw[opacity=1.000000, fill=none, line width=0.031114cm, colorJLDJAMBQBLOIIMS] (4.996889, -6.123211) -- (4.996889, -6.558805) -- (1.400124, -6.558805) -- (1.400124, -3.618544) -- (2.227754, -3.618544);
        \verbatimfont{\normalsize\fontJLDJAMBQAHQNFMGN}
        \node[opacity=1.000000, above right, colorJLDJAMBQBLOIIMS] at(8.417154, -3.431861) {\verb|-|};
        \draw[opacity=1.000000, fill=none, line width=0.031114cm, colorJLDJAMBQBLOIIMS] (7.766024, -3.618544) -- (8.593653, -3.618544) -- (8.593653, -6.869944) -- (4.996889, -6.869944);
        \draw[opacity=1.000000, fill=none, line width=0.031114 cm, colorJLDJAMBQBLOIIMS] (4.996889,-1.798382) -- (4.996889,-2.233976);
        \draw[opacity=1.000000, fill=colorJLDJAMBQGLJFIB, draw=none] (2.227754, -8.690106) -- (3.612321, -7.305538) -- (6.381456, -7.305538) -- (7.766024, -8.690106) -- (6.381456, -10.074673) -- (3.612321, -10.074673) --cycle;
        \draw[opacity=1.000000, fill=none, line width=0.031114cm, colorJLDJAMBQBLOIIMS] (2.227754, -8.690106) -- (3.612321, -7.305538) -- (6.381456, -7.305538) -- (7.766024, -8.690106) -- (6.381456, -10.074673) -- (3.612321, -10.074673) --cycle;
        \verbatimfont{\normalsize\fontJLDJAMBQAHQNFMGN}
        \node[opacity=1.000000, above right, colorJLDJAMBQBLOIIMS] at(3.796698, -8.758892) {\verb|i := 0, nn - 1|};
        \draw[fill=colorJLDJAMBQGLJFIB, draw=none, opacity=1.000000] (3.612321,-10.510268) -- (6.381456,-10.510268) -- (6.381456,-11.194773) -- (3.612321,-11.194773) -- cycle;
        \draw[opacity=1.000000, fill=none, line width=0.031114cm, colorJLDJAMBQBLOIIMS] (3.612321,-10.510268) -- (6.381456,-10.510268) -- (6.381456,-11.194773) -- (3.612321,-11.194773) -- cycle;
        \verbatimfont{\normalsize\fontJLDJAMBQAHQNFMGN}
        \node[opacity=1.000000, above right, colorJLDJAMBQBLOIIMS] at(3.796698, -10.995980) {\verb|res[0][i] := 1|};
        \draw[opacity=1.000000, fill=none, line width=0.031114 cm, colorJLDJAMBQBLOIIMS] (4.996889,-10.074673) -- (4.996889,-10.510268);
        \verbatimfont{\normalsize\fontJLDJAMBQAHQNFMGN}
        \node[opacity=1.000000, above right, colorJLDJAMBQBLOIIMS] at(4.801722, -10.392371) {\verb|+|};
        \draw[opacity=1.000000, fill=none, line width=0.031114cm, colorJLDJAMBQBLOIIMS] (4.996889, -11.194773) -- (4.996889, -11.630367) -- (1.916615, -11.630367) -- (1.916615, -8.690106) -- (2.227754, -8.690106);
        \verbatimfont{\normalsize\fontJLDJAMBQAHQNFMGN}
        \node[opacity=1.000000, above right, colorJLDJAMBQBLOIIMS] at(7.900664, -8.503423) {\verb|-|};
        \draw[opacity=1.000000, fill=none, line width=0.031114cm, colorJLDJAMBQBLOIIMS] (7.766024, -8.690106) -- (8.077162, -8.690106) -- (8.077162, -11.941506) -- (4.996889, -11.941506);
        \draw[opacity=1.000000, fill=none, line width=0.031114 cm, colorJLDJAMBQBLOIIMS] (4.996889,-6.869944) -- (4.996889,-7.305538);
        \draw[opacity=1.000000, fill=colorJLDJAMBQGLJFIB, draw=none] (2.912259, -13.419415) -- (3.954574, -12.377100) -- (6.039203, -12.377100) -- (7.081518, -13.419415) -- (6.039203, -14.461730) -- (3.954574, -14.461730) --cycle;
        \draw[opacity=1.000000, fill=none, line width=0.031114cm, colorJLDJAMBQBLOIIMS] (2.912259, -13.419415) -- (3.954574, -12.377100) -- (6.039203, -12.377100) -- (7.081518, -13.419415) -- (6.039203, -14.461730) -- (3.954574, -14.461730) --cycle;
        \verbatimfont{\normalsize\fontJLDJAMBQAHQNFMGN}
        \node[opacity=1.000000, above right, colorJLDJAMBQBLOIIMS] at(4.132045, -13.488202) {\verb|m := 1, nn|};
        \draw[opacity=1.000000, fill=colorJLDJAMBQGLJFIB, draw=none] (2.227754, -16.281892) -- (3.612321, -14.897324) -- (6.381456, -14.897324) -- (7.766024, -16.281892) -- (6.381456, -17.666459) -- (3.612321, -17.666459) --cycle;
        \draw[opacity=1.000000, fill=none, line width=0.031114cm, colorJLDJAMBQBLOIIMS] (2.227754, -16.281892) -- (3.612321, -14.897324) -- (6.381456, -14.897324) -- (7.766024, -16.281892) -- (6.381456, -17.666459) -- (3.612321, -17.666459) --cycle;
        \verbatimfont{\normalsize\fontJLDJAMBQAHQNFMGN}
        \node[opacity=1.000000, above right, colorJLDJAMBQBLOIIMS] at(3.796698, -16.350678) {\verb|n := 1, nn - 1|};
        \draw[fill=colorJLDJAMBQGLJFIB, draw=none, opacity=1.000000] (0.616055,-18.102054) -- (9.377722,-18.102054) -- (9.377722,-19.116366) -- (0.616055,-19.116366) -- cycle;
        \draw[opacity=1.000000, fill=none, line width=0.031114cm, colorJLDJAMBQBLOIIMS] (0.616055,-18.102054) -- (9.377722,-18.102054) -- (9.377722,-19.116366) -- (0.616055,-19.116366) -- cycle;
        \verbatimfont{\normalsize\fontJLDJAMBQAHQNFMGN}
        \node[opacity=1.000000, above right, colorJLDJAMBQBLOIIMS] at(0.796221, -18.587766) {\verb|res[n][m - 1] := (m + res[n - 1][m - 1] - 1) mod |};
        \verbatimfont{\normalsize\fontJLDJAMBQAHQNFMGN}
        \node[opacity=1.000000, above right, colorJLDJAMBQBLOIIMS] at(3.955547, -18.924132) {\verb|(n + 1) + 1;|};
        \draw[opacity=1.000000, fill=none, line width=0.031114 cm, colorJLDJAMBQBLOIIMS] (4.996889,-17.666459) -- (4.996889,-18.102054);
        \verbatimfont{\normalsize\fontJLDJAMBQAHQNFMGN}
        \node[opacity=1.000000, above right, colorJLDJAMBQBLOIIMS] at(4.801722, -17.984157) {\verb|+|};
        \draw[opacity=1.000000, fill=none, line width=0.031114cm, colorJLDJAMBQBLOIIMS] (4.996889, -19.116366) -- (4.996889, -19.551960) -- (0.304916, -19.551960) -- (0.304916, -16.281892) -- (2.227754, -16.281892);
        \verbatimfont{\normalsize\fontJLDJAMBQAHQNFMGN}
        \node[opacity=1.000000, above right, colorJLDJAMBQBLOIIMS] at(9.512363, -16.095208) {\verb|-|};
        \draw[opacity=1.000000, fill=none, line width=0.031114cm, colorJLDJAMBQBLOIIMS] (7.766024, -16.281892) -- (9.688861, -16.281892) -- (9.688861, -19.863099) -- (4.996889, -19.863099);
        \draw[opacity=1.000000, fill=none, line width=0.031114 cm, colorJLDJAMBQBLOIIMS] (4.996889,-14.461730) -- (4.996889,-14.897324);
        \verbatimfont{\normalsize\fontJLDJAMBQAHQNFMGN}
        \node[opacity=1.000000, above right, colorJLDJAMBQBLOIIMS] at(4.801722, -14.779428) {\verb|+|};
        \draw[opacity=1.000000, fill=none, line width=0.031114cm, colorJLDJAMBQBLOIIMS] (4.996889, -19.863099) -- (4.996889, -20.298693) -- (-0.006223, -20.298693) -- (-0.006223, -13.419415) -- (2.912259, -13.419415);
        \verbatimfont{\normalsize\fontJLDJAMBQAHQNFMGN}
        \node[opacity=1.000000, above right, colorJLDJAMBQBLOIIMS] at(9.823501, -13.232732) {\verb|-|};
        \draw[opacity=1.000000, fill=none, line width=0.031114cm, colorJLDJAMBQBLOIIMS] (7.081518, -13.419415) -- (10.000000, -13.419415) -- (10.000000, -20.609832) -- (4.996889, -20.609832);
        \draw[opacity=1.000000, fill=none, line width=0.031114 cm, colorJLDJAMBQBLOIIMS] (4.996889,-11.941506) -- (4.996889,-12.377100);
        \draw[opacity=1.000000, fill=colorJLDJAMBQGLJFIB, draw=none] (4.069695, -21.509023) -- (4.996889, -21.972620) -- (5.924082, -21.509023) -- (4.996889, -21.045426) --cycle;
        \draw[opacity=1.000000, fill=none, line width=0.031114cm, colorJLDJAMBQBLOIIMS] (4.069695, -21.509023) -- (4.996889, -21.972620) -- (5.924082, -21.509023) -- (4.996889, -21.045426) --cycle;
        \verbatimfont{\normalsize\fontJLDJAMBQAHQNFMGN}
        \node[opacity=1.000000, above right, colorJLDJAMBQBLOIIMS] at(4.555642, -21.615146) {\verb|n ≤ 0|};
        \draw[opacity=1.000000, fill=none, line width=0.031114cm, colorJLDJAMBQBLOIIMS] (4.069695, -21.509023) -- (2.358432, -21.509023) -- (2.358432, -22.408214);
        \draw[fill=colorJLDJAMBQGLJFIB, draw=none, opacity=1.000000] (0.031114,-22.408214) -- (4.685750,-22.408214) -- (4.685750,-23.092719) -- (0.031114,-23.092719) -- cycle;
        \draw[opacity=1.000000, fill=none, line width=0.031114cm, colorJLDJAMBQBLOIIMS] (0.031114,-22.408214) -- (4.685750,-22.408214) -- (4.685750,-23.092719) -- (0.031114,-23.092719) -- cycle;
        \verbatimfont{\normalsize\fontJLDJAMBQAHQNFMGN}
        \node[opacity=1.000000, above right, colorJLDJAMBQBLOIIMS] at(0.205148, -22.893927) {\verb|resultMatrix := {NULL, 0}|};
        \verbatimfont{\normalsize\fontJLDJAMBQAHQNFMGN}
        \node[opacity=1.000000, above right, colorJLDJAMBQBLOIIMS] at(2.358432, -21.322340) {\verb|+|};
        \draw[opacity=1.000000, fill=none, line width=0.031114cm, colorJLDJAMBQBLOIIMS] (5.924082, -21.509023) -- (7.635345, -21.509023) -- (7.635345, -22.408214);
        \verbatimfont{\normalsize\fontJLDJAMBQAHQNFMGN}
        \node[opacity=1.000000, above right, colorJLDJAMBQBLOIIMS] at(7.458847, -21.322340) {\verb|-|};
        \draw[fill=colorJLDJAMBQGLJFIB, draw=none, opacity=1.000000] (5.308027,-22.408214) -- (9.962663,-22.408214) -- (9.962663,-23.092719) -- (5.308027,-23.092719) -- cycle;
        \draw[opacity=1.000000, fill=none, line width=0.031114cm, colorJLDJAMBQBLOIIMS] (5.308027,-22.408214) -- (9.962663,-22.408214) -- (9.962663,-23.092719) -- (5.308027,-23.092719) -- cycle;
        \verbatimfont{\normalsize\fontJLDJAMBQAHQNFMGN}
        \node[opacity=1.000000, above right, colorJLDJAMBQBLOIIMS] at(5.482062, -22.893927) {\verb|resultMatrix := {res, nn}|};
        \draw[opacity=1.000000, fill=none, line width=0.031114cm, colorJLDJAMBQBLOIIMS] (7.635345, -23.092719) -- (7.635345, -23.528314) -- (4.996889, -23.528314);
        \draw[opacity=1.000000, fill=none, line width=0.031114cm, colorJLDJAMBQBLOIIMS] (2.358432, -23.092719) -- (2.358432, -23.528314) -- (4.996889, -23.528314);
        \draw[opacity=1.000000, fill=none, line width=0.031114 cm, colorJLDJAMBQBLOIIMS] (4.996889,-20.609832) -- (4.996889,-21.045426);
        \draw[opacity=1.000000, rounded corners=0.342253 cm, fill=colorJLDJAMBQGLJFIB, draw=none] (2.943373,-23.963908) -- (7.050404,-23.963908) -- (7.050404,-24.648413) -- (2.943373,-24.648413) -- cycle;
        \draw[opacity=1.000000, rounded corners=0.342253 cm, fill=none, line width=0.031114cm, colorJLDJAMBQBLOIIMS] (2.943373,-23.963908) -- (7.050404,-23.963908) -- (7.050404,-24.648413) -- (2.943373,-24.648413) -- cycle;
        \verbatimfont{\normalsize\fontJLDJAMBQAHQNFMGN}
        \node[opacity=1.000000, above right, colorJLDJAMBQBLOIIMS] at(3.284852, -24.449620) {\verb|Вернуть resultMatrix|};
        \draw[opacity=1.000000, fill=none, line width=0.031114 cm, colorJLDJAMBQBLOIIMS] (4.996889,-23.528314) -- (4.996889,-23.963908);
    \end{tikzpicture}
\end{document}}{Функция createJosephProblemMatrix}{ris:task5}\\
    \makenewfig{\documentclass{article}

\usepackage[utf8]{inputenc}
\usepackage[english,russian]{babel}
\usepackage{pgfplots}
\usepackage{verbatim}
\usetikzlibrary{positioning}
\usetikzlibrary{shapes.geometric}
\usetikzlibrary{shapes.misc}
\usetikzlibrary{calc}
\usetikzlibrary{chains}
\usetikzlibrary{matrix}
\usetikzlibrary{decorations.text}
\usepackage{fontspec}
\usetikzlibrary{backgrounds}

\begin{document}
    \begin{tikzpicture}[every node/.style={inner sep=0,outer sep=0}, background rectangle/.style={opacity=1.000000, fill=colorPANHKENGOSTOEFQF}, show background rectangle]
        \makeatletter
        \newcommand{\verbatimfont}[1]{\def\verbatim@font{#1}}%
        \makeatother
        \newfontfamily\fontPANHKENGKLGLCLRM[Scale=1.074543, SizeFeatures={Size=10.000000}]{Consolas}
        \definecolor{colorPANHKENGOSTOEFQF}{rgb}{1.000000,1.000000,1.000000}
        \definecolor{colorPANHKENGKQKLEN}{rgb}{0.949020,0.949020,0.949020}
        \definecolor{colorPANHKENGAHPERHGS}{rgb}{0.000000,0.000000,0.000000}
        \draw[opacity=1.000000, rounded corners=0.416983 cm, fill=colorPANHKENGKQKLEN, draw=none] (1.345716,0.007582) -- (8.646702,0.007582) -- (8.646702,-0.826384) -- (1.345716,-0.826384) -- cycle;
        \draw[opacity=1.000000, rounded corners=0.416983 cm, fill=none, line width=0.037908cm, colorPANHKENGAHPERHGS] (1.345716,0.007582) -- (8.646702,0.007582) -- (8.646702,-0.826384) -- (1.345716,-0.826384) -- cycle;
        \verbatimfont{\normalsize\fontPANHKENGKLGLCLRM}
        \node[opacity=1.000000, above right, colorPANHKENGAHPERHGS] at(1.748193, -0.585469) {\verb|freeJosephProblemMatrix(matrix)|};
        \draw[opacity=1.000000, fill=colorPANHKENGKQKLEN, draw=none] (0.371494, -3.669447) -- (2.683851, -1.357089) -- (7.308567, -1.357089) -- (9.620925, -3.669447) -- (7.308567, -5.981804) -- (2.683851, -5.981804) --cycle;
        \draw[opacity=1.000000, fill=none, line width=0.037908cm, colorPANHKENGAHPERHGS] (0.371494, -3.669447) -- (2.683851, -1.357089) -- (7.308567, -1.357089) -- (9.620925, -3.669447) -- (7.308567, -5.981804) -- (2.683851, -5.981804) --cycle;
        \verbatimfont{\normalsize\fontPANHKENGKLGLCLRM}
        \node[opacity=1.000000, above right, colorPANHKENGAHPERHGS] at(2.895590, -3.754536) {\verb|i := 0, matrix.n - 1|};
        \draw[fill=colorPANHKENGKQKLEN, draw=none, opacity=1.000000] (2.350265,-6.512509) -- (7.642153,-6.512509) -- (7.642153,-7.346475) -- (2.350265,-7.346475) -- cycle;
        \draw[opacity=1.000000, fill=none, line width=0.037908cm, colorPANHKENGAHPERHGS] (2.350265,-6.512509) -- (7.642153,-6.512509) -- (7.642153,-7.346475) -- (2.350265,-7.346475) -- cycle;
        \draw[opacity=1.000000, fill=none, line width=0.037908cm, colorPANHKENGAHPERHGS] (2.475360,-6.512509) -- (7.517058,-6.512509) -- (7.517058,-7.346475) -- (2.475360,-7.346475) -- cycle;
        \verbatimfont{\normalsize\fontPANHKENGKLGLCLRM}
        \node[opacity=1.000000, above right, colorPANHKENGAHPERHGS] at(2.701415, -7.105560) {\verb|free(matrix.matrix[i])|};
        \draw[opacity=1.000000, fill=none, line width=0.037908 cm, colorPANHKENGAHPERHGS] (4.996209,-5.981804) -- (4.996209,-6.512509);
        \verbatimfont{\normalsize\fontPANHKENGKLGLCLRM}
        \node[opacity=1.000000, above right, colorPANHKENGAHPERHGS] at(4.761638, -6.370154) {\verb|+|};
        \draw[opacity=1.000000, fill=none, line width=0.037908cm, colorPANHKENGAHPERHGS] (4.996209, -7.346475) -- (4.996209, -7.877180) -- (-0.007582, -7.877180) -- (-0.007582, -3.669447) -- (0.371494, -3.669447);
        \verbatimfont{\normalsize\fontPANHKENGKLGLCLRM}
        \node[opacity=1.000000, above right, colorPANHKENGAHPERHGS] at(9.788173, -3.442002) {\verb|-|};
        \draw[opacity=1.000000, fill=none, line width=0.037908cm, colorPANHKENGAHPERHGS] (9.620925, -3.669447) -- (10.000000, -3.669447) -- (10.000000, -8.256255) -- (4.996209, -8.256255);
        \draw[opacity=1.000000, fill=none, line width=0.037908 cm, colorPANHKENGAHPERHGS] (4.996209,-0.826384) -- (4.996209,-1.357089);
        \draw[fill=colorPANHKENGKQKLEN, draw=none, opacity=1.000000] (2.661107,-8.786960) -- (7.331312,-8.786960) -- (7.331312,-9.620925) -- (2.661107,-9.620925) -- cycle;
        \draw[opacity=1.000000, fill=none, line width=0.037908cm, colorPANHKENGAHPERHGS] (2.661107,-8.786960) -- (7.331312,-8.786960) -- (7.331312,-9.620925) -- (2.661107,-9.620925) -- cycle;
        \draw[opacity=1.000000, fill=none, line width=0.037908cm, colorPANHKENGAHPERHGS] (2.786202,-8.786960) -- (7.206217,-8.786960) -- (7.206217,-9.620925) -- (2.786202,-9.620925) -- cycle;
        \verbatimfont{\normalsize\fontPANHKENGKLGLCLRM}
        \node[opacity=1.000000, above right, colorPANHKENGAHPERHGS] at(3.001504, -9.380010) {\verb|free(matrix.matrix)|};
        \draw[opacity=1.000000, fill=none, line width=0.037908 cm, colorPANHKENGAHPERHGS] (4.996209,-8.256255) -- (4.996209,-8.786960);
        \draw[opacity=1.000000, rounded corners=0.416983 cm, fill=colorPANHKENGKQKLEN, draw=none] (3.328279,-10.151630) -- (6.664139,-10.151630) -- (6.664139,-10.985595) -- (3.328279,-10.985595) -- cycle;
        \draw[opacity=1.000000, rounded corners=0.416983 cm, fill=none, line width=0.037908cm, colorPANHKENGAHPERHGS] (3.328279,-10.151630) -- (6.664139,-10.151630) -- (6.664139,-10.985595) -- (3.328279,-10.985595) -- cycle;
        \verbatimfont{\normalsize\fontPANHKENGKLGLCLRM}
        \node[opacity=1.000000, above right, colorPANHKENGAHPERHGS] at(4.466641, -10.744680) {\verb|Выход|};
        \draw[opacity=1.000000, fill=none, line width=0.037908 cm, colorPANHKENGAHPERHGS] (4.996209,-9.620925) -- (4.996209,-10.151630);
    \end{tikzpicture}
\end{document}}{Функция freeJosephProblemMatrix}{ris:task5_free}\\
    \makenewfig{\documentclass{article}

\usepackage[utf8]{inputenc}
\usepackage[english,russian]{babel}
\usepackage{pgfplots}
\usepackage{verbatim}
\usetikzlibrary{positioning}
\usetikzlibrary{shapes.geometric}
\usetikzlibrary{shapes.misc}
\usetikzlibrary{calc}
\usetikzlibrary{chains}
\usetikzlibrary{matrix}
\usetikzlibrary{decorations.text}
\usepackage{fontspec}
\usetikzlibrary{backgrounds}

\begin{document}
    \begin{tikzpicture}[every node/.style={inner sep=0,outer sep=0}, background rectangle/.style={opacity=1.000000, fill=colorHIDQCFBOKNFQGLB}, show background rectangle]
        \makeatletter
        \newcommand{\verbatimfont}[1]{\def\verbatim@font{#1}}%
        \makeatother
        \newfontfamily\fontHIDQCFBORAKLTQ[Scale=0.870859, SizeFeatures={Size=10.000000}]{Consolas}
        \definecolor{colorHIDQCFBOKNFQGLB}{rgb}{1.000000,1.000000,1.000000}
        \definecolor{colorHIDQCFBOLSJKMHK}{rgb}{0.949020,0.949020,0.949020}
        \definecolor{colorHIDQCFBOJQEQCJSD}{rgb}{0.000000,0.000000,0.000000}
        \draw[opacity=1.000000, rounded corners=0.337942 cm, fill=colorHIDQCFBOLSJKMHK, draw=none] (3.628264,0.006144) -- (9.545315,0.006144) -- (9.545315,-0.669739) -- (3.628264,-0.669739) -- cycle;
        \draw[opacity=1.000000, rounded corners=0.337942 cm, fill=none, line width=0.030722cm, colorHIDQCFBOJQEQCJSD] (3.628264,0.006144) -- (9.545315,0.006144) -- (9.545315,-0.669739) -- (3.628264,-0.669739) -- cycle;
        \verbatimfont{\normalsize\fontHIDQCFBORAKLTQ}
        \node[opacity=1.000000, above right, colorHIDQCFBOJQEQCJSD] at(3.956980, -0.477449) {\verb|josephProblemFindSurvivor(n, m)|};
        \draw[fill=colorHIDQCFBOLSJKMHK, draw=none, opacity=1.000000] (3.173579,-1.099846) -- (10.000000,-1.099846) -- (10.000000,-1.775730) -- (3.173579,-1.775730) -- cycle;
        \draw[opacity=1.000000, fill=none, line width=0.030722cm, colorHIDQCFBOJQEQCJSD] (3.173579,-1.099846) -- (10.000000,-1.099846) -- (10.000000,-1.775730) -- (3.173579,-1.775730) -- cycle;
        \draw[opacity=1.000000, fill=none, line width=0.030722cm, colorHIDQCFBOJQEQCJSD] (3.274962,-1.099846) -- (9.898618,-1.099846) -- (9.898618,-1.775730) -- (3.274962,-1.775730) -- cycle;
        \verbatimfont{\normalsize\fontHIDQCFBORAKLTQ}
        \node[opacity=1.000000, above right, colorHIDQCFBOJQEQCJSD] at(3.445138, -1.583440) {\verb|matrix = createJosephProblemMatrix(n)|};
        \draw[opacity=1.000000, fill=none, line width=0.030722 cm, colorHIDQCFBOJQEQCJSD] (6.586790,-0.669739) -- (6.586790,-1.099846);
        \draw[fill=colorHIDQCFBOLSJKMHK, draw=none, opacity=1.000000] (5.640553,-2.205837) -- (7.533026,-2.205837) -- (7.533026,-2.881720) -- (5.640553,-2.881720) -- cycle;
        \draw[opacity=1.000000, fill=none, line width=0.030722cm, colorHIDQCFBOJQEQCJSD] (5.640553,-2.205837) -- (7.533026,-2.205837) -- (7.533026,-2.881720) -- (5.640553,-2.881720) -- cycle;
        \verbatimfont{\normalsize\fontHIDQCFBORAKLTQ}
        \node[opacity=1.000000, above right, colorHIDQCFBOJQEQCJSD] at(5.810202, -2.689431) {\verb|res := -1|};
        \draw[opacity=1.000000, fill=none, line width=0.030722 cm, colorHIDQCFBOJQEQCJSD] (6.586790,-1.775730) -- (6.586790,-2.205837);
        \draw[opacity=1.000000, fill=colorHIDQCFBOLSJKMHK, draw=none] (4.405530, -4.402458) -- (6.586790, -5.493088) -- (8.768049, -4.402458) -- (6.586790, -3.311828) --cycle;
        \draw[opacity=1.000000, fill=none, line width=0.030722cm, colorHIDQCFBOJQEQCJSD] (4.405530, -4.402458) -- (6.586790, -5.493088) -- (8.768049, -4.402458) -- (6.586790, -3.311828) --cycle;
        \verbatimfont{\normalsize\fontHIDQCFBORAKLTQ}
        \node[opacity=1.000000, above right, colorHIDQCFBOJQEQCJSD] at(4.892416, -4.511243) {\verb|matrix.matrix ≠ NULL|};
        \draw[opacity=1.000000, fill=none, line width=0.030722cm, colorHIDQCFBOJQEQCJSD] (4.405530, -4.402458) -- (3.136713, -4.402458) -- (3.136713, -5.923195);
        \draw[fill=colorHIDQCFBOLSJKMHK, draw=none, opacity=1.000000] (1.007680,-5.923195) -- (5.265745,-5.923195) -- (5.265745,-6.599078) -- (1.007680,-6.599078) -- cycle;
        \draw[opacity=1.000000, fill=none, line width=0.030722cm, colorHIDQCFBOJQEQCJSD] (1.007680,-5.923195) -- (5.265745,-5.923195) -- (5.265745,-6.599078) -- (1.007680,-6.599078) -- cycle;
        \verbatimfont{\normalsize\fontHIDQCFBORAKLTQ}
        \node[opacity=1.000000, above right, colorHIDQCFBOJQEQCJSD] at(1.177593, -6.406789) {\verb|indexM := (m - 1) mod n|};
        \draw[fill=colorHIDQCFBOLSJKMHK, draw=none, opacity=1.000000] (-0.006144,-7.029186) -- (6.279570,-7.029186) -- (6.279570,-7.705069) -- (-0.006144,-7.705069) -- cycle;
        \draw[opacity=1.000000, fill=none, line width=0.030722cm, colorHIDQCFBOJQEQCJSD] (-0.006144,-7.029186) -- (6.279570,-7.029186) -- (6.279570,-7.705069) -- (-0.006144,-7.705069) -- cycle;
        \verbatimfont{\normalsize\fontHIDQCFBORAKLTQ}
        \node[opacity=1.000000, above right, colorHIDQCFBOJQEQCJSD] at(0.171559, -7.512780) {\verb|res := matrix.matrix[n - 1][indexM]|};
        \draw[opacity=1.000000, fill=none, line width=0.030722 cm, colorHIDQCFBOJQEQCJSD] (3.136713,-6.599078) -- (3.136713,-7.029186);
        \verbatimfont{\normalsize\fontHIDQCFBORAKLTQ}
        \node[opacity=1.000000, above right, colorHIDQCFBOJQEQCJSD] at(3.136713, -4.218126) {\verb|+|};
        \draw[opacity=1.000000, fill=none, line width=0.030722cm, colorHIDQCFBOJQEQCJSD] (8.768049, -4.402458) -- (9.075269, -4.402458) -- (9.075269, -5.923195);
        \verbatimfont{\normalsize\fontHIDQCFBORAKLTQ}
        \node[opacity=1.000000, above right, colorHIDQCFBOJQEQCJSD] at(8.898772, -4.218126) {\verb|-|};
        \draw[opacity=1.000000, fill=none, line width=0.030722cm, colorHIDQCFBOJQEQCJSD] (9.075269, -5.923195) -- (9.075269, -8.135177) -- (6.586790, -8.135177);
        \draw[opacity=1.000000, fill=none, line width=0.030722cm, colorHIDQCFBOJQEQCJSD] (3.136713, -7.705069) -- (3.136713, -8.135177) -- (6.586790, -8.135177);
        \draw[opacity=1.000000, fill=none, line width=0.030722 cm, colorHIDQCFBOJQEQCJSD] (6.586790,-2.881720) -- (6.586790,-3.311828);
        \draw[fill=colorHIDQCFBOLSJKMHK, draw=none, opacity=1.000000] (3.680492,-8.565284) -- (9.493088,-8.565284) -- (9.493088,-9.241167) -- (3.680492,-9.241167) -- cycle;
        \draw[opacity=1.000000, fill=none, line width=0.030722cm, colorHIDQCFBOJQEQCJSD] (3.680492,-8.565284) -- (9.493088,-8.565284) -- (9.493088,-9.241167) -- (3.680492,-9.241167) -- cycle;
        \draw[opacity=1.000000, fill=none, line width=0.030722cm, colorHIDQCFBOJQEQCJSD] (3.781874,-8.565284) -- (9.391705,-8.565284) -- (9.391705,-9.241167) -- (3.781874,-9.241167) -- cycle;
        \verbatimfont{\normalsize\fontHIDQCFBORAKLTQ}
        \node[opacity=1.000000, above right, colorHIDQCFBOJQEQCJSD] at(3.956980, -9.048878) {\verb|freeJosephProblemMatrix(matrix)|};
        \draw[opacity=1.000000, fill=none, line width=0.030722 cm, colorHIDQCFBOJQEQCJSD] (6.586790,-8.135177) -- (6.586790,-8.565284);
        \draw[opacity=1.000000, rounded corners=0.337942 cm, fill=colorHIDQCFBOLSJKMHK, draw=none] (5.235023,-9.671275) -- (7.938556,-9.671275) -- (7.938556,-10.347158) -- (5.235023,-10.347158) -- cycle;
        \draw[opacity=1.000000, rounded corners=0.337942 cm, fill=none, line width=0.030722cm, colorHIDQCFBOJQEQCJSD] (5.235023,-9.671275) -- (7.938556,-9.671275) -- (7.938556,-10.347158) -- (5.235023,-10.347158) -- cycle;
        \verbatimfont{\normalsize\fontHIDQCFBORAKLTQ}
        \node[opacity=1.000000, above right, colorHIDQCFBOJQEQCJSD] at(5.651354, -10.154869) {\verb|Вернуть res|};
        \draw[opacity=1.000000, fill=none, line width=0.030722 cm, colorHIDQCFBOJQEQCJSD] (6.586790,-9.241167) -- (6.586790,-9.671275);
    \end{tikzpicture}
\end{document}}{Функция josephProblemFindSurvivor}{ris:task5_josephProblemFindSurvivor}\\
    \makenewfig{\documentclass{article}

\usepackage[utf8]{inputenc}
\usepackage[english,russian]{babel}
\usepackage{pgfplots}
\usepackage{verbatim}
\usetikzlibrary{positioning}
\usetikzlibrary{shapes.geometric}
\usetikzlibrary{shapes.misc}
\usetikzlibrary{calc}
\usetikzlibrary{chains}
\usetikzlibrary{matrix}
\usetikzlibrary{decorations.text}
\usepackage{fontspec}
\usetikzlibrary{backgrounds}

\begin{document}
    \begin{tikzpicture}[every node/.style={inner sep=0,outer sep=0}, background rectangle/.style={opacity=1.000000, fill=colorGMFLMDSTNCJA}, show background rectangle]
        \makeatletter
        \newcommand{\verbatimfont}[1]{\def\verbatim@font{#1}}%
        \makeatother
        \newfontfamily\fontGMFLMDHSOKIM[Scale=0.988026, SizeFeatures={Size=10.000000}]{Consolas}
        \definecolor{colorGMFLMDSTNCJA}{rgb}{1.000000,1.000000,1.000000}
        \definecolor{colorGMFLMDFPFCHKT}{rgb}{0.949020,0.949020,0.949020}
        \definecolor{colorGMFLMDTIIJSIBT}{rgb}{0.000000,0.000000,0.000000}
        \draw[opacity=1.000000, rounded corners=0.383409 cm, fill=colorGMFLMDFPFCHKT, draw=none] (1.648658,0.006971) -- (9.128616,0.006971) -- (9.128616,-0.759847) -- (1.648658,-0.759847) -- cycle;
        \draw[opacity=1.000000, rounded corners=0.383409 cm, fill=none, line width=0.034855cm, colorGMFLMDTIIJSIBT] (1.648658,0.006971) -- (9.128616,0.006971) -- (9.128616,-0.759847) -- (1.648658,-0.759847) -- cycle;
        \verbatimfont{\normalsize\fontGMFLMDHSOKIM}
        \node[opacity=1.000000, above right, colorGMFLMDTIIJSIBT] at(2.034912, -0.534272) {\verb|josephProblemFindMForSurvivorOne(n)|};
        \draw[fill=colorGMFLMDFPFCHKT, draw=none, opacity=1.000000] (4.506797,-1.247822) -- (6.270478,-1.247822) -- (6.270478,-2.014639) -- (4.506797,-2.014639) -- cycle;
        \draw[opacity=1.000000, fill=none, line width=0.034855cm, colorGMFLMDTIIJSIBT] (4.506797,-1.247822) -- (6.270478,-1.247822) -- (6.270478,-2.014639) -- (4.506797,-2.014639) -- cycle;
        \verbatimfont{\normalsize\fontGMFLMDHSOKIM}
        \node[opacity=1.000000, above right, colorGMFLMDTIIJSIBT] at(4.717892, -1.789064) {\verb|m := -1|};
        \draw[opacity=1.000000, fill=none, line width=0.034855 cm, colorGMFLMDTIIJSIBT] (5.388637,-0.759847) -- (5.388637,-1.247822);
        \draw[opacity=1.000000, fill=colorGMFLMDFPFCHKT, draw=none] (3.241548, -3.576159) -- (4.315092, -2.502614) -- (6.462182, -2.502614) -- (7.535727, -3.576159) -- (6.462182, -4.649704) -- (4.315092, -4.649704) --cycle;
        \draw[opacity=1.000000, fill=none, line width=0.034855cm, colorGMFLMDTIIJSIBT] (3.241548, -3.576159) -- (4.315092, -2.502614) -- (6.462182, -2.502614) -- (7.535727, -3.576159) -- (6.462182, -4.649704) -- (4.315092, -4.649704) --cycle;
        \verbatimfont{\normalsize\fontGMFLMDHSOKIM}
        \node[opacity=1.000000, above right, colorGMFLMDTIIJSIBT] at(4.523729, -3.650340) {\verb|i := 2, n|};
        \draw[opacity=1.000000, fill=colorGMFLMDFPFCHKT, draw=none] (1.474381, -7.094807) -- (5.388637, -9.051934) -- (9.302893, -7.094807) -- (5.388637, -5.137679) --cycle;
        \draw[opacity=1.000000, fill=none, line width=0.034855cm, colorGMFLMDTIIJSIBT] (1.474381, -7.094807) -- (5.388637, -9.051934) -- (9.302893, -7.094807) -- (5.388637, -5.137679) --cycle;
        \verbatimfont{\normalsize\fontGMFLMDHSOKIM}
        \node[opacity=1.000000, above right, colorGMFLMDTIIJSIBT] at(2.034912, -7.210814) {\verb|josephProblemFindSurvivor(n, i) = 1|};
        \draw[opacity=1.000000, fill=none, line width=0.034855cm, colorGMFLMDTIIJSIBT] (1.474381, -7.094807) -- (1.125828, -7.094807) -- (1.125828, -9.539909);
        \draw[fill=colorGMFLMDFPFCHKT, draw=none, opacity=1.000000] (0.341582,-9.539909) -- (1.910073,-9.539909) -- (1.910073,-10.306727) -- (0.341582,-10.306727) -- cycle;
        \draw[opacity=1.000000, fill=none, line width=0.034855cm, colorGMFLMDTIIJSIBT] (0.341582,-9.539909) -- (1.910073,-9.539909) -- (1.910073,-10.306727) -- (0.341582,-10.306727) -- cycle;
        \verbatimfont{\normalsize\fontGMFLMDHSOKIM}
        \node[opacity=1.000000, above right, colorGMFLMDTIIJSIBT] at(0.543339, -10.081152) {\verb|m := i|};
        \verbatimfont{\normalsize\fontGMFLMDHSOKIM}
        \node[opacity=1.000000, above right, colorGMFLMDTIIJSIBT] at(1.125828, -6.885674) {\verb|+|};
        \draw[opacity=1.000000, fill=none, line width=0.034855cm, colorGMFLMDTIIJSIBT] (9.302893, -7.094807) -- (9.651446, -7.094807) -- (9.651446, -9.539909);
        \verbatimfont{\normalsize\fontGMFLMDHSOKIM}
        \node[opacity=1.000000, above right, colorGMFLMDTIIJSIBT] at(9.439632, -6.885674) {\verb|-|};
        \draw[opacity=1.000000, fill=none, line width=0.034855cm, colorGMFLMDTIIJSIBT] (9.651446, -9.539909) -- (9.651446, -10.794702) -- (5.388637, -10.794702);
        \draw[opacity=1.000000, fill=none, line width=0.034855cm, colorGMFLMDTIIJSIBT] (1.125828, -10.306727) -- (1.125828, -10.794702) -- (5.388637, -10.794702);
        \draw[opacity=1.000000, fill=none, line width=0.034855 cm, colorGMFLMDTIIJSIBT] (5.388637,-4.649704) -- (5.388637,-5.137679);
        \verbatimfont{\normalsize\fontGMFLMDHSOKIM}
        \node[opacity=1.000000, above right, colorGMFLMDTIIJSIBT] at(5.155910, -5.002727) {\verb|+|};
        \draw[opacity=1.000000, fill=none, line width=0.034855cm, colorGMFLMDTIIJSIBT] (5.388637, -10.794702) -- (5.388637, -11.282677) -- (-0.006971, -11.282677) -- (-0.006971, -3.576159) -- (3.241548, -3.576159);
        \verbatimfont{\normalsize\fontGMFLMDHSOKIM}
        \node[opacity=1.000000, above right, colorGMFLMDTIIJSIBT] at(9.788186, -3.367027) {\verb|-|};
        \draw[opacity=1.000000, fill=none, line width=0.034855cm, colorGMFLMDTIIJSIBT] (7.535727, -3.576159) -- (10.000000, -3.576159) -- (10.000000, -11.631230) -- (5.388637, -11.631230);
        \draw[opacity=1.000000, fill=none, line width=0.034855 cm, colorGMFLMDTIIJSIBT] (5.388637,-2.014639) -- (5.388637,-2.502614);
        \draw[opacity=1.000000, rounded corners=0.383409 cm, fill=colorGMFLMDFPFCHKT, draw=none] (3.855002,-12.119205) -- (6.922273,-12.119205) -- (6.922273,-12.886023) -- (3.855002,-12.886023) -- cycle;
        \draw[opacity=1.000000, rounded corners=0.383409 cm, fill=none, line width=0.034855cm, colorGMFLMDTIIJSIBT] (3.855002,-12.119205) -- (6.922273,-12.119205) -- (6.922273,-12.886023) -- (3.855002,-12.886023) -- cycle;
        \verbatimfont{\normalsize\fontGMFLMDHSOKIM}
        \node[opacity=1.000000, above right, colorGMFLMDTIIJSIBT] at(4.523729, -12.660448) {\verb|Вернуть m|};
        \draw[opacity=1.000000, fill=none, line width=0.034855 cm, colorGMFLMDTIIJSIBT] (5.388637,-11.631230) -- (5.388637,-12.119205);
    \end{tikzpicture}
\end{document}}{Функция josephProblemFindMForSurvivorOne}{ris:task5_josephProblemFindMForSurvivorOne}

    \subsection{Код программы}
    \begin{minted}{C}
#include <stddef.h>
#include <stdlib.h>

#include "Tasks.h"

typedef struct JosephProblemMatrix {
    int **matrix;
    int n;
} JosephProblemMatrix;

static JosephProblemMatrix createJosephProblemMatrix(int nn) {
    // Сформируем матрицу таким образом, что res[i][j] содержит номер "выжившего"
    // При этом i - изначальное количество людей в круге, j - шаг с которым будут
    // Казнить людей.

    int **res = malloc(nn * sizeof(int *));

    for (int i = 0; i < nn; i++)
        res[i] = malloc(sizeof(int) * nn);

    if (nn <= 0)
        return (JosephProblemMatrix) {NULL, 0};

    // Логично предположить что в круге из одного человека при любом шаге
    // он будет победителем. Так и делаем, всю первую строку матрицы
    // инициализируем единицей
    for (int i = 0; i < nn; i++)
        res[0][i] = 1;

    for (int m = 1; m <= nn; m++)
        // Теперь попробуем рассчитать победителя для конкретной ячейки
        // Допустим, мы казним одного человека. В итоге в круге останется
        // n - 1 человек. Шаг мы тоже знаем - m, значит нам остаётся только
        // вычислить победителя на основе информации о номере человека выжившем при кол-ве
        // людей меньше на 1 с тем же шагом.
        for (int n = 1; n < nn; n++) {
            res[n][m - 1] = (m + res[n - 1][m - 1] - 1) % (n + 1) + 1;
        }

    // Возвращаем матрицу
    return (JosephProblemMatrix) {res, nn};
}

void freeJosephProblemMatrix(JosephProblemMatrix matrix) {
    for (int i = 0; i < matrix.n; i++)
        free(matrix.matrix[i]);

    free(matrix.matrix);
}

int josephProblemFindSurvivor(int n, int m) {
    // Получаем матрицу
    JosephProblemMatrix matrix = createJosephProblemMatrix(n);

    int res = -1;
    if (matrix.matrix != NULL) {
        int indexM = (m - 1) % n;
        // Получаем результат для конкретной ячейки
        res = matrix.matrix[n - 1][indexM];
    }

    // Удаляем матрицу
    freeJosephProblemMatrix(matrix);

    // Возвращаем рзультат
    return res;
}

int josephProblemFindMForSurvivorOne(int n) {
    int m = -1;

    // Здесь просто переберём все возможные шаги
    for (int i = 2; i <= n; i++)
        if (josephProblemFindSurvivor(n, i) == 1)
            m = i;

    return m;
}
    \end{minted}

    \subsection{Тестовые данные}
    \labeltable{
        \begin{tabular}{|c|c|}
            \hline
            Тестовые данные & Ожидаемый результат \\
            \hline
            n = 5, m = 1    & 5                   \\
            \hline
            n = 5, m = 2    & 3                   \\
            \hline
            n = 5, m = 3    & 4                   \\
            \hline
            n = 5, m = 4    & 1                   \\
            \hline
            n = 5, m = 5    & 2                   \\
            \hline
            n = 5, m = 6    & 5                   \\
            \hline
            n = 5, m = 7    & 3                   \\
            \hline
            n = 5, m = 8    & 4                   \\
            \hline
            n = 5, m = 9    & 1                   \\
            \hline
            n = 5, m = 10   & 2                   \\
            \hline
            n = 0, m = 0    & -1                  \\
            \hline
        \end{tabular}
    }{Тестовые данные josephProblemFindSurvivor}{tab:task5-test1-table}
    \labeltable{
        \begin{tabular}{|c|c|}
            \hline
            Тестовые данные & Ожидаемый результат \\
            \hline
            n = 7           & -1                  \\
            \hline
            n = 0           & -1                  \\
            \hline
            n = 10          & 8                   \\
            \hline
        \end{tabular}
    }{Тестовые данные josephProblemFindMForSurvivorOne}{tab:task5-test2-table}

    \subsection{Результат выполнения тестов}
    \makenewfig{
        \includegraphics[width=120mm]{/task5tests}}{Результат выполнения тестов функции josephProblemFindMForSurvivorOne и josephProblemFindSurvivor}


    \section{Линейный поиск}

    \subsection{Задание варианта}
    Матрица $L(n, k)$ состоит из нулей и единиц. Найти в ней самую длинную цепочку подряд стоящих нулей по горизонтали, вертикали или диагонали.

    \subsection{Обоснование}
    Чтобы найти самую длинную последовательность, необходимо пройтись по всем линиям, столбцам и диагоналям матрицы и найти в ней самую длинную последовательность.
    Для определения наибольшей длины последовательности создадим два счётчика - максимальная длина и текущая.
    Дальше переберём элементы.
    Если элемент - ноль, увеличим счётчик на 1, сравнивая с максимальной длиной.
    Если она больше максимальной - присваиваем текущую длину.
    Иначе если элемент 1 - сбрасываем счётчик.
    С проходом по линиям и столбцам всё довольно легко.
    Для прохода по всем диагоналям будем выполнять следующие действия.
    Для каждого элемента крайнего левого столбца будем перебирать элементы диагонали начинающейся с текущей точки проходящей сверху слева - вправо вниз.
    Для каждого элемента крайнего правого столбца будем перебирать элементы диагонали начинающейся с текущей точки проходящей сверху справа - влево вниз.
    Для верхней границы будем перебирать все элементы двух видов диагонали, как слева вверх направо вниз, так и справа вверх налево вниз.

    \subsection{Спецификация}

    \begin{enumerate}
        \item Заголовок:  \mintinline{C}|static void findLongestZeroSubsequence(bool *arr, int size, int *begin, int *end)|
        \item Назначение: находит начало и конец наидлиннейшей подпоследовательности последовательности \mintinline{C}|arr|
        размером \mintinline{C}|size| из нулей и сохраняет её начало в \mintinline{C}|begin| и конец в \mintinline{C}|end|
    \end{enumerate}
    \hline
    \begin{enumerate}
        \item Заголовок:  \mintinline{C}|static void getLine(bool **matrix, int k, int lineIndex, bool *line)|
        \item Назначение: сохраняет в \mintinline{C}|line| ряд по индексу \mintinline{C}|lineIndex| матрицы \mintinline{C}|matrix|
        с количеством столбцов \mintinline{C}|k|
    \end{enumerate}
    \hline
    \begin{enumerate}
        \item Заголовок:  \mintinline{C}|static void getColumn(bool **matrix, int n, int columnIndex, bool *column)|
        \item Назначение: сохраняет в \mintinline{C}|column| столбец по индексу \mintinline{C}|columnIndex| матрицы \mintinline{C}|matrix|
        с количеством рядов \mintinline{C}|n|
    \end{enumerate}
    \hline
    \begin{enumerate}
        \item Заголовок:  \mintinline{C}|static void getDiagonalTopLeftToBottomRight(bool **matrix, int n, int k, |\\
        \mintinline{C}|int iStart, int jStart, bool *diagonal, int *diagonalSize)|
        \item Назначение: сохраняет в \mintinline{C}|diagonal| диагональ и её размер в \mintinline{C}|diagonalSize| по индексу (\mintinline{C}|iStart|; \mintinline{C}|jStart|) матрицы \mintinline{C}|matrix|
        размером \mintinline{C}|n| $\times$ \mintinline{C}|k|.
        Диагональ проходит сверху слева - справа вниз.
    \end{enumerate}
    \hline
    \begin{enumerate}
        \item Заголовок:  \mintinline{C}|static void getDiagonalTopRightToBottomLeft(bool **matrix, int n, int k, |\\
        \mintinline{C}|int iStart, int jStart, bool *diagonal, int *diagonalSize)|
        \item Назначение: сохраняет в \mintinline{C}|diagonal| диагональ и её размер в \mintinline{C}|diagonalSize| по индексу (\mintinline{C}|iStart|; \mintinline{C}|jStart|) матрицы \mintinline{C}|matrix|
        размером \mintinline{C}|n| $\times$ \mintinline{C}|k|.
        Диагональ проходит сверху справа - влево вниз.
    \end{enumerate}
    \hline
    \begin{enumerate}
        \item Заголовок:  \mintinline{C}|static void processLine(bool **matrix, int k, int lineIndex, |\\
        \mintinline{C}|MatrixIndex *longestZeroSubsequenceBegin, MatrixIndex *longestZeroSubsequenceEnd,|\\
        \mintinline{C}|int *maxSubsequenceSize, bool *buf)|
        \item Назначение: находит в ряду по индексу \mintinline{C}|lineIndex| в матрице \mintinline{C}|matrix|
        с количеством столбцов \mintinline{C}|k| наидлиннейшую подпоследовательность из нулей и сохраняет её начало в
        \mintinline{C}|longestZeroSubsequenceBegin|, конец в \mintinline{C}|longestZeroSubsequenceEnd|, размер подпоследовательности в
        \mintinline{C}|maxSubsequenceSize|.
        Принимает ссылку на буфер \mintinline{C}|buf| для хранения временных значений.
    \end{enumerate}
    \hline
    \begin{enumerate}
        \item Заголовок:  \mintinline{C}|static void processColumn(bool **matrix, int n, int columnIndex, |\\
        \mintinline{C}|MatrixIndex *longestZeroSubsequenceBegin, MatrixIndex *longestZeroSubsequenceEnd,|\\
        \mintinline{C}|int *maxSubsequenceSize, bool *buf)|
        \item Назначение: находит в столбце по индексу \mintinline{C}|columnIndex| в матрице \mintinline{C}|matrix|
        с количеством рядов \mintinline{C}|n| наидлиннейшую подпоследовательность из нулей и сохраняет её начало в
        \mintinline{C}|longestZeroSubsequenceBegin|, конец в \mintinline{C}|longestZeroSubsequenceEnd|, размер подпоследовательности в
        \mintinline{C}|maxSubsequenceSize|.
        Принимает ссылку на буфер \mintinline{C}|buf| для хранения временных значений.
    \end{enumerate}
    \hline
    \begin{enumerate}
        \item Заголовок:  \mintinline{C}|static void processDiagonalTopLeftToBottomRight(bool **matrix, int n, int k, |\\
        \mintinline{C}|int row, int column, MatrixIndex *longestZeroSubsequenceBegin, MatrixIndex *longestZeroSubsequenceEnd,|\\
        \mintinline{C}|int *maxSubsequenceSize, bool *buf)|
        \item Назначение: находит в диагонале проходящей сверху слева - вправо вниз по индексу (\mintinline{C}|row|; \mintinline{C}|column|)
        в матрице \mintinline{C}|matrix| размером \mintinline{C}|n| $\times$ \mintinline{C}|k| наидлиннейшую
        подпоследовательность из нулей и сохраняет её начало в
        \mintinline{C}|longestZeroSubsequenceBegin|, конец в \mintinline{C}|longestZeroSubsequenceEnd|, размер подпоследовательности в
        \mintinline{C}|maxSubsequenceSize|.
        Принимает ссылку на буфер \mintinline{C}|buf| для хранения временных значений.
    \end{enumerate}
    \hline
    \begin{enumerate}
        \item Заголовок:  \mintinline{C}|static void processDiagonalTopRightToBottomLeft(bool **matrix, int n, int k, |\\
        \mintinline{C}|int row, int column, MatrixIndex *longestZeroSubsequenceBegin, MatrixIndex *longestZeroSubsequenceEnd,|\\
        \mintinline{C}|int *maxSubsequenceSize, bool *buf)|
        \item Назначение: находит в диагонале проходящей сверху справа - влево вниз по индексу (\mintinline{C}|row|; \mintinline{C}|column|)
        в матрице \mintinline{C}|matrix| размером \mintinline{C}|n| $\times$ \mintinline{C}|k| наидлиннейшую
        подпоследовательность из нулей и сохраняет её начало в
        \mintinline{C}|longestZeroSubsequenceBegin|, конец в \mintinline{C}|longestZeroSubsequenceEnd|, размер подпоследовательности в
        \mintinline{C}|maxSubsequenceSize|.
        Принимает ссылку на буфер \mintinline{C}|buf| для хранения временных значений.
    \end{enumerate}
    \hline
    \begin{enumerate}
        \item Заголовок:  \mintinline{C}|static int max(int a, int b)|
        \item Назначение: Возвращает максимальное из \mintinline{C}|a| и \mintinline{C}|b|
    \end{enumerate}
    \hline
    \begin{enumerate}
        \item Заголовок:  \mintinline{C}|void findLongestZeroLineInMatrix(bool **matrix, int n, int k,|\\
        \mintinline{C}|MatrixIndex *longestZeroSubsequenceBegin, MatrixIndex *longestZeroSubsequenceEnd,|\\
        \mintinline{C}|int *maxSubsequenceSize)|
        \item Назначение: Находит в матрице \mintinline{C}|matrix| размером \mintinline{C}|n| $\times$ \mintinline{C}|k|
        наидлиннейшую
        подпоследовательность из нулей и сохраняет её начало в
        \mintinline{C}|longestZeroSubsequenceBegin|, конец в \mintinline{C}|longestZeroSubsequenceEnd|, размер подпоследовательности в
        \mintinline{C}|maxSubsequenceSize|.
    \end{enumerate}

    \subsection{Блок-схема}
    \makenewfig{% Созданная блок схема работает только с компиляторами XeTeX и LuaTeX.
\documentclass{article}

% Необходимые зависимости
\usepackage[utf8]{inputenc}
\usepackage[english,russian]{babel}
\usepackage{pgfplots}
\usepackage{verbatim}
\usetikzlibrary{positioning}
\usetikzlibrary{shapes.geometric}
\usetikzlibrary{shapes.misc}
\usetikzlibrary{calc}
\usetikzlibrary{chains}
\usetikzlibrary{matrix}
\usetikzlibrary{decorations.text}
\usepackage{fontspec}
\usetikzlibrary{backgrounds}

\begin{document}
% Блок-схема
% Если Вы хотите добавить блок схему в свой документ, скопируйте код между комментариями
% С линияи и вставьте в документ.

% --------------------------
\begin{tikzpicture}[every node/.style={inner sep=0,outer sep=0}, background rectangle/.style={opacity=1.000000, fill=colorPATRTQFHSHOEODS}, show background rectangle]
\makeatletter
\newcommand{\verbatimfont}[1]{\def\verbatim@font{#1}}
\makeatother
% Шрифты
\newfontfamily\fontPATRTQFHKGLKNMS[Scale=0.839895, SizeFeatures={Size=10.000000}]{Consolas}

% Цвета
\definecolor{colorPATRTQFHSHOEODS}{rgb}{1.000000,1.000000,1.000000}
\definecolor{colorPATRTQFHCIQEITPR}{rgb}{0.949020,0.949020,0.949020}
\definecolor{colorPATRTQFHNOFLCFFL}{rgb}{0.000000,0.000000,0.000000}
\draw[opacity=1.000000, rounded corners=0.325926 cm, fill=colorPATRTQFHCIQEITPR, draw=none] (1.360000,0.005926) -- (10.000000,0.005926) -- (10.000000,-0.645926) -- (1.360000,-0.645926) -- cycle;
\draw[opacity=1.000000, rounded corners=0.325926 cm, fill=none, line width=0.029630cm, colorPATRTQFHNOFLCFFL] (1.360000,0.005926) -- (10.000000,0.005926) -- (10.000000,-0.645926) -- (1.360000,-0.645926) -- cycle;
\verbatimfont{\normalsize\fontPATRTQFHKGLKNMS}
\node[opacity=1.000000, above right, colorPATRTQFHNOFLCFFL] at(1.673599, -0.471762) {\verb|findLongestZeroSubsequence(arr, size, begin, end)|};
\draw[fill=colorPATRTQFHCIQEITPR, draw=none, opacity=1.000000] (4.767407,-1.060741) -- (6.592593,-1.060741) -- (6.592593,-1.712593) -- (4.767407,-1.712593) -- cycle;
\draw[opacity=1.000000, fill=none, line width=0.029630cm, colorPATRTQFHNOFLCFFL] (4.767407,-1.060741) -- (6.592593,-1.060741) -- (6.592593,-1.712593) -- (4.767407,-1.712593) -- cycle;
\verbatimfont{\normalsize\fontPATRTQFHKGLKNMS}
\node[opacity=1.000000, above right, colorPATRTQFHNOFLCFFL] at(4.938728, -1.538429) {\verb|cLen := 0|};
\draw[opacity=1.000000, fill=none, line width=0.029630 cm, colorPATRTQFHNOFLCFFL] (5.680000,-0.645926) -- (5.680000,-1.060741);
\draw[fill=colorPATRTQFHCIQEITPR, draw=none, opacity=1.000000] (4.767407,-2.127407) -- (6.592593,-2.127407) -- (6.592593,-2.779259) -- (4.767407,-2.779259) -- cycle;
\draw[opacity=1.000000, fill=none, line width=0.029630cm, colorPATRTQFHNOFLCFFL] (4.767407,-2.127407) -- (6.592593,-2.127407) -- (6.592593,-2.779259) -- (4.767407,-2.779259) -- cycle;
\verbatimfont{\normalsize\fontPATRTQFHKGLKNMS}
\node[opacity=1.000000, above right, colorPATRTQFHNOFLCFFL] at(4.938728, -2.605096) {\verb|mLen := 0|};
\draw[opacity=1.000000, fill=none, line width=0.029630 cm, colorPATRTQFHNOFLCFFL] (5.680000,-1.712593) -- (5.680000,-2.127407);
\draw[fill=colorPATRTQFHCIQEITPR, draw=none, opacity=1.000000] (4.441481,-3.194074) -- (6.918519,-3.194074) -- (6.918519,-3.845926) -- (4.441481,-3.845926) -- cycle;
\draw[opacity=1.000000, fill=none, line width=0.029630cm, colorPATRTQFHNOFLCFFL] (4.441481,-3.194074) -- (6.918519,-3.194074) -- (6.918519,-3.845926) -- (4.441481,-3.845926) -- cycle;
\verbatimfont{\normalsize\fontPATRTQFHKGLKNMS}
\node[opacity=1.000000, above right, colorPATRTQFHNOFLCFFL] at(4.621040, -3.671762) {\verb|tmpBegin := 0|};
\draw[opacity=1.000000, fill=none, line width=0.029630 cm, colorPATRTQFHNOFLCFFL] (5.680000,-2.779259) -- (5.680000,-3.194074);
\draw[fill=colorPATRTQFHCIQEITPR, draw=none, opacity=1.000000] (4.687407,-4.260741) -- (6.672593,-4.260741) -- (6.672593,-4.912593) -- (4.687407,-4.912593) -- cycle;
\draw[opacity=1.000000, fill=none, line width=0.029630cm, colorPATRTQFHNOFLCFFL] (4.687407,-4.260741) -- (6.672593,-4.260741) -- (6.672593,-4.912593) -- (4.687407,-4.912593) -- cycle;
\verbatimfont{\normalsize\fontPATRTQFHKGLKNMS}
\node[opacity=1.000000, above right, colorPATRTQFHNOFLCFFL] at(4.850481, -4.738429) {\verb|begin := 0|};
\draw[opacity=1.000000, fill=none, line width=0.029630 cm, colorPATRTQFHNOFLCFFL] (5.680000,-3.845926) -- (5.680000,-4.260741);
\draw[fill=colorPATRTQFHCIQEITPR, draw=none, opacity=1.000000] (4.850370,-5.327407) -- (6.509630,-5.327407) -- (6.509630,-5.979259) -- (4.850370,-5.979259) -- cycle;
\draw[opacity=1.000000, fill=none, line width=0.029630cm, colorPATRTQFHNOFLCFFL] (4.850370,-5.327407) -- (6.509630,-5.327407) -- (6.509630,-5.979259) -- (4.850370,-5.979259) -- cycle;
\verbatimfont{\normalsize\fontPATRTQFHKGLKNMS}
\node[opacity=1.000000, above right, colorPATRTQFHNOFLCFFL] at(5.026974, -5.805096) {\verb|end := 0|};
\draw[opacity=1.000000, fill=none, line width=0.029630 cm, colorPATRTQFHNOFLCFFL] (5.680000,-4.912593) -- (5.680000,-5.327407);
\draw[opacity=1.000000, fill=colorPATRTQFHCIQEITPR, draw=none] (2.717037, -7.875556) -- (4.198519, -6.394074) -- (7.161481, -6.394074) -- (8.642963, -7.875556) -- (7.161481, -9.357037) -- (4.198519, -9.357037) --cycle;
\draw[opacity=1.000000, fill=none, line width=0.029630cm, colorPATRTQFHNOFLCFFL] (2.717037, -7.875556) -- (4.198519, -6.394074) -- (7.161481, -6.394074) -- (8.642963, -7.875556) -- (7.161481, -9.357037) -- (4.198519, -9.357037) --cycle;
\verbatimfont{\normalsize\fontPATRTQFHKGLKNMS}
\node[opacity=1.000000, above right, colorPATRTQFHNOFLCFFL] at(4.373949, -7.956207) {\verb|i := 0, size - 1|};
\draw[opacity=1.000000, fill=colorPATRTQFHCIQEITPR, draw=none] (4.391111, -10.416296) -- (5.680000, -11.060741) -- (6.968889, -10.416296) -- (5.680000, -9.771852) --cycle;
\draw[opacity=1.000000, fill=none, line width=0.029630cm, colorPATRTQFHNOFLCFFL] (4.391111, -10.416296) -- (5.680000, -11.060741) -- (6.968889, -10.416296) -- (5.680000, -9.771852) --cycle;
\verbatimfont{\normalsize\fontPATRTQFHKGLKNMS}
\node[opacity=1.000000, above right, colorPATRTQFHNOFLCFFL] at(4.850481, -10.532503) {\verb|arr[i] = 0|};
\draw[opacity=1.000000, fill=none, line width=0.029630cm, colorPATRTQFHNOFLCFFL] (4.391111, -10.416296) -- (3.715556, -10.416296) -- (3.715556, -11.475556);
\draw[fill=colorPATRTQFHCIQEITPR, draw=none, opacity=1.000000] (2.234074,-11.475556) -- (5.197037,-11.475556) -- (5.197037,-12.127407) -- (2.234074,-12.127407) -- cycle;
\draw[opacity=1.000000, fill=none, line width=0.029630cm, colorPATRTQFHNOFLCFFL] (2.234074,-11.475556) -- (5.197037,-11.475556) -- (5.197037,-12.127407) -- (2.234074,-12.127407) -- cycle;
\verbatimfont{\normalsize\fontPATRTQFHKGLKNMS}
\node[opacity=1.000000, above right, colorPATRTQFHNOFLCFFL] at(2.409504, -11.953244) {\verb|cLen := cLen + 1|};
\draw[opacity=1.000000, fill=colorPATRTQFHCIQEITPR, draw=none] (2.343704, -13.228148) -- (3.715556, -13.914074) -- (5.087407, -13.228148) -- (3.715556, -12.542222) --cycle;
\draw[opacity=1.000000, fill=none, line width=0.029630cm, colorPATRTQFHNOFLCFFL] (2.343704, -13.228148) -- (3.715556, -13.914074) -- (5.087407, -13.228148) -- (3.715556, -12.542222) --cycle;
\verbatimfont{\normalsize\fontPATRTQFHKGLKNMS}
\node[opacity=1.000000, above right, colorPATRTQFHNOFLCFFL] at(2.815439, -13.344355) {\verb|cLen > mLen|};
\draw[opacity=1.000000, fill=none, line width=0.029630cm, colorPATRTQFHNOFLCFFL] (2.343704, -13.228148) -- (1.854815, -13.228148) -- (1.854815, -14.328889);
\draw[fill=colorPATRTQFHCIQEITPR, draw=none, opacity=1.000000] (0.699259,-14.328889) -- (3.010370,-14.328889) -- (3.010370,-14.980741) -- (0.699259,-14.980741) -- cycle;
\draw[opacity=1.000000, fill=none, line width=0.029630cm, colorPATRTQFHNOFLCFFL] (0.699259,-14.328889) -- (3.010370,-14.328889) -- (3.010370,-14.980741) -- (0.699259,-14.980741) -- cycle;
\verbatimfont{\normalsize\fontPATRTQFHKGLKNMS}
\node[opacity=1.000000, above right, colorPATRTQFHNOFLCFFL] at(0.866452, -14.806577) {\verb|mLen := cLen|};
\draw[fill=colorPATRTQFHCIQEITPR, draw=none, opacity=1.000000] (0.290370,-15.395556) -- (3.419259,-15.395556) -- (3.419259,-16.047407) -- (0.290370,-16.047407) -- cycle;
\draw[opacity=1.000000, fill=none, line width=0.029630cm, colorPATRTQFHNOFLCFFL] (0.290370,-15.395556) -- (3.419259,-15.395556) -- (3.419259,-16.047407) -- (0.290370,-16.047407) -- cycle;
\verbatimfont{\normalsize\fontPATRTQFHKGLKNMS}
\node[opacity=1.000000, above right, colorPATRTQFHNOFLCFFL] at(0.460517, -15.873244) {\verb|begin := tmpBegin|};
\draw[opacity=1.000000, fill=none, line width=0.029630 cm, colorPATRTQFHNOFLCFFL] (1.854815,-14.980741) -- (1.854815,-15.395556);
\draw[fill=colorPATRTQFHCIQEITPR, draw=none, opacity=1.000000] (0.699259,-16.462222) -- (3.010370,-16.462222) -- (3.010370,-17.114074) -- (0.699259,-17.114074) -- cycle;
\draw[opacity=1.000000, fill=none, line width=0.029630cm, colorPATRTQFHNOFLCFFL] (0.699259,-16.462222) -- (3.010370,-16.462222) -- (3.010370,-17.114074) -- (0.699259,-17.114074) -- cycle;
\verbatimfont{\normalsize\fontPATRTQFHKGLKNMS}
\node[opacity=1.000000, above right, colorPATRTQFHNOFLCFFL] at(0.866452, -16.939910) {\verb|end := i + 1|};
\draw[opacity=1.000000, fill=none, line width=0.029630 cm, colorPATRTQFHNOFLCFFL] (1.854815,-16.047407) -- (1.854815,-16.462222);
\verbatimfont{\normalsize\fontPATRTQFHKGLKNMS}
\node[opacity=1.000000, above right, colorPATRTQFHNOFLCFFL] at(1.854815, -13.050370) {\verb|+|};
\draw[opacity=1.000000, fill=none, line width=0.029630cm, colorPATRTQFHNOFLCFFL] (5.087407, -13.228148) -- (5.383704, -13.228148) -- (5.383704, -14.328889);
\verbatimfont{\normalsize\fontPATRTQFHKGLKNMS}
\node[opacity=1.000000, above right, colorPATRTQFHNOFLCFFL] at(5.207210, -13.050370) {\verb|-|};
\draw[opacity=1.000000, fill=none, line width=0.029630cm, colorPATRTQFHNOFLCFFL] (5.383704, -14.328889) -- (5.383704, -17.528889) -- (3.715556, -17.528889);
\draw[opacity=1.000000, fill=none, line width=0.029630cm, colorPATRTQFHNOFLCFFL] (1.854815, -17.114074) -- (1.854815, -17.528889) -- (3.715556, -17.528889);
\draw[opacity=1.000000, fill=none, line width=0.029630 cm, colorPATRTQFHNOFLCFFL] (3.715556,-12.127407) -- (3.715556,-12.542222);
\verbatimfont{\normalsize\fontPATRTQFHKGLKNMS}
\node[opacity=1.000000, above right, colorPATRTQFHNOFLCFFL] at(3.715556, -10.238519) {\verb|+|};
\draw[opacity=1.000000, fill=none, line width=0.029630cm, colorPATRTQFHNOFLCFFL] (6.968889, -10.416296) -- (7.540741, -10.416296) -- (7.540741, -11.475556);
\verbatimfont{\normalsize\fontPATRTQFHKGLKNMS}
\node[opacity=1.000000, above right, colorPATRTQFHNOFLCFFL] at(7.364247, -10.238519) {\verb|-|};
\draw[fill=colorPATRTQFHCIQEITPR, draw=none, opacity=1.000000] (5.976296,-11.475556) -- (9.105185,-11.475556) -- (9.105185,-12.127407) -- (5.976296,-12.127407) -- cycle;
\draw[opacity=1.000000, fill=none, line width=0.029630cm, colorPATRTQFHNOFLCFFL] (5.976296,-11.475556) -- (9.105185,-11.475556) -- (9.105185,-12.127407) -- (5.976296,-12.127407) -- cycle;
\verbatimfont{\normalsize\fontPATRTQFHKGLKNMS}
\node[opacity=1.000000, above right, colorPATRTQFHNOFLCFFL] at(6.146443, -11.953244) {\verb|tmpBegin := i + 1|};
\draw[fill=colorPATRTQFHCIQEITPR, draw=none, opacity=1.000000] (6.628148,-12.542222) -- (8.453333,-12.542222) -- (8.453333,-13.194074) -- (6.628148,-13.194074) -- cycle;
\draw[opacity=1.000000, fill=none, line width=0.029630cm, colorPATRTQFHNOFLCFFL] (6.628148,-12.542222) -- (8.453333,-12.542222) -- (8.453333,-13.194074) -- (6.628148,-13.194074) -- cycle;
\verbatimfont{\normalsize\fontPATRTQFHKGLKNMS}
\node[opacity=1.000000, above right, colorPATRTQFHNOFLCFFL] at(6.799468, -13.019910) {\verb|cLen := 0|};
\draw[opacity=1.000000, fill=none, line width=0.029630 cm, colorPATRTQFHNOFLCFFL] (7.540741,-12.127407) -- (7.540741,-12.542222);
\draw[opacity=1.000000, fill=none, line width=0.029630cm, colorPATRTQFHNOFLCFFL] (7.540741, -13.194074) -- (7.540741, -17.943704) -- (5.680000, -17.943704);
\draw[opacity=1.000000, fill=none, line width=0.029630cm, colorPATRTQFHNOFLCFFL] (3.715556, -17.528889) -- (3.715556, -17.943704) -- (5.680000, -17.943704);
\draw[opacity=1.000000, fill=none, line width=0.029630 cm, colorPATRTQFHNOFLCFFL] (5.680000,-9.357037) -- (5.680000,-9.771852);
\verbatimfont{\normalsize\fontPATRTQFHKGLKNMS}
\node[opacity=1.000000, above right, colorPATRTQFHNOFLCFFL] at(5.485729, -9.674725) {\verb|+|};
\draw[opacity=1.000000, fill=none, line width=0.029630cm, colorPATRTQFHNOFLCFFL] (5.680000, -17.943704) -- (5.680000, -18.358519) -- (-0.005926, -18.358519) -- (-0.005926, -7.875556) -- (2.717037, -7.875556);
\verbatimfont{\normalsize\fontPATRTQFHKGLKNMS}
\node[opacity=1.000000, above right, colorPATRTQFHNOFLCFFL] at(9.224988, -7.697778) {\verb|-|};
\draw[opacity=1.000000, fill=none, line width=0.029630cm, colorPATRTQFHNOFLCFFL] (8.642963, -7.875556) -- (9.401481, -7.875556) -- (9.401481, -18.654815) -- (5.680000, -18.654815);
\draw[opacity=1.000000, fill=none, line width=0.029630 cm, colorPATRTQFHNOFLCFFL] (5.680000,-5.979259) -- (5.680000,-6.394074);
\draw[opacity=1.000000, rounded corners=0.325926 cm, fill=colorPATRTQFHCIQEITPR, draw=none] (3.887407,-19.069630) -- (7.472593,-19.069630) -- (7.472593,-19.721481) -- (3.887407,-19.721481) -- cycle;
\draw[opacity=1.000000, rounded corners=0.325926 cm, fill=none, line width=0.029630cm, colorPATRTQFHNOFLCFFL] (3.887407,-19.069630) -- (7.472593,-19.069630) -- (7.472593,-19.721481) -- (3.887407,-19.721481) -- cycle;
\verbatimfont{\normalsize\fontPATRTQFHKGLKNMS}
\node[opacity=1.000000, above right, colorPATRTQFHNOFLCFFL] at(4.197455, -19.547318) {\verb|Выход (begin, end)|};
\draw[opacity=1.000000, fill=none, line width=0.029630 cm, colorPATRTQFHNOFLCFFL] (5.680000,-18.654815) -- (5.680000,-19.069630);
\end{tikzpicture}
% --------------------------

% Конец блок схемы
\end{document}}{Функция findLongestZeroSubsequence}{ris:task6_1_findLongestZeroSubsequence}\\
    \makenewfig{% Созданная блок схема работает только с компиляторами XeTeX и LuaTeX.
\documentclass{article}

% Необходимые зависимости
\usepackage[utf8]{inputenc}
\usepackage[english,russian]{babel}
\usepackage{pgfplots}
\usepackage{verbatim}
\usetikzlibrary{positioning}
\usetikzlibrary{shapes.geometric}
\usetikzlibrary{shapes.misc}
\usetikzlibrary{calc}
\usetikzlibrary{chains}
\usetikzlibrary{matrix}
\usetikzlibrary{decorations.text}
\usepackage{fontspec}
\usetikzlibrary{backgrounds}

\begin{document}
% Блок-схема
% Если Вы хотите добавить блок схему в свой документ, скопируйте код между комментариями
% С линияи и вставьте в документ.

% --------------------------
\begin{tikzpicture}[every node/.style={inner sep=0,outer sep=0}, background rectangle/.style={opacity=1.000000, fill=colorBBSGDNJKBRJGS}, show background rectangle]
\makeatletter
\newcommand{\verbatimfont}[1]{\def\verbatim@font{#1}}
\makeatother
% Шрифты
\newfontfamily\fontBBSGDNGMJIPNQ[Scale=0.661065, SizeFeatures={Size=10.000000}]{Consolas}

% Цвета
\definecolor{colorBBSGDNJKBRJGS}{rgb}{1.000000,1.000000,1.000000}
\definecolor{colorBBSGDNSJNTFSQ}{rgb}{0.949020,0.949020,0.949020}
\definecolor{colorBBSGDNNRERGJ}{rgb}{0.000000,0.000000,0.000000}
\draw[opacity=1.000000, rounded corners=0.256530 cm, fill=colorBBSGDNSJNTFSQ, draw=none] (-0.004664,0.004664) -- (5.000000,0.004664) -- (5.000000,-0.508396) -- (-0.004664,-0.508396) -- cycle;
\draw[opacity=1.000000, rounded corners=0.256530 cm, fill=none, line width=0.023321cm, colorBBSGDNNRERGJ] (-0.004664,0.004664) -- (5.000000,0.004664) -- (5.000000,-0.508396) -- (-0.004664,-0.508396) -- cycle;
\verbatimfont{\normalsize\fontBBSGDNGMJIPNQ}
\node[opacity=1.000000, above right, colorBBSGDNNRERGJ] at(0.237784, -0.368443) {\verb|getLine(matrix, k, lineIndex, line)|};
\draw[opacity=1.000000, fill=colorBBSGDNSJNTFSQ, draw=none] (0.548041, -1.809701) -- (1.522854, -0.834888) -- (3.472481, -0.834888) -- (4.447295, -1.809701) -- (3.472481, -2.784515) -- (1.522854, -2.784515) --cycle;
\draw[opacity=1.000000, fill=none, line width=0.023321cm, colorBBSGDNNRERGJ] (0.548041, -1.809701) -- (1.522854, -0.834888) -- (3.472481, -0.834888) -- (4.447295, -1.809701) -- (3.472481, -2.784515) -- (1.522854, -2.784515) --cycle;
\verbatimfont{\normalsize\fontBBSGDNGMJIPNQ}
\node[opacity=1.000000, above right, colorBBSGDNNRERGJ] at(1.650211, -1.870309) {\verb|i := 0, k - 1|};
\draw[fill=colorBBSGDNSJNTFSQ, draw=none, opacity=1.000000] (0.368470,-3.111007) -- (4.626866,-3.111007) -- (4.626866,-3.624067) -- (0.368470,-3.624067) -- cycle;
\draw[opacity=1.000000, fill=none, line width=0.023321cm, colorBBSGDNNRERGJ] (0.368470,-3.111007) -- (4.626866,-3.111007) -- (4.626866,-3.624067) -- (0.368470,-3.624067) -- cycle;
\verbatimfont{\normalsize\fontBBSGDNGMJIPNQ}
\node[opacity=1.000000, above right, colorBBSGDNNRERGJ] at(0.502614, -3.484115) {\verb|line[j] := matrix[lineIndex][j]|};
\draw[opacity=1.000000, fill=none, line width=0.023321 cm, colorBBSGDNNRERGJ] (2.497668,-2.784515) -- (2.497668,-3.111007);
\verbatimfont{\normalsize\fontBBSGDNGMJIPNQ}
\node[opacity=1.000000, above right, colorBBSGDNNRERGJ] at(2.342433, -3.031690) {\verb|+|};
\draw[opacity=1.000000, fill=none, line width=0.023321cm, colorBBSGDNNRERGJ] (2.497668, -3.624067) -- (2.497668, -3.950560) -- (0.135261, -3.950560) -- (0.135261, -1.809701) -- (0.548041, -1.809701);
\verbatimfont{\normalsize\fontBBSGDNGMJIPNQ}
\node[opacity=1.000000, above right, colorBBSGDNNRERGJ] at(4.718832, -1.669776) {\verb|-|};
\draw[opacity=1.000000, fill=none, line width=0.023321cm, colorBBSGDNNRERGJ] (4.447295, -1.809701) -- (4.860075, -1.809701) -- (4.860075, -4.183769) -- (2.497668, -4.183769);
\draw[opacity=1.000000, fill=none, line width=0.023321 cm, colorBBSGDNNRERGJ] (2.497668,-0.508396) -- (2.497668,-0.834888);
\draw[opacity=1.000000, rounded corners=0.256530 cm, fill=colorBBSGDNSJNTFSQ, draw=none] (1.471549,-4.510261) -- (3.523787,-4.510261) -- (3.523787,-5.023321) -- (1.471549,-5.023321) -- cycle;
\draw[opacity=1.000000, rounded corners=0.256530 cm, fill=none, line width=0.023321cm, colorBBSGDNNRERGJ] (1.471549,-4.510261) -- (3.523787,-4.510261) -- (3.523787,-5.023321) -- (1.471549,-5.023321) -- cycle;
\verbatimfont{\normalsize\fontBBSGDNGMJIPNQ}
\node[opacity=1.000000, above right, colorBBSGDNNRERGJ] at(1.720833, -4.883369) {\verb|Выход (line)|};
\draw[opacity=1.000000, fill=none, line width=0.023321 cm, colorBBSGDNNRERGJ] (2.497668,-4.183769) -- (2.497668,-4.510261);
\end{tikzpicture}
% --------------------------

% Конец блок схемы
\end{document}}{Функция getLine}{ris:task6_2_getLine}\\
    \makenewfig{% Созданная блок схема работает только с компиляторами XeTeX и LuaTeX.
\documentclass{article}

% Необходимые зависимости
\usepackage[utf8]{inputenc}
\usepackage[english,russian]{babel}
\usepackage{pgfplots}
\usepackage{verbatim}
\usetikzlibrary{positioning}
\usetikzlibrary{shapes.geometric}
\usetikzlibrary{shapes.misc}
\usetikzlibrary{calc}
\usetikzlibrary{chains}
\usetikzlibrary{matrix}
\usetikzlibrary{decorations.text}
\usepackage{fontspec}
\usetikzlibrary{backgrounds}

\begin{document}
% Блок-схема
% Если Вы хотите добавить блок схему в свой документ, скопируйте код между комментариями
% С линияи и вставьте в документ.

% --------------------------
\begin{tikzpicture}[every node/.style={inner sep=0,outer sep=0}, background rectangle/.style={opacity=1.000000, fill=colorTDHQLEPDJQDBH}, show background rectangle]
\makeatletter
\newcommand{\verbatimfont}[1]{\def\verbatim@font{#1}}
\makeatother
% Шрифты
\newfontfamily\fontTDHQLEPMKNPTPOB[Scale=0.572887, SizeFeatures={Size=10.000000}]{Consolas}

% Цвета
\definecolor{colorTDHQLEPDJQDBH}{rgb}{1.000000,1.000000,1.000000}
\definecolor{colorTDHQLEPDMCMIFP}{rgb}{0.949020,0.949020,0.949020}
\definecolor{colorTDHQLEPGCMHCQKS}{rgb}{0.000000,0.000000,0.000000}
\draw[opacity=1.000000, rounded corners=0.222312 cm, fill=colorTDHQLEPDMCMIFP, draw=none] (-0.004042,0.004042) -- (5.000000,0.004042) -- (5.000000,-0.440582) -- (-0.004042,-0.440582) -- cycle;
\draw[opacity=1.000000, rounded corners=0.222312 cm, fill=none, line width=0.020210cm, colorTDHQLEPGCMHCQKS] (-0.004042,0.004042) -- (5.000000,0.004042) -- (5.000000,-0.440582) -- (-0.004042,-0.440582) -- cycle;
\verbatimfont{\normalsize\fontTDHQLEPMKNPTPOB}
\node[opacity=1.000000, above right, colorTDHQLEPGCMHCQKS] at(0.203070, -0.316931) {\verb|getColumn(matrix, n, columnIndex, column)|};
\draw[opacity=1.000000, fill=colorTDHQLEPDMCMIFP, draw=none] (0.808407, -1.568310) -- (1.653193, -0.723525) -- (3.342765, -0.723525) -- (4.187551, -1.568310) -- (3.342765, -2.413096) -- (1.653193, -2.413096) --cycle;
\draw[opacity=1.000000, fill=none, line width=0.020210cm, colorTDHQLEPGCMHCQKS] (0.808407, -1.568310) -- (1.653193, -0.723525) -- (3.342765, -0.723525) -- (4.187551, -1.568310) -- (3.342765, -2.413096) -- (1.653193, -2.413096) --cycle;
\verbatimfont{\normalsize\fontTDHQLEPMKNPTPOB}
\node[opacity=1.000000, above right, colorTDHQLEPGCMHCQKS] at(1.774200, -1.618467) {\verb|j := 0, n - 1|};
\draw[fill=colorTDHQLEPDMCMIFP, draw=none, opacity=1.000000] (0.430477,-2.696039) -- (4.565481,-2.696039) -- (4.565481,-3.140663) -- (0.430477,-3.140663) -- cycle;
\draw[opacity=1.000000, fill=none, line width=0.020210cm, colorTDHQLEPGCMHCQKS] (0.430477,-2.696039) -- (4.565481,-2.696039) -- (4.565481,-3.140663) -- (0.430477,-3.140663) -- cycle;
\verbatimfont{\normalsize\fontTDHQLEPMKNPTPOB}
\node[opacity=1.000000, above right, colorTDHQLEPGCMHCQKS] at(0.538480, -3.017012) {\verb|column[j] := matrix[j][columnIndex]|};
\draw[opacity=1.000000, fill=none, line width=0.020210 cm, colorTDHQLEPGCMHCQKS] (2.497979,-2.413096) -- (2.497979,-2.696039);
\verbatimfont{\normalsize\fontTDHQLEPMKNPTPOB}
\node[opacity=1.000000, above right, colorTDHQLEPGCMHCQKS] at(2.344628, -2.624934) {\verb|+|};
\draw[opacity=1.000000, fill=none, line width=0.020210cm, colorTDHQLEPGCMHCQKS] (2.497979, -3.140663) -- (2.497979, -3.423605) -- (0.228375, -3.423605) -- (0.228375, -1.568310) -- (0.808407, -1.568310);
\verbatimfont{\normalsize\fontTDHQLEPMKNPTPOB}
\node[opacity=1.000000, above right, colorTDHQLEPGCMHCQKS] at(4.626358, -1.447049) {\verb|-|};
\draw[opacity=1.000000, fill=none, line width=0.020210cm, colorTDHQLEPGCMHCQKS] (4.187551, -1.568310) -- (4.767583, -1.568310) -- (4.767583, -3.625707) -- (2.497979, -3.625707);
\draw[opacity=1.000000, fill=none, line width=0.020210 cm, colorTDHQLEPGCMHCQKS] (2.497979,-0.440582) -- (2.497979,-0.723525);
\draw[opacity=1.000000, rounded corners=0.222312 cm, fill=colorTDHQLEPDMCMIFP, draw=none] (1.497575,-3.908650) -- (3.498383,-3.908650) -- (3.498383,-4.353274) -- (1.497575,-4.353274) -- cycle;
\draw[opacity=1.000000, rounded corners=0.222312 cm, fill=none, line width=0.020210cm, colorTDHQLEPGCMHCQKS] (1.497575,-3.908650) -- (3.498383,-3.908650) -- (3.498383,-4.353274) -- (1.497575,-4.353274) -- cycle;
\verbatimfont{\normalsize\fontTDHQLEPMKNPTPOB}
\node[opacity=1.000000, above right, colorTDHQLEPGCMHCQKS] at(1.703587, -4.229623) {\verb|Выход (column)|};
\draw[opacity=1.000000, fill=none, line width=0.020210 cm, colorTDHQLEPGCMHCQKS] (2.497979,-3.625707) -- (2.497979,-3.908650);
\end{tikzpicture}
% --------------------------

% Конец блок схемы
\end{document}}{Функция getColumn}{ris:task6_3_getColumn}\\
    \makenewfig{% Созданная блок схема работает только с компиляторами XeTeX и LuaTeX.
\documentclass{article}

% Необходимые зависимости
\usepackage[utf8]{inputenc}
\usepackage[english,russian]{babel}
\usepackage{pgfplots}
\usepackage{verbatim}
\usetikzlibrary{positioning}
\usetikzlibrary{shapes.geometric}
\usetikzlibrary{shapes.misc}
\usetikzlibrary{calc}
\usetikzlibrary{chains}
\usetikzlibrary{matrix}
\usetikzlibrary{decorations.text}
\usepackage{fontspec}
\usetikzlibrary{backgrounds}

\begin{document}
% Блок-схема
% Если Вы хотите добавить блок схему в свой документ, скопируйте код между комментариями
% С линияи и вставьте в документ.

% --------------------------
\begin{tikzpicture}[every node/.style={inner sep=0,outer sep=0}, background rectangle/.style={opacity=1.000000, fill=colorOGRLRHTFADPCG}, show background rectangle]
\makeatletter
\newcommand{\verbatimfont}[1]{\def\verbatim@font{#1}}
\makeatother
% Шрифты
\newfontfamily\fontOGRLRHTODFBDK[Scale=0.824025, SizeFeatures={Size=10.000000}]{Consolas}

% Цвета
\definecolor{colorOGRLRHTFADPCG}{rgb}{1.000000,1.000000,1.000000}
\definecolor{colorOGRLRHTRFLQATL}{rgb}{0.949020,0.949020,0.949020}
\definecolor{colorOGRLRHTDITJEAHA}{rgb}{0.000000,0.000000,0.000000}
\draw[opacity=1.000000, rounded corners=0.627907 cm, fill=colorOGRLRHTRFLQATL, draw=none] (0.311047,0.005814) -- (6.683140,0.005814) -- (6.683140,-1.250000) -- (0.311047,-1.250000) -- cycle;
\draw[opacity=1.000000, rounded corners=0.627907 cm, fill=none, line width=0.029070cm, colorOGRLRHTDITJEAHA] (0.311047,0.005814) -- (6.683140,0.005814) -- (6.683140,-1.250000) -- (0.311047,-1.250000) -- cycle;
\verbatimfont{\normalsize\fontOGRLRHTODFBDK}
\node[opacity=1.000000, above right, colorOGRLRHTDITJEAHA] at(0.937330, -0.468926) {\verb|getDiagonalTopLeftToBottomRight(|};
\verbatimfont{\normalsize\fontOGRLRHTODFBDK}
\node[opacity=1.000000, above right, colorOGRLRHTDITJEAHA] at(1.096212, -0.804132) {\verb|matrix, n, k, iStart, jStart, |};
\verbatimfont{\normalsize\fontOGRLRHTODFBDK}
\node[opacity=1.000000, above right, colorOGRLRHTDITJEAHA] at(1.643471, -1.139338) {\verb|diagonal, diagonalSize)|};
\draw[fill=colorOGRLRHTRFLQATL, draw=none, opacity=1.000000] (2.441860,-1.656977) -- (4.552326,-1.656977) -- (4.552326,-2.296512) -- (2.441860,-2.296512) -- cycle;
\draw[opacity=1.000000, fill=none, line width=0.029070cm, colorOGRLRHTDITJEAHA] (2.441860,-1.656977) -- (4.552326,-1.656977) -- (4.552326,-2.296512) -- (2.441860,-2.296512) -- cycle;
\verbatimfont{\normalsize\fontOGRLRHTODFBDK}
\node[opacity=1.000000, above right, colorOGRLRHTDITJEAHA] at(2.614416, -2.131717) {\verb|i := iStart|};
\draw[opacity=1.000000, fill=none, line width=0.029070 cm, colorOGRLRHTDITJEAHA] (3.497093,-1.250000) -- (3.497093,-1.656977);
\draw[fill=colorOGRLRHTRFLQATL, draw=none, opacity=1.000000] (2.441860,-2.703488) -- (4.552326,-2.703488) -- (4.552326,-3.343023) -- (2.441860,-3.343023) -- cycle;
\draw[opacity=1.000000, fill=none, line width=0.029070cm, colorOGRLRHTDITJEAHA] (2.441860,-2.703488) -- (4.552326,-2.703488) -- (4.552326,-3.343023) -- (2.441860,-3.343023) -- cycle;
\verbatimfont{\normalsize\fontOGRLRHTODFBDK}
\node[opacity=1.000000, above right, colorOGRLRHTDITJEAHA] at(2.614416, -3.178229) {\verb|j := jStart|};
\draw[opacity=1.000000, fill=none, line width=0.029070 cm, colorOGRLRHTDITJEAHA] (3.497093,-2.296512) -- (3.497093,-2.703488);
\draw[opacity=1.000000, fill=colorOGRLRHTRFLQATL, draw=none] (1.831395, -4.582849) -- (3.497093, -5.415698) -- (5.162791, -4.582849) -- (3.497093, -3.750000) --cycle;
\draw[opacity=1.000000, fill=none, line width=0.029070cm, colorOGRLRHTDITJEAHA] (1.831395, -4.582849) -- (3.497093, -5.415698) -- (5.162791, -4.582849) -- (3.497093, -3.750000) --cycle;
\verbatimfont{\normalsize\fontOGRLRHTODFBDK}
\node[opacity=1.000000, above right, colorOGRLRHTDITJEAHA] at(2.296652, -4.702938) {\verb|i < n and j < k|};
\draw[fill=colorOGRLRHTRFLQATL, draw=none, opacity=1.000000] (0.284884,-5.822674) -- (6.709302,-5.822674) -- (6.709302,-6.462209) -- (0.284884,-6.462209) -- cycle;
\draw[opacity=1.000000, fill=none, line width=0.029070cm, colorOGRLRHTDITJEAHA] (0.284884,-5.822674) -- (6.709302,-5.822674) -- (6.709302,-6.462209) -- (0.284884,-6.462209) -- cycle;
\verbatimfont{\normalsize\fontOGRLRHTODFBDK}
\node[opacity=1.000000, above right, colorOGRLRHTDITJEAHA] at(0.443031, -6.297415) {\verb|diagonal[diagonalSize] := matrix[i][j]|};
\draw[fill=colorOGRLRHTRFLQATL, draw=none, opacity=1.000000] (0.764535,-6.869186) -- (6.229651,-6.869186) -- (6.229651,-7.508721) -- (0.764535,-7.508721) -- cycle;
\draw[opacity=1.000000, fill=none, line width=0.029070cm, colorOGRLRHTDITJEAHA] (0.764535,-6.869186) -- (6.229651,-6.869186) -- (6.229651,-7.508721) -- (0.764535,-7.508721) -- cycle;
\verbatimfont{\normalsize\fontOGRLRHTODFBDK}
\node[opacity=1.000000, above right, colorOGRLRHTDITJEAHA] at(0.937330, -7.343926) {\verb|diagonalSize := diagonalSize + 1|};
\draw[opacity=1.000000, fill=none, line width=0.029070 cm, colorOGRLRHTDITJEAHA] (3.497093,-6.462209) -- (3.497093,-6.869186);
\draw[fill=colorOGRLRHTRFLQATL, draw=none, opacity=1.000000] (2.523256,-7.915698) -- (4.470930,-7.915698) -- (4.470930,-8.555233) -- (2.523256,-8.555233) -- cycle;
\draw[opacity=1.000000, fill=none, line width=0.029070cm, colorOGRLRHTDITJEAHA] (2.523256,-7.915698) -- (4.470930,-7.915698) -- (4.470930,-8.555233) -- (2.523256,-8.555233) -- cycle;
\verbatimfont{\normalsize\fontOGRLRHTODFBDK}
\node[opacity=1.000000, above right, colorOGRLRHTDITJEAHA] at(2.685030, -8.390438) {\verb|i := i + 1|};
\draw[opacity=1.000000, fill=none, line width=0.029070 cm, colorOGRLRHTDITJEAHA] (3.497093,-7.508721) -- (3.497093,-7.915698);
\draw[fill=colorOGRLRHTRFLQATL, draw=none, opacity=1.000000] (2.523256,-8.962209) -- (4.470930,-8.962209) -- (4.470930,-9.601744) -- (2.523256,-9.601744) -- cycle;
\draw[opacity=1.000000, fill=none, line width=0.029070cm, colorOGRLRHTDITJEAHA] (2.523256,-8.962209) -- (4.470930,-8.962209) -- (4.470930,-9.601744) -- (2.523256,-9.601744) -- cycle;
\verbatimfont{\normalsize\fontOGRLRHTODFBDK}
\node[opacity=1.000000, above right, colorOGRLRHTDITJEAHA] at(2.685030, -9.436950) {\verb|j := j + 1|};
\draw[opacity=1.000000, fill=none, line width=0.029070 cm, colorOGRLRHTDITJEAHA] (3.497093,-8.555233) -- (3.497093,-8.962209);
\draw[opacity=1.000000, fill=none, line width=0.029070 cm, colorOGRLRHTDITJEAHA] (3.497093,-5.415698) -- (3.497093,-5.822674);
\verbatimfont{\normalsize\fontOGRLRHTODFBDK}
\node[opacity=1.000000, above right, colorOGRLRHTDITJEAHA] at(3.303116, -5.733461) {\verb|+|};
\draw[opacity=1.000000, fill=none, line width=0.029070cm, colorOGRLRHTDITJEAHA] (3.497093, -9.601744) -- (3.497093, -10.008721) -- (-0.005814, -10.008721) -- (-0.005814, -3.546512) -- (3.497093, -3.546512);
\verbatimfont{\normalsize\fontOGRLRHTODFBDK}
\node[opacity=1.000000, above right, colorOGRLRHTDITJEAHA] at(6.823465, -4.408430) {\verb|-|};
\draw[opacity=1.000000, fill=none, line width=0.029070cm, colorOGRLRHTDITJEAHA] (5.162791, -4.582849) -- (7.000000, -4.582849) -- (7.000000, -10.299419) -- (3.497093, -10.299419);
\draw[opacity=1.000000, fill=none, line width=0.029070 cm, colorOGRLRHTDITJEAHA] (3.497093,-3.343023) -- (3.497093,-3.750000);
\draw[opacity=1.000000, rounded corners=0.319767 cm, fill=colorOGRLRHTRFLQATL, draw=none] (0.779070,-10.706395) -- (6.215116,-10.706395) -- (6.215116,-11.345930) -- (0.779070,-11.345930) -- cycle;
\draw[opacity=1.000000, rounded corners=0.319767 cm, fill=none, line width=0.029070cm, colorOGRLRHTDITJEAHA] (0.779070,-10.706395) -- (6.215116,-10.706395) -- (6.215116,-11.345930) -- (0.779070,-11.345930) -- cycle;
\verbatimfont{\normalsize\fontOGRLRHTODFBDK}
\node[opacity=1.000000, above right, colorOGRLRHTDITJEAHA] at(1.096212, -11.181136) {\verb|Выход (diagonal, diagonalSize)|};
\draw[opacity=1.000000, fill=none, line width=0.029070 cm, colorOGRLRHTDITJEAHA] (3.497093,-10.299419) -- (3.497093,-10.706395);
\end{tikzpicture}
% --------------------------

% Конец блок схемы
\end{document}}{Функция getDiagonalTopLeftToBottomRight}{ris:task6_4_getDiagonalTopLeftToBottomRight}\\
    \makenewfig{\input{dbg_texs/task6_5_getDiagonalTopRightToBottomLeft}}{Функция getDiagonalTopRightToBottomLeft}{ris:task6_5_getDiagonalTopRightToBottomLeft}\\
    \makenewfig{% Созданная блок схема работает только с компиляторами XeTeX и LuaTeX.
\documentclass{article}

% Необходимые зависимости
\usepackage[utf8]{inputenc}
\usepackage[english,russian]{babel}
\usepackage{pgfplots}
\usepackage{verbatim}
\usetikzlibrary{positioning}
\usetikzlibrary{shapes.geometric}
\usetikzlibrary{shapes.misc}
\usetikzlibrary{calc}
\usetikzlibrary{chains}
\usetikzlibrary{matrix}
\usetikzlibrary{decorations.text}
\usepackage{fontspec}
\usetikzlibrary{backgrounds}

\begin{document}
% Блок-схема
% Если Вы хотите добавить блок схему в свой документ, скопируйте код между комментариями
% С линияи и вставьте в документ.

% --------------------------
\begin{tikzpicture}[every node/.style={inner sep=0,outer sep=0}, background rectangle/.style={opacity=1.000000, fill=colorDSJDGBERCFQAP}, show background rectangle]
\makeatletter
\newcommand{\verbatimfont}[1]{\def\verbatim@font{#1}}
\makeatother
% Шрифты
\newfontfamily\fontDSJDGBECQNRIGQ[Scale=0.748323, SizeFeatures={Size=10.000000}]{Consolas}

% Цвета
\definecolor{colorDSJDGBERCFQAP}{rgb}{1.000000,1.000000,1.000000}
\definecolor{colorDSJDGBEMJTPLQ}{rgb}{0.949020,0.949020,0.949020}
\definecolor{colorDSJDGBELSSHKDDC}{rgb}{0.000000,0.000000,0.000000}
\draw[opacity=1.000000, rounded corners=0.430306 cm, fill=colorDSJDGBEMJTPLQ, draw=none] (-0.005280,0.005280) -- (10.000000,0.005280) -- (10.000000,-0.855333) -- (-0.005280,-0.855333) -- cycle;
\draw[opacity=1.000000, rounded corners=0.430306 cm, fill=none, line width=0.026399cm, colorDSJDGBELSSHKDDC] (-0.005280,0.005280) -- (10.000000,0.005280) -- (10.000000,-0.855333) -- (-0.005280,-0.855333) -- cycle;
\verbatimfont{\normalsize\fontDSJDGBECQNRIGQ}
\node[opacity=1.000000, above right, colorDSJDGBELSSHKDDC] at(0.408828, -0.419647) {\verb|processLine(matrix, k, lineIndex, longestZeroSubsequenceBegin, |};
\verbatimfont{\normalsize\fontDSJDGBECQNRIGQ}
\node[opacity=1.000000, above right, colorDSJDGBELSSHKDDC] at(1.291238, -0.717858) {\verb|longestZeroSubsequenceEnd, maxSubsequenceSize, buf)|};
\draw[fill=colorDSJDGBEMJTPLQ, draw=none, opacity=1.000000] (2.283527,-1.224921) -- (7.711193,-1.224921) -- (7.711193,-1.805702) -- (2.283527,-1.805702) -- cycle;
\draw[opacity=1.000000, fill=none, line width=0.026399cm, colorDSJDGBELSSHKDDC] (2.283527,-1.224921) -- (7.711193,-1.224921) -- (7.711193,-1.805702) -- (2.283527,-1.805702) -- cycle;
\draw[opacity=1.000000, fill=none, line width=0.026399cm, colorDSJDGBELSSHKDDC] (2.370644,-1.224921) -- (7.624076,-1.224921) -- (7.624076,-1.805702) -- (2.370644,-1.805702) -- cycle;
\verbatimfont{\normalsize\fontDSJDGBECQNRIGQ}
\node[opacity=1.000000, above right, colorDSJDGBELSSHKDDC] at(2.526612, -1.649847) {\verb|getLine(matrix, k, lineIndex, buf)|};
\draw[opacity=1.000000, fill=none, line width=0.026399 cm, colorDSJDGBELSSHKDDC] (4.997360,-0.855333) -- (4.997360,-1.224921);
\draw[fill=colorDSJDGBEMJTPLQ, draw=none, opacity=1.000000] (3.748680,-2.175290) -- (6.246040,-2.175290) -- (6.246040,-2.756072) -- (3.748680,-2.756072) -- cycle;
\draw[opacity=1.000000, fill=none, line width=0.026399cm, colorDSJDGBELSSHKDDC] (3.748680,-2.175290) -- (6.246040,-2.175290) -- (6.246040,-2.756072) -- (3.748680,-2.756072) -- cycle;
\verbatimfont{\normalsize\fontDSJDGBECQNRIGQ}
\node[opacity=1.000000, above right, colorDSJDGBELSSHKDDC] at(3.903172, -2.600217) {\verb|indexBegin := 0|};
\draw[opacity=1.000000, fill=none, line width=0.026399 cm, colorDSJDGBELSSHKDDC] (4.997360,-1.805702) -- (4.997360,-2.175290);
\draw[fill=colorDSJDGBEMJTPLQ, draw=none, opacity=1.000000] (3.893875,-3.125660) -- (6.100845,-3.125660) -- (6.100845,-3.706441) -- (3.893875,-3.706441) -- cycle;
\draw[opacity=1.000000, fill=none, line width=0.026399cm, colorDSJDGBELSSHKDDC] (3.893875,-3.125660) -- (6.100845,-3.125660) -- (6.100845,-3.706441) -- (3.893875,-3.706441) -- cycle;
\verbatimfont{\normalsize\fontDSJDGBECQNRIGQ}
\node[opacity=1.000000, above right, colorDSJDGBELSSHKDDC] at(4.044357, -3.550587) {\verb|indexEnd := 0|};
\draw[opacity=1.000000, fill=none, line width=0.026399 cm, colorDSJDGBELSSHKDDC] (4.997360,-2.756072) -- (4.997360,-3.125660);
\draw[fill=colorDSJDGBEMJTPLQ, draw=none, opacity=1.000000] (0.686378,-4.076030) -- (9.308342,-4.076030) -- (9.308342,-4.656811) -- (0.686378,-4.656811) -- cycle;
\draw[opacity=1.000000, fill=none, line width=0.026399cm, colorDSJDGBELSSHKDDC] (0.686378,-4.076030) -- (9.308342,-4.076030) -- (9.308342,-4.656811) -- (0.686378,-4.656811) -- cycle;
\draw[opacity=1.000000, fill=none, line width=0.026399cm, colorDSJDGBELSSHKDDC] (0.773495,-4.076030) -- (9.221225,-4.076030) -- (9.221225,-4.656811) -- (0.773495,-4.656811) -- cycle;
\verbatimfont{\normalsize\fontDSJDGBECQNRIGQ}
\node[opacity=1.000000, above right, colorDSJDGBELSSHKDDC] at(0.920625, -4.500956) {\verb|findLongestZeroSubsequence(buf, k, indexBegin, indexEnd)|};
\draw[opacity=1.000000, fill=none, line width=0.026399 cm, colorDSJDGBELSSHKDDC] (4.997360,-3.706441) -- (4.997360,-4.076030);
\draw[fill=colorDSJDGBEMJTPLQ, draw=none, opacity=1.000000] (1.935058,-5.026399) -- (8.059662,-5.026399) -- (8.059662,-5.607181) -- (1.935058,-5.607181) -- cycle;
\draw[opacity=1.000000, fill=none, line width=0.026399cm, colorDSJDGBELSSHKDDC] (1.935058,-5.026399) -- (8.059662,-5.026399) -- (8.059662,-5.607181) -- (1.935058,-5.607181) -- cycle;
\verbatimfont{\normalsize\fontDSJDGBECQNRIGQ}
\node[opacity=1.000000, above right, colorDSJDGBELSSHKDDC] at(2.085407, -5.451326) {\verb|subSequenceSize := indexEnd - indexBegin|};
\draw[opacity=1.000000, fill=none, line width=0.026399 cm, colorDSJDGBELSSHKDDC] (4.997360,-4.656811) -- (4.997360,-5.026399);
\draw[opacity=1.000000, fill=colorDSJDGBEMJTPLQ, draw=none] (2.988384, -6.981257) -- (4.997360, -7.985744) -- (7.006336, -6.981257) -- (4.997360, -5.976769) --cycle;
\draw[opacity=1.000000, fill=none, line width=0.026399cm, colorDSJDGBELSSHKDDC] (2.988384, -6.981257) -- (4.997360, -7.985744) -- (7.006336, -6.981257) -- (4.997360, -5.976769) --cycle;
\verbatimfont{\normalsize\fontDSJDGBECQNRIGQ}
\node[opacity=1.000000, above right, colorDSJDGBELSSHKDDC] at(3.673745, -6.944198) {\verb|subSequenceSize > |};
\verbatimfont{\normalsize\fontDSJDGBECQNRIGQ}
\node[opacity=1.000000, above right, colorDSJDGBELSSHKDDC] at(3.673745, -7.242409) {\verb|maxSubsequenceSize|};
\draw[opacity=1.000000, fill=none, line width=0.026399cm, colorDSJDGBELSSHKDDC] (2.988384, -6.981257) -- (2.397043, -6.981257) -- (2.397043, -8.355333);
\draw[fill=colorDSJDGBEMJTPLQ, draw=none, opacity=1.000000] (0.641499,-8.355333) -- (4.152587,-8.355333) -- (4.152587,-9.215945) -- (0.641499,-9.215945) -- cycle;
\draw[opacity=1.000000, fill=none, line width=0.026399cm, colorDSJDGBELSSHKDDC] (0.641499,-8.355333) -- (4.152587,-8.355333) -- (4.152587,-9.215945) -- (0.641499,-9.215945) -- cycle;
\verbatimfont{\normalsize\fontDSJDGBECQNRIGQ}
\node[opacity=1.000000, above right, colorDSJDGBELSSHKDDC] at(0.791057, -8.780259) {\verb|maxSubsequenceSize := |};
\verbatimfont{\normalsize\fontDSJDGBECQNRIGQ}
\node[opacity=1.000000, above right, colorDSJDGBELSSHKDDC] at(1.302855, -9.078470) {\verb|subSequenceSize|};
\draw[fill=colorDSJDGBEMJTPLQ, draw=none, opacity=1.000000] (0.060718,-9.585533) -- (4.733369,-9.585533) -- (4.733369,-10.446146) -- (0.060718,-10.446146) -- cycle;
\draw[opacity=1.000000, fill=none, line width=0.026399cm, colorDSJDGBELSSHKDDC] (0.060718,-9.585533) -- (4.733369,-9.585533) -- (4.733369,-10.446146) -- (0.060718,-10.446146) -- cycle;
\verbatimfont{\normalsize\fontDSJDGBECQNRIGQ}
\node[opacity=1.000000, above right, colorDSJDGBELSSHKDDC] at(0.208666, -10.010460) {\verb|longestZeroSubsequenceBegin = |};
\verbatimfont{\normalsize\fontDSJDGBECQNRIGQ}
\node[opacity=1.000000, above right, colorDSJDGBELSSHKDDC] at(0.720464, -10.308671) {\verb|{lineIndex, indexBegin}|};
\draw[opacity=1.000000, fill=none, line width=0.026399 cm, colorDSJDGBELSSHKDDC] (2.397043,-9.215945) -- (2.397043,-9.585533);
\draw[fill=colorDSJDGBEMJTPLQ, draw=none, opacity=1.000000] (0.131996,-10.815734) -- (4.662091,-10.815734) -- (4.662091,-11.676346) -- (0.131996,-11.676346) -- cycle;
\draw[opacity=1.000000, fill=none, line width=0.026399cm, colorDSJDGBELSSHKDDC] (0.131996,-10.815734) -- (4.662091,-10.815734) -- (4.662091,-11.676346) -- (0.131996,-11.676346) -- cycle;
\verbatimfont{\normalsize\fontDSJDGBECQNRIGQ}
\node[opacity=1.000000, above right, colorDSJDGBELSSHKDDC] at(0.279259, -11.240661) {\verb|longestZeroSubsequenceEnd := |};
\verbatimfont{\normalsize\fontDSJDGBECQNRIGQ}
\node[opacity=1.000000, above right, colorDSJDGBELSSHKDDC] at(0.579278, -11.538871) {\verb|{lineIndex, indexEnd - 1}|};
\draw[opacity=1.000000, fill=none, line width=0.026399 cm, colorDSJDGBELSSHKDDC] (2.397043,-10.446146) -- (2.397043,-10.815734);
\verbatimfont{\normalsize\fontDSJDGBECQNRIGQ}
\node[opacity=1.000000, above right, colorDSJDGBELSSHKDDC] at(2.397043, -6.822862) {\verb|+|};
\draw[opacity=1.000000, fill=none, line width=0.026399cm, colorDSJDGBELSSHKDDC] (7.006336, -6.981257) -- (7.270327, -6.981257) -- (7.270327, -8.355333);
\verbatimfont{\normalsize\fontDSJDGBECQNRIGQ}
\node[opacity=1.000000, above right, colorDSJDGBELSSHKDDC] at(7.093845, -6.822862) {\verb|-|};
\draw[opacity=1.000000, fill=none, line width=0.026399cm, colorDSJDGBELSSHKDDC] (7.270327, -8.355333) -- (7.270327, -12.045935) -- (4.997360, -12.045935);
\draw[opacity=1.000000, fill=none, line width=0.026399cm, colorDSJDGBELSSHKDDC] (2.397043, -11.676346) -- (2.397043, -12.045935) -- (4.997360, -12.045935);
\draw[opacity=1.000000, fill=none, line width=0.026399 cm, colorDSJDGBELSSHKDDC] (4.997360,-5.607181) -- (4.997360,-5.976769);
\draw[opacity=1.000000, rounded corners=0.570222 cm, fill=colorDSJDGBEMJTPLQ, draw=none] (1.884900,-12.415523) -- (8.109820,-12.415523) -- (8.109820,-13.555966) -- (1.884900,-13.555966) -- cycle;
\draw[opacity=1.000000, rounded corners=0.570222 cm, fill=none, line width=0.026399cm, colorDSJDGBELSSHKDDC] (1.884900,-12.415523) -- (8.109820,-12.415523) -- (8.109820,-13.555966) -- (1.884900,-13.555966) -- cycle;
\verbatimfont{\normalsize\fontDSJDGBECQNRIGQ}
\node[opacity=1.000000, above right, colorDSJDGBELSSHKDDC] at(2.456019, -12.840449) {\verb|Выход (longestZeroSubsequenceBegin,|};
\verbatimfont{\normalsize\fontDSJDGBECQNRIGQ}
\node[opacity=1.000000, above right, colorDSJDGBELSSHKDDC] at(3.109002, -13.138660) {\verb|longestZeroSubsequenceEnd,|};
\verbatimfont{\normalsize\fontDSJDGBECQNRIGQ}
\node[opacity=1.000000, above right, colorDSJDGBELSSHKDDC] at(3.603152, -13.436871) {\verb|maxSubsequenceSize)|};
\draw[opacity=1.000000, fill=none, line width=0.026399 cm, colorDSJDGBELSSHKDDC] (4.997360,-12.045935) -- (4.997360,-12.415523);
\end{tikzpicture}
% --------------------------

% Конец блок схемы
\end{document}}{Функция processLine}{ris:task6_6_processLine}\\
    \makenewfig{% Созданная блок схема работает только с компиляторами XeTeX и LuaTeX.
\documentclass{article}

% Необходимые зависимости
\usepackage[utf8]{inputenc}
\usepackage[english,russian]{babel}
\usepackage{pgfplots}
\usepackage{verbatim}
\usetikzlibrary{positioning}
\usetikzlibrary{shapes.geometric}
\usetikzlibrary{shapes.misc}
\usetikzlibrary{calc}
\usetikzlibrary{chains}
\usetikzlibrary{matrix}
\usetikzlibrary{decorations.text}
\usepackage{fontspec}
\usetikzlibrary{backgrounds}

\begin{document}
% Блок-схема
% Если Вы хотите добавить блок схему в свой документ, скопируйте код между комментариями
% С линияи и вставьте в документ.

% --------------------------
\begin{tikzpicture}[every node/.style={inner sep=0,outer sep=0}, background rectangle/.style={opacity=1.000000, fill=colorRQGNQNIDIOFKOJKD}, show background rectangle]
\makeatletter
\newcommand{\verbatimfont}[1]{\def\verbatim@font{#1}}
\makeatother
% Шрифты
\newfontfamily\fontRQGNQNIDLQJFNJCN[Scale=0.707247, SizeFeatures={Size=10.000000}]{Consolas}

% Цвета
\definecolor{colorRQGNQNIDIOFKOJKD}{rgb}{1.000000,1.000000,1.000000}
\definecolor{colorRQGNQNIDSEDCAA}{rgb}{0.949020,0.949020,0.949020}
\definecolor{colorRQGNQNIDQSTOCFLA}{rgb}{0.000000,0.000000,0.000000}
\draw[opacity=1.000000, rounded corners=0.406687 cm, fill=colorRQGNQNIDSEDCAA, draw=none] (-0.004990,0.004990) -- (10.000000,0.004990) -- (10.000000,-0.808383) -- (-0.004990,-0.808383) -- cycle;
\draw[opacity=1.000000, rounded corners=0.406687 cm, fill=none, line width=0.024950cm, colorRQGNQNIDQSTOCFLA] (-0.004990,0.004990) -- (10.000000,0.004990) -- (10.000000,-0.808383) -- (-0.004990,-0.808383) -- cycle;
\verbatimfont{\normalsize\fontRQGNQNIDLQJFNJCN}
\node[opacity=1.000000, above right, colorRQGNQNIDQSTOCFLA] at(0.391458, -0.412104) {\verb|processColumn(matrix, k, columnIndex, longestZeroSubsequenceBegin, |};
\verbatimfont{\normalsize\fontRQGNQNIDLQJFNJCN}
\node[opacity=1.000000, above right, colorRQGNQNIDQSTOCFLA] at(1.485615, -0.709437) {\verb|longestZeroSubsequenceEnd, maxSubsequenceSize, buf)|};
\draw[fill=colorRQGNQNIDSEDCAA, draw=none, opacity=1.000000] (2.158184,-1.157685) -- (7.836826,-1.157685) -- (7.836826,-1.706587) -- (2.158184,-1.706587) -- cycle;
\draw[opacity=1.000000, fill=none, line width=0.024950cm, colorRQGNQNIDQSTOCFLA] (2.158184,-1.157685) -- (7.836826,-1.157685) -- (7.836826,-1.706587) -- (2.158184,-1.706587) -- cycle;
\draw[opacity=1.000000, fill=none, line width=0.024950cm, colorRQGNQNIDQSTOCFLA] (2.240519,-1.157685) -- (7.754491,-1.157685) -- (7.754491,-1.706587) -- (2.240519,-1.706587) -- cycle;
\verbatimfont{\normalsize\fontRQGNQNIDLQJFNJCN}
\node[opacity=1.000000, above right, colorRQGNQNIDQSTOCFLA] at(2.385647, -1.574778) {\verb|getColumn(matrix, k, columnIndex, buf)|};
\draw[opacity=1.000000, fill=none, line width=0.024950 cm, colorRQGNQNIDQSTOCFLA] (4.997505,-0.808383) -- (4.997505,-1.157685);
\draw[fill=colorRQGNQNIDSEDCAA, draw=none, opacity=1.000000] (3.817365,-2.055888) -- (6.177645,-2.055888) -- (6.177645,-2.604790) -- (3.817365,-2.604790) -- cycle;
\draw[opacity=1.000000, fill=none, line width=0.024950cm, colorRQGNQNIDQSTOCFLA] (3.817365,-2.055888) -- (6.177645,-2.055888) -- (6.177645,-2.604790) -- (3.817365,-2.604790) -- cycle;
\verbatimfont{\normalsize\fontRQGNQNIDLQJFNJCN}
\node[opacity=1.000000, above right, colorRQGNQNIDQSTOCFLA] at(3.956291, -2.472982) {\verb|indexBegin := 0|};
\draw[opacity=1.000000, fill=none, line width=0.024950 cm, colorRQGNQNIDQSTOCFLA] (4.997505,-1.706587) -- (4.997505,-2.055888);
\draw[fill=colorRQGNQNIDSEDCAA, draw=none, opacity=1.000000] (3.954591,-2.954092) -- (6.040419,-2.954092) -- (6.040419,-3.502994) -- (3.954591,-3.502994) -- cycle;
\draw[opacity=1.000000, fill=none, line width=0.024950cm, colorRQGNQNIDQSTOCFLA] (3.954591,-2.954092) -- (6.040419,-2.954092) -- (6.040419,-3.502994) -- (3.954591,-3.502994) -- cycle;
\verbatimfont{\normalsize\fontRQGNQNIDLQJFNJCN}
\node[opacity=1.000000, above right, colorRQGNQNIDQSTOCFLA] at(4.097473, -3.371185) {\verb|indexEnd := 0|};
\draw[opacity=1.000000, fill=none, line width=0.024950 cm, colorRQGNQNIDQSTOCFLA] (4.997505,-2.604790) -- (4.997505,-2.954092);
\draw[fill=colorRQGNQNIDSEDCAA, draw=none, opacity=1.000000] (0.923154,-3.852295) -- (9.071856,-3.852295) -- (9.071856,-4.401198) -- (0.923154,-4.401198) -- cycle;
\draw[opacity=1.000000, fill=none, line width=0.024950cm, colorRQGNQNIDQSTOCFLA] (0.923154,-3.852295) -- (9.071856,-3.852295) -- (9.071856,-4.401198) -- (0.923154,-4.401198) -- cycle;
\draw[opacity=1.000000, fill=none, line width=0.024950cm, colorRQGNQNIDQSTOCFLA] (1.005489,-3.852295) -- (8.989521,-3.852295) -- (8.989521,-4.401198) -- (1.005489,-4.401198) -- cycle;
\verbatimfont{\normalsize\fontRQGNQNIDLQJFNJCN}
\node[opacity=1.000000, above right, colorRQGNQNIDQSTOCFLA] at(1.150308, -4.269389) {\verb|findLongestZeroSubsequence(buf, k, indexBegin, indexEnd)|};
\draw[opacity=1.000000, fill=none, line width=0.024950 cm, colorRQGNQNIDQSTOCFLA] (4.997505,-3.502994) -- (4.997505,-3.852295);
\draw[fill=colorRQGNQNIDSEDCAA, draw=none, opacity=1.000000] (2.103293,-4.750499) -- (7.891717,-4.750499) -- (7.891717,-5.299401) -- (2.103293,-5.299401) -- cycle;
\draw[opacity=1.000000, fill=none, line width=0.024950cm, colorRQGNQNIDQSTOCFLA] (2.103293,-4.750499) -- (7.891717,-4.750499) -- (7.891717,-5.299401) -- (2.103293,-5.299401) -- cycle;
\verbatimfont{\normalsize\fontRQGNQNIDLQJFNJCN}
\node[opacity=1.000000, above right, colorRQGNQNIDQSTOCFLA] at(2.244465, -5.167593) {\verb|subSequenceSize := indexEnd - indexBegin|};
\draw[opacity=1.000000, fill=none, line width=0.024950 cm, colorRQGNQNIDQSTOCFLA] (4.997505,-4.401198) -- (4.997505,-4.750499);
\draw[opacity=1.000000, fill=colorRQGNQNIDSEDCAA, draw=none] (3.098802, -6.598054) -- (4.997505, -7.547405) -- (6.896208, -6.598054) -- (4.997505, -5.648703) --cycle;
\draw[opacity=1.000000, fill=none, line width=0.024950cm, colorRQGNQNIDQSTOCFLA] (3.098802, -6.598054) -- (4.997505, -7.547405) -- (6.896208, -6.598054) -- (4.997505, -5.648703) --cycle;
\verbatimfont{\normalsize\fontRQGNQNIDLQJFNJCN}
\node[opacity=1.000000, above right, colorRQGNQNIDQSTOCFLA] at(3.762167, -6.578521) {\verb|subSequenceSize > |};
\verbatimfont{\normalsize\fontRQGNQNIDLQJFNJCN}
\node[opacity=1.000000, above right, colorRQGNQNIDQSTOCFLA] at(3.762167, -6.875854) {\verb|maxSubsequenceSize|};
\draw[opacity=1.000000, fill=none, line width=0.024950cm, colorRQGNQNIDQSTOCFLA] (3.098802, -6.598054) -- (2.539920, -6.598054) -- (2.539920, -7.896707);
\draw[fill=colorRQGNQNIDSEDCAA, draw=none, opacity=1.000000] (0.880739,-7.896707) -- (4.199102,-7.896707) -- (4.199102,-8.710080) -- (0.880739,-8.710080) -- cycle;
\draw[opacity=1.000000, fill=none, line width=0.024950cm, colorRQGNQNIDQSTOCFLA] (0.880739,-7.896707) -- (4.199102,-7.896707) -- (4.199102,-8.710080) -- (0.880739,-8.710080) -- cycle;
\verbatimfont{\normalsize\fontRQGNQNIDLQJFNJCN}
\node[opacity=1.000000, above right, colorRQGNQNIDQSTOCFLA] at(1.022219, -8.313800) {\verb|maxSubsequenceSize := |};
\verbatimfont{\normalsize\fontRQGNQNIDLQJFNJCN}
\node[opacity=1.000000, above right, colorRQGNQNIDQSTOCFLA] at(1.498706, -8.611133) {\verb|subSequenceSize|};
\draw[fill=colorRQGNQNIDSEDCAA, draw=none, opacity=1.000000] (0.331836,-9.059381) -- (4.748004,-9.059381) -- (4.748004,-9.872754) -- (0.331836,-9.872754) -- cycle;
\draw[opacity=1.000000, fill=none, line width=0.024950cm, colorRQGNQNIDQSTOCFLA] (0.331836,-9.059381) -- (4.748004,-9.059381) -- (4.748004,-9.872754) -- (0.331836,-9.872754) -- cycle;
\verbatimfont{\normalsize\fontRQGNQNIDLQJFNJCN}
\node[opacity=1.000000, above right, colorRQGNQNIDQSTOCFLA] at(0.475140, -9.476475) {\verb|longestZeroSubsequenceBegin = |};
\verbatimfont{\normalsize\fontRQGNQNIDLQJFNJCN}
\node[opacity=1.000000, above right, colorRQGNQNIDQSTOCFLA] at(0.810446, -9.773808) {\verb|{indexBegin, columnIndex}|};
\draw[opacity=1.000000, fill=none, line width=0.024950 cm, colorRQGNQNIDQSTOCFLA] (2.539920,-8.710080) -- (2.539920,-9.059381);
\draw[fill=colorRQGNQNIDSEDCAA, draw=none, opacity=1.000000] (0.399202,-10.222056) -- (4.680639,-10.222056) -- (4.680639,-11.035429) -- (0.399202,-11.035429) -- cycle;
\draw[opacity=1.000000, fill=none, line width=0.024950cm, colorRQGNQNIDQSTOCFLA] (0.399202,-10.222056) -- (4.680639,-10.222056) -- (4.680639,-11.035429) -- (0.399202,-11.035429) -- cycle;
\verbatimfont{\normalsize\fontRQGNQNIDLQJFNJCN}
\node[opacity=1.000000, above right, colorRQGNQNIDQSTOCFLA] at(0.545731, -10.639149) {\verb|longestZeroSubsequenceEnd := |};
\verbatimfont{\normalsize\fontRQGNQNIDLQJFNJCN}
\node[opacity=1.000000, above right, colorRQGNQNIDQSTOCFLA] at(0.686913, -10.936483) {\verb|{indexEnd - 1, columnIndex}|};
\draw[opacity=1.000000, fill=none, line width=0.024950 cm, colorRQGNQNIDQSTOCFLA] (2.539920,-9.872754) -- (2.539920,-10.222056);
\verbatimfont{\normalsize\fontRQGNQNIDLQJFNJCN}
\node[opacity=1.000000, above right, colorRQGNQNIDQSTOCFLA] at(2.539920, -6.448353) {\verb|+|};
\draw[opacity=1.000000, fill=none, line width=0.024950cm, colorRQGNQNIDQSTOCFLA] (6.896208, -6.598054) -- (7.145709, -6.598054) -- (7.145709, -7.896707);
\verbatimfont{\normalsize\fontRQGNQNIDLQJFNJCN}
\node[opacity=1.000000, above right, colorRQGNQNIDQSTOCFLA] at(7.004527, -6.448353) {\verb|-|};
\draw[opacity=1.000000, fill=none, line width=0.024950cm, colorRQGNQNIDQSTOCFLA] (7.145709, -7.896707) -- (7.145709, -11.384731) -- (4.997505, -11.384731);
\draw[opacity=1.000000, fill=none, line width=0.024950cm, colorRQGNQNIDQSTOCFLA] (2.539920, -11.035429) -- (2.539920, -11.384731) -- (4.997505, -11.384731);
\draw[opacity=1.000000, fill=none, line width=0.024950 cm, colorRQGNQNIDQSTOCFLA] (4.997505,-5.299401) -- (4.997505,-5.648703);
\draw[opacity=1.000000, rounded corners=0.538922 cm, fill=colorRQGNQNIDSEDCAA, draw=none] (2.055888,-11.734032) -- (7.939122,-11.734032) -- (7.939122,-12.811876) -- (2.055888,-12.811876) -- cycle;
\draw[opacity=1.000000, rounded corners=0.538922 cm, fill=none, line width=0.024950cm, colorRQGNQNIDQSTOCFLA] (2.055888,-11.734032) -- (7.939122,-11.734032) -- (7.939122,-12.811876) -- (2.055888,-12.811876) -- cycle;
\verbatimfont{\normalsize\fontRQGNQNIDLQJFNJCN}
\node[opacity=1.000000, above right, colorRQGNQNIDQSTOCFLA] at(2.579771, -12.151126) {\verb|Выход (longestZeroSubsequenceBegin,|};
\verbatimfont{\normalsize\fontRQGNQNIDLQJFNJCN}
\node[opacity=1.000000, above right, colorRQGNQNIDQSTOCFLA] at(3.197441, -12.448459) {\verb|longestZeroSubsequenceEnd,|};
\verbatimfont{\normalsize\fontRQGNQNIDLQJFNJCN}
\node[opacity=1.000000, above right, colorRQGNQNIDQSTOCFLA] at(3.691576, -12.745792) {\verb|maxSubsequenceSize)|};
\draw[opacity=1.000000, fill=none, line width=0.024950 cm, colorRQGNQNIDQSTOCFLA] (4.997505,-11.384731) -- (4.997505,-11.734032);
\end{tikzpicture}
% --------------------------

% Конец блок схемы
\end{document}}{Функция processColumn}{ris:task6_7_processColumn}\\
    \makenewfig{\input{dbg_texs/task6_8_processDiagonalTopLeftToBottomRight}}{Функция processDiagonalTopLeftToBottomRight}{ris:task6_8_processDiagonalTopLeftToBottomRight}\\
    \makenewfig{% Созданная блок схема работает только с компиляторами XeTeX и LuaTeX.
\documentclass{article}

% Необходимые зависимости
\usepackage[utf8]{inputenc}
\usepackage[english,russian]{babel}
\usepackage{pgfplots}
\usepackage{verbatim}
\usetikzlibrary{positioning}
\usetikzlibrary{shapes.geometric}
\usetikzlibrary{shapes.misc}
\usetikzlibrary{calc}
\usetikzlibrary{chains}
\usetikzlibrary{matrix}
\usetikzlibrary{decorations.text}
\usepackage{fontspec}
\usetikzlibrary{backgrounds}

\begin{document}
% Блок-схема
% Если Вы хотите добавить блок схему в свой документ, скопируйте код между комментариями
% С линияи и вставьте в документ.

% --------------------------
\begin{tikzpicture}[every node/.style={inner sep=0,outer sep=0}, background rectangle/.style={opacity=1.000000, fill=colorEGQKDLNETSDRSSI}, show background rectangle]
\makeatletter
\newcommand{\verbatimfont}[1]{\def\verbatim@font{#1}}
\makeatother
% Шрифты
\newfontfamily\fontEGQKDLNQGKIRER[Scale=0.579564, SizeFeatures={Size=10.000000}]{Consolas}

% Цвета
\definecolor{colorEGQKDLNETSDRSSI}{rgb}{1.000000,1.000000,1.000000}
\definecolor{colorEGQKDLNSNLBBA}{rgb}{0.949020,0.949020,0.949020}
\definecolor{colorEGQKDLNCECBPCTI}{rgb}{0.000000,0.000000,0.000000}
\draw[opacity=1.000000, rounded corners=0.333265 cm, fill=colorEGQKDLNSNLBBA, draw=none] (0.337354,0.004089) -- (10.000000,0.004089) -- (10.000000,-0.662441) -- (0.337354,-0.662441) -- cycle;
\draw[opacity=1.000000, rounded corners=0.333265 cm, fill=none, line width=0.020446cm, colorEGQKDLNCECBPCTI] (0.337354,0.004089) -- (10.000000,0.004089) -- (10.000000,-0.662441) -- (0.337354,-0.662441) -- cycle;
\verbatimfont{\normalsize\fontEGQKDLNQGKIRER}
\node[opacity=1.000000, above right, colorEGQKDLNCECBPCTI] at(1.621811, -0.318071) {\verb|processDiagonalTopRightToBottomLeft(matrix, n, k, row, column, |};
\verbatimfont{\normalsize\fontEGQKDLNQGKIRER}
\node[opacity=1.000000, above right, colorEGQKDLNCECBPCTI] at(0.668921, -0.542092) {\verb|longestZeroSubsequenceBegin, longestZeroSubsequenceEnd, maxSubsequenceSize, buf)|};
\draw[fill=colorEGQKDLNSNLBBA, draw=none, opacity=1.000000] (4.371294,-0.948681) -- (5.966060,-0.948681) -- (5.966060,-1.398487) -- (4.371294,-1.398487) -- cycle;
\draw[opacity=1.000000, fill=none, line width=0.020446cm, colorEGQKDLNCECBPCTI] (4.371294,-0.948681) -- (5.966060,-0.948681) -- (5.966060,-1.398487) -- (4.371294,-1.398487) -- cycle;
\verbatimfont{\normalsize\fontEGQKDLNQGKIRER}
\node[opacity=1.000000, above right, colorEGQKDLNCECBPCTI] at(4.480479, -1.270841) {\verb|bufSize := 0|};
\draw[opacity=1.000000, fill=none, line width=0.020446 cm, colorEGQKDLNCECBPCTI] (5.168677,-0.662441) -- (5.168677,-0.948681);
\draw[fill=colorEGQKDLNSNLBBA, draw=none, opacity=1.000000] (0.875077,-1.684727) -- (9.462278,-1.684727) -- (9.462278,-2.134533) -- (0.875077,-2.134533) -- cycle;
\draw[opacity=1.000000, fill=none, line width=0.020446cm, colorEGQKDLNCECBPCTI] (0.875077,-1.684727) -- (9.462278,-1.684727) -- (9.462278,-2.134533) -- (0.875077,-2.134533) -- cycle;
\draw[opacity=1.000000, fill=none, line width=0.020446cm, colorEGQKDLNCECBPCTI] (0.942548,-1.684727) -- (9.394807,-1.684727) -- (9.394807,-2.134533) -- (0.942548,-2.134533) -- cycle;
\verbatimfont{\normalsize\fontEGQKDLNQGKIRER}
\node[opacity=1.000000, above right, colorEGQKDLNCECBPCTI] at(1.057135, -2.006887) {\verb|getDiagonalTopRightToBottomLeft(matrix, n, k, row, column, buf, bufSize);|};
\draw[opacity=1.000000, fill=none, line width=0.020446 cm, colorEGQKDLNCECBPCTI] (5.168677,-1.398487) -- (5.168677,-1.684727);
\draw[fill=colorEGQKDLNSNLBBA, draw=none, opacity=1.000000] (4.201595,-2.420773) -- (6.135760,-2.420773) -- (6.135760,-2.870579) -- (4.201595,-2.870579) -- cycle;
\draw[opacity=1.000000, fill=none, line width=0.020446cm, colorEGQKDLNCECBPCTI] (4.201595,-2.420773) -- (6.135760,-2.420773) -- (6.135760,-2.870579) -- (4.201595,-2.870579) -- cycle;
\verbatimfont{\normalsize\fontEGQKDLNQGKIRER}
\node[opacity=1.000000, above right, colorEGQKDLNCECBPCTI] at(4.321664, -2.742933) {\verb|indexBegin := 0|};
\draw[opacity=1.000000, fill=none, line width=0.020446 cm, colorEGQKDLNCECBPCTI] (5.168677,-2.134533) -- (5.168677,-2.420773);
\draw[fill=colorEGQKDLNSNLBBA, draw=none, opacity=1.000000] (4.314046,-3.156819) -- (6.023308,-3.156819) -- (6.023308,-3.606624) -- (4.314046,-3.606624) -- cycle;
\draw[opacity=1.000000, fill=none, line width=0.020446cm, colorEGQKDLNCECBPCTI] (4.314046,-3.156819) -- (6.023308,-3.156819) -- (6.023308,-3.606624) -- (4.314046,-3.606624) -- cycle;
\verbatimfont{\normalsize\fontEGQKDLNQGKIRER}
\node[opacity=1.000000, above right, colorEGQKDLNCECBPCTI] at(4.427541, -3.478979) {\verb|indexEnd := 0|};
\draw[opacity=1.000000, fill=none, line width=0.020446 cm, colorEGQKDLNCECBPCTI] (5.168677,-2.870579) -- (5.168677,-3.156819);
\draw[fill=colorEGQKDLNSNLBBA, draw=none, opacity=1.000000] (1.492537,-3.892864) -- (8.844817,-3.892864) -- (8.844817,-4.342670) -- (1.492537,-4.342670) -- cycle;
\draw[opacity=1.000000, fill=none, line width=0.020446cm, colorEGQKDLNCECBPCTI] (1.492537,-3.892864) -- (8.844817,-3.892864) -- (8.844817,-4.342670) -- (1.492537,-4.342670) -- cycle;
\draw[opacity=1.000000, fill=none, line width=0.020446cm, colorEGQKDLNCECBPCTI] (1.560008,-3.892864) -- (8.777346,-3.892864) -- (8.777346,-4.342670) -- (1.560008,-4.342670) -- cycle;
\verbatimfont{\normalsize\fontEGQKDLNQGKIRER}
\node[opacity=1.000000, above right, colorEGQKDLNCECBPCTI] at(1.674749, -4.215025) {\verb|findLongestZeroSubsequence(buf, bufSize, indexBegin, indexEnd)|};
\draw[opacity=1.000000, fill=none, line width=0.020446 cm, colorEGQKDLNCECBPCTI] (5.168677,-3.606624) -- (5.168677,-3.892864);
\draw[fill=colorEGQKDLNSNLBBA, draw=none, opacity=1.000000] (2.796974,-4.628910) -- (7.540380,-4.628910) -- (7.540380,-5.078716) -- (2.796974,-5.078716) -- cycle;
\draw[opacity=1.000000, fill=none, line width=0.020446cm, colorEGQKDLNCECBPCTI] (2.796974,-4.628910) -- (7.540380,-4.628910) -- (7.540380,-5.078716) -- (2.796974,-5.078716) -- cycle;
\verbatimfont{\normalsize\fontEGQKDLNQGKIRER}
\node[opacity=1.000000, above right, colorEGQKDLNCECBPCTI] at(2.909976, -4.951070) {\verb|subSequenceSize := indexEnd - indexBegin|};
\draw[opacity=1.000000, fill=none, line width=0.020446 cm, colorEGQKDLNCECBPCTI] (5.168677,-4.342670) -- (5.168677,-4.628910);
\draw[opacity=1.000000, fill=colorEGQKDLNSNLBBA, draw=none] (3.612758, -6.142916) -- (5.168677, -6.920875) -- (6.724596, -6.142916) -- (5.168677, -5.364956) --cycle;
\draw[opacity=1.000000, fill=none, line width=0.020446cm, colorEGQKDLNCECBPCTI] (3.612758, -6.142916) -- (5.168677, -6.920875) -- (6.724596, -6.142916) -- (5.168677, -5.364956) --cycle;
\verbatimfont{\normalsize\fontEGQKDLNQGKIRER}
\node[opacity=1.000000, above right, colorEGQKDLNCECBPCTI] at(4.145203, -6.107276) {\verb|subSequenceSize > |};
\verbatimfont{\normalsize\fontEGQKDLNQGKIRER}
\node[opacity=1.000000, above right, colorEGQKDLNCECBPCTI] at(4.145203, -6.331296) {\verb|maxSubsequenceSize|};
\draw[opacity=1.000000, fill=none, line width=0.020446cm, colorEGQKDLNCECBPCTI] (3.612758, -6.142916) -- (2.480065, -6.142916) -- (2.480065, -7.207115);
\draw[fill=colorEGQKDLNSNLBBA, draw=none, opacity=1.000000] (1.120425,-7.207115) -- (3.839706,-7.207115) -- (3.839706,-7.873645) -- (1.120425,-7.873645) -- cycle;
\draw[opacity=1.000000, fill=none, line width=0.020446cm, colorEGQKDLNCECBPCTI] (1.120425,-7.207115) -- (3.839706,-7.207115) -- (3.839706,-7.873645) -- (1.120425,-7.873645) -- cycle;
\verbatimfont{\normalsize\fontEGQKDLNQGKIRER}
\node[opacity=1.000000, above right, colorEGQKDLNCECBPCTI] at(1.227192, -7.529275) {\verb|maxSubsequenceSize := |};
\verbatimfont{\normalsize\fontEGQKDLNQGKIRER}
\node[opacity=1.000000, above right, colorEGQKDLNCECBPCTI] at(1.633053, -7.753296) {\verb|subSequenceSize|};
\draw[fill=colorEGQKDLNSNLBBA, draw=none, opacity=1.000000] (0.163566,-8.159886) -- (4.796565,-8.159886) -- (4.796565,-8.826416) -- (0.163566,-8.826416) -- cycle;
\draw[opacity=1.000000, fill=none, line width=0.020446cm, colorEGQKDLNCECBPCTI] (0.163566,-8.159886) -- (4.796565,-8.159886) -- (4.796565,-8.826416) -- (0.163566,-8.826416) -- cycle;
\verbatimfont{\normalsize\fontEGQKDLNQGKIRER}
\node[opacity=1.000000, above right, colorEGQKDLNCECBPCTI] at(0.786040, -8.482046) {\verb|longestZeroSubsequenceBegin = |};
\verbatimfont{\normalsize\fontEGQKDLNQGKIRER}
\node[opacity=1.000000, above right, colorEGQKDLNCECBPCTI] at(0.274303, -8.706066) {\verb|{row + indexBegin, column - indexBegin}|};
\draw[opacity=1.000000, fill=none, line width=0.020446 cm, colorEGQKDLNCECBPCTI] (2.480065,-7.873645) -- (2.480065,-8.159886);
\draw[fill=colorEGQKDLNSNLBBA, draw=none, opacity=1.000000] (-0.004089,-9.112656) -- (4.964220,-9.112656) -- (4.964220,-9.995911) -- (-0.004089,-9.995911) -- cycle;
\draw[opacity=1.000000, fill=none, line width=0.020446cm, colorEGQKDLNCECBPCTI] (-0.004089,-9.112656) -- (4.964220,-9.112656) -- (4.964220,-9.995911) -- (-0.004089,-9.995911) -- cycle;
\verbatimfont{\normalsize\fontEGQKDLNQGKIRER}
\node[opacity=1.000000, above right, colorEGQKDLNCECBPCTI] at(0.838978, -9.434816) {\verb|longestZeroSubsequenceEnd := |};
\verbatimfont{\normalsize\fontEGQKDLNQGKIRER}
\node[opacity=1.000000, above right, colorEGQKDLNCECBPCTI] at(0.168426, -9.658837) {\verb|{row + indexBegin + subSequenceSize - 1, |};
\verbatimfont{\normalsize\fontEGQKDLNQGKIRER}
\node[opacity=1.000000, above right, colorEGQKDLNCECBPCTI] at(0.115488, -9.882857) {\verb|column - indexBegin - subSequenceSize + 1}|};
\draw[opacity=1.000000, fill=none, line width=0.020446 cm, colorEGQKDLNCECBPCTI] (2.480065,-8.826416) -- (2.480065,-9.112656);
\verbatimfont{\normalsize\fontEGQKDLNQGKIRER}
\node[opacity=1.000000, above right, colorEGQKDLNCECBPCTI] at(2.480065, -6.020241) {\verb|+|};
\draw[opacity=1.000000, fill=none, line width=0.020446cm, colorEGQKDLNCECBPCTI] (6.724596, -6.142916) -- (6.929053, -6.142916) -- (6.929053, -7.207115);
\verbatimfont{\normalsize\fontEGQKDLNQGKIRER}
\node[opacity=1.000000, above right, colorEGQKDLNCECBPCTI] at(6.787885, -6.020241) {\verb|-|};
\draw[opacity=1.000000, fill=none, line width=0.020446cm, colorEGQKDLNCECBPCTI] (6.929053, -7.207115) -- (6.929053, -10.282151) -- (5.168677, -10.282151);
\draw[opacity=1.000000, fill=none, line width=0.020446cm, colorEGQKDLNCECBPCTI] (2.480065, -9.995911) -- (2.480065, -10.282151) -- (5.168677, -10.282151);
\draw[opacity=1.000000, fill=none, line width=0.020446 cm, colorEGQKDLNCECBPCTI] (5.168677,-5.078716) -- (5.168677,-5.364956);
\draw[opacity=1.000000, rounded corners=0.441627 cm, fill=colorEGQKDLNSNLBBA, draw=none] (2.758127,-10.568391) -- (7.579227,-10.568391) -- (7.579227,-11.451646) -- (2.758127,-11.451646) -- cycle;
\draw[opacity=1.000000, rounded corners=0.441627 cm, fill=none, line width=0.020446cm, colorEGQKDLNCECBPCTI] (2.758127,-10.568391) -- (7.579227,-10.568391) -- (7.579227,-11.451646) -- (2.758127,-11.451646) -- cycle;
\verbatimfont{\normalsize\fontEGQKDLNQGKIRER}
\node[opacity=1.000000, above right, colorEGQKDLNCECBPCTI] at(3.192314, -10.890551) {\verb|Выход (longestZeroSubsequenceBegin,|};
\verbatimfont{\normalsize\fontEGQKDLNQGKIRER}
\node[opacity=1.000000, above right, colorEGQKDLNCECBPCTI] at(3.704051, -11.114572) {\verb|longestZeroSubsequenceEnd,|};
\verbatimfont{\normalsize\fontEGQKDLNQGKIRER}
\node[opacity=1.000000, above right, colorEGQKDLNCECBPCTI] at(4.092265, -11.338592) {\verb|maxSubsequenceSize)|};
\draw[opacity=1.000000, fill=none, line width=0.020446 cm, colorEGQKDLNCECBPCTI] (5.168677,-10.282151) -- (5.168677,-10.568391);
\end{tikzpicture}
% --------------------------

% Конец блок схемы
\end{document}}{Функция processDiagonalTopRightToBottomLeft}{ris:task6_9_processDiagonalTopRightToBottomLeft}\\
    \makenewfig{% Созданная блок схема работает только с компиляторами XeTeX и LuaTeX.
\documentclass{article}

% Необходимые зависимости
\usepackage[utf8]{inputenc}
\usepackage[english,russian]{babel}
\usepackage{pgfplots}
\usepackage{verbatim}
\usetikzlibrary{positioning}
\usetikzlibrary{shapes.geometric}
\usetikzlibrary{shapes.misc}
\usetikzlibrary{calc}
\usetikzlibrary{chains}
\usetikzlibrary{matrix}
\usetikzlibrary{decorations.text}
\usepackage{fontspec}
\usetikzlibrary{backgrounds}

\begin{document}
% Блок-схема
% Если Вы хотите добавить блок схему в свой документ, скопируйте код между комментариями
% С линияи и вставьте в документ.

% --------------------------
\begin{tikzpicture}[every node/.style={inner sep=0,outer sep=0}, background rectangle/.style={opacity=1.000000, fill=colorTBMQPDOQOCFAFR}, show background rectangle]
\makeatletter
\newcommand{\verbatimfont}[1]{\def\verbatim@font{#1}}
\makeatother
% Шрифты
\newfontfamily\fontTBMQPDOKSSIBMS[Scale=0.858984, SizeFeatures={Size=10.000000}]{Consolas}

% Цвета
\definecolor{colorTBMQPDOQOCFAFR}{rgb}{1.000000,1.000000,1.000000}
\definecolor{colorTBMQPDORLMSCMPH}{rgb}{0.949020,0.949020,0.949020}
\definecolor{colorTBMQPDONLTLPPG}{rgb}{0.000000,0.000000,0.000000}
\draw[opacity=1.000000, rounded corners=0.333333 cm, fill=colorTBMQPDORLMSCMPH, draw=none] (1.163636,0.006061) -- (3.830303,0.006061) -- (3.830303,-0.660606) -- (1.163636,-0.660606) -- cycle;
\draw[opacity=1.000000, rounded corners=0.333333 cm, fill=none, line width=0.030303cm, colorTBMQPDONLTLPPG] (1.163636,0.006061) -- (3.830303,0.006061) -- (3.830303,-0.660606) -- (1.163636,-0.660606) -- cycle;
\verbatimfont{\normalsize\fontTBMQPDOKSSIBMS}
\node[opacity=1.000000, above right, colorTBMQPDONLTLPPG] at(1.737578, -0.475461) {\verb|max(a, b)|};
\draw[opacity=1.000000, fill=colorTBMQPDORLMSCMPH, draw=none] (1.593939, -1.536364) -- (2.496970, -1.987879) -- (3.400000, -1.536364) -- (2.496970, -1.084848) --cycle;
\draw[opacity=1.000000, fill=none, line width=0.030303cm, colorTBMQPDONLTLPPG] (1.593939, -1.536364) -- (2.496970, -1.987879) -- (3.400000, -1.536364) -- (2.496970, -1.084848) --cycle;
\verbatimfont{\normalsize\fontTBMQPDOKSSIBMS}
\node[opacity=1.000000, above right, colorTBMQPDONLTLPPG] at(2.073123, -1.648188) {\verb|a > b|};
\draw[opacity=1.000000, fill=none, line width=0.030303cm, colorTBMQPDONLTLPPG] (1.593939, -1.536364) -- (1.093939, -1.536364) -- (1.093939, -2.412121);
\draw[fill=colorTBMQPDORLMSCMPH, draw=none, opacity=1.000000] (-0.006061,-2.412121) -- (2.193939,-2.412121) -- (2.193939,-3.078788) -- (-0.006061,-3.078788) -- cycle;
\draw[opacity=1.000000, fill=none, line width=0.030303cm, colorTBMQPDONLTLPPG] (-0.006061,-2.412121) -- (2.193939,-2.412121) -- (2.193939,-3.078788) -- (-0.006061,-3.078788) -- cycle;
\verbatimfont{\normalsize\fontTBMQPDOKSSIBMS}
\node[opacity=1.000000, above right, colorTBMQPDONLTLPPG] at(0.175605, -2.893642) {\verb|result := a|};
\verbatimfont{\normalsize\fontTBMQPDOKSSIBMS}
\node[opacity=1.000000, above right, colorTBMQPDONLTLPPG] at(1.093939, -1.354545) {\verb|+|};
\draw[opacity=1.000000, fill=none, line width=0.030303cm, colorTBMQPDONLTLPPG] (3.400000, -1.536364) -- (3.900000, -1.536364) -- (3.900000, -2.412121);
\verbatimfont{\normalsize\fontTBMQPDOKSSIBMS}
\node[opacity=1.000000, above right, colorTBMQPDONLTLPPG] at(3.723397, -1.354545) {\verb|-|};
\draw[fill=colorTBMQPDORLMSCMPH, draw=none, opacity=1.000000] (2.800000,-2.412121) -- (5.000000,-2.412121) -- (5.000000,-3.078788) -- (2.800000,-3.078788) -- cycle;
\draw[opacity=1.000000, fill=none, line width=0.030303cm, colorTBMQPDONLTLPPG] (2.800000,-2.412121) -- (5.000000,-2.412121) -- (5.000000,-3.078788) -- (2.800000,-3.078788) -- cycle;
\verbatimfont{\normalsize\fontTBMQPDOKSSIBMS}
\node[opacity=1.000000, above right, colorTBMQPDONLTLPPG] at(2.981666, -2.893642) {\verb|result := b|};
\draw[opacity=1.000000, fill=none, line width=0.030303cm, colorTBMQPDONLTLPPG] (3.900000, -3.078788) -- (3.900000, -3.503030) -- (2.496970, -3.503030);
\draw[opacity=1.000000, fill=none, line width=0.030303cm, colorTBMQPDONLTLPPG] (1.093939, -3.078788) -- (1.093939, -3.503030) -- (2.496970, -3.503030);
\draw[opacity=1.000000, fill=none, line width=0.030303 cm, colorTBMQPDONLTLPPG] (2.496970,-0.660606) -- (2.496970,-1.084848);
\draw[opacity=1.000000, rounded corners=0.333333 cm, fill=colorTBMQPDORLMSCMPH, draw=none] (0.996970,-3.927273) -- (3.996970,-3.927273) -- (3.996970,-4.593939) -- (0.996970,-4.593939) -- cycle;
\draw[opacity=1.000000, rounded corners=0.333333 cm, fill=none, line width=0.030303cm, colorTBMQPDONLTLPPG] (0.996970,-3.927273) -- (3.996970,-3.927273) -- (3.996970,-4.593939) -- (0.996970,-4.593939) -- cycle;
\verbatimfont{\normalsize\fontTBMQPDOKSSIBMS}
\node[opacity=1.000000, above right, colorTBMQPDONLTLPPG] at(1.313732, -4.408794) {\verb|Вернуть result|};
\draw[opacity=1.000000, fill=none, line width=0.030303 cm, colorTBMQPDONLTLPPG] (2.496970,-3.503030) -- (2.496970,-3.927273);
\end{tikzpicture}
% --------------------------

% Конец блок схемы
\end{document}}{Функция max}{ris:task6_10_max}\\
    \makenewfig{% Созданная блок схема работает только с компиляторами XeTeX и LuaTeX.
\documentclass{article}

% Необходимые зависимости
\usepackage[utf8]{inputenc}
\usepackage[english,russian]{babel}
\usepackage{pgfplots}
\usepackage{verbatim}
\usetikzlibrary{positioning}
\usetikzlibrary{shapes.geometric}
\usetikzlibrary{shapes.misc}
\usetikzlibrary{calc}
\usetikzlibrary{chains}
\usetikzlibrary{matrix}
\usetikzlibrary{decorations.text}
\usepackage{fontspec}
\usetikzlibrary{backgrounds}

\begin{document}
% Блок-схема
% Если Вы хотите добавить блок схему в свой документ, скопируйте код между комментариями
% С линияи и вставьте в документ.

% --------------------------
\begin{tikzpicture}[every node/.style={inner sep=0,outer sep=0}, background rectangle/.style={opacity=1.000000, fill=colorQGKTDBLJCGLAT}, show background rectangle]
\makeatletter
\newcommand{\verbatimfont}[1]{\def\verbatim@font{#1}}
\makeatother
% Шрифты
\newfontfamily\fontQGKTDBGSMHCRBD[Scale=0.871580, SizeFeatures={Size=10.000000}]{Consolas}

% Цвета
\definecolor{colorQGKTDBLJCGLAT}{rgb}{1.000000,1.000000,1.000000}
\definecolor{colorQGKTDBPKRQMG}{rgb}{0.949020,0.949020,0.949020}
\definecolor{colorQGKTDBAMEKTCJQ}{rgb}{0.000000,0.000000,0.000000}
\draw[opacity=1.000000, rounded corners=0.501183 cm, fill=colorQGKTDBPKRQMG, draw=none] (-0.006149,0.006149) -- (13.000000,0.006149) -- (13.000000,-0.996216) -- (-0.006149,-0.996216) -- cycle;
\draw[opacity=1.000000, rounded corners=0.501183 cm, fill=none, line width=0.030747cm, colorQGKTDBAMEKTCJQ] (-0.006149,0.006149) -- (13.000000,0.006149) -- (13.000000,-0.996216) -- (-0.006149,-0.996216) -- cycle;
\verbatimfont{\normalsize\fontQGKTDBGSMHCRBD}
\node[opacity=1.000000, above right, colorQGKTDBAMEKTCJQ] at(0.479219, -0.477537) {\verb|findLongestZeroLineInMatrix(matrix, n, k, longestZeroSubsequenceBegin, |};
\verbatimfont{\normalsize\fontQGKTDBGSMHCRBD}
\node[opacity=1.000000, above right, colorQGKTDBAMEKTCJQ] at(2.596887, -0.813635) {\verb|longestZeroSubsequenceEnd, maxSubsequenceSize)|};
\draw[fill=colorQGKTDBPKRQMG, draw=none, opacity=1.000000] (2.911779,-1.426679) -- (10.082072,-1.426679) -- (10.082072,-2.103122) -- (2.911779,-2.103122) -- cycle;
\draw[opacity=1.000000, fill=none, line width=0.030747cm, colorQGKTDBAMEKTCJQ] (2.911779,-1.426679) -- (10.082072,-1.426679) -- (10.082072,-2.103122) -- (2.911779,-2.103122) -- cycle;
\draw[opacity=1.000000, fill=none, line width=0.030747cm, colorQGKTDBAMEKTCJQ] (3.013245,-1.426679) -- (9.980605,-1.426679) -- (9.980605,-2.103122) -- (3.013245,-2.103122) -- cycle;
\verbatimfont{\normalsize\fontQGKTDBGSMHCRBD}
\node[opacity=1.000000, above right, colorQGKTDBAMEKTCJQ] at(3.196893, -1.910365) {\verb|buf := malloc(sizeof(bool) * max(n, k))|};
\draw[opacity=1.000000, fill=none, line width=0.030747 cm, colorQGKTDBAMEKTCJQ] (6.496925,-0.996216) -- (6.496925,-1.426679);
\draw[fill=colorQGKTDBPKRQMG, draw=none, opacity=1.000000] (4.366131,-2.533586) -- (8.627720,-2.533586) -- (8.627720,-3.210028) -- (4.366131,-3.210028) -- cycle;
\draw[opacity=1.000000, fill=none, line width=0.030747cm, colorQGKTDBAMEKTCJQ] (4.366131,-2.533586) -- (8.627720,-2.533586) -- (8.627720,-3.210028) -- (4.366131,-3.210028) -- cycle;
\verbatimfont{\normalsize\fontQGKTDBGSMHCRBD}
\node[opacity=1.000000, above right, colorQGKTDBAMEKTCJQ] at(4.538082, -3.017272) {\verb|maxSubsequenceSize := 0|};
\draw[opacity=1.000000, fill=none, line width=0.030747 cm, colorQGKTDBAMEKTCJQ] (6.496925,-2.103122) -- (6.496925,-2.533586);
\draw[fill=colorQGKTDBPKRQMG, draw=none, opacity=1.000000] (3.182356,-3.640492) -- (9.811495,-3.640492) -- (9.811495,-4.316935) -- (3.182356,-4.316935) -- cycle;
\draw[opacity=1.000000, fill=none, line width=0.030747cm, colorQGKTDBAMEKTCJQ] (3.182356,-3.640492) -- (9.811495,-3.640492) -- (9.811495,-4.316935) -- (3.182356,-4.316935) -- cycle;
\verbatimfont{\normalsize\fontQGKTDBGSMHCRBD}
\node[opacity=1.000000, above right, colorQGKTDBAMEKTCJQ] at(3.355718, -4.124178) {\verb|longestZeroSubsequenceBegin := {0, 0}|};
\draw[opacity=1.000000, fill=none, line width=0.030747 cm, colorQGKTDBAMEKTCJQ] (6.496925,-3.210028) -- (6.496925,-3.640492);
\draw[fill=colorQGKTDBPKRQMG, draw=none, opacity=1.000000] (3.351466,-4.747398) -- (9.642384,-4.747398) -- (9.642384,-5.423841) -- (3.351466,-5.423841) -- cycle;
\draw[opacity=1.000000, fill=none, line width=0.030747cm, colorQGKTDBAMEKTCJQ] (3.351466,-4.747398) -- (9.642384,-4.747398) -- (9.642384,-5.423841) -- (3.351466,-5.423841) -- cycle;
\verbatimfont{\normalsize\fontQGKTDBGSMHCRBD}
\node[opacity=1.000000, above right, colorQGKTDBAMEKTCJQ] at(3.532190, -5.231084) {\verb|longestZeroSubsequenceEnd := {0, 0}|};
\draw[opacity=1.000000, fill=none, line width=0.030747 cm, colorQGKTDBAMEKTCJQ] (6.496925,-4.316935) -- (6.496925,-4.747398);
\draw[opacity=1.000000, fill=colorQGKTDBPKRQMG, draw=none] (3.926443, -7.139546) -- (5.211684, -5.854305) -- (7.782167, -5.854305) -- (9.067408, -7.139546) -- (7.782167, -8.424787) -- (5.211684, -8.424787) --cycle;
\draw[opacity=1.000000, fill=none, line width=0.030747cm, colorQGKTDBAMEKTCJQ] (3.926443, -7.139546) -- (5.211684, -5.854305) -- (7.782167, -5.854305) -- (9.067408, -7.139546) -- (7.782167, -8.424787) -- (5.211684, -8.424787) --cycle;
\verbatimfont{\normalsize\fontQGKTDBGSMHCRBD}
\node[opacity=1.000000, above right, colorQGKTDBAMEKTCJQ] at(5.385150, -7.211217) {\verb|i := 0, n - 1|};
\draw[fill=colorQGKTDBPKRQMG, draw=none, opacity=1.000000] (1.513079,-8.855251) -- (11.480771,-8.855251) -- (11.480771,-9.857616) -- (1.513079,-9.857616) -- cycle;
\draw[opacity=1.000000, fill=none, line width=0.030747cm, colorQGKTDBAMEKTCJQ] (1.513079,-8.855251) -- (11.480771,-8.855251) -- (11.480771,-9.857616) -- (1.513079,-9.857616) -- cycle;
\draw[opacity=1.000000, fill=none, line width=0.030747cm, colorQGKTDBAMEKTCJQ] (1.663434,-8.855251) -- (11.330416,-8.855251) -- (11.330416,-9.857616) -- (1.663434,-9.857616) -- cycle;
\verbatimfont{\normalsize\fontQGKTDBGSMHCRBD}
\node[opacity=1.000000, above right, colorQGKTDBAMEKTCJQ] at(1.838056, -9.338937) {\verb|processLine(matrix, k, i, longestZeroSubsequenceBegin, |};
\verbatimfont{\normalsize\fontQGKTDBGSMHCRBD}
\node[opacity=1.000000, above right, colorQGKTDBAMEKTCJQ] at(2.173353, -9.675035) {\verb|longestZeroSubsequenceEnd, maxSubsequenceSize, buf)|};
\draw[fill=colorQGKTDBPKRQMG, draw=none, opacity=1.000000] (1.799338,-10.288079) -- (11.194513,-10.288079) -- (11.194513,-11.616367) -- (1.799338,-11.616367) -- cycle;
\draw[opacity=1.000000, fill=none, line width=0.030747cm, colorQGKTDBAMEKTCJQ] (1.799338,-10.288079) -- (11.194513,-10.288079) -- (11.194513,-11.616367) -- (1.799338,-11.616367) -- cycle;
\draw[opacity=1.000000, fill=none, line width=0.030747cm, colorQGKTDBAMEKTCJQ] (1.998581,-10.288079) -- (10.995270,-10.288079) -- (10.995270,-11.616367) -- (1.998581,-11.616367) -- cycle;
\verbatimfont{\normalsize\fontQGKTDBGSMHCRBD}
\node[opacity=1.000000, above right, colorQGKTDBAMEKTCJQ] at(3.443954, -10.771766) {\verb|processDiagonalTopLeftToBottomRight(|};
\verbatimfont{\normalsize\fontQGKTDBGSMHCRBD}
\node[opacity=1.000000, above right, colorQGKTDBAMEKTCJQ] at(2.349826, -11.107864) {\verb|matrix, n, k, i, 0, longestZeroSubsequenceBegin, |};
\verbatimfont{\normalsize\fontQGKTDBGSMHCRBD}
\node[opacity=1.000000, above right, colorQGKTDBAMEKTCJQ] at(2.173353, -11.443963) {\verb|longestZeroSubsequenceEnd, maxSubsequenceSize, buf)|};
\draw[opacity=1.000000, fill=none, line width=0.030747 cm, colorQGKTDBAMEKTCJQ] (6.496925,-9.857616) -- (6.496925,-10.288079);
\draw[fill=colorQGKTDBPKRQMG, draw=none, opacity=1.000000] (1.716320,-12.046831) -- (11.277531,-12.046831) -- (11.277531,-13.375118) -- (1.716320,-13.375118) -- cycle;
\draw[opacity=1.000000, fill=none, line width=0.030747cm, colorQGKTDBAMEKTCJQ] (1.716320,-12.046831) -- (11.277531,-12.046831) -- (11.277531,-13.375118) -- (1.716320,-13.375118) -- cycle;
\draw[opacity=1.000000, fill=none, line width=0.030747cm, colorQGKTDBAMEKTCJQ] (1.915563,-12.046831) -- (11.078288,-12.046831) -- (11.078288,-13.375118) -- (1.915563,-13.375118) -- cycle;
\verbatimfont{\normalsize\fontQGKTDBGSMHCRBD}
\node[opacity=1.000000, above right, colorQGKTDBAMEKTCJQ] at(3.443954, -12.530517) {\verb|processDiagonalTopRightToBottomLeft(|};
\verbatimfont{\normalsize\fontQGKTDBGSMHCRBD}
\node[opacity=1.000000, above right, colorQGKTDBAMEKTCJQ] at(2.085117, -12.866615) {\verb|matrix, n, k, i, k - 1, longestZeroSubsequenceBegin,|};
\verbatimfont{\normalsize\fontQGKTDBGSMHCRBD}
\node[opacity=1.000000, above right, colorQGKTDBAMEKTCJQ] at(2.173353, -13.202714) {\verb|longestZeroSubsequenceEnd, maxSubsequenceSize, buf)|};
\draw[opacity=1.000000, fill=none, line width=0.030747 cm, colorQGKTDBAMEKTCJQ] (6.496925,-11.616367) -- (6.496925,-12.046831);
\draw[opacity=1.000000, fill=none, line width=0.030747 cm, colorQGKTDBAMEKTCJQ] (6.496925,-8.424787) -- (6.496925,-8.855251);
\verbatimfont{\normalsize\fontQGKTDBGSMHCRBD}
\node[opacity=1.000000, above right, colorQGKTDBAMEKTCJQ] at(6.302004, -8.742437) {\verb|+|};
\draw[opacity=1.000000, fill=none, line width=0.030747cm, colorQGKTDBAMEKTCJQ] (6.496925, -13.375118) -- (6.496925, -13.805582) -- (1.205605, -13.805582) -- (1.205605, -7.139546) -- (3.926443, -7.139546);
\verbatimfont{\normalsize\fontQGKTDBGSMHCRBD}
\node[opacity=1.000000, above right, colorQGKTDBAMEKTCJQ] at(11.611773, -6.955061) {\verb|-|};
\draw[opacity=1.000000, fill=none, line width=0.030747cm, colorQGKTDBAMEKTCJQ] (9.067408, -7.139546) -- (11.788245, -7.139546) -- (11.788245, -14.113056) -- (6.496925, -14.113056);
\draw[opacity=1.000000, fill=none, line width=0.030747 cm, colorQGKTDBAMEKTCJQ] (6.496925,-5.423841) -- (6.496925,-5.854305);
\draw[opacity=1.000000, fill=colorQGKTDBPKRQMG, draw=none] (3.926443, -15.828761) -- (5.211684, -14.543519) -- (7.782167, -14.543519) -- (9.067408, -15.828761) -- (7.782167, -17.114002) -- (5.211684, -17.114002) --cycle;
\draw[opacity=1.000000, fill=none, line width=0.030747cm, colorQGKTDBAMEKTCJQ] (3.926443, -15.828761) -- (5.211684, -14.543519) -- (7.782167, -14.543519) -- (9.067408, -15.828761) -- (7.782167, -17.114002) -- (5.211684, -17.114002) --cycle;
\verbatimfont{\normalsize\fontQGKTDBGSMHCRBD}
\node[opacity=1.000000, above right, colorQGKTDBAMEKTCJQ] at(5.385150, -15.900432) {\verb|i := 0, k - 1|};
\draw[fill=colorQGKTDBPKRQMG, draw=none, opacity=1.000000] (1.343969,-17.544465) -- (11.649882,-17.544465) -- (11.649882,-18.546831) -- (1.343969,-18.546831) -- cycle;
\draw[opacity=1.000000, fill=none, line width=0.030747cm, colorQGKTDBAMEKTCJQ] (1.343969,-17.544465) -- (11.649882,-17.544465) -- (11.649882,-18.546831) -- (1.343969,-18.546831) -- cycle;
\draw[opacity=1.000000, fill=none, line width=0.030747cm, colorQGKTDBAMEKTCJQ] (1.494324,-17.544465) -- (11.499527,-17.544465) -- (11.499527,-18.546831) -- (1.494324,-18.546831) -- cycle;
\verbatimfont{\normalsize\fontQGKTDBGSMHCRBD}
\node[opacity=1.000000, above right, colorQGKTDBAMEKTCJQ] at(1.661583, -18.028152) {\verb|processColumn(matrix, n, i, longestZeroSubsequenceBegin, |};
\verbatimfont{\normalsize\fontQGKTDBGSMHCRBD}
\node[opacity=1.000000, above right, colorQGKTDBAMEKTCJQ] at(2.173353, -18.364250) {\verb|longestZeroSubsequenceEnd, maxSubsequenceSize, buf)|};
\draw[fill=colorQGKTDBPKRQMG, draw=none, opacity=1.000000] (1.799338,-18.977294) -- (11.194513,-18.977294) -- (11.194513,-20.305582) -- (1.799338,-20.305582) -- cycle;
\draw[opacity=1.000000, fill=none, line width=0.030747cm, colorQGKTDBAMEKTCJQ] (1.799338,-18.977294) -- (11.194513,-18.977294) -- (11.194513,-20.305582) -- (1.799338,-20.305582) -- cycle;
\draw[opacity=1.000000, fill=none, line width=0.030747cm, colorQGKTDBAMEKTCJQ] (1.998581,-18.977294) -- (10.995270,-18.977294) -- (10.995270,-20.305582) -- (1.998581,-20.305582) -- cycle;
\verbatimfont{\normalsize\fontQGKTDBGSMHCRBD}
\node[opacity=1.000000, above right, colorQGKTDBAMEKTCJQ] at(3.443954, -19.460980) {\verb|processDiagonalTopLeftToBottomRight(|};
\verbatimfont{\normalsize\fontQGKTDBGSMHCRBD}
\node[opacity=1.000000, above right, colorQGKTDBAMEKTCJQ] at(2.438062, -19.797079) {\verb|matrix, n, k, 0, i, longestZeroSubsequenceBegin,|};
\verbatimfont{\normalsize\fontQGKTDBGSMHCRBD}
\node[opacity=1.000000, above right, colorQGKTDBAMEKTCJQ] at(2.173353, -20.133178) {\verb|longestZeroSubsequenceEnd, maxSubsequenceSize, buf)|};
\draw[opacity=1.000000, fill=none, line width=0.030747 cm, colorQGKTDBAMEKTCJQ] (6.496925,-18.546831) -- (6.496925,-18.977294);
\draw[fill=colorQGKTDBPKRQMG, draw=none, opacity=1.000000] (1.799338,-20.736045) -- (11.194513,-20.736045) -- (11.194513,-22.064333) -- (1.799338,-22.064333) -- cycle;
\draw[opacity=1.000000, fill=none, line width=0.030747cm, colorQGKTDBAMEKTCJQ] (1.799338,-20.736045) -- (11.194513,-20.736045) -- (11.194513,-22.064333) -- (1.799338,-22.064333) -- cycle;
\draw[opacity=1.000000, fill=none, line width=0.030747cm, colorQGKTDBAMEKTCJQ] (1.998581,-20.736045) -- (10.995270,-20.736045) -- (10.995270,-22.064333) -- (1.998581,-22.064333) -- cycle;
\verbatimfont{\normalsize\fontQGKTDBGSMHCRBD}
\node[opacity=1.000000, above right, colorQGKTDBAMEKTCJQ] at(3.443954, -21.219732) {\verb|processDiagonalTopRightToBottomLeft(|};
\verbatimfont{\normalsize\fontQGKTDBGSMHCRBD}
\node[opacity=1.000000, above right, colorQGKTDBAMEKTCJQ] at(2.438062, -21.555830) {\verb|matrix, n, k, 0, i, longestZeroSubsequenceBegin,|};
\verbatimfont{\normalsize\fontQGKTDBGSMHCRBD}
\node[opacity=1.000000, above right, colorQGKTDBAMEKTCJQ] at(2.173353, -21.891929) {\verb|longestZeroSubsequenceEnd, maxSubsequenceSize, buf)|};
\draw[opacity=1.000000, fill=none, line width=0.030747 cm, colorQGKTDBAMEKTCJQ] (6.496925,-20.305582) -- (6.496925,-20.736045);
\draw[opacity=1.000000, fill=none, line width=0.030747 cm, colorQGKTDBAMEKTCJQ] (6.496925,-17.114002) -- (6.496925,-17.544465);
\verbatimfont{\normalsize\fontQGKTDBGSMHCRBD}
\node[opacity=1.000000, above right, colorQGKTDBAMEKTCJQ] at(6.302004, -17.431652) {\verb|+|};
\draw[opacity=1.000000, fill=none, line width=0.030747cm, colorQGKTDBAMEKTCJQ] (6.496925, -22.064333) -- (6.496925, -22.494797) -- (1.036495, -22.494797) -- (1.036495, -15.828761) -- (3.926443, -15.828761);
\verbatimfont{\normalsize\fontQGKTDBGSMHCRBD}
\node[opacity=1.000000, above right, colorQGKTDBAMEKTCJQ] at(11.780883, -15.644276) {\verb|-|};
\draw[opacity=1.000000, fill=none, line width=0.030747cm, colorQGKTDBAMEKTCJQ] (9.067408, -15.828761) -- (11.957356, -15.828761) -- (11.957356, -22.802271) -- (6.496925, -22.802271);
\draw[opacity=1.000000, fill=none, line width=0.030747 cm, colorQGKTDBAMEKTCJQ] (6.496925,-14.113056) -- (6.496925,-14.543519);
\draw[fill=colorQGKTDBPKRQMG, draw=none, opacity=1.000000] (5.448439,-23.232734) -- (7.545412,-23.232734) -- (7.545412,-23.909177) -- (5.448439,-23.909177) -- cycle;
\draw[opacity=1.000000, fill=none, line width=0.030747cm, colorQGKTDBAMEKTCJQ] (5.448439,-23.232734) -- (7.545412,-23.232734) -- (7.545412,-23.909177) -- (5.448439,-23.909177) -- cycle;
\draw[opacity=1.000000, fill=none, line width=0.030747cm, colorQGKTDBAMEKTCJQ] (5.549905,-23.232734) -- (7.443945,-23.232734) -- (7.443945,-23.909177) -- (5.549905,-23.909177) -- cycle;
\verbatimfont{\normalsize\fontQGKTDBGSMHCRBD}
\node[opacity=1.000000, above right, colorQGKTDBAMEKTCJQ] at(5.720447, -23.716420) {\verb|free(buf)|};
\draw[opacity=1.000000, fill=none, line width=0.030747 cm, colorQGKTDBAMEKTCJQ] (6.496925,-22.802271) -- (6.496925,-23.232734);
\draw[opacity=1.000000, rounded corners=0.501183 cm, fill=colorQGKTDBPKRQMG, draw=none] (2.106197,-24.339640) -- (10.887654,-24.339640) -- (10.887654,-25.342006) -- (2.106197,-25.342006) -- cycle;
\draw[opacity=1.000000, rounded corners=0.501183 cm, fill=none, line width=0.030747cm, colorQGKTDBAMEKTCJQ] (2.106197,-24.339640) -- (10.887654,-24.339640) -- (10.887654,-25.342006) -- (2.106197,-25.342006) -- cycle;
\verbatimfont{\normalsize\fontQGKTDBGSMHCRBD}
\node[opacity=1.000000, above right, colorQGKTDBAMEKTCJQ] at(3.443954, -24.823327) {\verb|Выход (longestZeroSubsequenceBegin, |};
\verbatimfont{\normalsize\fontQGKTDBGSMHCRBD}
\node[opacity=1.000000, above right, colorQGKTDBAMEKTCJQ] at(2.596887, -25.159425) {\verb|longestZeroSubsequenceEnd, maxSubsequenceSize)|};
\draw[opacity=1.000000, fill=none, line width=0.030747 cm, colorQGKTDBAMEKTCJQ] (6.496925,-23.909177) -- (6.496925,-24.339640);
\end{tikzpicture}
% --------------------------

% Конец блок схемы
\end{document}}{Функция findLongestZeroLineInMatrix}{ris:task6_11_findLongestZeroLineInMatrix}

    \subsection{Код программы}
    \begin{minted}{C}
#include <stdlib.h>
#include <stdbool.h>

#include "Tasks.h"

static void findLongestZeroSubsequence(bool *arr, int size, int *begin, int *end) {
    int cLen = 0;
    int mLen = 0;
    int tmpBegin = 0;
    *begin = 0;
    *end = 0;

    for (int i = 0; i < size; i++) {
        if (!arr[i]) {
            cLen++;

            if (cLen > mLen) {
                mLen = cLen;
                *begin = tmpBegin;
                *end = i + 1;
            }
        } else {
            tmpBegin = i + 1;
            cLen = 0;
        }
    }
}

static void getLine(bool **matrix, int k, int lineIndex, bool *line) {
    for (int j = 0; j < k; j++)
        line[j] = matrix[lineIndex][j];
}

static void getColumn(bool **matrix, int n, int columnIndex, bool *column) {
    for (int j = 0; j < n; j++)
        column[j] = matrix[j][columnIndex];
}

static void getDiagonalTopLeftToBottomRight(bool **matrix, int n, int k, int iStart, int jStart, bool *diagonal, int *diagonalSize) {
    for (int i = iStart, j = jStart; i < n && j < k; i++, j++)
        diagonal[(*diagonalSize)++] = matrix[i][j];
}

static void getDiagonalTopRightToBottomLeft(bool **matrix, int n, int k, int iStart, int jStart, bool *diagonal, int *diagonalSize) {
    for (int i = iStart, j = jStart; i < n && j >= 0; i++, j--)
        diagonal[(*diagonalSize)++] = matrix[i][j];
}

static void processLine(bool **matrix, int k, int lineIndex, MatrixIndex *longestZeroSubsequenceBegin,
                 MatrixIndex *longestZeroSubsequenceEnd, int *maxSubsequenceSize, bool *buf) {
    getLine(matrix, k, lineIndex, buf);

    int indexBegin, indexEnd;
    findLongestZeroSubsequence(buf, k, &indexBegin, &indexEnd);

    int subSequenceSize = indexEnd - indexBegin;
    if (subSequenceSize > *maxSubsequenceSize) {
        *maxSubsequenceSize = subSequenceSize;
        *longestZeroSubsequenceBegin = (MatrixIndex) {lineIndex, indexBegin};
        *longestZeroSubsequenceEnd = (MatrixIndex) {lineIndex, indexEnd - 1};
    }
}

static void processColumn(bool **matrix, int n, int columnIndex, MatrixIndex *longestZeroSubsequenceBegin,
                   MatrixIndex *longestZeroSubsequenceEnd, int *maxSubsequenceSize, bool *buf) {
    getColumn(matrix, n, columnIndex, buf);

    int indexBegin, indexEnd;
    findLongestZeroSubsequence(buf, n, &indexBegin, &indexEnd);

    int subSequenceSize = indexEnd - indexBegin;
    if (subSequenceSize > *maxSubsequenceSize) {
        *maxSubsequenceSize = subSequenceSize;
        *longestZeroSubsequenceBegin = (MatrixIndex) {indexBegin, columnIndex};
        *longestZeroSubsequenceEnd = (MatrixIndex) {indexEnd - 1, columnIndex};
    }
}

static void processDiagonalTopLeftToBottomRight(bool **matrix, int n, int k, int row, int column,
                                         MatrixIndex *longestZeroSubsequenceBegin,
                                         MatrixIndex *longestZeroSubsequenceEnd, int *maxSubsequenceSize, bool *buf) {
    int bufSize = 0;
    getDiagonalTopLeftToBottomRight(matrix, n, k, row, column, buf, &bufSize);

    int indexBegin, indexEnd;
    findLongestZeroSubsequence(buf, bufSize, &indexBegin, &indexEnd);

    int subSequenceSize = indexEnd - indexBegin;
    if (subSequenceSize > *maxSubsequenceSize) {
        *maxSubsequenceSize = subSequenceSize;
        *longestZeroSubsequenceBegin = (MatrixIndex) {row + indexBegin, column + indexBegin};
        *longestZeroSubsequenceEnd = (MatrixIndex) {row + indexBegin + subSequenceSize - 1, column + indexBegin + subSequenceSize - 1};
    }
}

static void processDiagonalTopRightToBottomLeft(bool **matrix, int n, int k, int row, int column,
                                         MatrixIndex *longestZeroSubsequenceBegin,
                                         MatrixIndex *longestZeroSubsequenceEnd, int *maxSubsequenceSize, bool *buf) {
    int bufSize = 0;
    getDiagonalTopRightToBottomLeft(matrix, n, k, row, column, buf, &bufSize);

    int indexBegin, indexEnd;
    findLongestZeroSubsequence(buf, bufSize, &indexBegin, &indexEnd);

    int subSequenceSize = indexEnd - indexBegin;
    if (subSequenceSize > *maxSubsequenceSize) {
        *maxSubsequenceSize = subSequenceSize;
        *longestZeroSubsequenceBegin = (MatrixIndex) {row + indexBegin, column - indexBegin};
        *longestZeroSubsequenceEnd = (MatrixIndex) {row + indexBegin + subSequenceSize - 1, column - indexBegin - subSequenceSize + 1};
    }
}

int max(int a, int b) {
    return a > b ? a : b;
}

void findLongestZeroLineInMatrix(bool **matrix, int n, int k, MatrixIndex *longestZeroSubsequenceBegin,
                                 MatrixIndex *longestZeroSubsequenceEnd, int *maxSubsequenceSize) {
    bool *buf = malloc(sizeof(bool) * max(n, k));
    *maxSubsequenceSize = 0;
    *longestZeroSubsequenceBegin = (MatrixIndex) {0, 0};
    *longestZeroSubsequenceEnd = (MatrixIndex) {0, 0};

    // Сначала пройдёмся по строчкам
    for (int i = 0; i < n; i++) {
        processLine(matrix, k, i, longestZeroSubsequenceBegin, longestZeroSubsequenceEnd, maxSubsequenceSize, buf);
        // Для каждой строчки здесь же будем находить диагональ слева сверху до нижней правой части с левого края матрицы
        processDiagonalTopLeftToBottomRight(matrix, n, k, i, 0, longestZeroSubsequenceBegin,
                                            longestZeroSubsequenceEnd, maxSubsequenceSize, buf);
        // И диагональ сверху справа до нижней левой части с правого края матрицы
        processDiagonalTopRightToBottomLeft(matrix, n, k, i, k - 1, longestZeroSubsequenceBegin,
                                            longestZeroSubsequenceEnd, maxSubsequenceSize, buf);
    }

    // Потом пройдёмся по столбцам
     for (int i = 0; i < k; i++) {
        processColumn(matrix, n, i, longestZeroSubsequenceBegin, longestZeroSubsequenceEnd, maxSubsequenceSize, buf);
        // Для каждого столбца будем находить оба вида диагоналей.
        processDiagonalTopLeftToBottomRight(matrix, n, k, 0, i, longestZeroSubsequenceBegin,
                                            longestZeroSubsequenceEnd, maxSubsequenceSize, buf);
        processDiagonalTopRightToBottomLeft(matrix, n, k, 0, i, longestZeroSubsequenceBegin,
                                            longestZeroSubsequenceEnd, maxSubsequenceSize, buf);
    }

    free(buf);
}
    \end{minted}

    \subsection{Тестовые данные}
    \labeltable{
        \begin{tabular}{|c|c|}
            \hline
            Тестовые данные & Ожидаемый результат \\
            \hline
            $()$ & Пустая последовательность \\
            \hline
            \begin{pmatrix}
                1 & 1 & 1 \\
                1 & 1 & 1 \\
                1 & 1 & 1 \\
            \end{pmatrix} & Пустая последовательность \\
            \hline
            \begin{pmatrix}
                1 & 1 & 1 \\
                1 & 0 & 1 \\
                1 & 1 & 1 \\
            \end{pmatrix} &
            \begin{pmatrix}
                1 & 1          & 1 \\
                1 & \mathbf{0} & 1 \\
                1 & 1          & 1 \\
            \end{pmatrix} \\
            \hline
            \begin{pmatrix}
                1 & 1 & 1 & 1 & 1 \\
                0 & 0 & 0 & 0 & 0 \\
                1 & 1 & 1 & 1 & 1 \\
            \end{pmatrix} &
            \begin{pmatrix}
                1          & 1          & 1          & 1          & 1          \\
                \mathbf{0} & \mathbf{0} & \mathbf{0} & \mathbf{0} & \mathbf{0} \\
                1          & 1          & 1          & 1          & 1          \\
            \end{pmatrix} \\
            \hline
            \begin{pmatrix}
                1 & 1 & 1 & 1 & 1 \\
                1 & 1 & 1 & 1 & 1 \\
                1 & 1 & 0 & 0 & 0 \\
            \end{pmatrix} &
            \begin{pmatrix}
                1 & 1 & 1          & 1          & 1          \\
                1 & 1 & 1          & 1          & 1          \\
                1 & 1 & \mathbf{0} & \mathbf{0} & \mathbf{0} \\
            \end{pmatrix} \\
            \hline
            \begin{pmatrix}
                1 & 1 & 0 \\
                1 & 1 & 0 \\
                1 & 1 & 0 \\
                1 & 1 & 0 \\
                1 & 1 & 0 \\
            \end{pmatrix} &
            \begin{pmatrix}
                1 & 1 & \mathbf{0} \\
                1 & 1 & \mathbf{0} \\
                1 & 1 & \mathbf{0} \\
                1 & 1 & \mathbf{0} \\
                1 & 1 & \mathbf{0} \\
            \end{pmatrix} \\
            \hline
            \begin{pmatrix}
                1 & 1 & 1 \\
                1 & 1 & 1 \\
                1 & 1 & 1 \\
                1 & 0 & 1 \\
                1 & 0 & 1 \\
            \end{pmatrix} &
            \begin{pmatrix}
                1 & 1          & 1 \\
                1 & 1          & 1 \\
                1 & 1          & 1 \\
                1 & \mathbf{0} & 1 \\
                1 & \mathbf{0} & 1 \\
            \end{pmatrix} \\
            \hline
            \begin{pmatrix}
                1 & 1 & 1 & 1 \\
                1 & 1 & 1 & 1 \\
                1 & 1 & 1 & 0 \\
                1 & 1 & 0 & 1 \\
                1 & 0 & 1 & 1 \\
            \end{pmatrix} &
            \begin{pmatrix}
                1 & 1          & 1          & 1          \\
                1 & 1          & 1          & 1          \\
                1 & 1          & 1          & \mathbf{0} \\
                1 & 1          & \mathbf{0} & 1          \\
                1 & \mathbf{0} & 1          & 1          \\
            \end{pmatrix} \\
            \hline
            \begin{pmatrix}
                1 & 1 & 1 & 1 \\
                1 & 1 & 0 & 1 \\
                1 & 0 & 1 & 1 \\
                1 & 1 & 1 & 1 \\
                1 & 1 & 1 & 1 \\
            \end{pmatrix} &
            \begin{pmatrix}
                1 & 1          & 1          & 1 \\
                1 & 1          & \mathbf{0} & 1 \\
                1 & \mathbf{0} & 1          & 1 \\
                1 & 1          & 1          & 1 \\
                1 & 1          & 1          & 1 \\
            \end{pmatrix} \\
            \hline
        \end{tabular}
    }{Тестовые данные findLongestZeroLineInMatrix}{tab:task6-test-table}
    \begin{center}
        \begin{tabular}{|c|c|}
            \hline
            \begin{pmatrix}
                1 & 1 & 1 & 1 \\
                0 & 1 & 1 & 1 \\
                1 & 0 & 1 & 1 \\
                1 & 1 & 0 & 1 \\
                1 & 1 & 1 & 0 \\
            \end{pmatrix} &
            \begin{pmatrix}
                1          & 1          & 1          & 1          \\
                \mathbf{0} & 1          & 1          & 1          \\
                1          & \mathbf{0} & 1          & 1          \\
                1          & 1          & \mathbf{0} & 1          \\
                1          & 1          & 1          & \mathbf{0} \\
            \end{pmatrix} \\
            \hline
            \begin{pmatrix}
                1 & 0 & 1 & 1 \\
                1 & 1 & 0 & 1 \\
                1 & 1 & 1 & 1 \\
                1 & 1 & 1 & 1 \\
                1 & 1 & 1 & 1 \\
            \end{pmatrix} &
            \begin{pmatrix}
                1 & \mathbf{0} & 1          & 1 \\
                1 & 1          & \mathbf{0} & 1 \\
                1 & 1          & 1          & 1 \\
                1 & 1          & 1          & 1 \\
                1 & 1          & 1          & 1 \\
            \end{pmatrix} \\
            \hline
            \begin{pmatrix}
                1 & 0 & 1 & 0 & 1 \\
                1 & 1 & 0 & 0 & 0 \\
                1 & 1 & 0 & 1 & 1 \\
                1 & 0 & 0 & 0 & 1 \\
                0 & 1 & 1 & 1 & 1 \\
            \end{pmatrix} &
            \begin{pmatrix}
                1          & 0          & 1          & 0          & 1 \\
                1          & 1          & 0          & \mathbf{0} & 0 \\
                1          & 1          & \mathbf{0} & 1          & 1 \\
                1          & \mathbf{0} & 0          & 0          & 1 \\
                \mathbf{0} & 1          & 1          & 1          & 1 \\
            \end{pmatrix} \\
            \hline

        \end{tabular}
    \end{center}

    \subsection{Результат выполнения тестов}
    \makenewfig{
        \includegraphics[width=120mm]{/task6tests}}{Результат выполнения тестов функции findLongestZeroLineInMatrix}


    \section{Арифметика}

    \subsection{Задание варианта}
    Пусть $m$ натуральных чисел заданы своими цифрами в $q$-ичной системе счисления, хранящимися в строках матрицы $K(m, n)$.
    Найти сумму этих чисел в той же системе, не вычисляя самих чисел $(q \leq 10)$.

    \subsection{Обоснование}
    Будем перебирать цифры числа с конца.
    Для всех чисел будем находить сумму всех цифр и перенос из предыдущего разряда, остаток от деления этой суммы на основание сформирует в новом числе цифру, целочисленное деление - перенос в следующий разряд.
    Будем выполнять это жействие до тех пор пока не закончатся цифры.

    \subsection{Спецификация}

    \begin{enumerate}
        \item Заголовок:  \mintinline{C}|void sum(unsigned int **numbers, int m, int n, int q, unsigned int *ans)|
        \item Назначение: вычисляет сумму чисел \mintinline{C}|numbers| количеством \mintinline{C}|m| с количеством цифр \mintinline{C}|n|
        в \mintinline{C}|q|-ичной системе счисления и записывает сумму в той же системе счисления в \mintinline{C}|ans|.
    \end{enumerate}

    \subsection{Блок-схема}
    \makenewfig{% Созданная блок схема работает только с компиляторами XeTeX и LuaTeX.
\documentclass{article}

% Необходимые зависимости
\usepackage[utf8]{inputenc}
\usepackage[english,russian]{babel}
\usepackage{pgfplots}
\usepackage{verbatim}
\usetikzlibrary{positioning}
\usetikzlibrary{shapes.geometric}
\usetikzlibrary{shapes.misc}
\usetikzlibrary{calc}
\usetikzlibrary{chains}
\usetikzlibrary{matrix}
\usetikzlibrary{decorations.text}
\usepackage{fontspec}
\usetikzlibrary{backgrounds}

\begin{document}
% Блок-схема
% Если Вы хотите добавить блок схему в свой документ, скопируйте код между комментариями
% С линияи и вставьте в документ.

% --------------------------
\begin{tikzpicture}[every node/.style={inner sep=0,outer sep=0}, background rectangle/.style={opacity=1.000000, fill=colorFLOFRFEDHOJID}, show background rectangle]
\makeatletter
\newcommand{\verbatimfont}[1]{\def\verbatim@font{#1}}
\makeatother
% Шрифты
\newfontfamily\fontFLOFRFEABEHPJ[Scale=0.938624, SizeFeatures={Size=10.000000}]{Consolas}

% Цвета
\definecolor{colorFLOFRFEDHOJID}{rgb}{1.000000,1.000000,1.000000}
\definecolor{colorFLOFRFEPBIJGNFE}{rgb}{0.949020,0.949020,0.949020}
\definecolor{colorFLOFRFELATEIA}{rgb}{0.000000,0.000000,0.000000}
\draw[opacity=1.000000, rounded corners=0.364238 cm, fill=colorFLOFRFEPBIJGNFE, draw=none] (0.764901,0.006623) -- (6.228477,0.006623) -- (6.228477,-0.721854) -- (0.764901,-0.721854) -- cycle;
\draw[opacity=1.000000, rounded corners=0.364238 cm, fill=none, line width=0.033113cm, colorFLOFRFELATEIA] (0.764901,0.006623) -- (6.228477,0.006623) -- (6.228477,-0.721854) -- (0.764901,-0.721854) -- cycle;
\verbatimfont{\normalsize\fontFLOFRFEABEHPJ}
\node[opacity=1.000000, above right, colorFLOFRFELATEIA] at(1.113186, -0.525297) {\verb|sum(numbers, m, n, q, ans)|};
\draw[fill=colorFLOFRFEPBIJGNFE, draw=none, opacity=1.000000] (2.112583,-1.185430) -- (4.880795,-1.185430) -- (4.880795,-1.913907) -- (2.112583,-1.913907) -- cycle;
\draw[opacity=1.000000, fill=none, line width=0.033113cm, colorFLOFRFELATEIA] (2.112583,-1.185430) -- (4.880795,-1.185430) -- (4.880795,-1.913907) -- (2.112583,-1.913907) -- cycle;
\verbatimfont{\normalsize\fontFLOFRFEABEHPJ}
\node[opacity=1.000000, above right, colorFLOFRFELATEIA] at(2.296110, -1.717350) {\verb|nextRank := 0|};
\draw[opacity=1.000000, fill=none, line width=0.033113 cm, colorFLOFRFELATEIA] (3.496689,-0.721854) -- (3.496689,-1.185430);
\draw[opacity=1.000000, fill=colorFLOFRFEPBIJGNFE, draw=none] (0.549669, -3.850993) -- (2.023179, -2.377483) -- (4.970199, -2.377483) -- (6.443709, -3.850993) -- (4.970199, -5.324503) -- (2.023179, -5.324503) --cycle;
\draw[opacity=1.000000, fill=none, line width=0.033113cm, colorFLOFRFELATEIA] (0.549669, -3.850993) -- (2.023179, -2.377483) -- (4.970199, -2.377483) -- (6.443709, -3.850993) -- (4.970199, -5.324503) -- (2.023179, -5.324503) --cycle;
\verbatimfont{\normalsize\fontFLOFRFEABEHPJ}
\node[opacity=1.000000, above right, colorFLOFRFELATEIA] at(2.207832, -3.763708) {\verb|digitIndex := |};
\verbatimfont{\normalsize\fontFLOFRFEABEHPJ}
\node[opacity=1.000000, above right, colorFLOFRFELATEIA] at(2.402043, -4.136687) {\verb|n - 1, 0, -1|};
\draw[fill=colorFLOFRFEPBIJGNFE, draw=none, opacity=1.000000] (1.930464,-5.788079) -- (5.062914,-5.788079) -- (5.062914,-6.516556) -- (1.930464,-6.516556) -- cycle;
\draw[opacity=1.000000, fill=none, line width=0.033113cm, colorFLOFRFELATEIA] (1.930464,-5.788079) -- (5.062914,-5.788079) -- (5.062914,-6.516556) -- (1.930464,-6.516556) -- cycle;
\verbatimfont{\normalsize\fontFLOFRFEABEHPJ}
\node[opacity=1.000000, above right, colorFLOFRFELATEIA] at(2.119554, -6.319999) {\verb|sum := nextRank|};
\draw[opacity=1.000000, fill=colorFLOFRFEPBIJGNFE, draw=none] (1.092715, -8.182119) -- (2.294702, -6.980132) -- (4.698675, -6.980132) -- (5.900662, -8.182119) -- (4.698675, -9.384106) -- (2.294702, -9.384106) --cycle;
\draw[opacity=1.000000, fill=none, line width=0.033113cm, colorFLOFRFELATEIA] (1.092715, -8.182119) -- (2.294702, -6.980132) -- (4.698675, -6.980132) -- (5.900662, -8.182119) -- (4.698675, -9.384106) -- (2.294702, -9.384106) --cycle;
\verbatimfont{\normalsize\fontFLOFRFEABEHPJ}
\node[opacity=1.000000, above right, colorFLOFRFELATEIA] at(2.490321, -8.094833) {\verb|numIndex :=|};
\verbatimfont{\normalsize\fontFLOFRFEABEHPJ}
\node[opacity=1.000000, above right, colorFLOFRFELATEIA] at(2.755155, -8.467812) {\verb|0, m - 1|};
\draw[fill=colorFLOFRFEPBIJGNFE, draw=none, opacity=1.000000] (0.655629,-9.847682) -- (6.337748,-9.847682) -- (6.337748,-10.927152) -- (0.655629,-10.927152) -- cycle;
\draw[opacity=1.000000, fill=none, line width=0.033113cm, colorFLOFRFELATEIA] (0.655629,-9.847682) -- (6.337748,-9.847682) -- (6.337748,-10.927152) -- (0.655629,-10.927152) -- cycle;
\verbatimfont{\normalsize\fontFLOFRFEABEHPJ}
\node[opacity=1.000000, above right, colorFLOFRFELATEIA] at(2.296110, -10.379602) {\verb|sum := sum + |};
\verbatimfont{\normalsize\fontFLOFRFEABEHPJ}
\node[opacity=1.000000, above right, colorFLOFRFELATEIA] at(0.848352, -10.752581) {\verb|numbers[numIndex][digitIndex]|};
\draw[opacity=1.000000, fill=none, line width=0.033113 cm, colorFLOFRFELATEIA] (3.496689,-9.384106) -- (3.496689,-9.847682);
\verbatimfont{\normalsize\fontFLOFRFEABEHPJ}
\node[opacity=1.000000, above right, colorFLOFRFELATEIA] at(3.264954, -9.737217) {\verb|+|};
\draw[opacity=1.000000, fill=none, line width=0.033113cm, colorFLOFRFELATEIA] (3.496689, -10.927152) -- (3.496689, -11.390728) -- (0.324503, -11.390728) -- (0.324503, -8.182119) -- (1.092715, -8.182119);
\verbatimfont{\normalsize\fontFLOFRFEABEHPJ}
\node[opacity=1.000000, above right, colorFLOFRFELATEIA] at(6.457007, -7.983444) {\verb|-|};
\draw[opacity=1.000000, fill=none, line width=0.033113cm, colorFLOFRFELATEIA] (5.900662, -8.182119) -- (6.668874, -8.182119) -- (6.668874, -11.721854) -- (3.496689, -11.721854);
\draw[opacity=1.000000, fill=none, line width=0.033113 cm, colorFLOFRFELATEIA] (3.496689,-6.516556) -- (3.496689,-6.980132);
\draw[fill=colorFLOFRFEPBIJGNFE, draw=none, opacity=1.000000] (0.837748,-12.185430) -- (6.155629,-12.185430) -- (6.155629,-12.913907) -- (0.837748,-12.913907) -- cycle;
\draw[opacity=1.000000, fill=none, line width=0.033113cm, colorFLOFRFELATEIA] (0.837748,-12.185430) -- (6.155629,-12.185430) -- (6.155629,-12.913907) -- (0.837748,-12.913907) -- cycle;
\verbatimfont{\normalsize\fontFLOFRFEABEHPJ}
\node[opacity=1.000000, above right, colorFLOFRFELATEIA] at(1.024908, -12.717350) {\verb|ans[digitIndex] = sum mod q|};
\draw[opacity=1.000000, fill=none, line width=0.033113 cm, colorFLOFRFELATEIA] (3.496689,-11.721854) -- (3.496689,-12.185430);
\draw[fill=colorFLOFRFEPBIJGNFE, draw=none, opacity=1.000000] (1.384106,-13.377483) -- (5.609272,-13.377483) -- (5.609272,-14.105960) -- (1.384106,-14.105960) -- cycle;
\draw[opacity=1.000000, fill=none, line width=0.033113cm, colorFLOFRFELATEIA] (1.384106,-13.377483) -- (5.609272,-13.377483) -- (5.609272,-14.105960) -- (1.384106,-14.105960) -- cycle;
\verbatimfont{\normalsize\fontFLOFRFEABEHPJ}
\node[opacity=1.000000, above right, colorFLOFRFELATEIA] at(1.572231, -13.909403) {\verb|nextRank := sum div q|};
\draw[opacity=1.000000, fill=none, line width=0.033113 cm, colorFLOFRFELATEIA] (3.496689,-12.913907) -- (3.496689,-13.377483);
\draw[opacity=1.000000, fill=none, line width=0.033113 cm, colorFLOFRFELATEIA] (3.496689,-5.324503) -- (3.496689,-5.788079);
\verbatimfont{\normalsize\fontFLOFRFEABEHPJ}
\node[opacity=1.000000, above right, colorFLOFRFELATEIA] at(3.264954, -5.677615) {\verb|+|};
\draw[opacity=1.000000, fill=none, line width=0.033113cm, colorFLOFRFELATEIA] (3.496689, -14.105960) -- (3.496689, -14.569536) -- (-0.006623, -14.569536) -- (-0.006623, -3.850993) -- (0.549669, -3.850993);
\verbatimfont{\normalsize\fontFLOFRFEABEHPJ}
\node[opacity=1.000000, above right, colorFLOFRFELATEIA] at(6.788133, -3.652318) {\verb|-|};
\draw[opacity=1.000000, fill=none, line width=0.033113cm, colorFLOFRFELATEIA] (6.443709, -3.850993) -- (7.000000, -3.850993) -- (7.000000, -14.900662) -- (3.496689, -14.900662);
\draw[opacity=1.000000, fill=none, line width=0.033113 cm, colorFLOFRFELATEIA] (3.496689,-1.913907) -- (3.496689,-2.377483);
\draw[opacity=1.000000, rounded corners=0.364238 cm, fill=colorFLOFRFEPBIJGNFE, draw=none] (2.039735,-15.364238) -- (4.953642,-15.364238) -- (4.953642,-16.092715) -- (2.039735,-16.092715) -- cycle;
\draw[opacity=1.000000, rounded corners=0.364238 cm, fill=none, line width=0.033113cm, colorFLOFRFELATEIA] (2.039735,-15.364238) -- (4.953642,-15.364238) -- (4.953642,-16.092715) -- (2.039735,-16.092715) -- cycle;
\verbatimfont{\normalsize\fontFLOFRFEABEHPJ}
\node[opacity=1.000000, above right, colorFLOFRFELATEIA] at(2.490321, -15.896158) {\verb|Выход (ans)|};
\draw[opacity=1.000000, fill=none, line width=0.033113 cm, colorFLOFRFELATEIA] (3.496689,-14.900662) -- (3.496689,-15.364238);
\end{tikzpicture}
% --------------------------

% Конец блок схемы
\end{document}}{Функция sum}{ris:task7}

    \subsection{Код программы}
    \begin{minted}{C}
void sum(unsigned int **numbers, int m, int n, int q, unsigned int *ans) {
    unsigned int nextRank = 0;

    // Будем идти с конца чисел
    for (int digitIndex = n - 1; digitIndex >= 0; digitIndex--) {
        // Посчитаем сумму цифр. Сумме цифр сразу присваиваем перенос из предыдущего разряда.
        unsigned int sum = nextRank;

        // Из каждого числа добавляем цифру
        for (int numIndex = 0; numIndex < m; numIndex++)
            sum += numbers[numIndex][digitIndex];

        // В текущий разряд записываем остаток от суммы
        ans[digitIndex] = sum % q;
        // В следующий разряд переносим целочисленное деление
        nextRank = sum / q;
    }
}
    \end{minted}

    \subsection{Тестовые данные}
    \labeltable{
        \begin{tabular}{|c|c|}
            \hline
            Тестовые данные & Ожидаемый результат \\
            \hline
            \begin{tabular}{c}
                numbers = $()$, \\m = 0, n = 0, q = 3
            \end{tabular} & ans = $000_3$ \\
            \hline
            \begin{tabular}{c}
                numbers = \begin{pmatrix}
                              0 & 1 & 0 \\
                              1 & 1 & 1 \\
                \end{pmatrix}, \\m = 2, n = 3, q = 2
            \end{tabular} & ans = $101_2$ \\
            \hline
            \begin{tabular}{c}
                numbers = \begin{pmatrix}
                              0 & 3 & 2 \\
                              1 & 1 & 1 \\
                \end{pmatrix}, \\m = 2, n = 3, q = 4
            \end{tabular} & ans = $103_4$ \\
            \hline
            \begin{tabular}{c}
                numbers = \begin{pmatrix}
                              0 & 3 & 2 \\
                              1 & 1 & 1 \\
                              1 & 2 & 1 \\
                              0 & 0 & 3 \\
                \end{pmatrix}, \\m = 4, n = 3, q = 4
            \end{tabular} & ans = $232_4$ \\
            \hline
            \begin{tabular}{c}
                numbers = \begin{pmatrix}
                              0 & 3 & 2 & 0 \\
                              0 & 0 & 1 & 1 \\
                              0 & 0 & 0 & 1 \\
                              1 & 0 & 0 & 1 \\
                \end{pmatrix}, \\m = 4, n = 4, q = 2
            \end{tabular} & ans = $1111_2$ \\
            \hline
            \begin{tabular}{c}
                numbers = \begin{pmatrix}
                              0 & 3 & 2 \\
                              2 & 1 & 1 \\
                              1 & 2 & 0 \\
                              0 & 3 & 0 \\
                \end{pmatrix}, \\m = 4, n = 3, q = 4
            \end{tabular} & ans = $113_4$ \\
            \hline
            \begin{tabular}{c}
                numbers = \begin{pmatrix}
                              0 & 0 & 0 \\
                              0 & 0 & 0 \\
                              0 & 0 & 0 \\
                \end{pmatrix}, \\m = 3, n = 3, q = 3
            \end{tabular} & ans = $000_3$ \\
            \hline
        \end{tabular}
    }{Тестовые данные sum}{tab:task7-test-table}

    \subsection{Результат выполнения тестов}
    \makenewfig{
        \includegraphics[width=120mm]{/task7tests}}{Результат выполнения тестов функции sum}


    \section{Геометрия и теория множеств}

    \subsection{Задание варианта}
    Найти площадь многоугольника (не обязательно выпуклого), заданного координатами своих вершин на плоскости в порядке обхода по или против часовой стрелки.

    \subsection{Обоснование}
    Площадь многоугольника можно найти методом трапеций.
    Опустим из каждой точки многоугольника перпендикуляр к $Ox$, две соседние точки и их перпендикуляры будут образовывать трапецию.
    Если вторая точка находится правеке текущей, прибавляем сумму к итоговой, иначе - отнимаем.
    Итоговая сумма будет равна сумме по модулю, так как знак суммарной площади зависит от порядка обхода.
    Формула для вычисления площади:\\
    $S=|\sum_{i = 0}^{n}\frac{(points_i.x - points{(i + 1) \% n}.x) \cdot (points_i.y + points_{(i + 1) \% n}.y)}{2}|$, где points - массив точек.

    \subsection{Спецификация}

    \begin{enumerate}
        \item Заголовок:  \mintinline{C}|double getArea(Point *points, int pointsAmount)|
        \item Назначение: воозвращает сумму многоугольника, заданного массивом точек \mintinline{C}|points| размером \mintinline{C}|pointsAmount|.
    \end{enumerate}

    \subsection{Блок-схема}
    \makenewfig{% Созданная блок схема работает только с компиляторами XeTeX и LuaTeX.
\documentclass{article}

% Необходимые зависимости
\usepackage[utf8]{inputenc}
\usepackage[english,russian]{babel}
\usepackage{pgfplots}
\usepackage{verbatim}
\usetikzlibrary{positioning}
\usetikzlibrary{shapes.geometric}
\usetikzlibrary{shapes.misc}
\usetikzlibrary{calc}
\usetikzlibrary{chains}
\usetikzlibrary{matrix}
\usetikzlibrary{decorations.text}
\usepackage{fontspec}
\usetikzlibrary{backgrounds}

\begin{document}
% Блок-схема
% Если Вы хотите добавить блок схему в свой документ, скопируйте код между комментариями
% С линияи и вставьте в документ.

% --------------------------
\begin{tikzpicture}[every node/.style={inner sep=0,outer sep=0}, background rectangle/.style={opacity=1.000000, fill=colorHQMAHGFGTIFSDS}, show background rectangle]
\makeatletter
\newcommand{\verbatimfont}[1]{\def\verbatim@font{#1}}
\makeatother
% Шрифты
\newfontfamily\fontHQMAHGFOHRLCT[Scale=0.862118, SizeFeatures={Size=10.000000}]{Consolas}

% Цвета
\definecolor{colorHQMAHGFGTIFSDS}{rgb}{1.000000,1.000000,1.000000}
\definecolor{colorHQMAHGFOOHRNQ}{rgb}{0.949020,0.949020,0.949020}
\definecolor{colorHQMAHGFQPTJPJTM}{rgb}{0.000000,0.000000,0.000000}
\draw[opacity=1.000000, rounded corners=0.334550 cm, fill=colorHQMAHGFOOHRNQ, draw=none] (2.235401,0.006083) -- (7.758516,0.006083) -- (7.758516,-0.663017) -- (2.235401,-0.663017) -- cycle;
\draw[opacity=1.000000, rounded corners=0.334550 cm, fill=none, line width=0.030414cm, colorHQMAHGFQPTJPJTM] (2.235401,0.006083) -- (7.758516,0.006083) -- (7.758516,-0.663017) -- (2.235401,-0.663017) -- cycle;
\verbatimfont{\normalsize\fontHQMAHGFOHRLCT}
\node[opacity=1.000000, above right, colorHQMAHGFQPTJPJTM] at(2.561311, -0.475844) {\verb|getArea(points, pointsAmount)|};
\draw[fill=colorHQMAHGFOOHRNQ, draw=none, opacity=1.000000] (3.643552,-1.088808) -- (6.350365,-1.088808) -- (6.350365,-1.757908) -- (3.643552,-1.757908) -- cycle;
\draw[opacity=1.000000, fill=none, line width=0.030414cm, colorHQMAHGFQPTJPJTM] (3.643552,-1.088808) -- (6.350365,-1.088808) -- (6.350365,-1.757908) -- (3.643552,-1.757908) -- cycle;
\verbatimfont{\normalsize\fontHQMAHGFOHRLCT}
\node[opacity=1.000000, above right, colorHQMAHGFQPTJPJTM] at(3.814434, -1.570734) {\verb|areaTotal := 0|};
\draw[opacity=1.000000, fill=none, line width=0.030414 cm, colorHQMAHGFQPTJPJTM] (4.996959,-0.663017) -- (4.996959,-1.088808);
\draw[opacity=1.000000, fill=colorHQMAHGFOOHRNQ, draw=none] (2.119830, -3.622263) -- (3.558394, -2.183698) -- (6.435523, -2.183698) -- (7.874088, -3.622263) -- (6.435523, -5.060827) -- (3.558394, -5.060827) --cycle;
\draw[opacity=1.000000, fill=none, line width=0.030414cm, colorHQMAHGFQPTJPJTM] (2.119830, -3.622263) -- (3.558394, -2.183698) -- (6.435523, -2.183698) -- (7.874088, -3.622263) -- (6.435523, -5.060827) -- (3.558394, -5.060827) --cycle;
\verbatimfont{\normalsize\fontHQMAHGFOHRLCT}
\node[opacity=1.000000, above right, colorHQMAHGFQPTJPJTM] at(4.661616, -3.535455) {\verb|i :=|};
\verbatimfont{\normalsize\fontHQMAHGFOHRLCT}
\node[opacity=1.000000, above right, colorHQMAHGFQPTJPJTM] at(3.726186, -3.871396) {\verb|0, pointsAmount|};
\draw[fill=colorHQMAHGFOOHRNQ, draw=none, opacity=1.000000] (0.298054,-5.486618) -- (9.695864,-5.486618) -- (9.695864,-6.800487) -- (0.298054,-6.800487) -- cycle;
\draw[opacity=1.000000, fill=none, line width=0.030414cm, colorHQMAHGFQPTJPJTM] (0.298054,-5.486618) -- (9.695864,-5.486618) -- (9.695864,-6.800487) -- (0.298054,-6.800487) -- cycle;
\verbatimfont{\normalsize\fontHQMAHGFOHRLCT}
\node[opacity=1.000000, above right, colorHQMAHGFQPTJPJTM] at(2.896654, -5.968545) {\verb|areaTotal := areaTotal + |};
\verbatimfont{\normalsize\fontHQMAHGFOHRLCT}
\node[opacity=1.000000, above right, colorHQMAHGFQPTJPJTM] at(0.478656, -6.304486) {\verb|(points[i].x - points[(i + 1) mod pointsAmount].x) *  |};
\verbatimfont{\normalsize\fontHQMAHGFOHRLCT}
\node[opacity=1.000000, above right, colorHQMAHGFQPTJPJTM] at(0.478656, -6.640427) {\verb|(points[i].y + points[(i + 1) mod pointsAmount].y) / 2|};
\draw[opacity=1.000000, fill=none, line width=0.030414 cm, colorHQMAHGFQPTJPJTM] (4.996959,-5.060827) -- (4.996959,-5.486618);
\verbatimfont{\normalsize\fontHQMAHGFOHRLCT}
\node[opacity=1.000000, above right, colorHQMAHGFQPTJPJTM] at(4.802214, -5.378520) {\verb|+|};
\draw[opacity=1.000000, fill=none, line width=0.030414cm, colorHQMAHGFQPTJPJTM] (4.996959, -6.800487) -- (4.996959, -7.226277) -- (-0.006083, -7.226277) -- (-0.006083, -3.622263) -- (2.119830, -3.622263);
\verbatimfont{\normalsize\fontHQMAHGFOHRLCT}
\node[opacity=1.000000, above right, colorHQMAHGFQPTJPJTM] at(9.823504, -3.439781) {\verb|-|};
\draw[opacity=1.000000, fill=none, line width=0.030414cm, colorHQMAHGFQPTJPJTM] (7.874088, -3.622263) -- (10.000000, -3.622263) -- (10.000000, -7.530414) -- (4.996959, -7.530414);
\draw[opacity=1.000000, fill=none, line width=0.030414 cm, colorHQMAHGFQPTJPJTM] (4.996959,-1.757908) -- (4.996959,-2.183698);
\draw[fill=colorHQMAHGFOOHRNQ, draw=none, opacity=1.000000] (2.372263,-7.956204) -- (7.621655,-7.956204) -- (7.621655,-8.625304) -- (2.372263,-8.625304) -- cycle;
\draw[opacity=1.000000, fill=none, line width=0.030414cm, colorHQMAHGFQPTJPJTM] (2.372263,-7.956204) -- (7.621655,-7.956204) -- (7.621655,-8.625304) -- (2.372263,-8.625304) -- cycle;
\draw[opacity=1.000000, fill=none, line width=0.030414cm, colorHQMAHGFQPTJPJTM] (2.472628,-7.956204) -- (7.521290,-7.956204) -- (7.521290,-8.625304) -- (2.472628,-8.625304) -- cycle;
\verbatimfont{\normalsize\fontHQMAHGFOHRLCT}
\node[opacity=1.000000, above right, colorHQMAHGFQPTJPJTM] at(2.649559, -8.438131) {\verb|areaTotal := fabs(areaTotal)|};
\draw[opacity=1.000000, fill=none, line width=0.030414 cm, colorHQMAHGFQPTJPJTM] (4.996959,-7.530414) -- (4.996959,-7.956204);
\draw[opacity=1.000000, rounded corners=0.334550 cm, fill=colorHQMAHGFOOHRNQ, draw=none] (3.239051,-9.051095) -- (6.754866,-9.051095) -- (6.754866,-9.720195) -- (3.239051,-9.720195) -- cycle;
\draw[opacity=1.000000, rounded corners=0.334550 cm, fill=none, line width=0.030414cm, colorHQMAHGFQPTJPJTM] (3.239051,-9.051095) -- (6.754866,-9.051095) -- (6.754866,-9.720195) -- (3.239051,-9.720195) -- cycle;
\verbatimfont{\normalsize\fontHQMAHGFOHRLCT}
\node[opacity=1.000000, above right, colorHQMAHGFQPTJPJTM] at(3.567340, -9.533022) {\verb|Вернуть areaTotal|};
\draw[opacity=1.000000, fill=none, line width=0.030414 cm, colorHQMAHGFQPTJPJTM] (4.996959,-8.625304) -- (4.996959,-9.051095);
\end{tikzpicture}
% --------------------------

% Конец блок схемы
\end{document}}{Функция getArea}{ris:task8}

    \subsection{Код программы}
    \begin{minted}{C}
#include <math.h>

#include "Tasks.h"

double getArea(Point *points, int pointsAmount) {
    double areaTotal = 0;

    for (int i = 0; i < pointsAmount; i++)
        areaTotal += (points[i].x - points[(i + 1) % pointsAmount].x) *
                (points[i].y + points[(i + 1) % pointsAmount].y) / 2;

    areaTotal = fabs(areaTotal);

    return areaTotal;
}
    \end{minted}

    \subsection{Тестовые данные}
    \labeltable{
        \begin{tabular}{|c|c|}
            \hline
            Тестовые данные & Ожидаемый результат \\
            \hline
            \begin{tabular}{c}
                pointsAmount = 0
            \end{tabular}   & 0                   \\
            \hline
            \begin{tabular}{c}
                points =\\(1, 1)\\(5, 5)\\pointsAmount = 2
            \end{tabular}   & 0                \\
            \hline
            \begin{tabular}{c}
                points =\\(0, 0)\\(5, 5)\\(5, 0)\\pointsAmount = 3
            \end{tabular}   & 12.5                \\
            \hline
            \begin{tabular}{c}
                points =\\(2, 1)\\(5, 3)\\(6, 6)\\(5, 7)\\(3, 6)\\(2, 8)\\(1, 5)\\pointsAmount = 7
            \end{tabular}   & 19.5                \\
            \hline
            \begin{tabular}{c}
                points =\\(2, 1)\\(1, 5)\\(2, 8)\\(3, 6)\\(5, 7)\\(6, 6)\\(5, 3)\\pointsAmount = 7
            \end{tabular}   & 19.5                \\
            \hline
        \end{tabular}
    }{Тестовые данные getArea}{tab:task8-test-table}

    \subsection{Результат выполнения тестов}
    \makenewfig{
        \includegraphics[width=120mm]{/task8tests}}{Результат выполнения тестов функции getArea}


    \section{Линейная алгебра и сжатие информации}

    \subsection{Задание варианта}
    Матрица $A(n, n)$ системы линейных уравнений $AX = B$ приведена к
    верхнетреугольному виду и упакована в одномерный массив.
    Найти вектор решения $X$ последовательной подстановкой, не распаковывая $A$.

    \subsection{Обоснование}
    Рассмотрим для примера матрицу 3 на 3\\
    \begin{pmatrix}
        x_1 & x_2 & x_3 \\
        0   & x_4 & x_5 \\
        0   & 0   & x_6 \\
    \end{pmatrix} $\times$
    \begin{pmatrix}
        t_1 & t_2 & t_3 \\
        t_4 & t_5 & t_6 \\
        t_7 & t_8 & t_9 \\
    \end{pmatrix} $=$
    \begin{pmatrix}
        b_1 & b_2 & b_3 \\
        b_4 & b_5 & b_6 \\
        b_7 & b_8 & b_9 \\
    \end{pmatrix}\\
    \begin{pmatrix}
        x_1 \cdot t_1 + x_2 \cdot t_4 + x_3 \cdot t_7 & x_1 \cdot t_2 + x_2 \cdot t_5 + x_3 \cdot t_8 & x_1 \cdot t_3 + x_2 \cdot t_6 + x_3 \cdot t_9 \\
        x_4\cdot t_4 + x_5\cdot t_7                   & x_4\cdot t_2 + x_5\cdot t_5                   & x_4\cdot t_6 + x_5\cdot t_9                   \\
        x_6\cdot t_7                                  & x_6\cdot t_8                                  & x_6\cdot t_9                                  \\
    \end{pmatrix}
    $=$
    \begin{pmatrix}
        b_1 & b_2 & b_3 \\
        b_4 & b_5 & b_6 \\
        b_7 & b_8 & b_9 \\
    \end{pmatrix}\\
    Можно заметить, что легко можно вычислить $t_7, t_8, t_9$.
    Далее на основе вычисленных $t_7, t_8, t_9$ можно вычислить $t_4, t_5, t_6$, и так далее до тех пор пока не получим полную матрицу $X$.

    \subsection{Спецификация}

    \begin{enumerate}
        \item Заголовок:  \mintinline{C}|void findX(double *matrixA, int n, double **matrixB, double **matrixX)|
        \item Назначение: находит в уравнении вида $AX=B$ вектор $X$, где $A$ -  \mintinline{C}|matrixA|, верхнетреугольная упакованная матрица,
        \mintinline{C}|n| - размер стороны матриц, $B$ -  \mintinline{C}|matrixB|. Сохраняет ответ $X$ в \mintinline{C}|matrixX|.
    \end{enumerate}

    \subsection{Блок-схема}

    \makenewfig{% Созданная блок схема работает только с компиляторами XeTeX и LuaTeX.
\documentclass{article}

% Необходимые зависимости
\usepackage[utf8]{inputenc}
\usepackage[english,russian]{babel}
\usepackage{pgfplots}
\usepackage{verbatim}
\usetikzlibrary{positioning}
\usetikzlibrary{shapes.geometric}
\usetikzlibrary{shapes.misc}
\usetikzlibrary{calc}
\usetikzlibrary{chains}
\usetikzlibrary{matrix}
\usetikzlibrary{decorations.text}
\usepackage{fontspec}
\usetikzlibrary{backgrounds}

\begin{document}
% Блок-схема
% Если Вы хотите добавить блок схему в свой документ, скопируйте код между комментариями
% С линияи и вставьте в документ.

% --------------------------
\begin{tikzpicture}[every node/.style={inner sep=0,outer sep=0}, background rectangle/.style={opacity=1.000000, fill=colorNEIGTJJNKOKSLT}, show background rectangle]
\makeatletter
\newcommand{\verbatimfont}[1]{\def\verbatim@font{#1}}
\makeatother
% Шрифты
\newfontfamily\fontNEIGTJJNFNNKCLJT[Scale=0.696962, SizeFeatures={Size=10.000000}]{Consolas}

% Цвета
\definecolor{colorNEIGTJJNKOKSLT}{rgb}{1.000000,1.000000,1.000000}
\definecolor{colorNEIGTJJNFOAEMES}{rgb}{0.949020,0.949020,0.949020}
\definecolor{colorNEIGTJJNTMQRBC}{rgb}{0.000000,0.000000,0.000000}
\draw[opacity=1.000000, rounded corners=0.270460 cm, fill=colorNEIGTJJNFOAEMES, draw=none] (4.428170,0.004917) -- (9.704601,0.004917) -- (9.704601,-0.536003) -- (4.428170,-0.536003) -- cycle;
\draw[opacity=1.000000, rounded corners=0.270460 cm, fill=none, line width=0.024587cm, colorNEIGTJJNTMQRBC] (4.428170,0.004917) -- (9.704601,0.004917) -- (9.704601,-0.536003) -- (4.428170,-0.536003) -- cycle;
\verbatimfont{\normalsize\fontNEIGTJJNFNNKCLJT}
\node[opacity=1.000000, above right, colorNEIGTJJNTMQRBC] at(4.684299, -0.374885) {\verb|findX(matrixA, n, matrixB, matrixX)|};
\draw[fill=colorNEIGTJJNFOAEMES, draw=none, opacity=1.000000] (5.431331,-0.880225) -- (8.701440,-0.880225) -- (8.701440,-1.421145) -- (5.431331,-1.421145) -- cycle;
\draw[opacity=1.000000, fill=none, line width=0.024587cm, colorNEIGTJJNTMQRBC] (5.431331,-0.880225) -- (8.701440,-0.880225) -- (8.701440,-1.421145) -- (5.431331,-1.421145) -- cycle;
\verbatimfont{\normalsize\fontNEIGTJJNFNNKCLJT}
\node[opacity=1.000000, above right, colorNEIGTJJNTMQRBC] at(5.566553, -1.260027) {\verb|elementsAmountOnA := 1|};
\draw[opacity=1.000000, fill=none, line width=0.024587 cm, colorNEIGTJJNTMQRBC] (7.066386,-0.536003) -- (7.066386,-0.880225);
\draw[fill=colorNEIGTJJNFOAEMES, draw=none, opacity=1.000000] (4.755181,-1.765367) -- (9.377590,-1.765367) -- (9.377590,-2.566913) -- (4.755181,-2.566913) -- cycle;
\draw[opacity=1.000000, fill=none, line width=0.024587cm, colorNEIGTJJNTMQRBC] (4.755181,-1.765367) -- (9.377590,-1.765367) -- (9.377590,-2.566913) -- (4.755181,-2.566913) -- cycle;
\verbatimfont{\normalsize\fontNEIGTJJNFNNKCLJT}
\node[opacity=1.000000, above right, colorNEIGTJJNTMQRBC] at(4.896040, -2.145170) {\verb|totalCompressedMatrixElements = |};
\verbatimfont{\normalsize\fontNEIGTJJNFNNKCLJT}
\node[opacity=1.000000, above right, colorNEIGTJJNTMQRBC] at(5.778295, -2.406953) {\verb|((n + 1) * n) div 2|};
\draw[opacity=1.000000, fill=none, line width=0.024587 cm, colorNEIGTJJNTMQRBC] (7.066386,-1.421145) -- (7.066386,-1.765367);
\draw[opacity=1.000000, fill=colorNEIGTJJNFOAEMES, draw=none] (6.092729, -3.397963) -- (6.579557, -2.911135) -- (7.553214, -2.911135) -- (8.040042, -3.397963) -- (7.553214, -3.884791) -- (6.579557, -3.884791) --cycle;
\draw[opacity=1.000000, fill=none, line width=0.024587cm, colorNEIGTJJNTMQRBC] (6.092729, -3.397963) -- (6.579557, -2.911135) -- (7.553214, -2.911135) -- (8.040042, -3.397963) -- (7.553214, -3.884791) -- (6.579557, -3.884791) --cycle;
\verbatimfont{\normalsize\fontNEIGTJJNFNNKCLJT}
\node[opacity=1.000000, above right, colorNEIGTJJNTMQRBC] at(6.713484, -3.317983) {\verb|i := |};
\verbatimfont{\normalsize\fontNEIGTJJNFNNKCLJT}
\node[opacity=1.000000, above right, colorNEIGTJJNTMQRBC] at(6.784064, -3.579767) {\verb|0, n|};
\draw[opacity=1.000000, fill=colorNEIGTJJNFOAEMES, draw=none] (6.230418, -4.646997) -- (6.648402, -4.229013) -- (7.484370, -4.229013) -- (7.902353, -4.646997) -- (7.484370, -5.064981) -- (6.648402, -5.064981) --cycle;
\draw[opacity=1.000000, fill=none, line width=0.024587cm, colorNEIGTJJNTMQRBC] (6.230418, -4.646997) -- (6.648402, -4.229013) -- (7.484370, -4.229013) -- (7.902353, -4.646997) -- (7.484370, -5.064981) -- (6.648402, -5.064981) --cycle;
\verbatimfont{\normalsize\fontNEIGTJJNFNNKCLJT}
\node[opacity=1.000000, above right, colorNEIGTJJNTMQRBC] at(6.784064, -4.567017) {\verb|j :=|};
\verbatimfont{\normalsize\fontNEIGTJJNFNNKCLJT}
\node[opacity=1.000000, above right, colorNEIGTJJNTMQRBC] at(6.784064, -4.828801) {\verb|0, n|};
\draw[fill=colorNEIGTJJNFOAEMES, draw=none, opacity=1.000000] (5.701791,-5.409203) -- (8.430980,-5.409203) -- (8.430980,-5.950123) -- (5.701791,-5.950123) -- cycle;
\draw[opacity=1.000000, fill=none, line width=0.024587cm, colorNEIGTJJNTMQRBC] (5.701791,-5.409203) -- (8.430980,-5.409203) -- (8.430980,-5.950123) -- (5.701791,-5.950123) -- cycle;
\verbatimfont{\normalsize\fontNEIGTJJNFNNKCLJT}
\node[opacity=1.000000, above right, colorNEIGTJJNTMQRBC] at(5.848875, -5.789005) {\verb|matrixX[i][j] := 0|};
\draw[opacity=1.000000, fill=none, line width=0.024587 cm, colorNEIGTJJNTMQRBC] (7.066386,-5.064981) -- (7.066386,-5.409203);
\verbatimfont{\normalsize\fontNEIGTJJNFNNKCLJT}
\node[opacity=1.000000, above right, colorNEIGTJJNTMQRBC] at(6.910473, -5.312012) {\verb|+|};
\draw[opacity=1.000000, fill=none, line width=0.024587cm, colorNEIGTJJNTMQRBC] (7.066386, -5.950123) -- (7.066386, -6.294345) -- (5.455919, -6.294345) -- (5.455919, -4.646997) -- (6.230418, -4.646997);
\verbatimfont{\normalsize\fontNEIGTJJNFNNKCLJT}
\node[opacity=1.000000, above right, colorNEIGTJJNTMQRBC] at(8.535692, -4.499473) {\verb|-|};
\draw[opacity=1.000000, fill=none, line width=0.024587cm, colorNEIGTJJNTMQRBC] (7.902353, -4.646997) -- (8.676853, -4.646997) -- (8.676853, -6.540218) -- (7.066386, -6.540218);
\draw[opacity=1.000000, fill=none, line width=0.024587 cm, colorNEIGTJJNTMQRBC] (7.066386,-3.884791) -- (7.066386,-4.229013);
\verbatimfont{\normalsize\fontNEIGTJJNFNNKCLJT}
\node[opacity=1.000000, above right, colorNEIGTJJNTMQRBC] at(6.910473, -4.131822) {\verb|+|};
\draw[opacity=1.000000, fill=none, line width=0.024587cm, colorNEIGTJJNTMQRBC] (7.066386, -6.540218) -- (7.066386, -6.884440) -- (5.210046, -6.884440) -- (5.210046, -3.397963) -- (6.092729, -3.397963);
\verbatimfont{\normalsize\fontNEIGTJJNFNNKCLJT}
\node[opacity=1.000000, above right, colorNEIGTJJNTMQRBC] at(8.781565, -3.250439) {\verb|-|};
\draw[opacity=1.000000, fill=none, line width=0.024587cm, colorNEIGTJJNTMQRBC] (8.040042, -3.397963) -- (8.922726, -3.397963) -- (8.922726, -7.130313) -- (7.066386, -7.130313);
\draw[opacity=1.000000, fill=none, line width=0.024587 cm, colorNEIGTJJNTMQRBC] (7.066386,-2.566913) -- (7.066386,-2.911135);
\draw[fill=colorNEIGTJJNFOAEMES, draw=none, opacity=1.000000] (6.038637,-7.474535) -- (8.094134,-7.474535) -- (8.094134,-8.015455) -- (6.038637,-8.015455) -- cycle;
\draw[opacity=1.000000, fill=none, line width=0.024587cm, colorNEIGTJJNTMQRBC] (6.038637,-7.474535) -- (8.094134,-7.474535) -- (8.094134,-8.015455) -- (6.038637,-8.015455) -- cycle;
\verbatimfont{\normalsize\fontNEIGTJJNFNNKCLJT}
\node[opacity=1.000000, above right, colorNEIGTJJNTMQRBC] at(6.184131, -7.854337) {\verb|line := n - 1|};
\draw[opacity=1.000000, fill=none, line width=0.024587 cm, colorNEIGTJJNTMQRBC] (7.066386,-7.130313) -- (7.066386,-7.474535);
\draw[opacity=1.000000, fill=colorNEIGTJJNFOAEMES, draw=none] (6.132069, -8.826835) -- (7.066386, -9.293994) -- (8.000702, -8.826835) -- (7.066386, -8.359677) --cycle;
\draw[opacity=1.000000, fill=none, line width=0.024587cm, colorNEIGTJJNTMQRBC] (6.132069, -8.826835) -- (7.066386, -9.293994) -- (8.000702, -8.826835) -- (7.066386, -8.359677) --cycle;
\verbatimfont{\normalsize\fontNEIGTJJNFNNKCLJT}
\node[opacity=1.000000, above right, colorNEIGTJJNTMQRBC] at(6.519388, -8.906673) {\verb|line ≥ 0|};
\draw[fill=colorNEIGTJJNFOAEMES, draw=none, opacity=1.000000] (4.890411,-9.638216) -- (9.242360,-9.638216) -- (9.242360,-10.700386) -- (4.890411,-10.700386) -- cycle;
\draw[opacity=1.000000, fill=none, line width=0.024587cm, colorNEIGTJJNTMQRBC] (4.890411,-9.638216) -- (9.242360,-9.638216) -- (9.242360,-10.700386) -- (4.890411,-10.700386) -- cycle;
\verbatimfont{\normalsize\fontNEIGTJJNFNNKCLJT}
\node[opacity=1.000000, above right, colorNEIGTJJNTMQRBC] at(5.778295, -10.018018) {\verb|matrixABegIndex := |};
\verbatimfont{\normalsize\fontNEIGTJJNFNNKCLJT}
\node[opacity=1.000000, above right, colorNEIGTJJNTMQRBC] at(5.637134, -10.279802) {\verb|(elementsAmountOnA * |};
\verbatimfont{\normalsize\fontNEIGTJJNFNNKCLJT}
\node[opacity=1.000000, above right, colorNEIGTJJNTMQRBC] at(5.037201, -10.541585) {\verb|(elementsAmountOnA - 1)) div 2|};
\draw[opacity=1.000000, fill=colorNEIGTJJNFOAEMES, draw=none] (4.066737, -12.544433) -- (5.566561, -11.044608) -- (8.566210, -11.044608) -- (10.066034, -12.544433) -- (8.566210, -14.044257) -- (5.566561, -14.044257) --cycle;
\draw[opacity=1.000000, fill=none, line width=0.024587cm, colorNEIGTJJNTMQRBC] (4.066737, -12.544433) -- (5.566561, -11.044608) -- (8.566210, -11.044608) -- (10.066034, -12.544433) -- (8.566210, -14.044257) -- (5.566561, -14.044257) --cycle;
\verbatimfont{\normalsize\fontNEIGTJJNFNNKCLJT}
\node[opacity=1.000000, above right, colorNEIGTJJNTMQRBC] at(5.972390, -12.464453) {\verb|elementNumber :=|};
\verbatimfont{\normalsize\fontNEIGTJJNFNNKCLJT}
\node[opacity=1.000000, above right, colorNEIGTJJNTMQRBC] at(5.707714, -12.726237) {\verb|0, elementsAmountOnA|};
\draw[opacity=1.000000, fill=colorNEIGTJJNFOAEMES, draw=none] (4.991219, -15.426063) -- (7.066386, -16.463646) -- (9.141553, -15.426063) -- (7.066386, -14.388479) --cycle;
\draw[opacity=1.000000, fill=none, line width=0.024587cm, colorNEIGTJJNTMQRBC] (4.991219, -15.426063) -- (7.066386, -16.463646) -- (9.141553, -15.426063) -- (7.066386, -14.388479) --cycle;
\verbatimfont{\normalsize\fontNEIGTJJNFNNKCLJT}
\node[opacity=1.000000, above right, colorNEIGTJJNTMQRBC] at(5.972390, -15.375588) {\verb|elementNumber = |};
\verbatimfont{\normalsize\fontNEIGTJJNFNNKCLJT}
\node[opacity=1.000000, above right, colorNEIGTJJNTMQRBC] at(5.637134, -15.637371) {\verb|elementsAmountOnA - 1|};
\draw[opacity=1.000000, fill=none, line width=0.024587cm, colorNEIGTJJNTMQRBC] (4.991219, -15.426063) -- (3.653671, -15.426063) -- (3.653671, -16.807868);
\draw[opacity=1.000000, fill=colorNEIGTJJNFOAEMES, draw=none] (2.817703, -17.225852) -- (3.235687, -16.807868) -- (4.071654, -16.807868) -- (4.489638, -17.225852) -- (4.071654, -17.643836) -- (3.235687, -17.643836) --cycle;
\draw[opacity=1.000000, fill=none, line width=0.024587cm, colorNEIGTJJNTMQRBC] (2.817703, -17.225852) -- (3.235687, -16.807868) -- (4.071654, -16.807868) -- (4.489638, -17.225852) -- (4.071654, -17.643836) -- (3.235687, -17.643836) --cycle;
\verbatimfont{\normalsize\fontNEIGTJJNFNNKCLJT}
\node[opacity=1.000000, above right, colorNEIGTJJNTMQRBC] at(3.371349, -17.145872) {\verb|j :=|};
\verbatimfont{\normalsize\fontNEIGTJJNFNNKCLJT}
\node[opacity=1.000000, above right, colorNEIGTJJNTMQRBC] at(3.371349, -17.407656) {\verb|0, n|};
\draw[fill=colorNEIGTJJNFOAEMES, draw=none, opacity=1.000000] (0.732701,-17.988058) -- (6.574640,-17.988058) -- (6.574640,-19.310854) -- (0.732701,-19.310854) -- cycle;
\draw[opacity=1.000000, fill=none, line width=0.024587cm, colorNEIGTJJNTMQRBC] (0.732701,-17.988058) -- (6.574640,-17.988058) -- (6.574640,-19.310854) -- (0.732701,-19.310854) -- cycle;
\verbatimfont{\normalsize\fontNEIGTJJNFNNKCLJT}
\node[opacity=1.000000, above right, colorNEIGTJJNTMQRBC] at(2.365579, -18.367860) {\verb|matrixX[line][j] :=|};
\verbatimfont{\normalsize\fontNEIGTJJNFNNKCLJT}
\node[opacity=1.000000, above right, colorNEIGTJJNTMQRBC] at(0.865747, -18.629644) {\verb|(matrixB[line][j] - matrixX[line][j]) div|};
\verbatimfont{\normalsize\fontNEIGTJJNFNNKCLJT}
\node[opacity=1.000000, above right, colorNEIGTJJNTMQRBC] at(0.936328, -18.891427) {\verb|matrixA[totalCompressedMatrixElements - |};
\verbatimfont{\normalsize\fontNEIGTJJNFNNKCLJT}
\node[opacity=1.000000, above right, colorNEIGTJJNTMQRBC] at(1.218649, -19.153211) {\verb|matrixABegIndex - 1 - elementNumber]|};
\draw[opacity=1.000000, fill=none, line width=0.024587 cm, colorNEIGTJJNTMQRBC] (3.653671,-17.643836) -- (3.653671,-17.988058);
\verbatimfont{\normalsize\fontNEIGTJJNFNNKCLJT}
\node[opacity=1.000000, above right, colorNEIGTJJNTMQRBC] at(3.497757, -17.890867) {\verb|+|};
\draw[opacity=1.000000, fill=none, line width=0.024587cm, colorNEIGTJJNTMQRBC] (3.653671, -19.310854) -- (3.653671, -19.655076) -- (0.486828, -19.655076) -- (0.486828, -17.225852) -- (2.817703, -17.225852);
\verbatimfont{\normalsize\fontNEIGTJJNFNNKCLJT}
\node[opacity=1.000000, above right, colorNEIGTJJNTMQRBC] at(6.679352, -17.078328) {\verb|-|};
\draw[opacity=1.000000, fill=none, line width=0.024587cm, colorNEIGTJJNTMQRBC] (4.489638, -17.225852) -- (6.820513, -17.225852) -- (6.820513, -19.900948) -- (3.653671, -19.900948);
\verbatimfont{\normalsize\fontNEIGTJJNFNNKCLJT}
\node[opacity=1.000000, above right, colorNEIGTJJNTMQRBC] at(3.653671, -15.278539) {\verb|+|};
\draw[opacity=1.000000, fill=none, line width=0.024587cm, colorNEIGTJJNTMQRBC] (9.141553, -15.426063) -- (10.410256, -15.426063) -- (10.410256, -16.807868);
\verbatimfont{\normalsize\fontNEIGTJJNFNNKCLJT}
\node[opacity=1.000000, above right, colorNEIGTJJNTMQRBC] at(10.269096, -15.278539) {\verb|-|};
\draw[opacity=1.000000, fill=colorNEIGTJJNFOAEMES, draw=none] (9.574289, -17.225852) -- (9.992273, -16.807868) -- (10.828240, -16.807868) -- (11.246224, -17.225852) -- (10.828240, -17.643836) -- (9.992273, -17.643836) --cycle;
\draw[opacity=1.000000, fill=none, line width=0.024587cm, colorNEIGTJJNTMQRBC] (9.574289, -17.225852) -- (9.992273, -16.807868) -- (10.828240, -16.807868) -- (11.246224, -17.225852) -- (10.828240, -17.643836) -- (9.992273, -17.643836) --cycle;
\verbatimfont{\normalsize\fontNEIGTJJNFNNKCLJT}
\node[opacity=1.000000, above right, colorNEIGTJJNTMQRBC] at(10.127935, -17.145872) {\verb|j :=|};
\verbatimfont{\normalsize\fontNEIGTJJNFNNKCLJT}
\node[opacity=1.000000, above right, colorNEIGTJJNTMQRBC] at(10.127935, -17.407656) {\verb|0, n|};
\draw[fill=colorNEIGTJJNFOAEMES, draw=none, opacity=1.000000] (7.558131,-17.988058) -- (13.262381,-17.988058) -- (13.262381,-19.310854) -- (7.558131,-19.310854) -- cycle;
\draw[opacity=1.000000, fill=none, line width=0.024587cm, colorNEIGTJJNTMQRBC] (7.558131,-17.988058) -- (13.262381,-17.988058) -- (13.262381,-19.310854) -- (7.558131,-19.310854) -- cycle;
\verbatimfont{\normalsize\fontNEIGTJJNFNNKCLJT}
\node[opacity=1.000000, above right, colorNEIGTJJNTMQRBC] at(7.834074, -18.367860) {\verb|matrixX[line][j] := matrixX[line][j] +|};
\verbatimfont{\normalsize\fontNEIGTJJNFNNKCLJT}
\node[opacity=1.000000, above right, colorNEIGTJJNTMQRBC] at(7.692913, -18.629644) {\verb|matrixA[totalCompressedMatrixElements - |};
\verbatimfont{\normalsize\fontNEIGTJJNFNNKCLJT}
\node[opacity=1.000000, above right, colorNEIGTJJNTMQRBC] at(7.834074, -18.891427) {\verb|matrixABegIndex - 1 - elementNumber] *|};
\verbatimfont{\normalsize\fontNEIGTJJNFNNKCLJT}
\node[opacity=1.000000, above right, colorNEIGTJJNTMQRBC] at(8.169331, -19.153211) {\verb|matrixX[n - 1 - elementNumber][j]|};
\draw[opacity=1.000000, fill=none, line width=0.024587 cm, colorNEIGTJJNTMQRBC] (10.410256,-17.643836) -- (10.410256,-17.988058);
\verbatimfont{\normalsize\fontNEIGTJJNFNNKCLJT}
\node[opacity=1.000000, above right, colorNEIGTJJNTMQRBC] at(10.254343, -17.890867) {\verb|+|};
\draw[opacity=1.000000, fill=none, line width=0.024587cm, colorNEIGTJJNTMQRBC] (10.410256, -19.310854) -- (10.410256, -19.655076) -- (7.312259, -19.655076) -- (7.312259, -17.225852) -- (9.574289, -17.225852);
\verbatimfont{\normalsize\fontNEIGTJJNFNNKCLJT}
\node[opacity=1.000000, above right, colorNEIGTJJNTMQRBC] at(13.367094, -17.078328) {\verb|-|};
\draw[opacity=1.000000, fill=none, line width=0.024587cm, colorNEIGTJJNTMQRBC] (11.246224, -17.225852) -- (13.508254, -17.225852) -- (13.508254, -19.900948) -- (10.410256, -19.900948);
\draw[opacity=1.000000, fill=none, line width=0.024587cm, colorNEIGTJJNTMQRBC] (10.410256, -19.900948) -- (10.410256, -20.245170) -- (7.066386, -20.245170);
\draw[opacity=1.000000, fill=none, line width=0.024587cm, colorNEIGTJJNTMQRBC] (3.653671, -19.900948) -- (3.653671, -20.245170) -- (7.066386, -20.245170);
\draw[opacity=1.000000, fill=none, line width=0.024587 cm, colorNEIGTJJNTMQRBC] (7.066386,-14.044257) -- (7.066386,-14.388479);
\verbatimfont{\normalsize\fontNEIGTJJNFNNKCLJT}
\node[opacity=1.000000, above right, colorNEIGTJJNTMQRBC] at(6.910473, -14.291288) {\verb|+|};
\draw[opacity=1.000000, fill=none, line width=0.024587cm, colorNEIGTJJNTMQRBC] (7.066386, -20.245170) -- (7.066386, -20.589392) -- (0.240955, -20.589392) -- (0.240955, -12.544433) -- (4.066737, -12.544433);
\verbatimfont{\normalsize\fontNEIGTJJNFNNKCLJT}
\node[opacity=1.000000, above right, colorNEIGTJJNTMQRBC] at(13.612966, -12.396909) {\verb|-|};
\draw[opacity=1.000000, fill=none, line width=0.024587cm, colorNEIGTJJNTMQRBC] (10.066034, -12.544433) -- (13.754127, -12.544433) -- (13.754127, -20.835265) -- (7.066386, -20.835265);
\draw[opacity=1.000000, fill=none, line width=0.024587 cm, colorNEIGTJJNTMQRBC] (7.066386,-10.700386) -- (7.066386,-11.044608);
\draw[fill=colorNEIGTJJNFOAEMES, draw=none, opacity=1.000000] (4.079031,-21.179487) -- (10.053741,-21.179487) -- (10.053741,-21.720407) -- (4.079031,-21.720407) -- cycle;
\draw[opacity=1.000000, fill=none, line width=0.024587cm, colorNEIGTJJNTMQRBC] (4.079031,-21.179487) -- (10.053741,-21.179487) -- (10.053741,-21.720407) -- (4.079031,-21.720407) -- cycle;
\verbatimfont{\normalsize\fontNEIGTJJNFNNKCLJT}
\node[opacity=1.000000, above right, colorNEIGTJJNTMQRBC] at(4.225527, -21.559290) {\verb|elementsAmountOnA := elementsAmountOnA + 1|};
\draw[opacity=1.000000, fill=none, line width=0.024587 cm, colorNEIGTJJNTMQRBC] (7.066386,-20.835265) -- (7.066386,-21.179487);
\draw[fill=colorNEIGTJJNFOAEMES, draw=none, opacity=1.000000] (5.837021,-22.064629) -- (8.295750,-22.064629) -- (8.295750,-22.605550) -- (5.837021,-22.605550) -- cycle;
\draw[opacity=1.000000, fill=none, line width=0.024587cm, colorNEIGTJJNTMQRBC] (5.837021,-22.064629) -- (8.295750,-22.064629) -- (8.295750,-22.605550) -- (5.837021,-22.605550) -- cycle;
\verbatimfont{\normalsize\fontNEIGTJJNFNNKCLJT}
\node[opacity=1.000000, above right, colorNEIGTJJNTMQRBC] at(5.972390, -22.444432) {\verb|line := line - 1|};
\draw[opacity=1.000000, fill=none, line width=0.024587 cm, colorNEIGTJJNTMQRBC] (7.066386,-21.720407) -- (7.066386,-22.064629);
\draw[opacity=1.000000, fill=none, line width=0.024587 cm, colorNEIGTJJNTMQRBC] (7.066386,-9.293994) -- (7.066386,-9.638216);
\verbatimfont{\normalsize\fontNEIGTJJNFNNKCLJT}
\node[opacity=1.000000, above right, colorNEIGTJJNTMQRBC] at(6.910473, -9.541025) {\verb|+|};
\draw[opacity=1.000000, fill=none, line width=0.024587cm, colorNEIGTJJNTMQRBC] (7.066386, -22.605550) -- (7.066386, -22.949772) -- (-0.004917, -22.949772) -- (-0.004917, -8.187566) -- (7.066386, -8.187566);
\verbatimfont{\normalsize\fontNEIGTJJNFNNKCLJT}
\node[opacity=1.000000, above right, colorNEIGTJJNTMQRBC] at(13.858839, -8.679312) {\verb|-|};
\draw[opacity=1.000000, fill=none, line width=0.024587cm, colorNEIGTJJNTMQRBC] (8.000702, -8.826835) -- (14.000000, -8.826835) -- (14.000000, -23.195645) -- (7.066386, -23.195645);
\draw[opacity=1.000000, fill=none, line width=0.024587 cm, colorNEIGTJJNTMQRBC] (7.066386,-8.015455) -- (7.066386,-8.359677);
\draw[opacity=1.000000, rounded corners=0.270460 cm, fill=colorNEIGTJJNFOAEMES, draw=none] (5.780471,-23.539867) -- (8.352301,-23.539867) -- (8.352301,-24.080787) -- (5.780471,-24.080787) -- cycle;
\draw[opacity=1.000000, rounded corners=0.270460 cm, fill=none, line width=0.024587cm, colorNEIGTJJNTMQRBC] (5.780471,-23.539867) -- (8.352301,-23.539867) -- (8.352301,-24.080787) -- (5.780471,-24.080787) -- cycle;
\verbatimfont{\normalsize\fontNEIGTJJNFNNKCLJT}
\node[opacity=1.000000, above right, colorNEIGTJJNTMQRBC] at(6.042971, -23.919669) {\verb|Выход (matrixX)|};
\draw[opacity=1.000000, fill=none, line width=0.024587 cm, colorNEIGTJJNTMQRBC] (7.066386,-23.195645) -- (7.066386,-23.539867);
\end{tikzpicture}
% --------------------------

% Конец блок схемы
\end{document}}{Функция findX}{ris:task9}

    \subsection{Код программы}
    \begin{minted}{C}
void findX(double *matrixA, int n, double **matrixB, double **matrixX) {
    /*
    Здесь верхнетреугольная матрица matrixA

               {1, -2, 0}
     matrixA = {0, 4, 17} -> {1, -2, 0, 4, 17, 1}
               {0, 0, 1}

     */
    // Кол-во элементов в одной линии верхнетреугольной матрицы. Будем идти снизу-вверх и справа-налево по этой матрице
    int elementsAmountOnA = 1;
    // Общее количество элементов в упакованной верхнетреугольной матрице
    int totalCompressedMatrixElements = ((n + 1) * n) / 2;

    // Заполним матрицу нулями
    for (int i = 0; i < n; i++)
        for (int j = 0; j < n; j++)
            matrixX[i][j] = 0;

    // Счётчик для линии
    for (int line = n - 1; line >= 0; line--) {

        // Индекс с которого будет начинаться упакованная верхнетреугольная матрица в line
        int matrixABegIndex = (elementsAmountOnA * (elementsAmountOnA - 1)) / 2;

        // Далее будем перебирать все элементы верхнетреугольной матрицы в этой линии
        for (int elementNumber = 0; elementNumber < elementsAmountOnA; elementNumber++)
            // Если мы достигли конца линии (самый левый элемент), значит мы может легко вычислить элемент из x
            if (elementNumber == elementsAmountOnA - 1)
                for (int j = 0; j < n; j++)
                    matrixX[line][j] =
                            (matrixB[line][j] - matrixX[line][j]) /
                            matrixA[totalCompressedMatrixElements - matrixABegIndex - 1 - elementNumber];
        // Иначе будем вычислять части массива X суммируя произведение текущего элемента упакованной матрицы и
        // элемента ниже.
        else
                for (int j = 0; j < n; j++)
                    matrixX[line][j] += matrixA[totalCompressedMatrixElements - matrixABegIndex - 1 - elementNumber] *
                            matrixX[n - 1 - elementNumber][j];

        // На следующей линии будет на 1 элемент больше
        elementsAmountOnA++;
    }
}
    \end{minted}

    \subsection{Тестовые данные}
    \labeltable{
        \begin{tabular}{|c|c|}
            \hline
            Тестовые данные & Ожидаемый результат \\
            \hline
            $A=(), B=()$ & $X=()$ \\
            \hline
            $A=(4), B=(-8)$ & $X=(-2)$ \\
            \hline

            \begin{tabular}{c}
                $A=\begin{pmatrix}
                       4 & 1 & 3  \\
                       0 & 2 & 0  \\
                       0 & 0 & -9 \\
                \end{pmatrix}$ или \\
                $A=\{4, 1, 3, 2, 0, -9\}$ \\
                в упакованном виде, \\
                $B=\begin{pmatrix}
                       -31 & 99  & 18  \\
                       8   & 0   & 2   \\
                       81  & -45 & -63 \\
                \end{pmatrix}$
            \end{tabular} &
            $X=\begin{pmatrix}
                   -2 & 21 & -1 \\
                   4  & 0  & 1  \\
                   -9 & 5  & 7  \\
            \end{pmatrix}$ \\
            \hline
            \begin{tabular}{c}
                $A=\begin{pmatrix}
                       4 & 1 & 3  & 3 \\
                       0 & 2 & 0  & 0 \\
                       0 & 0 & -9 & 5 \\
                       0 & 0 & 0  & 6 \\
                \end{pmatrix}$ или \\
                $A=\{4, 1, 3, 3, 2, 0, 0,-9, 5, 6\}$ \\
                в упакованном виде, \\
                $B=\begin{pmatrix}
                       -19 & 72  & 54 & 87  \\
                       8   & 0   & 2  & 12  \\
                       101 & -90 & -3 & -37 \\
                       24  & -54 & 72 & 42  \\
                \end{pmatrix}$
            \end{tabular} &
            $X=\begin{pmatrix}
                   -2 & 21 & -1 & 9 \\
                   4  & 0  & 1  & 6 \\
                   -9 & 5  & 7  & 8 \\
                   4  & -9 & 12 & 7 \\
            \end{pmatrix}$ \\
            \hline
        \end{tabular}
    }{Тестовые данные findX}{tab:task9-test-table}

    \subsection{Результат выполнения тестов}
    \makenewfig{
        \includegraphics[width=120mm]{/task9tests}}{Результат выполнения тестов функции findX}


    \section{Алгоритмы обработки символьной информации}

    \subsection{Задание варианта}
    Заданный список русских фамилий (вместе с именами и отчествами) упорядочить по алфавиту.
    Проверить (и исправить, если нужно) написание собственных имен с прописных букв.

    \subsection{Обоснование}

    Задача разделяется на две подзадачи: сделать первые буквы ФИО заглавными и отсортировать список ФИО по фамилии.
    Русские символы в кодировке UTF-8 состоят из двух char и имеют следующие значения\\
    \label{g}
    \labeltable{
        \begin{tabular}{|c|c|}
            \hline
            Символ & HEX        \\
            \hline
            Ë      & 0xDO, 0x81 \\
            \hline
            А      & 0xDO, 0x90 \\
            \hline
            Б      & 0xDO, 0x91 \\
            \hline
            В      & 0xDO, 0x92 \\
            \hline
            Г      & 0xDO, 0x93 \\
            \hline
            Д      & 0xDO, 0x94 \\
            \hline
            E      & 0xDO, 0x95 \\
            \hline
            Ж      & 0xDO, 0x96 \\
            \hline
            З      & 0xDO, 0x97 \\
            \hline
            И      & 0xDO, 0x98 \\
            \hline
            Й      & 0xDO, 0x99 \\
            \hline
            К      & 0xDO, 0x9A \\
            \hline
            Л      & 0xDO, 0x9B \\
            \hline
            М      & 0xDO, 0x9C \\
            \hline
            Н      & 0xDO, 0x9D \\
            \hline
            О      & 0xDO, 0x9E \\
            \hline
            П      & 0xDO, 0x9F \\
            \hline
            Р      & 0xDO, 0xA0 \\
            \hline
            С      & 0xDO, 0xA1 \\
            \hline
            Т      & 0xDO, 0xA2 \\
            \hline
            У      & 0xDO, 0xA3 \\
            \hline
            Ф      & 0xDO, 0xA4 \\
            \hline
            Х      & 0xDO, 0xA5 \\
            \hline
            Ц      & 0xDO, 0xA6 \\
            \hline
            Ч      & 0xDO, 0xA7 \\
            \hline
            Ш      & 0xDO, 0xA8 \\
            \hline
            Щ      & 0xDO, 0xA9 \\
            \hline
            Ъ      & 0xDO, 0xAA \\
            \hline
            Ы      & 0xDO, 0xAB \\
            \hline
        \end{tabular}
        \begin{tabular}{|c|c|}
            \hline
            Ь & 0xDO, 0xAC \\
            \hline
            Э & 0xDO, 0xAD \\
            \hline
            Ю & 0xDO, 0xAE \\
            \hline
            Я & 0xDO, 0xAF \\
            \hline
        \end{tabular}
        \begin{tabular}{|c|c|}
            \hline
            Символ & HEX        \\
            \hline
            а      & 0xDO, 0xB0 \\
            \hline
            б      & 0xDO, 0xB1 \\
            \hline
            в      & 0xDO, 0xB2 \\
            \hline
            г      & 0xDO, 0xB3 \\
            \hline
            д      & 0xDO, 0xB4 \\
            \hline
            е      & 0xDO, 0xB5 \\
            \hline
            ж      & 0xDO, 0xB6 \\
            \hline
            з      & 0xDO, 0xB7 \\
            \hline
            и      & 0xDO, 0xB8 \\
            \hline
            й      & 0xDO, 0xB9 \\
            \hline
            к      & 0xDO, 0xBA \\
            \hline
            л      & 0xDO, 0xBB \\
            \hline
            м      & 0xDO, 0xBC \\
            \hline
            н      & 0xDO, 0xBD \\
            \hline
            о      & 0xDO, 0xBE \\
            \hline
            п      & 0xDO, 0xBF \\
            \hline
            р      & 0xD1, 0x80 \\
            \hline
            с      & 0xD1, 0x81 \\
            \hline
            т      & 0xD1, 0x82 \\
            \hline
            у      & 0xD1, 0x83 \\
            \hline
            ф      & 0xD1, 0x84 \\
            \hline
            х      & 0xD1, 0x85 \\
            \hline
            ц      & 0xD1, 0x86 \\
            \hline
            ч      & 0xD1, 0x87 \\
            \hline
            ш      & 0xD1, 0x88 \\
            \hline
            щ      & 0xD1, 0x89 \\
            \hline
            ъ      & 0xD1, 0x8A \\
            \hline
            ы      & 0xD1, 0x8B \\
            \hline
            ь      & 0xD1, 0x8C \\
            \hline
        \end{tabular}
        \begin{tabular}{|c|c|}
            \hline
            э & 0xD1, 0x8D \\
            \hline
            ю & 0xD1, 0x8E \\
            \hline
            я & 0xD1, 0x8F \\
            \hline
            ё & 0xD1, 0x91 \\
            \hline
        \end{tabular}
    }{Русские символы в UTF-8 кодировке}{tab:task10-char-table}
    Символ в UTF-8 нельзя уместить в char, однако можно поместить в short.
    Первая вспомогательная функция будет возвращать код русского символа в short - он будет содержать два char.
    Вторая вспомогательная функция - запись short в массив.
    Согласно \textit{таблице \ref{tab:task10-char-table}}, прописные буквы подразделяются на три неразрывные области: ё (0xD191), а-п [0xDOB0; 0xDOBF]
    и р-я [0xD180; 0xD18F].
    На основании этих областей будем отнимать от кода текущего символа разницу между кодом прописного и заглавного.
    Например для е получение кода заглавного Е будет выглядеть так: code(е) - (code(а) - code(А)).
    Полученный код заглавного символа запишем в массив.
    Перейдём к второй подзадаче.
    Для сортировки воспользуемся функцией qsort, если бы исходный текст был в кодировке ASCII, где каждый символ занимает 8 бит, в качестве компаратора мы могли бы использовать стандартный strcmp,
    однако мы используем русские символы в кодировке UTF-8, где используется 16 бит на русский символ.
    Кроме того, порядок символов согласно \textit{таблице \ref{tab:task10-char-table}} имеет другой порядок относительно алфавита,
    поэтому необходимо определить ещё одну функцию, которая определяет приоритет символов в алфавите.
    И уже с использованием этой функции можно сранивать русские строки аналогично strcmp.

    \subsection{Спецификация}

    \begin{enumerate}
        \item Заголовок:  \mintinline{C}|static uint16_t getRuCode(unsigned char *beg)|
        \item Назначение: возвращает значение двухбайтового символа \mintinline{C}|beg|.
    \end{enumerate}
    \hline
    \begin{enumerate}
        \item Заголовок:  \mintinline{C}|static void writeRuCodeToChar(uint16_t code, unsigned char *beg)|
        \item Назначение: записывает двухбайтовое значение символа \mintinline{C}|code| в \mintinline{C}|beg|.
    \end{enumerate}
    \hline
    \begin{enumerate}
        \item Заголовок:  \mintinline{C}|static void firstLetterToupperRu(unsigned char *beg)|
        \item Назначение: делает первый символ русской строки \mintinline{C}|beg| заглавным.
    \end{enumerate}
    \hline
    \begin{enumerate}
        \item Заголовок:  \mintinline{C}|void validateProperName(ProperName name)|
        \item Назначение: проверяет и исправляет имя собственное \mintinline{C}|name|.
    \end{enumerate}
    \hline
    \begin{enumerate}
        \item Заголовок:  \mintinline{C}|void validateFullName(FullName fullName)|
        \item Назначение: проверяет и исправляет фамилию, имя, отчество в \mintinline{C}|fullName|.
    \end{enumerate}
    \hline
    \begin{enumerate}
        \item Заголовок:  \mintinline{C}|int getRuPriority(unsigned char *a)|
        \item Назначение: возвращает приоритет русского символа \mintinline{C}|a|.
    \end{enumerate}
    \hline
    \begin{enumerate}
        \item Заголовок:  \mintinline{C}|int getRuPriority(unsigned char *a)|
        \item Назначение: возвращает приоритет русского символа \mintinline{C}|a|.
    \end{enumerate}
    \hline
    \begin{enumerate}
        \item Заголовок:  \mintinline{C}|static int fullNameComparator(const void *a, const void *b)|
        \item Назначение: возвращает 0, если фамилия \mintinline{C}|a| = фамилии \mintinline{C}|b|, значение > 0 если фамилия \mintinline{C}|a| > фамилии \mintinline{C}|b|, иначе - значение < 0.
    \end{enumerate}
    \hline
    \begin{enumerate}
        \item Заголовок:  \mintinline{C}|void sortBySurname(FullName *namesList, int size)|
        \item Назначение: проверяет и исправляет имена собственые а также сортирует по фамилии список ФИО \mintinline{C}|namesList| размером \mintinline{C}|size|.
    \end{enumerate}

    \subsection{Блок-схема}

    \makenewfig{% Созданная блок схема работает только с компиляторами XeTeX и LuaTeX.
\documentclass{article}

% Необходимые зависимости
\usepackage[utf8]{inputenc}
\usepackage[english,russian]{babel}
\usepackage{pgfplots}
\usepackage{verbatim}
\usetikzlibrary{positioning}
\usetikzlibrary{shapes.geometric}
\usetikzlibrary{shapes.misc}
\usetikzlibrary{calc}
\usetikzlibrary{chains}
\usetikzlibrary{matrix}
\usetikzlibrary{decorations.text}
\usepackage{fontspec}
\usetikzlibrary{backgrounds}

\begin{document}
% Блок-схема
% Если Вы хотите добавить блок схему в свой документ, скопируйте код между комментариями
% С линияи и вставьте в документ.

% --------------------------
\begin{tikzpicture}[every node/.style={inner sep=0,outer sep=0}, background rectangle/.style={opacity=1.000000, fill=colorLBEILTTGFOLGCP}, show background rectangle]
\makeatletter
\newcommand{\verbatimfont}[1]{\def\verbatim@font{#1}}
\makeatother
% Шрифты
\newfontfamily\fontLBEILTTGADCCDAB[Scale=0.641323, SizeFeatures={Size=10.000000}]{Consolas}

% Цвета
\definecolor{colorLBEILTTGFOLGCP}{rgb}{1.000000,1.000000,1.000000}
\definecolor{colorLBEILTTGHEISADPG}{rgb}{0.949020,0.949020,0.949020}
\definecolor{colorLBEILTTGKLJKFHQI}{rgb}{0.000000,0.000000,0.000000}
\draw[opacity=1.000000, rounded corners=0.248869 cm, fill=colorLBEILTTGHEISADPG, draw=none] (0.877828,0.004525) -- (3.117647,0.004525) -- (3.117647,-0.493213) -- (0.877828,-0.493213) -- cycle;
\draw[opacity=1.000000, rounded corners=0.248869 cm, fill=none, line width=0.022624cm, colorLBEILTTGKLJKFHQI] (0.877828,0.004525) -- (3.117647,0.004525) -- (3.117647,-0.493213) -- (0.877828,-0.493213) -- cycle;
\verbatimfont{\normalsize\fontLBEILTTGADCCDAB}
\node[opacity=1.000000, above right, colorLBEILTTGKLJKFHQI] at(1.114795, -0.364871) {\verb|getRuCode(beg)|};
\draw[fill=colorLBEILTTGHEISADPG, draw=none, opacity=1.000000] (-0.004525,-0.809955) -- (4.000000,-0.809955) -- (4.000000,-1.307692) -- (-0.004525,-1.307692) -- cycle;
\draw[opacity=1.000000, fill=none, line width=0.022624cm, colorLBEILTTGKLJKFHQI] (-0.004525,-0.809955) -- (4.000000,-0.809955) -- (4.000000,-1.307692) -- (-0.004525,-1.307692) -- cycle;
\verbatimfont{\normalsize\fontLBEILTTGADCCDAB}
\node[opacity=1.000000, above right, colorLBEILTTGKLJKFHQI] at(0.125900, -1.179350) {\verb|code := (beg[0] << 8) + beg[1]|};
\draw[opacity=1.000000, fill=none, line width=0.022624 cm, colorLBEILTTGKLJKFHQI] (1.997738,-0.493213) -- (1.997738,-0.809955);
\draw[opacity=1.000000, rounded corners=0.248869 cm, fill=colorLBEILTTGHEISADPG, draw=none] (1.002262,-1.624434) -- (2.993213,-1.624434) -- (2.993213,-2.122172) -- (1.002262,-2.122172) -- cycle;
\draw[opacity=1.000000, rounded corners=0.248869 cm, fill=none, line width=0.022624cm, colorLBEILTTGKLJKFHQI] (1.002262,-1.624434) -- (2.993213,-1.624434) -- (2.993213,-2.122172) -- (1.002262,-2.122172) -- cycle;
\verbatimfont{\normalsize\fontLBEILTTGADCCDAB}
\node[opacity=1.000000, above right, colorLBEILTTGKLJKFHQI] at(1.238407, -1.993830) {\verb|Вернуть code|};
\draw[opacity=1.000000, fill=none, line width=0.022624 cm, colorLBEILTTGKLJKFHQI] (1.997738,-1.307692) -- (1.997738,-1.624434);
\end{tikzpicture}
% --------------------------

% Конец блок схемы
\end{document}}{Функция getRuCode}{ris:task10_1_getRuCode}\\
    \makenewfig{% Созданная блок схема работает только с компиляторами XeTeX и LuaTeX.
\documentclass{article}

% Необходимые зависимости
\usepackage[utf8]{inputenc}
\usepackage[english,russian]{babel}
\usepackage{pgfplots}
\usepackage{verbatim}
\usetikzlibrary{positioning}
\usetikzlibrary{shapes.geometric}
\usetikzlibrary{shapes.misc}
\usetikzlibrary{calc}
\usetikzlibrary{chains}
\usetikzlibrary{matrix}
\usetikzlibrary{decorations.text}
\usepackage{fontspec}
\usetikzlibrary{backgrounds}

\begin{document}
% Блок-схема
% Если Вы хотите добавить блок схему в свой документ, скопируйте код между комментариями
% С линияи и вставьте в документ.

% --------------------------
\begin{tikzpicture}[every node/.style={inner sep=0,outer sep=0}, background rectangle/.style={opacity=1.000000, fill=colorSNRCFQQENPREA}, show background rectangle]
\makeatletter
\newcommand{\verbatimfont}[1]{\def\verbatim@font{#1}}
\makeatother
% Шрифты
\newfontfamily\fontSNRCFQTOAPGES[Scale=0.801653, SizeFeatures={Size=10.000000}]{Consolas}

% Цвета
\definecolor{colorSNRCFQQENPREA}{rgb}{1.000000,1.000000,1.000000}
\definecolor{colorSNRCFQDDRINGEM}{rgb}{0.949020,0.949020,0.949020}
\definecolor{colorSNRCFQGNHRTICT}{rgb}{0.000000,0.000000,0.000000}
\draw[opacity=1.000000, rounded corners=0.311086 cm, fill=colorSNRCFQDDRINGEM, draw=none] (0.008484,0.005656) -- (4.985860,0.005656) -- (4.985860,-0.616516) -- (0.008484,-0.616516) -- cycle;
\draw[opacity=1.000000, rounded corners=0.311086 cm, fill=none, line width=0.028281cm, colorSNRCFQGNHRTICT] (0.008484,0.005656) -- (4.985860,0.005656) -- (4.985860,-0.616516) -- (0.008484,-0.616516) -- cycle;
\verbatimfont{\normalsize\fontSNRCFQTOAPGES}
\node[opacity=1.000000, above right, colorSNRCFQGNHRTICT] at(0.307475, -0.464918) {\verb|writeRuCodeToChar(code, beg)|};
\draw[fill=colorSNRCFQDDRINGEM, draw=none, opacity=1.000000] (-0.005656,-1.012443) -- (5.000000,-1.012443) -- (5.000000,-1.634615) -- (-0.005656,-1.634615) -- cycle;
\draw[opacity=1.000000, fill=none, line width=0.028281cm, colorSNRCFQGNHRTICT] (-0.005656,-1.012443) -- (5.000000,-1.012443) -- (5.000000,-1.634615) -- (-0.005656,-1.634615) -- cycle;
\verbatimfont{\normalsize\fontSNRCFQTOAPGES}
\node[opacity=1.000000, above right, colorSNRCFQGNHRTICT] at(0.148546, -1.483018) {\verb|beg[0] := (code & 0xFF00) >> 8|};
\draw[opacity=1.000000, fill=none, line width=0.028281 cm, colorSNRCFQGNHRTICT] (2.497172,-0.616516) -- (2.497172,-1.012443);
\draw[fill=colorSNRCFQDDRINGEM, draw=none, opacity=1.000000] (0.537330,-2.030543) -- (4.457014,-2.030543) -- (4.457014,-2.652715) -- (0.537330,-2.652715) -- cycle;
\draw[opacity=1.000000, fill=none, line width=0.028281cm, colorSNRCFQGNHRTICT] (0.537330,-2.030543) -- (4.457014,-2.030543) -- (4.457014,-2.652715) -- (0.537330,-2.652715) -- cycle;
\verbatimfont{\normalsize\fontSNRCFQTOAPGES}
\node[opacity=1.000000, above right, colorSNRCFQGNHRTICT] at(0.695970, -2.501117) {\verb|beg[1] := code & 0x00FF|};
\draw[opacity=1.000000, fill=none, line width=0.028281 cm, colorSNRCFQGNHRTICT] (2.497172,-1.634615) -- (2.497172,-2.030543);
\draw[opacity=1.000000, rounded corners=0.311086 cm, fill=colorSNRCFQDDRINGEM, draw=none] (1.252828,-3.048643) -- (3.741516,-3.048643) -- (3.741516,-3.670814) -- (1.252828,-3.670814) -- cycle;
\draw[opacity=1.000000, rounded corners=0.311086 cm, fill=none, line width=0.028281cm, colorSNRCFQGNHRTICT] (1.252828,-3.048643) -- (3.741516,-3.048643) -- (3.741516,-3.670814) -- (1.252828,-3.670814) -- cycle;
\verbatimfont{\normalsize\fontSNRCFQTOAPGES}
\node[opacity=1.000000, above right, colorSNRCFQGNHRTICT] at(1.631889, -3.519217) {\verb|Выход (beg)|};
\draw[opacity=1.000000, fill=none, line width=0.028281 cm, colorSNRCFQGNHRTICT] (2.497172,-2.652715) -- (2.497172,-3.048643);
\end{tikzpicture}
% --------------------------

% Конец блок схемы
\end{document}}{Функция writeRuCodeToChar}{ris:task10_2_writeRuCodeToChar}\\
    \makenewfig{% Созданная блок схема работает только с компиляторами XeTeX и LuaTeX.
\documentclass{article}

% Необходимые зависимости
\usepackage[utf8]{inputenc}
\usepackage[english,russian]{babel}
\usepackage{pgfplots}
\usepackage{verbatim}
\usetikzlibrary{positioning}
\usetikzlibrary{shapes.geometric}
\usetikzlibrary{shapes.misc}
\usetikzlibrary{calc}
\usetikzlibrary{chains}
\usetikzlibrary{matrix}
\usetikzlibrary{decorations.text}
\usepackage{fontspec}
\usetikzlibrary{backgrounds}

\begin{document}
% Блок-схема
% Если Вы хотите добавить блок схему в свой документ, скопируйте код между комментариями
% С линияи и вставьте в документ.

% --------------------------
\begin{tikzpicture}[every node/.style={inner sep=0,outer sep=0}, background rectangle/.style={opacity=1.000000, fill=colorPDRPQOQCHTOPBNP}, show background rectangle]
\makeatletter
\newcommand{\verbatimfont}[1]{\def\verbatim@font{#1}}
\makeatother
% Шрифты
\newfontfamily\fontPDRPQOQCREFQFK[Scale=0.609648, SizeFeatures={Size=10.000000}]{Consolas}

% Цвета
\definecolor{colorPDRPQOQCHTOPBNP}{rgb}{1.000000,1.000000,1.000000}
\definecolor{colorPDRPQOQCJQHRHS}{rgb}{0.949020,0.949020,0.949020}
\definecolor{colorPDRPQOQCIFPMMG}{rgb}{0.000000,0.000000,0.000000}
\draw[opacity=1.000000, rounded corners=0.236577 cm, fill=colorPDRPQOQCJQHRHS, draw=none] (1.556894,0.004301) -- (4.989415,0.004301) -- (4.989415,-0.468853) -- (1.556894,-0.468853) -- cycle;
\draw[opacity=1.000000, rounded corners=0.236577 cm, fill=none, line width=0.021507cm, colorPDRPQOQCIFPMMG] (1.556894,0.004301) -- (4.989415,0.004301) -- (4.989415,-0.468853) -- (1.556894,-0.468853) -- cycle;
\verbatimfont{\normalsize\fontPDRPQOQCREFQFK}
\node[opacity=1.000000, above right, colorPDRPQOQCIFPMMG] at(1.791033, -0.358857) {\verb|firstLetterToupperRu(beg)|};
\draw[fill=colorPDRPQOQCJQHRHS, draw=none, opacity=1.000000] (1.771964,-0.769952) -- (4.774345,-0.769952) -- (4.774345,-1.243106) -- (1.771964,-1.243106) -- cycle;
\draw[opacity=1.000000, fill=none, line width=0.021507cm, colorPDRPQOQCIFPMMG] (1.771964,-0.769952) -- (4.774345,-0.769952) -- (4.774345,-1.243106) -- (1.771964,-1.243106) -- cycle;
\draw[opacity=1.000000, fill=none, line width=0.021507cm, colorPDRPQOQCIFPMMG] (1.842937,-0.769952) -- (4.703372,-0.769952) -- (4.703372,-1.243106) -- (1.842937,-1.243106) -- cycle;
\verbatimfont{\normalsize\fontPDRPQOQCREFQFK}
\node[opacity=1.000000, above right, colorPDRPQOQCIFPMMG] at(1.967476, -1.133110) {\verb|code := getRuCode(beg)|};
\draw[opacity=1.000000, fill=none, line width=0.021507 cm, colorPDRPQOQCIFPMMG] (3.273155,-0.468853) -- (3.273155,-0.769952);
\draw[opacity=1.000000, fill=colorPDRPQOQCJQHRHS, draw=none] (1.627867, -2.366848) -- (3.273155, -3.189492) -- (4.918442, -2.366848) -- (3.273155, -1.544205) --cycle;
\draw[opacity=1.000000, fill=none, line width=0.021507cm, colorPDRPQOQCIFPMMG] (1.627867, -2.366848) -- (3.273155, -3.189492) -- (4.918442, -2.366848) -- (3.273155, -1.544205) --cycle;
\verbatimfont{\normalsize\fontPDRPQOQCREFQFK}
\node[opacity=1.000000, above right, colorPDRPQOQCIFPMMG] at(1.967476, -2.467621) {\verb|code = getRuCode("ё") |};
\draw[opacity=1.000000, fill=none, line width=0.021507cm, colorPDRPQOQCIFPMMG] (1.627867, -2.366848) -- (1.412797, -2.366848) -- (1.412797, -3.490591);
\draw[fill=colorPDRPQOQCJQHRHS, draw=none, opacity=1.000000] (-0.004301,-3.490591) -- (2.829895,-3.490591) -- (2.829895,-4.191720) -- (-0.004301,-4.191720) -- cycle;
\draw[opacity=1.000000, fill=none, line width=0.021507cm, colorPDRPQOQCIFPMMG] (-0.004301,-3.490591) -- (2.829895,-3.490591) -- (2.829895,-4.191720) -- (-0.004301,-4.191720) -- cycle;
\draw[opacity=1.000000, fill=none, line width=0.021507cm, colorPDRPQOQCIFPMMG] (0.100868,-3.490591) -- (2.724725,-3.490591) -- (2.724725,-4.191720) -- (0.100868,-4.191720) -- cycle;
\verbatimfont{\normalsize\fontPDRPQOQCREFQFK}
\node[opacity=1.000000, above right, colorPDRPQOQCIFPMMG] at(0.336494, -3.853749) {\verb|writeRuCodeToChar(|};
\verbatimfont{\normalsize\fontPDRPQOQCREFQFK}
\node[opacity=1.000000, above right, colorPDRPQOQCIFPMMG] at(0.212984, -4.113673) {\verb|getRuCode("Ë"), beg)|};
\verbatimfont{\normalsize\fontPDRPQOQCREFQFK}
\node[opacity=1.000000, above right, colorPDRPQOQCIFPMMG] at(1.412797, -2.237806) {\verb|+|};
\draw[opacity=1.000000, fill=none, line width=0.021507cm, colorPDRPQOQCIFPMMG] (4.918442, -2.366848) -- (7.503157, -2.366848) -- (7.503157, -3.490591);
\verbatimfont{\normalsize\fontPDRPQOQCREFQFK}
\node[opacity=1.000000, above right, colorPDRPQOQCIFPMMG] at(7.362002, -2.237806) {\verb|-|};
\draw[opacity=1.000000, fill=colorPDRPQOQCJQHRHS, draw=none] (5.451386, -4.516476) -- (7.503157, -5.542361) -- (9.554927, -4.516476) -- (7.503157, -3.490591) --cycle;
\draw[opacity=1.000000, fill=none, line width=0.021507cm, colorPDRPQOQCIFPMMG] (5.451386, -4.516476) -- (7.503157, -5.542361) -- (9.554927, -4.516476) -- (7.503157, -3.490591) --cycle;
\verbatimfont{\normalsize\fontPDRPQOQCREFQFK}
\node[opacity=1.000000, above right, colorPDRPQOQCIFPMMG] at(6.021035, -4.503261) {\verb|code ≥ getRuCode("а") and|};
\verbatimfont{\normalsize\fontPDRPQOQCREFQFK}
\node[opacity=1.000000, above right, colorPDRPQOQCIFPMMG] at(6.250411, -4.763186) {\verb|code ≤ getRuCode("п")|};
\draw[opacity=1.000000, fill=none, line width=0.021507cm, colorPDRPQOQCIFPMMG] (5.451386, -4.516476) -- (5.236316, -4.516476) -- (5.236316, -5.843460);
\draw[fill=colorPDRPQOQCJQHRHS, draw=none, opacity=1.000000] (3.488225,-5.843460) -- (6.984407,-5.843460) -- (6.984407,-6.772563) -- (3.488225,-6.772563) -- cycle;
\draw[opacity=1.000000, fill=none, line width=0.021507cm, colorPDRPQOQCIFPMMG] (3.488225,-5.843460) -- (6.984407,-5.843460) -- (6.984407,-6.772563) -- (3.488225,-6.772563) -- cycle;
\draw[opacity=1.000000, fill=none, line width=0.021507cm, colorPDRPQOQCIFPMMG] (3.627590,-5.843460) -- (6.845042,-5.843460) -- (6.845042,-6.772563) -- (3.627590,-6.772563) -- cycle;
\verbatimfont{\normalsize\fontPDRPQOQCREFQFK}
\node[opacity=1.000000, above right, colorPDRPQOQCIFPMMG] at(4.160013, -6.206618) {\verb|writeRuCodeToChar(|};
\verbatimfont{\normalsize\fontPDRPQOQCREFQFK}
\node[opacity=1.000000, above right, colorPDRPQOQCIFPMMG] at(3.754194, -6.466542) {\verb|code - (getRuCode("а") - |};
\verbatimfont{\normalsize\fontPDRPQOQCREFQFK}
\node[opacity=1.000000, above right, colorPDRPQOQCIFPMMG] at(3.983570, -6.726467) {\verb|getRuCode("А")), beg)|};
\verbatimfont{\normalsize\fontPDRPQOQCREFQFK}
\node[opacity=1.000000, above right, colorPDRPQOQCIFPMMG] at(5.236316, -4.387434) {\verb|+|};
\draw[opacity=1.000000, fill=none, line width=0.021507cm, colorPDRPQOQCIFPMMG] (9.554927, -4.516476) -- (11.733159, -4.516476) -- (11.733159, -5.843460);
\verbatimfont{\normalsize\fontPDRPQOQCREFQFK}
\node[opacity=1.000000, above right, colorPDRPQOQCIFPMMG] at(11.592005, -4.387434) {\verb|-|};
\draw[opacity=1.000000, fill=colorPDRPQOQCJQHRHS, draw=none] (9.681389, -6.869345) -- (11.733159, -7.895230) -- (13.784930, -6.869345) -- (11.733159, -5.843460) --cycle;
\draw[opacity=1.000000, fill=none, line width=0.021507cm, colorPDRPQOQCIFPMMG] (9.681389, -6.869345) -- (11.733159, -7.895230) -- (13.784930, -6.869345) -- (11.733159, -5.843460) --cycle;
\verbatimfont{\normalsize\fontPDRPQOQCREFQFK}
\node[opacity=1.000000, above right, colorPDRPQOQCIFPMMG] at(10.251037, -6.856130) {\verb|code ≥ getRuCode("р") and|};
\verbatimfont{\normalsize\fontPDRPQOQCREFQFK}
\node[opacity=1.000000, above right, colorPDRPQOQCIFPMMG] at(10.480413, -7.116055) {\verb|code ≤ getRuCode("я")|};
\draw[opacity=1.000000, fill=none, line width=0.021507cm, colorPDRPQOQCIFPMMG] (9.681389, -6.869345) -- (9.466318, -6.869345) -- (9.466318, -8.196328);
\draw[fill=colorPDRPQOQCJQHRHS, draw=none, opacity=1.000000] (7.718227,-8.196328) -- (11.214410,-8.196328) -- (11.214410,-9.125432) -- (7.718227,-9.125432) -- cycle;
\draw[opacity=1.000000, fill=none, line width=0.021507cm, colorPDRPQOQCIFPMMG] (7.718227,-8.196328) -- (11.214410,-8.196328) -- (11.214410,-9.125432) -- (7.718227,-9.125432) -- cycle;
\draw[opacity=1.000000, fill=none, line width=0.021507cm, colorPDRPQOQCIFPMMG] (7.857593,-8.196328) -- (11.075044,-8.196328) -- (11.075044,-9.125432) -- (7.857593,-9.125432) -- cycle;
\verbatimfont{\normalsize\fontPDRPQOQCREFQFK}
\node[opacity=1.000000, above right, colorPDRPQOQCIFPMMG] at(8.390016, -8.559487) {\verb|writeRuCodeToChar(|};
\verbatimfont{\normalsize\fontPDRPQOQCREFQFK}
\node[opacity=1.000000, above right, colorPDRPQOQCIFPMMG] at(7.984197, -8.819411) {\verb|code - (getRuCode("р") - |};
\verbatimfont{\normalsize\fontPDRPQOQCREFQFK}
\node[opacity=1.000000, above right, colorPDRPQOQCIFPMMG] at(8.213573, -9.079336) {\verb|getRuCode("Р")), beg)|};
\verbatimfont{\normalsize\fontPDRPQOQCREFQFK}
\node[opacity=1.000000, above right, colorPDRPQOQCIFPMMG] at(9.466318, -6.740303) {\verb|+|};
\draw[opacity=1.000000, fill=none, line width=0.021507cm, colorPDRPQOQCIFPMMG] (13.784930, -6.869345) -- (14.000000, -6.869345) -- (14.000000, -8.196328);
\verbatimfont{\normalsize\fontPDRPQOQCREFQFK}
\node[opacity=1.000000, above right, colorPDRPQOQCIFPMMG] at(13.858846, -6.740303) {\verb|-|};
\draw[opacity=1.000000, fill=none, line width=0.021507cm, colorPDRPQOQCIFPMMG] (14.000000, -8.196328) -- (14.000000, -9.426530) -- (11.733159, -9.426530);
\draw[opacity=1.000000, fill=none, line width=0.021507cm, colorPDRPQOQCIFPMMG] (9.466318, -9.125432) -- (9.466318, -9.426530) -- (11.733159, -9.426530);
\draw[opacity=1.000000, fill=none, line width=0.021507cm, colorPDRPQOQCIFPMMG] (11.733159, -9.426530) -- (11.733159, -9.727629) -- (7.503157, -9.727629);
\draw[opacity=1.000000, fill=none, line width=0.021507cm, colorPDRPQOQCIFPMMG] (5.236316, -6.772563) -- (5.236316, -9.727629) -- (7.503157, -9.727629);
\draw[opacity=1.000000, fill=none, line width=0.021507cm, colorPDRPQOQCIFPMMG] (7.503157, -9.727629) -- (7.503157, -10.028727) -- (3.273155, -10.028727);
\draw[opacity=1.000000, fill=none, line width=0.021507cm, colorPDRPQOQCIFPMMG] (1.412797, -4.191720) -- (1.412797, -10.028727) -- (3.273155, -10.028727);
\draw[opacity=1.000000, fill=none, line width=0.021507 cm, colorPDRPQOQCIFPMMG] (3.273155,-1.243106) -- (3.273155,-1.544205);
\draw[opacity=1.000000, rounded corners=0.236577 cm, fill=colorPDRPQOQCJQHRHS, draw=none] (2.326845,-10.329826) -- (4.219464,-10.329826) -- (4.219464,-10.802980) -- (2.326845,-10.802980) -- cycle;
\draw[opacity=1.000000, rounded corners=0.236577 cm, fill=none, line width=0.021507cm, colorPDRPQOQCIFPMMG] (2.326845,-10.329826) -- (4.219464,-10.329826) -- (4.219464,-10.802980) -- (2.326845,-10.802980) -- cycle;
\verbatimfont{\normalsize\fontPDRPQOQCREFQFK}
\node[opacity=1.000000, above right, colorPDRPQOQCIFPMMG] at(2.549738, -10.692984) {\verb|Вернуть code|};
\draw[opacity=1.000000, fill=none, line width=0.021507 cm, colorPDRPQOQCIFPMMG] (3.273155,-10.028727) -- (3.273155,-10.329826);
\end{tikzpicture}
% --------------------------

% Конец блок схемы
\end{document}}{Функция firstLetterToupperRu}{ris:task10_3_firstLetterToupperRu}\\
    \makenewfig{% Созданная блок схема работает только с компиляторами XeTeX и LuaTeX.
\documentclass{article}

% Необходимые зависимости
\usepackage[utf8]{inputenc}
\usepackage[english,russian]{babel}
\usepackage{pgfplots}
\usepackage{verbatim}
\usetikzlibrary{positioning}
\usetikzlibrary{shapes.geometric}
\usetikzlibrary{shapes.misc}
\usetikzlibrary{calc}
\usetikzlibrary{chains}
\usetikzlibrary{matrix}
\usetikzlibrary{decorations.text}
\usepackage{fontspec}
\usetikzlibrary{backgrounds}

\begin{document}
% Блок-схема
% Если Вы хотите добавить блок схему в свой документ, скопируйте код между комментариями
% С линияи и вставьте в документ.

% --------------------------
\begin{tikzpicture}[every node/.style={inner sep=0,outer sep=0}, background rectangle/.style={opacity=1.000000, fill=colorKJJDTSEHSBDKN}, show background rectangle]
\makeatletter
\newcommand{\verbatimfont}[1]{\def\verbatim@font{#1}}
\makeatother
% Шрифты
\newfontfamily\fontKJJDTSEAHCDCB[Scale=0.749906, SizeFeatures={Size=10.000000}]{Consolas}

% Цвета
\definecolor{colorKJJDTSEHSBDKN}{rgb}{1.000000,1.000000,1.000000}
\definecolor{colorKJJDTSHMLRBN}{rgb}{0.949020,0.949020,0.949020}
\definecolor{colorKJJDTSRKAPMBMC}{rgb}{0.000000,0.000000,0.000000}
\draw[opacity=1.000000, rounded corners=0.291005 cm, fill=colorKJJDTSHMLRBN, draw=none] (0.460317,0.005291) -- (4.534392,0.005291) -- (4.534392,-0.576720) -- (0.460317,-0.576720) -- cycle;
\draw[opacity=1.000000, rounded corners=0.291005 cm, fill=none, line width=0.026455cm, colorKJJDTSRKAPMBMC] (0.460317,0.005291) -- (4.534392,0.005291) -- (4.534392,-0.576720) -- (0.460317,-0.576720) -- cycle;
\verbatimfont{\normalsize\fontKJJDTSEAHCDCB}
\node[opacity=1.000000, above right, colorKJJDTSRKAPMBMC] at(0.749257, -0.420087) {\verb|validateProperName(name)|};
\draw[fill=colorKJJDTSHMLRBN, draw=none, opacity=1.000000] (-0.005291,-0.947090) -- (5.000000,-0.947090) -- (5.000000,-1.529101) -- (-0.005291,-1.529101) -- cycle;
\draw[opacity=1.000000, fill=none, line width=0.026455cm, colorKJJDTSRKAPMBMC] (-0.005291,-0.947090) -- (5.000000,-0.947090) -- (5.000000,-1.529101) -- (-0.005291,-1.529101) -- cycle;
\draw[opacity=1.000000, fill=none, line width=0.026455cm, colorKJJDTSRKAPMBMC] (0.082011,-0.947090) -- (4.912698,-0.947090) -- (4.912698,-1.529101) -- (0.082011,-1.529101) -- cycle;
\verbatimfont{\normalsize\fontKJJDTSEAHCDCB}
\node[opacity=1.000000, above right, colorKJJDTSRKAPMBMC] at(0.237188, -1.372468) {\verb|firstLetterToupperRu(name.name)|};
\draw[opacity=1.000000, fill=none, line width=0.026455 cm, colorKJJDTSRKAPMBMC] (2.497354,-0.576720) -- (2.497354,-0.947090);
\draw[opacity=1.000000, rounded corners=0.291005 cm, fill=colorKJJDTSHMLRBN, draw=none] (1.333333,-1.899471) -- (3.661376,-1.899471) -- (3.661376,-2.481481) -- (1.333333,-2.481481) -- cycle;
\draw[opacity=1.000000, rounded corners=0.291005 cm, fill=none, line width=0.026455cm, colorKJJDTSRKAPMBMC] (1.333333,-1.899471) -- (3.661376,-1.899471) -- (3.661376,-2.481481) -- (1.333333,-2.481481) -- cycle;
\verbatimfont{\normalsize\fontKJJDTSEAHCDCB}
\node[opacity=1.000000, above right, colorKJJDTSRKAPMBMC] at(1.614477, -2.324849) {\verb|Выход (name)|};
\draw[opacity=1.000000, fill=none, line width=0.026455 cm, colorKJJDTSRKAPMBMC] (2.497354,-1.529101) -- (2.497354,-1.899471);
\end{tikzpicture}
% --------------------------

% Конец блок схемы
\end{document}}{Функция validateProperName}{ris:task10_4_validateProperName}\\
    \makenewfig{% Созданная блок схема работает только с компиляторами XeTeX и LuaTeX.
\documentclass{article}

% Необходимые зависимости
\usepackage[utf8]{inputenc}
\usepackage[english,russian]{babel}
\usepackage{pgfplots}
\usepackage{verbatim}
\usetikzlibrary{positioning}
\usetikzlibrary{shapes.geometric}
\usetikzlibrary{shapes.misc}
\usetikzlibrary{calc}
\usetikzlibrary{chains}
\usetikzlibrary{matrix}
\usetikzlibrary{decorations.text}
\usepackage{fontspec}
\usetikzlibrary{backgrounds}

\begin{document}
% Блок-схема
% Если Вы хотите добавить блок схему в свой документ, скопируйте код между комментариями
% С линияи и вставьте в документ.

% --------------------------
\begin{tikzpicture}[every node/.style={inner sep=0,outer sep=0}, background rectangle/.style={opacity=1.000000, fill=colorELHHCPDFGAIQ}, show background rectangle]
\makeatletter
\newcommand{\verbatimfont}[1]{\def\verbatim@font{#1}}
\makeatother
% Шрифты
\newfontfamily\fontELHHCPDCDDSO[Scale=0.608293, SizeFeatures={Size=10.000000}]{Consolas}

% Цвета
\definecolor{colorELHHCPDFGAIQ}{rgb}{1.000000,1.000000,1.000000}
\definecolor{colorELHHCPCNKENEP}{rgb}{0.949020,0.949020,0.949020}
\definecolor{colorELHHCPSLSECIS}{rgb}{0.000000,0.000000,0.000000}
\draw[opacity=1.000000, rounded corners=0.236052 cm, fill=colorELHHCPCNKENEP, draw=none] (0.727468,0.004292) -- (4.268240,0.004292) -- (4.268240,-0.467811) -- (0.727468,-0.467811) -- cycle;
\draw[opacity=1.000000, rounded corners=0.236052 cm, fill=none, line width=0.021459cm, colorELHHCPSLSECIS] (0.727468,0.004292) -- (4.268240,0.004292) -- (4.268240,-0.467811) -- (0.727468,-0.467811) -- cycle;
\verbatimfont{\normalsize\fontELHHCPDCDDSO}
\node[opacity=1.000000, above right, colorELHHCPSLSECIS] at(0.961954, -0.358744) {\verb|validateFullName(fullName)|};
\draw[fill=colorELHHCPCNKENEP, draw=none, opacity=1.000000] (0.349785,-0.768240) -- (4.645923,-0.768240) -- (4.645923,-1.240343) -- (0.349785,-1.240343) -- cycle;
\draw[opacity=1.000000, fill=none, line width=0.021459cm, colorELHHCPSLSECIS] (0.349785,-0.768240) -- (4.645923,-0.768240) -- (4.645923,-1.240343) -- (0.349785,-1.240343) -- cycle;
\draw[opacity=1.000000, fill=none, line width=0.021459cm, colorELHHCPSLSECIS] (0.420601,-0.768240) -- (4.575107,-0.768240) -- (4.575107,-1.240343) -- (0.420601,-1.240343) -- cycle;
\verbatimfont{\normalsize\fontELHHCPDCDDSO}
\node[opacity=1.000000, above right, colorELHHCPSLSECIS] at(0.538257, -1.131277) {\verb|validateProperName(fullName.name)|};
\draw[opacity=1.000000, fill=none, line width=0.021459 cm, colorELHHCPSLSECIS] (2.497854,-0.467811) -- (2.497854,-0.768240);
\draw[fill=colorELHHCPCNKENEP, draw=none, opacity=1.000000] (0.173820,-1.540773) -- (4.821888,-1.540773) -- (4.821888,-2.012876) -- (0.173820,-2.012876) -- cycle;
\draw[opacity=1.000000, fill=none, line width=0.021459cm, colorELHHCPSLSECIS] (0.173820,-1.540773) -- (4.821888,-1.540773) -- (4.821888,-2.012876) -- (0.173820,-2.012876) -- cycle;
\draw[opacity=1.000000, fill=none, line width=0.021459cm, colorELHHCPSLSECIS] (0.244635,-1.540773) -- (4.751073,-1.540773) -- (4.751073,-2.012876) -- (0.244635,-2.012876) -- cycle;
\verbatimfont{\normalsize\fontELHHCPDCDDSO}
\node[opacity=1.000000, above right, colorELHHCPSLSECIS] at(0.361716, -1.903809) {\verb|validateProperName(fullName.surname)|};
\draw[opacity=1.000000, fill=none, line width=0.021459 cm, colorELHHCPSLSECIS] (2.497854,-1.240343) -- (2.497854,-1.540773);
\draw[fill=colorELHHCPCNKENEP, draw=none, opacity=1.000000] (-0.004292,-2.313305) -- (5.000000,-2.313305) -- (5.000000,-2.785408) -- (-0.004292,-2.785408) -- cycle;
\draw[opacity=1.000000, fill=none, line width=0.021459cm, colorELHHCPSLSECIS] (-0.004292,-2.313305) -- (5.000000,-2.313305) -- (5.000000,-2.785408) -- (-0.004292,-2.785408) -- cycle;
\draw[opacity=1.000000, fill=none, line width=0.021459cm, colorELHHCPSLSECIS] (0.066524,-2.313305) -- (4.929185,-2.313305) -- (4.929185,-2.785408) -- (0.066524,-2.785408) -- cycle;
\verbatimfont{\normalsize\fontELHHCPDCDDSO}
\node[opacity=1.000000, above right, colorELHHCPSLSECIS] at(0.185176, -2.676341) {\verb|validateProperName(fullName.patronymic)|};
\draw[opacity=1.000000, fill=none, line width=0.021459 cm, colorELHHCPSLSECIS] (2.497854,-2.012876) -- (2.497854,-2.313305);
\draw[opacity=1.000000, rounded corners=0.236052 cm, fill=colorELHHCPCNKENEP, draw=none] (1.317597,-3.085837) -- (3.678112,-3.085837) -- (3.678112,-3.557940) -- (1.317597,-3.557940) -- cycle;
\draw[opacity=1.000000, rounded corners=0.236052 cm, fill=none, line width=0.021459cm, colorELHHCPSLSECIS] (1.317597,-3.085837) -- (3.678112,-3.085837) -- (3.678112,-3.557940) -- (1.317597,-3.557940) -- cycle;
\verbatimfont{\normalsize\fontELHHCPDCDDSO}
\node[opacity=1.000000, above right, colorELHHCPSLSECIS] at(1.544536, -3.448873) {\verb|Выход (fullName)|};
\draw[opacity=1.000000, fill=none, line width=0.021459 cm, colorELHHCPSLSECIS] (2.497854,-2.785408) -- (2.497854,-3.085837);
\end{tikzpicture}
% --------------------------

% Конец блок схемы
\end{document}}{Функция validateFullName}{ris:task10_5_validateFullName}\\
    \makenewfig{\rotatebox{90}{
        % Созданная блок схема работает только с компиляторами XeTeX и LuaTeX.
\documentclass{article}

% Необходимые зависимости
\usepackage[utf8]{inputenc}
\usepackage[english,russian]{babel}
\usepackage{pgfplots}
\usepackage{verbatim}
\usetikzlibrary{positioning}
\usetikzlibrary{shapes.geometric}
\usetikzlibrary{shapes.misc}
\usetikzlibrary{calc}
\usetikzlibrary{chains}
\usetikzlibrary{matrix}
\usetikzlibrary{decorations.text}
\usepackage{fontspec}
\usetikzlibrary{backgrounds}

\begin{document}
% Блок-схема
% Если Вы хотите добавить блок схему в свой документ, скопируйте код между комментариями
% С линияи и вставьте в документ.

% --------------------------
\begin{tikzpicture}[every node/.style={inner sep=0,outer sep=0}, background rectangle/.style={opacity=1.000000, fill=colorICGEFLSTHCFBF}, show background rectangle]
\makeatletter
\newcommand{\verbatimfont}[1]{\def\verbatim@font{#1}}
\makeatother
% Шрифты
\newfontfamily\fontICGEFLSOHLDJCOH[Scale=0.392488, SizeFeatures={Size=10.000000}]{Consolas}

% Цвета
\definecolor{colorICGEFLSTHCFBF}{rgb}{1.000000,1.000000,1.000000}
\definecolor{colorICGEFLSRKMDGA}{rgb}{0.949020,0.949020,0.949020}
\definecolor{colorICGEFLSCONPFG}{rgb}{0.000000,0.000000,0.000000}
\draw[opacity=1.000000, rounded corners=0.152307 cm, fill=colorICGEFLSRKMDGA, draw=none] (0.707536,0.002769) -- (2.230607,0.002769) -- (2.230607,-0.301845) -- (0.707536,-0.301845) -- cycle;
\draw[opacity=1.000000, rounded corners=0.152307 cm, fill=none, line width=0.013846cm, colorICGEFLSCONPFG] (0.707536,0.002769) -- (2.230607,0.002769) -- (2.230607,-0.301845) -- (0.707536,-0.301845) -- cycle;
\verbatimfont{\normalsize\fontICGEFLSOHLDJCOH}
\node[opacity=1.000000, above right, colorICGEFLSCONPFG] at(0.851625, -0.213130) {\verb|getRuPriority(a)|};
\draw[opacity=1.000000, fill=colorICGEFLSRKMDGA, draw=none] (0.714459, -0.872997) -- (1.469071, -1.250303) -- (2.223684, -0.872997) -- (1.469071, -0.495690) --cycle;
\draw[opacity=1.000000, fill=none, line width=0.013846cm, colorICGEFLSCONPFG] (0.714459, -0.872997) -- (1.469071, -1.250303) -- (2.223684, -0.872997) -- (1.469071, -0.495690) --cycle;
\verbatimfont{\normalsize\fontICGEFLSOHLDJCOH}
\node[opacity=1.000000, above right, colorICGEFLSCONPFG] at(0.922190, -0.919974) {\verb|!a[0] or !a[1]|};
\draw[opacity=1.000000, fill=none, line width=0.013846cm, colorICGEFLSCONPFG] (0.714459, -0.872997) -- (0.575998, -0.872997) -- (0.575998, -1.444148);
\draw[fill=colorICGEFLSRKMDGA, draw=none, opacity=1.000000] (-0.002769,-1.444148) -- (1.154765,-1.444148) -- (1.154765,-1.748763) -- (-0.002769,-1.748763) -- cycle;
\draw[opacity=1.000000, fill=none, line width=0.013846cm, colorICGEFLSCONPFG] (-0.002769,-1.444148) -- (1.154765,-1.444148) -- (1.154765,-1.748763) -- (-0.002769,-1.748763) -- cycle;
\verbatimfont{\normalsize\fontICGEFLSOHLDJCOH}
\node[opacity=1.000000, above right, colorICGEFLSCONPFG] at(0.064399, -1.660048) {\verb|priority := 0|};
\verbatimfont{\normalsize\fontICGEFLSOHLDJCOH}
\node[opacity=1.000000, above right, colorICGEFLSCONPFG] at(0.575998, -0.789920) {\verb|+|};
\draw[opacity=1.000000, fill=none, line width=0.013846cm, colorICGEFLSCONPFG] (2.223684, -0.872997) -- (3.979231, -0.872997) -- (3.979231, -1.444148);
\verbatimfont{\normalsize\fontICGEFLSOHLDJCOH}
\node[opacity=1.000000, above right, colorICGEFLSCONPFG] at(3.873383, -0.789920) {\verb|-|};
\draw[fill=colorICGEFLSRKMDGA, draw=none, opacity=1.000000] (3.095850,-1.444148) -- (4.862612,-1.444148) -- (4.862612,-1.748763) -- (3.095850,-1.748763) -- cycle;
\draw[opacity=1.000000, fill=none, line width=0.013846cm, colorICGEFLSCONPFG] (3.095850,-1.444148) -- (4.862612,-1.444148) -- (4.862612,-1.748763) -- (3.095850,-1.748763) -- cycle;
\verbatimfont{\normalsize\fontICGEFLSOHLDJCOH}
\node[opacity=1.000000, above right, colorICGEFLSCONPFG] at(3.167730, -1.660048) {\verb|aCode := getRuCode(a)|};
\draw[opacity=1.000000, fill=colorICGEFLSRKMDGA, draw=none] (2.620928, -2.621759) -- (3.979231, -3.300910) -- (5.337533, -2.621759) -- (3.979231, -1.942608) --cycle;
\draw[opacity=1.000000, fill=none, line width=0.013846cm, colorICGEFLSCONPFG] (2.620928, -2.621759) -- (3.979231, -3.300910) -- (5.337533, -2.621759) -- (3.979231, -1.942608) --cycle;
\verbatimfont{\normalsize\fontICGEFLSOHLDJCOH}
\node[opacity=1.000000, above right, colorICGEFLSCONPFG] at(2.973675, -2.595352) {\verb|aCode ≥ getRuCode("А") and|};
\verbatimfont{\normalsize\fontICGEFLSOHLDJCOH}
\node[opacity=1.000000, above right, colorICGEFLSCONPFG] at(3.132447, -2.744790) {\verb|aCode ≤ getRuCode("Е")|};
\draw[opacity=1.000000, fill=none, line width=0.013846cm, colorICGEFLSCONPFG] (2.620928, -2.621759) -- (2.482467, -2.621759) -- (2.482467, -3.494756);
\draw[fill=colorICGEFLSRKMDGA, draw=none, opacity=1.000000] (1.607532,-3.494756) -- (3.357402,-3.494756) -- (3.357402,-3.946139) -- (1.607532,-3.946139) -- cycle;
\draw[opacity=1.000000, fill=none, line width=0.013846cm, colorICGEFLSCONPFG] (1.607532,-3.494756) -- (3.357402,-3.494756) -- (3.357402,-3.946139) -- (1.607532,-3.946139) -- cycle;
\draw[opacity=1.000000, fill=none, line width=0.013846cm, colorICGEFLSCONPFG] (1.675240,-3.494756) -- (3.289695,-3.494756) -- (3.289695,-3.946139) -- (1.675240,-3.946139) -- cycle;
\verbatimfont{\normalsize\fontICGEFLSOHLDJCOH}
\node[opacity=1.000000, above right, colorICGEFLSCONPFG] at(1.759173, -3.710655) {\verb|priority := aCode -|};
\verbatimfont{\normalsize\fontICGEFLSOHLDJCOH}
\node[opacity=1.000000, above right, colorICGEFLSCONPFG] at(1.794455, -3.860094) {\verb|getRuCode("А") + 1|};
\verbatimfont{\normalsize\fontICGEFLSOHLDJCOH}
\node[opacity=1.000000, above right, colorICGEFLSCONPFG] at(2.482467, -2.538683) {\verb|+|};
\draw[opacity=1.000000, fill=none, line width=0.013846cm, colorICGEFLSCONPFG] (5.337533, -2.621759) -- (5.894147, -2.621759) -- (5.894147, -3.494756);
\verbatimfont{\normalsize\fontICGEFLSOHLDJCOH}
\node[opacity=1.000000, above right, colorICGEFLSCONPFG] at(5.788299, -2.538683) {\verb|-|};
\draw[opacity=1.000000, fill=colorICGEFLSRKMDGA, draw=none] (4.834920, -4.024369) -- (5.894147, -4.553982) -- (6.953373, -4.024369) -- (5.894147, -3.494756) --cycle;
\draw[opacity=1.000000, fill=none, line width=0.013846cm, colorICGEFLSCONPFG] (4.834920, -4.024369) -- (5.894147, -4.553982) -- (6.953373, -4.024369) -- (5.894147, -3.494756) --cycle;
\verbatimfont{\normalsize\fontICGEFLSOHLDJCOH}
\node[opacity=1.000000, above right, colorICGEFLSCONPFG] at(5.047363, -4.071346) {\verb|aCode = getRuCode("Ë")|};
\draw[opacity=1.000000, fill=none, line width=0.013846cm, colorICGEFLSCONPFG] (4.834920, -4.024369) -- (4.696459, -4.024369) -- (4.696459, -4.747828);
\draw[fill=colorICGEFLSRKMDGA, draw=none, opacity=1.000000] (4.117692,-4.747828) -- (5.275226,-4.747828) -- (5.275226,-5.052442) -- (4.117692,-5.052442) -- cycle;
\draw[opacity=1.000000, fill=none, line width=0.013846cm, colorICGEFLSCONPFG] (4.117692,-4.747828) -- (5.275226,-4.747828) -- (5.275226,-5.052442) -- (4.117692,-5.052442) -- cycle;
\verbatimfont{\normalsize\fontICGEFLSOHLDJCOH}
\node[opacity=1.000000, above right, colorICGEFLSCONPFG] at(4.184860, -4.963727) {\verb|priority := 7|};
\verbatimfont{\normalsize\fontICGEFLSOHLDJCOH}
\node[opacity=1.000000, above right, colorICGEFLSCONPFG] at(4.696459, -3.941293) {\verb|+|};
\draw[opacity=1.000000, fill=none, line width=0.013846cm, colorICGEFLSCONPFG] (6.953373, -4.024369) -- (8.404306, -4.024369) -- (8.404306, -4.747828);
\verbatimfont{\normalsize\fontICGEFLSOHLDJCOH}
\node[opacity=1.000000, above right, colorICGEFLSCONPFG] at(8.298458, -3.941293) {\verb|-|};
\draw[opacity=1.000000, fill=colorICGEFLSRKMDGA, draw=none] (7.046004, -5.426979) -- (8.404306, -6.106130) -- (9.762609, -5.426979) -- (8.404306, -4.747828) --cycle;
\draw[opacity=1.000000, fill=none, line width=0.013846cm, colorICGEFLSCONPFG] (7.046004, -5.426979) -- (8.404306, -6.106130) -- (9.762609, -5.426979) -- (8.404306, -4.747828) --cycle;
\verbatimfont{\normalsize\fontICGEFLSOHLDJCOH}
\node[opacity=1.000000, above right, colorICGEFLSCONPFG] at(7.398750, -5.400572) {\verb|aCode ≥ getRuCode("Ж") and|};
\verbatimfont{\normalsize\fontICGEFLSOHLDJCOH}
\node[opacity=1.000000, above right, colorICGEFLSCONPFG] at(7.557522, -5.550010) {\verb|aCode ≤ getRuCode("Я")|};
\draw[opacity=1.000000, fill=none, line width=0.013846cm, colorICGEFLSCONPFG] (7.046004, -5.426979) -- (6.907543, -5.426979) -- (6.907543, -6.299976);
\draw[fill=colorICGEFLSRKMDGA, draw=none, opacity=1.000000] (6.032608,-6.299976) -- (7.782478,-6.299976) -- (7.782478,-6.751359) -- (6.032608,-6.751359) -- cycle;
\draw[opacity=1.000000, fill=none, line width=0.013846cm, colorICGEFLSCONPFG] (6.032608,-6.299976) -- (7.782478,-6.299976) -- (7.782478,-6.751359) -- (6.032608,-6.751359) -- cycle;
\draw[opacity=1.000000, fill=none, line width=0.013846cm, colorICGEFLSCONPFG] (6.100315,-6.299976) -- (7.714770,-6.299976) -- (7.714770,-6.751359) -- (6.100315,-6.751359) -- cycle;
\verbatimfont{\normalsize\fontICGEFLSOHLDJCOH}
\node[opacity=1.000000, above right, colorICGEFLSCONPFG] at(6.184248, -6.515875) {\verb|priority := aCode -|};
\verbatimfont{\normalsize\fontICGEFLSOHLDJCOH}
\node[opacity=1.000000, above right, colorICGEFLSCONPFG] at(6.219531, -6.665314) {\verb|getRuCode("Ж") + 8|};
\verbatimfont{\normalsize\fontICGEFLSOHLDJCOH}
\node[opacity=1.000000, above right, colorICGEFLSCONPFG] at(6.907543, -5.343903) {\verb|+|};
\draw[opacity=1.000000, fill=none, line width=0.013846cm, colorICGEFLSCONPFG] (9.762609, -5.426979) -- (10.914466, -5.426979) -- (10.914466, -6.299976);
\verbatimfont{\normalsize\fontICGEFLSOHLDJCOH}
\node[opacity=1.000000, above right, colorICGEFLSCONPFG] at(10.808618, -5.343903) {\verb|-|};
\draw[opacity=1.000000, fill=colorICGEFLSRKMDGA, draw=none] (9.556163, -6.979127) -- (10.914466, -7.658278) -- (12.272768, -6.979127) -- (10.914466, -6.299976) --cycle;
\draw[opacity=1.000000, fill=none, line width=0.013846cm, colorICGEFLSCONPFG] (9.556163, -6.979127) -- (10.914466, -7.658278) -- (12.272768, -6.979127) -- (10.914466, -6.299976) --cycle;
\verbatimfont{\normalsize\fontICGEFLSOHLDJCOH}
\node[opacity=1.000000, above right, colorICGEFLSCONPFG] at(9.908910, -6.952720) {\verb|aCode ≥ getRuCode("а") and|};
\verbatimfont{\normalsize\fontICGEFLSOHLDJCOH}
\node[opacity=1.000000, above right, colorICGEFLSCONPFG] at(10.067682, -7.102158) {\verb|aCode ≤ getRuCode("е")|};
\draw[opacity=1.000000, fill=none, line width=0.013846cm, colorICGEFLSCONPFG] (9.556163, -6.979127) -- (9.417702, -6.979127) -- (9.417702, -7.852124);
\draw[fill=colorICGEFLSRKMDGA, draw=none, opacity=1.000000] (8.542767,-7.852124) -- (10.292637,-7.852124) -- (10.292637,-8.303507) -- (8.542767,-8.303507) -- cycle;
\draw[opacity=1.000000, fill=none, line width=0.013846cm, colorICGEFLSCONPFG] (8.542767,-7.852124) -- (10.292637,-7.852124) -- (10.292637,-8.303507) -- (8.542767,-8.303507) -- cycle;
\draw[opacity=1.000000, fill=none, line width=0.013846cm, colorICGEFLSCONPFG] (8.610475,-7.852124) -- (10.224930,-7.852124) -- (10.224930,-8.303507) -- (8.610475,-8.303507) -- cycle;
\verbatimfont{\normalsize\fontICGEFLSOHLDJCOH}
\node[opacity=1.000000, above right, colorICGEFLSCONPFG] at(8.694408, -8.068023) {\verb|priority := aCode -|};
\verbatimfont{\normalsize\fontICGEFLSOHLDJCOH}
\node[opacity=1.000000, above right, colorICGEFLSCONPFG] at(8.694408, -8.217462) {\verb|getRuCode("а") + 34|};
\verbatimfont{\normalsize\fontICGEFLSOHLDJCOH}
\node[opacity=1.000000, above right, colorICGEFLSCONPFG] at(9.417702, -6.896050) {\verb|+|};
\draw[opacity=1.000000, fill=none, line width=0.013846cm, colorICGEFLSCONPFG] (12.272768, -6.979127) -- (12.866766, -6.979127) -- (12.866766, -7.852124);
\verbatimfont{\normalsize\fontICGEFLSOHLDJCOH}
\node[opacity=1.000000, above right, colorICGEFLSCONPFG] at(12.760918, -6.896050) {\verb|-|};
\draw[opacity=1.000000, fill=colorICGEFLSRKMDGA, draw=none] (11.807539, -8.381737) -- (12.866766, -8.911350) -- (13.925993, -8.381737) -- (12.866766, -7.852124) --cycle;
\draw[opacity=1.000000, fill=none, line width=0.013846cm, colorICGEFLSCONPFG] (11.807539, -8.381737) -- (12.866766, -8.911350) -- (13.925993, -8.381737) -- (12.866766, -7.852124) --cycle;
\verbatimfont{\normalsize\fontICGEFLSOHLDJCOH}
\node[opacity=1.000000, above right, colorICGEFLSCONPFG] at(12.019982, -8.428714) {\verb|aCode = getRuCode("ё")|};
\draw[opacity=1.000000, fill=none, line width=0.013846cm, colorICGEFLSCONPFG] (11.807539, -8.381737) -- (11.669078, -8.381737) -- (11.669078, -9.105196);
\draw[fill=colorICGEFLSRKMDGA, draw=none, opacity=1.000000] (11.052927,-9.105196) -- (12.285230,-9.105196) -- (12.285230,-9.409810) -- (11.052927,-9.409810) -- cycle;
\draw[opacity=1.000000, fill=none, line width=0.013846cm, colorICGEFLSCONPFG] (11.052927,-9.105196) -- (12.285230,-9.105196) -- (12.285230,-9.409810) -- (11.052927,-9.409810) -- cycle;
\verbatimfont{\normalsize\fontICGEFLSOHLDJCOH}
\node[opacity=1.000000, above right, colorICGEFLSCONPFG] at(11.122197, -9.321095) {\verb|priority := 40|};
\verbatimfont{\normalsize\fontICGEFLSOHLDJCOH}
\node[opacity=1.000000, above right, colorICGEFLSCONPFG] at(11.669078, -8.298660) {\verb|+|};
\draw[opacity=1.000000, fill=none, line width=0.013846cm, colorICGEFLSCONPFG] (13.925993, -8.381737) -- (15.376925, -8.381737) -- (15.376925, -9.105196);
\verbatimfont{\normalsize\fontICGEFLSOHLDJCOH}
\node[opacity=1.000000, above right, colorICGEFLSCONPFG] at(15.271077, -8.298660) {\verb|-|};
\draw[opacity=1.000000, fill=colorICGEFLSRKMDGA, draw=none] (14.018623, -9.784347) -- (15.376925, -10.463498) -- (16.735228, -9.784347) -- (15.376925, -9.105196) --cycle;
\draw[opacity=1.000000, fill=none, line width=0.013846cm, colorICGEFLSCONPFG] (14.018623, -9.784347) -- (15.376925, -10.463498) -- (16.735228, -9.784347) -- (15.376925, -9.105196) --cycle;
\verbatimfont{\normalsize\fontICGEFLSOHLDJCOH}
\node[opacity=1.000000, above right, colorICGEFLSCONPFG] at(14.371370, -9.757940) {\verb|aCode ≥ getRuCode("ж") and|};
\verbatimfont{\normalsize\fontICGEFLSOHLDJCOH}
\node[opacity=1.000000, above right, colorICGEFLSCONPFG] at(14.530142, -9.907378) {\verb|aCode ≤ getRuCode("п")|};
\draw[opacity=1.000000, fill=none, line width=0.013846cm, colorICGEFLSCONPFG] (14.018623, -9.784347) -- (13.880162, -9.784347) -- (13.880162, -10.657344);
\draw[fill=colorICGEFLSRKMDGA, draw=none, opacity=1.000000] (13.005227,-10.657344) -- (14.755097,-10.657344) -- (14.755097,-11.108727) -- (13.005227,-11.108727) -- cycle;
\draw[opacity=1.000000, fill=none, line width=0.013846cm, colorICGEFLSCONPFG] (13.005227,-10.657344) -- (14.755097,-10.657344) -- (14.755097,-11.108727) -- (13.005227,-11.108727) -- cycle;
\draw[opacity=1.000000, fill=none, line width=0.013846cm, colorICGEFLSCONPFG] (13.072934,-10.657344) -- (14.687390,-10.657344) -- (14.687390,-11.108727) -- (13.072934,-11.108727) -- cycle;
\verbatimfont{\normalsize\fontICGEFLSOHLDJCOH}
\node[opacity=1.000000, above right, colorICGEFLSCONPFG] at(13.156867, -10.873243) {\verb|priority := aCode -|};
\verbatimfont{\normalsize\fontICGEFLSOHLDJCOH}
\node[opacity=1.000000, above right, colorICGEFLSCONPFG] at(13.156867, -11.022682) {\verb|getRuCode("ж") + 41|};
\verbatimfont{\normalsize\fontICGEFLSOHLDJCOH}
\node[opacity=1.000000, above right, colorICGEFLSCONPFG] at(13.880162, -9.701270) {\verb|+|};
\draw[opacity=1.000000, fill=none, line width=0.013846cm, colorICGEFLSCONPFG] (16.735228, -9.784347) -- (17.887085, -9.784347) -- (17.887085, -10.657344);
\verbatimfont{\normalsize\fontICGEFLSOHLDJCOH}
\node[opacity=1.000000, above right, colorICGEFLSCONPFG] at(17.781237, -9.701270) {\verb|-|};
\draw[opacity=1.000000, fill=colorICGEFLSRKMDGA, draw=none] (16.528783, -11.336495) -- (17.887085, -12.015646) -- (19.245388, -11.336495) -- (17.887085, -10.657344) --cycle;
\draw[opacity=1.000000, fill=none, line width=0.013846cm, colorICGEFLSCONPFG] (16.528783, -11.336495) -- (17.887085, -12.015646) -- (19.245388, -11.336495) -- (17.887085, -10.657344) --cycle;
\verbatimfont{\normalsize\fontICGEFLSOHLDJCOH}
\node[opacity=1.000000, above right, colorICGEFLSCONPFG] at(16.881529, -11.310088) {\verb|aCode ≥ getRuCode("р") and|};
\verbatimfont{\normalsize\fontICGEFLSOHLDJCOH}
\node[opacity=1.000000, above right, colorICGEFLSCONPFG] at(17.040301, -11.459526) {\verb|aCode ≤ getRuCode("я")|};
\draw[opacity=1.000000, fill=none, line width=0.013846cm, colorICGEFLSCONPFG] (16.528783, -11.336495) -- (16.390322, -11.336495) -- (16.390322, -12.209492);
\draw[fill=colorICGEFLSRKMDGA, draw=none, opacity=1.000000] (15.515386,-12.209492) -- (17.265257,-12.209492) -- (17.265257,-12.660874) -- (15.515386,-12.660874) -- cycle;
\draw[opacity=1.000000, fill=none, line width=0.013846cm, colorICGEFLSCONPFG] (15.515386,-12.209492) -- (17.265257,-12.209492) -- (17.265257,-12.660874) -- (15.515386,-12.660874) -- cycle;
\draw[opacity=1.000000, fill=none, line width=0.013846cm, colorICGEFLSCONPFG] (15.583094,-12.209492) -- (17.197549,-12.209492) -- (17.197549,-12.660874) -- (15.583094,-12.660874) -- cycle;
\verbatimfont{\normalsize\fontICGEFLSOHLDJCOH}
\node[opacity=1.000000, above right, colorICGEFLSCONPFG] at(15.667027, -12.425391) {\verb|priority := aCode -|};
\verbatimfont{\normalsize\fontICGEFLSOHLDJCOH}
\node[opacity=1.000000, above right, colorICGEFLSCONPFG] at(15.667027, -12.574829) {\verb|getRuCode("р") + 51|};
\verbatimfont{\normalsize\fontICGEFLSOHLDJCOH}
\node[opacity=1.000000, above right, colorICGEFLSCONPFG] at(16.390322, -11.253418) {\verb|+|};
\draw[opacity=1.000000, fill=none, line width=0.013846cm, colorICGEFLSCONPFG] (19.245388, -11.336495) -- (19.383849, -11.336495) -- (19.383849, -12.209492);
\verbatimfont{\normalsize\fontICGEFLSOHLDJCOH}
\node[opacity=1.000000, above right, colorICGEFLSCONPFG] at(19.278001, -11.253418) {\verb|-|};
\draw[fill=colorICGEFLSRKMDGA, draw=none, opacity=1.000000] (18.767697,-12.209492) -- (20.000000,-12.209492) -- (20.000000,-12.514106) -- (18.767697,-12.514106) -- cycle;
\draw[opacity=1.000000, fill=none, line width=0.013846cm, colorICGEFLSCONPFG] (18.767697,-12.209492) -- (20.000000,-12.209492) -- (20.000000,-12.514106) -- (18.767697,-12.514106) -- cycle;
\verbatimfont{\normalsize\fontICGEFLSOHLDJCOH}
\node[opacity=1.000000, above right, colorICGEFLSCONPFG] at(18.836967, -12.425391) {\verb|priority := 67|};
\draw[opacity=1.000000, fill=none, line width=0.013846cm, colorICGEFLSCONPFG] (19.383849, -12.514106) -- (19.383849, -12.854720) -- (17.887085, -12.854720);
\draw[opacity=1.000000, fill=none, line width=0.013846cm, colorICGEFLSCONPFG] (16.390322, -12.660874) -- (16.390322, -12.854720) -- (17.887085, -12.854720);
\draw[opacity=1.000000, fill=none, line width=0.013846cm, colorICGEFLSCONPFG] (17.887085, -12.854720) -- (17.887085, -13.048565) -- (15.376925, -13.048565);
\draw[opacity=1.000000, fill=none, line width=0.013846cm, colorICGEFLSCONPFG] (13.880162, -11.108727) -- (13.880162, -13.048565) -- (15.376925, -13.048565);
\draw[opacity=1.000000, fill=none, line width=0.013846cm, colorICGEFLSCONPFG] (15.376925, -13.048565) -- (15.376925, -13.242411) -- (12.866766, -13.242411);
\draw[opacity=1.000000, fill=none, line width=0.013846cm, colorICGEFLSCONPFG] (11.669078, -9.409810) -- (11.669078, -13.242411) -- (12.866766, -13.242411);
\draw[opacity=1.000000, fill=none, line width=0.013846cm, colorICGEFLSCONPFG] (12.866766, -13.242411) -- (12.866766, -13.436256) -- (10.914466, -13.436256);
\draw[opacity=1.000000, fill=none, line width=0.013846cm, colorICGEFLSCONPFG] (9.417702, -8.303507) -- (9.417702, -13.436256) -- (10.914466, -13.436256);
\draw[opacity=1.000000, fill=none, line width=0.013846cm, colorICGEFLSCONPFG] (10.914466, -13.436256) -- (10.914466, -13.630101) -- (8.404306, -13.630101);
\draw[opacity=1.000000, fill=none, line width=0.013846cm, colorICGEFLSCONPFG] (6.907543, -6.751359) -- (6.907543, -13.630101) -- (8.404306, -13.630101);
\draw[opacity=1.000000, fill=none, line width=0.013846cm, colorICGEFLSCONPFG] (8.404306, -13.630101) -- (8.404306, -13.823947) -- (5.894147, -13.823947);
\draw[opacity=1.000000, fill=none, line width=0.013846cm, colorICGEFLSCONPFG] (4.696459, -5.052442) -- (4.696459, -13.823947) -- (5.894147, -13.823947);
\draw[opacity=1.000000, fill=none, line width=0.013846cm, colorICGEFLSCONPFG] (5.894147, -13.823947) -- (5.894147, -14.017792) -- (3.979231, -14.017792);
\draw[opacity=1.000000, fill=none, line width=0.013846cm, colorICGEFLSCONPFG] (2.482467, -3.946139) -- (2.482467, -14.017792) -- (3.979231, -14.017792);
\draw[opacity=1.000000, fill=none, line width=0.013846 cm, colorICGEFLSCONPFG] (3.979231,-1.748763) -- (3.979231,-1.942608);
\draw[opacity=1.000000, fill=none, line width=0.013846cm, colorICGEFLSCONPFG] (3.979231, -14.017792) -- (3.979231, -14.211638) -- (1.469071, -14.211638);
\draw[opacity=1.000000, fill=none, line width=0.013846cm, colorICGEFLSCONPFG] (0.575998, -1.748763) -- (0.575998, -14.211638) -- (1.469071, -14.211638);
\draw[opacity=1.000000, fill=none, line width=0.013846 cm, colorICGEFLSCONPFG] (1.469071,-0.301845) -- (1.469071,-0.495690);
\draw[opacity=1.000000, rounded corners=0.152307 cm, fill=colorICGEFLSRKMDGA, draw=none] (0.707536,-14.405483) -- (2.230607,-14.405483) -- (2.230607,-14.710097) -- (0.707536,-14.710097) -- cycle;
\draw[opacity=1.000000, rounded corners=0.152307 cm, fill=none, line width=0.013846cm, colorICGEFLSCONPFG] (0.707536,-14.405483) -- (2.230607,-14.405483) -- (2.230607,-14.710097) -- (0.707536,-14.710097) -- cycle;
\verbatimfont{\normalsize\fontICGEFLSOHLDJCOH}
\node[opacity=1.000000, above right, colorICGEFLSCONPFG] at(0.851625, -14.621383) {\verb|Вернуть priority|};
\draw[opacity=1.000000, fill=none, line width=0.013846 cm, colorICGEFLSCONPFG] (1.469071,-14.211638) -- (1.469071,-14.405483);
\end{tikzpicture}
% --------------------------

% Конец блок схемы
\end{document}
    }}{Функция getRuPriority}{ris:task10_6_getRuPriority}\\
    \makenewfig{% Созданная блок схема работает только с компиляторами XeTeX и LuaTeX.
\documentclass{article}

% Необходимые зависимости
\usepackage[utf8]{inputenc}
\usepackage[english,russian]{babel}
\usepackage{pgfplots}
\usepackage{verbatim}
\usetikzlibrary{positioning}
\usetikzlibrary{shapes.geometric}
\usetikzlibrary{shapes.misc}
\usetikzlibrary{calc}
\usetikzlibrary{chains}
\usetikzlibrary{matrix}
\usetikzlibrary{decorations.text}
\usepackage{fontspec}
\usetikzlibrary{backgrounds}

\begin{document}
% Блок-схема
% Если Вы хотите добавить блок схему в свой документ, скопируйте код между комментариями
% С линияи и вставьте в документ.

% --------------------------
\begin{tikzpicture}[every node/.style={inner sep=0,outer sep=0}, background rectangle/.style={opacity=1.000000, fill=colorIKRMARLDFOGK}, show background rectangle]
\makeatletter
\newcommand{\verbatimfont}[1]{\def\verbatim@font{#1}}
\makeatother
% Шрифты
\newfontfamily\fontIKRMARJADIIEH[Scale=1.045995, SizeFeatures={Size=10.000000}]{Consolas}

% Цвета
\definecolor{colorIKRMARLDFOGK}{rgb}{1.000000,1.000000,1.000000}
\definecolor{colorIKRMARNRRSMLBK}{rgb}{0.949020,0.949020,0.949020}
\definecolor{colorIKRMARFEGDKI}{rgb}{0.000000,0.000000,0.000000}
\draw[opacity=1.000000, rounded corners=0.405904 cm, fill=colorIKRMARNRRSMLBK, draw=none] (2.154982,0.007380) -- (7.837638,0.007380) -- (7.837638,-0.804428) -- (2.154982,-0.804428) -- cycle;
\draw[opacity=1.000000, rounded corners=0.405904 cm, fill=none, line width=0.036900cm, colorIKRMARFEGDKI] (2.154982,0.007380) -- (7.837638,0.007380) -- (7.837638,-0.804428) -- (2.154982,-0.804428) -- cycle;
\verbatimfont{\normalsize\fontIKRMARJADIIEH}
\node[opacity=1.000000, above right, colorIKRMARFEGDKI] at(2.560347, -0.580224) {\verb|fullNameComparator(a, b)|};
\draw[fill=colorIKRMARNRRSMLBK, draw=none, opacity=1.000000] (1.730627,-1.321033) -- (8.261993,-1.321033) -- (8.261993,-2.132841) -- (1.730627,-2.132841) -- cycle;
\draw[opacity=1.000000, fill=none, line width=0.036900cm, colorIKRMARFEGDKI] (1.730627,-1.321033) -- (8.261993,-1.321033) -- (8.261993,-2.132841) -- (1.730627,-2.132841) -- cycle;
\verbatimfont{\normalsize\fontIKRMARJADIIEH}
\node[opacity=1.000000, above right, colorIKRMARFEGDKI] at(1.942530, -1.908637) {\verb|aNameSurname := a.surname.name|};
\draw[opacity=1.000000, fill=none, line width=0.036900 cm, colorIKRMARFEGDKI] (4.996310,-0.804428) -- (4.996310,-1.321033);
\draw[fill=colorIKRMARNRRSMLBK, draw=none, opacity=1.000000] (1.730627,-2.649446) -- (8.261993,-2.649446) -- (8.261993,-3.461255) -- (1.730627,-3.461255) -- cycle;
\draw[opacity=1.000000, fill=none, line width=0.036900cm, colorIKRMARFEGDKI] (1.730627,-2.649446) -- (8.261993,-2.649446) -- (8.261993,-3.461255) -- (1.730627,-3.461255) -- cycle;
\verbatimfont{\normalsize\fontIKRMARJADIIEH}
\node[opacity=1.000000, above right, colorIKRMARFEGDKI] at(1.942530, -3.237050) {\verb|bNameSurname := b.surname.name|};
\draw[opacity=1.000000, fill=none, line width=0.036900 cm, colorIKRMARFEGDKI] (4.996310,-2.132841) -- (4.996310,-2.649446);
\draw[fill=colorIKRMARNRRSMLBK, draw=none, opacity=1.000000] (1.852768,-3.977860) -- (8.139852,-3.977860) -- (8.139852,-5.180812) -- (1.852768,-5.180812) -- cycle;
\draw[opacity=1.000000, fill=none, line width=0.036900cm, colorIKRMARFEGDKI] (1.852768,-3.977860) -- (8.139852,-3.977860) -- (8.139852,-5.180812) -- (1.852768,-5.180812) -- cycle;
\draw[opacity=1.000000, fill=none, line width=0.036900cm, colorIKRMARFEGDKI] (2.033210,-3.977860) -- (7.959410,-3.977860) -- (7.959410,-5.180812) -- (2.033210,-5.180812) -- cycle;
\verbatimfont{\normalsize\fontIKRMARJADIIEH}
\node[opacity=1.000000, above right, colorIKRMARFEGDKI] at(3.672417, -4.565464) {\verb|aNameCode := |};
\verbatimfont{\normalsize\fontIKRMARJADIIEH}
\node[opacity=1.000000, above right, colorIKRMARFEGDKI] at(2.242613, -4.975946) {\verb|getRuPriority(aNameSurname)|};
\draw[opacity=1.000000, fill=none, line width=0.036900 cm, colorIKRMARFEGDKI] (4.996310,-3.461255) -- (4.996310,-3.977860);
\draw[fill=colorIKRMARNRRSMLBK, draw=none, opacity=1.000000] (1.852768,-5.697417) -- (8.139852,-5.697417) -- (8.139852,-6.900369) -- (1.852768,-6.900369) -- cycle;
\draw[opacity=1.000000, fill=none, line width=0.036900cm, colorIKRMARFEGDKI] (1.852768,-5.697417) -- (8.139852,-5.697417) -- (8.139852,-6.900369) -- (1.852768,-6.900369) -- cycle;
\draw[opacity=1.000000, fill=none, line width=0.036900cm, colorIKRMARFEGDKI] (2.033210,-5.697417) -- (7.959410,-5.697417) -- (7.959410,-6.900369) -- (2.033210,-6.900369) -- cycle;
\verbatimfont{\normalsize\fontIKRMARJADIIEH}
\node[opacity=1.000000, above right, colorIKRMARFEGDKI] at(3.672417, -6.285021) {\verb|bNameCode := |};
\verbatimfont{\normalsize\fontIKRMARJADIIEH}
\node[opacity=1.000000, above right, colorIKRMARFEGDKI] at(2.242613, -6.695503) {\verb|getRuPriority(bNameSurname)|};
\draw[opacity=1.000000, fill=none, line width=0.036900 cm, colorIKRMARFEGDKI] (4.996310,-5.180812) -- (4.996310,-5.697417);
\draw[opacity=1.000000, fill=colorIKRMARNRRSMLBK, draw=none] (0.361624, -9.734317) -- (4.996310, -12.051661) -- (9.630996, -9.734317) -- (4.996310, -7.416974) --cycle;
\draw[opacity=1.000000, fill=none, line width=0.036900cm, colorIKRMARFEGDKI] (0.361624, -9.734317) -- (4.996310, -12.051661) -- (9.630996, -9.734317) -- (4.996310, -7.416974) --cycle;
\verbatimfont{\normalsize\fontIKRMARJADIIEH}
\node[opacity=1.000000, above right, colorIKRMARFEGDKI] at(1.342365, -9.676165) {\verb|aNameCode ≠ 0 and bNameCode ≠ 0 and |};
\verbatimfont{\normalsize\fontIKRMARJADIIEH}
\node[opacity=1.000000, above right, colorIKRMARFEGDKI] at(2.860429, -10.086647) {\verb|aNameCode = bNameCode|};
\draw[fill=colorIKRMARNRRSMLBK, draw=none, opacity=1.000000] (1.527675,-12.568266) -- (8.464945,-12.568266) -- (8.464945,-13.380074) -- (1.527675,-13.380074) -- cycle;
\draw[opacity=1.000000, fill=none, line width=0.036900cm, colorIKRMARFEGDKI] (1.527675,-12.568266) -- (8.464945,-12.568266) -- (8.464945,-13.380074) -- (1.527675,-13.380074) -- cycle;
\verbatimfont{\normalsize\fontIKRMARJADIIEH}
\node[opacity=1.000000, above right, colorIKRMARFEGDKI] at(1.748359, -13.155870) {\verb|aNameSurname := aNameSurname + 2|};
\draw[fill=colorIKRMARNRRSMLBK, draw=none, opacity=1.000000] (1.527675,-13.896679) -- (8.464945,-13.896679) -- (8.464945,-14.708487) -- (1.527675,-14.708487) -- cycle;
\draw[opacity=1.000000, fill=none, line width=0.036900cm, colorIKRMARFEGDKI] (1.527675,-13.896679) -- (8.464945,-13.896679) -- (8.464945,-14.708487) -- (1.527675,-14.708487) -- cycle;
\verbatimfont{\normalsize\fontIKRMARJADIIEH}
\node[opacity=1.000000, above right, colorIKRMARFEGDKI] at(1.748359, -14.484283) {\verb|bNameSurname := bNameSurname + 2|};
\draw[opacity=1.000000, fill=none, line width=0.036900 cm, colorIKRMARFEGDKI] (4.996310,-13.380074) -- (4.996310,-13.896679);
\draw[fill=colorIKRMARNRRSMLBK, draw=none, opacity=1.000000] (1.852768,-15.225092) -- (8.139852,-15.225092) -- (8.139852,-16.428044) -- (1.852768,-16.428044) -- cycle;
\draw[opacity=1.000000, fill=none, line width=0.036900cm, colorIKRMARFEGDKI] (1.852768,-15.225092) -- (8.139852,-15.225092) -- (8.139852,-16.428044) -- (1.852768,-16.428044) -- cycle;
\draw[opacity=1.000000, fill=none, line width=0.036900cm, colorIKRMARFEGDKI] (2.033210,-15.225092) -- (7.959410,-15.225092) -- (7.959410,-16.428044) -- (2.033210,-16.428044) -- cycle;
\verbatimfont{\normalsize\fontIKRMARJADIIEH}
\node[opacity=1.000000, above right, colorIKRMARFEGDKI] at(3.672417, -15.812696) {\verb|aNameCode := |};
\verbatimfont{\normalsize\fontIKRMARJADIIEH}
\node[opacity=1.000000, above right, colorIKRMARFEGDKI] at(2.242613, -16.223178) {\verb|getRuPriority(aNameSurname)|};
\draw[opacity=1.000000, fill=none, line width=0.036900 cm, colorIKRMARFEGDKI] (4.996310,-14.708487) -- (4.996310,-15.225092);
\draw[fill=colorIKRMARNRRSMLBK, draw=none, opacity=1.000000] (1.852768,-16.944649) -- (8.139852,-16.944649) -- (8.139852,-18.147601) -- (1.852768,-18.147601) -- cycle;
\draw[opacity=1.000000, fill=none, line width=0.036900cm, colorIKRMARFEGDKI] (1.852768,-16.944649) -- (8.139852,-16.944649) -- (8.139852,-18.147601) -- (1.852768,-18.147601) -- cycle;
\draw[opacity=1.000000, fill=none, line width=0.036900cm, colorIKRMARFEGDKI] (2.033210,-16.944649) -- (7.959410,-16.944649) -- (7.959410,-18.147601) -- (2.033210,-18.147601) -- cycle;
\verbatimfont{\normalsize\fontIKRMARJADIIEH}
\node[opacity=1.000000, above right, colorIKRMARFEGDKI] at(3.672417, -17.532253) {\verb|bNameCode := |};
\verbatimfont{\normalsize\fontIKRMARJADIIEH}
\node[opacity=1.000000, above right, colorIKRMARFEGDKI] at(2.242613, -17.942736) {\verb|getRuPriority(bNameSurname)|};
\draw[opacity=1.000000, fill=none, line width=0.036900 cm, colorIKRMARFEGDKI] (4.996310,-16.428044) -- (4.996310,-16.944649);
\draw[opacity=1.000000, fill=none, line width=0.036900 cm, colorIKRMARFEGDKI] (4.996310,-12.051661) -- (4.996310,-12.568266);
\verbatimfont{\normalsize\fontIKRMARJADIIEH}
\node[opacity=1.000000, above right, colorIKRMARFEGDKI] at(4.762347, -12.440002) {\verb|+|};
\draw[opacity=1.000000, fill=none, line width=0.036900cm, colorIKRMARFEGDKI] (4.996310, -18.147601) -- (4.996310, -18.664207) -- (-0.007380, -18.664207) -- (-0.007380, -7.158672) -- (4.996310, -7.158672);
\verbatimfont{\normalsize\fontIKRMARJADIIEH}
\node[opacity=1.000000, above right, colorIKRMARFEGDKI] at(9.788177, -9.512915) {\verb|-|};
\draw[opacity=1.000000, fill=none, line width=0.036900cm, colorIKRMARFEGDKI] (9.630996, -9.734317) -- (10.000000, -9.734317) -- (10.000000, -19.033210) -- (4.996310, -19.033210);
\draw[opacity=1.000000, fill=none, line width=0.036900 cm, colorIKRMARFEGDKI] (4.996310,-6.900369) -- (4.996310,-7.416974);
\draw[fill=colorIKRMARNRRSMLBK, draw=none, opacity=1.000000] (1.830258,-19.549815) -- (8.162362,-19.549815) -- (8.162362,-20.361624) -- (1.830258,-20.361624) -- cycle;
\draw[opacity=1.000000, fill=none, line width=0.036900cm, colorIKRMARFEGDKI] (1.830258,-19.549815) -- (8.162362,-19.549815) -- (8.162362,-20.361624) -- (1.830258,-20.361624) -- cycle;
\verbatimfont{\normalsize\fontIKRMARJADIIEH}
\node[opacity=1.000000, above right, colorIKRMARFEGDKI] at(2.048442, -20.137419) {\verb|diff := aNameCode - bNameCode|};
\draw[opacity=1.000000, fill=none, line width=0.036900 cm, colorIKRMARFEGDKI] (4.996310,-19.033210) -- (4.996310,-19.549815);
\draw[opacity=1.000000, rounded corners=0.405904 cm, fill=colorIKRMARNRRSMLBK, draw=none] (3.372694,-20.878229) -- (6.619926,-20.878229) -- (6.619926,-21.690037) -- (3.372694,-21.690037) -- cycle;
\draw[opacity=1.000000, rounded corners=0.405904 cm, fill=none, line width=0.036900cm, colorIKRMARFEGDKI] (3.372694,-20.878229) -- (6.619926,-20.878229) -- (6.619926,-21.690037) -- (3.372694,-21.690037) -- cycle;
\verbatimfont{\normalsize\fontIKRMARJADIIEH}
\node[opacity=1.000000, above right, colorIKRMARFEGDKI] at(3.778328, -21.465833) {\verb|Вернуть diff|};
\draw[opacity=1.000000, fill=none, line width=0.036900 cm, colorIKRMARFEGDKI] (4.996310,-20.361624) -- (4.996310,-20.878229);
\end{tikzpicture}
% --------------------------

% Конец блок схемы
\end{document}}{Функция fullNameComparator}{ris:task10_7_fullNameComparator}\\
    \makenewfig{% Созданная блок схема работает только с компиляторами XeTeX и LuaTeX.
\documentclass{article}

% Необходимые зависимости
\usepackage[utf8]{inputenc}
\usepackage[english,russian]{babel}
\usepackage{pgfplots}
\usepackage{verbatim}
\usetikzlibrary{positioning}
\usetikzlibrary{shapes.geometric}
\usetikzlibrary{shapes.misc}
\usetikzlibrary{calc}
\usetikzlibrary{chains}
\usetikzlibrary{matrix}
\usetikzlibrary{decorations.text}
\usepackage{fontspec}
\usetikzlibrary{backgrounds}

\begin{document}
% Блок-схема
% Если Вы хотите добавить блок схему в свой документ, скопируйте код между комментариями
% С линияи и вставьте в документ.

% --------------------------
\begin{tikzpicture}[every node/.style={inner sep=0,outer sep=0}, background rectangle/.style={opacity=1.000000, fill=colorDIJKSPBEQNQBT}, show background rectangle]
\makeatletter
\newcommand{\verbatimfont}[1]{\def\verbatim@font{#1}}
\makeatother
% Шрифты
\newfontfamily\fontDIJKSPBKLPOKEO[Scale=0.975542, SizeFeatures={Size=10.000000}]{Consolas}

% Цвета
\definecolor{colorDIJKSPBEQNQBT}{rgb}{1.000000,1.000000,1.000000}
\definecolor{colorDIJKSPBINQDPT}{rgb}{0.949020,0.949020,0.949020}
\definecolor{colorDIJKSPBCJPNILG}{rgb}{0.000000,0.000000,0.000000}
\draw[opacity=1.000000, rounded corners=0.378564 cm, fill=colorDIJKSPBINQDPT, draw=none] (0.278761,0.006883) -- (6.714356,0.006883) -- (6.714356,-0.750246) -- (0.278761,-0.750246) -- cycle;
\draw[opacity=1.000000, rounded corners=0.378564 cm, fill=none, line width=0.034415cm, colorDIJKSPBCJPNILG] (0.278761,0.006883) -- (6.714356,0.006883) -- (6.714356,-0.750246) -- (0.278761,-0.750246) -- cycle;
\verbatimfont{\normalsize\fontDIJKSPBKLPOKEO}
\node[opacity=1.000000, above right, colorDIJKSPBCJPNILG] at(0.653905, -0.532082) {\verb|sortBySurname(namesList, size)|};
\draw[opacity=1.000000, fill=colorDIJKSPBINQDPT, draw=none] (1.755162, -2.102753) -- (2.625860, -1.232055) -- (4.367257, -1.232055) -- (5.237955, -2.102753) -- (4.367257, -2.973451) -- (2.625860, -2.973451) --cycle;
\draw[opacity=1.000000, fill=none, line width=0.034415cm, colorDIJKSPBCJPNILG] (1.755162, -2.102753) -- (2.625860, -1.232055) -- (4.367257, -1.232055) -- (5.237955, -2.102753) -- (4.367257, -2.973451) -- (2.625860, -2.973451) --cycle;
\verbatimfont{\normalsize\fontDIJKSPBKLPOKEO}
\node[opacity=1.000000, above right, colorDIJKSPBCJPNILG] at(3.108121, -1.998159) {\verb|i :=|};
\verbatimfont{\normalsize\fontDIJKSPBKLPOKEO}
\node[opacity=1.000000, above right, colorDIJKSPBCJPNILG] at(2.825622, -2.371933) {\verb|0, size|};
\draw[fill=colorDIJKSPBINQDPT, draw=none, opacity=1.000000] (0.337266,-3.455261) -- (6.655851,-3.455261) -- (6.655851,-4.212389) -- (0.337266,-4.212389) -- cycle;
\draw[opacity=1.000000, fill=none, line width=0.034415cm, colorDIJKSPBCJPNILG] (0.337266,-3.455261) -- (6.655851,-3.455261) -- (6.655851,-4.212389) -- (0.337266,-4.212389) -- cycle;
\draw[opacity=1.000000, fill=none, line width=0.034415cm, colorDIJKSPBCJPNILG] (0.450836,-3.455261) -- (6.542281,-3.455261) -- (6.542281,-4.212389) -- (0.450836,-4.212389) -- cycle;
\verbatimfont{\normalsize\fontDIJKSPBKLPOKEO}
\node[opacity=1.000000, above right, colorDIJKSPBCJPNILG] at(0.653905, -3.994226) {\verb|validateFullName(namesList[i])|};
\draw[opacity=1.000000, fill=none, line width=0.034415 cm, colorDIJKSPBCJPNILG] (3.496559,-2.973451) -- (3.496559,-3.455261);
\verbatimfont{\normalsize\fontDIJKSPBKLPOKEO}
\node[opacity=1.000000, above right, colorDIJKSPBCJPNILG] at(3.264035, -3.326576) {\verb|+|};
\draw[opacity=1.000000, fill=none, line width=0.034415cm, colorDIJKSPBCJPNILG] (3.496559, -4.212389) -- (3.496559, -4.694199) -- (-0.006883, -4.694199) -- (-0.006883, -2.102753) -- (1.755162, -2.102753);
\verbatimfont{\normalsize\fontDIJKSPBKLPOKEO}
\node[opacity=1.000000, above right, colorDIJKSPBCJPNILG] at(6.788125, -1.896264) {\verb|-|};
\draw[opacity=1.000000, fill=none, line width=0.034415cm, colorDIJKSPBCJPNILG] (5.237955, -2.102753) -- (7.000000, -2.102753) -- (7.000000, -5.038348) -- (3.496559, -5.038348);
\draw[opacity=1.000000, fill=none, line width=0.034415 cm, colorDIJKSPBCJPNILG] (3.496559,-0.750246) -- (3.496559,-1.232055);
\draw[fill=colorDIJKSPBINQDPT, draw=none, opacity=1.000000] (0.888594,-5.520157) -- (6.104523,-5.520157) -- (6.104523,-7.006883) -- (0.888594,-7.006883) -- cycle;
\draw[opacity=1.000000, fill=none, line width=0.034415cm, colorDIJKSPBCJPNILG] (0.888594,-5.520157) -- (6.104523,-5.520157) -- (6.104523,-7.006883) -- (0.888594,-7.006883) -- cycle;
\draw[opacity=1.000000, fill=none, line width=0.034415cm, colorDIJKSPBCJPNILG] (1.111603,-5.520157) -- (5.881514,-5.520157) -- (5.881514,-7.006883) -- (1.111603,-7.006883) -- cycle;
\verbatimfont{\normalsize\fontDIJKSPBKLPOKEO}
\node[opacity=1.000000, above right, colorDIJKSPBCJPNILG] at(1.307186, -6.059123) {\verb|qsort(namesList, size, |};
\verbatimfont{\normalsize\fontDIJKSPBKLPOKEO}
\node[opacity=1.000000, above right, colorDIJKSPBCJPNILG] at(1.783904, -6.432896) {\verb|sizeof(FullName), |};
\verbatimfont{\normalsize\fontDIJKSPBKLPOKEO}
\node[opacity=1.000000, above right, colorDIJKSPBCJPNILG] at(1.695623, -6.806670) {\verb|fullNameComparator)|};
\draw[opacity=1.000000, fill=none, line width=0.034415 cm, colorDIJKSPBCJPNILG] (3.496559,-5.038348) -- (3.496559,-5.520157);
\draw[opacity=1.000000, rounded corners=0.378564 cm, fill=colorDIJKSPBINQDPT, draw=none] (1.507375,-7.488692) -- (5.485742,-7.488692) -- (5.485742,-8.245821) -- (1.507375,-8.245821) -- cycle;
\draw[opacity=1.000000, rounded corners=0.378564 cm, fill=none, line width=0.034415cm, colorDIJKSPBCJPNILG] (1.507375,-7.488692) -- (5.485742,-7.488692) -- (5.485742,-8.245821) -- (1.507375,-8.245821) -- cycle;
\verbatimfont{\normalsize\fontDIJKSPBKLPOKEO}
\node[opacity=1.000000, above right, colorDIJKSPBCJPNILG] at(1.872185, -8.027658) {\verb|Выход (namesList)|};
\draw[opacity=1.000000, fill=none, line width=0.034415 cm, colorDIJKSPBCJPNILG] (3.496559,-7.006883) -- (3.496559,-7.488692);
\end{tikzpicture}
% --------------------------

% Конец блок схемы
\end{document}}{Функция sortBySurname}{ris:task10_8_sortBySurname}

    \subsection{Код программы}

    \textbf{Код программы:}\\
    \begin{minted}{C}
#include <stdlib.h>
#include <stdint.h>

#include "Tasks.h"

// beg содержит ссылку на русский символ в кодировке UTF-8
static uint16_t getRuCode(unsigned char *beg) {
    // Русский символ в UTF-8 состоит из двух байтов, поэтому сохраняем результат в двухбайтовый беззнаковый тип.
    return (((uint16_t) beg[0]) << 8) + ((uint16_t) beg[1]);
}

// beg содержит ссылку на русский символ в кодировке UTF-8
static void writeRuCodeToChar(uint16_t code, unsigned char *beg) {
    // Русский символ в UTF-8 состоит из двух байтов, поэтому просто копируем их
    beg[0] = (code & 0xFF00) >> 8;
    beg[1] = code & 0x00FF;
}

static void firstLetterToupperRu(unsigned char *beg) {
    uint16_t code = getRuCode(beg);

    // Отдельная проверка для ё, потому что она совсем в другом месте в таблице UTF-8
    if (code == getRuCode("ё"))
        writeRuCodeToChar(getRuCode("Ë"), beg);
    // Проверка для а-п, их коды: 0xD0B0 - 0xD0BF
    else if (code >= getRuCode("а") && code <= getRuCode("п"))
        writeRuCodeToChar(code - (getRuCode("а") - getRuCode("А")), beg);
    // Почему-то между п и р большой пробел, причём только для прописных букв.
    // Поэтому выполняем отдельную проверку для них. Границы р-я: 0xD1080 - 0xD18F
    else if (code >= getRuCode("р") && code <= getRuCode("я"))
        writeRuCodeToChar(code - (getRuCode("р") - getRuCode("Р")), beg);
}

void validateProperName(ProperName name) {
    // Делаем первый символ имени заглавным
    firstLetterToupperRu(name.name);
}

void validateFullName(FullName fullName) {
    // Проверяем имя, фамилию и отчество на валидность
    validateProperName(fullName.name);
    validateProperName(fullName.surname);
    validateProperName(fullName.patronymic);
}

// Приоритет русских символов
int getRuPriority(unsigned char *a) {
    // Если хоть один из символов ноль-символ - возвращаем 0. Для удобства работы с аналогом strcmp.
    if (!*a || !a[1]) return 0;

    // Получаем код символа
    int aCode = getRuCode(a);

    if (aCode >= getRuCode("А") && aCode <= getRuCode("Е"))
        return aCode - getRuCode("А") + 1;
    else if (aCode == getRuCode("Ë"))
        return 7;
    else if (aCode >= getRuCode("Ж") && aCode <= getRuCode("Я"))
        return aCode - getRuCode("Ж") + 8;
    else if (aCode >= getRuCode("а") && aCode <= getRuCode("е"))
        return aCode - getRuCode("а") + 34;
    else if (aCode == getRuCode("ё"))
        return 40;
    else if (aCode >= getRuCode("ж") && aCode <= getRuCode("п"))
        return aCode - getRuCode("ж") + 41;
    else if (aCode >= getRuCode("р") && aCode <= getRuCode("я"))
        return aCode - getRuCode("р") + 51;

    // Если получить приоритет не удалось - ставим символ в самую последнюю очередь.
    return 67;
}

// Замена strcmp, но для русских символов в UTF-8
static int fullNameComparator(const void *a, const void *b) {
    unsigned char* aNameSurname = (*(FullName *) a).surname.name;
    unsigned char* bNameSurname = (*(FullName *) b).surname.name;

    int aNameCode = getRuPriority(aNameSurname);
    int bNameCode = getRuPriority(bNameSurname);
    while (aNameCode && bNameCode && aNameCode == bNameCode) {
        aNameSurname += 2;
        bNameSurname += 2;

        aNameCode = getRuPriority(aNameSurname);
        bNameCode = getRuPriority(bNameSurname);
    }

    return aNameCode - bNameCode;
}

// Так как кодировка в задании не была уточнена, реализовал алгоритм для кодировки UTF-8.
// получилось "самую капельку" сложнее, зато интереснее.
void sortBySurname(FullName *namesList, int size) {
    for (int i = 0; i < size; i++)
        validateFullName(namesList[i]);

    qsort(namesList, size, sizeof(FullName), fullNameComparator);
}
    \end{minted}

    \subsection{Тестовые данные}
    \labeltable{
        \begin{tabular}{|c|c|}
            \hline
            Тестовые данные & Ожидаемый результат \\
            \hline
            \begin{tabular}{c}
                namesList =                    \\
                фадей трифонов витальевич      \\
                инесса филатова фомовна        \\
                белла пестова даниловна        \\
                ибрагим пестовааееее романович \\
                степан пестоваб игнатович      \\
                фадей богданов витальевич      \\
                инесса ё фомовна               \\
                белла ёа даниловна             \\
                ибрагим ёааа романович         \\
                степан ёаав игнатович          \\
            \end{tabular} &
            \begin{tabular}{c}
                namesList =                    \\
                Фадей Богданов Витальевич      \\
                Инесса Ë Фомовна               \\
                Белла Ëа Даниловна             \\
                Ибрагим Ëааа Романович         \\
                Степан Ëаав Игнатович          \\
                Белла Пестова Даниловна        \\
                Ибрагим Пестовааееее Романович \\
                Степан Пестоваб Игнатович      \\
                Фадей Трифонов Витальевич      \\
                Инесса Филатова Фомовна        \\
            \end{tabular} \\
            \hline
        \end{tabular}
    }{Тестовые данные findX}{tab:task10-test-table}

    \subsection{Результат выполнения тестов}
    \makenewfig{
        \includegraphics[width=120mm]{/task10tests}}{Результат выполнения тестов функции sortBySurname}


    \section{Аналитическая геометрия}

    \subsection{Задание варианта}
    Даны точки $A (-4; 0), B(1; -3), C(4; -2)$. Построить прямую $l$, проходящую через $A$ и параллельно $BC$ в диапазоне
    $x \in [-1; 3]$ с шагом $\triangle = 0.25$

    \subsection{Обоснование}
    Для уравнения $y = kx + b$ найдём коэффициенты. Так как $y$ должен быть параллелен $BC$, то $k_y = k_{BC}$\\
    $k_{BC} = \frac{\triangle y}{\triangle x} = \frac{-3 - (-2)}{1 - 4} = \frac{-1}{-3} = \frac{1}{3}$\\
    Прямая $y$ должна проходить через точку $A$, следовательно\\
    $0 = \frac{1}{3} \cdot (-4) + b$\\
    $b = \frac{4}{3}$
    \bigbreak
    Таким образом получили уравнение: $y = \frac{1}{3}x + \frac{4}{3}$.
    Уравнение $y$ описывает искомую прямую $l$.\\

    \subsection{Таблица значений}
    \labeltable{
        \begin{tabular}{|c|c|}
            \hline
            x     & y                \\
            \hline
            -1    & 1                \\
            \hline
            -0.75 & $1\frac{1}{12}$  \\
            \hline
            -0.5  & $1\frac{1}{6}$   \\
            \hline
            -0.25 & $1\frac{1}{4}$   \\
            \hline
            0     & $1\frac{1}{3}$   \\
            \hline
            0.25  & $1\frac{5}{12}$  \\
            \hline
            0.5   & $1\frac{1}{2}$   \\
            \hline
            0.75  & $1\frac{7}{12}$  \\
            \hline
            1     & $1\frac{2}{3}$   \\
            \hline
            1.25  & $1\frac{3}{4}$   \\
            \hline
            1.5   & $1\frac{5}{6}$   \\
            \hline
            1.75  & $1\frac{11}{12}$ \\
            \hline
            2     & $2$              \\
            \hline
            2.25  & $2\frac{1}{12}$  \\
            \hline
            2.5   & $2\frac{1}{6}$   \\
            \hline
            2.75  & $2\frac{1}{4}$   \\
            \hline
            3     & $2\frac{1}{3}$   \\
            \hline
        \end{tabular}
    }{Таблица значений функции $y$}{tab:task11-values-table}

    \subsection{График}
    \makenewfig{
        \begin{tikzpicture}

            \begin{axis}
                [
                grid=both,
                axis lines = middle,
                xlabel = {$x$},
                ylabel = {$y$},
                minor tick num = 1,
                width = 10cm,
                height = 10cm]
                \addplot coordinates {
                    (-1.00000, 1.00000)
                    (-0.75000, 1.08333)
                    (-0.50000, 1.16667)
                    (-0.25000, 1.25000)
                    (0.00000, 1.33333)
                    (0.25000, 1.41667)
                    (0.50000, 1.50000)
                    (0.75000, 1.58333)
                    (1.00000, 1.66667)
                    (1.25000, 1.75000)
                    (1.50000, 1.83333)
                    (1.75000, 1.91667)
                    (2.00000, 2.00000)
                    (2.25000, 2.08333)
                    (2.50000, 2.16667)
                    (2.75000, 2.25000)
                    (3.00000, 2.33333)
                };

                \addplot[mark=*] coordinates {(-4,0)} node[pin=45:{$A$}]{} ;
                \addplot[mark=*] coordinates {(1,-3)} node[pin=100:{$B$}]{} ;
                \addplot[mark=*] coordinates {(4,-2)} node[pin=150:{$C$}]{} ;

                \addplot[blue] {(1/3) * x + (4 / 3)} node[pin=230:{$l$}]{};
                \addplot[red] {(1/3) * x - (10 / 3)};
            \end{axis}

        \end{tikzpicture}
    }{Прямая $l$, проходящая через $A$ параллельная $BC$}{ris:task11}


    \section{Кривые второго порядка на плоскости}

    \subsection{Задание варианта}
    Постройте гиперболы (диапазон и шаг выберите самостоятельно).\\
    $1) 16x^2 - 9y^2 = 144\\2) \frac{y^2}{81} - \frac{x^2}{64} = 1$\\

    \subsection{Обоснование}
    \begin{enumerate}
        \item $16x^2 - 9y^2 = 144$\\
        $y^2 = \frac{16x^2 - 144}{9}$\\
        $y = \frac{\pm \sqrt{16x^2 - 144}}{3}$\\
        $D(y): 16x^2 - 144 \geq 0$\\
        $(4x - 12)(4x + 12) \geq 0$ - гипербола с ветвями направленными вверх, значит\\
        $D(y) \in (-\infty; -3] \cup [3; \infty)$.\\

        \item $\frac{y^2}{81} - \frac{x^2}{64} = 1$\\
        $y^2 = 81(\frac{x^2 + 64}{64})$\\
        $y = \frac{9}{8}\sqrt{64 + x^2}$\\
        $D(y) \in (-\infty; \infty)$, так как $64 + x^2 > 0$ при любом $x$.\\
    \end{enumerate}

    \subsection{Таблица значений}

    \begin{enumerate}
        \item Область перебора: $[-5; -3] \cup [3; 5]; \triangle = 0.5$.\bigbreak
        \labeltable{
            \begin{tabular}{|c|c|}
                \hline
                x    & y                         \\
                \hline
                -5   & $\pm5\frac{1}{3}$         \\
                \hline
                -4.5 & $\pm2\sqrt{5}$            \\
                \hline
                -4   & $\pm\frac{4\sqrt{7}}{3}$  \\
                \hline
                -3.5 & $\pm\frac{2\sqrt{13}}{3}$ \\
                \hline
                -3   & $0$                       \\
                \hline
                3    & $0$                       \\
                \hline
                3.5  & $\pm\frac{2\sqrt{13}}{3}$ \\
                \hline
                4    & $\pm\frac{4\sqrt{7}}{3}$  \\
                \hline
                4.5  & $\pm2\sqrt{5}$            \\
                \hline
                5    & $\pm5\frac{1}{3}$         \\
                \hline
            \end{tabular}
        }{Таблица значений функции $y$}{tab:task12-values1-table}
        \item Область перебора: $[-5; 5]; \triangle = 0.5$.\bigbreak
        \labeltable{
            \begin{tabular}{|c|c|}
                \hline
                x    & y                           \\
                \hline
                -5   & $\pm\frac{9\sqrt{89}}{8}$   \\
                \hline
                -4.5 & $\pm\frac{9\sqrt{337}}{16}$ \\
                \hline
                -4   & $\pm\frac{9\sqrt{5}}{2}$    \\
                \hline
                -3.5 & $\pm\frac{9\sqrt{305}}{16}$ \\
                \hline
                -3   & $\pm\frac{9\sqrt{73}}{8}$   \\
                \hline
                -2.5 & $\pm\frac{9\sqrt{281}}{16}$ \\
                \hline
                -2   & $\pm\frac{9\sqrt{17}}{4}$   \\
                \hline
                -1.5 & $\pm\frac{9\sqrt{265}}{16}$ \\
                \hline
                -1   & $\pm\frac{9\sqrt{65}}{8}$   \\
                \hline
                -0.5 & $\pm\frac{9\sqrt{257}}{16}$ \\
                \hline
                0    & $\pm9$                      \\
                \hline
                0.5  & $\pm\frac{9\sqrt{257}}{16}$ \\
                \hline
                1    & $\pm\frac{9\sqrt{65}}{8}$   \\
                \hline
                1.5  & $\pm\frac{9\sqrt{265}}{16}$ \\
                \hline
                2    & $\pm\frac{9\sqrt{17}}{4}$   \\
                \hline
                2.5  & $\pm\frac{9\sqrt{281}}{16}$ \\
                \hline
                3    & $\pm\frac{9\sqrt{73}}{8}$   \\
                \hline
                3.5  & $\pm\frac{9\sqrt{305}}{16}$ \\
                \hline
                4    & $\pm\frac{9\sqrt{5}}{2}$    \\
                \hline
                4.5  & $\pm\frac{9\sqrt{337}}{16}$ \\
                \hline
                5    & $\pm\frac{9\sqrt{89}}{8}$   \\
                \hline
            \end{tabular}
        }{Таблица значений функции $y$}{tab:task12-values2-table}
    \end{enumerate}

    \subsection{График}
    \begin{enumerate}
        \item \makenewfig{
            \begin{tikzpicture}

                \begin{axis}
                    [
                    grid=both,
                    axis lines = middle,
                    xlabel = {$x$},
                    ylabel = {$y$},
                    minor tick num = 1,
                    width = 10cm,
                    height = 10cm]
                    \addplot [blue] coordinates {
                        (-5.00000, 5.33333)
                        (-4.50000, 4.47214)
                        (-4.00000, 3.52767)
                        (-3.50000, 2.40370)
                        (-3.00000, 0.00000)
                    };
                    \addplot [blue] coordinates {
                        (3.0000, 0.00000)
                        (3.50000, 2.40370)
                        (4.00000, 3.52767)
                        (4.50000, 4.47214)
                        (5.00000, 5.33333)
                    };
                    \addplot [blue] coordinates {
                        (-5.00000, -5.33333)
                        (-4.50000, -4.47214)
                        (-4.00000, -3.52767)
                        (-3.50000, -2.40370)
                        (-3.00000, -0.00000)
                    };
                    \addplot [blue] coordinates {
                        (3.0000, 0.00000)
                        (3.50000, -2.40370)
                        (4.00000, -3.52767)
                        (4.50000, -4.47214)
                        (5.00000, -5.33333)
                    };
                \end{axis}

            \end{tikzpicture}
        }{График гиперболы $16x^2 - 9y^2 = 144$}{ris:task12_1}
        \item \makenewfig{
            \begin{tikzpicture}

                \begin{axis}
                    [
                    grid=both,
                    axis lines = middle,
                    xlabel = {$x$},
                    ylabel = {$y$},
                    minor tick num = 1,
                    width = 10cm,
                    height = 10cm]
                    \addplot [blue] coordinates {
                        (-5.00000, 10.61323)
                        (-4.50000, 10.32613)
                        (-4.00000, 10.06231)
                        (-3.50000, 9.82364)
                        (-3.00000, 9.61200)
                        (-2.50000, 9.42922)
                        (-2.00000, 9.27699)
                        (-1.50000, 9.15684)
                        (-1.00000, 9.07004)
                        (-0.50000, 9.01756)
                        (0.00000, 9.00000)
                        (0.50000, 9.01756)
                        (1.00000, 9.07004)
                        (1.50000, 9.15684)
                        (2.00000, 9.27699)
                        (2.50000, 9.42922)
                        (3.00000, 9.61200)
                        (3.50000, 9.82364)
                        (4.00000, 10.06231)
                        (4.50000, 10.32613)
                        (5.00000, 10.61323)
                    };
                    \addplot [blue] coordinates {
                        (-5.00000, -10.61323)
                        (-4.50000, -10.32613)
                        (-4.00000, -10.06231)
                        (-3.50000, -9.82364)
                        (-3.00000, -9.61200)
                        (-2.50000, -9.42922)
                        (-2.00000, -9.27699)
                        (-1.50000, -9.15684)
                        (-1.00000, -9.07004)
                        (-0.50000, -9.01756)
                        (0.00000, -9.00000)
                        (0.50000, -9.01756)
                        (1.00000, -9.07004)
                        (1.50000, -9.15684)
                        (2.00000, -9.27699)
                        (2.50000, -9.42922)
                        (3.00000, -9.61200)
                        (3.50000, -9.82364)
                        (4.00000, -10.06231)
                        (4.50000, -10.32613)
                        (5.00000, -10.61323)
                    };
                \end{axis}

            \end{tikzpicture}
        }{График гиперболы $\frac{y^2}{81} - \frac{x^2}{64} = 1$}{ris:task12_2}
    \end{enumerate}


    \section{Графическое решение систем уравнений}

    \subsection{Задание варианта}
    $\begin{cases}
         y^2 = x\\
         y^2 = -x^2 + 9
    \end{cases}$
    в диапазоне $0 \leq x \leq 4$ с шагом $\triangle = 0.2$

    \subsection{Обоснование}
    $y_1^2 = x$\\
    $y_1 = \pm\sqrt{x}$\bigbreak
    $y_2^2=9-x^2$\\
    $y_2=\pm\sqrt{9-x^2}$

    \subsection{Таблица значений}
    \labeltable{
        \begin{tabular}{|c|c|c|}
            \hline
            x   & $y_1$            & $y_2$            \\
            \hline
            0   & 0                & $\pm3$           \\
            \hline
            0.2 & $\pm0.447213595$ & $\pm2.993325909$ \\
            \hline
            0.4 & $\pm0.632455532$ & $\pm2.973213749$ \\
            \hline
            0.6 & $\pm0.774596669$ & $\pm2.939387691$ \\
            \hline
            0.8 & $\pm0.894427191$ & $\pm2.891366459$ \\
            \hline
            1   & $\pm1$           & $\pm2.828427125$ \\
            \hline
            1.2 & $\pm1.095445115$ & $\pm2.749545417$ \\
            \hline
            1.4 & $\pm1.183215957$ & $\pm2.653299832$ \\
            \hline
            1.6 & $\pm1.264911064$ & $\pm2.537715508$ \\
            \hline
            1.8 & $\pm1.341640786$ & $\pm2.4$         \\
            \hline
            2   & $\pm1.414213562$ & $\pm2.236067977$ \\
            \hline
            2.2 & $\pm1.483239697$ & $\pm2.039607805$ \\
            \hline
            2.4 & $\pm1.549193338$ & $\pm1.8$         \\
            \hline
            2.6 & $\pm1.61245155$  & $\pm1.496662955$ \\
            \hline
            2.8 & $\pm1.673320053$ & $\pm1.077032961$ \\
            \hline
            3   & $\pm1.732050808$ & 0                \\
            \hline
            3.2 & $\pm1.788854382$ & Нет решения      \\
            \hline
            3.4 & $\pm1.843908891$ & Нет решения      \\
            \hline
            3.6 & $\pm1.897366596$ & Нет решения      \\
            \hline
            3.8 & $\pm1.949358869$ & Нет решения      \\
            \hline
            4   & $\pm2$           & Нет решения      \\
            \hline
        \end{tabular}
    }{Таблица значений функций $y_1$ и $y_2$}{tab:task13-values-table}
    \subsection{График}

    \makenewfig{
        \begin{tikzpicture}

            \begin{axis}
                [
                grid=both,
                axis lines = middle,
                title = Система уравнений,
                xlabel = {$x$},
                ylabel = {$y$},
                minor tick num = 1,
                width = 10cm,
                height = 10cm]
                \addplot [blue] coordinates {
                    (0.00000, 0.00000)
                    (0.20000, 0.44721)
                    (0.40000, 0.63246)
                    (0.60000, 0.77460)
                    (0.80000, 0.89443)
                    (1.00000, 1.00000)
                    (1.20000, 1.09545)
                    (1.40000, 1.18322)
                    (1.60000, 1.26491)
                    (1.80000, 1.34164)
                    (2.00000, 1.41421)
                    (2.20000, 1.48324)
                    (2.40000, 1.54919)
                    (2.60000, 1.61245)
                    (2.80000, 1.67332)
                    (3.00000, 1.73205)
                    (3.20000, 1.78885)
                    (3.40000, 1.84391)
                    (3.60000, 1.89737)
                    (3.80000, 1.94936)
                    (4.00000, 2.00000)
                } ;
                \addplot [blue] coordinates {
                    (0.00000, 0.00000)
                    (0.20000, -0.44721)
                    (0.40000, -0.63246)
                    (0.60000, -0.77460)
                    (0.80000, -0.89443)
                    (1.00000, -1.00000)
                    (1.20000, -1.09545)
                    (1.40000, -1.18322)
                    (1.60000, -1.26491)
                    (1.80000, -1.34164)
                    (2.00000, -1.41421)
                    (2.20000, -1.48324)
                    (2.40000, -1.54919)
                    (2.60000, -1.61245)
                    (2.80000, -1.67332)
                    (3.00000, -1.73205)
                    (3.20000, -1.78885)
                    (3.40000, -1.84391)
                    (3.60000, -1.89737)
                    (3.80000, -1.94936)
                    (4.00000, -2.00000)
                }node[pin=130:{$y_1$}]{};

                \addplot [red] coordinates {
                    (0.00000, 3.00000)
                    (0.20000, 2.99333)
                    (0.40000, 2.97321)
                    (0.60000, 2.93939)
                    (0.80000, 2.89137)
                    (1.00000, 2.82843)
                    (1.20000, 2.74955)
                    (1.40000, 2.65330)
                    (1.60000, 2.53772)
                    (1.80000, 2.40000)
                    (2.00000, 2.23607)
                    (2.20000, 2.03961)
                    (2.40000, 1.80000)
                    (2.60000, 1.49666)
                    (2.80000, 1.07703)
                    (3.00000, 0.00000)
                } node[pin=45:{$y_2$}]{};
                \addplot [red] coordinates {
                    (0.00000, -3.00000)
                    (0.20000, -2.99333)
                    (0.40000, -2.97321)
                    (0.60000, -2.93939)
                    (0.80000, -2.89137)
                    (1.00000, -2.82843)
                    (1.20000, -2.74955)
                    (1.40000, -2.65330)
                    (1.60000, -2.53772)
                    (1.80000, -2.40000)
                    (2.00000, -2.23607)
                    (2.20000, -2.03961)
                    (2.40000, -1.80000)
                    (2.60000, -1.49666)
                    (2.80000, -1.07703)
                    (3.00000, 0.00000)
                };

                \addplot[mark=*] coordinates {(2.541381265, 1.594171027)} node[pin=45:{$x_1$}]{} ;
                \addplot[mark=*] coordinates {(2.541381265, -1.594171027)} node[pin=280:{$x_2$}]{} ;
            \end{axis}
        \end{tikzpicture}
    }{Графики функций $y_1$, $y_2$}{ris:task13}
    \bigbreak
    Решение системы уравнений: $(2.541381265, 1.594171027)$,\\$(2.541381265, -1.594171027)$.


    \section{Плоскость в трёхмерном пространстве}

    \subsection{Задание варианта}
    Построить плоскость, параллельную плоскости $Oxy$ и пересекающую ось $Oz$ в точке $M(0, 0, 20)$; при $0 \leq x $ ≤ $ 10$
    с шагом $\triangle = 1$; $-5 \leq y $ ≤ $ 5$ с шагом $\triangle = 1$.

    \subsection{Обоснование}
    Плоскость, параллельная $Oxy$ имеет вид $Cz + D = 0$.
    Подставим в уравнение координаты точки M.\\
    $C \cdot 20 + D = 0$\\
    $D = -C \cdot 20$\\
    $Cz - C \cdot 20 = 0  | :C$\\
    $z = 20$ - искомое уравнение.

    \subsection{Таблица значений}
    \labeltable{
        \begin{tabular}{|c|c|c|c|c|c|c|c|c|c|c|c|}
            \hline
            x / y & 0  & 1  & 2  & 3  & 4  & 5  & 6  & 7  & 8  & 9  & 10 \\
            \hline
            -5    & 20 & 20 & 20 & 20 & 20 & 20 & 20 & 20 & 20 & 20 & 20 \\
            \hline
            -4    & 20 & 20 & 20 & 20 & 20 & 20 & 20 & 20 & 20 & 20 & 20 \\
            \hline
            -3    & 20 & 20 & 20 & 20 & 20 & 20 & 20 & 20 & 20 & 20 & 20 \\
            \hline
            -2    & 20 & 20 & 20 & 20 & 20 & 20 & 20 & 20 & 20 & 20 & 20 \\
            \hline
            -1    & 20 & 20 & 20 & 20 & 20 & 20 & 20 & 20 & 20 & 20 & 20 \\
            \hline
            0     & 20 & 20 & 20 & 20 & 20 & 20 & 20 & 20 & 20 & 20 & 20 \\
            \hline
            1     & 20 & 20 & 20 & 20 & 20 & 20 & 20 & 20 & 20 & 20 & 20 \\
            \hline
            2     & 20 & 20 & 20 & 20 & 20 & 20 & 20 & 20 & 20 & 20 & 20 \\
            \hline
            3     & 20 & 20 & 20 & 20 & 20 & 20 & 20 & 20 & 20 & 20 & 20 \\
            \hline
            4     & 20 & 20 & 20 & 20 & 20 & 20 & 20 & 20 & 20 & 20 & 20 \\
            \hline
            5     & 20 & 20 & 20 & 20 & 20 & 20 & 20 & 20 & 20 & 20 & 20 \\
            \hline
        \end{tabular}
    }{Таблица значений функции плоскости}{tab:task14-values-table}

    \subsection{График}

    \makenewfig{
        \begin{tikzpicture}

            \begin{axis}
                [
                xlabel = {$x$},
                ylabel = {$y$},
                zlabel={$z$}]
                \addplot3[
                samples = 20,
                samples y = 8,
                only marks,
                blue
                ] coordinates {
                    (-5, 0, 20) (-5, 1, 20) (-5, 2, 20) (-5, 3, 20) (-5, 4, 20) (-5, 5, 20) (-5, 6, 20) (-5, 7, 20) (-5, 8, 20) (-5, 9, 20) (-5, 10, 20)
                    (-4, 0, 20) (-4, 1, 20) (-4, 2, 20) (-4, 3, 20) (-4, 4, 20) (-4, 5, 20) (-4, 6, 20) (-4, 7, 20) (-4, 8, 20) (-4, 9, 20) (-4, 10, 20)
                    (-3, 0, 20) (-3, 1, 20) (-3, 2, 20) (-3, 3, 20) (-3, 4, 20) (-3, 5, 20) (-3, 6, 20) (-3, 7, 20) (-3, 8, 20) (-3, 9, 20) (-3, 10, 20)
                    (-2, 0, 20) (-2, 1, 20) (-2, 2, 20) (-2, 3, 20) (-2, 4, 20) (-2, 5, 20) (-2, 6, 20) (-2, 7, 20) (-2, 8, 20) (-2, 9, 20) (-2, 10, 20)
                    (-1, 0, 20) (-1, 1, 20) (-1, 2, 20) (-1, 3, 20) (-1, 4, 20) (-1, 5, 20) (-1, 6, 20) (-1, 7, 20) (-1, 8, 20) (-1, 9, 20) (-1, 10, 20)
                    (0, 0, 20) (0, 1, 20) (0, 2, 20) (0, 3, 20) (0, 4, 20) (0, 5, 20) (0, 6, 20) (0, 7, 20) (0, 8, 20) (0, 9, 20) (0, 10, 20)
                    (1, 0, 20) (1, 1, 20) (1, 2, 20) (1, 3, 20) (1, 4, 20) (1, 5, 20) (1, 6, 20) (1, 7, 20) (1, 8, 20) (1, 9, 20) (1, 10, 20)
                    (2, 0, 20) (2, 1, 20) (2, 2, 20) (2, 3, 20) (2, 4, 20) (2, 5, 20) (2, 6, 20) (2, 7, 20) (2, 8, 20) (2, 9, 20) (2, 10, 20)
                    (3, 0, 20) (3, 1, 20) (3, 2, 20) (3, 3, 20) (3, 4, 20) (3, 5, 20) (3, 6, 20) (3, 7, 20) (3, 8, 20) (3, 9, 20) (3, 10, 20)
                    (4, 0, 20) (4, 1, 20) (4, 2, 20) (4, 3, 20) (4, 4, 20) (4, 5, 20) (4, 6, 20) (4, 7, 20) (4, 8, 20) (4, 9, 20) (4, 10, 20)
                    (5, 0, 20) (5, 1, 20) (5, 2, 20) (5, 3, 20) (5, 4, 20) (5, 5, 20) (5, 6, 20) (5, 7, 20) (5, 8, 20) (5, 9, 20) (5, 10, 20)
                };
            \end{axis}
        \end{tikzpicture}
    }{График плоскости, параллельной $Oxy$ проходящей через $M(0, 0, 20)$}{ris:task14}


    \section{Поверхность второго порядка в трёхмерном пространстве}

    \subsection{Задание варианта}
    Построить верхнюю часть гиперболоида, заданного уравнением $\frac{x^2}{25} + \frac{y^2}{16} - \frac{z^2}{9} = 1$,
    лежащую в диапазоне $-5 \leq x $ ≤ $ 5$ с шагом $\triangle = 1$ и $-3 \leq y $ ≤ $ 3$ с шагом $\triangle = 0.5$.

    \subsection{Обоснование}
    $\frac{x^2}{25} + \frac{y^2}{16} - \frac{z^2}{9} = 1$\\
    $\frac{z^2}{9} = \frac{x^2}{25} + \frac{y^2}{16} - 1$\\
    $z^2 = \frac{9x^2}{25} + \frac{9y^2}{16} - 9$\\
    $z = \sqrt{\frac{9x^2}{25} + \frac{9y^2}{16} - 9}$, так как по условию нужно получить только верхнюю часть гиперболоида,
    будем брать результат с плюсом.

    \subsection{Таблица значений}
    \labeltable{
        \begin{tabular}{|c|c|c|c|c|c|c|c|c|c|c|c|}
            \hline
            x / y & -5         & -4  & -3          & -2          & -1          & 0           & 1           & 2           & 3           & 4   & 5          \\
            \hline
            -3    & 2.88140591 & 1.8 & $\emptyset$ & $\emptyset$ & $\emptyset$ & $\emptyset$ & $\emptyset$ & $\emptyset$ & $\emptyset$ & 1.8 & 2.88140591 \\
            \hline
            -2.5  & 2.70416346 & 1.5 & $\emptyset$ & $\emptyset$ & $\emptyset$ & $\emptyset$ & $\emptyset$ & $\emptyset$ & $\emptyset$ & 1.5 & 2.70416346 \\
            \hline
            -2    & 2.55       & 1.2 & $\emptyset$ & $\emptyset$ & $\emptyset$ & $\emptyset$ & $\emptyset$ & $\emptyset$ & $\emptyset$ & 1.2 & 2.55 \\
            \hline
            -1.5  & 2.42332416 & 0.9 & $\emptyset$ & $\emptyset$ & $\emptyset$ & $\emptyset$ & $\emptyset$ & $\emptyset$ & $\emptyset$ & 0.9 & 2.42332416 \\
            \hline
            -1    & 2.3286262  & 0.6 & $\emptyset$ & $\emptyset$ & $\emptyset$ & $\emptyset$ & $\emptyset$ & $\emptyset$ & $\emptyset$ & 0.6 & 2.3286262 \\
            \hline
            -0.5  & 2.26991189 & 0.3 & $\emptyset$ & $\emptyset$ & $\emptyset$ & $\emptyset$ & $\emptyset$ & $\emptyset$ & $\emptyset$ & 0.3 & 2.26991189 \\
            \hline
            0     & 2.25       & 0   & $\emptyset$ & $\emptyset$ & $\emptyset$ & $\emptyset$ & $\emptyset$ & $\emptyset$ & $\emptyset$ & 0 & 2.25 \\
            \hline
            0.5   & 2.26991189 & 0.3 & $\emptyset$ & $\emptyset$ & $\emptyset$ & $\emptyset$ & $\emptyset$ & $\emptyset$ & $\emptyset$ & 0.3 & 2.26991189 \\
            \hline
            1     & 2.3286262  & 0.6 & $\emptyset$ & $\emptyset$ & $\emptyset$ & $\emptyset$ & $\emptyset$ & $\emptyset$ & $\emptyset$ & 0.6 & 2.3286262 \\
            \hline
            1.5   & 2.42332416 & 0.9 & $\emptyset$ & $\emptyset$ & $\emptyset$ & $\emptyset$ & $\emptyset$ & $\emptyset$ & $\emptyset$ & 0.9 & 2.42332416 \\
            \hline
            2     & 2.55       & 1.2 & $\emptyset$ & $\emptyset$ & $\emptyset$ & $\emptyset$ & $\emptyset$ & $\emptyset$ & $\emptyset$ & 1.2 & 2.55 \\
            \hline
            2.5   & 2.70416346 & 1.5 & $\emptyset$ & $\emptyset$ & $\emptyset$ & $\emptyset$ & $\emptyset$ & $\emptyset$ & $\emptyset$ & 1.5 & 2.70416346 \\
            \hline
            3     & 2.88140591 & 1.8 & $\emptyset$ & $\emptyset$ & $\emptyset$ & $\emptyset$ & $\emptyset$ & $\emptyset$ & $\emptyset$ & 1.8 & 2.88140591 \\
            \hline
        \end{tabular}
    }{Таблица значений гиперболоида}{tab:task15-values-table}

    \subsection{График}
    \makenewfig{
        \begin{tikzpicture}

            \begin{axis}
                [
                xlabel = {$x$},
                ylabel = {$y$},
                zlabel={$z$}]
                \addplot3[
                samples = 20,
                samples y = 8,
                only marks,
                blue
                ] coordinates {
                    (-5, -3, 2.881405907) (-4, -3, 1.8) (4, -3, 1.8) (5, -3, 2.881405907)
                    (-5, -2.5, 2.704163457) (-4, -2.5, 1.5)  (4, -2.5, 1.5) (5, -2.5, 2.704163457)
                    (-5, -2, 2.55) (-4, -2, 1.2)  (4, -2, 1.2) (5, -2, 2.55)
                    (-5, -1.5, 2.423324163) (-4, -1.5, 0.9) (4, -1.5, 0.9) (5, -1.5, 2.423324163)
                    (-5, -1, 2.328626204) (-4, -1, 0.6)  (4, -1, 0.6) (5, -1, 2.328626204)
                    (-5, -0.5, 2.269911893) (-4, -0.5, 0.3)  (4, -0.5, 0.3) (5, -0.5, 2.269911893)
                    (-5, 0, 2.25) (-4, 0, 0)  (4, 0, 0) (5, 0, 2.25)
                    (-5, 0.5, 2.269911893) (-4, 0.5, 0.3) (4, 0.5, 0.3) (5, 0.5, 2.269911893)
                    (-5, 1, 2.328626204) (-4, 1, 0.6)  (4, 1, 0.6) (5, 1, 2.328626204)
                    (-5, 1.5, 2.423324163) (-4, 1.5, 0.9)  (4, 1.5, 0.9) (5, 1.5, 2.423324163)
                    (-5, 2, 2.55) (-4, 2, 1.2)  (4, 2, 1.2) (5, 2, 2.55)
                    (-5, 2.5, 2.704163457) (-4, 2.5, 1.5) (4, 2.5, 1.5) (5, 2.5, 2.704163457)
                    (-5, 3, 2.881405907) (-4, 3, 1.8) (4, 3, 1.8) (5, 3, 2.881405907)
                };
            \end{axis}
        \end{tikzpicture}
    }{График поверхности $\frac{x^2}{25} + \frac{y^2}{16} - \frac{z^2}{9} = 1$}{ris:task15}

% Литра http://www.pm298.ru/reshenie/dsere.php

\end{document}