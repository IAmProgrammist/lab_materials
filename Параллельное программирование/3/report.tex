\documentclass[a4paper,14pt]{extarticle}


\usepackage[utf8]{inputenc}
\usepackage[T2A]{fontenc}
\usepackage[english,russian]{babel}
\usepackage{ragged2e}
\usepackage[utf8]{inputenc}
\usepackage{hyperref}
\usepackage{minted}
\setmintedinline{frame=lines, framesep=2mm, baselinestretch=1.5, bgcolor=LightGray, breaklines,fontsize=\scriptsize}
\setminted{frame=lines, framesep=2mm, baselinestretch=1.5, bgcolor=LightGray, breaklines,fontsize=\scriptsize}
\usepackage{xcolor}
\definecolor{LightGray}{gray}{0.9}
\usepackage{graphicx}
\usepackage[export]{adjustbox}
\usepackage[left=1cm,right=1cm, top=1cm,bottom=1cm,bindingoffset=0cm]{geometry}
\usepackage{fontspec}
\usepackage{ upgreek }
\usepackage[shortlabels]{enumitem}
\usepackage{adjustbox}
\usepackage{multirow}
\usepackage{amsmath}
\usepackage{amssymb}
\usepackage{pifont}
\usepackage{pgfplots}
\usepackage{longtable}
\usepackage{array}
\graphicspath{ {./images/} }
\makeatletter
\AddEnumerateCounter{\asbuk}{\russian@alph}{щ}
\makeatother
\setmonofont{Ubuntu Mono}
\setmainfont{Times New Roman}

\newcommand\textbox[1]{
	\parbox{.45\textwidth}{#1}
} 

\newcommand{\specialcell}[2][c]{%
	\begin{tabular}[#1]{@{}c@{}}#2\end{tabular}}

\begin{document}
\pagenumbering{gobble}
\begin{center}
    \small{
        \textbf{МИНИCТЕРCТВО НАУКИ И ВЫCШЕГО ОБРАЗОВАНИЯ РОCCИЙCКОЙ ФЕДЕРАЦИИ}\\
        ФЕДЕРАЛЬНОЕ ГОCУДАРCТВЕННОЕ БЮДЖЕТНОЕ ОБРАЗОВАТЕЛЬНОЕ УЧРЕЖДЕНИЕ\\ВЫCШЕГО ОБРАЗОВАНИЯ \\
        \textbf{«БЕЛГОРОДCКИЙ ГОCУДАРCТВЕННЫЙ ТЕХНОЛОГИЧЕCКИЙ\\УНИВЕРCИТЕТ им. В. Г. ШУХОВА»\\ (БГТУ им. В.Г. Шухова)} \\
        \bigbreak
        \includegraphics[width=70mm]{log}\\
        ИНСТИТУТ ИНФОРМАЦИОННЫХ ТЕХНОЛОГИЙ И УПРАВЛЯЮЩИХ СИСТЕМ\\}
\end{center}

\vfill
\begin{center}
    \large{
        \textbf{
            Лабораторная работа №3}}\\
    \normalsize{
        по дисциплине: Параллельное программирование \\
        тема: «Гибридное параллельное программирование с использованием OpenMP + MPI»}
\end{center}
\vfill
\hfill\textbox{
    Выполнил: ст. группы ПВ-223\\Пахомов Владислав Андреевич
    \bigbreak
    Проверили: \\доц. Островский Алексей Мичеславович
}
\vfill\begin{center}
    Белгород 2025 г.
\end{center}
\newpage
\underline{\textbf{Цель работы: }}Изучить возможности гибридного подхода к параллельному
программированию с использованием стандартов MPI и OpenMP, реализовать смешанную
модель параллельных вычислений, оценить её эффективность по сравнению с отдельными
технологиями, изучить механизмы межпроцессного взаимодействия (MPI) и многопоточного
параллелизма (OpenMP).\\
\underline{\textbf{Условие индивидуального задания: }}\\
Гибридная генерация и проверка простых чисел (решето Эратосфена). Реализовать
гибридную версию алгоритма решета Эратосфена для поиска всех простых чисел до N.
Разделить диапазон между процессами MPI. Внутри каждого процесса параллелизовать
вычёркивание кратных с помощью OpenMP. После завершения — объединить частичные
результаты. Провести измерение времени при разных N и конфигурациях потоков/процессов.
\begin{center}
\textbf{Ход выполнения работы}
\end{center}

При решении задачи недостаточно использовать классическую схему решения решета Эратосфена. 
Для решения её в парадигме параллельного программирования необходимо использовать другой подход - 
сегментированное решето Эратосфена. Решение основывается на том, что для вычисления отрезка
решета Эратосфена до N достаточно вычислить решето Эратосфена классическим способом только до 
$\sqrt{N}$. В однопоточном программировании такой подход даст улучшение относительно памяти, 
так как нам больше не нужно хранить весь массив. 

Для параллельной задачи мы можем синхронно подготовить решето до $\sqrt{N}$ после чего разбить 
оставшийся массив и разделить между процессами MPI. При помощи OMP будем проставлять флаги того, 
что число является непростым.

\textbf{Исходный код:}
\begin{minted}{c}
#include <stdio.h>
#include <stdlib.h>
#include <stdbool.h>
#include <mpi.h>
#include <omp.h>
#include <math.h>

#define N 80000000

void classic_sieve(int *sieve, int sieve_size)
{
    for (int i = 2; i < sieve_size; i++)
    {
        if (sieve[i])
            continue;
        for (int j = 2 * i; j < N; j += i)
            sieve[j] = true;
    }
}

int main(int argc, char *argv[])
{
    // Инициализировать MPI, получить ранг
    // и количество процессов MPI
    int rank, size;

    // Просчёт локального решета Эратосфена
    int *local_sieve = (int *)calloc(N, sizeof(int));
    // Просчитываем первые sqrt(N) чисел
    classic_sieve(local_sieve, sqrt(N));

    MPI_Init(&argc, &argv);
    MPI_Comm_rank(MPI_COMM_WORLD, &rank);
    MPI_Comm_size(MPI_COMM_WORLD, &size);

    // Определяем границы для которых будет вычисляться локальное решето
    int left = sqrt(N) + ((N - sqrt(N)) / size) * rank;
    int right = rank == size - 1 ? N : (sqrt(N) + ((N - sqrt(N)) / size) * (rank + 1));

    // Проходим по всем простым числам до sqrt(N)
    for (int i = 2; i < sqrt(N); i++)
    {
        if (local_sieve[i])
            continue;

        // Находим первое число кратное i в рамках определённых границ
        int j = left / i * i;
        if (j < left)
        {
            j += i;
        }

// Устанавливаем кратные числа
#pragma omp parallel for schedule(guided)
        for (int jj = j; jj < right; jj += i)
        {
            local_sieve[jj] = true;
        }
    }

    int *global_sieve = (int *)calloc(N, sizeof(int));

    MPI_Reduce(
        local_sieve,
        global_sieve,
        N,
        MPI_INT,
        MPI_LOR,
        0,
        MPI_COMM_WORLD);

    // Очистить ресурсы MPI
    MPI_Finalize();
    return 0;
}
\end{minted}

\textbf{Результаты вычислений по времени:}\\
\includegraphics[width=140mm]{exp}


\textbf{Вывод: } в ходе лабораторной работы изучили возможности гибридного подхода к параллельному
программированию с использованием стандартов MPI и OpenMP, реализовать смешанную
модель параллельных вычислений, оценить её эффективность по сравнению с отдельными
технологиями, изучить механизмы межпроцессного взаимодействия (MPI) и многопоточного
параллелизма (OpenMP).

С увеличением количества потоков уменьшалось и время работы, для генерации решета Эратосфена
была использована стратегия static, так как присвоение - равномерная нагрузка. 

Однако с увеличением процессов время выполнения работы программы увеличивалось, возможно это связано
с тем что N недостаточно большой, и время по организации процессов перевешивает время вычислений. 


\end{document}