\documentclass[a4paper,14pt]{extarticle}


\usepackage[utf8]{inputenc}
\usepackage[T2A]{fontenc}
\usepackage[english,russian]{babel}
\usepackage{ragged2e}
\usepackage[utf8]{inputenc}
\usepackage{hyperref}
\usepackage{minted}
\setmintedinline{frame=lines, framesep=2mm, baselinestretch=1.5, bgcolor=LightGray, breaklines,fontsize=\scriptsize}
\setminted{frame=lines, framesep=2mm, baselinestretch=1.5, bgcolor=LightGray, breaklines,fontsize=\scriptsize}
\usepackage{xcolor}
\definecolor{LightGray}{gray}{0.9}
\usepackage{graphicx}
\usepackage[export]{adjustbox}
\usepackage[left=1cm,right=1cm, top=1cm,bottom=1cm,bindingoffset=0cm]{geometry}
\usepackage{fontspec}
\usepackage{ upgreek }
\usepackage[shortlabels]{enumitem}
\usepackage{adjustbox}
\usepackage{multirow}
\usepackage{amsmath}
\usepackage{amssymb}
\usepackage{pifont}
\usepackage{pgfplots}
\usepackage{longtable}
\usepackage{array}
\graphicspath{ {./images/} }
\makeatletter
\AddEnumerateCounter{\asbuk}{\russian@alph}{щ}
\makeatother
\setmonofont{Ubuntu Mono}
\setmainfont{Times New Roman}

\newcommand\textbox[1]{
	\parbox{.45\textwidth}{#1}
} 

\newcommand{\specialcell}[2][c]{%
	\begin{tabular}[#1]{@{}c@{}}#2\end{tabular}}

\begin{document}
\pagenumbering{gobble}
\begin{center}
    \small{
        \textbf{МИНИCТЕРCТВО НАУКИ И ВЫCШЕГО ОБРАЗОВАНИЯ РОCCИЙCКОЙ ФЕДЕРАЦИИ}\\
        ФЕДЕРАЛЬНОЕ ГОCУДАРCТВЕННОЕ БЮДЖЕТНОЕ ОБРАЗОВАТЕЛЬНОЕ УЧРЕЖДЕНИЕ\\ВЫCШЕГО ОБРАЗОВАНИЯ \\
        \textbf{«БЕЛГОРОДCКИЙ ГОCУДАРCТВЕННЫЙ ТЕХНОЛОГИЧЕCКИЙ\\УНИВЕРCИТЕТ им. В. Г. ШУХОВА»\\ (БГТУ им. В.Г. Шухова)} \\
        \bigbreak
        \includegraphics[width=70mm]{log}\\
        ИНСТИТУТ ИНФОРМАЦИОННЫХ ТЕХНОЛОГИЙ И УПРАВЛЯЮЩИХ СИСТЕМ\\}
\end{center}

\vfill
\begin{center}
    \large{
        \textbf{
            Лабораторная работа №2}}\\
    \normalsize{
        по дисциплине: Параллельное программирование \\
        тема: «Реализация параллелизма в рамках стандарта OpenMP»}
\end{center}
\vfill
\hfill\textbox{
    Выполнил: ст. группы ПВ-223\\Пахомов Владислав Андреевич
    \bigbreak
    Проверили: \\доц. Островский Алексей Мичеславович
}
\vfill\begin{center}
    Белгород 2025 г.
\end{center}
\newpage
\underline{\textbf{Цель работы: }}изучить возможности стандарта OpenMP для создания многопоточных
программ, изучить механизмы управления потоками и различные стратегии распределения
работы между потоками, их влияние на производительность вычислений.\\
\underline{\textbf{Условие индивидуального задания: }}\\

$S = \sum_{i = 1}^{N} \frac{cos(i^3) + i^4 e^{-i} + ln(i + 1)}{\sqrt{i ^ 2 + tan(i) + 1} + i!}$
\begin{center}
\textbf{Ход выполнения работы}
\end{center}

В отличие от прошлого задания, самая тяжёлая задача, которую можно выполнить только синхронно (а именно расчёт
факториалов) была выполнена перед запуском кода, что позволило составить таблицу факториалов.

\textbf{Исходный код:}
\begin{minted}{c}
#include <stdio.h>
#include <omp.h>
#include <math.h>
#include <stdlib.h>

#define NUM_ITERATIONS 10000000

int main()
{
    // Непараллельные вычисления факториалов
    long double *factorials = malloc(NUM_ITERATIONS * sizeof(long double));
    factorials[0] = 1;

    for (int i = 1; i < NUM_ITERATIONS; i++)
    {
        factorials[i] = factorials[i - 1] * (i + 1.0);
    }

    long double *sums = malloc(NUM_ITERATIONS * sizeof(long double));
    long double sum = 0.0;

#pragma omp parallel
    {
#pragma omp for schedule(auto)
        for (int i = 0; i < NUM_ITERATIONS; i++)
        {
            sums[i] = (cos(pow(i, 3)) + pow(i, 4) * exp(-i) + log(i + 1)) /
                      (sqrt(i * i + tan(i) + 1) + factorials[i]);
        }

#pragma omp for reduction(+ : sum)
        for (int i = 0; i < NUM_ITERATIONS; i++)
        {
            sum += sums[i];
        }
    }

    free(factorials);
    free(sums);

    return 0;
}
\end{minted}

\begin{minted}{Makefile}
# Компилятор
CC=gcc

# Имя исполняемого файла
EXECUTABLE=lab2


# Основная задача
all: 
	$(CC) -fopenmp lab2.c -o $(EXECUTABLE) -lm

clean:
	rm -rf *.o $(EXECUTABLE)
\end{minted}

\textbf{Результаты вычислений по времени:}\\
\begin{tabular}{|c|c|c|c|c|}
    \hline
    \textbf{Стратегия} & static & dynamic & auto & guided \\ 
    \hline
    \textbf{Время выполнения, с} & 1.344 & 1.516 & 1.353 & 1.285 \\
    \hline
\end{tabular}\\

\textbf{Вывод: } в ходе лабораторной работы изучили возможности стандарта OpenMP для создания многопоточных
программ, изучить механизмы управления потоками и различные стратегии распределения
работы между потоками, их влияние на производительность вычислений. Оптимизация вычислений и вынос
недекомпозируемой задачи и её кеширование в дальнейших вычислениях позволило сократить время выполнения программы, 
в лабораторной работе №1 было достигнуто время в 1.5 секунды, в то время как минимальное время при новом подходе
1.3 секунды. Неожиданным оказался тот факт, что самым быстрым оказалась стратегия guided, судя по всему один из компонентов
формулы имеет нелинейную сложность вычисления, следовательно и стратегия guided оказала больший эффект, так как 
при малых значениях программа выполняется быстрее следовательно и поток может выполнить больше итераций. 
Самой долгой оказалась стратегия dynamic, она здесь действительно не подходит, так как компоненты вычисления
не имеют случайную стоимость вычисления по времени. Хорошо себя показал и static, что довольно логично, 
влияние нелинейных компонентов было невелико и следовательно время увеличилось незначительно по сравнению с 
guided. Стратегия auto, если судить по времени, отдала предпочтение стратегии static. Ручная настройка 
стратегии дала небольшой рост производительности.

\end{document}