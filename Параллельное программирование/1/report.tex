\documentclass[a4paper,14pt]{extarticle}


\usepackage[utf8]{inputenc}
\usepackage[T2A]{fontenc}
\usepackage[english,russian]{babel}
\usepackage{ragged2e}
\usepackage[utf8]{inputenc}
\usepackage{hyperref}
\usepackage{minted}
\setmintedinline{frame=lines, framesep=2mm, baselinestretch=1.5, bgcolor=LightGray, breaklines,fontsize=\scriptsize}
\setminted{frame=lines, framesep=2mm, baselinestretch=1.5, bgcolor=LightGray, breaklines,fontsize=\scriptsize}
\usepackage{xcolor}
\definecolor{LightGray}{gray}{0.9}
\usepackage{graphicx}
\usepackage[export]{adjustbox}
\usepackage[left=1cm,right=1cm, top=1cm,bottom=1cm,bindingoffset=0cm]{geometry}
\usepackage{fontspec}
\usepackage{ upgreek }
\usepackage[shortlabels]{enumitem}
\usepackage{adjustbox}
\usepackage{multirow}
\usepackage{amsmath}
\usepackage{amssymb}
\usepackage{pifont}
\usepackage{pgfplots}
\usepackage{longtable}
\usepackage{array}
\graphicspath{ {./images/} }
\makeatletter
\AddEnumerateCounter{\asbuk}{\russian@alph}{щ}
\makeatother
\setmonofont{Ubuntu Mono}
\setmainfont{Times New Roman}

\newcommand\textbox[1]{
	\parbox{.45\textwidth}{#1}
} 

\newcommand{\specialcell}[2][c]{%
	\begin{tabular}[#1]{@{}c@{}}#2\end{tabular}}

\begin{document}
\pagenumbering{gobble}
\begin{center}
    \small{
        \textbf{МИНИCТЕРCТВО НАУКИ И ВЫCШЕГО ОБРАЗОВАНИЯ РОCCИЙCКОЙ ФЕДЕРАЦИИ}\\
        ФЕДЕРАЛЬНОЕ ГОCУДАРCТВЕННОЕ БЮДЖЕТНОЕ ОБРАЗОВАТЕЛЬНОЕ УЧРЕЖДЕНИЕ\\ВЫCШЕГО ОБРАЗОВАНИЯ \\
        \textbf{«БЕЛГОРОДCКИЙ ГОCУДАРCТВЕННЫЙ ТЕХНОЛОГИЧЕCКИЙ\\УНИВЕРCИТЕТ им. В. Г. ШУХОВА»\\ (БГТУ им. В.Г. Шухова)} \\
        \bigbreak
        \includegraphics[width=70mm]{log}\\
        ИНСТИТУТ ИНФОРМАЦИОННЫХ ТЕХНОЛОГИЙ И УПРАВЛЯЮЩИХ СИСТЕМ\\}
\end{center}

\vfill
\begin{center}
    \large{
        \textbf{
            Лабораторная работа №1}}\\
    \normalsize{
        по дисциплине: Параллельное программирование \\
        тема: «Сравнение парадигм конкурентности и параллелизма при разработке
        многопоточных программ в ОС Linux»}
\end{center}
\vfill
\hfill\textbox{
    Выполнил: ст. группы ПВ-223\\Пахомов Владислав Андреевич
    \bigbreak
    Проверили: \\доц. Островский Алексей Мичеславович
}
\vfill\begin{center}
    Белгород 2025 г.
\end{center}
\newpage
\underline{\textbf{Цель работы: }}исследовать чувствительность вычислительной схемы из индивидуального задания к ситуациям конкурентности, 
когда несколько потоков разделяют одно процессорное ядро; 
ситуациям параллелизма, когда каждый поток выполняется на отдельном ядре процессора
(нет конкуренции за вычислительные ресурсы).\\
\underline{\textbf{Условие индивидуального задания: }}\\

$S = \sum_{i = 1}^{N} \frac{cos(i^3) + i^4 e^{-i} + ln(i + 1)}{\sqrt{i ^ 2 + tan(i) + 1} + i!}$
\begin{center}
\textbf{Ход выполнения работы}
\end{center}

Вычислить сумму, N подобрать эмпирически для обеспечения эффективной нагрузки в
рамках вычислительного эксперимента. При необходимости предусмотреть проверку
подвыражений на принадлежность областям допустимых значений.


\textbf{Вывод: } в ходе лабораторной работы изучили основы разработки драйверов для ядра Linux с использованием языка
программирования Rust, включая настройку окружения, создание драйвера и его тестирование. Поддержка драйверов на Rust 
всё ещё очень сырая. Кроме вышеозвученных причин можно также отметить отсутствие std для модулей написанных на Rust. 
Большинство стандартных функций просто недоступны, взаимодействие происходит через create kernel, а его функционал
крайне узкий (просмотрите, например, документацию к kernel::alloc::kvec::Vec, очень много функций для вектора просто нет).
Rust for Linux всё ещё активно развивается и дополняется, однако уже сегодня можно писапть простые и безопасные драйверы
на Rust.

\end{document}