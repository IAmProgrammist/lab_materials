\documentclass[a4paper,14pt]{extarticle}


\usepackage[utf8]{inputenc}
\usepackage[T2A]{fontenc}
\usepackage[english,russian]{babel}
\usepackage{ragged2e}
\usepackage[utf8]{inputenc}
\usepackage{hyperref}
\usepackage{minted}
\setmintedinline{frame=lines, framesep=2mm, baselinestretch=1.5, bgcolor=LightGray, breaklines,fontsize=\scriptsize}
\setminted{frame=lines, framesep=2mm, baselinestretch=1.5, bgcolor=LightGray, breaklines,fontsize=\scriptsize}
\usepackage{xcolor}
\definecolor{LightGray}{gray}{0.9}
\usepackage{graphicx}
\usepackage[export]{adjustbox}
\usepackage[left=1cm,right=1cm, top=1cm,bottom=1cm,bindingoffset=0cm]{geometry}
\usepackage{fontspec}
\usepackage{ upgreek }
\usepackage[shortlabels]{enumitem}
\usepackage{adjustbox}
\usepackage{multirow}
\usepackage{amsmath}
\usepackage{amssymb}
\usepackage{pifont}
\usepackage{pgfplots}
\usepackage{longtable}
\usepackage{array}
\graphicspath{ {./images/} }
\makeatletter
\AddEnumerateCounter{\asbuk}{\russian@alph}{щ}
\makeatother
\setmonofont{Ubuntu Mono}
\setmainfont{Times New Roman}

\newcommand\textbox[1]{
	\parbox{.45\textwidth}{#1}
} 

\newcommand{\specialcell}[2][c]{%
	\begin{tabular}[#1]{@{}c@{}}#2\end{tabular}}

\begin{document}
\pagenumbering{gobble}
\begin{center}
    \small{
        \textbf{МИНИCТЕРCТВО НАУКИ И ВЫCШЕГО ОБРАЗОВАНИЯ РОCCИЙCКОЙ ФЕДЕРАЦИИ}\\
        ФЕДЕРАЛЬНОЕ ГОCУДАРCТВЕННОЕ БЮДЖЕТНОЕ ОБРАЗОВАТЕЛЬНОЕ УЧРЕЖДЕНИЕ\\ВЫCШЕГО ОБРАЗОВАНИЯ \\
        \textbf{«БЕЛГОРОДCКИЙ ГОCУДАРCТВЕННЫЙ ТЕХНОЛОГИЧЕCКИЙ\\УНИВЕРCИТЕТ им. В. Г. ШУХОВА»\\ (БГТУ им. В.Г. Шухова)} \\
        \bigbreak
        \includegraphics[width=70mm]{log}\\
        ИНСТИТУТ ИНФОРМАЦИОННЫХ ТЕХНОЛОГИЙ И УПРАВЛЯЮЩИХ СИСТЕМ\\}
\end{center}

\vfill
\begin{center}
    \large{
        \textbf{
            Лабораторная работа №1}}\\
    \normalsize{
        по дисциплине: Параллельное программирование \\
        тема: «Сравнение парадигм конкурентности и параллелизма при разработке
        многопоточных программ в ОС Linux»}
\end{center}
\vfill
\hfill\textbox{
    Выполнил: ст. группы ПВ-223\\Пахомов Владислав Андреевич
    \bigbreak
    Проверили: \\доц. Островский Алексей Мичеславович
}
\vfill\begin{center}
    Белгород 2025 г.
\end{center}
\newpage
\underline{\textbf{Цель работы: }}исследовать чувствительность вычислительной схемы из индивидуального задания к ситуациям конкурентности, 
когда несколько потоков разделяют одно процессорное ядро; 
ситуациям параллелизма, когда каждый поток выполняется на отдельном ядре процессора
(нет конкуренции за вычислительные ресурсы).\\
\underline{\textbf{Условие индивидуального задания: }}\\

$S = \sum_{i = 1}^{N} \frac{cos(i^3) + i^4 e^{-i} + ln(i + 1)}{\sqrt{i ^ 2 + tan(i) + 1} + i!}$
\begin{center}
\textbf{Ход выполнения работы}
\end{center}

Декомпозируем вычислительную задачу. Будем в каждом потоке вычислять не все элементы от 1 до N элемента, а 
N / M элементов, где M - количество потоков. 

\textbf{Исходный код:}
\begin{minted}{c}
#define _GNU_SOURCE
#include <pthread.h>
#include <stdio.h>
#include <stdlib.h>
#include <sched.h>
#include <unistd.h>
#include <stdint.h>
#include <math.h>

#ifndef NUM_THREADS
#define NUM_THREADS 4
#endif

#define NUM_ITERATIONS 10000000

volatile long double sums[NUM_THREADS];

long double factorial(const long double n)
{
    long double f = 1;
    for (long double i = 1; i <= n; ++i)
        f *= i;
    return f;
}

void *compute(void *arg)
{
    size_t iteration_num = (size_t)arg;
    size_t lower_bound = (NUM_ITERATIONS / NUM_THREADS) * iteration_num + 1;
    size_t upper_bound = (NUM_ITERATIONS / NUM_THREADS) * (iteration_num + 1) + 1;

    long double iter_factorial = factorial(lower_bound);

    for (uint64_t iter = lower_bound; iter < upper_bound; iter++)
    {
        sums[iteration_num] += (cos(pow(iter, 3)) + pow(iter, 4) * exp(-iter) + log(iter + 1)) /
               (sqrt(iter * iter + tan(iter) + 1) + iter_factorial);
        iter_factorial *= iter;
    }
}
// Принудительное закрепление потока за конкретным ядром
void pin_thread_to_core(int core_id)
{
    cpu_set_t cpuset;
    CPU_ZERO(&cpuset);
    CPU_SET(core_id, &cpuset);
    pthread_t current_thread = pthread_self();
    pthread_setaffinity_np(current_thread, sizeof(cpu_set_t), &cpuset);
}

void *compute_pinned(void *arg)
{
    int core_id = (int)(long)arg;
    pin_thread_to_core(core_id);
    return compute(arg);
}
int main(int argc, char *argv[])
{
    pthread_t threads[NUM_THREADS];
    if (argc < 2)
    {
        fprintf(stderr, "Запуск: %s <mode>\n"
                        "Опции:\n 1 - конкурентность, одно ядро\n"
                        "2 - параллелизм, разные ядра\n",
                argv[0]);
        return EXIT_FAILURE;
    }
    int mode = atoi(argv[1]);
    printf("Старт: %s, %d поток(а)(ов)...\n",
           mode == 1 ? "конкурентность, одно ядро" : "параллелизм, разные ядра", NUM_THREADS);
    for (size_t iter = 0; iter < NUM_THREADS; iter++)
    {
        sums[iter] = 0;
        if (mode == 1)
        {
            pthread_create(&threads[iter], NULL, compute, (void *)iter);
        }
        else
        {
            pthread_create(&threads[iter], NULL, compute_pinned, (void *)iter);
        }
    }
    for (size_t iter = 0; iter < NUM_THREADS; iter++)
        pthread_join(threads[iter], NULL);
    
    long double sum = 0;
    for (int i = 0; i < NUM_THREADS; i++) {
        sum += sums[i];
    }

    return EXIT_SUCCESS;
}
\end{minted}

\begin{minted}{Makefile}
# Компилятор
CC=gcc

# Флаги компиляции
CFLAGS=-c -Wall

# Флаги линковки
LDFLAGS=-pthread

# Оверрайд переменных компиляции
DFLAGS="-DNUM_THREADS=2"

# Имя исполняемого файла
EXECUTABLE=lab1

# Исходники
SOURCES=lab1.c

# Объектные файлы
OBJECTS=$(SOURCES:.c=.o)


# Основная задача
all: $(SOURCES) $(EXECUTABLE)
	
# Задача по линковке. Линкует все .o файлы в один
$(EXECUTABLE): $(OBJECTS) 
	$(CC) $(LDFLAGS) $(OBJECTS) -o $@ -lm

# Задача по компиляции, компилирует все .c файлы в .o
%.o : %.c
	$(CC) $(DFLAGS) -c $(CFLAGS) $< -o $@

clean:
	rm -rf *.o $(EXECUTABLE)
\end{minted}

\begin{minted}{bash}
#!/bin/bash

echo "Компиляция..."
echo ""

MAX_THREADS=10

make clean
for i in $(seq 1 $MAX_THREADS); do 
    make DFLAGS="-DNUM_THREADS=$i";
    mv ./lab1 ./lab1_$((i))_threads
    make clean
done

echo "Запуск конкуренции: "
echo ""

for i in $(seq 1 $MAX_THREADS); do 
    time taskset -c 0 ./lab1_$((i))_threads 1
done

echo "Запуск параллелизма: "
echo ""

for i in $(seq 1 $MAX_THREADS); do 
    time ./lab1_$((i))_threads 2
done

echo "Очистка: "
echo ""

for i in $(seq 1 $MAX_THREADS); do 
    rm ./lab1_$((i))_threads
done
\end{minted}

Результаты тестирования:\\
\includegraphics[width=140mm]{graph}

\textbf{Вывод: } в ходе лабораторной работы исследовали чувствительность вычислительной схемы из индивидуального задания 
к ситуациям конкурентности, когда несколько потоков разделяют одно процессорное ядро; 
ситуациям параллелизма, когда каждый поток выполняется на отдельном ядре процессора
(нет конкуренции за вычислительные ресурсы). Ожидаемо при конкуренции за ресурсы росла и нагрузка на выбранное ядро, 
следовательно и время работы тоже увеличивалось. Опыт показал, что время работы увеличивалось линейно с 
увеличением потоков. В условиях отсутствия конкурентности время выполнения лишь падало, так как ядра процессора
могли выполнять только свою задачу и не забивали задачами одно ядро. Причём после увеличения количества ядер до 6
дальнейшее увеличение количества задействованных ядер сильно на результат не влияло (что соответствует
количеству производительных ядер процессора автора отчёта Intel Core i5 12600K). В условиях конкуренции за 
ресурсы увеличение потоков и декомпозиция задачи бессмысленна. А также в условиях отсутствия конкуренции 
бессмысленна и отсутствие декомпозиция задачи и её распараллеливание. Причём эффективное количество потоков 
соответствует количеству ядер процессора.

\end{document}