\documentclass[a4paper,14pt]{extarticle}

\usepackage{subfiles}

\usepackage[english,russian]{babel}
\usepackage[T2A]{fontenc}
\usepackage[utf8]{inputenc}
\usepackage{ragged2e}
\usepackage{hyperref}
\usepackage{minted}
\setmintedinline{frame=lines, framesep=2mm, baselinestretch=1.5, bgcolor=LightGray, breaklines,fontsize=\scriptsize}
\setminted{frame=lines, framesep=2mm, baselinestretch=1.5, bgcolor=LightGray, breaklines,fontsize=\scriptsize}
\usepackage{xcolor}
\definecolor{LightGray}{gray}{0.9}
\usepackage{graphicx}
\usepackage[export]{adjustbox}
\usepackage[left=1cm,right=1cm, top=1cm,bottom=1cm,bindingoffset=0cm]{geometry}
\usepackage{fontspec}
\usepackage{ upgreek }
\usepackage[shortlabels]{enumitem}
\usepackage{adjustbox}
\usepackage{multirow}
\usepackage{amsmath}
\usepackage{amssymb}
\usepackage{pifont}
\usepackage{pgfplots}
\usepackage{longtable}
\usepackage{slashbox}
\usepackage{array}

\graphicspath{ {./images/} }
\makeatletter
\AddEnumerateCounter{\asbuk}{\russian@alph}{щ}
\makeatother
\setmonofont{Consolas}
\setmainfont{Times New Roman}

\newcommand\textbox[1]{
	\parbox{.45\textwidth}{#1}
} 

\newcommand{\specialcell}[2][c]{%
	\begin{tabular}[#1]{@{}c@{}}#2\end{tabular}}

\begin{document}
\pagenumbering{gobble}
\begin{center}
    \small{
        \textbf{МИНИCТЕРCТВО НАУКИ И ВЫCШЕГО ОБРАЗОВАНИЯ РОCCИЙCКОЙ ФЕДЕРАЦИИ}\\
        ФЕДЕРАЛЬНОЕ ГОCУДАРCТВЕННОЕ БЮДЖЕТНОЕ ОБРАЗОВАТЕЛЬНОЕ УЧРЕЖДЕНИЕ\\ВЫCШЕГО ОБРАЗОВАНИЯ \\
        \textbf{«БЕЛГОРОДCКИЙ ГОCУДАРCТВЕННЫЙ ТЕХНОЛОГИЧЕCКИЙ\\УНИВЕРCИТЕТ им. В. Г. ШУХОВА»\\ (БГТУ им. В.Г. Шухова)} \\
        \bigbreak
        \includegraphics[width=70mm]{log}\\
        ИНСТИТУТ ИНФОРМАЦИОННЫХ ТЕХНОЛОГИЙ И УПРАВЛЯЮЩИХ СИСТЕМ\\}
\end{center}

\vfill
\begin{center}
    \large{
        \textbf{
            Лабораторная работа №7}}\\
    \normalsize{
        по дисциплине: Исследование операций \\
        тема: «Решение полностью целочисленных задач с помощью первого
        алгоритма Гомори, а также методом ветвей и границ»}
\end{center}
\vfill
\hfill\textbox{
    Выполнил: ст. группы ПВ-223\\Пахомов Владислав Андреевич
    \bigbreak
    Проверили: \\проф. Вирченко Юрий Петрович
}
\vfill\begin{center}
    Белгород 2024 г.
\end{center}
\newpage
\begin{center}
    \textbf{Лабораторная работа №7}\\
    Решение полностью целочисленных задач с помощью первого
    алгоритма Гомори, а также методом ветвей и границ\\
\end{center}
\textbf{Цель работы: }освоить метод отсечения Гомори для полностью
целочисленных задач. Изучить алгоритм этого метода. Программно
реализовать этот алгоритм.\\

\textbf{Задание: }выяснить для каких задач применяется первый алгоритм Гомори.
Изучить этот алгоритм и написать реализующую его программу для
ПЭВМ. Изучить и программно реализовать алгоритм метода ветвей
и границ. В качестве тестовых данных использовать, решенную
вручную одну из нижеследующих задач.

\begin{center}
    \textbf{Вариант 10}
\end{center}
\begin{equation*}
    \begin{aligned}
        z = 9x_1 + x_2 + 3x_5 \rightarrow max; \\
        \begin{cases}
            10x_1 + 3x_2 + x_3 = 40, \\
            9x_1 - 4x_2 + x_4 = 7,   \\
            -4x_1 + x_2 + x_5 = 14,  \\
        \end{cases}               \\
        x_i \geq 0 (i=\overline{1, 5}).
    \end{aligned}
\end{equation*}\\

\textbf{Блок-схемы: }
\begin{center}
    \subfile{sources/solveInInt.tex}\bigbreak
    \subfile{sources/method_branch_and_bound.tex}\bigbreak
\end{center}

\textbf{Листинг программы: }
\begin{minted}{C++}
#include <vector>
#include <array>

#include "../libs/alg/alg.h"

template <std::size_t T, std::size_t MatrixLines>
std::tuple<Fraction, std::vector<Fraction>> solveInInt(std::vector<std::array<Fraction, T>> matrix, std::array<Fraction, T> function);

template <>
std::tuple<Fraction, std::vector<Fraction>> solveInInt<15ULL, 12ULL>(std::vector<std::array<Fraction, 15ULL>> matrix, std::array<Fraction, 15ULL> function) {
    throw std::invalid_argument("Too much!");
}

template <std::size_t T, std::size_t MatrixLines>
std::tuple<Fraction, std::vector<Fraction>> solveInInt(std::vector<std::array<Fraction, T>> matrix, std::array<Fraction, T> function) {
    std::cout << "Simplex" << std::endl;
    // Применяем симплекс метод
    auto res = solveSimplexMethodMaxRaw(matrix, function, Fraction());

    // Ищем нецелые свободные члены с максимальным остатком
    int maxFracIndex = -1;
    std::vector<int> freeIndices;
    for (int i = 0; i < matrix.size(); i++) {
        if (!matrix[i].back().isInt()) {
            if (maxFracIndex == -1) {
                maxFracIndex = i;
                continue;
            }

            auto newDouble = matrix[i].back().getFrac();
            auto oldDouble = matrix[maxFracIndex].back().getFrac();
            if (newDouble > oldDouble) maxFracIndex = i;
        }
    }

    // Определяем свободные переменные
    for (int j = 0; j < T; j++) {
        bool metOne = false;
        bool onlyOnesAndZeros = true;
        for (int i = 0; i < MatrixLines; i++) {
            if (!(matrix[i][j] == Fraction() || matrix[i][j] == Fraction(1))) {
                onlyOnesAndZeros = false;
                break;
            }

            if (matrix[i][j] == Fraction(1)) {
                if (metOne) {
                    onlyOnesAndZeros = false;
                    break;
                } else {
                    metOne = true;
                }
            }
        }

        if (!onlyOnesAndZeros) freeIndices.push_back(j);
    }

    // Если все свободные члены целые
    if (maxFracIndex == -1) {
        // Возвращаем ответ
        std::vector<Fraction> ansB;
        for (int i = 0; i < MatrixLines; i++) {
            ansB.push_back(matrix[i].back());
        }

        return {res, ansB};
    }

    // Формируем сечение
    std::array<Fraction, T + 1> newLine;
    newLine[T] = matrix[maxFracIndex].back().getFrac() * Fraction(-1);
    newLine[T - 1] = Fraction(1);
    for (int i = 0; i < T - 1; i++) {
        if (std::find(freeIndices.begin(), freeIndices.end(), i) != freeIndices.end()) {
            newLine[i] = Fraction(-1) * matrix[maxFracIndex][i].getFrac();
        }
    }

    std::vector<std::array<Fraction, T + 1>> simplexMatrix;
    for (int i = 0; i < matrix.size(); i++) {
        simplexMatrix.push_back({});
        for (int j = 0; j < T - 1; j++) {
            simplexMatrix[i][j] = matrix[i][j];
        }

        simplexMatrix[i][T] = matrix[i].back();
    }
    simplexMatrix.push_back(newLine);
    std::array<Fraction, T + 1> funcLine = {};
    for (int i = 0; i < T - 1; i++) {
        funcLine[i] = function[i];
    }
    funcLine[T] = function.back();
    simplexMatrix.push_back(funcLine);


    std::cout << "Marks" << std::endl;
    // Используем метод последовательного уточнения оценок
    while (true) {
        for (int i = 0; i < simplexMatrix.size(); i++) {
            for (int j = 0; j < T + 1; j++) {
                std::cout << std::setw(10) << simplexMatrix[i][j] << " ";
            }
            std::cout << std::endl;
        }
        std::cout << std::endl;

        int minRowIndex = -1;
        for (int i = 0; i < MatrixLines + 1; i++)
            if (simplexMatrix[i].back() < Fraction() && (minRowIndex == -1 || simplexMatrix[i].back() < simplexMatrix[minRowIndex].back()))
                minRowIndex = i;
        
        if (minRowIndex == -1) break;

        int minColumnIndex = -1;
        for (int i = 0; i < T + 1; i++)
            if (simplexMatrix[minRowIndex][i] < Fraction() && (minColumnIndex == -1 || 
            (Fraction(-1) * simplexMatrix.back()[i] / simplexMatrix[minRowIndex][i]) < (Fraction(-1) * simplexMatrix.back()[minColumnIndex] / simplexMatrix[minRowIndex][minColumnIndex])))
                minColumnIndex = i;

        if (minColumnIndex == -1) throw std::invalid_argument("No solution"); 

        subtractLineFromOther(simplexMatrix, minRowIndex, minColumnIndex, Fraction());
    }

    std::array<Fraction, T + 1> newFunction = {};
    for (int i = 0; i < T + 1; i++) {
        newFunction[i] = simplexMatrix.back()[i] * Fraction(-1);
    }

    simplexMatrix.pop_back();

    // Решаем следующую задачу
    return solveInInt<T + 1, MatrixLines + 1>(simplexMatrix, newFunction);
}

int main() {
    std::vector<std::array<Fraction, 6>> matrix;
    matrix.push_back({{{10}, {3}, {1}, {0}, {0}, {40}}});
    matrix.push_back({{{9}, {-4}, {0}, {1}, {0}, {7}}});
    matrix.push_back({{{-4}, {1}, {0}, {0}, {1}, {14}}});
    std::array<Fraction, 6> function{{{9}, {1}, {0}, {0}, {3}, {0}}};
    Fraction EPS;

    auto res = solveInInt<6, 3>(matrix, function);
    std::cout << std::get<0>(res) << std::endl;
}
\end{minted}

\begin{minted}{C++}
from scipy.optimize import linprog
import numpy as np
import math as m

A = np.array([[7, 5, 1, 0, 0],
              [4, -6, 0, 3, 0],
              [-3, 4, 0, 0, 1]])
b = np.array([28, 14, 6])
c = np.array([3, 1, 0, 0, - 1])


max_res_simplex_method = None
def method_branch_and_bound(A, b, c, bounds):
    global max_res_simplex_method
    # получаем решение симплекс методом
    res = linprog(-c, A_ub=A, b_ub=b, bounds=bounds, method='highs')
    # если решения нет выходим из рекурсии
    if not res.success:
        return None

    print("Р:", res.x)
    print("Макс ф:", -res.fun)

    # обновляем результат
    if all(is_int(res.x)) and (max_res_simplex_method is None or -res.fun > -max_res_simplex_method.fun):
        max_res_simplex_method = res

    # цикл, если имеются нецелые решения делаем ветку
    for index, x in zip(range(len(res.x)), res.x):
        if not is_int(x):
            bounds_for_bigger_than = bounds.copy()
            bounds_for_bigger_than[index] = (m.ceil(x), None)
            bounds_for_less_than = bounds.copy()
            bounds_for_less_than[index] = (0, m.floor(x))

            method_branch_and_bound(A, b, c, bounds_for_bigger_than)
            method_branch_and_bound(A, b, c, bounds_for_less_than)

    # выводим результат
    return max_res_simplex_method


def is_int(n, epsilon=0.001):
    return abs(n % 1) <= epsilon


if __name__ == '__main__':
    print("Промежуточные решения и целевые функции:")
    res = method_branch_and_bound(A, b, c, [(0, None)]*len(c))
    print()
    print("Решение:", res.x)
    print("Максимизированная функция:", -res.fun)
\end{minted}

Результаты выполнения программы:
\begin{minted}{console}
78
\end{minted}
\begin{minted}{console}
Решение: [ 2.  3.  0.  0. 19.]
Максимизированная функция: 78.0
\end{minted}

\textbf{Результаты вычислений: }\\
\begin{equation*}
    \begin{aligned}
        z = 9x_1 + x_2 + 3x_5 \rightarrow max; \\
        \begin{cases}
            10x_1 + 3x_2 + x_3 = 40, \\
            9x_1 - 4x_2 + x_4 = 7,   \\
            -4x_1 + x_2 + x_5 = 14,  \\
        \end{cases}               \\
        x_i \geq 0 (i=\overline{1, 5}).
    \end{aligned}
\end{equation*}\\

Сформируем задачу З0:
\begin{center}
    \begin{longtable}{|c|c|c|c|c|c|c|}
        \hline
        Баз. пер. & Св. чл. & $x_1$ & $x_2$ & $x_3$ & $x_4 $ & $x_5$ \\
        \hline
        $x_3$     & $40$    & $10$  & $3$   & $1$   & $0$    & $0$   \\
        \hline
        $x_4$     & $7$     & $9$   & $-4$  & $0$   & $1$    & $0$   \\
        \hline
        $x_5$     & $14$    & $-4$  & $1$   & $0$   & $0$    & $1$   \\
        \hline
        $z$       & $0$     & $-9$  & $-1$  & $0$   & $0$    & $-3$  \\
        \hline
        \multicolumn{7}{c}{}                                         \\
        \hline
        Баз. пер. & Св. чл. & $x_1$ & $x_2$ & $x_3$ & $x_4 $ & $x_5$ \\
        \hline
        $x_3$     & 290/9   & 0     & 67/9  & $1$   & -10/9  & $0$   \\
        \hline
        $x_1$     & 7/9     & 1     & -4/9  & $0$   & 1/9    & $0$   \\
        \hline
        $x_5$     & 154/9   & 0     & -7/9  & $0$   & 4/9    & $1$   \\
        \hline
        $z$       & 7       & 0     & -5    & $0$   & 1      & $-3$  \\
        \hline
        \multicolumn{7}{c}{}                                         \\
        \hline
        Баз. пер. & Св. чл. & $x_1$ & $x_2$ & $x_3$ & $x_4 $ & $x_5$ \\
        \hline
        $x_2$     & 290/67  & 0     & 1     & 9/67  & -10/67 & $0$   \\
        \hline
        $x_1$     & 181/67  & 1     & 0     & 4/67  & 3/67   & $0$   \\
        \hline
        $x_5$     & 1372/67 & 0     & 0     & 7/67  & 22/67  & $1$   \\
        \hline
        $z$       & 1919/67 & 0     & 0     & 45/67 & 17/67  & $-3$  \\
        \hline
        \multicolumn{7}{c}{}                                         \\
        \hline
        Баз. пер. & Св. чл. & $x_1$ & $x_2$ & $x_3$ & $x_4 $ & $x_5$ \\
        \hline
        $x_2$     & 290/67  & 0     & 1     & 9/67  & -10/67 & $0$   \\
        \hline
        $x_1$     & 181/67  & 1     & 0     & 4/67  & 3/67   & $0$   \\
        \hline
        $x_5$     & 1372/67 & 0     & 0     & 7/67  & 22/67  & $1$   \\
        \hline
        $z$       & 6035/67 & 0     & 0     & 66/67 & 83/67  & 0     \\
        \hline
    \end{longtable}
\end{center}

Так как план содержит нецелые числа, необходимо сделать новое сечение
И получить задачу З1.
\begin{center}
    \begin{longtable}{|c|c|c|c|c|c|c|c|}
        \hline
        Баз. пер. & Св. чл. & $x_1$ & $x_2$ & $x_3$ & $x_4 $ & $x_5$ & $v_1$ \\
        \hline
        $x_2$     & 290/67  & 0     & 1     & 9/67  & -10/67 & $0$   & 0     \\
        \hline
        $x_1$     & 181/67  & 1     & 0     & 4/67  & 3/67   & $0$   & 0     \\
        \hline
        $x_5$     & 1372/67 & 0     & 0     & 7/67  & 22/67  & $1$   & 0     \\
        \hline
        $v_1$     & -47/67  & 0     & 0     & -4/67 & -3/67  & 0     & 1     \\
        \hline
        $z$       & 6035/67 & 0     & 0     & 66/67 & 83/67  & 0     & 0     \\
        \hline
        \multicolumn{8}{c}{}\\
        \hline
        Баз. пер. & Св. чл. & $x_1$ & $x_2$ & $x_3$ & $x_4 $ & $x_5$ & $v_1$ \\
        \hline
        $x_2$     & 11/4    & 0     & 1     & 0     & -1/4   & $0$   & 9/4     \\
        \hline
        $x_1$     & 2       & 1     & 0     & 0     & 0      & $0$   & 1     \\
        \hline
        $x_5$     & 77/4    & 0     & 0     & 0     & 1/4    & $1$   & 7/4     \\
        \hline
        $x_3$     & 47/4    & 0     & 0     & 1     & 3/4    & 0     & -67/4     \\
        \hline
        $z$       & 157/2   & 0     & 0     & 0     & 1/2    & $0$  & 33/2     \\
        \hline
    \end{longtable}
\end{center}
Симплекс метод неприменим.
Так как план содержит нецелые числа, необходимо сделать новое сечение
И получить задачу З2.

\begin{center}
    \begin{longtable}{|c|c|c|c|c|c|c|c|c|}
        \hline
        Баз. пер. & Св. чл. & $x_1$ & $x_2$ & $x_3$ & $x_4 $ & $x_5$ & $v_1$ & $v_2$ \\
        \hline
        $x_2$     & 11/4    & 0     & 1     & 0     & -1/4   & $0$   & 9/4  & 0   \\
        \hline
        $x_1$     & 2       & 1     & 0     & 0     & 0      & $0$   & 1   & 0  \\
        \hline
        $x_5$     & 77/4    & 0     & 0     & 0     & 1/4    & $1$   & 7/4  & 0   \\
        \hline
        $x_3$     & 47/4    & 0     & 0     & 1     & 3/4    & 0     & -67/4  & 0   \\
        \hline
        $v_2$     & -3/4    & 0     & 0     & 0     & -3/4    & 0     & -1/4  & 1   \\
        \hline
        $z$       & 157/2   & 0     & 0     & 0     & 1/2    & $0$  & 33/2  & 0   \\
        \hline
        \multicolumn{9}{c}{}\\
        \hline
        $x_2$     & 3    & 0     & 1     & 0     & 0   & $0$   & 7/3  & -1/3   \\
        \hline
        $x_1$     & 2       & 1     & 0     & 0     & 0      & $0$   & 1   & 0  \\
        \hline
        $x_5$     & 19    & 0     & 0     & 0     & 0    & $1$   & 5/3  & 1/3   \\
        \hline
        $x_3$     & 11    & 0     & 0     & 1     & 0    & 0     & -17  & 1   \\
        \hline
        $x_5$     & 1    & 0     & 0     & 0     & 1    & 0     & 1/3  & -4/3   \\
        \hline
        $z$       & 78   & 0     & 0     & 0     & 0    & $0$  & 49/3  & 2/3   \\
        \hline
    \end{longtable}
\end{center}

Симплекс метод неприменим.\\
План содержит только целые числа.\\
Ответ: [2, 3, 0, 0, 19], $z_{max} = 78$\\

\textbf{Вывод: } в ходе лабораторной работы освоили метод отсечения Гомори для полностью
целочисленных задач. Изучить алгоритм этого метода. Программно
реализовали и отладили программу, реализующую этот алгоритм. 

\end{document}