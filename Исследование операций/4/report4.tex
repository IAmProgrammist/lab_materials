\documentclass[a4paper,14pt]{extarticle}

\usepackage{subfiles}

\usepackage[english,russian]{babel}
\usepackage[T2A]{fontenc}
\usepackage[utf8]{inputenc}
\usepackage{ragged2e}
\usepackage{hyperref}
\usepackage{minted}
\setmintedinline{frame=lines, framesep=2mm, baselinestretch=1.5, bgcolor=LightGray, breaklines,fontsize=\scriptsize}
\setminted{frame=lines, framesep=2mm, baselinestretch=1.5, bgcolor=LightGray, breaklines,fontsize=\scriptsize}
\usepackage{xcolor}
\definecolor{LightGray}{gray}{0.9}
\usepackage{graphicx}
\usepackage[export]{adjustbox}
\usepackage[left=1cm,right=1cm, top=1cm,bottom=1cm,bindingoffset=0cm]{geometry}
\usepackage{fontspec}
\usepackage{ upgreek }
\usepackage[shortlabels]{enumitem}
\usepackage{adjustbox}
\usepackage{multirow}
\usepackage{amsmath}
\usepackage{amssymb}
\usepackage{pifont}
\usepackage{pgfplots}
\usepackage{longtable}
\usepackage{slashbox}
\usepackage{array}

\graphicspath{ {./images/} }
\makeatletter
\AddEnumerateCounter{\asbuk}{\russian@alph}{щ}
\makeatother
\setmonofont{Consolas}
\setmainfont{Times New Roman}

\newcommand\textbox[1]{
	\parbox{.45\textwidth}{#1}
} 

\newcommand{\specialcell}[2][c]{%
	\begin{tabular}[#1]{@{}c@{}}#2\end{tabular}}

\begin{document}
\pagenumbering{gobble}
\begin{center}
    \small{
        \textbf{МИНИCТЕРCТВО НАУКИ И ВЫCШЕГО ОБРАЗОВАНИЯ РОCCИЙCКОЙ ФЕДЕРАЦИИ}\\
        ФЕДЕРАЛЬНОЕ ГОCУДАРCТВЕННОЕ БЮДЖЕТНОЕ ОБРАЗОВАТЕЛЬНОЕ УЧРЕЖДЕНИЕ\\ВЫCШЕГО ОБРАЗОВАНИЯ \\
        \textbf{«БЕЛГОРОДCКИЙ ГОCУДАРCТВЕННЫЙ ТЕХНОЛОГИЧЕCКИЙ\\УНИВЕРCИТЕТ им. В. Г. ШУХОВА»\\ (БГТУ им. В.Г. Шухова)} \\
        \bigbreak
        \includegraphics[width=70mm]{log}\\
        ИНСТИТУТ ИНФОРМАЦИОННЫХ ТЕХНОЛОГИЙ И УПРАВЛЯЮЩИХ СИСТЕМ\\}
\end{center}

\vfill
\begin{center}
    \large{
        \textbf{
            Лабораторная работа №4}}\\
    \normalsize{
        по дисциплине: Исследование операций \\
        тема: «Закрытая транспортная задача»}
\end{center}
\vfill
\hfill\textbox{
    Выполнил: ст. группы ПВ-223\\Пахомов Владислав Андреевич
    \bigbreak
    Проверили: \\проф. Вирченко Юрий Петрович
}
\vfill\begin{center}
    Белгород 2024 г.
\end{center}
\newpage
\begin{center}
    \textbf{Лабораторная работа №4}\\
    Закрытая транспортная задача\\
\end{center}
\textbf{Цель работы: }изучить математическую модель транспортной
задачи, овладеть методами решения этой задачи.\\

\textbf{Задание: }составить и отладить программы решения транспортной задачи
распределительным методом и методом потенциалов. В рамках
подготовки тестовых данных решить задачу вручную.

\begin{center}
    \textbf{Вариант 10}
\end{center}
\begin{center}
    $\vec{a}=(14,14,14,14);$\\
    $\vec{b}=(13,5,13,12,13);$\\
    $C = \begin{pmatrix}
        16&26&12&24&3\\
        5&2&19&27&2\\
        29&23&25&16&8\\
        2&25&14&15&21
    \end{pmatrix}$\\
\end{center}

\textbf{Блок-схемы: }
\begin{center}
    \subfile{sources/main.tex}\bigbreak
    \subfile{sources/getTransportTaskBasisLeastCost.tex}\bigbreak
    \subfile{sources/getTransportTaskCycles.tex}\bigbreak
    \subfile{sources/moveTransportTaskCycle.tex}\bigbreak
    \subfile{sources/getAnswerTransportTask.tex}\bigbreak
    \subfile{sources/solveTransportTaskDistributionMethod.tex}\bigbreak
    \subfile{sources/recalculatePotentials.tex}\bigbreak
    \subfile{sources/solveTransportTaskPotentials.tex}\bigbreak
\end{center}

\textbf{Листинг программы: }
\begin{minted}{C++}
#include <iostream>
#include <iomanip>
#include <windows.h>
#include <limits>
#include <algorithm>
#include <array>

#include "../libs/alg/alg.h"

int main() {
    // Подготовить входные данные
    std::vector<std::array<double, 5>> c = {
    {{{16}, {26}, {12}, {24}, {3}}},
    {{{5}, {2}, {19}, {27}, {2}}},
    {{{29}, {23}, {25}, {16}, {8}}},
    {{{2}, {25}, {14}, {15}, {21}}}};
    
    std::array<double, 4> a{{{14}, {14 }, {14}, {14}}};
    std::array<double, 5> b{{{13}, {5}, {13}, {12}, {13}}};

    std::cout << solveTransportTaskPotentials(c, a, b, 0.00000001);
}
\end{minted}
\href{https://github.com/IAmProgrammist/operations_research/blob/master/src/lab4/main.cpp}{Ссылка на репозиторий}\\

\begin{minted}{C++}
#pragma once 

#include <vector>
#include <tuple>

#include "../fraction.hpp"

template <std::size_t T, std::size_t MatrixLines, typename CountType>
void getTransportTaskBasisNorthEast(std::vector<std::array<CountType, T>> &c, std::vector<std::array<CountType, T>> &x, std::array<CountType, MatrixLines> &a, std::array<CountType, T> &b, CountType EPS) {
    int k = 0;
    int r = 0;

    // Пока не дошли до края матрицы
    while (k < MatrixLines && r < T) {
        // Если ak == br
        if (abs(a[k] - b[r]) < EPS) {
            if (r == T - 1) {
                // Вычёркиваем строчку
                x[k][r] = b[r];
                a[k] -= b[r];
                r++;
            } else {
                // Вычёркиваем столбец
                x[k][r] = a[k];
                b[r] -= a[k];
                k++;
            }
        // Если ak > br
        } else if (a[k] > b[r]) {
            // Вычёркиваем строчку
            x[k][r] = b[r];
            a[k] -= b[r];
            r++;
        // Если ak < br
        } else {
            // Вычёркиваем столбец
            x[k][r] = a[k];
            b[r] -= a[k];
            k++;
        }
    }
}

template <std::size_t T, std::size_t MatrixLines, typename CountType>
void getTransportTaskBasisLeastCost(std::vector<std::array<CountType, T>> &c, std::vector<std::array<CountType, T>> &x, std::array<CountType, MatrixLines> &a, std::array<CountType, T> &b, CountType EPS) {
    std::array<bool, T> usedCols = {};
    std::array<bool, MatrixLines> usedRows = {};
    
    // Бесконечный цикл
    while(true) {
        bool foundAny = false;
        int iMin, jMin;

        // Для каждого элемента cij, оставшегося в матрице
        for (int i = 0; i < MatrixLines; i++) {
            if (usedRows[i]) continue;
            
            for (int j = 0; j < T; j++) {
                if (usedCols[j]) continue;

                // Если cij < cmin
                if (!foundAny || c[iMin][jMin] > c[i][j]) {
                    // Обновлям данные
                    iMin = i;
                    jMin = j;

                    foundAny = true;
                }
            }
        }

        // Если элемент не найден, выходим из цикла
        if (!foundAny) break;

        // Если ai >= bj
        if (abs(a[iMin] - b[jMin]) < EPS || a[iMin] > b[jMin]) {
            // Вычёркиваем строчку, обновляем x, a
            usedCols[jMin] = true;
            x[iMin][jMin] = b[jMin];
            a[iMin] -= b[jMin];
        // Иначе
        } else {
            // Вычёркиваем столбец, обновляем x, b
            usedRows[iMin] = true;
            x[iMin][jMin] = a[iMin];
            b[jMin] -= a[iMin];
        }
    }
}

template <std::size_t T, std::size_t MatrixLines, typename CountType>
std::vector<std::tuple<CountType, CountType, std::vector<std::pair<int, int>>>> getTransportTaskCycles(std::vector<std::array<CountType, T>> c, std::vector<std::array<CountType, T>> x, 
std::array<CountType, MatrixLines> a, std::array<CountType, T> b, std::pair<int, int> init, std::vector<std::pair<int, int>> result, std::pair<int, int> curr,
CountType min, bool minAssigned, CountType sum, int it, int state, CountType EPS) {
    std::vector<std::tuple<CountType, CountType, std::vector<std::pair<int, int>>>> recResult;
    std::vector<std::pair<int, int>> selectedNodes;

    // Ищем подходящие вершины и сохраняем в selectedNodes
    if (state == -1 || state == 0) {
        // Выполняем поиск по столбцу, если ранее был выполнен поиск по строчке
        // или мы находимся в начальной вершине
        state = 1;
        
        // Пропускаем текущий элемент и элементы, которые уже находятся в пути. 
        // Остальные элементы с xij != 0 добавляем в selectedNodes
        for (int k = 0; k < MatrixLines; k++) {
            if (k == curr.first) continue;
            if ((abs(x[k][curr.second]) > EPS && std::find(result.begin(), result.end(), std::pair<int, int>(k, curr.second)) == result.end()) || 
            (it > 1 && std::pair<int, int>(k, curr.second) == init)) {
                selectedNodes.push_back({k, curr.second});
            } 
        }
    } else {
        // Иначе по строке
        state = 0;
        
        // Пропускаем текущий элемент и элементы, которые уже находятся в пути. 
        // Остальные элементы с xij != 0 добавляем в selectedNodes
        for (int k = 0; k < T; k++) {
            if (k == curr.second) continue;
            if ((abs(x[curr.first][k]) > EPS && std::find(result.begin(), result.end(), std::pair<int, int>(curr.first, k)) == result.end()) || 
            (it > 1 && std::pair<int, int>(curr.first, k) == init)) {
                selectedNodes.push_back({curr.first, k});
            } 
        }
    }

    // Если вершин не найдено, возвращаем пустой массив
    if (selectedNodes.empty()) {
        return {};
    }

    it++;
    // Для каждой найденной вершины из selectedNodes
    for (auto& node : selectedNodes) {
        // Если достигли начальной вершины, добавляем в конечный результат цикл
        // и переходим к следующей вершине
        if (node == init) {
            recResult.push_back({sum, min, result});
            continue;
        }

        // Обновляем минимум, если вершина отрицательная
        auto newMin = min;
        auto newMinAssigned = minAssigned;
        if (it % 2 && (x[node.first][node.second] < newMin || !newMinAssigned)) {
            newMin = x[node.first][node.second];

            newMinAssigned = true;
        }

        // Добавляем cij к sum, если вершина со знаком плюс, иначе - вычитаем
        auto newSum = sum;
        if (it % 2 == 0) {
            newSum += c[node.first][node.second];
        } else {
            newSum -= c[node.first][node.second];
        }

        // Добавляем вершину в цикл
        auto newResult = result;
        newResult.push_back(node);

        // Рекурсивно вызываем поиск дальнейших вершин и вставляем
        // в результат выоплнения функции
        auto recNextResult = getTransportTaskCycles(c, x, a, b, init, newResult, node, newMin, newMinAssigned, newSum, it, state, EPS);
        recResult.insert(recResult.end(), recNextResult.begin(), recNextResult.end());
    }

    // Возвращаем массив циклов
    return recResult;
}

template <std::size_t T, std::size_t MatrixLines, typename CountType>
std::vector<std::tuple<CountType, CountType, std::vector<std::pair<int, int>>>> getTransportTaskCycles(std::vector<std::array<CountType, T>> c, std::vector<std::array<CountType, T>> x, 
std::array<CountType, MatrixLines> a, std::array<CountType, T> b, int i, int j, CountType EPS) {
    // Инициализируем путь result, сумму sum и min
    std::pair<int, int> init = {i, j};
    std::vector<std::pair<int, int>> result = {init};
    std::pair<int, int> curr = init;
    CountType sum = c[i][j];
    bool minAssigned = false;
    CountType min;
    int it = 0;
    int state = -1;

    // Рекурсивно вызываем метод поиска циклов
    return getTransportTaskCycles(c, x, a, b, init, result, curr, min, minAssigned, sum, it, state, EPS);
}

template <std::size_t T, typename CountType>
void moveTransportTaskCycle(std::vector<std::array<CountType, T>> &x, std::vector<std::pair<int, int>> path, CountType min, CountType EPS) {
    // Для всех вершин в пути
    for (int i = 0; i < path.size(); i++) {
        // Если вершина с плюсом, добавляем сдвиг min, иначе - вычитаем
        if (i % 2 == 0) {
            x[path[i].first][path[i].second] += min;
        } else {
            x[path[i].first][path[i].second] -= min;
        }
    }
}

template <std::size_t T, std::size_t MatrixLines, typename CountType>
CountType getAnswerTransportTask(std::vector<std::array<CountType, T>> &c, std::vector<std::array<CountType, T>> &x) {
    // Считаем сумму result = cij * xij, для всех i, j 
    CountType result = c[0][0] * x[0][0];
    for (int i = 0; i < MatrixLines; i++) {
        for (int j = 0; j < T; j++) {
            if (i == 0 && j == 0) continue;

            result += c[i][j] * x[i][j];
        }
    }

    // Вернуть result
    return result;
}

template <std::size_t T, std::size_t MatrixLines, typename CountType>
CountType solveTransportTaskDistributionMethod(std::vector<std::array<CountType, T>> c, std::array<CountType, MatrixLines> a, std::array<CountType, T> b, CountType EPS) {
    // Инициализируем x
    std::vector<std::array<CountType, T>> x;
    for (int i = 0; i < MatrixLines; i++) {
        x.push_back({});
    }

    // Приводим систему к опорному решению
    getTransportTaskBasisLeastCost(c, x, a, b, EPS);

    // В бесконечном цикле
    while (true) {
        bool foundAny = false;
        std::tuple<CountType, CountType, std::vector<std::pair<int, int>>> search;

        // Ищем незаполненную клетку, для которой сумма будет отрицательна
        for (int i = 0; i < MatrixLines && !foundAny; i++) {
            for (int j = 0; j < T; j++) {
                if (abs(x[i][j]) > EPS) continue;

                auto resArray = getTransportTaskCycles(c, x, a, b, i, j, EPS);
                if (resArray.empty()) continue;
                auto res = resArray[0];

                if (std::get<0>(res) < -EPS) {
                    search = res;
                    foundAny = true;
                    break;
                }
            }
        }

        if (!foundAny) break;

        std::vector<std::pair<int, int>> path = std::get<2>(search);
        CountType minv = std::get<1>(search);

        // Если такая сумма есть, то выполняем сдвиг на min, иначе - выходим из цикла
        moveTransportTaskCycle(x, path, minv, EPS);
    }

    // Возвращаем ответ
    return getAnswerTransportTask<T, MatrixLines, CountType>(c, x);
}

template <std::size_t T, std::size_t MatrixLines, typename CountType>
void recalculatePotentials(std::vector<std::array<CountType, T>> &c, std::vector<std::array<CountType, T>> &x, std::array<CountType, T> &potentialsV, std::array<CountType, MatrixLines> &potentialsU, CountType EPS) {
    // Инициализируем просчитанные потенциалы
    std::array<bool, T> potentialsVFound = {};
    std::array<bool, MatrixLines> potentialsUFound = {};

    // Предположим, что u1 = 0
    potentialsUFound[0] = true;
    potentialsU = {0};
    
    // Пока мы находим непросчитанные потенциалы
    bool foundAny = true;
    while (foundAny) {
        foundAny = false;

        // Для всех заполненных клеток, для которых ui и vj не вычислено
        for (int i = 0; i < MatrixLines; i++) {
            for (int j = 0; j < T; j++) {
                if (abs(x[i][j]) < EPS) continue;
                if (potentialsUFound[i] && potentialsVFound[j]) continue;

                // Если вычислен ui, вычисляем vj
                if (potentialsUFound[i]) {
                    potentialsVFound[j] = true;
                    potentialsV[j] = c[i][j] - potentialsU[i];

                    foundAny = true;
                }

                // Если вычислен vj, вычисляем ui
                if (potentialsVFound[j]) {
                    potentialsUFound[i] = true;
                    potentialsU[i] = c[i][j] - potentialsV[j];

                    foundAny = true;
                }
            }
        }
    }
    
}

template <std::size_t T, std::size_t MatrixLines, typename CountType>
CountType solveTransportTaskPotentials(std::vector<std::array<CountType, T>> c, std::array<CountType, MatrixLines> a, std::array<CountType, T> b, CountType EPS) {
    // Инициализируем x
    std::vector<std::array<CountType, T>> x;
    for (int i = 0; i < MatrixLines; i++) {
        x.push_back({});
    }

    // Приводим систему к опорному решению
    getTransportTaskBasisLeastCost(c, x, a, b, EPS);

    std::array<CountType, T> potentialsV;
    std::array<CountType, MatrixLines> potentialsU;

    // В бесконечном цикле
    while (true) {
        // Пересчитываем потенциалы
        recalculatePotentials(c, x, potentialsV, potentialsU, EPS);
        int i, j;
        bool foundAny = false;
        // Для каждой пустой клетки
        for (i = 0; i < MatrixLines; i++) {
            for (j = 0; j < T; j++) {
                if (x[i][j] > EPS) continue;

                CountType t = c[i][j] - (potentialsU[i] + potentialsV[j]);
                
                // Ищем первую клетку, для которой сумма будет отрицательна
                if (t < -EPS) {
                    foundAny = true;
                    break;
                }
            }
            if (foundAny) break;
        }

        // Если такой суммы нет, выходим из цикла
        if (!foundAny) break;

        // Иначе находим цикл для найденной вершины
        auto search = getTransportTaskCycles(c, x, a, b, i, j, EPS)[0];
        // И выполняем сдвиг на min
        moveTransportTaskCycle(x, std::get<2>(search), std::get<1>(search), EPS);
    }

    // Возвращаем ответ
    return getAnswerTransportTask<T, MatrixLines, CountType>(c, x);
}
    \end{minted}
\href{https://github.com/IAmProgrammist/operations_research/blob/master/src/libs/alg/lab4/task2.tpp}{Ссылка на репозиторий}\bigbreak

Результат выполнения программы:
\begin{minted}{console}
426
\end{minted}

\textbf{Результаты вычислений: }\\
\begin{center}
    $\vec{a}=(14,14,14,14);$\\
    $\vec{b}=(13,5,13,12,13);$\\
    $C = \begin{pmatrix}
        16&26&12&24&3\\
        5&2&19&27&2\\
        29&23&25&16&8\\
        2&25&14&15&21
    \end{pmatrix}$\\
\end{center}

\begin{tabular}{|c|c|c|c|c|c|}
    \hline
    \backslashbox{a}{b} & 13 & 5 & 13 & 12 & 13\\
    \hline
    14 & \backslashbox{}{16} & \backslashbox{}{26} & \backslashbox{}{12} & \backslashbox{}{24} & \backslashbox{}{3}\\
    \hline
    14 & \backslashbox{}{5} & \backslashbox{}{2} & \backslashbox{}{27} & \backslashbox{}{27} & \backslashbox{}{2}\\
    \hline
    14 & \backslashbox{}{29} & \backslashbox{}{23} & \backslashbox{}{25} & \backslashbox{}{16} & \backslashbox{}{8}\\
    \hline
    14 & \backslashbox{}{2} & \backslashbox{}{25} & \backslashbox{}{14} & \backslashbox{}{15} & \backslashbox{}{21}\\
    \hline
\end{tabular}

Для приведения к опорному виду используем метод наименьшей стоимости

\begin{tabular}{|c|c|c|c|c|c|}
    \hline
    \backslashbox{a}{b} & 13 & 5 & 13 & 12 & 13\\
    \hline
    14 & \backslashbox{}{16} & \backslashbox{}{26} & \backslashbox{10}{12} & \backslashbox{}{24} & \backslashbox{4}{3}\\
    \hline
    14 & \backslashbox{}{5} & \backslashbox{5}{2} & \backslashbox{}{27} & \backslashbox{}{27} & \backslashbox{9}{2}\\
    \hline
    14 & \backslashbox{}{29} & \backslashbox{}{23} & \backslashbox{2}{25} & \backslashbox{12}{16} & \backslashbox{}{8}\\
    \hline
    14 & \backslashbox{13}{2} & \backslashbox{}{25} & \backslashbox{1}{14} & \backslashbox{}{15} & \backslashbox{}{21}\\
    \hline
\end{tabular}

Введём потенциалы, где $u_1 = 0$.\\


\begin{tabular}{|c|c|c|c|c|c|c|}
    \hline
    &&$v_1$&$v_2$&$v_3$&$v_4$&$v_5$\\
    \hline
    &\backslashbox{a}{b} & 13 & 5 & 13 & 12 & 13\\
    \hline
    $u_1$ & 14 & \backslashbox{}{16} & \backslashbox{}{26} & \backslashbox{10}{12} & \backslashbox{}{24} & \backslashbox{4}{3}\\
    \hline
    $u_2$ & 14 & \backslashbox{}{5} & \backslashbox{5}{2} & \backslashbox{}{27} & \backslashbox{}{27} & \backslashbox{9}{2}\\
    \hline
    $u_3$ & 14 & \backslashbox{}{29} & \backslashbox{}{23} & \backslashbox{2}{25} & \backslashbox{12}{16} & \backslashbox{}{8}\\
    \hline
    $u_4$ & 14 & \backslashbox{13}{2} & \backslashbox{}{25} & \backslashbox{1}{14} & \backslashbox{}{15} & \backslashbox{}{21}\\
    \hline
\end{tabular}

$\begin{cases}
    u_1 + v_3 = 12\\
    u_1 + v_5 = 3\\
    u_2 + v_2 = 2\\
    u_2 + v_5 = 2\\
    u_3 + v_3 = 25\\
    u_3 + v_4 = 16\\
    u_4 + v_1 = 2\\
    u_4 + v_3 = 14
\end{cases} $\\

$\begin{cases}
    u_1 = 0\\
    u_2 = -1\\
    u_3 = 13\\
    u_4 = 2
\end{cases} \begin{cases}
    v_1 = 0\\
    v_2 = 3\\
    v_3 = 12\\
    v_4 = 3\\
    v_5 = 3
\end{cases}$

Просчитаем $\gamma$:\\
$\gamma_{11} = 16$\\
$\gamma_{12} = 23$\\
$\gamma_{14} = 21$\\
$\gamma_{21} = 6$\\
$\gamma_{23} = 8$\\
$\gamma_{24} = 25$\\
$\gamma_{31} = 16$\\
$\gamma_{32} = 7$\\
$\gamma_{35} = -8$\\
$\gamma_{42} = 20$\\
$\gamma_{44} = 10$\\
$\gamma_{45} = 16$\\

$\gamma_{35} < 0$, цикл в этой точке:

$3, 4 \rightarrow 1, 5 \rightarrow 1, 3 \rightarrow 3, 3$. Минимум: 2. Выполним сдвиг, получим новое
опорное решение:\\
\begin{tabular}{|c|c|c|c|c|c|c|}
    \hline
    &&$v_1$&$v_2$&$v_3$&$v_4$&$v_5$\\
    \hline
    &\backslashbox{a}{b} & 13 & 5 & 13 & 12 & 13\\
    \hline
    $u_1$ & 14 & \backslashbox{}{16} & \backslashbox{}{26} & \backslashbox{12}{12} & \backslashbox{}{24} & \backslashbox{2}{3}\\
    \hline
    $u_2$ & 14 & \backslashbox{}{5} & \backslashbox{5}{2} & \backslashbox{}{27} & \backslashbox{}{27} & \backslashbox{9}{2}\\
    \hline
    $u_3$ & 14 & \backslashbox{}{29} & \backslashbox{}{23} & \backslashbox{}{25} & \backslashbox{12}{16} & \backslashbox{2}{8}\\
    \hline
    $u_4$ & 14 & \backslashbox{13}{2} & \backslashbox{}{25} & \backslashbox{1}{14} & \backslashbox{}{15} & \backslashbox{}{21}\\
    \hline
\end{tabular}

$\begin{cases}
    u_1 + v_3 = 12\\
    u_1 + v_5 = 3\\
    u_2 + v_2 = 2\\
    u_2 + v_5 = 2\\
    u_3 + v_4 = 16\\
    u_3 + v_5 = 8\\
    u_4 + v_1 = 2\\
    u_4 + v_3 = 14
\end{cases} $\\

$\begin{cases}
    u_1 = 0\\
    u_2 = -1\\
    u_3 = 5\\
    u_4 = 2
\end{cases} \begin{cases}
    v_1 = 0\\
    v_2 = 3\\
    v_3 = 12\\
    v_4 = 11\\
    v_5 = 3
\end{cases}$

Просчитаем $\gamma$:\\
$\gamma_{11} = 16$\\
$\gamma_{12} = 23$\\
$\gamma_{14} = 13$\\
$\gamma_{21} = 6$\\
$\gamma_{23} = 8$\\
$\gamma_{24} = 17$\\
$\gamma_{31} = 24$\\
$\gamma_{32} = 15$\\
$\gamma_{33} = 8$\\
$\gamma_{42} = 20$\\
$\gamma_{44} = 2$\\
$\gamma_{45} = 16$\\

Так как все $\gamma > 0$, мы достигли оптимального плана.\\

$Z_{min} = 12\cdot12 + 2\cdot 3 + 5\cdot2 + 9\cdot2 + 12\cdot16 + 2\cdot8 + 13\cdot2 + 1\cdot14 = 426$.

\textbf{Вывод: } в ходе лабораторной работы изучили математическую модель транспортной
задачи, овладели методами решения этой задачи.

\end{document}