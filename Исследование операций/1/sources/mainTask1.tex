% Созданная блок схема работает только с компиляторами XeTeX и LuaTeX.
\documentclass[../report1.tex]{subfiles}

% Необходимые зависимости
\usepackage[utf8]{inputenc}
\usepackage[english,russian]{babel}
\usepackage{pgfplots}
\usepackage{verbatim}
\usetikzlibrary{positioning}
\usetikzlibrary{shapes.geometric}
\usetikzlibrary{shapes.misc}
\usetikzlibrary{calc}
\usetikzlibrary{chains}
\usetikzlibrary{matrix}
\usetikzlibrary{decorations.text}
\usepackage{fontspec}
\usetikzlibrary{backgrounds}

\begin{document}
% Блок-схема
% Если Вы хотите добавить блок схему в свой документ, скопируйте код между комментариями
% С линияи и вставьте в документ.

% --------------------------
\newsavebox{\bdgBMQTDDFI}
\begin{lrbox}{\bdgBMQTDDFI}\begin{tikzpicture}[every node/.style={inner sep=0,outer sep=0}]
\makeatletter
\newcommand{\verbatimfont}[1]{\def\verbatim@font{#1}}
\makeatother
% Шрифты
\newfontfamily\fontBMQTDDFIRPASPS[Scale=0.647159, SizeFeatures={Size=10.000000}]{Consolas}

% Цвета
\definecolor{colorBMQTDDFIEDHSHEA}{rgb}{1.000000,1.000000,1.000000}
\definecolor{colorBMQTDDFITBTRTQ}{rgb}{1.000000,1.000000,1.000000}
\definecolor{colorBMQTDDFIEOTKGJPP}{rgb}{0.000000,0.000000,0.000000}
\draw[opacity=1.000000, rounded corners=0.210835 cm, fill=colorBMQTDDFIEDHSHEA, draw=none] (1.654376,0.004566) -- (3.341058,0.004566) -- (3.341058,-0.417104) -- (1.654376,-0.417104) -- cycle;
\draw[opacity=1.000000, rounded corners=0.210835 cm, fill=none, line width=0.022830cm, colorBMQTDDFIEOTKGJPP] (1.654376,0.004566) -- (3.341058,0.004566) -- (3.341058,-0.417104) -- (1.654376,-0.417104) -- cycle;
\verbatimfont{\normalsize\fontBMQTDDFIRPASPS}
\node[opacity=1.000000, above right, colorBMQTDDFIEOTKGJPP] at(2.132465, -0.280122) {\verb|Начало|};
\draw[fill=colorBMQTDDFIEDHSHEA, draw=none, opacity=1.000000] (0.990691,-0.736729) -- (4.004743,-0.736729) -- (4.004743,-1.202767) -- (0.990691,-1.202767) -- cycle;
\draw[opacity=1.000000, fill=none, line width=0.022830cm, colorBMQTDDFIEOTKGJPP] (0.990691,-0.736729) -- (4.004743,-0.736729) -- (4.004743,-1.202767) -- (0.990691,-1.202767) -- cycle;
\verbatimfont{\normalsize\fontBMQTDDFIRPASPS}
\node[opacity=1.000000, above right, colorBMQTDDFIEOTKGJPP] at(1.127673, -1.065785) {\verb|Инициализируем матрицу|};
\draw[opacity=1.000000, fill=none, line width=0.013698 cm, colorBMQTDDFIEOTKGJPP] (2.497717,-0.417104) -- (2.497717,-0.736729);
\draw[fill=colorBMQTDDFIEDHSHEA, draw=none, opacity=1.000000] (-0.004566,-1.522392) -- (5.000000,-1.522392) -- (5.000000,-2.300474) -- (-0.004566,-2.300474) -- cycle;
\draw[opacity=1.000000, fill=none, line width=0.022830cm, colorBMQTDDFIEOTKGJPP] (-0.004566,-1.522392) -- (5.000000,-1.522392) -- (5.000000,-2.300474) -- (-0.004566,-2.300474) -- cycle;
\draw[opacity=1.000000, fill=none, line width=0.022830cm, colorBMQTDDFIEOTKGJPP] (0.112146,-1.522392) -- (4.883288,-1.522392) -- (4.883288,-2.300474) -- (0.112146,-2.300474) -- cycle;
\verbatimfont{\normalsize\fontBMQTDDFIRPASPS}
\node[opacity=1.000000, above right, colorBMQTDDFIEOTKGJPP] at(0.693751, -1.866497) {\verb|Получаем все базисные решения |};
\verbatimfont{\normalsize\fontBMQTDDFIRPASPS}
\node[opacity=1.000000, above right, colorBMQTDDFIEOTKGJPP] at(0.249127, -2.163492) {\verb|для матрицы при помощи getAllBasises|};
\draw[opacity=1.000000, fill=none, line width=0.013698 cm, colorBMQTDDFIEOTKGJPP] (2.497717,-1.202767) -- (2.497717,-1.522392);
\draw[opacity=1.000000, fill=colorBMQTDDFIEDHSHEA, draw=none] (1.580613, -3.083350) -- (1.696425, -2.620099) -- (3.414821, -2.620099) -- (3.299009, -3.083350) --cycle;
\draw[opacity=1.000000, fill=none, line width=0.022830cm, colorBMQTDDFIEOTKGJPP] (1.580613, -3.083350) -- (1.696425, -2.620099) -- (3.414821, -2.620099) -- (3.299009, -3.083350) --cycle;
\verbatimfont{\normalsize\fontBMQTDDFIRPASPS}
\node[opacity=1.000000, above right, colorBMQTDDFIEOTKGJPP] at(1.696425, -2.946368) {\verb|Вывод базисов|};
\draw[opacity=1.000000, fill=none, line width=0.013698 cm, colorBMQTDDFIEOTKGJPP] (2.497717,-2.300474) -- (2.497717,-2.620099);
\draw[opacity=1.000000, rounded corners=0.224101 cm, fill=colorBMQTDDFIEDHSHEA, draw=none] (1.601314,-3.402975) -- (3.394120,-3.402975) -- (3.394120,-3.851176) -- (1.601314,-3.851176) -- cycle;
\draw[opacity=1.000000, rounded corners=0.224101 cm, fill=none, line width=0.022830cm, colorBMQTDDFIEOTKGJPP] (1.601314,-3.402975) -- (3.394120,-3.402975) -- (3.394120,-3.851176) -- (1.601314,-3.851176) -- cycle;
\verbatimfont{\normalsize\fontBMQTDDFIRPASPS}
\node[opacity=1.000000, above right, colorBMQTDDFIEOTKGJPP] at(2.195226, -3.714194) {\verb|Конец|};
\draw[opacity=1.000000, fill=none, line width=0.013698 cm, colorBMQTDDFIEOTKGJPP] (2.497717,-3.083350) -- (2.497717,-3.402975);
\end{tikzpicture}
\end{lrbox}
% Здесь Вы можете поменять размер блок схемы. Оношение ширина/высота будет сохранено.
% Для изменения размеров блок схемы Вы можете изменять первый параметр resizebox, он задаёт желаемую ширину.
\resizebox{5.000000cm}{!}{\usebox{\bdgBMQTDDFI}}
% --------------------------

% Конец блок схемы
\end{document}