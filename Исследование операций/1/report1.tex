\documentclass[a4paper,14pt]{extarticle}

\usepackage{subfiles}

\usepackage[english,russian]{babel}
\usepackage[T2A]{fontenc}
\usepackage[utf8]{inputenc}
\usepackage{ragged2e}
\usepackage{hyperref}
\usepackage{minted}
\setmintedinline{frame=lines, framesep=2mm, baselinestretch=1.5, bgcolor=LightGray, breaklines,fontsize=\scriptsize}
\setminted{frame=lines, framesep=2mm, baselinestretch=1.5, bgcolor=LightGray, breaklines,fontsize=\scriptsize}
\usepackage{xcolor}
\definecolor{LightGray}{gray}{0.9}
\usepackage{graphicx}
\usepackage[export]{adjustbox}
\usepackage[left=1cm,right=1cm, top=1cm,bottom=1cm,bindingoffset=0cm]{geometry}
\usepackage{fontspec}
\usepackage{ upgreek }
\usepackage[shortlabels]{enumitem}
\usepackage{adjustbox}
\usepackage{multirow}
\usepackage{amsmath}
\usepackage{amssymb}
\usepackage{pifont}
\usepackage{pgfplots}
\usepackage{longtable}
\usepackage{array}

\graphicspath{ {./images/} }
\makeatletter
\AddEnumerateCounter{\asbuk}{\russian@alph}{щ}
\makeatother
\setmonofont{Consolas}
\setmainfont{Times New Roman}

\newcommand\textbox[1]{
	\parbox{.45\textwidth}{#1}
} 

\newcommand{\specialcell}[2][c]{%
	\begin{tabular}[#1]{@{}c@{}}#2\end{tabular}}

\begin{document}
\pagenumbering{gobble}
\begin{center}
    \small{
        \textbf{МИНИCТЕРCТВО НАУКИ И ВЫCШЕГО ОБРАЗОВАНИЯ РОCCИЙCКОЙ ФЕДЕРАЦИИ}\\
        ФЕДЕРАЛЬНОЕ ГОCУДАРCТВЕННОЕ БЮДЖЕТНОЕ ОБРАЗОВАТЕЛЬНОЕ УЧРЕЖДЕНИЕ\\ВЫCШЕГО ОБРАЗОВАНИЯ \\
        \textbf{«БЕЛГОРОДCКИЙ ГОCУДАРCТВЕННЫЙ ТЕХНОЛОГИЧЕCКИЙ\\УНИВЕРCИТЕТ им. В. Г. ШУХОВА»\\ (БГТУ им. В.Г. Шухова)} \\
        \bigbreak
        \includegraphics[width=70mm]{log}\\
        ИНСТИТУТ ИНФОРМАЦИОННЫХ ТЕХНОЛОГИЙ И УПРАВЛЯЮЩИХ СИСТЕМ\\}
\end{center}

\vfill
\begin{center}
    \large{
        \textbf{
            Лабораторная работа №1}}\\
    \normalsize{
        по дисциплине: Исследование операций \\
        тема: «Исследование множества опорных планов системы ограничений
        задачи линейного программирования (задачи ЛП) в канонической форме»}
\end{center}
\vfill
\hfill\textbox{
    Выполнил: ст. группы ПВ-223\\Пахомов Владислав Андреевич
    \bigbreak
    Проверили: \\проф. Вирченко Юрий Петрович
}
\vfill\begin{center}
    Белгород 2024 г.
\end{center}
\newpage
\begin{center}
    \textbf{Лабораторная работа №1}\\
    Исследование множества опорных планов системы ограничений
    задачи линейного программирования (задачи ЛП) в канонической форме.\\
\end{center}
\textbf{Цель работы: }изучить метод Гаусса-Жордана и операцию
замещения, а также освоить их применение к отысканию множества
допустимых базисных видов системы линейных уравнений, и решению
задачи линейного программирования простым перебором опорных
решений.\\
\begin{center}
    \textbf{Вариант 10}
\end{center}
\begin{equation*}
    \begin{cases}
        x_1 - 4x_2 + 8x_3 + 9x_4 - 3x_5 - x_6 = 87  \\
        8x_1 + x_2 - 3x_3 + 4x_4 + 5x_5 + 6x_6 = 11 \\
        4x_1 + x_3 + 3x_4 - 2x_5 - 5x_6 = 17        \\
        -3x_1 - 4x_2 + 7x_3 + 6x_4 - x_5 + 4x_6 = 70
    \end{cases}
\end{equation*}\\
\textbf{Задание 1}\\
Составить программу для отыскания всех базисных видов системы
линейных уравнений.\bigbreak

Блок-схемы:
\begin{center}
    \subfile{sources/getAllBasises.tex}
    \subfile{sources/subtractLineFromOther.tex}
\end{center}
Исходный код:
\begin{minted}{C++}
void subtractLineFromOther(std::vector<std::vector<double>>& origin, int indexLeadingLine, int indexEnablingElement) {
    // Преобразовать ведущую строку таким образом, чтобы разрещающий элемент стал равен 1.
    double originEnablingElement = origin[indexLeadingLine][indexEnablingElement];
    for (int i = 0; i < origin[indexLeadingLine].size(); i++) 
        // Разделить каждый элемент строки по индексу indexLeadingLine
        // на разрешающий элемент
        origin[indexLeadingLine][i] /= originEnablingElement;
    
    // Для всех остальных строк с номером i
    for (int i = 0; i < origin.size(); i++) {
        if (i == indexLeadingLine) continue;

        // k - коэффициент, на который нужно домножить ведущую строку и вычесть её
        double k = origin[i][indexEnablingElement];

        // Для всех остальных элементов j в строке i
        for (int j = 0; j < origin[indexLeadingLine].size(); j++)
            // Вычесть ведущую строку indexLeadingLine из строки i
            origin[i][j] -= origin[indexLeadingLine][j] * k;
    }
}

// Вспомогательная СД
struct Basis {
    std::vector<int> indices;
    std::vector<std::vector<double>> matrix;
};

std::vector<Basis> getAllBasises(std::vector<std::vector<double>> origin) {
    // result - массив полученных систем 
    std::vector<Basis> result;

    // indices - массив неизвестных в системе, заполняем
    // его индексами 0 ... l - 1, где l - длина 
    // матрицы origin
    std::vector<int> indices;
    for (int i = 0; i < origin[0].size() - 1; i++) 
        indices.push_back(i);
    
    /*
    Для каждого сочетания в indices из l по h (h - высота матрицы) 
    получить набор базисных неизвестных basis.
    */
    for (auto basis : getCombinations(indices, origin.size())) {
        // Для каждой перестановки строк в матрице origin matrixPermutation
        for (auto matrixPermutation : getPermutations(origin)) {
            auto copyMatrixPermutation = matrixPermutation;
            bool badPermutation = false;
            
            // Для каждой строки в матрице (ввод счётчика строк i)
            for (int i = 0; i < matrixPermutation.size(); i++) {
                // Разрешающий элемент равен 0? 
                if (std::abs(matrixPermutation[i][basis[i]]) < EPS) {
                    bool allZeros = true;
                    for (int j = 0; (j < matrixPermutation[i].size() - 1) && allZeros; j++) {
                        if (std::abs(matrixPermutation[i][j]) > EPS) {
                            // Неудачное расположение строк, необходимо
                            // перейти к другой перестановке
                            badPermutation = true;
                            allZeros = false;
                            break;
                        }
                    }

                    // Коэффициенты в i строке содержат только 0?
                    if (!allZeros) {
                        // Перейти к следующей перестановке строк
                        break;
                    }

                    // b_i равен 0?
                    if (std::abs(matrixPermutation[i].back()) > EPS) {
                        // Система несовместима. Вернуть пустой массив.
                        return {};
                    }

                    /*
                    Удалить строку i из матрицы, полученной перестановкой строк,
                    и вызвать функцию getAllBasises с полученной матрицей.
                    */
                    if (allZeros) {
                        copyMatrixPermutation.erase(copyMatrixPermutation.begin() + i);
                        return getAllBasises(copyMatrixPermutation);
                    }

                    break;
                }

                // Выбрать ведущую переменную из basis[i], преобразовать i строку и вычесть её из остальных
                subtractLineFromOther(matrixPermutation, i, basis[i]);
            }

            if (badPermutation) continue;

            // Полученную матрицу matrixPermutation добавить в result, закончить перебор перестановок
            result.push_back({basis, matrixPermutation});
            break;
        }
    }

    return result;
}
    \end{minted}
\href{https://github.com/IAmProgrammist/operations_research/blob/master/src/libs/alg/lab1/task1.tpp}{Ссылка на репозиторий}\\

\begin{minted}{C++}
#include <iostream>
#include <iomanip>
#include <windows.h>

#include "libs/alg/lab1/task1.tpp"

int main() {
    SetConsoleOutputCP(CP_UTF8);
    std::vector<std::vector<double>> matrix = {
    {1,  -4,  8,  9, -3, -1, 87},
    {8,   1, -3,  4,  5,  6, 11},
    {4,   0,  1,  3, -2, -5, 17},
    {-3, -4,  7,  6, -1,  4, 70}};;

    auto res = getAllBasises(matrix);

    std::cout << "==================================================================================================================\n";
    for (auto& matrix : res) {
        std::cout << "Выбранные базисные переменные: ";
        for (auto& bas : matrix.indices) {
            std::cout << "x" << (bas + 1) << " ";
        }
        std::cout << "\n\nПолученная система: " << std::endl;


        for (int i = 0; i < matrix.matrix[0].size() - 1; i++) {
            std::stringstream buf;
            buf << "a" << (i + 1);
            std::cout << std::setw(15) << buf.str() << " ";
        }

        std::cout << std::setw(15) << "b" << std::endl;

        for (auto & line : matrix.matrix) {
            for (auto & element : line) {
                std::cout << std::setw(15) << element << " ";
            }

            std::cout << "\n";
        }
        
        std::cout << "==================================================================================================================\n";
    }
}
    \end{minted}
\href{https://github.com/IAmProgrammist/operations_research/blob/master/src/main.cpp}{Ссылка на репозиторий}\bigbreak

Результаты выполнения программы:
\begin{minted}{console}
==================================================================================================================
Выбранные базисные переменные: x4 x5 x6 

Полученная система: 
             a1              a2              a3              a4              a5              a6               b
         0.8125         -0.1875           0.375               1               0               0          7.1875 
        2.54808        0.932692        -1.78846              -0               1               0        -8.93269 
       -1.33173       -0.485577        0.740385               0               0               1         4.48558 
==================================================================================================================
Выбранные базисные переменные: x3 x5 x6 

Полученная система: 
             a1              a2              a3              a4              a5              a6               b
        2.16667            -0.5               1         2.66667               0               0         19.1667 
        6.42308       0.0384615               0         4.76923               1               0         25.3462 
        -2.9359       -0.115385              -0        -1.97436              -0               1        -9.70513 
==================================================================================================================
Выбранные базисные переменные: x3 x4 x6 

Полученная система:
             a1              a2              a3              a4              a5              a6               b
       -1.42473       -0.521505               1               0        -0.55914               0         4.99462 
        1.34677      0.00806452               0               1        0.209677               0         5.31452
      -0.276882      -0.0994624               0               0        0.413978               1        0.787634
==================================================================================================================
Выбранные базисные переменные: x3 x4 x5

Полученная система:
             a1              a2              a3              a4              a5              a6               b
        -1.7987       -0.655844               1               0               0         1.35065         6.05844
        1.48701       0.0584416               0               1               0       -0.506494         4.91558
      -0.668831        -0.24026               0               0               1         2.41558          1.9026
==================================================================================================================
Выбранные базисные переменные: x2 x5 x6

Полученная система:
             a1              a2              a3              a4              a5              a6               b
       -4.33333               1              -2        -5.33333               0               0        -38.3333
        6.58974               0       0.0769231         4.97436               1               0         26.8205
        -3.4359              -0       -0.230769        -2.58974              -0               1        -14.1282
==================================================================================================================
Выбранные базисные переменные: x2 x4 x6

Полученная система:
             a1              a2              a3              a4              a5              a6               b
        2.73196               1        -1.91753               0         1.07216               0        -9.57732
        1.32474               0       0.0154639               1        0.201031               0         5.39175
    -0.00515464               0       -0.190722               0        0.520619               1       -0.164948
==================================================================================================================
Выбранные базисные переменные: x2 x4 x5

Полученная система:
             a1              a2              a3              a4              a5              a6               b
        2.74257               1        -1.52475               0               0        -2.05941        -9.23762
        1.32673               0       0.0891089               1               0       -0.386139         5.45545
    -0.00990099               0       -0.366337               0               1         1.92079       -0.316832
==================================================================================================================
Выбранные базисные переменные: x2 x3 x6

Полученная система:
             a1              a2              a3              a4              a5              a6               b
            167               1               0             124              26               0             659
        85.6667               0               1         64.6667              13               0         348.667
        16.3333               0               0         12.3333               3               1         66.3333
==================================================================================================================
Выбранные базисные переменные: x2 x3 x5

Полученная система:
             a1              a2              a3              a4              a5              a6               b
        25.4444               1               0         17.1111               0        -8.66667         84.1111
        14.8889               0               1         11.2222               0        -4.33333         61.2222
        5.44444               0               0         4.11111               1        0.333333         22.1111
==================================================================================================================
Выбранные базисные переменные: x2 x3 x4

Полученная система:
             a1              a2              a3              a4              a5              a6               b
        2.78378               1               0               0        -4.16216        -10.0541        -7.91892
       0.027027               0               1               0        -2.72973        -5.24324        0.864865
        1.32432               0               0               1        0.243243       0.0810811         5.37838
==================================================================================================================
Выбранные базисные переменные: x1 x5 x6

Полученная система:
             a1              a2              a3              a4              a5              a6               b
              1       -0.230769        0.461538         1.23077               0               0         8.84615
             -0         1.52071         -2.9645        -3.13609               1               0        -31.4734
              0       -0.792899         1.35503         1.63905               0               1         16.2663
==================================================================================================================
Выбранные базисные переменные: x1 x4 x6 

Полученная система:
             a1              a2              a3              a4              a5              a6               b
              1        0.366038       -0.701887               0        0.392453               0        -3.50566
              0       -0.484906        0.945283               1       -0.318868               0         10.0358
              0      0.00188679        -0.19434               0        0.522642               1       -0.183019
==================================================================================================================
Выбранные базисные переменные: x1 x4 x5

Полученная система:
             a1              a2              a3              a4              a5              a6               b
              1        0.364621       -0.555957               0               0       -0.750903        -3.36823
              0       -0.483755        0.826715               1               0        0.610108         9.92419
              0      0.00361011       -0.371841               0               1         1.91336       -0.350181
==================================================================================================================
Выбранные базисные переменные: x1 x3 x6

Полученная система:
             a1              a2              a3              a4              a5              a6               b
              1      0.00598802               0        0.742515        0.155689               0         3.94611
              0       -0.512974               1         1.05788       -0.337325               0         10.6168
              0      -0.0978044               0        0.205589        0.457086               1         1.88024
==================================================================================================================
Выбранные базисные переменные: x1 x3 x5

Полученная система:
             a1              a2              a3              a4              a5              a6               b
              1       0.0393013               0        0.672489               0       -0.340611         3.30568
              0       -0.585153               1         1.20961               0        0.737991         12.0044
              0       -0.213974               0        0.449782               1         2.18777         4.11354
==================================================================================================================
Выбранные базисные переменные: x1 x3 x4

Полученная система:
             a1              a2              a3              a4              a5              a6               b
              1        0.359223               0               0        -1.49515        -3.61165        -2.84466
              0     -0.00970874               1               0        -2.68932        -5.14563        0.941748
              0       -0.475728               0               1          2.2233         4.86408         9.14563
==================================================================================================================
Выбранные базисные переменные: x1 x2 x6 

Полученная система:
             a1              a2              a3              a4              a5              a6               b
              1               0       0.0116732        0.754864        0.151751               0         4.07004
             -0               1        -1.94942        -2.06226        0.657588               0        -20.6965
              0               0       -0.190661      0.00389105        0.521401               1       -0.143969
==================================================================================================================
Выбранные базисные переменные: x1 x2 x5

Полученная система:
             a1              a2              a3              a4              a5              a6               b
              1               0       0.0671642        0.753731               0       -0.291045         4.11194
             -0               1        -1.70896        -2.06716               0        -1.26119        -20.5149
              0               0       -0.365672      0.00746269               1         1.91791       -0.276119
==================================================================================================================
Выбранные базисные переменные: x1 x2 x4

Полученная система:
             a1              a2              a3              a4              a5              a6               b
              1               0              37               0            -101            -194              32 
              0               1            -103               0             277             530             -97
              0               0             -49               1             134             257             -37
==================================================================================================================
Выбранные базисные переменные: x1 x2 x3

Полученная система:
             a1              a2              a3              a4              a5              a6               b
              1               0               0        0.755102        0.183673       0.0612245         4.06122
             -0               1               0        -2.10204        -4.67347        -10.2245        -19.2245
             -0              -0               1      -0.0204082        -2.73469         -5.2449        0.755102
==================================================================================================================
\end{minted}


\textbf{Задание 2}\\
Организовать отбор опорных планов среди всех базисных решений, а также нахождение
оптимального опорного плана методом прямого перебора. Целевая функция выбирается
произвольно.\bigbreak

\textbf{Задание 3}\\
Организовать отбор опорных планов среди всех базисных решений, а также нахождение
оптимального опорного плана методом прямого перебора. Целевая функция выбирается
произвольно.\bigbreak
Решение:
Построим расширенную матрицу:\bigbreak
$\begin{pmatrix}
        1  & -4 & 8  & 9 & -3 & -1 & 87 \\
        8  & 1  & -3 & 4 & 5  & 6  & 11 \\
        4  & 0  & 1  & 3 & -2 & -5 & 17 \\
        -3 & -4 & 7  & 6 & -1 & 4  & 70
    \end{pmatrix}$\bigbreak
В качестве базисных переменных выберем $x_2$, $x_3$ и $x_5$. Выберем ведущий элемент $x_2$ в первый строке:\bigbreak

$\begin{pmatrix}
        1  & \|-4\| & 8  & 9 & -3 & -1 & 87 \\
        8  & 1      & -3 & 4 & 5  & 6  & 11 \\
        4  & 0      & 1  & 3 & -2 & -5 & 17 \\
        -3 & -4     & 7  & 6 & -1 & 4  & 70
    \end{pmatrix}$\bigbreak
Разделим ведущую строку на -4. Вычтем 1 строку из 2, домножив её на -1 и сложив; 3 строка останется неизменной; из 4, домножив на 4 и сложив:\bigbreak
$\begin{pmatrix}
        -\frac{1}{4} & \|1\| & -2 & -2\frac{1}{4} & \frac{3}{4}  & \frac{1}{4}  & -21\frac{3}{4} \\
        8\frac{1}{4} & 0     & -1 & 6\frac{1}{4}  & 4\frac{1}{4} & 5\frac{3}{4} & 32\frac{3}{4}  \\
        4            & 0     & 1  & 3             & -2           & -5           & 17             \\
        -4           & 0     & -1 & -3            & 2            & 5            & -17
    \end{pmatrix}$\bigbreak
Во второй строке выберем ведущим элементом $x_3$:\bigbreak
$\begin{pmatrix}
        -\frac{1}{4} & 1 & -2     & -2\frac{1}{4} & \frac{3}{4}  & \frac{1}{4}  & -21\frac{3}{4} \\
        8\frac{1}{4} & 0 & \|-1\| & 6\frac{1}{4}  & 4\frac{1}{4} & 5\frac{3}{4} & 32\frac{3}{4}  \\
        4            & 0 & 1      & 3             & -2           & -5           & 17             \\
        -4           & 0 & -1     & -3            & 2            & 5            & -17
    \end{pmatrix}$\bigbreak
Разделим ведущую строку на -1. Вычтем 2 строку из 1, домножив её на 2 и сложив; из 3, домножив на -1 сложив; из 4, сложив:\bigbreak
$\begin{pmatrix}
        -16\frac{3}{4} & 1 & 0     & -14\frac{3}{4} & -7\frac{3}{4} & -11\frac{1}{4} & -87\frac{1}{4} \\
        -8\frac{1}{4}  & 0 & \|1\| & -6\frac{1}{4}  & -4\frac{1}{4} & -5\frac{3}{4}  & -32\frac{3}{4} \\
        12\frac{1}{4}  & 0 & 0     & 9\frac{1}{4}   & 2\frac{1}{4}  & \frac{3}{4}    & 49\frac{3}{4}  \\
        -12\frac{1}{4} & 0 & 0     & -9\frac{1}{4}  & -2\frac{1}{4} & -\frac{3}{4}   & -49\frac{3}{4} \\
    \end{pmatrix}$\bigbreak

В третьей строке выберем ведущим элементом $x_5$:\bigbreak
$\begin{pmatrix}
        -16\frac{3}{4} & 1 & 0 & -14\frac{3}{4} & -7\frac{3}{4}    & -11\frac{1}{4} & -87\frac{1}{4} \\
        -8\frac{1}{4}  & 0 & 1 & -6\frac{1}{4}  & -4\frac{1}{4}    & -5\frac{3}{4}  & -32\frac{3}{4} \\
        12\frac{1}{4}  & 0 & 0 & 9\frac{1}{4}   & \|2\frac{1}{4}\| & \frac{3}{4}    & 49\frac{3}{4}  \\
        -12\frac{1}{4} & 0 & 0 & -9\frac{1}{4}  & -2\frac{1}{4}    & -\frac{3}{4}   & -49\frac{3}{4} \\
    \end{pmatrix}$\bigbreak
Разделим ведущую строку на $2\frac{1}{4}$. Вычтем 3 строку из 1, домножив её на $7\frac{3}{4}$ и сложив; из 2, домножив на $4\frac{1}{4}$ и сложив; из 4, домножив на $2\frac{1}{4}$ и сложив:\bigbreak
$\begin{pmatrix}
        25\frac{4}{9} & 1 & 0 & 17\frac{1}{9} & 0     & -8\frac{2}{3} & 84\frac{1}{9} \\
        14\frac{8}{9} & 0 & 1 & 11\frac{2}{9} & 0     & -4\frac{1}{3} & 61\frac{2}{9} \\
        5\frac{4}{9}  & 0 & 0 & 4\frac{1}{9}  & \|1\| & \frac{1}{3}   & 22\frac{1}{9} \\
        0             & 0 & 0 & 0             & 0     & 0             & 0             \\
    \end{pmatrix}$\bigbreak
Так как в 4 строке только нули и $b_4$ тоже равен 0, строчку можем убрать:\bigbreak
$\begin{pmatrix}
        25\frac{4}{9} & 1 & 0 & 17\frac{1}{9} & 0     & -8\frac{2}{3} & 84\frac{1}{9} \\
        14\frac{8}{9} & 0 & 1 & 11\frac{2}{9} & 0     & -4\frac{1}{3} & 61\frac{2}{9} \\
        5\frac{4}{9}  & 0 & 0 & 4\frac{1}{9}  & \|1\| & \frac{1}{3}   & 22\frac{1}{9}
    \end{pmatrix}$\bigbreak
Получили систему уравнений со свободными переменными $x_1$, $x_4$, $x_6$ и базисными $x_2$, $x_3$ и $x_5$:\bigbreak
\begin{equation*}
    \begin{cases}
        x_2 = 84\frac{1}{9} - (25\frac{4}{9}x_1 + 17\frac{1}{9}x_4 - 8\frac{2}{3}x_6) \\
        x_3 = 61\frac{2}{9} - (-14\frac{8}{9}x_1 + 11\frac{2}{9}x_4 -4\frac{1}{3}x_6) \\
        x_5 = 22\frac{1}{9} - (5\frac{4}{9}x_1 + 4\frac{1}{9}x_4 + \frac{1}{3}x_6)    \\
    \end{cases}
\end{equation*}\\

Введём произвольную функцию $z = -3x_1 + x_2 + 121x_3 - 7x_6$. Решим задачу:
\begin{center}
    $z = -3x_1 + x_2 + 121x_3 - 7x_6 \rightarrow max$\\
    \begin{equation*}
        \begin{cases}
            x_2 = 84\frac{1}{9} - (25\frac{4}{9}x_1 + 17\frac{1}{9}x_4 - 8\frac{2}{3}x_6) \\
            x_3 = 61\frac{2}{9} - (14\frac{8}{9}x_1 + 11\frac{2}{9}x_4 -4\frac{1}{3}x_6)  \\
            x_5 = 22\frac{1}{9} - (5\frac{4}{9}x_1 + 4\frac{1}{9}x_4 + \frac{1}{3}x_6)    \\
        \end{cases}
    \end{equation*}\\
    $x_1, x_2, x_3, \dotsb, x_6 \geq 0$
\end{center}
Оформим эту систему в виде таблицы:\\
\begin{tabular}{|c|c|c|c|c|c|c|c|}
    \hline
    Баз. пер.        & Св. чл.         & $\downarrow x_1$ & $x_2$ & $x_3$ & $x_4$           & $x_5$ & $x_6$            \\
    \hline
    $\leftarrow x_2$ & $84\frac{1}{9}$ & $25\frac{4}{9}$  & $1$   & $0$   & $17\frac{1}{9}$ & $0$   & $- 8\frac{2}{3}$ \\
    \hline
    $x_3$            & $61\frac{2}{9}$ & $14\frac{8}{9}$  & $0$   & $1$   & $11\frac{2}{9}$ & $0$   & $-4\frac{1}{3}$  \\
    \hline
    $x_5$            & $22\frac{1}{9}$ & $5\frac{4}{9}$   & $0$   & $0$   & $4\frac{1}{9}$  & $1$   & $\frac{1}{3}$    \\
    \hline
    $z$              & $0$             & $-3$             & $1$   & $121$ & $0$             & $0$   & $-7$             \\
    \hline
\end{tabular}\bigbreak
\begin{tabular}{|c|c|c|c|c|c|c|c|}
    \hline
    Баз. пер.        & Св. чл.            & $x_1$ & $x_2$              & $x_3$ & $x_4$             & $x_5$ & $\downarrow x_6$   \\
    \hline
    $x_1$            & $ 3\frac{70}{229}$ & $1$   & $\frac{9}{229}$    & $0$   & $\frac{154}{229}$ & $0$   & $- \frac{78}{229}$ \\
    \hline
    $\leftarrow x_3$ & $12\frac{1}{229}$  & $0$   & $-\frac{134}{229}$ & $1$   & $1\frac{48}{229}$ & $0$   & $\frac{169}{229}$  \\
    \hline
    $x_5$            & $4\frac{26}{229}$  & $0$   & $-\frac{49}{229}$  & $0$   & $\frac{103}{229}$ & $1$   & $2\frac{43}{229}$  \\
    \hline
    $z$              & $9\frac{210}{229}$ & $0$   & $1\frac{27}{229}$  & $121$ & $2\frac{4}{229}$  & $0$   & $-8\frac{5}{229}$  \\
    \hline
\end{tabular}\bigbreak
\begin{tabular}{|c|c|c|c|c|c|c|c|}
    \hline
    Баз. пер. & Св. чл.            & $x_1$ & $x_2$              & $x_3$ & $x_4$             & $x_5$ & $x_6$              \\
    \hline
    $x_1$     & $ 3\frac{70}{229}$ & $1$   & $\frac{9}{229}$    & $0$   & $\frac{154}{229}$ & $0$   & $- \frac{78}{229}$ \\
    \hline
    $x_3$     & $12\frac{1}{229}$  & $0$   & $-\frac{134}{229}$ & $1$   & $1\frac{48}{229}$ & $0$   & $\frac{169}{229}$  \\
    \hline
    $x_5$     & $4\frac{26}{229}$  & $0$   & $-\frac{49}{229}$  & $0$   & $\frac{103}{229}$ & $1$   & $2\frac{43}{229}$  \\
    \hline
    $z$       & $9\frac{210}{229}$ & $0$   & $1\frac{27}{229}$  & $121$ & $2\frac{4}{229}$  & $0$   & $-8\frac{5}{229}$  \\
    \hline
\end{tabular}\bigbreak

\textbf{Вывод: } в ходе лабораторной работы получили навыки модульной декомпозиции
предметной области, создания модулей. Разработали интерфейс. Реализовали программу.

\end{document}