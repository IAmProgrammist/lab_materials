% Созданная блок схема работает только с компиляторами XeTeX и LuaTeX.
\documentclass[../report5.tex]{subfiles}

% Необходимые зависимости
\usepackage[utf8]{inputenc}
\usepackage[english,russian]{babel}
\usepackage{pgfplots}
\usepackage{verbatim}
\usetikzlibrary{positioning}
\usetikzlibrary{shapes.geometric}
\usetikzlibrary{shapes.misc}
\usetikzlibrary{calc}
\usetikzlibrary{chains}
\usetikzlibrary{matrix}
\usetikzlibrary{decorations.text}
\usepackage{fontspec}
\usetikzlibrary{backgrounds}

\begin{document}
% Блок-схема
% Если Вы хотите добавить блок схему в свой документ, скопируйте код между комментариями
% С линияи и вставьте в документ.

% --------------------------
\newsavebox{\bdgJNDMDPH}
\begin{lrbox}{\bdgJNDMDPH}\begin{tikzpicture}[every node/.style={inner sep=0,outer sep=0}]
\makeatletter
\newcommand{\verbatimfont}[1]{\def\verbatim@font{#1}}
\makeatother
% Шрифты
\newfontfamily\fontJNDMDPHJMHPFEAK[Scale=0.746754, SizeFeatures={Size=10.000000}]{Consolas}

% Цвета
\definecolor{colorJNDMDPHEPSMMA}{rgb}{1.000000,1.000000,1.000000}
\definecolor{colorJNDMDPHJGPECFR}{rgb}{1.000000,1.000000,1.000000}
\definecolor{colorJNDMDPHMDPGJS}{rgb}{0.000000,0.000000,0.000000}
\draw[opacity=1.000000, rounded corners=0.243282 cm, fill=colorJNDMDPHEPSMMA, draw=none] (1.524239,0.005269) -- (3.470492,0.005269) -- (3.470492,-0.481294) -- (1.524239,-0.481294) -- cycle;
\draw[opacity=1.000000, rounded corners=0.243282 cm, fill=none, line width=0.026344cm, colorJNDMDPHMDPGJS] (1.524239,0.005269) -- (3.470492,0.005269) -- (3.470492,-0.481294) -- (1.524239,-0.481294) -- cycle;
\verbatimfont{\normalsize\fontJNDMDPHJMHPFEAK}
\node[opacity=1.000000, above right, colorJNDMDPHMDPGJS] at(2.075903, -0.323231) {\verb|Начало|};
\draw[fill=colorJNDMDPHEPSMMA, draw=none, opacity=1.000000] (0.471503,-0.850108) -- (4.523228,-0.850108) -- (4.523228,-1.367285) -- (0.471503,-1.367285) -- cycle;
\draw[opacity=1.000000, fill=none, line width=0.026344cm, colorJNDMDPHMDPGJS] (0.471503,-0.850108) -- (4.523228,-0.850108) -- (4.523228,-1.367285) -- (0.471503,-1.367285) -- cycle;
\verbatimfont{\normalsize\fontJNDMDPHJMHPFEAK}
\node[opacity=1.000000, above right, colorJNDMDPHMDPGJS] at(0.629566, -1.209222) {\verb|Подготовить входные данные|};
\draw[opacity=1.000000, fill=none, line width=0.015806 cm, colorJNDMDPHMDPGJS] (2.497366,-0.481294) -- (2.497366,-0.850108);
\draw[fill=colorJNDMDPHEPSMMA, draw=none, opacity=1.000000] (-0.005269,-1.736099) -- (5.000000,-1.736099) -- (5.000000,-3.139885) -- (-0.005269,-3.139885) -- cycle;
\draw[opacity=1.000000, fill=none, line width=0.026344cm, colorJNDMDPHMDPGJS] (-0.005269,-1.736099) -- (5.000000,-1.736099) -- (5.000000,-3.139885) -- (-0.005269,-3.139885) -- cycle;
\draw[opacity=1.000000, fill=none, line width=0.026344cm, colorJNDMDPHMDPGJS] (0.205299,-1.736099) -- (4.789432,-1.736099) -- (4.789432,-3.139885) -- (0.205299,-3.139885) -- cycle;
\verbatimfont{\normalsize\fontJNDMDPHJMHPFEAK}
\node[opacity=1.000000, above right, colorJNDMDPHMDPGJS] at(0.627765, -2.126600) {\verb|Решить задачу двойственным |};
\verbatimfont{\normalsize\fontJNDMDPHJMHPFEAK}
\node[opacity=1.000000, above right, colorJNDMDPHMDPGJS] at(1.351963, -2.396366) {\verb|симплекс методом|};
\verbatimfont{\normalsize\fontJNDMDPHJMHPFEAK}
\node[opacity=1.000000, above right, colorJNDMDPHMDPGJS] at(2.515023, -2.630888) {\verb||};
\verbatimfont{\normalsize\fontJNDMDPHJMHPFEAK}
\node[opacity=1.000000, above right, colorJNDMDPHMDPGJS] at(0.363363, -2.981822) {\verb|(вызов solveDualSimplexMethod)|};
\draw[opacity=1.000000, fill=none, line width=0.015806 cm, colorJNDMDPHMDPGJS] (2.497366,-1.367285) -- (2.497366,-1.736099);
\draw[opacity=1.000000, fill=colorJNDMDPHEPSMMA, draw=none] (1.450475, -3.995261) -- (1.572116, -3.508698) -- (3.544256, -3.508698) -- (3.422615, -3.995261) --cycle;
\draw[opacity=1.000000, fill=none, line width=0.026344cm, colorJNDMDPHMDPGJS] (1.450475, -3.995261) -- (1.572116, -3.508698) -- (3.544256, -3.508698) -- (3.422615, -3.995261) --cycle;
\verbatimfont{\normalsize\fontJNDMDPHJMHPFEAK}
\node[opacity=1.000000, above right, colorJNDMDPHMDPGJS] at(1.572116, -3.837198) {\verb|Вывести ответ|};
\draw[opacity=1.000000, fill=none, line width=0.015806 cm, colorJNDMDPHMDPGJS] (2.497366,-3.139885) -- (2.497366,-3.508698);
\draw[opacity=1.000000, rounded corners=0.258589 cm, fill=colorJNDMDPHEPSMMA, draw=none] (1.463011,-4.364075) -- (3.531721,-4.364075) -- (3.531721,-4.881252) -- (1.463011,-4.881252) -- cycle;
\draw[opacity=1.000000, rounded corners=0.258589 cm, fill=none, line width=0.026344cm, colorJNDMDPHMDPGJS] (1.463011,-4.364075) -- (3.531721,-4.364075) -- (3.531721,-4.881252) -- (1.463011,-4.881252) -- cycle;
\verbatimfont{\normalsize\fontJNDMDPHJMHPFEAK}
\node[opacity=1.000000, above right, colorJNDMDPHMDPGJS] at(2.148323, -4.723189) {\verb|Конец|};
\draw[opacity=1.000000, fill=none, line width=0.015806 cm, colorJNDMDPHMDPGJS] (2.497366,-3.995261) -- (2.497366,-4.364075);
\end{tikzpicture}
\end{lrbox}
% Здесь Вы можете поменять размер блок схемы. Оношение ширина/высота будет сохранено.
% Для изменения размеров блок схемы Вы можете изменять первый параметр resizebox, он задаёт желаемую ширину.
\resizebox{5.000000cm}{!}{\usebox{\bdgJNDMDPH}}
% --------------------------

% Конец блок схемы
\end{document}