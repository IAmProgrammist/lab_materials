\documentclass[a4paper,14pt]{extarticle}

\usepackage{subfiles}

\usepackage[english,russian]{babel}
\usepackage[T2A]{fontenc}
\usepackage[utf8]{inputenc}
\usepackage{ragged2e}
\usepackage{hyperref}
\usepackage{minted}
\setmintedinline{frame=lines, framesep=2mm, baselinestretch=1.5, bgcolor=LightGray, breaklines,fontsize=\scriptsize}
\setminted{frame=lines, framesep=2mm, baselinestretch=1.5, bgcolor=LightGray, breaklines,fontsize=\scriptsize}
\usepackage{xcolor}
\definecolor{LightGray}{gray}{0.9}
\usepackage{graphicx}
\usepackage[export]{adjustbox}
\usepackage[left=1cm,right=1cm, top=1cm,bottom=1cm,bindingoffset=0cm]{geometry}
\usepackage{fontspec}
\usepackage{ upgreek }
\usepackage[shortlabels]{enumitem}
\usepackage{adjustbox}
\usepackage{multirow}
\usepackage{amsmath}
\usepackage{amssymb}
\usepackage{pifont}
\usepackage{pgfplots}
\usepackage{longtable}
\usepackage{slashbox}
\usepackage{array}

\graphicspath{ {./images/} }
\makeatletter
\AddEnumerateCounter{\asbuk}{\russian@alph}{щ}
\makeatother
\setmonofont{Consolas}
\setmainfont{Times New Roman}

\newcommand\textbox[1]{
	\parbox{.45\textwidth}{#1}
} 

\newcommand{\specialcell}[2][c]{%
	\begin{tabular}[#1]{@{}c@{}}#2\end{tabular}}

\begin{document}
\pagenumbering{gobble}
\begin{center}
    \small{
        \textbf{МИНИCТЕРCТВО НАУКИ И ВЫCШЕГО ОБРАЗОВАНИЯ РОCCИЙCКОЙ ФЕДЕРАЦИИ}\\
        ФЕДЕРАЛЬНОЕ ГОCУДАРCТВЕННОЕ БЮДЖЕТНОЕ ОБРАЗОВАТЕЛЬНОЕ УЧРЕЖДЕНИЕ\\ВЫCШЕГО ОБРАЗОВАНИЯ \\
        \textbf{«БЕЛГОРОДCКИЙ ГОCУДАРCТВЕННЫЙ ТЕХНОЛОГИЧЕCКИЙ\\УНИВЕРCИТЕТ им. В. Г. ШУХОВА»\\ (БГТУ им. В.Г. Шухова)} \\
        \bigbreak
        \includegraphics[width=70mm]{log}\\
        ИНСТИТУТ ИНФОРМАЦИОННЫХ ТЕХНОЛОГИЙ И УПРАВЛЯЮЩИХ СИСТЕМ\\}
\end{center}

\vfill
\begin{center}
    \large{
        \textbf{
            Лабораторная работа №6}}\\
    \normalsize{
        по дисциплине: Исследование операций \\
        тема: «Нахождение седловой точки в смешанных стратегиях для
        матричной игры с нулевой суммой»}
\end{center}
\vfill
\hfill\textbox{
    Выполнил: ст. группы ПВ-223\\Пахомов Владислав Андреевич
    \bigbreak
    Проверили: \\проф. Вирченко Юрий Петрович
}
\vfill\begin{center}
    Белгород 2024 г.
\end{center}
\newpage
\begin{center}
    \textbf{Лабораторная работа №6}\\
    Нахождение седловой точки в смешанных стратегиях для
матричной игры с нулевой суммой\\
\end{center}
\textbf{Цель работы: }освоить метод нахождения седловой точки в
смешанных стратегиях с помощью построения пары двойственных
задач ЛП.\\

\textbf{Задание: }cоставить и отладить программу для нахождения седловой точки
игры с помощью решения пары симметрично двойственных задач
ЛП. Для подготовки тестовых данных решить вручную одну из
следующих ниже задач.

\begin{center}
    \textbf{Вариант 10}
\end{center}
$\begin{pmatrix}
12&6&5&10\\
5&9&7&8\\
9&4&6&8
\end{pmatrix}$\\

\textbf{Блок-схемы: }
\begin{center}
    \subfile{sources/main.tex}\bigbreak
\end{center}

\textbf{Листинг программы: }
\begin{minted}{C++}
#include <vector>
#include <array>

#include "../libs/alg/alg.h"

int main() {
    // Подготовить входные данные
    std::vector<std::array<Fraction, 5>> matrix;
    matrix.push_back({{{12}, {6}, {5}, {10}, {1}}});
    matrix.push_back({{{5}, {9}, {7}, {8}, {1}}});
    matrix.push_back({{{9}, {4}, {6}, {8}, {1}}});
    std::array<Fraction, 5> function{{{1}, {1}, {1}, {1}, {0}}};

    // Решить задачу двойственным симплекс методом
    auto ans = solveDualSimplexMethod<5, 3, Fraction>(matrix, function, MAX, Fraction());

    // Вывести ответ
    std::cout << "A = " << std::get<0>(ans) << std::endl;
    std::cout << "Q = ";
    for (int i = 0; i < 4; i++) {
        auto r = (std::get<1>(ans))[i] / std::get<0>(ans);
        std::cout << r << ", ";
    }
    std::cout << std::endl;

    std::cout << "P = ";
    for (int i = 0; i < 3; i++) {
        auto r = (std::get<1>(ans))[4 + i] / std::get<0>(ans);
        std::cout << r << ", ";
    }
}
\end{minted}

Результат выполнения программы:
\begin{minted}{console}
A = 5/33
Q = 0, 2/5, 0, 1_2/5,
P = 0, 3/5, 2/5,
\end{minted}

\textbf{Результаты вычислений: }\\
$
\begin{pmatrix}
12&6&5&10\\
5&9&7&8\\
9&4&6&8
\end{pmatrix}
$\\
Подготовим задачу к двойственному симплекс-методу
\begin{equation*}
    \begin{aligned}
        z = y_1 + y_2 + y_3 + y_4 \rightarrow max; \\
        \begin{cases}
            12y_1 + 6y_2 + 5y_3 + 10y_4 \leq 1, \\
            5y_1 + 9y_2 + 7y_3 + 8y_4 \leq 1,    \\
            9y_1 + 4y_2 + 6y_3 + 8y_4 \leq 1,  \\
        \end{cases}                \\
        y_i \geq 0 (i=\overline{1, 4}).
    \end{aligned}
\end{equation*}\\

Решим задачу двойственным симплекс-методом
\begin{center}
    \begin{longtable}{|c|c|c|c|c|c|c|c|c|}
        \hline
        Баз. пер. & Св. чл.    & $y_1$    & $y_2$    & $y_3$    & $y_4 $     & $y_5$     & $y_6$ & $y_7$     \\
        \hline
        $y_5$     & $1$        & $12$      & $6$      & $5$      & $10$        & $1$      & $0$   & $0$       \\
        \hline
        $y_6$     & $1$        & $5$     & $9$      & $7$      & $8$        & $0$      & $1$   & $0$       \\
        \hline
        $y_7$     & $1$        & $9$     & $4$      & $6$      & $8$        & $0$      & $0$   & $1$       \\
        \hline
        $z$      & $0$        & $-1$     & $-1$     & $-1$     & $-1$       & $0$       & $0$   & $0$       \\
        \hline
        \multicolumn{9}{c}{}                                                                                 \\
        \hline
        Баз. пер. & Св. чл.    & $y_1$    & $y_2$    & $y_3$    & $y_4 $     & $y_5$     & $y_6$ & $y_7$     \\
        \hline
        $y_4$     & $1/10$        & $6/5$      & $3/5$      & $1/2$      & $1$        & $1/10$      & $0$   & $0$       \\
        \hline
        $y_6$     & $1/5$        & $-23/5$     & $21/5$      & $3$      & $0$        & $-4/5$      & $1$   & $0$       \\
        \hline
        $y_7$     & $1/5$        & $-3/5$     & $-4/5$      & $2$      & $0$        & $-4/5$      & $0$   & $1$       \\
        \hline
        $z$      & $1/10$        & $1/5$     & $-2/5$     & $-1/2$     & $0$       & $1/10$       & $0$   & $0$       \\
        \hline
        \multicolumn{9}{c}{}                                                                                 \\
        \hline
        Баз. пер. & Св. чл.    & $y_1$    & $y_2$    & $y_3$    & $y_4 $     & $y_5$     & $y_6$ & $y_7$     \\
        \hline
        $y_4$     & $1/15$        & $59/30$      & $-1/10$      & $0$      & $1$        & $7/30$      & $-1/6$   & $0$       \\
        \hline
        $y_3$     & $1/15$        & $-23/15$     & $7/5$      & $1$      & $0$        & $-4/15$      & $1/3$   & $0$       \\
        \hline
        $y_7$     & $1/15$        & $37/15$     & $-18/5$      & $0$      & $0$        & $-4/15$      & $-2/3$   & $1$       \\
        \hline
        $z$      & $2/15$        & $-17/30$     & $3/10$     & $0$     & $0$       & $-1/30$       & $1/6$   & $0$       \\
        \hline
        \multicolumn{9}{c}{}                                                                                 \\
        \hline
        Баз. пер. & Св. чл.    & $y_1$    & $y_2$    & $y_3$    & $y_4 $     & $y_5$     & $y_6$ & $y_7$     \\
        \hline
        $y_4$     & $1/74$        & $0$      & $205/74$      & $0$      & $1$        & $33/74$      & $27/74$   & $-59/74$       \\
        \hline
        $y_3$     & $4/37$        & $0$     & $-31/37$      & $1$      & $0$        & $-16/37$      & $-3/37$   & $23/37$       \\
        \hline
        $y_1$     & $1/37$        & $1$     & $-54/37$      & $0$      & $0$        & $-4/37$      & $-10/37$   & $15/37$       \\
        \hline
        $z$      & $11/74$        & $0$     & $-39/74$     & $0$     & $0$       & $-7/74$       & $1/74$   & $17/74$ \\
        \hline
        \multicolumn{9}{c}{}                                                                                 \\
        \hline
        Баз. пер. & Св. чл.    & $y_1$    & $y_2$    & $y_3$    & $y_4 $     & $y_5$     & $y_6$ & $y_7$     \\
        \hline
        $y_2$     & $1/205$        & $0$      & $1$      & $0$      & $74/205$        & $33/205$      & $27/205$   & $-59/205$       \\
        \hline
        $y_3$     & $23/205$        & $0$     & $0$      & $1$      & $62/205$        & $-61/205$      & $6/205$   & $78/205$       \\
        \hline
        $y_1$     & $7/205$        & $1$     & $0$      & $0$      & $108/205$        & $26/205$      & $-16/205$   & $-3/205$       \\
        \hline
        $z$      & $31/205$        & $0$     & $0$     & $0$     & $39/205$       & $-2/205$       & $17/205$   & $16/205$ \\
        \hline
        \multicolumn{9}{c}{}                                                                                 \\
        \hline
        Баз. пер. & Св. чл.    & $y_1$    & $y_2$    & $y_3$    & $y_4 $     & $y_5$     & $y_6$ & $y_7$     \\
        \hline
        $y_5$     & $1/33$        & $0$      & $205/33$      & $0$      & $74/33$        & $1$      & $9/11$   & $-59/33$       \\
        \hline
        $y_3$     & $4/33$        & $0$     & $61/33$      & $1$      & $32/33$        & $0$      & $3/11$   & $-5/33$       \\
        \hline
        $y_1$     & $1/33$        & $1$     & $-26/33$      & $0$      & $8/33$        & $0$      & $-2/11$   & $7/33$       \\
        \hline
        $z$      & $5/33$        & $0$     & $2/33$     & $0$     & $7/33$       & $0$       & $1/11$   & $2/33$ \\
        \hline
    \end{longtable}
\end{center}

$A = 5/33; Q = [0, 2/5, 0, 7/5]; P = [0, 3/5, 2/5]$

\textbf{Вывод: } в ходе лабораторной работы освоили метод нахождения седловой точки в
смешанных стратегиях с помощью построения пары двойственных
задач ЛП. Написали и отладили программу для нахождения седловой точки
игры с помощью решения пары симметрично двойственных задач
ЛП.

\end{document}