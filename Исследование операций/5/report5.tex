\documentclass[a4paper,14pt]{extarticle}

\usepackage{subfiles}

\usepackage[english,russian]{babel}
\usepackage[T2A]{fontenc}
\usepackage[utf8]{inputenc}
\usepackage{ragged2e}
\usepackage{hyperref}
\usepackage{minted}
\setmintedinline{frame=lines, framesep=2mm, baselinestretch=1.5, bgcolor=LightGray, breaklines,fontsize=\scriptsize}
\setminted{frame=lines, framesep=2mm, baselinestretch=1.5, bgcolor=LightGray, breaklines,fontsize=\scriptsize}
\usepackage{xcolor}
\definecolor{LightGray}{gray}{0.9}
\usepackage{graphicx}
\usepackage[export]{adjustbox}
\usepackage[left=1cm,right=1cm, top=1cm,bottom=1cm,bindingoffset=0cm]{geometry}
\usepackage{fontspec}
\usepackage{ upgreek }
\usepackage[shortlabels]{enumitem}
\usepackage{adjustbox}
\usepackage{multirow}
\usepackage{amsmath}
\usepackage{amssymb}
\usepackage{pifont}
\usepackage{pgfplots}
\usepackage{longtable}
\usepackage{slashbox}
\usepackage{array}

\graphicspath{ {./images/} }
\makeatletter
\AddEnumerateCounter{\asbuk}{\russian@alph}{щ}
\makeatother
\setmonofont{Consolas}
\setmainfont{Times New Roman}

\newcommand\textbox[1]{
	\parbox{.45\textwidth}{#1}
} 

\newcommand{\specialcell}[2][c]{%
	\begin{tabular}[#1]{@{}c@{}}#2\end{tabular}}

\begin{document}
\pagenumbering{gobble}
\begin{center}
    \small{
        \textbf{МИНИCТЕРCТВО НАУКИ И ВЫCШЕГО ОБРАЗОВАНИЯ РОCCИЙCКОЙ ФЕДЕРАЦИИ}\\
        ФЕДЕРАЛЬНОЕ ГОCУДАРCТВЕННОЕ БЮДЖЕТНОЕ ОБРАЗОВАТЕЛЬНОЕ УЧРЕЖДЕНИЕ\\ВЫCШЕГО ОБРАЗОВАНИЯ \\
        \textbf{«БЕЛГОРОДCКИЙ ГОCУДАРCТВЕННЫЙ ТЕХНОЛОГИЧЕCКИЙ\\УНИВЕРCИТЕТ им. В. Г. ШУХОВА»\\ (БГТУ им. В.Г. Шухова)} \\
        \bigbreak
        \includegraphics[width=70mm]{log}\\
        ИНСТИТУТ ИНФОРМАЦИОННЫХ ТЕХНОЛОГИЙ И УПРАВЛЯЮЩИХ СИСТЕМ\\}
\end{center}

\vfill
\begin{center}
    \large{
        \textbf{
            Лабораторная работа №5}}\\
    \normalsize{
        по дисциплине: Исследование операций \\
        тема: «Двойственный симплекс метод»}
\end{center}
\vfill
\hfill\textbox{
    Выполнил: ст. группы ПВ-223\\Пахомов Владислав Андреевич
    \bigbreak
    Проверили: \\проф. Вирченко Юрий Петрович
}
\vfill\begin{center}
    Белгород 2024 г.
\end{center}
\newpage
\begin{center}
    \textbf{Лабораторная работа №5}\\
    Двойственный симплекс метод\\
\end{center}
\textbf{Цель работы: }изучить элементы теории двойственности,
двойственный симплекс-метод для пары симметрично двойственных
задач, а также метод последовательного уточнения оценок.\\

\textbf{Задание: }составить и отладить программы решения транспортной задачи
распределительным методом и методом потенциалов. В рамках
подготовки тестовых данных решить задачу вручную.

\begin{center}
    \textbf{Вариант 10}
\end{center}
\begin{equation*}
    \begin{aligned}
        z = -3x_1 + 9x_2 + x_3 + 4x_4 \rightarrow min; \\
        \begin{cases}
            -4x_1 + x_2 - 4x_3 + x_4 \geq 15,  \\
            2x_1 + 3x_2 + 4x_3 + 2x_4 \geq 30, \\
            3x_1 + 7x_2 + 2x_3 - 3x_4 = 25,    \\
        \end{cases}             \\
        x_i \geq 0 (i=\overline{1, 4}).
    \end{aligned}
\end{equation*}\\

\textbf{Блок-схемы: }
\begin{center}
    \subfile{sources/main.tex}\bigbreak
    \subfile{sources/solveDualSimplexMethod.tex}\bigbreak
    \subfile{sources/solveCommonDualSimplexMethod.tex}\bigbreak
\end{center}

\textbf{Листинг программы: }
\begin{minted}{C++}
#include <vector>
#include <array>

#include "../../libs/alg/alg.h"

int main() {
    // Подготовить входные данные
    std::vector<std::array<Fraction, 5>> matrix;
    matrix.push_back({{{-4}, {1}, {-4}, {1}, {15}}});
    matrix.push_back({{{2}, {3}, {4}, {2}, {30}}});
    matrix.push_back({{{3}, {7}, {2}, {-3}, {25}}});
    std::array<Fraction, 5> function{{{-3}, {9}, {1}, {4}, {0}}};

    // Вывод ответа
    auto g = solveDualSimplexMethod<5, 3, Fraction>(matrix, function, MIN, Fraction());
    std::cout << std::get<0>(g) << std::endl;
    auto l = std::get<1>(g);
    for (int i = 0; i < l.size(); i++) {
        std::cout << l[i] << " ";
    }
}
\end{minted}

\begin{minted}{C++}
#pragma once 

#include "../alg.h"

enum Extremum {
    MIN,
    MAX
};

template <std::size_t T, std::size_t MatrixLines, typename CountType>
std::tuple<std::vector<std::array<CountType, MatrixLines + 1>>, std::array<CountType, MatrixLines + 1>, Extremum> 
getDualProblem(std::vector<std::array<CountType, T>> sourceSystem, std::array<CountType, T> sourceFunc, Extremum extr) {
    std::vector<std::array<CountType, MatrixLines + 1>> resMatrix;
    for (int i = 0; i < T; i++) {
        resMatrix.push_back({});
    }
    std::array<CountType, MatrixLines + 1> resFunc;
    resFunc.back() = sourceFunc.back();

    for (int i = 0; i < MatrixLines; i++) {
        resFunc[i] = sourceSystem[i].back();

        for (int j = 0; j < T - 1; j++) {
            resMatrix[j][i] = sourceSystem[i][j];
        }
    }

    for (int j = 0; j < T - 1; j++) {
        resMatrix[j].back() = sourceFunc[j];
    }

    return {resMatrix, resFunc, extr == MIN ? MAX : MIN};
} 

template <std::size_t T, std::size_t MatrixLines, typename CountType, std::size_t ExtendedMatrixSize = T + MatrixLines>
std::tuple<std::array<CountType, ExtendedMatrixSize>, std::vector<std::array<CountType, ExtendedMatrixSize>>> introduceNewVariables(std::vector<std::array<CountType, T>> sourceSystem, std::array<CountType, T> sourceFunc) {
std::vector<std::array<CountType, ExtendedMatrixSize>> newSystem;
    for (int i = 0; i < MatrixLines; i++) {
        newSystem.push_back({});
    }
    std::array<CountType, ExtendedMatrixSize> newFunc;
    for (int i = 0; i < MatrixLines; i++) {
        newSystem[i].back() = sourceSystem[i].back();
        int j;
        for (j = 0; j < T - 1; j++) {
            newSystem[i][j] = sourceSystem[i][j];
        }

        newSystem[i][j + i] = {1};
    }
    newFunc.back() = sourceFunc.back();
    for (int i = 0; i < T - 1; i++) {
        newFunc[i] = sourceFunc[i];
    }

    return {newFunc, newSystem};
}

template <std::size_t T, std::size_t MatrixLines, typename CountType, std::size_t ExtendedMatrixSize = T + MatrixLines>
std::tuple<CountType, std::vector<CountType>> solveDualSimplexMethod(std::vector<std::array<CountType, T>> sourceSystem, std::array<CountType, T> sourceFunc, Extremum extr, CountType EPS) {
    // Если экстремум - минимум
    if (extr == MIN) {
        // То получаем двойственную задачу и решаем её той же функцией
        auto reversed = getDualProblem<T, MatrixLines, CountType>(sourceSystem, sourceFunc, extr);
        auto res = solveDualSimplexMethod<MatrixLines + 1, T - 1, CountType>(std::get<0>(reversed), std::get<1>(reversed), std::get<2>(reversed), EPS);
        auto newF = std::get<1>(res);
        std::rotate(newF.begin(), newF.begin() + T - 2, newF.end());

        return {std::get<0>(res), newF};
    }

    // Вводим дополнительные переменные
    auto newVars = introduceNewVariables<T, MatrixLines, CountType>(sourceSystem, sourceFunc);
    auto newFunc = std::get<0>(newVars);
    auto newSystem = std::get<1>(newVars);

    // Решаем обычным симплекс-методом
    CountType ans = solveSimplexMethodMax(newSystem, newFunc, EPS);
    // Возвращаем последнюю строчку
    return {ans, std::vector<CountType>(newFunc.begin(), newFunc.end() - 1)};
}
    \end{minted}
\begin{minted}{C++}
#pragma once 

#include "../alg.h"

template <std::size_t T, std::size_t MatrixLines, typename CountType, std::size_t ExtendedMatrixSize = T + MatrixLines>
std::tuple<CountType, std::vector<CountType>> solveCommonDualSimplexMethod(std::vector<std::array<CountType, T>> sourceSystem, std::array<CountType, T> sourceFunc, Extremum extr, CountType EPS) {
    // Если экстремум - минимум
    if (extr == MIN) {
        // То получаем двойственную задачу и решаем её той же функцией
        auto reversed = getDualProblem<T, MatrixLines, CountType>(sourceSystem, sourceFunc, extr);
        auto res = solveCommonDualSimplexMethod<MatrixLines + 1, T - 1, CountType>(std::get<0>(reversed), std::get<1>(reversed), std::get<2>(reversed), EPS);
        auto newF = std::get<1>(res);
        std::rotate(newF.begin(), newF.begin() + T - 2, newF.end());

        return {std::get<0>(res), newF};
    }

    // Вводим дополнительные переменные
    auto newVars = introduceNewVariables<T, MatrixLines, CountType>(sourceSystem, sourceFunc);
    auto newFunc = std::get<0>(newVars);
    auto newSystem = std::get<1>(newVars);

    // Строим симплекс-таблицу, копируя в неё матрицу newSystem
    std::vector<std::array<CountType, ExtendedMatrixSize>> simplexMatrix(newSystem);
    
    // Добавляем новую строку - целевую функцию, умножая её коэф. yi на -1
    simplexMatrix.push_back(newFunc);
    for (int i = 0; i < T; i++)
        simplexMatrix.back()[i] *= -1;

    CountType minusOne = {-1};
    CountType zero = EPS;
    // Используем метод последовательного уточнения оценок
    while (true) {
        int minRowIndex = -1;
        for (int i = 0; i < MatrixLines; i++)
            if (simplexMatrix[i].back() < zero && (minRowIndex == -1 || simplexMatrix[i].back() < simplexMatrix[minRowIndex].back()))
                minRowIndex = i;
        
        if (minRowIndex == -1) break;

        int minColumnIndex = -1;
        for (int i = 0; i < ExtendedMatrixSize - 1; i++)
            if (simplexMatrix[minRowIndex][i] < zero && (minColumnIndex == -1 || 
            (minusOne * simplexMatrix.back()[i] / simplexMatrix[minRowIndex][i]) < (minusOne * simplexMatrix.back()[minColumnIndex] / simplexMatrix[minRowIndex][minColumnIndex])))
                minColumnIndex = i;

        if (minColumnIndex == -1) throw std::invalid_argument("No solution"); 

        subtractLineFromOther(simplexMatrix, minRowIndex, minColumnIndex, EPS);
    }

    // Возвращам последнюю строчку
    return {simplexMatrix.back().back(), std::vector<CountType>(simplexMatrix.back().begin(), simplexMatrix.back().end() - 1)};
}
\end{minted}

Результат выполнения программы:
\begin{minted}{console}
95
0 7 0 8 0 7 0
\end{minted}

\textbf{Результаты вычислений: }\\
\begin{equation*}
    \begin{aligned}
        z = -3x_1 + 9x_2 + x_3 + 4x_4 \rightarrow min; \\
        \begin{cases}
            -4x_1 + x_2 - 4x_3 + x_4 \geq 15,  \\
            2x_1 + 3x_2 + 4x_3 + 2x_4 \geq 30, \\
            3x_1 + 7x_2 + 2x_3 - 3x_4 = 25,    \\
        \end{cases}             \\
        x_i \geq 0 (i=\overline{1, 4}).
    \end{aligned}
\end{equation*}\\
Получим двойственную задачу
\begin{equation*}
    \begin{aligned}
        z' = 15y_1 + 30y_2 + 25y_3 \rightarrow max; \\
        \begin{cases}
            -4y_1 + 2y_2 + 3y_3 \leq -3, \\
            y_1 + 3y_2 + 7y_3 \leq 9,    \\
            -4y_1 + 4y_2 + 2y_3 \leq 1,  \\
            y_1 + 2y_2 - 3y_3 \leq 4,    \\
        \end{cases}                \\
        y_i \geq 0 (i=\overline{1, 3}).
    \end{aligned}
\end{equation*}\\

Получим симплекс-таблицу
\begin{center}
    \begin{longtable}{|c|c|c|c|c|c|c|c|c|}
        \hline
        Баз. пер. & Св. чл.    & $y_1$    & $y_2$    & $y_3$    & $y_4 $     & $y_5$     & $y_6$ & $y_7$     \\
        \hline
        $y_4$     & $-3$       & $2$      & $-1$     & $1$      & $\textbf{3}$        & $1$       & $1$   & $0$       \\
        \hline
        $y_5$     & $9$        & $1$      & $4$      & $1$      & $1$        & $-2$      & $0$   & $1$       \\
        \hline
        $y_6$     & $1$        & $-1$     & $4$      & $6$      & $3$        & $-8$      & $0$   & $0$       \\
        \hline
        $y_7$     & $4$        & $-1$     & $4$      & $6$      & $3$        & $-8$      & $0$   & $0$       \\
        \hline
        $z'$      & $0$        & $-2$     & $-7$     & $-8$     & $-7$       & $9$       & $0$   & $0$       \\
        \hline
        \multicolumn{9}{c}{}                                                                                 \\
        \hline
        Баз. пер. & Св. чл.    & $y_1$    & $y_2$    & $y_3$    & $y_4 $     & $y_5$     & $y_6$ & $y_7$     \\
        \hline
        $y_3$     & $-1$       & $-4/3$   & $\textbf{2/3}$    & $1$      & $1/3$      & $0$       & $0$   & $0$       \\
        \hline
        $y_5$     & $16$       & $31/3$   & $-5/3$   & $0$      & $-7/3$     & $1$       & $0$   & $0$       \\
        \hline
        $y_6$     & $3$        & $-4/3$   & $8/3$    & $0$      & $1$        & $0$       & $1$   & $0$       \\
        \hline
        $y_7$     & $1$        & $-3$     & $4$      & $0$      & $1$        & $0$       & $0$   & $1$       \\
        \hline
        $z'$      & $-25$      & $-145/3$ & $-40/3$  & $0$      & $25/3$     & $0$       & $0$   & $0$       \\
        \hline
        \multicolumn{9}{c}{}                                                                                 \\
        \hline
        Баз. пер. & Св. чл.    & $y_1$    & $y_2$    & $y_3$    & $y_4 $     & $y_5$     & $y_6$ & $y_7$     \\
        \hline
        $y_2$     & $-3/2$     & $\textbf{-2}$     & $1$      & $3/2$    & $1/2$      & $0$       & $0$   & $0$       \\
        \hline
        $y_5$     & $27/2$     & $7$      & $0$      & $5/2$    & $-3/2$     & $1$       & $0$   & $0$       \\
        \hline
        $y_6$     & $7$        & $4$      & $0$      & $-4$     & $-2$       & $0$       & $1$   & $0$       \\
        \hline
        $y_7$     & $7$        & $5$      & $0$      & $-6$     & $-1$       & $0$       & $0$   & $1$       \\
        \hline
        $z'$      & $-45$      & $-75$    & $0$      & $20$     & $15$       & $0$       & $0$   & $0$       \\
        \hline
        \multicolumn{9}{c}{}                                                                                 \\
        \hline
        Баз. пер. & Св. чл.    & $y_1$    & $y_2$    & $y_3$    & $y_4 $     & $y_5$     & $y_6$ & $y_7$     \\
        \hline
        $y_1$     & $3/4$      & $1$      & $-1/2$   & $-3/4$   & $-1/4$     & $0$       & $0$   & $0$       \\
        \hline
        $y_5$     & $33/4$     & $0$      & $7/2$    & $31/4$   & $1/4$      & $1$       & $0$   & $0$       \\
        \hline
        $y_6$     & $4$        & $0$      & $2$      & $-1$     & $-1$       & $0$       & $1$   & $0$       \\
        \hline
        $y_7$     & $13/4$     & $0$      & $\textbf{5/2}$    & $-9/4$   & $1/4$      & $0$       & $0$   & $1$       \\
        \hline
        $z'$      & $45/4$     & $0$      & $-75/2$  & $-145/4$ & $-15/4$    & $0$       & $0$   & $0$       \\
        \hline
        \multicolumn{9}{c}{}                                                                                 \\
        \hline
        Баз. пер. & Св. чл.    & $y_1$    & $y_2$    & $y_3$    & $y_4 $     & $y_5$     & $y_6$ & $y_7$     \\
        \hline
        $y_1$     & $7/5$      & $1$      & $0$      & $-6/5$   & $-1/5$     & $0$       & $0$   & $1/5$     \\
        \hline
        $y_5$     & $37/10$    & $0$      & $0$      & $\textbf{109/10}$ & $-1/10$    & $1$       & $0$   & $-7/5$    \\
        \hline
        $y_6$     & $7/5$      & $0$      & $0$      & $4/5$    & $-6/5$     & $0$       & $1$   & $-4/5$    \\
        \hline
        $y_2$     & $13/10$    & $0$      & $1$      & $-9/10$  & $1/10$     & $0$       & $0$   & $2/5$     \\
        \hline
        $z'$      & $60$       & $0$      & $0$      & $-70$    & $0$        & $0$       & $0$   & $15$      \\
        \hline
        \multicolumn{9}{c}{}                                                                                 \\
        \hline
        Баз. пер. & Св. чл.    & $y_1$    & $y_2$    & $y_3$    & $y_4 $     & $y_5$     & $y_6$ & $y_7$     \\
        \hline
        $y_1$     & $197/109$  & $1$      & $0$      & $0$      & $-23/109$  & $12/109$  & $0$   & $5/109$   \\
        \hline
        $y_3$     & $37/109$   & $0$      & $0$      & $1$      & $-1/109$   & $10/109$  & $0$   & $-14/109$ \\
        \hline
        $y_6$     & $123/109$  & $0$      & $0$      & $0$      & $-130/109$ & $-8/109$  & $1$   & $-76/109$ \\
        \hline
        $y_2$     & $175/109$  & $0$      & $1$      & $0$      & $\textbf{10/109}$   & $9/109$   & $0$   & $31/109$  \\
        \hline
        $z'$      & $9130/109$ & $0$      & $0$      & $0$      & $-70/109$  & $700/109$ & $0$   & $655/109$ \\
        \hline
        \multicolumn{9}{c}{}                                                                                 \\
        \hline
        Баз. пер. & Св. чл.    & $y_1$    & $y_2$    & $y_3$    & $y_4 $     & $y_5$     & $y_6$ & $y_7$     \\
        \hline
        $y_1$     & $11/2$     & $1$      & $23/10$  & $0$      & $0$  & $3/10$  & $0$   & $7/10$   \\
        \hline
        $y_3$     & $1/2$      & $0$      & $1/10$   & $1$      & $0$   & $1/10$  & $0$   & $-1/10$ \\
        \hline
        $y_6$     & $22$       & $0$      & $13$     & $0$      & $0$ & $1$  & $1$   & $3$ \\
        \hline
        $y_4$     & $35/2$     & $0$      & $109/10$ & $0$      & $1$   & $9/10$   & $0$   & $31/10$  \\
        \hline
        $z'$      & $95$       & $0$      & $7$      & $0$      & $0$  & $7$ & $0$   & $8$ \\
        \hline
    \end{longtable}
\end{center}

$x_1 = 0$, $x_2 = 7$, $x_3 = 0$, $x_4 = 8$, $z_{min} = 95$

\textbf{Вывод: } в ходе лабораторной работы изучили элементы теории двойственности, двойственный симплекс-метод
для пары симметрично двойственных задач, а также метод последовательного уточнения
оценок.

\end{document}