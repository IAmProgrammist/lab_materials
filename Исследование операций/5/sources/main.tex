% Созданная блок схема работает только с компиляторами XeTeX и LuaTeX.
\documentclass[../report5.tex]{subfiles}

% Необходимые зависимости
\usepackage[utf8]{inputenc}
\usepackage[english,russian]{babel}
\usepackage{pgfplots}
\usepackage{verbatim}
\usetikzlibrary{positioning}
\usetikzlibrary{shapes.geometric}
\usetikzlibrary{shapes.misc}
\usetikzlibrary{calc}
\usetikzlibrary{chains}
\usetikzlibrary{matrix}
\usetikzlibrary{decorations.text}
\usepackage{fontspec}
\usetikzlibrary{backgrounds}

\begin{document}
% Блок-схема
% Если Вы хотите добавить блок схему в свой документ, скопируйте код между комментариями
% С линияи и вставьте в документ.

% --------------------------
\newsavebox{\bdgQFHQHMNF}
\begin{lrbox}{\bdgQFHQHMNF}\begin{tikzpicture}[every node/.style={inner sep=0,outer sep=0}]
\makeatletter
\newcommand{\verbatimfont}[1]{\def\verbatim@font{#1}}
\makeatother
% Шрифты
\newfontfamily\fontQFHQHMNFPOKFMFJC[Scale=0.553635, SizeFeatures={Size=10.000000}]{Consolas}

% Цвета
\definecolor{colorQFHQHMNFBIMOSSF}{rgb}{1.000000,1.000000,1.000000}
\definecolor{colorQFHQHMNFCONTBMSQ}{rgb}{1.000000,1.000000,1.000000}
\definecolor{colorQFHQHMNFMGOQQH}{rgb}{0.000000,0.000000,0.000000}
\draw[opacity=1.000000, rounded corners=0.180366 cm, fill=colorQFHQHMNFBIMOSSF, draw=none] (0.776581,0.003906) -- (2.219512,0.003906) -- (2.219512,-0.356827) -- (0.776581,-0.356827) -- cycle;
\draw[opacity=1.000000, rounded corners=0.180366 cm, fill=none, line width=0.019531cm, colorQFHQHMNFMGOQQH] (0.776581,0.003906) -- (2.219512,0.003906) -- (2.219512,-0.356827) -- (0.776581,-0.356827) -- cycle;
\verbatimfont{\normalsize\fontQFHQHMNFPOKFMFJC}
\node[opacity=1.000000, above right, colorQFHQHMNFMGOQQH] at(1.185579, -0.239640) {\verb|Начало|};
\draw[fill=colorQFHQHMNFBIMOSSF, draw=none, opacity=1.000000] (-0.003906,-0.630261) -- (3.000000,-0.630261) -- (3.000000,-1.013691) -- (-0.003906,-1.013691) -- cycle;
\draw[opacity=1.000000, fill=none, line width=0.019531cm, colorQFHQHMNFMGOQQH] (-0.003906,-0.630261) -- (3.000000,-0.630261) -- (3.000000,-1.013691) -- (-0.003906,-1.013691) -- cycle;
\verbatimfont{\normalsize\fontQFHQHMNFPOKFMFJC}
\node[opacity=1.000000, above right, colorQFHQHMNFMGOQQH] at(0.113280, -0.896505) {\verb|Подготовить входные данные|};
\draw[opacity=1.000000, fill=none, line width=0.011719 cm, colorQFHQHMNFMGOQQH] (1.498047,-0.356827) -- (1.498047,-0.630261);
\draw[fill=colorQFHQHMNFBIMOSSF, draw=none, opacity=1.000000] (0.694453,-1.287125) -- (2.301641,-1.287125) -- (2.301641,-1.670555) -- (0.694453,-1.670555) -- cycle;
\draw[opacity=1.000000, fill=none, line width=0.019531cm, colorQFHQHMNFMGOQQH] (0.694453,-1.287125) -- (2.301641,-1.287125) -- (2.301641,-1.670555) -- (0.694453,-1.670555) -- cycle;
\draw[opacity=1.000000, fill=none, line width=0.019531cm, colorQFHQHMNFMGOQQH] (0.751967,-1.287125) -- (2.244126,-1.287125) -- (2.244126,-1.670555) -- (0.751967,-1.670555) -- cycle;
\verbatimfont{\normalsize\fontQFHQHMNFPOKFMFJC}
\node[opacity=1.000000, above right, colorQFHQHMNFMGOQQH] at(0.869154, -1.553369) {\verb|Вывод ответа|};
\draw[opacity=1.000000, fill=none, line width=0.011719 cm, colorQFHQHMNFMGOQQH] (1.498047,-1.013691) -- (1.498047,-1.287125);
\draw[opacity=1.000000, rounded corners=0.191715 cm, fill=colorQFHQHMNFBIMOSSF, draw=none] (0.731187,-1.943989) -- (2.264907,-1.943989) -- (2.264907,-2.327419) -- (0.731187,-2.327419) -- cycle;
\draw[opacity=1.000000, rounded corners=0.191715 cm, fill=none, line width=0.019531cm, colorQFHQHMNFMGOQQH] (0.731187,-1.943989) -- (2.264907,-1.943989) -- (2.264907,-2.327419) -- (0.731187,-2.327419) -- cycle;
\verbatimfont{\normalsize\fontQFHQHMNFPOKFMFJC}
\node[opacity=1.000000, above right, colorQFHQHMNFMGOQQH] at(1.239270, -2.210233) {\verb|Конец|};
\draw[opacity=1.000000, fill=none, line width=0.011719 cm, colorQFHQHMNFMGOQQH] (1.498047,-1.670555) -- (1.498047,-1.943989);
\end{tikzpicture}
\end{lrbox}
% Здесь Вы можете поменять размер блок схемы. Оношение ширина/высота будет сохранено.
% Для изменения размеров блок схемы Вы можете изменять первый параметр resizebox, он задаёт желаемую ширину.
\resizebox{3.000000cm}{!}{\usebox{\bdgQFHQHMNF}}
% --------------------------

% Конец блок схемы
\end{document}