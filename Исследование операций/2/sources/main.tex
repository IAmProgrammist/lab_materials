% Созданная блок схема работает только с компиляторами XeTeX и LuaTeX.
\documentclass[../report2.tex]{subfiles}

% Необходимые зависимости
\usepackage[utf8]{inputenc}
\usepackage[english,russian]{babel}
\usepackage{pgfplots}
\usepackage{verbatim}
\usetikzlibrary{positioning}
\usetikzlibrary{shapes.geometric}
\usetikzlibrary{shapes.misc}
\usetikzlibrary{calc}
\usetikzlibrary{chains}
\usetikzlibrary{matrix}
\usetikzlibrary{decorations.text}
\usepackage{fontspec}
\usetikzlibrary{backgrounds}

\begin{document}
% Блок-схема
% Если Вы хотите добавить блок схему в свой документ, скопируйте код между комментариями
% С линияи и вставьте в документ.

% --------------------------
\newsavebox{\bdgTKAQDO}
\begin{lrbox}{\bdgTKAQDO}\begin{tikzpicture}[every node/.style={inner sep=0,outer sep=0}]
\makeatletter
\newcommand{\verbatimfont}[1]{\def\verbatim@font{#1}}
\makeatother
% Шрифты
\newfontfamily\fontTKAQDOAEPMCHT[Scale=0.653823, SizeFeatures={Size=10.000000}]{Consolas}

% Цвета
\definecolor{colorTKAQDOSJKCHPFH}{rgb}{1.000000,1.000000,1.000000}
\definecolor{colorTKAQDOBMQFAKL}{rgb}{1.000000,1.000000,1.000000}
\definecolor{colorTKAQDOAMSMJLL}{rgb}{0.000000,0.000000,0.000000}
\draw[opacity=1.000000, rounded corners=0.213006 cm, fill=colorTKAQDOSJKCHPFH, draw=none] (1.645669,0.004613) -- (3.349718,0.004613) -- (3.349718,-0.421399) -- (1.645669,-0.421399) -- cycle;
\draw[opacity=1.000000, rounded corners=0.213006 cm, fill=none, line width=0.023065cm, colorTKAQDOAMSMJLL] (1.645669,0.004613) -- (3.349718,0.004613) -- (3.349718,-0.421399) -- (1.645669,-0.421399) -- cycle;
\verbatimfont{\normalsize\fontTKAQDOAEPMCHT}
\node[opacity=1.000000, above right, colorTKAQDOAMSMJLL] at(2.128680, -0.283007) {\verb|Начало|};
\draw[fill=colorTKAQDOSJKCHPFH, draw=none, opacity=1.000000] (0.975150,-0.744315) -- (4.020237,-0.744315) -- (4.020237,-1.513403) -- (0.975150,-1.513403) -- cycle;
\draw[opacity=1.000000, fill=none, line width=0.023065cm, colorTKAQDOAMSMJLL] (0.975150,-0.744315) -- (4.020237,-0.744315) -- (4.020237,-1.513403) -- (0.975150,-1.513403) -- cycle;
\verbatimfont{\normalsize\fontTKAQDOAEPMCHT}
\node[opacity=1.000000, above right, colorTKAQDOAMSMJLL] at(1.113542, -1.076759) {\verb|Инициализируем матрицу|};
\verbatimfont{\normalsize\fontTKAQDOAEPMCHT}
\node[opacity=1.000000, above right, colorTKAQDOAMSMJLL] at(1.939021, -1.375011) {\verb|и функцию|};
\draw[opacity=1.000000, fill=none, line width=0.013839 cm, colorTKAQDOAMSMJLL] (2.497693,-0.421399) -- (2.497693,-0.744315);
\draw[opacity=1.000000, fill=colorTKAQDOSJKCHPFH, draw=none] (-0.004613, -2.596397) -- (0.185406, -1.836319) -- (5.000000, -1.836319) -- (4.809980, -2.596397) --cycle;
\draw[opacity=1.000000, fill=none, line width=0.023065cm, colorTKAQDOAMSMJLL] (-0.004613, -2.596397) -- (0.185406, -1.836319) -- (5.000000, -1.836319) -- (4.809980, -2.596397) --cycle;
\verbatimfont{\normalsize\fontTKAQDOAEPMCHT}
\node[opacity=1.000000, above right, colorTKAQDOAMSMJLL] at(1.875952, -2.150743) {\verb|Вывод Zmax|};
\verbatimfont{\normalsize\fontTKAQDOAEPMCHT}
\node[opacity=1.000000, above right, colorTKAQDOAMSMJLL] at(0.185406, -2.458005) {\verb|(вызов функции solveSimplexMethodMax)|};
\draw[opacity=1.000000, fill=none, line width=0.013839 cm, colorTKAQDOAMSMJLL] (2.497693,-1.513403) -- (2.497693,-1.836319);
\draw[opacity=1.000000, rounded corners=0.226408 cm, fill=colorTKAQDOSJKCHPFH, draw=none] (1.592060,-2.919313) -- (3.403327,-2.919313) -- (3.403327,-3.372130) -- (1.592060,-3.372130) -- cycle;
\draw[opacity=1.000000, rounded corners=0.226408 cm, fill=none, line width=0.023065cm, colorTKAQDOAMSMJLL] (1.592060,-2.919313) -- (3.403327,-2.919313) -- (3.403327,-3.372130) -- (1.592060,-3.372130) -- cycle;
\verbatimfont{\normalsize\fontTKAQDOAEPMCHT}
\node[opacity=1.000000, above right, colorTKAQDOAMSMJLL] at(2.192088, -3.233738) {\verb|Конец|};
\draw[opacity=1.000000, fill=none, line width=0.013839 cm, colorTKAQDOAMSMJLL] (2.497693,-2.596397) -- (2.497693,-2.919313);
\end{tikzpicture}
\end{lrbox}
% Здесь Вы можете поменять размер блок схемы. Оношение ширина/высота будет сохранено.
% Для изменения размеров блок схемы Вы можете изменять первый параметр resizebox, он задаёт желаемую ширину.
\resizebox{5.000000cm}{!}{\usebox{\bdgTKAQDO}}
% --------------------------

% Конец блок схемы
\end{document}