% Созданная блок схема работает только с компиляторами XeTeX и LuaTeX.
\documentclass[../report2.tex]{subfiles}

% Необходимые зависимости
\usepackage[utf8]{inputenc}
\usepackage[english,russian]{babel}
\usepackage{pgfplots}
\usepackage{verbatim}
\usetikzlibrary{positioning}
\usetikzlibrary{shapes.geometric}
\usetikzlibrary{shapes.misc}
\usetikzlibrary{calc}
\usetikzlibrary{chains}
\usetikzlibrary{matrix}
\usetikzlibrary{decorations.text}
\usepackage{fontspec}
\usetikzlibrary{backgrounds}

\begin{document}
% Блок-схема
% Если Вы хотите добавить блок схему в свой документ, скопируйте код между комментариями
% С линияи и вставьте в документ.

% --------------------------
\newsavebox{\bdgAMENPI}
\begin{lrbox}{\bdgAMENPI}\begin{tikzpicture}[every node/.style={inner sep=0,outer sep=0}]
\makeatletter
\newcommand{\verbatimfont}[1]{\def\verbatim@font{#1}}
\makeatother
% Шрифты
\newfontfamily\fontAMENPIPSCPLKL[Scale=0.720183, SizeFeatures={Size=10.000000}]{Consolas}

% Цвета
\definecolor{colorAMENPIRIDJCKAA}{rgb}{1.000000,1.000000,1.000000}
\definecolor{colorAMENPIDCHJDLMQ}{rgb}{1.000000,1.000000,1.000000}
\definecolor{colorAMENPIHLMIHPC}{rgb}{0.000000,0.000000,0.000000}
\draw[opacity=1.000000, rounded corners=0.270849 cm, fill=colorAMENPIRIDJCKAA, draw=none] (0.032631,0.005081) -- (5.962287,0.005081) -- (5.962287,-0.536617) -- (0.032631,-0.536617) -- cycle;
\draw[opacity=1.000000, rounded corners=0.270849 cm, fill=none, line width=0.025406cm, colorAMENPIHLMIHPC] (0.032631,0.005081) -- (5.962287,0.005081) -- (5.962287,-0.536617) -- (0.032631,-0.536617) -- cycle;
\verbatimfont{\normalsize\fontAMENPIPSCPLKL}
\node[opacity=1.000000, above right, colorAMENPIHLMIHPC] at(0.303480, -0.384179) {\verb|solveSimplexMethodMax(matrix, function)|};
\draw[fill=colorAMENPIRIDJCKAA, draw=none, opacity=1.000000] (-0.005081,-0.892308) -- (6.000000,-0.892308) -- (6.000000,-3.398566) -- (-0.005081,-3.398566) -- cycle;
\draw[opacity=1.000000, fill=none, line width=0.025406cm, colorAMENPIHLMIHPC] (-0.005081,-0.892308) -- (6.000000,-0.892308) -- (6.000000,-3.398566) -- (-0.005081,-3.398566) -- cycle;
\verbatimfont{\normalsize\fontAMENPIPSCPLKL}
\node[opacity=1.000000, above right, colorAMENPIHLMIHPC] at(0.725019, -1.275241) {\verb|Подготовим матрицу таким образом, |};
\verbatimfont{\normalsize\fontAMENPIPSCPLKL}
\node[opacity=1.000000, above right, colorAMENPIHLMIHPC] at(0.147357, -1.607361) {\verb|чтобы в целевой функции были использованы |};
\verbatimfont{\normalsize\fontAMENPIPSCPLKL}
\node[opacity=1.000000, above right, colorAMENPIHLMIHPC] at(1.072808, -1.938737) {\verb|только свободные переменные. |};
\verbatimfont{\normalsize\fontAMENPIPSCPLKL}
\node[opacity=1.000000, above right, colorAMENPIHLMIHPC] at(0.774269, -2.270112) {\verb|Также отберём опорное решение, в |};
\verbatimfont{\normalsize\fontAMENPIPSCPLKL}
\node[opacity=1.000000, above right, colorAMENPIHLMIHPC] at(0.641406, -2.601488) {\verb|котором все свободные члены больше |};
\verbatimfont{\normalsize\fontAMENPIPSCPLKL}
\node[opacity=1.000000, above right, colorAMENPIHLMIHPC] at(0.500356, -2.916861) {\verb|или равны нулю. Результат сохраним в |};
\verbatimfont{\normalsize\fontAMENPIPSCPLKL}
\node[opacity=1.000000, above right, colorAMENPIHLMIHPC] at(2.033366, -3.246128) {\verb|preparedMatrix|};
\draw[opacity=1.000000, fill=none, line width=0.015244 cm, colorAMENPIHLMIHPC] (2.997459,-0.536617) -- (2.997459,-0.892308);
\draw[fill=colorAMENPIRIDJCKAA, draw=none, opacity=1.000000] (0.994458,-3.754257) -- (5.000460,-3.754257) -- (5.000460,-4.584408) -- (0.994458,-4.584408) -- cycle;
\draw[opacity=1.000000, fill=none, line width=0.025406cm, colorAMENPIHLMIHPC] (0.994458,-3.754257) -- (5.000460,-3.754257) -- (5.000460,-4.584408) -- (0.994458,-4.584408) -- cycle;
\draw[opacity=1.000000, fill=none, line width=0.025406cm, colorAMENPIHLMIHPC] (1.118981,-3.754257) -- (4.875938,-3.754257) -- (4.875938,-4.584408) -- (1.118981,-4.584408) -- cycle;
\verbatimfont{\normalsize\fontAMENPIPSCPLKL}
\node[opacity=1.000000, above right, colorAMENPIHLMIHPC] at(1.271420, -4.100594) {\verb|Вызовем симплекс метод на |};
\verbatimfont{\normalsize\fontAMENPIPSCPLKL}
\node[opacity=1.000000, above right, colorAMENPIHLMIHPC] at(1.408315, -4.431969) {\verb|преобразованной матрице|};
\draw[opacity=1.000000, fill=none, line width=0.015244 cm, colorAMENPIHLMIHPC] (2.997459,-3.398566) -- (2.997459,-3.754257);
\draw[opacity=1.000000, rounded corners=0.395475 cm, fill=colorAMENPIRIDJCKAA, draw=none] (0.388347,-4.940099) -- (5.606572,-4.940099) -- (5.606572,-5.731049) -- (0.388347,-5.731049) -- cycle;
\draw[opacity=1.000000, rounded corners=0.395475 cm, fill=none, line width=0.025406cm, colorAMENPIHLMIHPC] (0.388347,-4.940099) -- (5.606572,-4.940099) -- (5.606572,-5.731049) -- (0.388347,-5.731049) -- cycle;
\verbatimfont{\normalsize\fontAMENPIPSCPLKL}
\node[opacity=1.000000, above right, colorAMENPIHLMIHPC] at(0.783821, -5.318442) {\verb|Вернём результат вызова симплекс|};
\verbatimfont{\normalsize\fontAMENPIPSCPLKL}
\node[opacity=1.000000, above right, colorAMENPIHLMIHPC] at(2.591675, -5.578610) {\verb|метода|};
\draw[opacity=1.000000, fill=none, line width=0.015244 cm, colorAMENPIHLMIHPC] (2.997459,-4.584408) -- (2.997459,-4.940099);
\end{tikzpicture}
\end{lrbox}
% Здесь Вы можете поменять размер блок схемы. Оношение ширина/высота будет сохранено.
% Для изменения размеров блок схемы Вы можете изменять первый параметр resizebox, он задаёт желаемую ширину.
\resizebox{6.000000cm}{!}{\usebox{\bdgAMENPI}}
% --------------------------

% Конец блок схемы
\end{document}