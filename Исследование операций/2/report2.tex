\documentclass[a4paper,14pt]{extarticle}

\usepackage{subfiles}

\usepackage[english,russian]{babel}
\usepackage[T2A]{fontenc}
\usepackage[utf8]{inputenc}
\usepackage{ragged2e}
\usepackage{hyperref}
\usepackage{minted}
\setmintedinline{frame=lines, framesep=2mm, baselinestretch=1.5, bgcolor=LightGray, breaklines,fontsize=\scriptsize}
\setminted{frame=lines, framesep=2mm, baselinestretch=1.5, bgcolor=LightGray, breaklines,fontsize=\scriptsize}
\usepackage{xcolor}
\definecolor{LightGray}{gray}{0.9}
\usepackage{graphicx}
\usepackage[export]{adjustbox}
\usepackage[left=1cm,right=1cm, top=1cm,bottom=1cm,bindingoffset=0cm]{geometry}
\usepackage{fontspec}
\usepackage{ upgreek }
\usepackage[shortlabels]{enumitem}
\usepackage{adjustbox}
\usepackage{multirow}
\usepackage{amsmath}
\usepackage{amssymb}
\usepackage{pifont}
\usepackage{pgfplots}
\usepackage{longtable}
\usepackage{array}

\graphicspath{ {./images/} }
\makeatletter
\AddEnumerateCounter{\asbuk}{\russian@alph}{щ}
\makeatother
\setmonofont{Consolas}
\setmainfont{Times New Roman}

\newcommand\textbox[1]{
	\parbox{.45\textwidth}{#1}
} 

\newcommand{\specialcell}[2][c]{%
	\begin{tabular}[#1]{@{}c@{}}#2\end{tabular}}

\begin{document}
\pagenumbering{gobble}
\begin{center}
    \small{
        \textbf{МИНИCТЕРCТВО НАУКИ И ВЫCШЕГО ОБРАЗОВАНИЯ РОCCИЙCКОЙ ФЕДЕРАЦИИ}\\
        ФЕДЕРАЛЬНОЕ ГОCУДАРCТВЕННОЕ БЮДЖЕТНОЕ ОБРАЗОВАТЕЛЬНОЕ УЧРЕЖДЕНИЕ\\ВЫCШЕГО ОБРАЗОВАНИЯ \\
        \textbf{«БЕЛГОРОДCКИЙ ГОCУДАРCТВЕННЫЙ ТЕХНОЛОГИЧЕCКИЙ\\УНИВЕРCИТЕТ им. В. Г. ШУХОВА»\\ (БГТУ им. В.Г. Шухова)} \\
        \bigbreak
        \includegraphics[width=70mm]{log}\\
        ИНСТИТУТ ИНФОРМАЦИОННЫХ ТЕХНОЛОГИЙ И УПРАВЛЯЮЩИХ СИСТЕМ\\}
\end{center}

\vfill
\begin{center}
    \large{
        \textbf{
            Лабораторная работа №2}}\\
    \normalsize{
        по дисциплине: Исследование операций \\
        тема: «Симплекс-метод в чистом виде»}
\end{center}
\vfill
\hfill\textbox{
    Выполнил: ст. группы ПВ-223\\Пахомов Владислав Андреевич
    \bigbreak
    Проверили: \\проф. Вирченко Юрий Петрович
}
\vfill\begin{center}
    Белгород 2024 г.
\end{center}
\newpage
\begin{center}
    \textbf{Лабораторная работа №2}\\
    Симплекс-метод в чистом виде\\
\end{center}
\textbf{Цель работы: }изучение симплекс-метода для решения задачи
линейного программирования с использованием симплекс-таблиц,
получение навыков кодирования изученного алгоритма, отладки и
тестирования соответствующих программ.\\

Запрограммировать и отладить изученный алгоритм. В рамках
подготовки тестовых данных решить задачу вручную.

\begin{center}
    \textbf{Вариант 10}
\end{center}
\begin{center}
    $z = 6 x_1 + 2x_2 -5x_4 \rightarrow max;$\\
    $\begin{cases}
        -x_1 + 3x_2 + 6x_4 + x_6 = 26, \\
        5x_1 + x_2 + 7x_3 -3x_4 = 14,  \\
        2x_1 - x_2 - 2x_4 + x_5 = 12,
    \end{cases}$\\
    $x_i \ge 0 (i = \overline{1, 6})$.
\end{center}

Блок-схемы:
\begin{center}
    \subfile{sources/main.tex}\bigbreak
    \subfile{sources/solveSimplexMethodMax.tex}\bigbreak
    \subfile{sources/getMatrixForSimplexMethod.tex}\bigbreak
    \subfile{sources/solveSimplexMethodMaxRaw.tex}
\end{center}

Исходный код:
\begin{minted}{C++}
#include <iostream>
#include <iomanip>
#include <array>

#include "../libs/alg/alg.h"

int main() {
    // Инициализируем матрицу и функцию
    std::vector<std::array<double, 7>> matrix = {
    {-1, 3, 0, 6, 0, 1, 26},
    {5, 1, 7, -3, 0, 0, 14},
    {2, -1, 0, -2, 1, 0, 12}};
    std::array<double, 7> function{{6.0, 2.0, 0.0, -5.0, 0.0, 0.0, 0.0}};

    // Вывод ответа
    std::cout << "Zmax: " << solveSimplexMethodMax(matrix, function);
}
    \end{minted}
\href{https://github.com/IAmProgrammist/operations_research/blob/master/src/lab2/main.cpp}{Ссылка на репозиторий}\\

\begin{minted}{C++}
#pragma once 

#include <optional>
#include <vector>
#include <algorithm>

#include "../lab1/task1.tpp"

template <std::size_t T>
auto getMatrixForSimplexMethod(std::vector<std::array<double, T>>& matrix, std::array<double, T>& function) {
    // Определим, какие переменные могут быть базисными в опорном решении и занесём их в список functionBasisVars
    // (Если yi = 0, то переменная может быть базисной)
    std::vector<int> functionBasisVars;
    for (int i = 0; i < function.size() - 1; i++) 
        if (std::abs(function[i]) < EPS) 
            functionBasisVars.push_back(i);
    
    // Получим все базисные решения и для каждого решения basis
    for (auto& basis : getAllBasises(matrix)) {
        bool isAllBsMoreOrEqualToZero = true;
        for (int i = 0; i < basis.matrix.size() && isAllBsMoreOrEqualToZero; i++) {
            if (basis.matrix[i].back() < EPS) 
                isAllBsMoreOrEqualToZero = false;
        }

        // Хоть один из br меньше 0?
        if (!isAllBsMoreOrEqualToZero) { 
            // Перейти к следующему базисному решению
            continue;
        }

        // Если список базисных переменных из опорного решения basis входит в список
        // functionBasisVars
        std::sort(basis.indices.begin(), basis.indices.end());
        if (std::includes(basis.indices.begin(), basis.indices.end(), functionBasisVars.begin(), functionBasisVars.end()))
            // Возвращаем искомый базис
            return basis;
    }

    // Возвращаем ошибку - получить подходящее преобразование матрицы невозможно
    throw std::invalid_argument("No basis for function found");
}

template <std::size_t T>
double solveSimplexMethodMaxRaw(std::vector<std::array<double, T>>& matrix, std::array<double, T>& function) {
    auto preparedMatrix = getMatrixForSimplexMethod(matrix, function);
    // Строим симплекс-таблицу, копируя в неё матрицу matrix
    std::vector<std::array<double, T>> simplexMatrix(preparedMatrix.matrix);
    // Добавляем новую строку - целевую функцию, умножая её коэф. yi на -1
    simplexMatrix.push_back(function);
    for (int i = 0; i < T; i++)
        simplexMatrix.back()[i] *= -1;

    // Бесконечный цикл
    while (true) {
        // Найдём наибольший по модулю отрицательный элемент в последней строке, кроме свободного члена.
        int minColumnIndex = -1;
        for (int i = 0; i < T - 1; i++) {
            if (simplexMatrix.back()[i] < 0 && (minColumnIndex == -1 || simplexMatrix.back()[minColumnIndex] > simplexMatrix.back()[i]))
                minColumnIndex = i;
        }

        // Такой элемент найден?
        if (minColumnIndex == -1) { 
            // Решение получено, можно выходить из цикла
            break;
        }

        // Определим генеральный элемент таблицы
        int minRowIndex = -1;
        for (int i = 0; i < simplexMatrix.size() - 1; i++) {
            if (simplexMatrix[i][minColumnIndex] <= EPS) continue;
            if (minRowIndex == -1) minRowIndex = i;
            else if (simplexMatrix[minRowIndex].back() / simplexMatrix[minRowIndex][minColumnIndex] > 
            simplexMatrix[i].back() / simplexMatrix[i][minColumnIndex]) minRowIndex = i;
        }

        // Такой элемент найден?
        if (minRowIndex == -1) {
            // Решение получить невозможно, возвращаем ошибку
            throw std::invalid_argument("No solution");
        }

        // Преобразуем матрицу к новому базисному виду
        subtractLineFromOther(simplexMatrix, minRowIndex, minColumnIndex);
    }

    // Возвращаем свободный член в последней строке
    return simplexMatrix.back().back();
}

template <std::size_t T>
double solveSimplexMethodMax(std::vector<std::array<double, T>>& matrix, std::array<double, T>& function) {
    // Подготовим матрицу таким образом, чтобы в целевой функции были использованы только свободные переменные. 
    // Также отберём опорное решение, в котором все свободные члены больше или равны нулю
    auto preparedMatrix = getMatrixForSimplexMethod(matrix, function);
    
    // Вызовем симплекс метод на преобразованной матрице
    return solveSimplexMethodMaxRaw(preparedMatrix.matrix, function);
}
    \end{minted}
\href{https://github.com/IAmProgrammist/operations_research/blob/master/src/libs/alg/lab2/task.tpp}{Ссылка на репозиторий}\bigbreak

Результат выполнения программы:
\begin{minted}{console}
Zmax: 24
\end{minted}

Дополнительно проведём тестирование при помощи программы отбора базисных решений (см. лабораторная работа №1 задание 2).\\
Результат выполнения программы:
\begin{minted}{console}
==================================================================================================================
Обнаружено опорное решение: {0; 0; 2; 0; 12; 26}

Значение функции z(B): 0
==================================================================================================================
Обнаружено опорное решение: {0; 0; 3.85714; 4.33333; 20.6667; 0}

Значение функции z(B): -21.6667
==================================================================================================================
Обнаружено опорное решение: {0; 8.66667; 0.761905; 0; 20.6667; 0}


Значение функции z(B): 16.8
==================================================================================================================
Обнаружено опорное решение: {6; 0; 0; 5.33333; 10.6667; 0}

Значение функции z(B): 9.33333
==================================================================================================================
Обнаружено опорное решение: {1; 9; 0; 0; 19; 0}

Значение функции z(B): 24
==================================================================================================================

Zmax: 24

Оптимальный план: {1; 9; 0; 0; 19; 0}
\end{minted}

Результаты выполнения программ совпали.

Результаты вычислений:\bigbreak
Имеем целевую функцию
$z = 6 x_1 + 2x_2 -5x_4 \rightarrow max$.
и систему уравнений
\begin{equation*}
    \begin{cases}
        -x_1 + 3x_2 + 6x_4 + x_6 = 26, \\
        5x_1 + x_2 + 7x_3 -3x_4 = 14,  \\
        2x_1 - x_2 - 2x_4 + x_5 = 12,
    \end{cases}
\end{equation*}

Приведём систему уравнений к базисному виду. Базисными переменными выберем $x_3$, $x_5$ и $x_6$.
Построим расширенную матрицу:\bigbreak
$\begin{pmatrix}
        -1 & 3  & 0 & 6  & 0 & 1 & 26 \\
        5  & 1  & 7 & -3 & 0 & 0 & 14 \\
        2  & -1 & 0 & -2 & 1 & 0 & 12
    \end{pmatrix}$\bigbreak
Поменяем местами 1 и 2 строки, 2 и 3.\bigbreak
$\begin{pmatrix}
        5  & 1  & 7 & -3 & 0 & 0 & 14 \\
        2  & -1 & 0 & -2 & 1 & 0 & 12 \\
        -1 & 3  & 0 & 6  & 0 & 1 & 26 \\
    \end{pmatrix}$\bigbreak

Выберем ведущий элемент $x_3$ в первый строке:\bigbreak

$\begin{pmatrix}
        5  & 1  & \|7\| & -3 & 0 & 0 & 14 \\
        2  & -1 & 0     & -2 & 1 & 0 & 12 \\
        -1 & 3  & 0     & 6  & 0 & 1 & 26 \\
    \end{pmatrix}$\bigbreak
Разделим ведущую строку на 7.\bigbreak
$\begin{pmatrix}
        \frac{5}{7} & \frac{1}{7} & 1 & -\frac{3}{7} & 0 & 0 & 2  \\
        2           & -1          & 0 & -2           & 1 & 0 & 12 \\
        -1          & 3           & 0 & 6            & 0 & 1 & 26 \\
    \end{pmatrix}$\bigbreak

Привели систему к базисному виду
\begin{equation*}
    \begin{cases}
        x_3 = 2 - (\frac{5}{7}x_1 + \frac{1}{7}x_2 -\frac{3}{7}x_4) \\
        x_5 = 12 - (2x_1 - x_2 - 2x_4)                              \\
        x_6 = 26 - (-x_1 + 3x_2 + 6x_4)
    \end{cases}
\end{equation*}\bigbreak

Построим симплекс-таблицу
\begin{center}
    \begin{tabular}{|c|c|c|c|c|c|c|c|}
        \hline
        Баз. пер.        & Св. чл.         & $x_1 \downarrow$ & $x_2$            & $x_3$            & $x_4 $           & $x_5$ & $x_6$          \\
        \hline
        $\leftarrow x_3$ & $2$             & $\frac{5}{7}$    & $\frac{1}{7}$    & $1$              & $-\frac{3}{7}$   & $0$   & $0$            \\
        \hline
        $x_5$            & $12$            & $2$              & $-1$             & $0$              & $-2$             & $1$   & $0$            \\
        \hline
        $x_6$            & $26$            & $-1$             & $3$              & $0$              & $6$              & $0$   & $1$            \\
        \hline
        $z$              & $0$             & $-6$             & $-2$             & $0$              & $5$              & $0$   & $0$            \\
        \hline
        \multicolumn{8}{c}{}                                                                                                                    \\
        \hline
        Баз. пер.        & Св. чл.         & $x_1$            & $x_2 \downarrow$ & $x_3$            & $x_4$            & $x_5$ & $x_6$          \\
        \hline
        $x_1$            & $2\frac{4}{5}$  & $1$              & $\frac{1}{5}$    & $1\frac{2}{5}$   & $-\frac{3}{5}$   & $0$   & $0$            \\
        \hline
        $x_5$            & $6\frac{2}{5}$  & $0$              & $-1\frac{2}{5}$  & $-2\frac{4}{5}$  & $-\frac{4}{5}$   & $1$   & $0$            \\
        \hline
        $\leftarrow x_6$ & $28\frac{4}{5}$ & $0$              & $3\frac{1}{5}$   & $1\frac{2}{5}$   & $5\frac{2}{5}$   & $0$   & $1$            \\
        \hline
        $z$              & $16\frac{4}{4}$ & $0$              & $-\frac{4}{5}$   & $8\frac{2}{5}$   & $1\frac{2}{5}$   & $0$   & $0$            \\
        \hline
        \multicolumn{8}{c}{}                                                                                                                    \\
        \hline
        Баз. пер.        & Св. чл.         & $x_1$            & $x_2$            & $x_3$            & $x_4$            & $x_5$ & $x_6$          \\
        \hline
        $x_1$            & $1$             & $1$              & $0$              & $1\frac{5}{16}$  & $-\frac{15}{16}$ & $0$   & $\frac{1}{16}$ \\
        \hline
        $x_2$            & $9$             & $0$              & $1$              & $\frac{7}{16}$   & $1\frac{11}{16}$ & $0$   & $\frac{5}{16}$ \\
        \hline
        $x_5$            & $19$            & $0$              & $0$              & $-2\frac{3}{16}$ & $1\frac{9}{16}$  & $1$   & $\frac{7}{16}$ \\
        \hline
        $z$              & $24$            & $0$              & $0$              & $8\frac{3}{4}$   & $2\frac{3}{4}$   & $0$   & $\frac{1}{4}$  \\
        \hline
    \end{tabular}
\end{center}

Ответ: $z_{max} = 24$. Координаты точки максимума: $x_1 = 1$, $x_2 = 9$, $x_3 = 0$, $x_4 = 0$, $x_5 = 19$, $x_6 = 0$. 
Результаты ручных вычислений совпали с результатами выполнения программы. 

\textbf{Вывод: } в ходе лабораторной работы разработали и отладили программу, находящую
оптимальное решение в системе линейных уравнений для целевой функции, и использующей 
симплекс метод.

\end{document}