% Созданная блок схема работает только с компиляторами XeTeX и LuaTeX.
\documentclass[../report3.tex]{subfiles}

% Необходимые зависимости
\usepackage[utf8]{inputenc}
\usepackage[english,russian]{babel}
\usepackage{pgfplots}
\usepackage{verbatim}
\usetikzlibrary{positioning}
\usetikzlibrary{shapes.geometric}
\usetikzlibrary{shapes.misc}
\usetikzlibrary{calc}
\usetikzlibrary{chains}
\usetikzlibrary{matrix}
\usetikzlibrary{decorations.text}
\usepackage{fontspec}
\usetikzlibrary{backgrounds}

\begin{document}
% Блок-схема
% Если Вы хотите добавить блок схему в свой документ, скопируйте код между комментариями
% С линияи и вставьте в документ.

% --------------------------
\newsavebox{\bdgEMPRIN}
\begin{lrbox}{\bdgEMPRIN}\begin{tikzpicture}[every node/.style={inner sep=0,outer sep=0}]
\makeatletter
\newcommand{\verbatimfont}[1]{\def\verbatim@font{#1}}
\makeatother
% Шрифты
\newfontfamily\fontEMPRINHQDOEK[Scale=0.755560, SizeFeatures={Size=10.000000}]{Consolas}

% Цвета
\definecolor{colorEMPRINQEQNDS}{rgb}{1.000000,1.000000,1.000000}
\definecolor{colorEMPRINHONAKA}{rgb}{1.000000,1.000000,1.000000}
\definecolor{colorEMPRINBCATDT}{rgb}{0.000000,0.000000,0.000000}
\draw[opacity=1.000000, rounded corners=0.284154 cm, fill=colorEMPRINQEQNDS, draw=none] (-0.005331,0.005331) -- (9.000000,0.005331) -- (9.000000,-0.562977) -- (-0.005331,-0.562977) -- cycle;
\draw[opacity=1.000000, rounded corners=0.284154 cm, fill=none, line width=0.026654cm, colorEMPRINBCATDT] (-0.005331,0.005331) -- (9.000000,0.005331) -- (9.000000,-0.562977) -- (-0.005331,-0.562977) -- cycle;
\verbatimfont{\normalsize\fontEMPRINHQDOEK}
\node[opacity=1.000000, above right, colorEMPRINBCATDT] at(0.278824, -0.403050) {\verb|solveSystemOfLinearEquationsBigPenalties(matrix, function)|};
\draw[fill=colorEMPRINQEQNDS, draw=none, opacity=1.000000] (2.226850,-0.936139) -- (6.767819,-0.936139) -- (6.767819,-2.169559) -- (2.226850,-2.169559) -- cycle;
\draw[opacity=1.000000, fill=none, line width=0.026654cm, colorEMPRINBCATDT] (2.226850,-0.936139) -- (6.767819,-0.936139) -- (6.767819,-2.169559) -- (2.226850,-2.169559) -- cycle;
\verbatimfont{\normalsize\fontEMPRINHQDOEK}
\node[opacity=1.000000, above right, colorEMPRINBCATDT] at(3.577118, -1.334108) {\verb|Сформируем M, |};
\verbatimfont{\normalsize\fontEMPRINHQDOEK}
\node[opacity=1.000000, above right, colorEMPRINBCATDT] at(3.571261, -1.664972) {\verb|где M = сумма |};
\verbatimfont{\normalsize\fontEMPRINHQDOEK}
\node[opacity=1.000000, above right, colorEMPRINBCATDT] at(2.386778, -2.009632) {\verb|коэффециентов ф-ции по модулю|};
\draw[opacity=1.000000, fill=none, line width=0.015993 cm, colorEMPRINBCATDT] (4.497335,-0.562977) -- (4.497335,-0.936139);
\draw[fill=colorEMPRINQEQNDS, draw=none, opacity=1.000000] (1.115851,-2.542721) -- (7.878818,-2.542721) -- (7.878818,-4.129131) -- (1.115851,-4.129131) -- cycle;
\draw[opacity=1.000000, fill=none, line width=0.026654cm, colorEMPRINBCATDT] (1.115851,-2.542721) -- (7.878818,-2.542721) -- (7.878818,-4.129131) -- (1.115851,-4.129131) -- cycle;
\draw[opacity=1.000000, fill=none, line width=0.026654cm, colorEMPRINBCATDT] (1.353813,-2.542721) -- (7.640857,-2.542721) -- (7.640857,-4.129131) -- (1.353813,-4.129131) -- cycle;
\verbatimfont{\normalsize\fontEMPRINHQDOEK}
\node[opacity=1.000000, above right, colorEMPRINBCATDT] at(2.458034, -2.926895) {\verb|Получаем расширенную матрицу |};
\verbatimfont{\normalsize\fontEMPRINHQDOEK}
\node[opacity=1.000000, above right, colorEMPRINBCATDT] at(2.018847, -3.267910) {\verb|для решения вспомогательной задачи |};
\verbatimfont{\normalsize\fontEMPRINHQDOEK}
\node[opacity=1.000000, above right, colorEMPRINBCATDT] at(1.513740, -3.614132) {\verb|с заданным M и исходной функцией function|};
\verbatimfont{\normalsize\fontEMPRINHQDOEK}
\node[opacity=1.000000, above right, colorEMPRINBCATDT] at(2.704536, -3.969204) {\verb|(вызов getDataForAuxTask)|};
\draw[opacity=1.000000, fill=none, line width=0.015993 cm, colorEMPRINBCATDT] (4.497335,-2.169559) -- (4.497335,-2.542721);
\draw[fill=colorEMPRINQEQNDS, draw=none, opacity=1.000000] (1.847792,-4.502293) -- (7.146877,-4.502293) -- (7.146877,-5.728295) -- (1.847792,-5.728295) -- cycle;
\draw[opacity=1.000000, fill=none, line width=0.026654cm, colorEMPRINBCATDT] (1.847792,-4.502293) -- (7.146877,-4.502293) -- (7.146877,-5.728295) -- (1.847792,-5.728295) -- cycle;
\draw[opacity=1.000000, fill=none, line width=0.026654cm, colorEMPRINBCATDT] (2.031692,-4.502293) -- (6.962977,-4.502293) -- (6.962977,-5.728295) -- (2.031692,-5.728295) -- cycle;
\verbatimfont{\normalsize\fontEMPRINHQDOEK}
\node[opacity=1.000000, above right, colorEMPRINBCATDT] at(2.613236, -4.865643) {\verb|Вызываем симплекс метод на|};
\verbatimfont{\normalsize\fontEMPRINHQDOEK}
\node[opacity=1.000000, above right, colorEMPRINBCATDT] at(2.830129, -5.213296) {\verb|преобразованной матрице|};
\verbatimfont{\normalsize\fontEMPRINHQDOEK}
\node[opacity=1.000000, above right, colorEMPRINBCATDT] at(2.191620, -5.568368) {\verb|(вызов solveSimplexMethodMaxRaw)|};
\draw[opacity=1.000000, fill=none, line width=0.015993 cm, colorEMPRINBCATDT] (4.497335,-4.129131) -- (4.497335,-4.502293);
\draw[opacity=1.000000, rounded corners=0.272050 cm, fill=colorEMPRINQEQNDS, draw=none] (1.315354,-6.101457) -- (7.679315,-6.101457) -- (7.679315,-6.645557) -- (1.315354,-6.645557) -- cycle;
\draw[opacity=1.000000, rounded corners=0.272050 cm, fill=none, line width=0.026654cm, colorEMPRINBCATDT] (1.315354,-6.101457) -- (7.679315,-6.101457) -- (7.679315,-6.645557) -- (1.315354,-6.645557) -- cycle;
\verbatimfont{\normalsize\fontEMPRINHQDOEK}
\node[opacity=1.000000, above right, colorEMPRINBCATDT] at(1.587404, -6.485630) {\verb|Вернуть результат вызова симплекс метода|};
\draw[opacity=1.000000, fill=none, line width=0.015993 cm, colorEMPRINBCATDT] (4.497335,-5.728295) -- (4.497335,-6.101457);
\end{tikzpicture}
\end{lrbox}
% Здесь Вы можете поменять размер блок схемы. Оношение ширина/высота будет сохранено.
% Для изменения размеров блок схемы Вы можете изменять первый параметр resizebox, он задаёт желаемую ширину.
\resizebox{9.000000cm}{!}{\usebox{\bdgEMPRIN}}
% --------------------------

% Конец блок схемы
\end{document}