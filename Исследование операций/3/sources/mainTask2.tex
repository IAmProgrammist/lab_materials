% Созданная блок схема работает только с компиляторами XeTeX и LuaTeX.
\documentclass[../report3.tex]{subfiles}

% Необходимые зависимости
\usepackage[utf8]{inputenc}
\usepackage[english,russian]{babel}
\usepackage{pgfplots}
\usepackage{verbatim}
\usetikzlibrary{positioning}
\usetikzlibrary{shapes.geometric}
\usetikzlibrary{shapes.misc}
\usetikzlibrary{calc}
\usetikzlibrary{chains}
\usetikzlibrary{matrix}
\usetikzlibrary{decorations.text}
\usepackage{fontspec}
\usetikzlibrary{backgrounds}

\begin{document}
% Блок-схема
% Если Вы хотите добавить блок схему в свой документ, скопируйте код между комментариями
% С линияи и вставьте в документ.

% --------------------------
\newsavebox{\bdgEAJTSFK}
\begin{lrbox}{\bdgEAJTSFK}\begin{tikzpicture}[every node/.style={inner sep=0,outer sep=0}]
\makeatletter
\newcommand{\verbatimfont}[1]{\def\verbatim@font{#1}}
\makeatother
% Шрифты
\newfontfamily\fontEAJTSFKHFHDICCB[Scale=0.693004, SizeFeatures={Size=10.000000}]{Consolas}

% Цвета
\definecolor{colorEAJTSFKGIKCSJN}{rgb}{1.000000,1.000000,1.000000}
\definecolor{colorEAJTSFKLHCEQEO}{rgb}{1.000000,1.000000,1.000000}
\definecolor{colorEAJTSFKRBHFPTR}{rgb}{0.000000,0.000000,0.000000}
\draw[opacity=1.000000, rounded corners=0.225771 cm, fill=colorEAJTSFKGIKCSJN, draw=none] (2.094473,0.004890) -- (3.900637,0.004890) -- (3.900637,-0.446652) -- (2.094473,-0.446652) -- cycle;
\draw[opacity=1.000000, rounded corners=0.225771 cm, fill=none, line width=0.024448cm, colorEAJTSFKRBHFPTR] (2.094473,0.004890) -- (3.900637,0.004890) -- (3.900637,-0.446652) -- (2.094473,-0.446652) -- cycle;
\verbatimfont{\normalsize\fontEAJTSFKHFHDICCB}
\node[opacity=1.000000, above right, colorEAJTSFKRBHFPTR] at(2.606429, -0.299966) {\verb|Начало|};
\draw[fill=colorEAJTSFKGIKCSJN, draw=none, opacity=1.000000] (1.117511,-0.788918) -- (4.877600,-0.788918) -- (4.877600,-1.538916) -- (1.117511,-1.538916) -- cycle;
\draw[opacity=1.000000, fill=none, line width=0.024448cm, colorEAJTSFKRBHFPTR] (1.117511,-0.788918) -- (4.877600,-0.788918) -- (4.877600,-1.538916) -- (1.117511,-1.538916) -- cycle;
\verbatimfont{\normalsize\fontEAJTSFKHFHDICCB}
\node[opacity=1.000000, above right, colorEAJTSFKRBHFPTR] at(1.264196, -1.122184) {\verb|Подготовить входные данные|};
\verbatimfont{\normalsize\fontEAJTSFKHFHDICCB}
\node[opacity=1.000000, above right, colorEAJTSFKRBHFPTR] at(1.869896, -1.392230) {\verb|matrix и function|};
\draw[opacity=1.000000, fill=none, line width=0.014669 cm, colorEAJTSFKRBHFPTR] (2.997555,-0.446652) -- (2.997555,-0.788918);
\draw[opacity=1.000000, fill=colorEAJTSFKGIKCSJN, draw=none] (-0.004890, -2.984072) -- (0.270833, -1.881183) -- (6.000000, -1.881183) -- (5.724278, -2.984072) --cycle;
\draw[opacity=1.000000, fill=none, line width=0.024448cm, colorEAJTSFKRBHFPTR] (-0.004890, -2.984072) -- (0.270833, -1.881183) -- (6.000000, -1.881183) -- (5.724278, -2.984072) --cycle;
\verbatimfont{\normalsize\fontEAJTSFKHFHDICCB}
\node[opacity=1.000000, above right, colorEAJTSFKRBHFPTR] at(2.138904, -2.186038) {\verb|Вывести ответ|};
\verbatimfont{\normalsize\fontEAJTSFKHFHDICCB}
\node[opacity=1.000000, above right, colorEAJTSFKRBHFPTR] at(2.084947, -2.511712) {\verb|(вызов функции |};
\verbatimfont{\normalsize\fontEAJTSFKHFHDICCB}
\node[opacity=1.000000, above right, colorEAJTSFKRBHFPTR] at(0.270831, -2.837386) {\verb|solveSystemOfLinearEquationsBigPenalties)|};
\draw[opacity=1.000000, fill=none, line width=0.014669 cm, colorEAJTSFKRBHFPTR] (2.997555,-1.538916) -- (2.997555,-1.881183);
\draw[opacity=1.000000, rounded corners=0.239976 cm, fill=colorEAJTSFKGIKCSJN, draw=none] (2.037651,-3.326339) -- (3.957459,-3.326339) -- (3.957459,-3.806291) -- (2.037651,-3.806291) -- cycle;
\draw[opacity=1.000000, rounded corners=0.239976 cm, fill=none, line width=0.024448cm, colorEAJTSFKRBHFPTR] (2.037651,-3.326339) -- (3.957459,-3.326339) -- (3.957459,-3.806291) -- (2.037651,-3.806291) -- cycle;
\verbatimfont{\normalsize\fontEAJTSFKHFHDICCB}
\node[opacity=1.000000, above right, colorEAJTSFKRBHFPTR] at(2.673636, -3.659605) {\verb|Конец|};
\draw[opacity=1.000000, fill=none, line width=0.014669 cm, colorEAJTSFKRBHFPTR] (2.997555,-2.984072) -- (2.997555,-3.326339);
\end{tikzpicture}
\end{lrbox}
% Здесь Вы можете поменять размер блок схемы. Оношение ширина/высота будет сохранено.
% Для изменения размеров блок схемы Вы можете изменять первый параметр resizebox, он задаёт желаемую ширину.
\resizebox{6.000000cm}{!}{\usebox{\bdgEAJTSFK}}
% --------------------------

% Конец блок схемы
\end{document}