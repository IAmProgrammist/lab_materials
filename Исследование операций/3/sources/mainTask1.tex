% Созданная блок схема работает только с компиляторами XeTeX и LuaTeX.
\documentclass[../report3.tex]{subfiles}

% Необходимые зависимости
\usepackage[utf8]{inputenc}
\usepackage[english,russian]{babel}
\usepackage{pgfplots}
\usepackage{verbatim}
\usetikzlibrary{positioning}
\usetikzlibrary{shapes.geometric}
\usetikzlibrary{shapes.misc}
\usetikzlibrary{calc}
\usetikzlibrary{chains}
\usetikzlibrary{matrix}
\usetikzlibrary{decorations.text}
\usepackage{fontspec}
\usetikzlibrary{backgrounds}

\begin{document}
% Блок-схема
% Если Вы хотите добавить блок схему в свой документ, скопируйте код между комментариями
% С линияи и вставьте в документ.

% --------------------------
\newsavebox{\bdgKBHANB}
\begin{lrbox}{\bdgKBHANB}\begin{tikzpicture}[every node/.style={inner sep=0,outer sep=0}]
\makeatletter
\newcommand{\verbatimfont}[1]{\def\verbatim@font{#1}}
\makeatother
% Шрифты
\newfontfamily\fontKBHANBCMSNBI[Scale=0.649362, SizeFeatures={Size=10.000000}]{Consolas}

% Цвета
\definecolor{colorKBHANBALLEPHCB}{rgb}{1.000000,1.000000,1.000000}
\definecolor{colorKBHANBFALDIES}{rgb}{1.000000,1.000000,1.000000}
\definecolor{colorKBHANBLTPGOG}{rgb}{0.000000,0.000000,0.000000}
\draw[opacity=1.000000, rounded corners=0.211553 cm, fill=colorKBHANBALLEPHCB, draw=none] (2.151498,0.004582) -- (3.843920,0.004582) -- (3.843920,-0.418524) -- (2.151498,-0.418524) -- cycle;
\draw[opacity=1.000000, rounded corners=0.211553 cm, fill=none, line width=0.022908cm, colorKBHANBLTPGOG] (2.151498,0.004582) -- (3.843920,0.004582) -- (3.843920,-0.418524) -- (2.151498,-0.418524) -- cycle;
\verbatimfont{\normalsize\fontKBHANBCMSNBI}
\node[opacity=1.000000, above right, colorKBHANBLTPGOG] at(2.631214, -0.281075) {\verb|Начало|};
\draw[fill=colorKBHANBALLEPHCB, draw=none, opacity=1.000000] (1.236060,-0.739236) -- (4.759358,-0.739236) -- (4.759358,-1.442003) -- (1.236060,-1.442003) -- cycle;
\draw[opacity=1.000000, fill=none, line width=0.022908cm, colorKBHANBLTPGOG] (1.236060,-0.739236) -- (4.759358,-0.739236) -- (4.759358,-1.442003) -- (1.236060,-1.442003) -- cycle;
\verbatimfont{\normalsize\fontKBHANBCMSNBI}
\node[opacity=1.000000, above right, colorKBHANBLTPGOG] at(1.373509, -1.051515) {\verb|Подготовить входные данные|};
\verbatimfont{\normalsize\fontKBHANBCMSNBI}
\node[opacity=1.000000, above right, colorKBHANBLTPGOG] at(1.941065, -1.304555) {\verb|matrix и function|};
\draw[opacity=1.000000, fill=none, line width=0.013745 cm, colorKBHANBLTPGOG] (2.997709,-0.418524) -- (2.997709,-0.739236);
\draw[opacity=1.000000, fill=colorKBHANBALLEPHCB, draw=none] (-0.004582, -2.796151) -- (0.253777, -1.762716) -- (6.000000, -1.762716) -- (5.741641, -2.796151) --cycle;
\draw[opacity=1.000000, fill=none, line width=0.022908cm, colorKBHANBLTPGOG] (-0.004582, -2.796151) -- (0.253777, -1.762716) -- (6.000000, -1.762716) -- (5.741641, -2.796151) --cycle;
\verbatimfont{\normalsize\fontKBHANBCMSNBI}
\node[opacity=1.000000, above right, colorKBHANBLTPGOG] at(2.193131, -2.048373) {\verb|Вывести ответ|};
\verbatimfont{\normalsize\fontKBHANBCMSNBI}
\node[opacity=1.000000, above right, colorKBHANBLTPGOG] at(2.142573, -2.353538) {\verb|(вызов функции |};
\verbatimfont{\normalsize\fontKBHANBCMSNBI}
\node[opacity=1.000000, above right, colorKBHANBLTPGOG] at(0.253777, -2.658703) {\verb|solveSystemOfLinearEquationsArtificialBasis)|};
\draw[opacity=1.000000, fill=none, line width=0.013745 cm, colorKBHANBLTPGOG] (2.997709,-1.442003) -- (2.997709,-1.762716);
\draw[opacity=1.000000, rounded corners=0.224864 cm, fill=colorKBHANBALLEPHCB, draw=none] (2.098255,-3.116863) -- (3.897163,-3.116863) -- (3.897163,-3.566590) -- (2.098255,-3.566590) -- cycle;
\draw[opacity=1.000000, rounded corners=0.224864 cm, fill=none, line width=0.022908cm, colorKBHANBLTPGOG] (2.098255,-3.116863) -- (3.897163,-3.116863) -- (3.897163,-3.566590) -- (2.098255,-3.566590) -- cycle;
\verbatimfont{\normalsize\fontKBHANBCMSNBI}
\node[opacity=1.000000, above right, colorKBHANBLTPGOG] at(2.694189, -3.429142) {\verb|Конец|};
\draw[opacity=1.000000, fill=none, line width=0.013745 cm, colorKBHANBLTPGOG] (2.997709,-2.796151) -- (2.997709,-3.116863);
\end{tikzpicture}
\end{lrbox}
% Здесь Вы можете поменять размер блок схемы. Оношение ширина/высота будет сохранено.
% Для изменения размеров блок схемы Вы можете изменять первый параметр resizebox, он задаёт желаемую ширину.
\resizebox{6.000000cm}{!}{\usebox{\bdgKBHANB}}
% --------------------------

% Конец блок схемы
\end{document}