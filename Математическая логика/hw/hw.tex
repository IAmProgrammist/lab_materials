\documentclass[a4paper,14pt]{extarticle}


\usepackage[english,russian]{babel}
\usepackage[T2A]{fontenc}
\usepackage[utf8]{inputenc}
\usepackage{ragged2e}
\usepackage[utf8]{inputenc}
\usepackage{hyperref}
\usepackage{minted}
\setmintedinline{frame=lines, framesep=2mm, baselinestretch=1.5, bgcolor=LightGray, breaklines,fontsize=\scriptsize}
\setminted{frame=lines, framesep=2mm, baselinestretch=1.5, bgcolor=LightGray, breaklines,fontsize=\scriptsize}
\usepackage{xcolor}
\definecolor{LightGray}{gray}{0.9}
\usepackage{graphicx}
\usepackage[export]{adjustbox}
\usepackage[left=1cm,right=1cm, top=1cm,bottom=1cm,bindingoffset=0cm]{geometry}
\usepackage{fontspec}
\usepackage{ upgreek }
\usepackage[shortlabels]{enumitem}
\usepackage{adjustbox}
\usepackage{multirow}
\usepackage{amsmath}
\usepackage{amssymb}
\usepackage{pifont}
\usepackage{pgfplots}
\usepackage{longtable}
\usepackage{array}
\graphicspath{ {./images/} }
\makeatletter
\AddEnumerateCounter{\asbuk}{\russian@alph}{щ}
\makeatother
\setmonofont{Consolas}
\setmainfont{Times New Roman}

\newcommand\textbox[1]{
	\parbox{.45\textwidth}{#1}
} 

\newcommand{\specialcell}[2][c]{%
	\begin{tabular}[#1]{@{}c@{}}#2\end{tabular}}

\begin{document}
\pagenumbering{gobble}
\begin{center}
    \small{
        \textbf{МИНИCТЕРCТВО НАУКИ И ВЫCШЕГО ОБРАЗОВАНИЯ РОCCИЙCКОЙ ФЕДЕРАЦИИ}\\
        ФЕДЕРАЛЬНОЕ ГОCУДАРCТВЕННОЕ БЮДЖЕТНОЕ ОБРАЗОВАТЕЛЬНОЕ УЧРЕЖДЕНИЕ\\ВЫCШЕГО ОБРАЗОВАНИЯ \\
        \textbf{«БЕЛГОРОДCКИЙ ГОCУДАРCТВЕННЫЙ ТЕХНОЛОГИЧЕCКИЙ\\УНИВЕРCИТЕТ им. В. Г. ШУХОВА»\\ (БГТУ им. В.Г. Шухова)} \\
        \bigbreak
        \includegraphics[width=70mm]{log}\\
        ИНСТИТУТ ИНФОРМАЦИОННЫХ ТЕХНОЛОГИЙ И УПРАВЛЯЮЩИХ СИСТЕМ\\}
\end{center}

\vfill
\begin{center}
	\large{
		\textbf{
			РГЗ}}\\
	\normalsize{
		по дисциплине: Математическая логика и теория алгоритмов}
\end{center}
\vfill
\hfill\textbox{
	Выполнил: ст. группы ПВ-223\\Пахомов Владислав Андреевич
	\bigbreak
	Проверили: асс. Солонченко Роман\\Евгеньевич
}
\vfill\begin{center}
	Белгород \the\year г.
\end{center}
\newpage
\begin{center}
	\textbf{РГЗ}
\end{center}
\begin{enumerate}
    \item (6.3) Вывести в исчислении высказываний: $A \rightarrow \overline{\overline{A}}$. 
\begin{minted}{console}
(* Задание 6.3

Проверим на то, что A -> ~~A непротиворечиво.*)

Theorem task1: forall A: Prop, A -> ~~A.

Proof.
intros.
tauto.
Qed.

(* Так как силлогизм A -> ~~A непротиворечив, можно добавить его в правила вывода
в натуральное исчисление высказываний. Можно использовать теорему task1 
для доказательства *)

Goal forall A: Prop, A -> ~~A.

Proof.
intros.
apply task1.
exact H.
Qed.

(* Доказано. *)
\end{minted}
\item (12.12) Доказать, что формула $\vdash \overline{A \wedge \overline{A}}$ выводима в исчислении высказываний.
\begin{minted}{console}
(* Задание 12.12

Проверим то, что теорема (~(A /\ ~A)) тождественно истинна.*)

Axiom demorgan: forall A: Prop, ~A \/ A -> (~(A /\ ~A)).
Axiom noThird: forall A: Prop, True -> ~A \/ A.
Theorem task2: forall A: Prop, True -> (~(A /\ ~A)).

Proof.
intros.
apply demorgan.
apply noThird.
exact H.
Qed.

(* Теорема (~(A /\ ~A)) тождественно истинна, 
а значит можно добавить её в качестве
аксиомы в аксиоматическое исчисление высказываний и 
использовать для доказательства*)

Goal forall A: Prop, True -> ~(A /\ ~A).

Proof.
intros.
apply task2.
exact H.
Qed.

(* Доказано.*) 
\end{minted}
\item Построить выводы формулы $(A \wedge B) \rightarrow C \vdash A \rightarrow (B \rightarrow C)$ в различных исчислениях высказываний
\begin{minted}{console}
(* Задание 15.10 

((A /\ B) -> C) -> (A -> (B -> C))
*)

Goal forall A B C: Prop, ((A /\ B) -> C) -> (A -> (B -> C)).

Axiom addImpl: forall A B C: Prop, ((A /\ B) -> C) -> (A /\ B) -> C.
Axiom remConj: forall A B: Prop, A -> B -> A /\ B.

Proof.
intros.
eapply addImpl.
- exact H.
- apply remConj.
  * exact H0.
  * exact H1.
Qed.

(* Доказано в натуральном исчислении высказываний. *)
\end{minted}

\item (20.11 - задание 20.10 было заменено из-за невозможности доказать его в Coq. см. \href{https://stackoverflow.com/questions/35518303/is-this-relationship-between-forall-and-exists-provable-in-coq-intuitionistic-lo}{здесь}.) Доказать $\exists x A(x) \vdash \overline{\forall x \overline{A(x)}}$ в исчислении предикатов.

\begin{minted}{console}
(* Задание 20.11

exists A(x) -> ~(forall (~A x))
*)


Section Task4.
Variable A : nat -> Prop.
Theorem task4: (exists x, A x) -> ~(forall y, ~A y).
Proof.
  intros H1 H2.
  destruct H1 as [y H3].
  apply H2 in H3.
  contradiction.
Qed.
End Task4.

(* Доказано. *)
\end{minted}

\end{enumerate}
\end{document}