\documentclass[a4paper,14pt]{extarticle}


\usepackage[english,russian]{babel}
\usepackage[T2A]{fontenc}
\usepackage[utf8]{inputenc}
\usepackage{ragged2e}
\usepackage[utf8]{inputenc}
\usepackage{hyperref}
\usepackage[cache=false]{minted}
\setmintedinline{frame=lines, framesep=2mm, baselinestretch=1.5, bgcolor=LightGray, breaklines,fontsize=\footnotesize}
\setminted{frame=lines, framesep=2mm, baselinestretch=1.5, bgcolor=LightGray, breaklines,fontsize=\footnotesize}
\usepackage{xcolor}
\definecolor{LightGray}{gray}{0.9}
\usepackage{graphicx}
\usepackage[export]{adjustbox}
\usepackage[left=1cm,right=1cm, top=1cm,bottom=1cm,bindingoffset=0cm]{geometry}
\usepackage{fontspec}
\usepackage{ upgreek }
\usepackage[shortlabels]{enumitem}
\usepackage{adjustbox}
\usepackage{multirow}
\usepackage{amsmath}
\usepackage{amssymb}
\usepackage{pifont}
\usepackage{pgfplots}
\graphicspath{ {./images/} }
\makeatletter
\AddEnumerateCounter{\asbuk}{\russian@alph}{щ}
\makeatother
\setmonofont{Consolas}
\setmainfont{Times New Roman}

\newcommand\textbox[1]{
	\parbox{.45\textwidth}{#1}
}

\newcommand{\specialcell}[2][c]{%
	\begin{tabular}[#1]{@{}c@{}}#2\end{tabular}}

\begin{document}
\pagenumbering{gobble}
\begin{center}
	\small{
		МИНИCТЕРCТВО НАУКИ И ВЫCШЕГО ОБРАЗОВАНИЯ \\РОCCИЙCКОЙ ФЕДЕРАЦИИ
		\bigbreak
		ФЕДЕРАЛЬНОЕ ГОCУДАРCТВЕННОЕ БЮДЖЕТНОЕ ОБРАЗОВАТЕЛЬНОЕ УЧРЕЖДЕНИЕ ВЫCШЕГО ОБРАЗОВАНИЯ \\
		\bigbreak
		\textbf{«БЕЛГОРОДCКИЙ ГОCУДАРCТВЕННЫЙ \\ТЕХНОЛОГИЧЕCКИЙ УНИВЕРCИТЕТ им. В. Г. ШУХОВА»\\ (БГТУ им. В.Г. Шухова)} \\
		\bigbreak
		Кафедра программного обеспечения вычислительной техники и автоматизированных систем\\}
\end{center}

\vfill
\begin{center}
	\large{
		\textbf{
			Лабораторная работа №1}}\\
	\normalsize{
		по дисциплине: Математическая логика и теория алгоритмов \\
		тема: «Логика высказываний»}
\end{center}
\vfill
\hfill\textbox{
	Выполнил: ст. группы ПВ-223\\Пахомов Владислав Андреевич
	\bigbreak
	Проверили: ст. пр. Бондаренко Татьяна \\Владимировна
}
\vfill\begin{center}
	Белгород 2023 г.
\end{center}

\newpage
\textbf{Цель работы:} решить задачи с использованием логических высказываний, разработать программу
для нахожданиея значение формулы, представленной в ДНФ или КНФ на данной интерпретации.\\
Вариант № 10
\begin{center}\textbf{Задания к работе:}\end{center}
\begin{enumerate}[1. ]
    \item Решение предложенных в теоретической части задач.

    \item Программа на выбранном языке программирования в виде исходных кодов (с пояс-
    няющими комментариями) и в электронном варианте для демонстрации на ЭВМ.
    
    \item Спецификация программы с указанием основных структур данных и алгоритмов.
    \item Наборы тестовых данных.
\end{enumerate}
\newpage

\end{document}