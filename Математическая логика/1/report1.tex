\documentclass[a4paper,14pt]{extarticle}


\usepackage[english,russian]{babel}
\usepackage[T2A]{fontenc}
\usepackage[utf8]{inputenc}
\usepackage{ragged2e}
\usepackage[utf8]{inputenc}
\usepackage{hyperref}
\usepackage[cache=false]{minted}
\setmintedinline{frame=lines, framesep=2mm, baselinestretch=1.5, bgcolor=LightGray, breaklines,fontsize=\footnotesize}
\setminted{frame=lines, framesep=2mm, baselinestretch=1.5, bgcolor=LightGray, breaklines,fontsize=\footnotesize}
\usepackage{xcolor}
\definecolor{LightGray}{gray}{0.9}
\usepackage{graphicx}
\usepackage[export]{adjustbox}
\usepackage[left=1cm,right=1cm, top=1cm,bottom=1cm,bindingoffset=0cm]{geometry}
\usepackage{fontspec}
\usepackage{ upgreek }
\usepackage[shortlabels]{enumitem}
\usepackage{adjustbox}
\usepackage{multirow}
\usepackage{amsmath}
\usepackage{amssymb}
\usepackage{pifont}
\usepackage{pgfplots}
\graphicspath{ {./images/} }
\makeatletter
\AddEnumerateCounter{\asbuk}{\russian@alph}{щ}
\makeatother
\setmonofont{Consolas}
\setmainfont{Times New Roman}

\newcommand\textbox[1]{
	\parbox{.45\textwidth}{#1}
}

\newcommand{\specialcell}[2][c]{%
	\begin{tabular}[#1]{@{}c@{}}#2\end{tabular}}

\begin{document}
\pagenumbering{gobble}
\begin{center}
	\small{
		МИНИCТЕРCТВО НАУКИ И ВЫCШЕГО ОБРАЗОВАНИЯ \\РОCCИЙCКОЙ ФЕДЕРАЦИИ
		\bigbreak
		ФЕДЕРАЛЬНОЕ ГОCУДАРCТВЕННОЕ БЮДЖЕТНОЕ ОБРАЗОВАТЕЛЬНОЕ УЧРЕЖДЕНИЕ ВЫCШЕГО ОБРАЗОВАНИЯ \\
		\bigbreak
		\textbf{«БЕЛГОРОДCКИЙ ГОCУДАРCТВЕННЫЙ \\ТЕХНОЛОГИЧЕCКИЙ УНИВЕРCИТЕТ им. В. Г. ШУХОВА»\\ (БГТУ им. В.Г. Шухова)} \\
		\bigbreak
		Кафедра программного обеспечения вычислительной техники и автоматизированных систем\\}
\end{center}

\vfill
\begin{center}
	\large{
		\textbf{
			Лабораторная работа №1}}\\
	\normalsize{
		по дисциплине: Математическая логика и теория алгоритмов \\
		тема: «Логика высказываний»}
\end{center}
\vfill
\hfill\textbox{
	Выполнил: ст. группы ПВ-223\\Пахомов Владислав Андреевич
	\bigbreak
	Проверили: ст. пр. Бондаренко Татьяна \\Владимировна
}
\vfill\begin{center}
	Белгород 2023 г.
\end{center}

\newpage
\textbf{Цель работы:} решить задачи с использованием логических высказываний, разработать программу
для нахожданиея значение формулы, представленной в ДНФ или КНФ на данной интерпретации.\\
Вариант № 10
\begin{center}\textbf{Задания к работе:}\end{center}
\begin{enumerate}[1. ]
    \item Решение предложенных в теоретической части задач.

    \item Программа на выбранном языке программирования в виде исходных кодов (с пояс-
    няющими комментариями) и в электронном варианте для демонстрации на ЭВМ.
    
    \item Спецификация программы с указанием основных структур данных и алгоритмов.
    \item Наборы тестовых данных.
\end{enumerate}
\newpage
\textbf{Исходный код}
\textit{log\_calculator.h}
\begin{minted}{C}
#ifndef MATH_LOGIC_LOG_CALCULATOR_H
#define MATH_LOGIC_LOG_CALCULATOR_H

#include <stdbool.h>

#define INPUT_TYPE_DISJUNCTIVE_NORMAL_FORM 0
#define INPUT_TYPE_CONJUNCTIVE_NORMAL_FORM 1
#define INPUT_TYPE_INVALID (-1)

#define LITERAL_POSITIVE 1
#define LITERAL_UNDEF 0
#define LITERAL_NEGATIVE (-1)

#define LATIN_ALPHABET_LENGTH 26

#define NON '!'
#define CONJUNCTION '&' // Стрелочка вверх
#define DISJUNCTION '+' // Стрелочка вниз
#define OPENING_BRACKET '('
#define CLOSING_BRACKET ')'
#define charAlphabetIndex(val) ((val) - 'A')

typedef struct Formula {
    // Элементы формулы, конъюнкты или дизъюнкты в зависимости от type
    char **val;
    // Длина массива элементов
    int amount;
    // Вместимость массива элементов
    int capacity;
    // Тип формулы
    int type;
} Formula;

// Принимает тип формулы type, возвращает INPUT_TYPE_DISJUNCTIVE_NORMAL_FORM, если переданный тип соответствует дизъюнктивной нормальной форме,
// INPUT_TYPE_CONJUNCTIVE_NORMAL_FORM, если переданный тип соответствует конъюнктивной нормальной форме, иначе - INPUT_TYPE_INVALID.
int checkInputType(int type);

// Принимает тип формулы type и на его основе инициализирует и возвращает формулу.
// Если type невалидный, возвращает формулу с типом INPUT_TYPE_INVALID.
Formula initFormula(int type);

// Освобождает из памяти ресурсы, занятые formula
void freeFormula(Formula *formula);

// Добавляет element (конъюнкт или дизъюнкт в зависимости от типа формулы) в formula.
void addElement(Formula *formula, char element[LATIN_ALPHABET_LENGTH]);

// Выполняет парсинг формулы и сохраняет её в formula из строки line.
// Формула в конъюнктивной нормальной форме должна иметь вид "(... + ... + ...) & (... + ... + ... + ...) & (... + ... + ...)"
// Формула в дизъюнктивной нормальной форме должна иметь вид "... & ... & ... + ... & ... & ... & ... + ... & ... & ..."
// Возвращает false, если парсинг был выполнен успешно, иначе - true.
bool processFormula(Formula *formula, char* line);

// Возвращает значение формулы f при значениях переменных из val.
bool findVal(Formula f, bool val[LATIN_ALPHABET_LENGTH]);

#endif //MATH_LOGIC_LOG_CALCULATOR_H
\end{minted}
\textit{log\_calculator.c}
\begin{minted}{C}
#include <malloc.h>
#include <string.h>
#include <ctype.h>
#include <stdio.h>
#include <stdbool.h>

#include "log_calculator.h"

int checkInputType(int type)
{
    // Если тип соответствует существующим возможным типам, возращаем этот тип. 
    if (type == INPUT_TYPE_CONJUNCTIVE_NORMAL_FORM || type == INPUT_TYPE_DISJUNCTIVE_NORMAL_FORM)
        return type;

    // Иначе - невалидный тип
    return INPUT_TYPE_INVALID;
}

Formula initFormula(int type)
{
    // Если тип невалидный - возвращаем формулу с ошибочным типом
    if (checkInputType(type) == INPUT_TYPE_INVALID)
        return (Formula){NULL, 0, 0, INPUT_TYPE_INVALID};
    return (Formula){malloc(0), 0, 0, type};
}

void freeFormula(Formula *formula)
{
    // Удаляем каждый элемент формулы
    for (int i = 0; i < formula->amount; i++)
        free(formula->val[i]);

    // Сам массив формул
    free(formula->val);

    // Делаем формулу "невалидной"
    formula->val = NULL;
    formula->amount = 0;
    formula->capacity = 0;
    formula->type = INPUT_TYPE_INVALID;
}

void addElement(Formula *formula, char element[LATIN_ALPHABET_LENGTH])
{
    // Если количество больше вместимости, увеличиваем размер массива элементов в 2 раза + 1
    if (formula->amount >= formula->capacity)
        formula->val = realloc(formula->val, formula->capacity = (formula->capacity * 2 + 1));

    // Инициализируем массив переменных элемента формулы
    formula->val[formula->amount++] = malloc(sizeof(char) * LATIN_ALPHABET_LENGTH);
    // Копируем переменные в элемент формулы
    memcpy(formula->val[formula->amount - 1], element, sizeof(char) * LATIN_ALPHABET_LENGTH);
}

static bool _processFormulaDisjunctive(Formula *formula, char *line)
{
    // ... & ... & ... + ... & ... & ... & ... + ... & ... & ...

    int presence = LITERAL_UNDEF;
    int letterIndex = -1;
    char element[LATIN_ALPHABET_LENGTH] = {};
    int index = 0;
    while (1)
    {
        char input = *line;
        if (isspace(input))
        {
            // Игнорируем, таким образом пробелы, табуляции и т.д. не будут иметь значения.
        }
        else if (input == NON)
        {
            // Отрицание должно стоять до переменной, а не после.
            if (letterIndex != -1)
            {
                fprintf(stderr, "Parsing error at index %d\n", index);
                return 1;
            }
            
            // Если отрицаний не было, значит переменная стоит с знаком !.
            // Если отрицание было, то !! = отсутствие отрицаний.
            presence = presence == LITERAL_UNDEF ? LITERAL_NEGATIVE : presence == LITERAL_POSITIVE ? LITERAL_NEGATIVE
                                                                                                   : LITERAL_POSITIVE;
        }
        else if (input == CONJUNCTION)
        {
            // Исключение ситуации, подобной "&&".
            if (letterIndex == -1)
            {
                fprintf(stderr, "No literal present at index %d\n", index);
                return 1;
            }

            // Если встречаем &, сохраняем предыдущую переменную в элемент
            element[letterIndex] = presence == LITERAL_UNDEF ? LITERAL_POSITIVE : presence;
            presence = LITERAL_UNDEF;
            letterIndex = -1;
        }
        else if (input == DISJUNCTION)
        {
            // Исключение ситуации, подобной "++".
            if (letterIndex == -1)
            {
                fprintf(stderr, "No literal present at index %d\n", index);
                return 1;
            }

            // Если встречаем +, сохраняем предыдущую переменную в элемент
            element[letterIndex] = presence == LITERAL_UNDEF ? LITERAL_POSITIVE : presence;
            presence = LITERAL_UNDEF;
            letterIndex = -1;
            
            // А также сам элемент в формулу. Также обновляем буфферный элемент element.
            addElement(formula, element);
            memset(element, LITERAL_UNDEF, LATIN_ALPHABET_LENGTH);
            
        }
        else if (isupper(input) || islower(input))
        {
            // Найдена буква латинского алфавита. Выполняем дополнииельные проверки и преобразования так, чтобы 'x' == 'X'.
            letterIndex = charAlphabetIndex(islower(input) ? toupper(input) : input);
        }
        else if (input == '\0')
        {
            // Если дошли до конца строки, сохраняем переменную в элемент, элемент в формулу.
            if (letterIndex == -1)
            {
                fprintf(stderr, "No literal present at index %d\n", index);
                return 1;
            }

            element[letterIndex] = presence == LITERAL_UNDEF ? LITERAL_POSITIVE : presence;

            addElement(formula, element);
            memset(element, LITERAL_UNDEF, LATIN_ALPHABET_LENGTH);

            // Заканчиваем цикл чтения.
            break;
        }
        else
        {
            fprintf(stderr, "Unknown character '%c' at index %d\n", input, index);
            return 1;
        }

        line++, index++;
    }

    return 0;
}

static bool _processFormulaConjunctive(Formula *formula, char *line)
{
    // (... + ... + ...) & (... + ... + ... + ...) & (... + ... + ...)

    int bracesCount = 0;
    int presence = LITERAL_UNDEF;
    int letterIndex = -1;
    char element[LATIN_ALPHABET_LENGTH] = {};
    bool elementEmpty = true;
    int index = 0;
    while (1)
    {
        char input = *line;
        if (isspace(input))
        {
            // Игнорируем, таким образом пробелы, табуляции и т.д. не будут иметь значения.
        }
        else if (input == OPENING_BRACKET)
        {
            // Открывающая скобка возможна, если степень вложенности равна 0, она не содержится внутри конъюнкта, конъюнкт пустой. 
            if (presence != LITERAL_UNDEF || letterIndex != -1 || bracesCount != 0 || !elementEmpty)
            {
                fprintf(stderr, "Parsing error at index %d\n", index);
                return 1;
            }

            bracesCount++;
        }
        else if (input == CLOSING_BRACKET)
        {
            // Открывающая скобка возможна, если степень вложенности равна 1, а также конъюнкт не пустой.
            if (letterIndex == -1 || bracesCount != 1)
            {
                fprintf(stderr, "Parsing error at index %d\n", index);
                return 1;
            }

            // Сохраняем переменную в элемент.
            element[letterIndex] = presence == LITERAL_UNDEF ? LITERAL_POSITIVE : presence;
            presence = LITERAL_UNDEF;
            letterIndex = -1;
            bracesCount--;
            elementEmpty = false;
        }
        else if (input == NON)
        {
            // Отрицание должно стоять до переменной, а не после.
            if (letterIndex != -1)
            {
                fprintf(stderr, "Parsing error at index %d\n", index);
                return 1;
            }

            // Если отрицаний не было, значит переменная стоит с знаком !.
            // Если отрицание было, то !! = отсутствие отрицаний.
            presence = presence == LITERAL_UNDEF ? LITERAL_NEGATIVE : presence == LITERAL_POSITIVE ? LITERAL_NEGATIVE
                                                                                                   : LITERAL_POSITIVE;
        }
        else if (input == CONJUNCTION)
        {
            // Встретили &. Если вложенность равна 0, записываем элемент в формулу.
            if (bracesCount == 0)
            {
                addElement(formula, element);
                memset(element, LITERAL_UNDEF, LATIN_ALPHABET_LENGTH);
                elementEmpty = true;
            }
            else
            {
                fprintf(stderr, "Brackets error at index %d\n", index);
                return 1;
            }
        }
        else if (input == DISJUNCTION)
        {
            // Встретили +. Сохраняем переменную в элемент.
            if (letterIndex == -1)
            {
                fprintf(stderr, "No literal present at index %d\n", index);
                return 1;
            }

            element[letterIndex] = presence == LITERAL_UNDEF ? LITERAL_POSITIVE : presence;
            presence = LITERAL_UNDEF;
            letterIndex = -1;
            elementEmpty = false;
        }
        else if (isupper(input) || islower(input))
        {
            // Найдена буква латинского алфавита. Выполняем дополнииельные проверки и преобразования так, чтобы 'x' == 'X'.
            letterIndex = charAlphabetIndex(islower(input) ? toupper(input) : input);
        }
        else if (input == '\0')
        {
            // Дошли до конца строки. Записываем элемент в формулу. Проверяем, что степень вложенности == 0.
            if (bracesCount == 0)
            {
                addElement(formula, element);
                memset(element, LITERAL_UNDEF, LATIN_ALPHABET_LENGTH);
            }
            else
            {
                fprintf(stderr, "Brackets error at index %d\n", index);
                return 1;
            }

            // Заканчиваем цикл чтения.
            break;
        }
        else
        {
            fprintf(stderr, "Unknown character '%c' at index %d\n", input, index);
            return 1;
        }

        line++, index++;
    }

    return 0;
}

bool processFormula(Formula *formula, char *line)
{
    // В зависимости от типа формулы выполняем парсинг при помощи соответствующей функции.
    int type = checkInputType(formula->type);
    if (type == INPUT_TYPE_INVALID)
        return 1;

    if (type == INPUT_TYPE_DISJUNCTIVE_NORMAL_FORM)
    {
        return _processFormulaDisjunctive(formula, line);
    }

    if (type == INPUT_TYPE_CONJUNCTIVE_NORMAL_FORM)
    {
        return _processFormulaConjunctive(formula, line);
    }
}

static bool _findValConjunctive(Formula f, bool val[LATIN_ALPHABET_LENGTH]) {
    // (... + ... + ...) & (... + ... + ... + ...) & (... + ... + ...)
    // Вычисления выполняются по ленивой схеме.
    
    for (int i = 0; i < f.amount; i++)
    {
        char *element = f.val[i];
        // Предположим, что результат элемента ... + ... + ... - false.
        bool elementResult = false;

        for (int j = 0; j < LATIN_ALPHABET_LENGTH; j++)
        {
            // Если переменной нет в формуле, пропускаем её.
            if (element[j] == LITERAL_UNDEF)
                continue;

            // Вычисляем значение элемента.
            int current = (element[j] == LITERAL_POSITIVE) ? val[j] : !val[j];

            // Если значение переменной true - присваиваем всему элементу true и заканчиваем цикл, дальнейшее вычисление не нужно.
            if (current)
            {
                elementResult = true;
                break;
            }
        }

        // Если элемент формулы false, то формула вида ... & ... & ... тоже будет иметь результат false 
        if (!elementResult)
            return false;
    }

    // Все элементы ... & ... & ... true, формула тоже true.
    return true;
}

static bool _findValDisjunctive(Formula f, bool val[LATIN_ALPHABET_LENGTH]) {
    // ... & ... & ... + ... & ... & ... & ... + ... & ... & ...
    // Вычисления выполняются по ленивой схеме.
    for (int i = 0; i < f.amount; i++)
    {
        char *element = f.val[i];
        // Предположим, что результат элемента ... & ... & ... - true.
        bool elementResult = true;

        for (int j = 0; j < LATIN_ALPHABET_LENGTH; j++)
        {
            // Если переменной нет в формуле, пропускаем её.
            if (element[j] == LITERAL_UNDEF)
                continue;

            // Вычисляем значение элемента.
            int current = (element[j] == LITERAL_POSITIVE) ? val[j] : !val[j];
            // Если значение переменной false - присваиваем всему элементу false и заканчиваем цикл, дальнейшее вычисление не нужно.
            if (!current)
            {
                elementResult = false;
                break;
            }
        }

        // Если элемент формулы true, то формула вида ... + ... + ... тоже будет иметь результат true 
        if (elementResult)
            return true;
    }

    // Все элементы оказались false, возвращаем false.
    return false;
}

bool findVal(Formula formula, bool args[LATIN_ALPHABET_LENGTH])
{
    // В зависимости от типа формулы выполняем вычисление значения формулы.
    int type = checkInputType(formula.type);
    if (type == INPUT_TYPE_INVALID)
        return false;

    if (type == INPUT_TYPE_DISJUNCTIVE_NORMAL_FORM)
    {
        return _findValDisjunctive(formula, args);
    }

    if (type == INPUT_TYPE_CONJUNCTIVE_NORMAL_FORM)
    {
        return _findValConjunctive(formula, args);
    }

    return false;
}
\end{minted}
\textit{main.c}
\begin{minted}{C}
#include <stdio.h>
#include <string.h>
#include <ctype.h>

#include "../libs/alg/lab1/log_calculator.h"

int main()
{
    int type = INPUT_TYPE_INVALID;
    while (checkInputType(type) == INPUT_TYPE_INVALID) {
        printf("Select normal form type (0 - disjunctive, 1 - conjunctive): ");
        scanf("%d", &type);
    }

    char buf[512];
    gets(buf);
    Formula f;
    do {
    printf("Enter formula: ");
        f = initFormula(type);

        gets(buf);
    } while(processFormula(&f, buf));

    bool isPresent[LATIN_ALPHABET_LENGTH] = {};
    for (int i = 0; buf[i] != '\0'; i++) {
        if (islower(buf[i]))
            isPresent[buf[i] - 'a'] = true;
        if (isupper(buf[i]))
            isPresent[buf[i] - 'A'] = true;
    }

    printf("Enter variables:\n");
    for (int i = 0; i < LATIN_ALPHABET_LENGTH; i++) {
        if (!isPresent[i]) continue;

        printf("%c ", i + 'A');
    }
    printf("\n");

    bool v[LATIN_ALPHABET_LENGTH];
    for (int i = 0; i < LATIN_ALPHABET_LENGTH; i++) {
        if (!isPresent[i]) continue;
        
        scanf("%d", v + i);
    }
        
    printf("Result: %d", findVal(f, v));

    return 0;
}
\end{minted}
\newpage
\textbf{Тесты}\\
\textit{check\_input\_type.c}
\begin{minted}{C}
#include "../../src/libs/alg/lab1/log_calculator.h"

#include <assert.h>

int main() {
    assert(checkInputType(0) == INPUT_TYPE_DISJUNCTIVE_NORMAL_FORM);
    assert(checkInputType(1) == INPUT_TYPE_CONJUNCTIVE_NORMAL_FORM);
    assert(checkInputType(99) == INPUT_TYPE_INVALID);
}
\end{minted}
\textit{init\_formula.c}
\begin{minted}{C}
#include <assert.h>
#include <string.h>
#include "../../src/libs/alg/lab1/log_calculator.h"

int main() {
    Formula formula = initFormula(INPUT_TYPE_DISJUNCTIVE_NORMAL_FORM);
    assert(formula.val != NULL);
    assert(formula.amount == 0);
    assert(formula.capacity == 0);
    assert(formula.type == INPUT_TYPE_DISJUNCTIVE_NORMAL_FORM);

    Formula formula2 = initFormula(INPUT_TYPE_CONJUNCTIVE_NORMAL_FORM);
    assert(formula2.val != NULL);
    assert(formula2.amount == 0);
    assert(formula2.capacity == 0);
    assert(formula2.type == INPUT_TYPE_CONJUNCTIVE_NORMAL_FORM);

    Formula formula3 = initFormula(31293);
    assert(formula3.val == NULL);
    assert(formula3.amount == 0);
    assert(formula3.capacity == 0);
    assert(formula3.type == INPUT_TYPE_INVALID);

    return 0;
}
\end{minted}
\textit{add\_element.c}
\begin{minted}{C}
#include <assert.h>
#include <string.h>
#include "../../src/libs/alg/lab1/log_calculator.h"

int main() {
    Formula formula = initFormula(INPUT_TYPE_DISJUNCTIVE_NORMAL_FORM);
    char element[LATIN_ALPHABET_LENGTH] = {LITERAL_POSITIVE, LITERAL_NEGATIVE, 99, LITERAL_NEGATIVE};

    addElement(&formula, element);
    addElement(&formula, element);

    assert(formula.amount == 2);
    assert(formula.capacity >= formula.amount);
    assert(!strncmp(formula.val[0], element, LATIN_ALPHABET_LENGTH * sizeof(char)));
    assert(!strncmp(formula.val[1], element, LATIN_ALPHABET_LENGTH * sizeof(char)));

    return 0;
}
\end{minted}
\textit{conjunctive\_formula.c}
\begin{minted}{C}
#include "../../src/libs/alg/lab1/log_calculator.h"

#include <assert.h>

bool formula(bool val[LATIN_ALPHABET_LENGTH])
{
    return (!val[0] || val[1]) && (val[0] + !val[1]);
}

int main()
{
    Formula f = initFormula(INPUT_TYPE_CONJUNCTIVE_NORMAL_FORM);
    processFormula(&f, "(!a + b) & (a + !b)");
    for (int i = 0; i < (2 << 26); i++)
    {
        bool val[LATIN_ALPHABET_LENGTH];

        for (int j = 0; j < LATIN_ALPHABET_LENGTH; j++)
        {
            val[j] = !!(i & (1 << j));
        }

        assert(findVal(f, val) == formula(val));
    }
}
\end{minted}
\newpage
\textit{conjunctive\_formula\_stress.c}
\begin{minted}{C}
#include "../../src/libs/alg/lab1/log_calculator.h"

#include <assert.h>

bool formula(bool val[LATIN_ALPHABET_LENGTH])
{
    return (val[15] || val[17] || !val[8] || val[14] || val[16] || !val[4] || val[22] || val[20] || val[3] || !val[7] || !val[2]) && (val[4] || val[7] || !val[20] || val[21] || val[5] || val[25] || val[18] || val[17] || val[3] || val[15] || val[11] || !val[1] || val[9] || !val[2] || !val[8] || !val[22] || val[10] || !val[19] || !val[14] || val[23] || !val[13] || !val[12] || val[6]) && (!val[6] || !val[11] || !val[16] || val[4] || !val[18] || val[24] || !val[15] || !val[5] || val[12] || val[9] || !val[14] || !val[25]) && (val[3] || !val[10] || val[20] || !val[6] || val[8] || val[9] || val[17] || val[13] || !val[5] || !val[21] || !val[11] || val[16] || !val[7] || val[23] || val[14] || val[24] || !val[0] || !val[2]);
}

int main()
{
    Formula f = initFormula(INPUT_TYPE_CONJUNCTIVE_NORMAL_FORM);
    processFormula(&f, "      (     !   !  !    !    !   ! p  +  !   !   R   +   !     !   !      !      ! i  +   !    !    !    !   O +     !   !   Q  +    !     !  !   e   +    !   !   W +     !    !   u   +  D  +    !    !     !      !    !  H +     !      !    !  C       )   &    (     !  !    !    !     !      !   E   + h  +   !     !   !     !  !    !   !    !     !  u  +   !     !    !  !    !  ! V +    !     !     !   !    !    !    !      ! f  +    !   ! Z  +      !    !   !   !    !    !   S  +    !    !  !     !   !    !   r   +   !     !      !    !    !   !    !   ! d   +   !    !    !  !     !   !     !    ! P   +   !   !   L  +    !    !     !   !   ! B  +  J   +  !      !   !    !    !  C +   !   !    !    !     !  !   !  I  +  !     !  !      !   !    !  !   w   +  !    !     !    ! k +     !    !    !  !     !  !    ! T  +    !    !  !    !     !  O  +     !     ! x   +      !   N +     !     !   !   !      !   M  +    !     !    !  !     !    ! g       )   &     (   ! G   +     !      !   !   l +      !    !     !  !      !   q +   !    !    !    !    !   ! e +   !     !    ! S  +     !   !   !   !    !     !   y  +     !     !    !     !  !  P +    !   !  !    !    !     !  ! f   +    !   !    !     !  m  +  !    !  J  +      !   !      !      !  !   O +    !   !    !      !     !  !      !   !     !   z       )   &      (    !    !   D  +    !    !  !    !    !     !   ! K  +   u   +   !    !    !  g +     !     !     !   !      !    !     !   ! i  +   !     !      !    !    !    !  J   +    !     !     !   !     !    !  R   +    !    !    !     ! n +   !  !      !   f  +     !  !     !    !  !     !    !   v +   !      !    !    !   !      !   !  L +   Q  +   !   !    !      !     !  H + X  +    !    !   !     ! O +  y  +     !     !    !      !    !   !  ! a  +    !      !      !    !     !  C   )  ");
    for (int i = 0; i < (2 << 26); i++)
    {
        bool val[LATIN_ALPHABET_LENGTH];

        for (int j = 0; j < LATIN_ALPHABET_LENGTH; j++)
        {
            val[j] = !!(i & (1 << j));
        }

        assert(findVal(f, val) == formula(val));
    }
}
\end{minted}
\textit{disjunctive\_formula.c}
\begin{minted}{C}
#include "../../src/libs/alg/lab1/log_calculator.h"

#include <assert.h>

bool formula(bool val[LATIN_ALPHABET_LENGTH])
{
    return !val[0] || val[1];
}

int main()
{
    Formula f = initFormula(INPUT_TYPE_DISJUNCTIVE_NORMAL_FORM);
    processFormula(&f, "!a + b");
    for (int i = 0; i < (2 << 26); i++)
    {
        bool val[LATIN_ALPHABET_LENGTH];

        for (int j = 0; j < LATIN_ALPHABET_LENGTH; j++)
        {
            val[j] = !!(i & (1 << j));
        }

        assert(findVal(f, val) == formula(val));
    }
}
\end{minted}
\newpage
\textit{disjunctive\_formula\_stress.c}
\begin{minted}{C}
#include "../../src/libs/alg/lab1/log_calculator.h"

#include <assert.h>

bool formula(bool val[LATIN_ALPHABET_LENGTH])
{
    return !val[7] && val[12] && val[19] && val[1] && val[5] && val[10] && !val[8] && val[17] && val[23] && val[3] && !val[20] && !val[4] && val[25] && val[2] || val[6] && !val[16] && val[22] && !val[12] && !val[25] && !val[5] || !val[17] && val[18] && val[1] && val[11] && val[21] && !val[2] && val[7] && !val[4] && !val[13] && !val[3] && val[8] && !val[15] && val[9] && !val[23] && !val[22] && val[12] && val[0] && val[10] && val[24] && val[5] && val[25] && !val[19] && val[20] && !val[6] || val[15] && !val[24] && val[5] && !val[21] && !val[18] && !val[12] && !val[11] && val[23] && !val[14] && val[0] && val[16];
}

int main()
{
    Formula f = initFormula(INPUT_TYPE_DISJUNCTIVE_NORMAL_FORM);
    processFormula(&f, "  !      !    !   !     !      !      !   H &   !  !    !  !   !    !    !   ! M &    !   !    !     !    !    !  T &   !    !     !  ! b & F  &    !    !   !     !     !   !  K  &      !    !   !     !   !   !   !  i   &   !     !   R   & x  & d  &   !   !      !     !    !  !    !    !     ! U  &     !  e & z &   !      !   C      +    !     !      !    !  G &      !    !      !     !     !      !      !    !   !  Q  &   !    !   !  !     !   !    !  !  w   &     !    !  !    !   !     !    !      !      ! M &    !   !      !  !     !    !     !  Z   &    !  !     !  !   !    !     !   !  !   F       +      !     !     ! r &  !  !    !     !  S  &   !    !     !    !    !   !  b &    !    !  l  &    !  !  !    !    !     ! v &    !     !     !  !    !   !      ! c   &  !   !  !     ! H  &    !   e &      !    !    !     !     !     !    !   n &   !      !    !      !   !   !    !    !    ! D  &   !   !    !    ! I   &     !     !   !     !  !    !    !   p   &  !    !    !    !    !     !   J   &      !     !    !   !    !  !    !    !    !   x &    !   !   !     !     !     !     !   !    ! w &   !     !   !      !   M   &     !   !  A &     !  ! k   & y &    !     !   !    !     !    !  !   ! f   &     !     !    !   !   Z  &   !      !   !   !    !  !    !   !    ! T   &   u &   !   g       +     !  !    !     !     !   !    !     ! p   &    !     !    !  !   !   !      !  !  !  Y   &    !     !     !      !     !   !      !     ! f   &     !  !     !   V   &  !  S  &  !    !   !   !   !   !    !      !    !  m   &   !    !    !    !    !  l  &     !    !  x &     !  O   &   A &     !   !     !   !    !     !     !  !   Q           ");
    for (int i = 0; i < (2 << 26); i++)
    {
        bool val[LATIN_ALPHABET_LENGTH];

        for (int j = 0; j < LATIN_ALPHABET_LENGTH; j++)
        {
            val[j] = !!(i & (1 << j));
        }

        assert(findVal(f, val) == formula(val));
    }
}
\end{minted}
Результат выполнения тестов: \\
\includegraphics[width=100mm]{testsresult}\\
\textbf{Вывод: }в ходе выполнения лабораторной работы решили задачи с использованием логических высказываний, разработали программу
для нахожданиея значение формулы, представленной в ДНФ или КНФ на данной интерпретации.\\
\end{document}