\documentclass[a4paper,14pt]{extarticle}


\usepackage[english,russian]{babel}
\usepackage[T2A]{fontenc}
\usepackage[utf8]{inputenc}
\usepackage{ragged2e}
\usepackage[utf8]{inputenc}
\usepackage{hyperref}
\usepackage{minted}
\setmintedinline{frame=lines, framesep=2mm, baselinestretch=1.5, bgcolor=LightGray, breaklines,fontsize=\scriptsize}
\setminted{frame=lines, framesep=2mm, baselinestretch=1.5, bgcolor=LightGray, breaklines,fontsize=\scriptsize}
\usepackage[table]{xcolor}
\definecolor{LightGray}{gray}{0.9}
\definecolor{Yellow}{rgb}{1,1,0}
\usepackage{graphicx}
\usepackage[export]{adjustbox}
\usepackage[left=1cm,right=1cm, top=1cm,bottom=1cm,bindingoffset=0cm]{geometry}
\usepackage{fontspec}
\usepackage{ upgreek }
\usepackage[shortlabels]{enumitem}
\usepackage{adjustbox}
\usepackage{multirow}
\usepackage{amsmath}
\usepackage{amssymb}
\usepackage{pifont}
\usepackage{pgfplots}
\usepackage{longtable}
\usepackage{array}

\graphicspath{ {./images/} }
\makeatletter
\AddEnumerateCounter{\asbuk}{\russian@alph}{щ}
\makeatother
\setmonofont{Consolas}
\setmainfont{Times New Roman}

\newcommand\textbox[1]{
	\parbox{.45\textwidth}{#1}
} 

\newcommand{\specialcell}[2][c]{%
	\begin{tabular}[#1]{@{}c@{}}#2\end{tabular}}

\begin{document}
\pagenumbering{gobble}
\begin{center}
    \small{
        \textbf{МИНИCТЕРCТВО НАУКИ И ВЫCШЕГО ОБРАЗОВАНИЯ РОCCИЙCКОЙ ФЕДЕРАЦИИ}\\
        ФЕДЕРАЛЬНОЕ ГОCУДАРCТВЕННОЕ БЮДЖЕТНОЕ ОБРАЗОВАТЕЛЬНОЕ УЧРЕЖДЕНИЕ\\ВЫCШЕГО ОБРАЗОВАНИЯ \\
        \textbf{«БЕЛГОРОДCКИЙ ГОCУДАРCТВЕННЫЙ ТЕХНОЛОГИЧЕCКИЙ\\УНИВЕРCИТЕТ им. В. Г. ШУХОВА»\\ (БГТУ им. В.Г. Шухова)} \\
        \bigbreak
        \includegraphics[width=70mm]{log}\\
        ИНСТИТУТ ИНФОРМАЦИОННЫХ ТЕХНОЛОГИЙ И УПРАВЛЯЮЩИХ СИСТЕМ\\}
\end{center}

\vfill
\begin{center}
    \large{
        \textbf{
            Лабораторная работа №1}}\\
    \normalsize{
        по дисциплине: Системное моделирование \\
        тема: «Поведение механических систем в статике»}
\end{center}
\vfill
\hfill\textbox{
    Выполнил: ст. группы ПВ-223\\Пахомов Владислав Андреевич
    \bigbreak
    Проверил: Полунин Александр Иванович
}
\vfill\begin{center}
    Белгород 2024 г.
\end{center}
\newpage
\begin{center}
    \textbf{Лабораторная работа №1}\\
    Поведение механических систем в статике\\
    Вариант 10
\end{center}
\textbf{Цель работы: }научиться моделировать на примере моделирования поведения механической системы в статике.
\begin{enumerate}[1. ]
    \item Разработать математическую модель, описывающую поведение элементов механической системы в статике.\\
          \begin{center}
              \includegraphics[width=140mm]{variant.png}
              \bigbreak
              \includegraphics[width=140mm]{mod1.jpg}
              \includegraphics[width=140mm]{mod2.jpg}
          \end{center}
    \item Разработать программу на основании математической модели и произвести расчёты.
          \begin{minted}{C++}
#include <algorithm>
#include <iostream>
#include <iomanip>

#define PI 3.141592654
#define g 9.81
#define k1 5000.0
#define k2 7000.0
#define I 5.0
#define R 1.0
#define r 0.2
#define m 10.0
#define es 0.0000000001

struct result {
    double angle;
    double P;
};

result getAngle(double P) {
    double angle = (180.0 * R * (m * g + P)) / (PI * r * r);

    return (result) {angle, P };   
}

double getY(double angle, double P) {
    return (angle * PI * R) / 180.0 + (m * g + P) / k2;
}

int main() {
    double P = 0;
    double stepP = 100.0;

    std::cout << std::setw(15) << "P" << "    " << std::setw(15) << "angle" << "    " << std::setw(15) << "Y" << std::endl;
    while (stepP > es) {
        result res = getAngle(-P);
        
        std::cout << std::setw(15) << res.P << "    " << std::setw(15) << res.angle << "    " << std::setw(15) << getY(res.angle, res.P) << std::endl;

        if (res.angle > 0)
            P += stepP;
        else {
            P -= stepP;
            stepP /= 2;
            P += stepP;
        } 
    }
}
    \end{minted}
          Результаты выполнения программы:

          \begin{center}
              \begin{longtable}{|c|c|c|}
                  \hline
                  $P$      & $\alpha$     & $y$          \\
                  \hline
                  -0       & 140518       & 2452.51      \\
                  \hline
                  -100     & -2721.55     & -47.5003     \\
                  \hline
                  -50      & 68898.2      & 1202.51      \\
                  \hline
                  -100     & -2721.55     & -47.5003     \\
                  \hline
                  -75      & 33088.3      & 577.503      \\
                  \hline
                  -100     & -2721.55     & -47.5003     \\
                  \hline
                  -87.5    & 15183.4      & 265.002      \\
                  \hline
                  -100     & -2721.55     & -47.5003     \\
                  \hline
                  -93.75   & 6230.92      & 108.751      \\
                  \hline
                  -100     & -2721.55     & -47.5003     \\
                  \hline
                  -96.875  & 1754.68      & 30.6252      \\
                  \hline
                  -100     & -2721.55     & -47.5003     \\
                  \hline
                  -98.4375 & -483.433     & -8.43755     \\
                  \hline
                  -97.6562 & 635.625      & 11.0938      \\
                  \hline
                  -98.4375 & -483.433     & -8.43755     \\
                  \hline
                  -98.0469 & 76.096       & 1.32813      \\
                  \hline
                  -98.4375 & -483.433     & -8.43755     \\
                  \hline
                  -98.2422 & -203.669     & -3.55471     \\
                  \hline
                  -98.1445 & -63.7863     & -1.11329     \\
                  \hline
                  -98.0957 & 6.15482      & 0.107422     \\
                  \hline
                  -98.1445 & -63.7863     & -1.11329     \\
                  \hline
                  -98.1201 & -28.8157     & -0.502933    \\
                  \hline
                  -98.1079 & -11.3305     & -0.197755    \\
                  \hline
                  -98.1018 & -2.58782     & -0.0451663   \\
                  \hline
                  -98.0988 & 1.7835       & 0.0311281    \\
                  \hline
                  -98.1018 & -2.58782     & -0.0451663   \\
                  \hline
                  -98.1003 & -0.402162    & -0.00701908  \\
                  \hline
                  -98.0995 & 0.690669     & 0.0120545    \\
                  \hline
                  -98.1003 & -0.402162    & -0.00701908  \\
                  \hline
                  -98.0999 & 0.144254     & 0.00251771   \\
                  \hline
                  -98.1003 & -0.402162    & -0.00701908  \\
                  \hline
                  -98.1001 & -0.128954    & -0.00225068  \\
                  \hline
                  -98.1    & 0.00764981   & 0.000133515  \\
                  \hline
                  -98.1001 & -0.128954    & -0.00225068  \\
                  \hline
                  -98.1    & -0.0606521   & -0.00105858  \\
                  \hline
                  -98.1    & -0.0265011   & -0.000462535 \\
                  \hline
                  -98.1    & -0.00942566  & -0.00016451  \\
                  \hline
                  -98.1    & -0.000887925 & -1.54973e-05 \\
                  \hline
                  -98.1    & 0.00338094   & 5.90089e-05  \\
                  \hline
                  -98.1    & -0.000887925 & -1.54973e-05 \\
                  \hline
                  -98.1    & 0.00124651   & 2.17558e-05  \\
                  \hline
                  -98.1    & -0.000887925 & -1.54973e-05 \\
                  \hline
                  -98.1    & 0.000179292  & 3.12926e-06  \\
                  \hline
                  -98.1    & -0.000887925 & -1.54973e-05 \\
                  \hline
                  -98.1    & -0.000354316 & -6.18402e-06 \\
                  \hline
                  -98.1    & -8.75118e-05 & -1.52738e-06 \\
                  \hline
                  -98.1    & 4.58903e-05  & 8.00942e-07  \\
                  \hline
                  -98.1    & -8.75118e-05 & -1.52738e-06 \\
                  \hline
                  -98.1    & -2.08107e-05 & -3.63218e-07 \\
                  \hline
                  -98.1    & 1.25398e-05  & 2.18862e-07  \\
                  \hline
                  -98.1    & -2.08107e-05 & -3.63218e-07 \\
                  \hline
                  -98.1    & -4.13545e-06 & -7.21777e-08 \\
                  \hline
                  -98.1    & 4.20218e-06  & 7.33423e-08  \\
                  \hline
                  -98.1    & -4.13545e-06 & -7.21777e-08 \\
                  \hline
                  -98.1    & 3.33627e-08  & 5.82293e-10  \\
                  \hline
                  -98.1    & -4.13545e-06 & -7.21777e-08 \\
                  \hline
                  -98.1    & -2.05105e-06 & -3.57977e-08 \\
                  \hline
                  -98.1    & -1.00884e-06 & -1.76077e-08 \\
                  \hline
                  -98.1    & -4.87739e-07 & -8.51271e-09 \\
                  \hline
                  \rowcolor{Yellow}
                  -98.1    & -2.27188e-07 & -3.96521e-09 \\
                  \hline
              \end{longtable}
          \end{center}

\end{enumerate}

\textbf{Вывод: } в ходе лабораторной работы изучили основные шаги моделирования,
промоделировали поведение механической системы в статике.

\end{document}