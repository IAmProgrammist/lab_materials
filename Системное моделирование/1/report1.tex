\documentclass[a4paper,14pt]{extarticle}


\usepackage[english,russian]{babel}
\usepackage[T2A]{fontenc}
\usepackage[utf8]{inputenc}
\usepackage{ragged2e}
\usepackage[utf8]{inputenc}
\usepackage{hyperref}
\usepackage{minted}
\setmintedinline{frame=lines, framesep=2mm, baselinestretch=1.5, bgcolor=LightGray, breaklines,fontsize=\scriptsize}
\setminted{frame=lines, framesep=2mm, baselinestretch=1.5, bgcolor=LightGray, breaklines,fontsize=\scriptsize}
\usepackage[table]{xcolor}
\definecolor{LightGray}{gray}{0.9}
\definecolor{Yellow}{rgb}{1,1,0}
\usepackage{graphicx}
\usepackage[export]{adjustbox}
\usepackage[left=1cm,right=1cm, top=1cm,bottom=1cm,bindingoffset=0cm]{geometry}
\usepackage{fontspec}
\usepackage{ upgreek }
\usepackage[shortlabels]{enumitem}
\usepackage{adjustbox}
\usepackage{multirow}
\usepackage{amsmath}
\usepackage{amssymb}
\usepackage{pifont}
\usepackage{pgfplots}
\usepackage{longtable}
\usepackage{array}

\graphicspath{ {./images/} }
\makeatletter
\AddEnumerateCounter{\asbuk}{\russian@alph}{щ}
\makeatother
\setmonofont{Consolas}
\setmainfont{Times New Roman}

\newcommand\textbox[1]{
	\parbox{.45\textwidth}{#1}
} 

\newcommand{\specialcell}[2][c]{%
	\begin{tabular}[#1]{@{}c@{}}#2\end{tabular}}

\begin{document}
\pagenumbering{gobble}
\begin{center}
    \small{
        \textbf{МИНИCТЕРCТВО НАУКИ И ВЫCШЕГО ОБРАЗОВАНИЯ РОCCИЙCКОЙ ФЕДЕРАЦИИ}\\
        ФЕДЕРАЛЬНОЕ ГОCУДАРCТВЕННОЕ БЮДЖЕТНОЕ ОБРАЗОВАТЕЛЬНОЕ УЧРЕЖДЕНИЕ\\ВЫCШЕГО ОБРАЗОВАНИЯ \\
        \textbf{«БЕЛГОРОДCКИЙ ГОCУДАРCТВЕННЫЙ ТЕХНОЛОГИЧЕCКИЙ\\УНИВЕРCИТЕТ им. В. Г. ШУХОВА»\\ (БГТУ им. В.Г. Шухова)} \\
        \bigbreak
        \includegraphics[width=70mm]{log}\\
        ИНСТИТУТ ИНФОРМАЦИОННЫХ ТЕХНОЛОГИЙ И УПРАВЛЯЮЩИХ СИСТЕМ\\}
\end{center}

\vfill
\begin{center}
    \large{
        \textbf{
            Лабораторная работа №1}}\\
    \normalsize{
        по дисциплине: Системное моделирование \\
        тема: «Поведение механических систем в статике»}
\end{center}
\vfill
\hfill\textbox{
    Выполнил: ст. группы ПВ-223\\Пахомов Владислав Андреевич
    \bigbreak
    Проверил: Полунин Александр Иванович
}
\vfill\begin{center}
    Белгород 2024 г.
\end{center}
\newpage
\begin{center}
    \textbf{Лабораторная работа №1}\\
    Поведение механических систем в статике\\
    Вариант 10
\end{center}
\textbf{Цель работы: }научиться моделировать на примере моделирования поведения механической системы в статике.
\begin{enumerate}[1. ]
    \item Разработать математическую модель, описывающую поведение элементов механической системы в статике.\\
          \begin{center}
              \includegraphics[width=140mm]{variant.png}
              \bigbreak
              \includegraphics[width=140mm]{mod1.jpg}
              \includegraphics[width=140mm]{mod2.jpg}
          \end{center}
    \item Разработать программу на основании математической модели и произвести расчёты.
          \begin{minted}{C++}
#include <algorithm>
#include <iostream>
#include <iomanip>

#define PI 3.141592654
#define g 9.81
#define k1 5000.0
#define k2 7000.0
#define I 5.0
#define R 1.0
#define r 0.2
#define m 10.0
#define e 0.0000000001
#define es 0.00001


double getM2(double P) {
    return -(m * g + P) * R;
}

double getM1(double angle) {
    return k1 * angle * PI * r * r / 180.0;
}

double getM(double angle, double P) {
    return getM1(angle) + getM2(P);
}

double getDeriative(double angle, double P) {
    return (getM(angle + e, P) - getM(angle, P)) / e;
}

double getDelta(double angle, double P) {
    return getM(angle, P) / getDeriative(angle, P);
}

struct result {
    double angle;
    bool succeed;
    double P;
};

result newton(double P) {
    double angle = 0;
    int c = 0;
    double delta = getDelta(angle, P);
    while (std::abs(getM(angle, P)) >= es && c < 100) {
        c++;
        angle -= delta;
        delta = getDelta(angle, P);
    }

    return (result) {angle, std::abs(getM(angle, P)) < es, P };   
}

double getY(double angle, double P) {
    return (angle * PI * R) / 180.0 + (m * g + P) / k2;
}

int main() {
    double P = 0;
    double stepP = 100.0;

    std::cout << std::setw(15) << "P" << "    " << std::setw(15) << "angle" << "    " << std::setw(15) << "Y" << std::endl;
    while (stepP > es) {
        result res = newton(-P);
        
        std::cout << std::setw(15) << res.P << "    " << std::setw(15) << res.angle << "    " << std::setw(15) << getY(res.angle, res.P) << std::endl;

        if (res.angle > 0)
            P += stepP;
        else {
            P -= stepP;
            stepP /= 2;
            P += stepP;
        } 
    }
}
    \end{minted}
          Результаты выполнения программы:

          \begin{center}
              \begin{tabular}{|c|c|c|}
                  \hline
                  $P$      & $\alpha$     & $y$          \\
                  \hline
                  -0       & 28.1036      & 0.504514     \\
                  \hline
                  -100     & -0.54431     & -0.00977143  \\
                  \hline
                  -50      & 13.7796      & 0.247371     \\
                  \hline
                  -100     & -0.54431     & -0.00977143  \\
                  \hline
                  -75      & 6.61766      & 0.1188       \\
                  \hline
                  -100     & -0.54431     & -0.00977143  \\
                  \hline
                  -87.5    & 3.03668      & 0.0545143    \\
                  \hline
                  -100     & -0.54431     & -0.00977143  \\
                  \hline
                  -93.75   & 1.24618      & 0.0223714    \\
                  \hline
                  -100     & -0.54431     & -0.00977143  \\
                  \hline
                  -96.875  & 0.350937     & 0.0063       \\
                  \hline
                  -100     & -0.54431     & -0.00977143  \\
                  \hline
                  -98.4375 & -0.0966866   & -0.00173571  \\
                  \hline
                  -97.6562 & 0.127125     & 0.00228214   \\
                  \hline
                  -98.4375 & -0.0966866   & -0.00173571  \\
                  \hline
                  -98.0469 & 0.0152192    & 0.000273214  \\
                  \hline
                  -98.4375 & -0.0966866   & -0.00173571  \\
                  \hline
                  -98.2422 & -0.0407337   & -0.00073125  \\
                  \hline
                  -98.1445 & -0.0127573   & -0.000229018 \\
                  \hline
                  -98.0957 & 0.00123096   & 2.20982e-05  \\
                  \hline
                  -98.1445 & -0.0127573   & -0.000229018 \\
                  \hline
                  -98.1201 & -0.00576315  & -0.00010346  \\
                  \hline
                  -98.1079 & -0.00226609  & -4.06808e-05 \\
                  \hline
                  -98.1018 & -0.000517564 & -9.29129e-06 \\
                  \hline
                  -98.0988 & 0.0003567    & 6.40346e-06  \\
                  \hline
                  -98.1018 & -0.000517564 & -9.29129e-06 \\
                  \hline
                  -98.1003 & -8.04323e-05 & -1.44392e-06 \\
                  \hline
                  -98.0995 & 0.000138134  & 2.47977e-06  \\
                  \hline
                  -98.1003 & -8.04323e-05 & -1.44392e-06 \\
                  \hline
                  -98.0999 & 2.88507e-05  & 5.17927e-07  \\
                  \hline
                  -98.1003 & -8.04323e-05 & -1.44392e-06 \\
                  \hline
                  -98.1001 & -2.57908e-05 & -4.62995e-07 \\
                  \hline
                  -98.1    & 0            & 7.62939e-10  \\
                  \hline
                  -98.0999 & 1.51903e-05  & 2.72696e-07  \\
                  \hline
                  -98.1    & 0            & 7.62939e-10  \\
                  \hline
                  -98.1    & 8.36015e-06  & 1.50081e-07  \\
                  \hline
                  -98.1    & 0            & 7.62939e-10  \\
                  \hline
                  -98.1    & 4.94506e-06  & 8.87735e-08  \\
                  \hline
                  \rowcolor{Yellow}
                  -98.1    & 0            & 7.62939e-10  \\
                  \hline
              \end{tabular}
          \end{center}

\end{enumerate}

\textbf{Вывод: } в ходе лабораторной работы изучили основные шаги моделирования, 
промоделировали поведение механической системы в статике.

\end{document}