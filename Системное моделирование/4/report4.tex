\documentclass[a4paper,14pt]{extarticle}


\usepackage[english,russian]{babel}
\usepackage[T2A]{fontenc}
\usepackage[utf8]{inputenc}
\usepackage{ragged2e}
\usepackage[utf8]{inputenc}
\usepackage{hyperref}
\usepackage{minted}
\setmintedinline{frame=lines, framesep=2mm, baselinestretch=1.5, bgcolor=LightGray, breaklines,fontsize=\scriptsize}
\setminted{frame=lines, framesep=2mm, baselinestretch=1.5, bgcolor=LightGray, breaklines,fontsize=\scriptsize}
\usepackage[table]{xcolor}
\definecolor{LightGray}{gray}{0.9}
\definecolor{Yellow}{rgb}{1,1,0}
\usepackage{graphicx}
\usepackage[export]{adjustbox}
\usepackage[left=1cm,right=1cm, top=1cm,bottom=1cm,bindingoffset=0cm]{geometry}
\usepackage{fontspec}
\usepackage{ upgreek }
\usepackage[shortlabels]{enumitem}
\usepackage{adjustbox}
\usepackage{multirow}
\usepackage{amsmath}
\usepackage{amssymb}
\usepackage{pifont}
\usepackage{pgfplots}
\usepackage{longtable}
\usepackage{array}

\graphicspath{ {./images/} }
\makeatletter
\AddEnumerateCounter{\asbuk}{\russian@alph}{щ}
\makeatother
\setmonofont{Consolas}
\setmainfont{Times New Roman}

\newcommand\textbox[1]{
	\parbox{.45\textwidth}{#1}
} 

\newcommand{\specialcell}[2][c]{%
	\begin{tabular}[#1]{@{}c@{}}#2\end{tabular}}

\begin{document}
\pagenumbering{gobble}
\begin{center}
    \small{
        \textbf{МИНИCТЕРCТВО НАУКИ И ВЫCШЕГО ОБРАЗОВАНИЯ РОCCИЙCКОЙ ФЕДЕРАЦИИ}\\
        ФЕДЕРАЛЬНОЕ ГОCУДАРCТВЕННОЕ БЮДЖЕТНОЕ ОБРАЗОВАТЕЛЬНОЕ УЧРЕЖДЕНИЕ\\ВЫCШЕГО ОБРАЗОВАНИЯ \\
        \textbf{«БЕЛГОРОДCКИЙ ГОCУДАРCТВЕННЫЙ ТЕХНОЛОГИЧЕCКИЙ\\УНИВЕРCИТЕТ им. В. Г. ШУХОВА»\\ (БГТУ им. В.Г. Шухова)} \\
        \bigbreak
        \includegraphics[width=70mm]{log}\\
        ИНСТИТУТ ИНФОРМАЦИОННЫХ ТЕХНОЛОГИЙ И УПРАВЛЯЮЩИХ СИСТЕМ\\}
\end{center}

\vfill
\begin{center}
    \large{
        \textbf{
            Лабораторная работа №4}}\\
    \normalsize{
        по дисциплине: Системное моделирование \\
        тема: «Получение дифференциальных уравнений движения с помощью вариационных принципов. Уравнение Лагранжа второго рода»}
\end{center}
\vfill
\hfill\textbox{
    Выполнил: ст. группы ПВ-223\\Пахомов Владислав Андреевич
    \bigbreak
    Проверил: Полунин Александр Иванович
}
\vfill\begin{center}
    Белгород 2024 г.
\end{center}
\newpage
\begin{center}
    \textbf{Лабораторная работа №4}\\
    Получение дифференциальных уравнений движения с помощью вариационных принципов. Уравнение Лагранжа второго рода\\
    Вариант 10
\end{center}
\textbf{Цель работы: }научиться моделировать на примере моделирования поведения механической системы в статике.
\begin{enumerate}[1. ]
    \item Разработать математическую модель, описывающую поведение элементов механической системы в статике.\\
          \begin{center}
              \includegraphics[width=140mm]{mod1.jpg}
              \includegraphics[width=140mm]{mod2.jpg}
          \end{center}
    \item Разработать программу на основании математической модели и произвести расчёты.
          \begin{minted}{python3}
import math
import matplotlib.pyplot as plt

dt = ((20.0) / (30000.0))
PI = 3.141592654
g = 9.81
k1 = 5000.0
k2 = 7000.0
I = 5.0
R = 1.0
r = 0.2
m = 10

def degreeToRadians(angle):
    return angle * (PI / 180.0)

def getAngleAcc(angle, y):
    return ((PI * PI * k1 * r * r * angle) / 32400.0 - (PI * k2 * R * (y - (angle * PI * R) / 180.0)) / 180.0) / -I

def getAngleVel(w):
    return w

def getYAcc(y, angle):
    return g + (k2 * (y - angle * PI * R / 180)) / -m

def getY(v):
    return v

angle = 0
aV = 0.0

y = 0.0
yV = 0.0

x_plot = list()
y_plot = list()
angle_plot = list()

t = 0.0
while t < 100:
    aVk1 = getAngleAcc(angle, y)
    aVk2 = getAngleAcc(angle + (dt/2) * aVk1, y + dt/2)
    aVk3 = getAngleAcc(angle + (dt/2) * aVk2, y + dt/2)
    aVk4 = getAngleAcc(angle + dt * aVk3, y + dt)
    aV += (dt/6) * (aVk1 + 2 * aVk2 + 2 * aVk3 + aVk4)
    
    # Recalc angle
    ak1 = getAngleVel(aV)
    ak2 = getAngleVel(aV + (dt/2) * ak1)
    ak3 = getAngleVel(aV + (dt/2) * ak2)
    ak4 = getAngleVel(aV + dt * ak3)
    angle += (dt/6) * (ak1 + 2 * ak2 + 2 * ak3 + ak4)
    
    # Recalc yV
    yVk1 = getYAcc(y, angle)
    yVk2 = getYAcc(y + (dt/2) * yVk1, angle + dt/2)
    yVk3 = getYAcc(y + (dt/2) * yVk2, angle + dt/2)
    yVk4 = getYAcc(y + dt * yVk3, angle + dt)
    yV += (dt/6) * (yVk1 + 2 * yVk2 + 2 * yVk3 + yVk4)
    
    # Recalc y
    yk1 = getY(yV)
    yk2 = getY(yV + (dt/2) * yk1)
    yk3 = getY(yV + (dt/2) * yk2)
    yk4 = getY(yV + dt * yk3)
    y += (dt/6) * (yk1 + 2 * yk2 + 2 * yk3 + yk4)

    x_plot.append(t)
    y_plot.append(y)
    angle_plot.append(angle)
    t += dt

# Визуализация
fig, (angle, y) = plt.subplots(2)
y.plot(x_plot, y_plot, 'r-', label='Отклонение')
angle.plot(x_plot, angle_plot, 'g-', label='Угол')
y.set_title('Координата y')
angle.set_title('Угол')
y.grid(True)
angle.grid(True)
plt.show()
    \end{minted}
          Результаты выполнения программы:

\begin{center}
\includegraphics[width=200mm]{graphs.png}
\end{center}

\end{enumerate}

\textbf{Вывод: } в ходе лабораторной работы изучили основные шаги моделирования,
промоделировали поведение механической системы.

\end{document}