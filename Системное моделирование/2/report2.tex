\documentclass[a4paper,14pt]{extarticle}


\usepackage[english,russian]{babel}
\usepackage[T2A]{fontenc}
\usepackage[utf8]{inputenc}
\usepackage{ragged2e}
\usepackage[utf8]{inputenc}
\usepackage{hyperref}
\usepackage{minted}
\setmintedinline{frame=lines, framesep=2mm, baselinestretch=1.5, bgcolor=LightGray, breaklines,fontsize=\scriptsize}
\setminted{frame=lines, framesep=2mm, baselinestretch=1.5, bgcolor=LightGray, breaklines,fontsize=\scriptsize}
\usepackage[table]{xcolor}
\definecolor{LightGray}{gray}{0.9}
\definecolor{Yellow}{rgb}{1,1,0}
\usepackage{graphicx}
\usepackage[export]{adjustbox}
\usepackage[left=1cm,right=1cm, top=1cm,bottom=1cm,bindingoffset=0cm]{geometry}
\usepackage{fontspec}
\usepackage{ upgreek }
\usepackage[shortlabels]{enumitem}
\usepackage{adjustbox}
\usepackage{multirow}
\usepackage{amsmath}
\usepackage{amssymb}
\usepackage{pifont}
\usepackage{pgfplots}
\usepackage{longtable}
\usepackage{array}

\graphicspath{ {./images/} }
\makeatletter
\AddEnumerateCounter{\asbuk}{\russian@alph}{щ}
\makeatother
\setmonofont{Consolas}
\setmainfont{Times New Roman}

\newcommand\textbox[1]{
	\parbox{.45\textwidth}{#1}
} 

\newcommand{\specialcell}[2][c]{%
	\begin{tabular}[#1]{@{}c@{}}#2\end{tabular}}

\begin{document}
\pagenumbering{gobble}
\begin{center}
    \small{
        \textbf{МИНИCТЕРCТВО НАУКИ И ВЫCШЕГО ОБРАЗОВАНИЯ РОCCИЙCКОЙ ФЕДЕРАЦИИ}\\
        ФЕДЕРАЛЬНОЕ ГОCУДАРCТВЕННОЕ БЮДЖЕТНОЕ ОБРАЗОВАТЕЛЬНОЕ УЧРЕЖДЕНИЕ\\ВЫCШЕГО ОБРАЗОВАНИЯ \\
        \textbf{«БЕЛГОРОДCКИЙ ГОCУДАРCТВЕННЫЙ ТЕХНОЛОГИЧЕCКИЙ\\УНИВЕРCИТЕТ им. В. Г. ШУХОВА»\\ (БГТУ им. В.Г. Шухова)} \\
        \bigbreak
        \includegraphics[width=70mm]{log}\\
        ИНСТИТУТ ИНФОРМАЦИОННЫХ ТЕХНОЛОГИЙ И УПРАВЛЯЮЩИХ СИСТЕМ\\}
\end{center}

\vfill
\begin{center}
    \large{
        \textbf{
            Лабораторная работа №2}}\\
    \normalsize{
        по дисциплине: Системное моделирование \\
        тема: «Движение механических систем»}
\end{center}
\vfill
\hfill\textbox{
    Выполнил: ст. группы ПВ-223\\Пахомов Владислав Андреевич
    \bigbreak
    Проверил: Полунин Александр Иванович
}
\vfill\begin{center}
    Белгород 2024 г.
\end{center}
\newpage
\begin{center}
    \textbf{Лабораторная работа №2}\\
    Движение механических систем\\
    Вариант 10
\end{center}
\textbf{Цель работы: }научиться моделировать на примере моделирования поведения механической системы.
\begin{enumerate}[1. ]
    \item Разработать математическую модель, описывающую поведение элементов механической системы.\\
          \begin{center}
              \includegraphics[width=140mm]{variant.png}
              \bigbreak
              \includegraphics[width=140mm]{mod1.jpg}
              \includegraphics[width=140mm]{mod2.jpg}
          \end{center}
    \item Разработать программу на основании математической модели и произвести расчёты.
          \begin{minted}{C++}
#include <iostream>

#define dt 0.001
#define PI 3.141592654
#define g 9.81
#define k1 5000.0
#define k2 7000.0
#define I 5.0
#define R 1.0
#define r 0.2
#define m 10.0
#define es 0.000001

double getY(double angle, double P) {
    return (angle * PI * R) / 180.0 + (m * g + P) / k2;
}

double getM(double angle, double P) {
    return 
    // M1
    (k1 * angle * PI * r * r) / 180.0 +
        
    // M2
    - (m*g + P) * R;
}

int main() {
    double angle = 0.0;
    double w = 0.0;
    double e = 0.0;
    double p = 0;

    for (double t = 0.0; t < .5; t += dt) {
        std::cout << t << " " << angle << " " << getY(angle, p) << std::endl;

        e = -getM(angle, p) / I;
        w = w + e * dt;
        angle = angle + w * dt + e * t * t / 2;
    }
}
    \end{minted}
          Результаты выполнения программы:

          \begin{center}
            \includegraphics[width=140mm]{graphs.png}
          \end{center}

\end{enumerate}

\textbf{Вывод: } в ходе лабораторной работы изучили основные шаги моделирования,
промоделировали поведение механической системы.

\end{document}